
\chapter{experimental tools}

In this study, the nanostructures were grown in a home-built chemical vapor deposition system and characterized using both electron microscopy and optical spectroscopy methods, including \gls{sem}, \gls{xrd}, \gls{edx}, \gls{tem}, Raman and UV-Vis. Other sample and substrate preparation methods are also briefly introduced. A detailed introduction will be given to the \gls{cvd} system. For other mature techniques, the contents will be limited to the extent where experienced material researchers could repeat the experiments performed here. Particular attentions will also be mentioned for properly characterizing the nanostrucutres synthesized in this work. This chapter is concluded with data processing discussion. 

\section{Synthesis Equipments}
\subsection{Home-built CVD System}
The syntheses were accomplished in a home-made hot-wall \gls{cvd} system. The reaction chamber part and a schematic drawing of whole system can be visualized in Fig.~\ref{fig:ch2cvd}.  The reaction chamber consists of fused silica tube and a thermal furnace. The furnace is made by two semi-cylindrical ceramic fiber heaters (WATLOW, Inc.) with power density from 0.8 to 4.6 \si{W cm^{-2}}. Quartz tube (Quartz Sci, Inc.) of 1 in diameter was inserted into the furnace serving as reaction chamber. A mechanical pump was connected to maintain the low pressure environment inside the chamber. The length of uniform heating zone is 6 in, with cooling zone extending outward. Carriers gas argon (Ar) and reactant gas oxygen (\ce{O2}) was regulated by mass flow controllers.

\begin{figure}[htb]
\centering
\subfloat[]{\includegraphics[width=0.7\textwidth]{CVD_d346.jpg}}

\subfloat[]{\includegraphics[width=0.9\textwidth]{CVD_model.jpg}}
\caption[Home-built low-pressure chemical vapor deposition system]{Home-built low-pressure chemical vapor deposition system. (a) Photograph of reaction chamber. (b) Sketch of CVD system, where A1: quartz tube; B1: bubbler; C1-2: gas cylinders; E1: mechanical pump; H1-2: ceramic heater; MFC: flow controller; P1: pressure gauge; V1-4: valves; V5-7: butterfly valves.}
\label{fig:ch2cvd}
\end{figure}

The controllable parameters of this \gls{cvd} system include central heating temperature, absolute gas flow and relative ratio of (Ar/\ce{O2}), amount of source material and location of the substrates. The operation capability is summarized in Table~\ref{tab:cvd}.

\begin{table}[htb]
\centering
\caption{Home-built CVD parameters}\label{tab:cvd}
    \begin{tabular}{lcccr}
    \toprule
     &&&\multicolumn{2}{c}{Flow} \\
    \cmidrule(l){4-5}
             & Temperature & Pressure & Ar & \ce{O2}  \\
    \midrule
             & \si{\degreeCelsius} & mTorr & sccm & sccm\\
    \midrule
    Range      & RT-1100    & 10 mTorr-1 atm & 0 - 100 & 0-30  \\
    Resolution & $\pm1$  & correlated to flow & 1   & 0.1  \\
    \bottomrule
    \end{tabular}
\end{table}

The heating temperature $T$ profile measured at different settings in ambient environment was shown in Fig.~\ref{fig:ch2temp}. There exists a uniform heating zone in the center area, spanning over about 2 in; then $T$ descends gradually within the 6 in heating zone, and decrease rapidly at the rest part. In this study, a zero position referred as 0 inch is defined at the upstream edge of reaction chamber covered by the furnace. 

\begin{figure}[htb]
\centering
\includegraphics[width=0.7\textwidth]{temp_profile.pdf}
\caption{Temperature profile of home-built CVD in ambient environment.}
\label{fig:ch2temp}
\end{figure}

The substrates were mostly positioned just outside the downstream heating zone since within this region, the vapor undergoes a rapid temperature gradient and precipitation occurs. The source material placed in the furnace center is oxidized and evaporated. The growth species, transported by carrier gas, bombard both substrate and chamber wall. Some will be adsorbed by the substrate and become adatoms while some may remain as gas molecules, waiting for another event. On the hot substrate, adatoms diffuse and do not settle down until finding an appropriate location where equilibrium is favored. Detailed growth conditions will be introduced in the corresponding chapters. Basic operations and several growth recipes using the home-built \gls{cvd} system are included in Appendix.


\subsection{Substrates Treatments}\label{ch2sub}

Silicon and silicon dioxides on Si wafer ($p$-Si(100), University Wafer Inc.) were primarily used, and other substrates, i.e., Mica\footnote{\ce{K(Al2)(Si3Al)O10(OH)2}}, glass (Fisher Scientific, microscope slide, 12-549),\footnote{Typical composition is 72.6\% \ce{SiO2}, 0.8\% \ce{B2O3}, 1.7\% \ce{Al2O3}, 4.6\% \ce{CaO}, 3.6\% \ce{MgO} and 15.2\% \ce{Na2O}} stainless steel, were also employed. Si and \ce{SiO2}-Si substrates were first cut into rectangular pieces of certain size and then ultrasonically cleaned (Branson 1510R-MTH) with acetone and alcohol for about 15 min, each followed by blow-drying with nitrogen gas. After solution cleaning, the substrate surface is usually hydrophobic. Depending on the specific experimental requirement, a hydrophilic surface may be required. A plasma cleaner (Kurt J Lesker, Plasma-Preen 862) was used to render a hydrophilic Si surface, where reaction among $e$, \ce{O+}, \ce{O2+}, \ce{O^-},\ce{O2} occurs, and \ce{OH}-terminated surface followed.\cite{Habib2010} The treatment proceeds at 2 Torr \ce{O2} for 3 min. Glass substrates cleaning is the same as that of Si. Mica is cleaved right before the growth without extra cleaning steps. In addition, substrates can be coated with a thin layer of metal before growth. This process will be covered in Sec.~\ref{sec:mag}.

\subsection{Magnetron Sputtering}\label{sec:mag}
Sputtering, a process in which atoms are ejected from a solid target material by bombarding it with energetic particles, is a well established \gls{pvd} process with a high degree of controllability. The high energy and controllable parameters of sputtering can result in the growth of well-structured and crystalline films. Further, sputtering can be easily implemented as a roll-to-roll process for large-scale manufacturing. It is widely utilized for deposition of \ce{WO_x} in industry.

In this work, magnetron sputtering (Denton Vacuum Desk IV) was used to coat thin layers of metals onto cleaned substrates, such as W, Mo, Au, Ag, Pd, and Pt. As shown in Fig.~\ref{fig:ch2magsp}, a high potential created between the target (cathode) and substrate ionizes the argon molecules. These ions are accelerated to bombard the target foils, knocking out active atoms which subsequently deposit onto the substrates. A magnetic field is applied to confine electrons to the vicinity of the foils. These electrons in turn increase the ionization yield. 
\begin{figure}[htb]
\centering
\includegraphics[width=0.7\textwidth]{magsp.jpg}
\caption[Schematic drawing of magnetron sputtering system]{Schematic drawing of magnetron sputtering system. Reproduced from Ref.\cite{Song2008}}
\label{fig:ch2magsp}
\end{figure}

Sputtering is a better PVD technique than e-beam deposition since it alleviates the adverse effects in the latter. \ce{Ar+} ions are accelerated to target foils. The pressure of \ce{Ar+} has two functions: to sputter off the target ions, and to influence the \gls{mfp}. These collisions between target ions and \ce{Ar+} leads to almost all arrival angle, thus uniform coverage. The sputter yield is number of target ions released per ions hitting on target. If \ce{Ar+} ion energy is less than 100 eV, the yield is zero; If larger than 10 keV, implantation of \ce{Ar+} into target foils occurs. Usually yield between 1 and 2 is desired. The \gls{mfp} in sputtering is on order of 3 cm (assuming pressure 30 mTorr), with source-to-substrate distance as 5 cm, average number of collisions is about 5/3$\sim$2. 

\section{Characterization Tools}
\subsection{Scanning Electron Microscopy (SEM)}\label{sec:sem}

The SEM instrument used in this study is JEOL JSM-6480. The \gls{ebeam} source is a tungsten filament heated at about 2800 K. The acceleration voltage can be up to 30 kV. After acceleration, electron beam strikes the sample mounted on specimen stage, producing rich signals as illustrated in Fig.~\ref{fig:ch2sem}. Depending on the energy and scattering angle, the out-coming electrons from specimen can be grouped into \gls{bse} and \gls{se}, respectively. \gls{se} is generated from the emission of valence band electrons. And due to the limited energy ($<50$ eV), the emission from deep region is mostly absorbed again; only those electrons that are at the vicinity of specimen surface can escape. Hence the \gls{se} image is topographical sensitive. The emission efficiency is strongly affected by the geometry (tilting angle of specimen surface) and voltage potential (charging effect), so does the contrast in \gls{se} image. On the other hand, backscattered electron could have large energy close to that of incident \gls{ebeam}. The scattering cross section strongly depends on atomic number Z, thus allowing for the compositional analysis. The X-ray emission stems from inner shell transition of the constitutive atoms, and cathodoluminescence from bandgap transition. Other \gls{ebeam} generated signals, such as Auger electrons, are usually not useful in SEM unless with ultra-high vacuum and clean surface.

\begin{figure}[htb]
\centering
\includegraphics[width=0.5\textwidth]{sem_sch.png}
\caption[Schematic drawing of electron materials interaction in SEM]{Schematic drawing of electron materials interaction in SEM. Some process is not shown, such as Auger emission.}
\label{fig:ch2sem}
\end{figure}

In this study, secondary electron images were mainly used to reveal the morphology of the samples. Unless otherwise noticed, the SEM images in this work were all acquired using 10 kV acceleration voltage, 10 mm working distance and 30 spot size.

Charge effect could arise on the samples grown in this study, leading to anomalous contrast on the images. Several measures could be taken to alleviate this adverse influence, including adjusting the acceleration voltage, tilting the specimen, and coating a thin layer of metal. Sometimes when an area is scanned for a long time at high magnification, the surrounding region could appear brighter after switching to a lower magnification. This is a sample contamination, due to the \gls{ebeam} induced polymerization and deposition of hydrocarbon molecules in the vicinity of that smaller area. 


\subsection{Energy Dispersive X-ray Spectroscopy (EDX)}\label{sec:edx}

Energy dispersive X-ray spectroscopy provides compositional information attaining to the studied sample because of the characteristic X-ray emission from each constitutive element. EDX unit is usually attached to another electron microscopy, SEM or TEM. In this study, the SEM-attached EDX was primarily used. The X-ray detector (INCA 7573-M, 10 \si{mm^2} Si cooled by liquid nitrogen) is equipped with a super thin window enabling light element detection such as boron. Typical working parameters of EDX detection in this study include 20 kV acceleration voltage, 10 mm working distance and 60 s analysis time.

To obtain quantitative results on EDX spectra, special attentions should be paid to the following aspects:
\begin{itemize}
\item elemental distribution in an X-ray generation area is uniform,
\item the specimen surface is flat,
\item the electron beam enters perpendicular to the specimen. 
\item a reference sample consisting of identical elements and known ratio is used. 
\end{itemize}
Even for qualitative analysis, ambiguity exists when the $K$ lines of lighter elements overlap with $L$ lines of heavier elements. For instance, Na-$K_\alpha$ (1.041 keV) interferes with $L_\alpha$ lines of Zn (1.009 keV) and Cu (0.928 keV). Overall spectra and growth conditions should be weighted against possible wrong assignment. 

\subsection{X-ray Diffraction (XRD)}
Crystal structures of the as-synthesized specimens in this study were characterized using PANXpert Pro MRD with Cu $K\alpha_{avg}$ radiation at $\lambda$=1.5418 \si{\angstrom}. The high energy photon ($E = 1239.8/0.15148= 8184.5$ eV) was generated using accelerated electron beam from tungsten filament to bombard Cu target. The $\theta-2\theta$ configuration was primarily used, as shown in Fig.~\ref{fig:ch2theta}. 
\begin{figure}[htb]
\centering
\includegraphics[width=0.5\textwidth]{xrd_2theta.png}
\caption{Bragg-Brentano geometry used in XRD configuration}
\label{fig:ch2theta}
\end{figure}
When striking onto the crystal, these photons are scattered by the lattices. Since the wavelength of X-ray is similar to the lattice spacing, the scattering events interact coherently. Only at specific angles constructive interference occurs, and a diffraction peak registers when the detector is scanned across that angle.  

The as-synthesized specimens were mounted onto diffractometer in a way that the X-ray will illuminate interested region during the whole scanning. Typical settings used was summarized in Table.~\ref{tab:ch2xrd}.

\begin{table}[htb]
\centering
\caption{XRD settings used in this study}\label{tab:ch2xrd}
\begin{tabular}{lp{1.5in}lp{1.5in}}
\toprule
Name & Value & Name & Value  \\
\midrule
Voltage   & 45 kV & Current & 40 mA \\
Divergence slit & 1/32$^\circ$(alignment) 1/2$^\circ$(scanning) & Receiving slit& Parallel collimator \\
Soller slit & 0.04 rad & Collimator & Parallel plate 0.27 rad \\
Scan range & $10 \sim 65 ^\circ$ & Step size & 0.02$^\circ$ \\
\bottomrule
\end{tabular}
\end{table}

The 1D nanostructures in this study usually grew with preferred orientation; therefore, they cannot be treated as powders. The interpretation of XRD patterns could become difficult somehow. Two possible approaches could be followed to bypass this difficulty: one is using grazing angle XRD which could reveal in-plane domain size and orientation;\cite{Tersigni2011,Goorsky2002} the other is combining XRD with alternative techniques, such as Raman, TEM. The latter approach was pursued in this work. 


\subsection{Transmission Electron Microscopy (TEM)}

TEM qualifies as a powerful and versatile tool for material characterization. Akin to the electron-materials interaction introduced in Sec.~\ref{sec:sem}, various signals are generated when the thin specimen was radiated by electron beam. Besides the \gls{saed} pattern and \gls{hrtem} imaging, other TEM based techniques include \gls{cbed}, EDX, \gls{eels} and \gls{cl}, etc. Therefore, TEM is essentially a signal-generating and detecting tool. Fundamental physics of electron is needed to obtain a comprehensive understanding of these TEM related techniques. Only a brief summary is given here.

Electron scattering was dictated by Coulomb interaction, and can be characterized by the following four aspects: the cross section (scattering probability), differential cross section (scattering angle), mean free path, and inelastic or elastic scattering. The cross section $\sigma$ in \si{m^2}, which when divided by area of targeting atom, represents a probability that scattering event will occur. Differential cross section for an isolated atom is 
\[
\frac{d\sigma}{d\Omega} = \frac{1}{2\pi \sin\theta} \frac{d\sigma}{d\theta},
\] 
where $\theta$ is scattering angle in radians, $\Omega$ is solid angle in steradians (sr), and $\theta$ is half of the scattering angle. The total cross section is just $\sigma$ times atom numbers in unit volume. The mean free path is the inverse of $\sigma_{total}$, that is $\lambda_{MFP} = 1/\sigma_{total}$. 

The spatial distribution of scattering is observed as contrast in images, and angular distribution is viewed in form of diffraction patterns. The positions of diffracted E-beams are determined by size and shape of unit cell, and intensities governed by distribution, number, and types of atoms in the specimen. 

JEOL JEM-2100 TEM with a \ce{LaB6} filament operated at 200 kV was used in this work. A numerical comparison between the scattering events in XRD and TEM is provided to visualize the difference. In both processes, the constructive interference is described by $2d\sin\theta = \lambda$, where $d$ is crystal plane distance, $\lambda$ is incident wavelength (photon or electron), and $\theta$, again, is half of the scattering angle. Table~\ref{tab:ch2tem} demonstrates the origin of difference in these two techniques.
\begin{table}[htb]
\centering
\caption{Scattering angle difference of Si(111) in XRD and TEM}\label{tab:ch2tem}
\begin{tabular}{lccr}
\toprule
Si(111) spacing & Source & Wavelength & scattering angle($2\theta$) \\
\midrule
3.135 \AA & Cu $K\alpha$ & 1.541 \AA & 28.45$^\circ$  \\
3.135 \AA & 200 kV & 0.0251 \si{\angstrom} & 0.45$^\circ$  \\
\bottomrule
\end{tabular}
\end{table}
This small deviation angle determines the way of secondary signal usage in TEM. In analog to the diffraction-limited resolution in conventional lens optics, the ultimate resolved point distance in TEM should be on the order of electron wavelength used, which is 0.025 \si{\angstrom} at 200 kV. However, it is generally more difficult to manipulate electron with magnetic field than do photon with dielectric lens. The current resolving power in TEM is on the order of 1.0 \si{\angstrom} without correction of spherical aberration. The nominal resolution of JEM-2100 is 2.3 \si{\angstrom}. On the other hand, it is much easier to fine tune the power (focal length) of lens in TEM than in conventional optics. This capacity enables other observation approaches that would be rather difficult, if not impossible, to be realized in lens optics, such as \gls{saed}. 

In this study, the TEM specimen was prepared either by directly scratching the as-synthesized sample using TEM grid (Cu, lacy carbon, 300 mesh), or dispersed into aqueous or organic solutions, then drop-cast onto TEM grid. A JEOL double tilt holder was used to rotate the crystal orientation. TEM images and SEADs were acquired using a \gls{ccd} from Gatan, Inc. EDX spectroscopy was captured by similar Oxford Instrument INCA attachment introduced in Sec.~\ref{sec:edx}. To make the best out of TEM analysis, one also needs to be aware of the limitations of TEM. Two aspects were pointed out here, including small sampling and interpreting of the image. The former one is inherent and can only be partially overcome by combining TEM with other ensemble characterization tools. As to the latter one, it should be noted that the specimen is usually in focus from top to bottom surface; therefore the images, \glspl{dp} and spectra are all averaged through the thickness of specimen. In other words, single TEM image has no depth sensitivity.\cite{Williams2009}  

\section{Optical Measurements}
\subsection{Raman Spectroscopy}

Raman spectroscopy, a common vibrational spectroscopy, is based on the inelastic scattering of a monochromatic excitation source with photon energy variation range of 100 to 4000 \si{cm^{-1}}. This energy difference between incident photon and inelastically scattered photon is closely related to a series of vibrational modes. Being nondestructive and requiring minimal preparation, it is an excellent tools to assess lattice dynamics and fingerprint species.\cite{McCreery2000} For instances, characteristic Raman shift can be used to determine material compositions; changes in Raman peak (\gls{fwhm}, frequency shift) are often related to strain; polarization Raman spectra can be used to derive crystal orientation and symmetry. 

Some common symbols for symmetry representations are as following:
\begin{enumerate}
\item $A$ representation indicates that the functions are symmetric with respect to rotation about the principal axis of rotation.
\item $B$ representations are asymmetric with respect to rotation about the principal axis.
\item $E$ representations are doubly degenerate.
\item $T$ representations are triply degenerate.
\item Subscripts u and g indicate asymmetric (\emph{ungerade}) or symmetric (\emph{gerade}) with respect to a center of inversion.
\end{enumerate}

In this study, Raman measurement was performed using a confocal micro-Raman system (Horiba Scentific, Labram HR800) with excitation wavelengths at 441, 532 and 632 nm, where the corresponding photon energy is 2.81 eV, 2.33 eV and 1.96 eV, respectively. For nanowire samples, the laser powers were kept between 0.2 and 0.3 mW, and typical acquisition time was 100 s to avoid possible thermal damage. The spectral resolution is about 1 \si{cm^{-1}} and the depth resolution about 2 $\mu$m.

\subsection{UV-Vis Spectroscopy}
UV-Vis-NIR spectrophotometer is a useful tool to obtain optical properties from various samples. Generally speaking, the UV spectrum ranges from 100 nm to 400 nm, and visible spectrum spans from 400 nm to 750 nm. Absorption from atmospheric \ce{CO2} becomes significant below 200 nm; therefore 100 to 200 nm region is usually not measured unless some vacuum technique is applied. Near-IR ranges from 0.75 to 3 $\mu$m. The cost of fabricating a spectrometer covering the whole NIR region can be quite high. The instrument used in this dissertation (Schimadzu, UV2600Plus) can measure from 220 nm to 1350 nm (5.6 eV to 0.92 eV) with an integrating sphere.

The absorption of materials can be described by Lambert-Beer law: $I = I_0 \exp(-A)=I_0\exp(-\alpha(\lambda) x)$, where $\alpha$ is the absorption coefficient in unit of \si{cm^{-1}}, and $x$ is the optical path in cm. When the sample is in liquid form, it is more convenient to use $A =\epsilon_\lambda C x$, where $C$ is the molar concentration (M = \si{\mole \per L}), and $\epsilon_\lambda$ is the extinction coefficient/molar absorptivity (\si{M^{-1} cm^{-1}}). For example, $\epsilon$ for methylene blue (MB) is $10^5$ \si{M^{-1}cm^{-1}} at 660 nm.\cite{Mills1999} So by measuring the maximum absorbance of sample with known composition, its concentration can be estimated accordingly. 

In the transmission mode, the reflection from materials phase boundary has been compensated with the usage of paired container and solvent. The scattering, however, cannot be eliminated this way. Usually scattering is negligible in molecular disperse media, yet should be considered in colloids or solids when the incident wavelength is comparable to the particle dimension. The size of nanostructures studied in this work is estimated to be 200 to 400 nm after sonication treatment; therefore, scattering effect should be included in the absorbance spectrum. 

In fact, there have been ongoing efforts to retrieve particle size profile from the dynamic light scattering, a technique also known as photon correlation spectroscopy. The experimental principle is as following: the sample in liquid solution is illuminated by a laser source and the fluctuations of the scattered light are detected at a known scattering angle $\theta$ by a fast photon detector. The fluctuations arise from the Brownian motion of small particles, equivalent to a random variation of scatter distance. By auto-correlation analysis, the decay rate of scattering intensity correlation can be related to the diffusion coefficient of small particles in the solvent.\cite{Maret1987} This approach, however, will not be pursued in this work. Instead, the scattering effect is partially circumvented by using diffuse reflection technique. 

\Gls{drs} serves as another tool to estimate the sample band gap using \gls{kmt}.\cite{Tandon1970} The original KM theory was proposed in 1931,\cite{Kubelka1931} and has been popular in the color-related industry, such as painting, pigments, and paper. This method is useful for the analysis of samples of difficultly soluble substances and for samples that will react with the solvents upon being dissolved. It is also particularly applicable when single crystals of the material could not be obtained. Some key points were recapitulated in this thesis.

The sample is under isotropic diffuse illumination. And the upward flux $J$ and downward flux $I$ are characterized by the K-M scattering and absorption coefficients denoted as $S$ and $K$, respectively; that is
\begin{align}
\ud I &= - (K + S)I \ud z + SJ\ud z,\\
\ud J &= + (K + S)J \ud z - SI\ud z.
\end{align}
Quantities $S$ and $K$, in unit of percentage of light scattered and absorbed per unit vertical length, have no direct physical meaning. For example, $S$ and $K$ depend on illumination geometry: diffuse or collimated. In the limit of infinite thick sample, the KM equation is given by
\begin{equation}
f(R_\infty) = \frac{(1-R_\infty)^2}{2R_\infty} = K/S,
\end{equation}
where $R_\infty$ is relative reflectance between sample and standard. In case of dilute species, one can approximate the KM function as $f(R) \propto \frac{\epsilon c}{s}$. When the diffuse scattering condition is fulfilled, $K$ can be related to absorption coefficient $\alpha$ by $K = 2\alpha$. If the scattering coefficient $S$ is fixed with respect to $\lambda$, one can then obtain
\begin{equation}
(f(R_\infty) h \nu)^n \propto (hv - E_g),
\end{equation}
where $n$ depends on the nature of electronic transition. The diffuse reflection spectroscopy is more proper to characterize nanomaterials because compared with UV-Vis transmission measurement for sample dispersed in liquid media, it takes the scattering effect into consideration. Favorable results have been obtained on the comparison of band gaps using KM model and other methods.\cite{Tandon1970,Morales2007} 

\begin{figure}[htb]
\centering
\includegraphics[width=0.5\textwidth]{UV-Vis_drs}
\caption[Schematic drawing of UV-Vis and DRS measurement]{Schematic drawing of (a) UV-Vis and (b) DRS measurement setup adopted from UV2600Plus manual.}
\label{fig:ch2uvvis}
\end{figure}

In this thesis, UV-Vis and diffuse reflectance spectra were recorded using Schimadzu, UV2600Plus, with schematic measurement setup shown in Fig.~\ref{fig:ch2uvvis}. For UV-Vis measurement in transmission mode, the as-synthesized sample was removed from substrates by light sonication(Branson 1510R-MTH, 70W) in ethanol or DI water for 15 seconds. The dispersion was left for 12 h to enable the possible sedimentation. Then the dispersion was transferred into one 10 mm quartz cuvette (Thorlabs, W005654) for absorption measurement with another paired cuvette containing pairing liquid only. To carry out diffuse reflectance measurement, integrating sphere (ISR2600Plus) was used with an incident angle of 0 degree. The baseline reflectance was first recorded with the standard white plate (barium sulfate powder, \ce{BaSO4}) placed at the exit window on the sample path side, then the target sample was set in for relative reflectance measurement. The \ce{BaSO4} powder should be replaced periodically, otherwise the surface might turn yellow. To replace the \ce{BaSO4} powder, one should supply new powder into the sample holder for several time, each followed by a compact pressing using the glass rod. Chemical wrapping paper can be inserted between the glass rod and \ce{BaSO4} powder to prevent the powder sticking. Nanostructured sample on transparent substrates, i.e., quartz, could be mounted directly with the growth substrate, or ground and spread evenly onto \ce{BaSO4} surface.



