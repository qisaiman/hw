
\chapter{experimental tools}

In this study, the nanostructures were grown in a home-built chemical vapor deposition system and characterized using both electron microscopy and optical spectroscopy methods, including SEM, XRD, EDX, TEM, Raman and UV-Vis. Other sample and substrate preparation methods are also briefly introduced. A detailed introduction will be given to the CVD system. For other mature techniques, the contents will be limited to the extent where experienced material researchers could repeat the experiments performed here. Particular attentions will also be mentioned to properly characterize the nanostrucutres synthesized in this work . This chapter is concluded with data processing discussion. 


\section{Home-built CVD System}

The synthesis was accomplished in a home-made hot-wall CVD system, as visualized in Fig.~\ref{fig:ch2cvd}. The furnace is made by two semi-cylindrical ceramic fiber heaters (WATLOW inc.) with power density from 0.8 to 4.6 \si{W cm^{-2}}. Quartz tube (Quartz Sci inc.) of 1 in diameter was primarily used as reaction chamber. A mechanical pump was connected to maintain the low pressure environment inside the chamber. The length of uniform heating zone is 6 in, with cooling zone extending outward. Carriers gas argon (Ar) and reactant gas oxygen (\ce{O2}) was regulated by two mass flow controllers.

\begin{figure}[htb]
\centering
\subfloat[]{\includegraphics[width=0.7\textwidth]{CVD_d346.jpg}}

\subfloat[]{\includegraphics[width=0.9\textwidth]{CVD_model.jpg}}
\caption[CVD system]{Home-built low-pressure chemical vapor system. (a) Photograph of reaction chamber. (b) Sketch of CVD system, where A1: quartz tube; B1: bubbler; C1-2: gas cylinders; E1: mechanical pump; H1-2: ceramic heater; MFC: flow controller; P1: pressure gauge; V1-4: valves; V5-7: butterfly valves.}
\label{fig:ch2cvd}
\end{figure}

The controllable parameters of this CVD system include central heating temperature, absolute gas flow and relative ratio of (Ar/\ce{O2}), amount of source material and location of the substrates. The operation capability is summarized in Table.~\ref{tab:cvd}.

\begin{table}[htb]
\centering
\caption{CVD parameters map}\label{tab:cvd}
    \begin{tabular}{lcccr}
    \toprule
     &&&\multicolumn{2}{c}{Flow} \\
    \cmidrule(l){4-5}
             & Temperature & Pressure & Ar & \ce{O2}  \\
    \midrule
             & \si{\degreeCelsius} & mTorr & sccm & sccm\\
    \midrule
    Range      & RT-1100    & 10 mTorr-1 atm & 0 - 100 & 0-30  \\
    Resolution & $\pm1$  & correlated to flow & 1   & 0.1  \\
    \bottomrule
    \end{tabular}
\end{table}

The heating temperature $T$ profile measured at different settings in ambient environment was shown in Fig.~\ref{fig:ch2temp}. There exists a uniform heating zone in the center area, spanning over about 2 in. Then $T$ descends gradually within the 6 in heating zone, and decrease rapidly at the rest part. In this study, a zero position is defined at the upstream edge of reaction chamber covered by the furnace. 

\begin{figure}[htb]
\centering
\includegraphics[width=0.8\textwidth]{temp_profile.pdf}
\caption[CVD temperature profile]{Temperature profile of home-built low-pressure chemical vapor system.}
\label{fig:ch2temp}
\end{figure}

The substrates were mostly positioned just outside the downstream heating zone since within this region, the vapor undergoes a rapid temperature gradient and precipitation occurs. The source material placed in the furnace center is oxidized and evaporated. The growth species, transported by carrier gas, bombard both substrate and chamber wall. Some will be adsorbed by the substrate and become adatoms while some may remain as gas molecules, waiting for another event. On the hot substrate, adatoms diffuse and do not settle down until finding an appropriate location where equilibrium is favored. Detailed growth conditions are introduced in corresponding chapters.


\section{Scanning Electron Microscopy (SEM) and Energy Dispersive X-ray Spectroscopy (EDX)}\label{sec:sem}

The SEM instrument used in this study is JEOL JSM-6480. The electron beam (E-beam) source is a tungsten filament heated at about 2800 K. The acceleration voltage can be up to 30 kV. After acceleration, electron beam strikes the sample mounted on specimen stage, producing rich signals as illustrated in Fig.~\ref{fig:ch2sem}. Depending on the energy and scattering angle, the out-coming electrons from specimen can be grouped as backscattering electron (BSE) and secondary electron (SE), respectively. SE is generated from the emission of valence band electrons. And due to the limited energy ($<50$ eV), the emission from deep region is mostly absorbed again; only those SE that are at the vicinity of specimen surface can escape. Hence the SE image is topographical sensitive. The emission efficiency is strongly affected by the geometry (tilting angle of specimen surface) and voltage potential (charging effect), so does the contrast in SE image. On the other hand, backscattered electron has large energy close to that of incident E-beam. The scattering efficiency strongly depends on atomic number (Z), thus allowing for the compositional analysis. The X-ray emission stems from inner shell transition of the constitutive atoms, and cathodoluminescence from bandgap transition. Other E-beam generated signals, such as Auger electrons, are not shown here.

\begin{figure}[htb]
\centering
\includegraphics[width=0.5\textwidth]{sem_sch.png}
\caption[SEM excitation volume]{schematic drawing of electron materials interaction in SEM. Some process is not shown, such as Auger emission.}
\label{fig:ch2sem}
\end{figure}

In this study, SE images were mainly used to reveal the morphology of the samples. Energy dispersive X-ray spectroscopy provides compositional information attaining to the studied sample because of the characteristic X-ray lines from each element. The X-ray detector (INCA, 7573-M, 10 \si{mm^2} Si cooled by liquid nitrogen) is equipped with a super thin window enabling light element detection such as boron. To obtain quantitative results on EDX spectra, special attentions should be paid to the following aspects:
\begin{itemize}
\item elemental distribution in an X-ray generation area is uniform,
\item the specimen surface is flat,
\item the electron beam enters perpendicular to the specimen. 
\item a reference sample consisting of identical elements and known ratio is used. 
\end{itemize}

Even for qualitative analysis, ambiguity exists when the $K$ lines of lighter elements overlap with $L$ lines of heavier elements. For instance, Na-$K_\alpha$ (1.041 keV) interferes with $L_\alpha$ lines of Zn (1.009 keV) and Cu (0.928 keV). Overall spectra and growth conditions should be weighted against possible wrong assignment. 

Charge effect could arise on the samples grown in this study, leading to anomalous contrast on the images. Several measures could be taken to alleviate this adverse influence, including adjusting the acceleration voltage, tilting the specimen, and coating a thin layer of metal. Sometimes when an area is scanned for a long time at high magnification, the surrounding region could appear brighter after switching to a lower magnification. This is a sample contamination, due to the E-beam induced polymerization and deposition of hydrocarbon molecules in the vicinity of that smaller area. 


\iffalse
An eucentric specimen stage is used, thus observation area remain fixed when tilting the specimen. The pressure in SEM chamber is on the order of $10^{-3} \sim 10^{-4}$ Pa, which is usually maintained by a diffusion pump, or turbo molecular pump when oil-free operation is needed. For a field emission electron source, a sputter ion pump becomes necessary due to the high vacuum requirement. 

The SEM instrument used in this study is JEOL JSM-6480 and EDX attachment from Oxford Instrument INCA. Typical observation conditions are listed as following:

\begin{enumerate}
\item SEM
\begin{itemize}

\item Acceleration voltage: 10 kV
\item Working distance: 10 mm
\item Scanning time: 80 s
\end{itemize}
\item EDX
\begin{itemize}

\item Acceleration voltage: 20 kV
\item Working distance: 10 mm
\item Dead time: $20\sim30$\%
\end{itemize}
\end{enumerate}


\begin{quotation}
Since the intensity of characteristic X-rays is proportional to the concentration of the corresponding element, quantitative analysis
can be performed. In actual experiment, a standard specimen containing elements with known concentrations is used. The
concentration of a certain element in an unknown specimen can be obtained by comparing the X-ray intensities of the certain element
between the standard specimen and unknown specimen. However, X-rays generated in the specimen may be absorbed in
this specimen or excite the X-rays from other elements before they are emitted in vacuum. Thus, quantitative correction is needed.
In the present EDS and WDS, correction calculation is easily made; however, a prerequisite is required for this correction. That is,
elemental distribution in an X-ray generation area is uniform, the specimen surface is flat, and the electron probe enters perpendicular to the specimen. Actually, many specimens observed with the SEM do not satisfy this prerequisite; therefore, it should be noted that a quantitative analysis result might have appreciable errors.
\end{quotation}


\fi

\section{X-ray Diffraction (XRD)}

Crystal structures of the as-synthesized specimens in this study were characterized using PANXpert X’pert Pro MRD with Cu $K\alpha_{avg}$ radiation at $\lambda$=1.5418 \AA. The high energy photon ($E = 1239.8/0.15148= 8184.5$ eV) was generated using accelerated electron beam from tungsten filament to bombard Cu target. The $\theta-2\theta$ configuration was primarily used, as shown in Fig.~\ref{fig:ch2theta}. 
\begin{figure}[htb]
\centering
\includegraphics[width=0.5\textwidth]{xrd_2theta.png}

\caption[XRD configuration]{Bragg-Brentano geometry used in powder diffractometer}
\label{fig:ch2theta}
\end{figure}
When striking onto the crystal, these photons are scattered by the lattices. Since the wavelength of X-ray is similar to the lattice spacing, the scattering events interact coherently. Only at specific angles constructive interference occurs, and a diffraction peak registers when the detector is scanned across that angle.  

The as-synthesized specimens were mounted onto diffractometer in a way that the X-ray will illuminate interested region during the whole scanning. Typical settings used was summarized in Table.~\ref{tab:ch2xrd}.

\begin{table}[htb]
\centering
\caption{XRD settings}\label{tab:ch2xrd}
\begin{tabular}{lp{1.5in}lp{1.5in}}
\toprule
Name & Value & Name & Value  \\
\midrule
Voltage   & 45 kV & Current & 40 mA \\
Divergence slit & 1/32$^\circ$(alignment) 1/2$^\circ$(scanning) & Receiving slit& Parallel collimator \\
Soller slit & 0.04 rad & Collimator & Parallel plate 0.27 rad \\
Scan range & $10 \sim 65 ^\circ$ & Step size & 0.02$^\circ$ \\
\bottomrule
\end{tabular}
\end{table}

The nanostructures usually grew with preferred orientation; therefore, they cannot be treated as powders. The interpretation of XRD patterns could become difficult somehow. Two possible approaches could be followed to resolve this problem: one is using grazing angle XRD which could reveal in-plane domain size and orientation;\cite{Tersigni2011,Goorsky2002} the other is combining XRD with alternative techniques, such as Raman, TEM. The latter approach was taken in this work. 


\section{Transmission Electron Microscopy (TEM)}



TEM qualifies as a powerful and versatile tool for material characterization. JEOL JEM-2100 TEM with a \ce{LaB6} filament operated at 200 kV was used in current thesis. Similar to the photon-lattice interaction in XRD, the process of TEM could be understood as electron scattering events by the same crystal plane. In both processes, the constructive interference is described by $2d\sin\theta = \lambda$, where $d$ is crystal plane distance, $\lambda$ is incident wavelength (photon or electron), and $\theta$ is half of the scattering angle. A numerical comparison in Table.~\ref{tab:ch2tem} demonstrates the origin of difference in these two techniques. 

\begin{table}[htb]
\centering
\caption{Scattering angle difference of Si(111)}\label{tab:ch2tem}
\begin{tabular}{lccr}
\toprule
Si(111) spacing & Source & Wavelength & scattering angle($2\theta$) \\
\midrule
3.135 \AA & Cu $K\alpha$ & 1.541 \AA & 28.45$^\circ$  \\
3.135 \AA & 200 kV & 0.0251 \AA \textsuperscript{\emph{a}}& 0.45$^\circ$  \\
\bottomrule
\end{tabular}

 \textsuperscript{\emph{a}} with relativistic correction;
\end{table}

This small deviation angle determines the design of TEM column. In analog to the diffraction-limited resolution in conventional lens optics, the ultimate resolved point distance in TEM at 200 kV could be as small as 0.025 \AA. However it is much harder to manipulate electron with magnetic field than photon with dielectric lens, the current resolution in TEM is on the order of 1.0 \AA without correcting the spherical aberration. The nominal resolution of JEM-2100 is 2.3 \AA. On the other hand, it is much easier to fine tune the power (focal length) of lens in TEM than in conventional optics. This capacity enables other observation approaches that would be rather difficult, if not impossible, to be realized in lens optics, such as selected area electron diffraction pattern (SAED). 

Akin to the electron-materials interaction introduced in Sec.~\ref{sec:sem}, X-ray emission and CL also occur in TEM, and these signals could be collected to obtain value insight into specimen at higher spatial resolution. In this study, the TEM specimen was prepared either by directly scratching the as-synthesized sample using TEM grid (Cu, lacy carbon, 300 mesh), or dispersed into aqueous or organic solutions, then drop-cast onto TEM grid. A JEOL double tilt holder was used to rotate the crystal orientation. TEM images and SEADs were acquired using a charge coupled device (CCD) from Gatan, inc. EDX spectroscopy was captured by similar Oxford Instrument INCA attachment. 

The cost of TEM adds up to \$10 per eV. seize the public's imaginations. One must be just as aware of the instrument's limitations as one is of its advantages. Fundamental physics of electrons, how it is controlled by magnetic fields, how it interacts with specimen. TEM is essentially a signal-generating and detecting tool. TEM is initially developed to overcome the image resolution imposed by light microscopes. constitute, draw analogies, resolving power, correction of spherical aberration ($C_s$) and chromatic aberration ($C_c$), $C_s$ is done by , $C_c$ by energy-filtering, which is more useful for thicker specimens. E-beam is one type of ionizing radiation, which is capable of removing the tightly bound, inner-shell electrons from attractive field of nucleus (visible light, is non-ionizing radiation to some extent). a wide range of secondary signal can be produced, the spectra exhibits characteristic peaks, which identify the elements present in the specimen. In analog to laser as a highly coherent source, . The diameter of e-beam in TEM is less than 5nm in general, and can be $< 0.1$ nm at best. $C_s$ correction permits the generation of smaller electron probes with higher currents, which significantly improves both analytical spatial resolution and sensitivity. $C_c$ correction offer the possibility to form band-gap imaging and chemical-bond imaging. The limiting apertures increase the depth of field for specimen, and the depth of focus for the image. And the specimen is usually in focus from top to bottom surface.\cite{Williams2009}

Point-group and space-group determination from convergent-beam patterns. crystal symmetry analysis. e-beam wavelength in metal. The high-resolution comes at the cost of poor sampling. The limitations of TEM include small sampling, interpreting of the image. Human eyes and brain understand reflected light image, and not well-trained for the transmission images. The images, DPs and spectra are all averaged through the thickness of specimen. In other words, single TEM image has no depth sensitivity. 

Both wave and particle approach, Electron scattering dictated by Coulomb interaction, Four aspects, the cross section (scattering probability), differential cross section (scattering angle), mean free path, and inelastic or elastic, non-scattering is invisible, backscattered in large angle and secondary electrons are of interest in SEM, where they provide Z contrast and surface-sensitive, topographical images. forward scattered is of interest in TEM. The spatial distribution of scattering is observed as contrast in images, and angular distribution is viewed in form of diffraction patterns. scattering events as billiard balls colliding, coherently scattered are those that remain in step, and incoherently scattered electrons have random phase relationship. Assuming single scattering events in TEM, 

angle in radians, solid angle in steradians, cross section $\sigma$ in \si{m^2}, which when divided by area of atom, represents a probability that scattering event will occur. differential cross section for an isolated atom is $\frac{d\sigma}{d\Omega} = \frac{1}{2\pi \sin\theta} \frac{d\sigma}{d\theta}$. The total cross section is just $\sigma$ times atom number in unit volume. The mean free path is the inverse of $\sigma_{total}$, that is $\lambda_{MFP} = 1/\sigma_{total}$. penetrate electron cloud, spherical wavelets, the cross section for electrons elastically scattered into angles larger than $\theta$ is $\sigma_{nucleus}= 1.62\times10^{-24} (\frac{Z}{E_0})^2\cot^2\frac{\theta}{2}$; scattering factor $f(\theta)$ for low angle ($< \sim 3^{\circ}$). The positions of diffracted e-beams determined by size and shape of unit cell, and intensities governed by distribution, number, and types of atoms in the specimen. 

XRD: X-ray scattered by electrons, electron scattered by both electrons and nuclei. Fresnel vs. Fraunhofer, high-angle scattered electron are incoherent; therefore, it can be used to form high-resolution Z-contrast image of a crystalline specimen, regardless of the orientation. Auger electron spectroscopy. EELS and XEDS constitute analytical electron microscopy (AEM). close approach the single-atom level. well suited to, energy-loss electrons cause Kikuchi lines to arise in DPs. Ionized atom enters excited state, 

\begin{itemize}
\item X-ray: characteristic X-ray for elemental analysis, Bremsstrahlung X-ray also useful for biological sample; $K_\alpha$ line from L to K transition, and $K_\beta$ from M to K transition, $L_\alpha$ line from M to L transition, Inelastic cross section, Bethe expression, X-ray energies are not identical to the ionized energy because after first emission, the atom is not in ground state until a free electron fill the last hole in the outermost shell. A cascade of transitions, Coster-Kroning transition, X-ray line shift slightly due to the chemical bonding to another atom. XEDS is not good at analyzing light elements due to the low fluorescence yield, which is strongly dependent on Z. Bremsstrahlung X-ray emission is strongly forward, 
\item SE: ejected from the conduction or valence bands, weak so only escaping if near the surface, STEM, complex cross section mechanism, 
\item Auger: Auger electron has specific energy similar to X-Ray, but is much more strongly absorbed than X-ray. Stated another way, Auger electron is hard to escape, so it is a surface sensitive. 
\item CL: spatial resolution around 100 nm, 
\item collective interaction, plasmon and phonon, plasmon excitation cross section in Lorenztian form $\frac{d\sigma_\theta}{d\Omega} = \frac{1}{2\pi a_0} \frac{\theta_E}{\theta^2 + \theta_E^2}$, where $\theta_E = E_p/2E_0$, 
\end{itemize}

 
The cross section of tungsten, moly is 

Focused ion beam (FIB) to prepare thin foils of individual gates from one of the many millions of such on a wafer. The events of electron passing through one crystal plane, The coherent length of e-beam, collection angle, 

detrimental, available means, safety as a primary concern, radiation-leak test, used in isolation, telepresence, increasingly attractive feature, give rise to, worth illustrating, remain unaffected, stated another way, line of thought, reduced and lowered, electron crystallography, interplanar spacing, intrinsic to the specimen, 

\section{Raman and UV-Vis Spectroscopy}

\url{http://www.kayelaby.npl.co.uk/chemistry/3_8/3_8_5.html}

Raman spectroscopy, a common vibrational spectroscopy, is based on the inelastic scattering of a monochromatic excitation source in energy range of 100 to 4000 \si{cm^{-1}}. Being nondestructive and requiring minimal preparation, 
it is an excellent tools to assess lattice dynamics and fingerprint species.\cite{McCreery2000} 

Some common symbols for symmetry representations are as following:
\begin{enumerate}
\item $A$ representation indicates that the functions are symmetric with respect to rotation about the principal axis of rotation.
\item $B$ representations are asymmetric with respect to rotation about the principal axis.
\item $E$ representations are doubly degenerate.
\item $T$ representations are triply degenerate.
\item Subscripts u and g indicate asymmetric (\emph{ungerade}) or symmetric (\emph{gerade}) with respect to a center of inversion.
\end{enumerate}

In this study, Raman measurement was performed using a confocal micro-Raman system (Horiba Scentific, Labram HR800) with excitation wavelengths of 441, 532 and 632 nm. For nanowire samples, the laser powers were kept between 0.2 and 0.3 mW, and the acquisition time was 100 s to avoid possible thermal damage. The spectral resolution is about 1 \si{cm^{-1}} and the depth resolution about 2 $\mu$m.  

A Raman pattern database can be found at \url{http://wwwobs.univ-bpclermont.fr/sfmc/ramandb2/index.html}. 

UV-Vis-NIR spectrophotometer is a useful tool to obtain absorption and reflectance from various samples. Generally speaking, the UV spectrum ranges from 100 nm to 400 nm, and visible spectrum spans from 400 nm to 750 nm. Absorption from atmospheric \ce{CO2} becomes significant below 200 nm; therefore 100 to 200 nm region is usually not measured unless some vacuum technique is applied. Near-IR ranges from 0.75 to 3 $\mu$m. The cost of fabricating a spectrometer covering the whole NIR region can be quite high. The instrument used in this thesis (Schimadzu, UV2600Plus) can measure from 220 nm to 1350 nm (5.6 eV to 0.92 eV) with an integrating sphere.

The absorption of materials can be described by Lambert-Beer law: $I = I_0 \exp(-A)=I_0\exp(-\alpha(\lambda) x)$, where $\alpha$ is the absorption coefficient in unit of \si{cm^{-1}}, and $x$ is the optical path in cm. When the sample is in liquid form, it is more convenient to use $A =\epsilon_\lambda C x$, where $C$ is the molar concentration (M = \si{\mole dm^{-3}}), and $\epsilon_\lambda$ is the extinction coefficient/molar absorptivity (\si{M^{-1} cm^{-1}}). For example, $\epsilon$ for methylene blue (MB) is $10^5 M^{-1}cm^{-1}$ at 660 nm.\cite{Mills1999} So by measuring the maximum absorbance of known sample, its concentration can be estimated accordingly. 

In the transmission mode, the reflection from materials phase boundary has been compensated with the usage of paired container and solvent. The scattering, however, cannot be eliminated this way. Usually scattering is negligible in molecular disperse media, yet should be considered in colloids or solids when the incident wavelength is comparable to the particle dimension. The size of nanostructures studied in this work is estimated to be 200 to 400 nm after sonication treatment; therefore, scattering effect should be included in the absorbance spectrum. 

In fact, there have been ongoing efforts to retrieve particle size profile from the dynamic light scattering, a technique also known as photon correlation spectroscopy. The experimental principle is as following: the sample in liquid solution is illuminated by a laser source and the fluctuations of the scattered light are detected at a known scattering angle $\theta$ by a fast photon detector. The fluctuations arise from the Brownian motion of small particles, equivalent to a random variation of scatter distance. By auto-correlation analysis, the decay rate of scattering intensity correlation can be related to the diffusion coefficient of small particles in the solvent.\cite{Maret1987} This approach, however, will not be pursued in this work. Instead, the scattering effect is partially circumvented by using diffuse reflection technique. 

Diffuse reflection spectroscopy (DRS) serves as another tool to estimate the sample band gap using Kubelka-Munk theory (KM).\cite{Tandon1970} The original KM theory was proposed in 1931,\cite{Kubelka1931} and has been popular in the color-related industry, such as painting, pigments, and paper. Some key points were recapitulated in this thesis.

The sample is under isotropic diffuse illumination. And the upward flux $J$ and downward flux $I$ are characterized by the K-M scattering and absorption coefficients denoted as $S$ and $K$, respectively; that is
\begin{align}
\ud I &= - (K + S)I \ud z + SJ\ud z,\\
\ud J &= + (K + S)J \ud z - SI\ud z.
\end{align}
Quantities $S$ and $K$, in unit of percentage of light scattered and absorbed per unit vertical length, have no direct physical meaning. For example, $S$ and $K$ depend on illumination geometry: diffuse or collimated. In the limit of infinite thick sample, the KM equation is given by
\begin{equation}
f(R_\infty) = \frac{(1-R_\infty)^2}{2R_\infty} = K/S,
\end{equation}
where $R_\infty$ is relative reflectance between sample and standard. In case of dilute species, one can approximate the KM function as $f(R) \propto \frac{\epsilon c}{s}$. When the diffuse scattering condition is fulfilled, $K$ can be related to absorption coefficient $\alpha$ by $K = 2\alpha$. If the scattering coefficient $S$ is fixed with respect to $\lambda$, one can then obtain
\begin{equation}
(f(R_\infty) h \nu)^n \propto (hv - E_g),
\end{equation}
where $n$ depends on the nature of electronic transition. The diffuse reflection spectroscopy is more proper to characterize nanomaterials because compared with UV-Vis transmission measurement for sample dispersed in liquid media, it takes the scattering effect into consideration. Favorable results have been obtained on the comparison of band gaps using KM model and other methods.\cite{Tandon1970,Morales2007} 

In this thesis, UV-Vis and diffuse reflectance spectra were recorded using Schimadzu, UV2600Plus. For UV-Vis measurement in transmission mode, the as-synthesized sample was removed from substrates by light sonication(Branson 1510R-MTH, 70W) in ethanol or DI water for 15 seconds. The dispersion was left for 12 h to enable the possible sedimentation. Then the dispersion was transferred into one 10 mm quartz cuvette (Thorlabs, W005654) for absorption measurement with another paired cuvette containing pairing liquid only. To carry out diffuse reflectance measurement, integrating sphere (ISR2600Plus) was used with an incident angle of 0 deg. The baseline reflectance was first recorded with the standard white plate (barium sulfate powder, \ce{BaSO4}) placed at the exit window on the sample path side, then the target sample was set in for relative reflectance measurement. The \ce{BaSO4} powder should be replaced periodically, otherwise the surface might turn yellow. To replace the \ce{BaSO4} powder, one should supply new powder into the sample holder for several time, each followed by a compact pressing using the glass rod. Chemical wrapping paper can be inserted between the glass rod and \ce{BaSO4} powder to prevent the powder sticking. Nanostructured sample could be mounted directly with the growth substrate, or ground and spread evenly onto \ce{BaSO4} surface.


\section{Substrates and Pre-growth Treatments}\label{ch2sub}

Silicon and silicon dioxides on Si wafer ($p$-Si(100),Unversity Wafer inc.) were primarily used, and other substrates (i.e. Mica\footnote{ \ce{K(Al2)(Si3Al)O10(OH)2}}, glass (Fisher Scientific, microscope slide, 12-549),\footnote{Typical composition is 72.6\% \ce{SiO2}, 0.8\% \ce{B2O3}, 1.7\% \ce{Al2O3}, 4.6\% \ce{CaO}, 3.6\% \ce{MgO} and 15.2\% \ce{Na2O}} stainless steel) were occasionally adapted. Substrates were first cut into rectangular pieces of certain size and then ultrasonically cleaned (Branson 1510R-MTH) with acetone and alcohol for about 15 minutes each followed by blow-drying with nitrogen gas. After solution cleaning, the Si surface is hydrophobic. Depending on the specific experimental requirement, sometimes a hydrophilic surface is desired. A plasma cleaning was used (Kurt J Lesker: Plasma-Preen 862) to render a hydrophilic Si surface. The treatment was done at 2 Torr \ce{O2} for 3 minutes. In addition, substrates can be coated with a thin layer of metal before growth. This process will be covered  in Sec.~\ref{sec:mag}.

oxygen plasma treatment on HF-etched Si (001). reaction among $e$, \ce{O^+}, \ce{O2^+}, \ce{O^-},\ce{O2}. \ce{OH}-terminated surface obtained.\cite{Habib2010}

\subsection{Magnetron Sputtering}\label{sec:mag}

Sputtering, a process in which atoms are ejected from a solid target material by bombarding it with energetic particles, is a well established physical vapor deposition (PVD) process with a high degree of controllability. The high energy and controllable parameters of sputtering can result in the growth of well-structured and crystalline films. Further, sputtering can be easily implemented as a roll-to-roll process for large-scale manufacturing. It is widely utilized for deposition of \ce{WO_x} in industry.

In this work, magnetron sputtering (Denton Vacuum Desk IV) was used to coat thin layers of metals onto cleaned substrates, such as W, Mo, Au, Ag, Pd, and Pt. As shown in Fig.~\ref{fig:ch2magsp}, a high potential created between the target (cathode) and substrate ionizes the argon molecules. These ions are accelerated to bombard the target foils, knocking out active atoms which subsequently deposit onto the substrates. A magnetic field is applied to confine electrons to the vicinity of the foils. These electrons in turn increase the ionization yield. 

\begin{figure}[htb]
\centering
\includegraphics[width=0.8\textwidth]{magsp.jpg}
\caption[magnetron sputtering system]{Schematic drawing of sputtering adopted from ref\cite{Song2008}}
\label{fig:ch2magsp}
\end{figure}

Sputtering is a better PVD technique than e-beam deposition since it alleviates the adverse effects in the latter. \ce{Ar+} ions are accelerated to target foils. The pressure of \ce{Ar+} has two functions: to sputter off the target ions, and to influence the mean free path (MFP). These collisions between target ions and \ce{Ar+} leads to almost all arrival angle, thus uniform coverage. The sputter yield is number of target ions released per ions hitting on target. If \ce{Ar+} ion energy is less than 100 eV, the yield is zero; If larger than 10 keV, implantation of \ce{Ar+} into target foils occurs. Usually yield between 1 and 2 is desired. The MFP in sputtering is on order of 3 cm (assuming pressure 30 mTorr), with source-to-substrate distance as 5 cm, average number of collisions is about 5/3$\sim$2. 


\section{data processing tools}

Image J.\cite{Schneider2012} TEM lattice measurement. 

Fityk. \cite{Wojdyr2010}

XRD pattern, Spectra(EDX, Raman, UV-Vis) redraw using Origin Pro 8.0. 















