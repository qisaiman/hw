%\providecommand{\setflag}{\newif \ifwhole \wholefalse}
\setflag
\ifwhole\else

    \documentclass[12pt,letterpaper,oneside]{book}

    %-------page layout--------%
% adapted from <http://www.khirevich.com/latex/page_layout/>
%\usepackage[DIV=14,BCOR=2mm,headinclude=true,footinclude=false]{typearea}

%\makeatletter
%\if@twoside % commands below work only for twoside option of \documentclass
%    \newlength{\textblockoffset}
%    \setlength{\textblockoffset}{12mm}
%    \addtolength{\hoffset}{\textblockoffset}
%    \addtolength{\evensidemargin}{-2.0\textblockoffset}
%\fi
%\makeatother

% packages used in uncc-thesis

%\RequirePackage{ifthen}
%\RequirePackage{setspace} % for double spacing
%\RequirePackage{comment}
%\RequirePackage{epsfig}
%\usepackage{sectsty} % for sectional header style. Alternative: titlesec package
% \usepackage{tocloft}
%\usepackage{geometry}

\usepackage{microtype} % better layout
%----inherent of article class-------%
%\usepackage[utf8]{inputenc} % set input encoding (not needed with XeLaTeX)

%--- for font ----
% \usepackage[T1]{fontenc}
% \usepackage{textcomp}

%\usepackage{mathptmx} % fine but not truetype
% \usepackage{newtxtext}
% \usepackage{pslatex} % not bad

\usepackage{fontspec} % to compile with LuaLatex
\setmainfont{Times New Roman} % to compile with LuaLatex

 %-- end of font adaption

\usepackage{graphicx} % support the \includegraphics command and options
% check pdftex option
\usepackage{placeins} %

\usepackage{subfig}
\usepackage{float} % for images
 \graphicspath{{./gallery/}} %--added by author

\usepackage[font=small, labelfont=bf]{caption}
%\usepackage{subcaption}


% It supplies a landscape environment, and anything inside is basically rotated.(http://en.wikibooks.org/wiki/LaTeX/Page_Layout)
%\usepackage{lscape}
\usepackage{pdflscape}
%\usepackage{rotating} % use \begin{sidewaystable}

% Helps format tables using the \toprule, \midrule, and \bottomrule commands (http://en.wikibooks.org/wiki/LaTeX/Tables#Using_booktabs)
\usepackage{booktabs}
%\usepackage{multirow} for multirow in tables
%  Helps format tables (http://en.wikibooks.org/wiki/LaTeX/Tables#Using_array)
%\usepackage{array}

%%-- Additional style---- modified by Tao Sheng 12/20/12
% for chemical formula, subscripts etc
\usepackage[version=3]{mhchem}
\usepackage{siunitx}
  \DeclareSIUnit \torr{Torr}
%%-- mathmatical symbols and equations-----
\usepackage{amsmath}
\usepackage{amssymb}
 %\numberwithin{equation}{section}
 %\numberwithin{figure}{section}
 \providecommand*{\ud}{\mathrm{d}}

%------- bibliography and citation ------
\usepackage[english]{babel}% Recommended
\usepackage{csquotes}% Recommended
\usepackage[style=numeric-comp,
		    sorting=nty,
            hyperref=true,
            url=false,
            isbn=false,
            backref=true,
            maxcitenames=2,
            maxbibnames=4,
            block=none,
            backend=bibtex,
            natbib=true]{biblatex}
% \usepackage[bibencoding=latin1]{biblatex}

\DefineBibliographyStrings{english}{%
    backrefpage  = {see p.}, % for single page number
    backrefpages = {see pp.} % for multiple page numbers
}
% suppress 'in:'
\renewbibmacro{in:}{%
  \ifentrytype{article}{}{\printtext{\bibstring{in}\intitlepunct}}}
% document preamble
% removes period at the very end of bibliographic record
\renewcommand{\finentrypunct}{}
% removes pagination (p./pp.) before page numbers
\DeclareFieldFormat{pages}{#1}


\providecommand*{\bibpath}{E:/spring2012/Ubuntu/Latex/Mendeley_Bib_lib}
\bibliography{\bibpath/arix.bib,\bibpath/ECD.bib,\bibpath/tungsten_newandgood.bib,\bibpath/ACSnano.bib,%
\bibpath/tungsten_old.bib,\bibpath/Raman.bib,\bibpath/Molybdenum.bib,%
\bibpath/tungsten_cl.bib,\bibpath/optics,\bibpath/CVD.bib,\bibpath/sodium.bib,\bibpath/VLS.bib}

%-- for works around, packages conflicts----

%-- redefine toc macros ------
\addto\captionsenglish{%
\renewcommand\chaptername{CHAPTER}%
\renewcommand\appendixname{APPENDIX}%
\renewcommand\indexname{INDEX}%
\renewcommand{\contentsname}{TABLE OF CONTENTS}%
\renewcommand{\listfigurename}{LIST OF FIGURES}%
\renewcommand{\listtablename}{LIST OF TABLES}%
}

%-- misc---
\usepackage{lipsum}
\usepackage{latexsym}
 \providecommand*{\thefootnote}{\fnsumbol{footnote}}

\usepackage{xcolor}
\usepackage{listings}
\lstset{
 frame = single,
 language = matlab,
 breaklines = true,
postbreak=\raisebox{0ex}[0ex][0ex]{\ensuremath{\color{red}\hookrightarrow\space}}
}

\usepackage{enumitem}
\setlist{nolistsep}

\setcounter{secnumdepth}{3} % show numbering of subsubsection

%% -- links ---
\usepackage{hyperref}
\hypersetup{
colorlinks,%
citecolor=black,%
filecolor=black,%
linkcolor=black,%
urlcolor=black
} % make all links black

\usepackage[acronym,toc,nonumberlist]{glossaries} % loaded after hyperref


    %\input{tweak.tex}
    %\input{commando.tex}
    %\input{font}

    \begin{document}

\fi  %  comment out when assembling

\chapter{introduction}

\section{Background and Motivation: Nanomaterials for Energy Applications}

Transition metals generally refer to any element in the d-block of periodic table, and \glspl{tmo} mean the chemical compounds that consist of at least one transition metal element and oxygen. Similarly, \glspl{tmdc} are compounds that combine at least one transition metal with chalcogen elements from group VIA (S, Se, Te, Po).\footnote{Oxygen also belongs to group VIA, but is often treated separately.} Both tungsten (W) and molybdenum (Mo) belong to Group VIB transition metal with outer shell electrons configuration as $4d^55s^1$ and $5d^46s^2$, respectively. This study focused on the following three materials: molybdenum oxide (\ce{MoO3}), tungsten oxide (\ce{WO3}), and tungsten disulfide (\ce{WS2}).
 
The two metal oxides, \ce{MoO3} and \ce{WO3}, have been studied for decades for energy related applications, e.g., photoelectrochemistry,\cite{Su2010} photocatalysis,\cite{Watcharenwong2008, Macphee2010} photovoltaics,\cite{Coridan2013} and electrochromism.\cite{Yoshimura1985, Mortimer2011} \ce{WS2} has also been investigated in various morphologies, i.e., thin films, nanotubes, and 2D atomic layers recently. \ce{WS2} exhibits potentials in solar cells,\cite{Britnell2013} water cleavage,\cite{Ballif1999} and field effect transistors.\cite{Perkins2013} Heterostructures formed between \gls{tmo} and \gls{tmdc} provide another degree of freedom to fine tune the band structures, which could enhance the solar energy harvesting performances.\cite{Chen2011b} 

\subsection{Electrochromism for Energy Saving}

Commercial buildings in U.S.\ consumed approximately 35\% of the electricity usage.\cite{us2008} And traditional windows in the buildings account for up to 40\% of the total heating, cooling, and lighting consumption, which is of low energy efficiency. Smart window is an energy efficient answer by allowing for separate control of the visible and NIR illumination/transmission. Several solutions have been proposed in both academia and industry to develop smart windows, including liquid crystal, suspended particles, and electrochromic materials.\cite{Lampert1998} Previous efforts show the electrochromic materials is the most promising component in smart window design in comparison to other options,\cite{Deb2008} in that it requires little power to switch between bleaching and coloration status (changing the optical transmission) and the failure mode is being clear instead of opaque, which is of great importance in commercial applications.

Electrochromism, where the optical property of the materials is changed upon applying low DC voltage (\textless 5 V), has been found in numerous inorganic and organic substances, e.g., \ce{V2O5}, \ce{WO3}, and \ce{NiO} for the former group, and prussian blue and polyaniline for the latter one. This study covers two inorganic metal oxides, \ce{WO3} and \ce{MoO3}. Detailed electrochromic mechanism will be discussed in Chapter 1; a brief review of electrochromism study on \ce{WO3} is presented here. 
\begin{figure}[htb]
\centering
\subfloat[]{\label{fig:introecd}\includegraphics[width=0.45\textwidth]{ecd_bw}}\hspace{0.04\textwidth}
\subfloat[]{\label{fig:introtran}\includegraphics[width=0.45\textwidth]{ECD_trans}}
\caption[ECD prototype and its transmission properties]{ECD prototype and its transmission properties. (a) Schematic diagram of \gls{ecd}, and (b) optical transmission in color and bleached states, reproduced from \cite{Lampert1998}}
\label{fig:introec}
\end{figure}

Prototype \gls{ecd} usually consists of a stacked configuration, as shown in Fig.~\ref{fig:introecd}. Typical transmission spectra of the prototype \gls{ecd} based on \ce{WO3} thin film are shown in Fig.~\ref{fig:introtran},\cite{Lampert1998} where optical transimission $T$ in bleaching and coloration states exhibit dynamic changes. The reflectance spectra of \ce{WO3} \gls{ecd} strongly depends on the device design.\cite{OBrien1999} As to the absorption, it was deduced that the infrared part was absorbed to a large extent, which was radiated again. Assuming the emission property of \gls{ecd} being that of the blackbody, Fig.~\ref{fig:introbb} displays the wavelength distribution of spectral radiance, where the spectra span from 3 $\mu$m up to 30 $\mu$m, with peaks around $8\sim10$ $\mu$m. Further thermal engineering is needed to better regulate this infrared radiation. 

\begin{figure}[htb]
\centering
\includegraphics[width=0.5\textwidth]{blackbody1.pdf}
\caption[Spectral radiance of blackbody at 0, 50 and 100 \si{\degreeCelsius} predicated by Planck's Law]{Spectral radiance of blackbody at 0, 50 and 100 \si{\degreeCelsius} predicated by Planck's Law.}
\label{fig:introbb}
\end{figure}

The successful deployment of \gls{ecd} based on \ce{WO3} relies on several factors:\cite{Granqvist2000}
\begin{itemize}
\item low cost transparent conductive materials with low resistivity;
\item low cost ion electrolyte of poor electron mobility;
\item high yield growth method of \ce{WO3} and integration with other manufacturing process; 
\item life cycle $> 10^5$ for long time outdoor usage.
\end{itemize}
The deficiency of \gls{ecd} based on amorphous or polycrystalline \ce{WO3} film was the accumulation of ions at the film surface preventing further intercalation,\cite{Dini1996} which led to low coloration efficiency and short life time. In contrast, nano-structured single crystalline \ce{WO3} could bypass this limitation, as detailed in Sec.~\ref{sec:nanoadv}.

\subsection{Photocatalysis for Solar Energy Harvesting}

Photocatalytic activity occurs at the semiconductor-electrolyte interface where charge carriers (electrons or holes) are produced in a semiconductor illuminated by photons of energy exceeding that of the band gap, then drift and generate free radicals, i.e. \ce{OH.}, enabling further reactions. A much desired goal is water splitting using photocatalytic activity from solar energy. Semiconductor is in the center of photoelectrochemical cell design for water splitting, e.g. the band gap should be within $2.0 \sim 3.0 $ eV, and \gls{cb} edge should be more negative than reduction potential of \ce{H^+/H2}, whereas the \gls{vb} top should be more positive than the oxidation potential of \ce{H2O}/\ce{O2}.\cite{Osterloh2008,Wang2012} The valence band holes (\ce{h^+}) oxidize water to oxygen, and conduction band electrons propel hydrogen generation, as depicted in Eq.~\ref{eq:h2o}.
\begin{subequations}\label{eq:h2o}
\begin{align}
4\ce{h^+} +  2\ce{H2O} &\rightarrow \ce{O2} + 4\ce{H^+} \label{eq:hervb}\\
4\ce{H^+} +  4e^- &\rightarrow 2\ce{H2} \label{eq:hercb}
\end{align}
\end{subequations}

% wo3 vs NHE
\begin{figure}[htb]
\centering
\includegraphics[width=0.6\textwidth]{woxnhe.jpg}
\caption[Band position of \ce{WO3} versus NHE]{Band position of \ce{WO3} in contact with aqueous electrolyte at pH 1. Reproduced from Ref.\cite{Gratzel2001}}
\label{fig:woxnhe}
\end{figure}
As shown in Fig.~\ref{fig:woxnhe}, \ce{WO3} has \gls{cb} edge positioned slightly more positive than the reduction potential of \ce{H^+/H2} versus \gls{nhe} and \gls{vb} edge much more positive than the oxidation potential of \ce{H2O}/\ce{O2}; therefore, photo-cleavage of water cannot be accomplished by \ce{WO3} alone. Nevertheless, a tandem cell approach by \ce{WO3} film and dye-sensitized \ce{TiO2} has been demonstrated with an efficiency of 4.5\%.\cite{Michael1999} It is worth noting that green plants also have two photosynthetic systems connected in series, one for oxidation of water into oxygen and the other for fixation of carbon dioxide. Moreover, favorable oxygen evolution of \ce{WO3} has brought good performance in degradation of organic compounds.\cite{Hepel2001,Luo2001,Watcharenwong2008} The formation of long-lived holes is recognized as a key requirement.\cite{Pesci2011} Besides, \ce{WO3} is remarkably stable in acid (pH $<5$), making it a significant candidate for treating water pollutants caused by organic acids.\cite{Monllor-Satoca2006}

\subsection{Advantages of Nano-Engineering}\label{sec:nanoadv}

Several differences are noted between bulk materials and nanostructured materials. Nanostructured materials possesses gigantic surface area in comparison to its bulk counterpart of the same volume; for instance, given 1 gram \ce{SiO2} nanoparticles in diameter of 10 nm and assuming the density being 1.2 \si{g\per cm^3}, the estimated surface area will be $1\times10^6$ \si{cm^2}. This huge surface-to-volume ratio is generally a favorable feature for catalytic reactions, which predominately occur on the surface.\cite{Sha2009} Other features due to nanostructure include 
\begin{enumerate*}[label=\itshape\alph*\upshape)]
\item surface energy state, thereby band structure modification;
\item quantum confinement effect; and
\item different charge-separation and transport mechanism.
\end{enumerate*} Overall, these features usually promise enhanced performance. 

In regarding to the contents of this study, electrochromic applications based on \gls{tmo} nanowires will benefit from the reduced diffusion depth by eliminating the deficiency of ions accumulation in amorphous oxide thin film devices, which in turn leads to shorter response time and higher coloration efficiency.\cite{Huang2008a,Scherer2012} And photocatalytic applications using \gls{tmo} nanostructures could harvest the huge surface area on which unsaturated bonds reside, thereby enabling a much higher efficiency.\cite{Mills1993,Merki2011,Chen2011b}

To fully use nano-engineering, one must just be as aware of the limitations as one is of its advantages; for example, an obvious disadvantage is the difficulty of device fabrication using nanomaterials due to the size and requirements of new integration processes. There also exist some uncertainty as to the energy band modification present in nanostructures, which could be either useful or not depending on the application.\cite{Wang2012}

\section{Scope and Dissertation Outline}
To fully harvest the advantages of these \gls{tmo} and \gls{tmdc} one-dimensional nanomaterials, a scalable growth method and a comprehensive understanding of the growth-structure-property relation must be established in the first place. This study primarily focuses on the \ce{WO3} and \ce{MoO3} nanostructure synthesis, the growth mechanism study, crystal structures characterization, and optical properties measurement.

The materials studied in this dissertation are tungsten oxides (\ce{WO3}), molybdenum oxides (\ce{MoO3}),and tungsten disulfide (\ce{WS2}). Experimental tools will be mentioned in Chapter 2, including the CVD apparatus and substrate treatments for growth, electron microscopes and optical spectrometers for characterization. Chapter 3 is devoted to tungsten oxides (\ce{WO3}) and sodium tungsten oxides (\ce{Na5W14O44}) nanowires, where the role of sodium in metallic tungsten precursors are thoroughly investigated; a potentially scalable growth method is also developed. Chapter 4 turns to molybdenum oxide (\ce{MoO3}) growth using alkali metal compounds as catalysts. In Chapter 5, a heterostructure based on \ce{WO3} \glspl{nw} is studied. Emphasis will be placed on the characterization of individual \ce{WO3}-\ce{WS2} core-shell \gls{nw}. This dissertation is concluded with summary and future work. 
