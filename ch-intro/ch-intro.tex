%\providecommand{\setflag}{\newif \ifwhole \wholefalse}
\setflag
\ifwhole\else

    \documentclass[12pt,letterpaper,oneside]{book}

    %-------page layout--------%
% adapted from <http://www.khirevich.com/latex/page_layout/>
%\usepackage[DIV=14,BCOR=2mm,headinclude=true,footinclude=false]{typearea}

%\makeatletter
%\if@twoside % commands below work only for twoside option of \documentclass
%    \newlength{\textblockoffset}
%    \setlength{\textblockoffset}{12mm}
%    \addtolength{\hoffset}{\textblockoffset}
%    \addtolength{\evensidemargin}{-2.0\textblockoffset}
%\fi
%\makeatother

% packages used in uncc-thesis

%\RequirePackage{ifthen}
%\RequirePackage{setspace} % for double spacing
%\RequirePackage{comment}
%\RequirePackage{epsfig}
%\usepackage{sectsty} % for sectional header style. Alternative: titlesec package
% \usepackage{tocloft}
%\usepackage{geometry}

\usepackage{microtype} % better layout
%----inherent of article class-------%
%\usepackage[utf8]{inputenc} % set input encoding (not needed with XeLaTeX)

%--- for font ----
% \usepackage[T1]{fontenc}
% \usepackage{textcomp}

%\usepackage{mathptmx} % fine but not truetype
% \usepackage{newtxtext}
% \usepackage{pslatex} % not bad

\usepackage{fontspec} % to compile with LuaLatex
\setmainfont{Times New Roman} % to compile with LuaLatex

 %-- end of font adaption

\usepackage{graphicx} % support the \includegraphics command and options
% check pdftex option
\usepackage{placeins} %

\usepackage{subfig}
\usepackage{float} % for images
 \graphicspath{{./gallery/}} %--added by author

\usepackage[font=small, labelfont=bf]{caption}
%\usepackage{subcaption}


% It supplies a landscape environment, and anything inside is basically rotated.(http://en.wikibooks.org/wiki/LaTeX/Page_Layout)
%\usepackage{lscape}
\usepackage{pdflscape}
%\usepackage{rotating} % use \begin{sidewaystable}

% Helps format tables using the \toprule, \midrule, and \bottomrule commands (http://en.wikibooks.org/wiki/LaTeX/Tables#Using_booktabs)
\usepackage{booktabs}
%\usepackage{multirow} for multirow in tables
%  Helps format tables (http://en.wikibooks.org/wiki/LaTeX/Tables#Using_array)
%\usepackage{array}

%%-- Additional style---- modified by Tao Sheng 12/20/12
% for chemical formula, subscripts etc
\usepackage[version=3]{mhchem}
\usepackage{siunitx}
  \DeclareSIUnit \torr{Torr}
%%-- mathmatical symbols and equations-----
\usepackage{amsmath}
\usepackage{amssymb}
 %\numberwithin{equation}{section}
 %\numberwithin{figure}{section}
 \providecommand*{\ud}{\mathrm{d}}

%------- bibliography and citation ------
\usepackage[english]{babel}% Recommended
\usepackage{csquotes}% Recommended
\usepackage[style=numeric-comp,
		    sorting=nty,
            hyperref=true,
            url=false,
            isbn=false,
            backref=true,
            maxcitenames=2,
            maxbibnames=4,
            block=none,
            backend=bibtex,
            natbib=true]{biblatex}
% \usepackage[bibencoding=latin1]{biblatex}

\DefineBibliographyStrings{english}{%
    backrefpage  = {see p.}, % for single page number
    backrefpages = {see pp.} % for multiple page numbers
}
% suppress 'in:'
\renewbibmacro{in:}{%
  \ifentrytype{article}{}{\printtext{\bibstring{in}\intitlepunct}}}
% document preamble
% removes period at the very end of bibliographic record
\renewcommand{\finentrypunct}{}
% removes pagination (p./pp.) before page numbers
\DeclareFieldFormat{pages}{#1}


\providecommand*{\bibpath}{E:/spring2012/Ubuntu/Latex/Mendeley_Bib_lib}
\bibliography{\bibpath/arix.bib,\bibpath/ECD.bib,\bibpath/tungsten_newandgood.bib,\bibpath/ACSnano.bib,%
\bibpath/tungsten_old.bib,\bibpath/Raman.bib,\bibpath/Molybdenum.bib,%
\bibpath/tungsten_cl.bib,\bibpath/optics,\bibpath/CVD.bib,\bibpath/sodium.bib,\bibpath/VLS.bib}

%-- for works around, packages conflicts----

%-- redefine toc macros ------
\addto\captionsenglish{%
\renewcommand\chaptername{CHAPTER}%
\renewcommand\appendixname{APPENDIX}%
\renewcommand\indexname{INDEX}%
\renewcommand{\contentsname}{TABLE OF CONTENTS}%
\renewcommand{\listfigurename}{LIST OF FIGURES}%
\renewcommand{\listtablename}{LIST OF TABLES}%
}

%-- misc---
\usepackage{lipsum}
\usepackage{latexsym}
 \providecommand*{\thefootnote}{\fnsumbol{footnote}}

\usepackage{xcolor}
\usepackage{listings}
\lstset{
 frame = single,
 language = matlab,
 breaklines = true,
postbreak=\raisebox{0ex}[0ex][0ex]{\ensuremath{\color{red}\hookrightarrow\space}}
}

\usepackage{enumitem}
\setlist{nolistsep}

\setcounter{secnumdepth}{3} % show numbering of subsubsection

%% -- links ---
\usepackage{hyperref}
\hypersetup{
colorlinks,%
citecolor=black,%
filecolor=black,%
linkcolor=black,%
urlcolor=black
} % make all links black

\usepackage[acronym,toc,nonumberlist]{glossaries} % loaded after hyperref


    %\input{tweak.tex}
    %\input{commando.tex}
    %\input{font}

    \begin{document}

\fi  %  comment out when assembling

\chapter{introduction}

\section{Background and Motivation: Nanomaterials for Energy Applications}

Transition metal generally refers to any element in the d-block of periodic table, and \glspl{tmo} mean the chemical compounds that consist of at least one transition metal element and oxygen. Similarly, \glspl{tmdc} are compounds that combine at least one transition metal with chalcogen elements from group VIA (S, Se, Te, Po). Oxygen also belongs to group VIA, but is often treated separately. This study focused on the following three materials: molybdenum oxide, tungsten oxide and tungsten disulfide. Both tungsten (W) and molybdenum (Mo) belong to Group VIB transition metal, with outer shell electrons configuration as $4d^55s^1$ and $5d^46s^2$, respectively. 
 
The two metal oxides, \ce{MoO3} and \ce{WO3}, have been studied for decades for energy related applications, e.g., photoelectrochemistry,\cite{Su2010} photocatalysis,\cite{Watcharenwong2008, Macphee2010} photovoltaics\cite{Coridan2013} and electrochromism.\cite{Yoshimura1985, Mortimer2011} \ce{WS2} has also been investigated in various forms, i.e., thin film, nanotube, and 2D atomic layer recently. \ce{WS2} exhibits some potentials in solar cell, water cleavage, and field effect transistor. 



\subsection{Electrochromism for Energy Saving}

Commercial buildings in U.S. consumed approximately 35\% of the electricity use.\footnote{U.S. Department of Energy, "Energy Efficiency Trends in Residential and Commercial Buildings", 2008.} And traditional windows in the buildings account for up to 40\% of the total heating, cooling and lighting consumption. The smart windows is an energy efficient solution by allowing for separate control of the visible and NIR illumination/transmission.

Electrochromic materials are a group of functional materials whose coloration status can be reversibly switched using a finite durance of low DC voltage (\textless 5 V). Another important advantage of smart windows based on EC materials is the failure mode being clear instead of opaque compared to other techniques, e.g, liquid crystal, suspended particles. 

coloration efficiency, the doping amount, 

some calculation on doping, uv-vis results on the thin film. 
 

\subsection{Photocatalysis for Energy Harvesting}

Photocatalytic activity occurs when a semiconductor in aqueous solution is illuminated by photons of energy larger than the bandgap, then electron-hole pairs generate free radicals, i.e., \ce{OH.}, which enable further reactions. For water splitting using solar energy, the band gap should be within $2.0 \sim 3.0 $ eV, and CB edge should be more negative than reduction potential of \ce{H^+/H2}, whereas the VB top should be more positive than the oxidation potential of \ce{H2O}/\ce{O2}.\cite{Osterloh2008,Wang2012} The valence band holes (\ce{h^+}) oxidize water to oxygen and conduction band electrons propel hydrogen generation, as depicted in Eq.~\ref{eq:h2o}.
\begin{subequations}\label{eq:h2o}
\begin{align}
4\ce{h^+} +  \ce{H2O} &\rightarrow \ce{O2} + 4\ce{H^+} \label{eq:hervb}\\
4\ce{H^+} +  4e^- &\rightarrow 2\ce{H2} \label{eq:hercb}
\end{align}
\end{subequations}
As shown in Fig.~\ref{fig:woxnhe}, \ce{WO3} has \gls{cb} edge positioned slightly more positive than reduction potential of \ce{H^+/H2} versus \gls{nhe}, and \gls{vb} edge much more positive than the oxidation potential of \ce{H2O}/\ce{O2}. So the photo-cleavage of water cannot be accomplished by \ce{WO3} alone. Nevertheless, a tandem cell approach by \ce{WO3} film and dye-sensitized \ce{TiO2} has been demonstrated with an efficiency of 4.5\%.\cite{Michael1999} It is worth noting that green plants also have two photosynthetic systems connected in series, one for oxidation of water into oxygen and the other for fixation of carbon dioxide.
% wo3 vs NHE
\begin{figure}[htb]
\centering
\includegraphics[width=0.7\textwidth]{woxnhe.jpg}
\caption[Bands positions of \ce{WO3} versus NHE]{Bands positions of \ce{WO3} in contact with aqueous electolyte at pH 1, adapted from Ref.\cite{Gratzel2001}}
\label{fig:woxnhe}
\end{figure}
Moreover, favorable oxygen evolution of \ce{WO3} brings good performance in degradation of organic compounds\cite{Hepel2001,Luo2001,Watcharenwong2008}. The formation of long-lived holes is recognized as a key requirement.\cite{Pesci2011} Besides, \ce{WO3} is remarkably stable in acid, making it a significant candidate for treating water pollutant caused by organic acids.\cite{Monllor-Satoca2006}

For photochemical water reduction to occur, the flat-band potential of the semiconductor (for highly doped semiconductors, this equals to the bottom of the conductance band) must exceed the proton reduction potential of 0.0 V vs NHE at pH=0.\cite{Osterloh2008} Flat-band potentials strongly depend on ion absorption (protonation of surface hydroxyl groups), crystallographic orientation of the exposed surface, surface defects, and surface oxidation processes. 

\subsection{Advantages of Nano-Engineering}

why nano? surface-to-volume ratio, more surface area for catalytic reaction; surface energy state: tuned by dimension; quantum confinement effects: exciton size vs physical dimension. easy for dopant diffusion, thereby band structure modification; charge-separation and transport mechanism may also differ from bulk.

how to show the crystalline TMO is better than amorphous thin film counterparts.

In comparison to oxides thin films, 1D \gls{tmo} nanostructures (e.g., nanowires) have following advantages to improve the response speed and enhance the coloration efficiency: (1) With reduced diameters down to nanometer scale, the distance for ion diffusing into and out of the nanowires is significantly shortened. (2) The gap between nanowires always provides channels for fast ion diffusion outside the nanowires. (3) The large surface area to bulk volume ratio of nanowires is desired to improve the overall usage of material, which results in enhanced the contrast between colored and bleached states. Furthermore, doped with certain metals (such as Ti and Ag), the crystalline structure of the TMO nanowires can be modified with enhanced electrochromic properties.\cite{Xiong2008} 

\section{Objective and Scope}
To fully harvest the advantages of these \gls{tmo} and \gls{tmdc} one-dimensional nanomaterials, a scalable growth method and a comprehensive understanding towards the growth-structure-property relation must first be established. Considering the nature of this Ph.D.\ study, this dissertation primarily focuses on synthesis, growth mechanism study and characterization, and will only present preliminary results on device and applications.

\section{Dissertation Outline}
The materials studied in this dissertation are tungsten oxides (\ce{WO3}), molybdenum oxides (\ce{MoO3}),and tungsten disulfide (\ce{WS2}). Experimental tools will be introduced in Chapter 2, including CVD apparatus and substrate treatments for growth, electron microscopes and optical spectrometers for characterization. Chapter 3 is devoted to tungsten oxides (\ce{WO3}) and sodium tungsten oxides (\ce{Na5W14O44}) nanowires, where the role of sodium in metallic tungsten precursors are thoroughly investigated; a potentially scalable growth method is also developed. Chapter 4 will turn to molybdenum oxide (\ce{MoO3}) growth using alkaline metal compounds as catalysts. In Chapter 5, a heterostructure based on \ce{WO3} \glspl{nw} is studied. Emphasis will be placed on the characterization of individual \ce{WO3}-\ce{WS2} core-shell \gls{nw}. The dissertation is then concluded with a summary and future work. 
