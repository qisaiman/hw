%\providecommand{\setflag}{\newif \ifwhole \wholefalse}
\setflag
\ifwhole\else

    \documentclass[12pt,letterpaper,oneside]{book}

    %-------page layout--------%
% adapted from <http://www.khirevich.com/latex/page_layout/>
%\usepackage[DIV=14,BCOR=2mm,headinclude=true,footinclude=false]{typearea}

%\makeatletter
%\if@twoside % commands below work only for twoside option of \documentclass
%    \newlength{\textblockoffset}
%    \setlength{\textblockoffset}{12mm}
%    \addtolength{\hoffset}{\textblockoffset}
%    \addtolength{\evensidemargin}{-2.0\textblockoffset}
%\fi
%\makeatother

% packages used in uncc-thesis

%\RequirePackage{ifthen}
%\RequirePackage{setspace} % for double spacing
%\RequirePackage{comment}
%\RequirePackage{epsfig}
%\usepackage{sectsty} % for sectional header style. Alternative: titlesec package
% \usepackage{tocloft}
%\usepackage{geometry}

\usepackage{microtype} % better layout
%----inherent of article class-------%
%\usepackage[utf8]{inputenc} % set input encoding (not needed with XeLaTeX)

%--- for font ----
% \usepackage[T1]{fontenc}
% \usepackage{textcomp}

%\usepackage{mathptmx} % fine but not truetype
% \usepackage{newtxtext}
% \usepackage{pslatex} % not bad

\usepackage{fontspec} % to compile with LuaLatex
\setmainfont{Times New Roman} % to compile with LuaLatex

 %-- end of font adaption

\usepackage{graphicx} % support the \includegraphics command and options
% check pdftex option
\usepackage{placeins} %

\usepackage{subfig}
\usepackage{float} % for images
 \graphicspath{{./gallery/}} %--added by author

\usepackage[font=small, labelfont=bf]{caption}
%\usepackage{subcaption}


% It supplies a landscape environment, and anything inside is basically rotated.(http://en.wikibooks.org/wiki/LaTeX/Page_Layout)
%\usepackage{lscape}
\usepackage{pdflscape}
%\usepackage{rotating} % use \begin{sidewaystable}

% Helps format tables using the \toprule, \midrule, and \bottomrule commands (http://en.wikibooks.org/wiki/LaTeX/Tables#Using_booktabs)
\usepackage{booktabs}
%\usepackage{multirow} for multirow in tables
%  Helps format tables (http://en.wikibooks.org/wiki/LaTeX/Tables#Using_array)
%\usepackage{array}

%%-- Additional style---- modified by Tao Sheng 12/20/12
% for chemical formula, subscripts etc
\usepackage[version=3]{mhchem}
\usepackage{siunitx}
  \DeclareSIUnit \torr{Torr}
%%-- mathmatical symbols and equations-----
\usepackage{amsmath}
\usepackage{amssymb}
 %\numberwithin{equation}{section}
 %\numberwithin{figure}{section}
 \providecommand*{\ud}{\mathrm{d}}

%------- bibliography and citation ------
\usepackage[english]{babel}% Recommended
\usepackage{csquotes}% Recommended
\usepackage[style=numeric-comp,
		    sorting=nty,
            hyperref=true,
            url=false,
            isbn=false,
            backref=true,
            maxcitenames=2,
            maxbibnames=4,
            block=none,
            backend=bibtex,
            natbib=true]{biblatex}
% \usepackage[bibencoding=latin1]{biblatex}

\DefineBibliographyStrings{english}{%
    backrefpage  = {see p.}, % for single page number
    backrefpages = {see pp.} % for multiple page numbers
}
% suppress 'in:'
\renewbibmacro{in:}{%
  \ifentrytype{article}{}{\printtext{\bibstring{in}\intitlepunct}}}
% document preamble
% removes period at the very end of bibliographic record
\renewcommand{\finentrypunct}{}
% removes pagination (p./pp.) before page numbers
\DeclareFieldFormat{pages}{#1}


\providecommand*{\bibpath}{E:/spring2012/Ubuntu/Latex/Mendeley_Bib_lib}
\bibliography{\bibpath/arix.bib,\bibpath/ECD.bib,\bibpath/tungsten_newandgood.bib,\bibpath/ACSnano.bib,%
\bibpath/tungsten_old.bib,\bibpath/Raman.bib,\bibpath/Molybdenum.bib,%
\bibpath/tungsten_cl.bib,\bibpath/optics,\bibpath/CVD.bib,\bibpath/sodium.bib,\bibpath/VLS.bib}

%-- for works around, packages conflicts----

%-- redefine toc macros ------
\addto\captionsenglish{%
\renewcommand\chaptername{CHAPTER}%
\renewcommand\appendixname{APPENDIX}%
\renewcommand\indexname{INDEX}%
\renewcommand{\contentsname}{TABLE OF CONTENTS}%
\renewcommand{\listfigurename}{LIST OF FIGURES}%
\renewcommand{\listtablename}{LIST OF TABLES}%
}

%-- misc---
\usepackage{lipsum}
\usepackage{latexsym}
 \providecommand*{\thefootnote}{\fnsumbol{footnote}}

\usepackage{xcolor}
\usepackage{listings}
\lstset{
 frame = single,
 language = matlab,
 breaklines = true,
postbreak=\raisebox{0ex}[0ex][0ex]{\ensuremath{\color{red}\hookrightarrow\space}}
}

\usepackage{enumitem}
\setlist{nolistsep}

\setcounter{secnumdepth}{3} % show numbering of subsubsection

%% -- links ---
\usepackage{hyperref}
\hypersetup{
colorlinks,%
citecolor=black,%
filecolor=black,%
linkcolor=black,%
urlcolor=black
} % make all links black

\usepackage[acronym,toc,nonumberlist]{glossaries} % loaded after hyperref


    %\input{tweak.tex}
    %\input{commando.tex}
    %\input{font}

    \begin{document}

\fi  %  comment out when assembling

\chapter{introduction}

\section{Background and Motivation: Nanomaterials for Energy Applications}
\today

TMO as electrochromic device and TMDC as newly 2D semiconductor, and some VLS.
Transition metal oxides (TMOs) exhibit rich structures and useful properties, and could been used in solar energy harvesting (\emph{e.g.}, photoelectrochemistry, photocatalysis, and photovoltaics) and energy saving (\emph{e.g.}, electrochromism and photochromism) applications. 

\iffalse
Materials that human can make define the age they live in. From Stone Age to Bronze Age and Iron Age, people evolve as mastering more and more sophisticated techniques of manipulating metals, such as alloying and annealing. Obtaining extreme high purity of silicon brings us into Information Age. Future is difficult to predict. But nanotechnology is one direction that we can not ignore. According to \gls{nni}, \gls{nanotechnology}. This definition alludes that dimension comes before compositions. It is often related to the quantum confinement or surface area in nanomaterials, which we will later revisit with specific scenario.

There are three states of matter under usual conditions: solid,liquid and gas. Solids materials could be further categorized into five groups: metals, ceramics, polymers, semiconductors, and composites.\cite{William2009} This classification is based on both composition and mechanical, electrical, and thermal properties as well as the associated functionality(i.e., \gls{ceramics} are typically hard yet brittle, insulating to electricity and resistant to heating).

\fi
Energy saving and harvesting. 

why nano? surface-to-volume ratio, more surface area for catalytic reaction; surface energy state: tuned by dimension; quantum confinement effects: exciton size vs physical dimension. easy for dopant diffusion, thereby band structure modification; charge-separation and transport mechanism may also differ from bulk.

how to show the crystalline TMO is better than amorphous thin film counterparts.

Commercial buildings in U.S. consumed approximately 35\% of the electricity use.\footnote{U.S. Department of Energy, "Energy Efficiency Trends in Residential and Commercial Buildings", 2008.} And traditional windows in the buildings account for up to 40\% of the total heating, cooling and lighting consumption. The smart windows is an energy efficient solution by allowing for separate control of the visible and NIR illumination/transmission.  

\subsection{Electrochromism for Energy Saving}

coloration efficiency, the doping amount, 

Electrochromic materials are a group of functional materials whose coloration status  can be reversibly switched using a finite durance of low DC voltage (\textless 5 V). Another important advantage of smart windows based on EC materials is the failure mode being clear instead of opaque compared to other techniques, e.g, liquid crystal, suspended particles. 

\subsection{Photocatalysis for Energy Harvesting}



\subsection{Advantage of Nano-engineering}

In comparison to oxides thin films, 1D TMO nanostructures (e.g., nanowires) have following advantages to improve the response speed and enhance the coloration efficiency: (1) With reduced diameters down to nanometer scale, the distance for ion diffusing into and out of the nanowires is significantly shortened. (2) The gap between nanowires always provides channels for fast ion diffusion outside the nanowires. (3) The large surface area to bulk volume ratio of nanowires is desired to improve the overall usage of material, which results in enhanced the contrast between colored and bleached states. Furthermore, doped with certain metals (such as Ti and Ag), the crystalline structure of the TMO nanowires can be modified with enhanced electrochromic properties.\cite{Xiong2008} 



\section{Dissertation Outline}
% what content should be presented

The materials studied in this work/dissertation are tungsten oxides (\ce{WO3}), molybdenum oxides (\ce{MoO3}),and their chalcogenide counterparts (\ce{WS2} and \ce{MoS2}). 

This dissertation primarily focuses on growth and characterization, and will only present some preliminary tests on device and applications. Experimental tools for growth and characterization will be introduced in Chapter 2. Chapter 3 will be devoted to tungsten oxides (\ce{WO3}) and sodium tungsten oxides (\ce{Na5W14O44}) nanowires, where the role of sodium compounds in metallic tungsten precursors are thoroughly investigated. Chapter 4 will turn to molybdenum oxides (\ce{MoO3}) growth using alkaline metal compounds as catalysts. In Chapter 5, the growth of \gls{tmdc} and heterostructures will be discussed. Emphasis will be placed on the characterization of individual \ce{WO3}-\ce{WS2} core-shell \gls{nw}. The dissertation is then concluded with a summary and future work. 

\iffalse

We have synthesized \gls{tmo} and \gls{tmdc} at nanoscale, measured their crystalline structures and optical properties and demonstrated some devices assembled using as-synthesized nanomaterials. We aim to illustrate that by nanoengineering these \gls{tmo} and \gls{tmdc}, enhanced performances over their bulk states could be expected and new properties will arise. In the remaining sections of this chapter, we will discuss some general perspectives of nanomaterials, the growth apparatus and characterization methods that apply to all experiments done in this work. Then chapter 2 will focus on growth of \ce{WO3} and its derivative. We employed thermal \gls{cvd} to synthesize \ce{WO3} \gls{nw}, and we investigated the role of impurity in tungsten metallic powders, during which we observed a new state of sodium tungsten oxides: \ce{Na5W14O44} nanowires. We also found a method to potentially obtain large yield of \ce{WO3} \gls{nw}. Chapter 3 will concentrate on \ce{MoO3}. We explored two different growth mechanism of \ce{MoO3}:\gls{vs} and \gls{vls}. We discovered that alkaline oxides can be used as catalyst to grow two distinct \ce{MoO3} morphologies: nanobelts and towers. We further demonstrated the application of as-synthesized \ce{MoO3} nanomaterials in electrochromic devices.  In chapter 4 we discuss  We synthesized  and inspected the growth of \gls{fl} \ce{WS2}. Chapter 5 will conclude with an overall summary.

\fi




%\input{../end.tex} %  comment out when assembling 