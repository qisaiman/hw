
\chapter{INTRO ON THE CONTROLLED GROWTH AND PROPERTIES MEASUREMENTS OF 1D NWS}
\chaptermark{header version}

\section{KK, optics modeling, doping}

how to show the crystalline TMO is better than amorphous thin film counterparts.

In comparison to oxides thin films, 1D \gls{tmo} nanostructures (e.g., nanowires) have following advantages to improve the response speed and enhance the coloration efficiency: (1) With reduced diameters down to nanometer scale, the distance for ion diffusing into and out of the nanowires is significantly shortened. (2) The gap between nanowires always provides channels for fast ion diffusion outside the nanowires. (3) The large surface area to bulk volume ratio of nanowires is desired to improve the overall usage of material, which results in enhanced the contrast between colored and bleached states. Furthermore, doped with certain metals (such as Ti and Ag), the crystalline structure of the TMO nanowires can be modified with enhanced electrochromic properties.\cite{Xiong2008}

optical properties of \ce{WO3} gaps\cite{Saygin-Hinczewski2008}

soda lime glass. thermal radiative transfer through windows, \cite{Rubin1985} \ce{FeO} absorb infrared, and \ce{Fe2O3} in ultraviolet,  

The phase of amplitude reflection coeffs
\[
\phi(\omega) = \frac{1}{2\pi} \int_0^\infty \frac{d\log r}{d\omega'}\log(abs(\frac{\omega' + \omega}{\omega' - \omega}))d\omega'
\]

two python projects available, KKcalc\cite{Watts2014} and hyperSpy. 
http://www1.univap.br/gaa/nkabs-database/data.htm

KK relations in optical materials research. 
\begin{align}
N(\omega) &= [N(-\omega)]^* \\
N(\omega) - 1 &\approx -\frac{\omega_p^2}{2\omega^2}\\
r(\omega) = \frac{N(\omega)-1}{N(\omega)+1} &= \frac{n-1+ik}{n+1+ik}\\
r(\omega) &= |r(\omega)| \exp[i\theta(\omega)]\\
\log r(\omega) &= \log|r(\omega)| + i\theta(\omega)
\end{align}

phase retrieval, 
Principal value:
\[
P\cdot\int_0^\infty f(x)dx = \lim_{\delta\rightarrow 0} (\int_0^{a-\delta} f(x)dx + \int_{a+\delta}^\infty f(x)dx)
\]

Maclaurin methodology, \cite{Rocha2014}. The singularity, 

\[
\nu_1, \nu_2, \nu_3, \ldots, \nu_m;\\
k_1,k_2,
\]

\[
n(\nu) = n_0 + 2/\pi \int_{\nu_1}^{\nu_2} \frac{\nu' k(\nu')}{\nu'^2 - \nu^2} d\nu'
\]
discretized as

\[
I_i = n_0 + 2/\pi \times 2h \times \sum_j f_j,
\]
where
\[
f_j = 0.5*k_j (\frac{1}{\nu_j - \nu_i} + \frac{1}{\nu_j + \nu_i})
\]

dispersive and dissipative mode, sum rules given. \cite{King1979}
\[
F(\omega) = \frac{\log r(\omega)}{\omega + i\omega'}
\]



Concentration of charge carrier $n$ is related to gate voltage $V_g$ by:
\[
n = \frac{\epsilon_0 \epsilon V_g}{ed}
\]
where $\epsilon_r = \epsilon_0 \epsilon$ is dielectric constant of gate materials.

intrinsic silicon equilibrium charge carrier concentration at RT is $n_i = p_i = 1.5 \times 10^{10} cm^{-3}$, much smaller than silicon atoms density as $5\times 10 ^{22} cm^{-3}$.

The average distance between dopant atoms is cubed root of concentration, $d = (10^{18})^{-1/3}$ \si{cm^{-3}}$^{-1/3}$ = 10 nm.

The electron mobility $\mu_n = 1500 cm^2/V\cdot sec $ at RT for Si, and hole mobility $\mu_p = 450 cm^2/V\cdot sec$ at RT.

for p-type silicon, when the conductivity $\sigma = 1 (ohm cm )^{-1}$, the doping level is
$N_A = \frac{\sigma}{q \mu_p}= 1 / (1.6E-19 \times 450)$ = 1.4E16 \si{cm^{-3}}.

Built-in voltage $V_0 = \frac{kT}{q}ln(N_A N_D/n_i^2)$, depletion region width $W = \sqrt{\frac{2 \epsilon_{Si} V_0}{q}(1/N_A + 1/N_D)}$, where $\epsilon_{Si} = 11.7 \epsilon_0$. When applying external field, depletion width $W = \sqrt{\frac{2 \epsilon_{Si} (V_0 - V) }{q}(1/N_A + 1/N_D)}$

The capacitance of p-n junction is $C = A \sqrt{\frac{q \epsilon_{Si}}{2(V_0 -V)}(N_D N_A/(N_A + N_D))}$.



\iffalse
Materials that human can make define the age they live in. From Stone Age to Bronze Age and Iron Age, people evolve as mastering more and more sophisticated techniques of manipulating metals, such as alloying and annealing. Obtaining extreme high purity of silicon brings us into Information Age. Future is difficult to predict. But nanotechnology is one direction that we can not ignore. According to \gls{nni}, \gls{nanotechnology}. This definition alludes that dimension comes before compositions. It is often related to the quantum confinement or surface area in nanomaterials, which we will later revisit with specific scenario.

There are three states of matter under usual conditions: solid,liquid and gas. Solids materials could be further categorized into five groups: metals, ceramics, polymers, semiconductors, and composites.\cite{William2009} This classification is based on both composition and mechanical, electrical, and thermal properties as well as the associated functionality(i.e., \gls{ceramics} are typically hard yet brittle, insulating to electricity and resistant to heating).

\fi

\iffalse

We have synthesized \gls{tmo} and \gls{tmdc} at nanoscale, measured their crystalline structures and optical properties and demonstrated some devices assembled using as-synthesized nanomaterials. We aim to illustrate that by nanoengineering these \gls{tmo} and \gls{tmdc}, enhanced performances over their bulk states could be expected and new properties will arise. In the remaining sections of this chapter, we will discuss some general perspectives of nanomaterials, the growth apparatus and characterization methods that apply to all experiments done in this work. Then chapter 2 will focus on growth of \ce{WO3} and its derivative. We employed thermal \gls{cvd} to synthesize \ce{WO3} \gls{nw}, and we investigated the role of impurity in tungsten metallic powders, during which we observed a new state of sodium tungsten oxides: \ce{Na5W14O44} nanowires. We also found a method to potentially obtain large yield of \ce{WO3} \gls{nw}. Chapter 3 will concentrate on \ce{MoO3}. We explored two different growth mechanism of \ce{MoO3}:\gls{vs} and \gls{vls}. We discovered that alkaline oxides can be used as catalyst to grow two distinct \ce{MoO3} morphologies: nanobelts and towers. We further demonstrated the application of as-synthesized \ce{MoO3} nanomaterials in electrochromic devices.  In chapter 4 we discuss  We synthesized  and inspected the growth of \gls{fl} \ce{WS2}. Chapter 5 will conclude with an overall summary.

\fi
\section{crystal growth}

Nucleation is a process of generating a new phase from a metastable old phase, where the Gibbs energy per molecule of the bulk of the emerging new phase is less than that of the old phase.

General CVD knowledge, substrate preparation, and\cite{MichealK.Zuraw2003}

\section{band and bond}

Solid and orderliness.

Two theories arise to describe the outer shell electrons and to correlate the structure and physical properties: \gls{cft} and band theory.\cite{Goodenough1971} \gls{cft} assumes weak interaction between neighboring atoms and localization of electron towards parent atom, whereas band theory assumes that electron is shared equally by all nuclei and therefore a many-electron problems follows. Description of a single electron in periodic potential fail to treat the electron correlations adequately, as the interaction between atoms becomes weaker. For transition metals, $s$ and $p$ electrons are well described by a collective-electron model, while the 4f or 5f electrons are tightly bound to nuclei and screened from the neighboring atoms by 5s, 5p or 6s, 6p core electrons, hence it matches well with a localized-electron model. d electrons show intermediate character.


\chapter{experimental}

\section{diffuse reflection}

The KM two-flux model has been extensively used to describe the optical properties of inhomogeneous materials that consist of scattering and absorbing particles in a matrix.\cite{Vargas1997} There are some limitation in the assumptions made in original KM model, e.g., the distribution of scattering on the optical path. And revised KM models have been proposed to eliminate these limitations.\cite{Yang2004}  

\citeauthor{Morandi2005} studied the absorbance and diffuse reflection of \ce{MoO3} and \ce{WO3} thin films using FT-IR and UV-Vis spectroscopy. The observed absorption in visible and middle IR region were attributed to oxygen defects induced donor levels.\cite{Morandi2005}

The ratio between absorption and scattering, not the absolute value of each quality, determines the reflection features.


\iffalse

hemispherical reflectance, 
diffusive reflectance is a mathematical artifice without direct physical meaning. 

p332-338

the reflecting power associated with a specular-type process is often termed reflectivity, while reflectance is the analogous term for diffusely reflected radiation. This two types of reflection are present simultaneously. 

Cauchy formula, empirical equation which has subsequently been placed on a firm theoretical basis. 

intensity for a complex quantities is obtained by multiplying the complex conjugate. 

For strongly absorbing materials, the reflected light is complementary to the color of light transmitted by it, such as gold transmission is green; 

For the weakly absorbing substances of rough surface, it will appear the same color by transmission or reflection. 

penetration depth is wavelength dependent $d = \frac{In2 \lambda}{2\pi \sqrt{\sin^2\theta_i - n^2}}$, longer wavelength penetrated further than did shorter wavelength. attenuated total reflectance, 

thin film, amorphous, mean free path of electron is diminished, and porous in structure. The maxwell-garnett theory provides a satisfactory explanation, where free electron density is reduced. 

diffuse reflection: $I = I_0 \exp(-\epsilon d)$, Lambert cosine law formulated. multiple scattering by individual particles. 
$R_tot = \alpha R_sp + (1- \alpha) R_diff $, 

multiple scattering of diffuse radiation in a medium composed of closely packed particles. 

\[
\log{f(r_\infty) = \log{k} - \log{s}}
\]

It is concluded that the KM function can be used to obtain the characteristic color curve of a given sample only in the limit of small total absorption. 

by mixing the sample with an abundance of the reflectance standard, the regular reflection is extensively eliminated. 

The color of an object depends on the spectral composition of the source. The color of many light sources can be specified in terms of the temperature to which a blackbody radiator must be heated in order to achieve a color match for the source, which is known as color temperature. 

integrating sphere: the intensity at any part of the sphere, due to the reflected light is a measure of the total flux from a particular source, independent of the spatial distribution or location of the source on the sphere. 

substitution method, comparison method, 

\fi
\section{tem text}


\iffalse
An eucentric specimen stage is used, thus observation area remain fixed when tilting the specimen. The pressure in SEM chamber is on the order of $10^{-3} \sim 10^{-4}$ Pa, which is usually maintained by a diffusion pump, or turbo molecular pump when oil-free operation is needed. For a field emission electron source, a sputter ion pump becomes necessary due to the high vacuum requirement. 

The SEM instrument used in this study is JEOL JSM-6480 and EDX attachment from Oxford Instrument INCA. Typical observation conditions are listed as following:

\begin{enumerate}
\item SEM
\begin{itemize}

\item Acceleration voltage: 10 kV
\item Working distance: 10 mm
\item Scanning time: 80 s
\end{itemize}
\item EDX
\begin{itemize}

\item Acceleration voltage: 20 kV
\item Working distance: 10 mm
\item Dead time: $20\sim30$\%
\end{itemize}
\end{enumerate}


\begin{quotation}
Since the intensity of characteristic X-rays is proportional to the concentration of the corresponding element, quantitative analysis
can be performed. In actual experiment, a standard specimen containing elements with known concentrations is used. The
concentration of a certain element in an unknown specimen can be obtained by comparing the X-ray intensities of the certain element
between the standard specimen and unknown specimen. However, X-rays generated in the specimen may be absorbed in
this specimen or excite the X-rays from other elements before they are emitted in vacuum. Thus, quantitative correction is needed.
In the present EDS and WDS, correction calculation is easily made; however, a prerequisite is required for this correction. That is,
elemental distribution in an X-ray generation area is uniform, the specimen surface is flat, and the electron probe enters perpendicular to the specimen. Actually, many specimens observed with the SEM do not satisfy this prerequisite; therefore, it should be noted that a quantitative analysis result might have appreciable errors.
\end{quotation}


\fi


\iffalse
Similar to the photon-lattice interaction in XRD, the process of TEM could be understood as electron scattering events by the same crystal plane. 

\begin{itemize}
\item X-ray: characteristic X-ray for elemental analysis, Bremsstrahlung X-ray also useful for biological sample; $K_\alpha$ line from L to K transition, and $K_\beta$ from M to K transition, $L_\alpha$ line from M to L transition, Inelastic cross section, Bethe expression, X-ray energies are not identical to the ionized energy because after first emission, the atom is not in ground state until a free electron fill the last hole in the outermost shell. A cascade of transitions, Coster-Kroning transition, X-ray line shift slightly due to the chemical bonding to another atom. XEDS is not good at analyzing light elements due to the low fluorescence yield, which is strongly dependent on Z. Bremsstrahlung X-ray emission is strongly forward, 
\item SE: ejected from the conduction or valence bands, weak so only escaping if near the surface, STEM, complex cross section mechanism, 
\item Auger: Auger electron has specific energy similar to X-Ray, but is much more strongly absorbed than X-ray. Stated another way, Auger electron is hard to escape, so it is surface sensitive. 
\item CL: spatial resolution around 100 nm, 
\item collective interaction, plasmon and phonon, plasmon excitation cross section in Lorenztian form $\frac{d\sigma_\theta}{d\Omega} = \frac{1}{2\pi a_0} \frac{\theta_E}{\theta^2 + \theta_E^2}$, where $\theta_E = E_p/2E_0$, 
\end{itemize}


The diameter of E-beam in TEM is less than 5 nm in general, and can be $< 0.1$ nm at best.

correction of spherical aberration ($C_s$) and chromatic aberration ($C_c$), $C_s$ is done by , $C_c$ by energy-filtering, which is more useful for thicker specimens. $C_s$ correction permits the generation of smaller electron probes with higher currents, which significantly improves both analytical spatial resolution and sensitivity. $C_c$ correction offer the possibility to form band-gap imaging and chemical-bond imaging. The limiting apertures increase the depth of field for specimen, and the depth of focus for the image.\cite{Williams2009} 

including controlling the interactions of electron with magnetic fields and with specimen.  

XRD: X-ray scattered by electrons, electron scattered by both electrons and nuclei. Fresnel vs. Fraunhofer, high-angle scattered electron are incoherent; therefore, it can be used to form high-resolution Z-contrast image of a crystalline specimen, regardless of the orientation. Auger electron spectroscopy. EELS and XEDS constitute analytical electron microscopy (AEM). close approach the single-atom level.  energy-loss electrons cause Kikuchi lines to arise in DPs. Ionized atom enters excited state, 

penetrate electron cloud, spherical wavelets, the cross section for electrons elastically scattered into angles larger than $\theta$ is $\sigma_{nucleus}= 1.62\times10^{-24} (\frac{Z}{E_0})^2\cot^2\frac{\theta}{2}$; scattering factor $f(\theta)$ for low angle ($< \sim 3^{\circ}$). 

Point-group and space-group determination from convergent-beam patterns. crystal symmetry analysis. e-beam wavelength in metal. The high-resolution comes at the cost of poor sampling.  Human eyes and brain understand reflected light image, and not well-trained for the transmission images.

One must be just as aware of the instrument's limitations as one is of its advantages.   TEM is initially developed to overcome the image resolution imposed by light microscopes. constitute, draw analogies, resolving power, E-beam is one type of ionizing radiation, which is capable of removing the tightly bound, inner-shell electrons from attractive field of nucleus (visible light, is non-ionizing radiation to some extent). a wide range of secondary signal can be produced, the spectra exhibits characteristic peaks, which identify the elements present in the specimen. In analog to laser as a highly coherent source, . 

Both wave and particle approach, non-scattering is invisible, backscattered in large angle and secondary electrons are of interest in SEM, where they provide Z contrast and surface-sensitive, topographical images. forward scattered is of interest in TEM.  scattering events as billiard balls colliding, coherently scattered are those that remain in step, and incoherently scattered electrons have random phase relationship. Assuming single scattering events in TEM, 

The cost of TEM adds up to \$10 per eV. seize the public's imaginations. 

The cross section of tungsten, moly is 

Focused ion beam (FIB) to prepare thin foils of individual gates from one of the many millions of such on a wafer. The events of electron passing through one crystal plane, The coherent length of e-beam, collection angle, 

\fi


\iffalse
incoherent illumination $s = \frac{1.22 \lambda_{vac}}{2n\sin i}$, 
coherent illumination in microscope $s = \frac{\lambda_{vac}}{n\sin i}$,, 
phase contrast: transforming phase change in object plane into amplitude variation in image plane. 
much better than needed, human eye resolving power 0.3 mrad, 

The illumination can be considered as incoherent (adding intensity) if the object is self-luminous, or if illuminated from all direction. 

The opposite occurs in TEM probably, where the specimen is irradiated with complete transverse coherence across its area. back 
focal plane displays the Fourier transform of complex amplitude in object plane. The optical system is a spatial low-pass filter: it builds up images from only the low Fourier components present at the object, having higher values cutoff. 

The information about an object is on display in the objective lens's back focal plane in a Fourier-transform form. dark field, removing zero order; Schlieren technique, removing half of diffraction from one side of back focal plane; applications: apodizing in telescope, satellite transmitting aerial structure, 

with the aid of Fig. 
conjugate plane, 

electron lens can be made of electrostatic field, and magnetic field, and the B field can be generated from ferromagnetic materials (soft iron, 2 T) or superconducting (100 T). 

$C_s$ has dimensions of length, and is approximately equal to the focal length of objective lens (1-3 mm in TEM). 

$r_{min} \approx 0.91(C_s \lambda^3)^{1/4}$ is about 0.3 nm typically; with $C_s$ correction, $r_{min}$ can be further reduced to 0.07 nm. And human eye can resolve a distance of 0.2 mm, therefore the maximum useful magnification is about $0.2 \times 10^{-3}/0.3 \times 10^{-9} \sim 670 K$. 

And after interacting with specimen, the energy spread of transmitted E-beam becomes about 20 eV, which will limit the resolution more than does the spherical aberration. 

\textbf{coherence of E-beam}

faint points, large disk, the feature of interest is what make the materials imperfect, 

20 keV E-beam in TV. 

In XRD, both diffraction peak position and intensity are used; whereas in TEM, most time only the positions of spots are of concern. 

physical process of diffraction where the atomic planes appear to behave as mirrors for incident E-beam; 

\fi

\section{raman}


\iffalse
A Raman pattern database can be found at \url{http://wwwobs.univ-bpclermont.fr/sfmc/ramandb2/index.html}. 
In analytical practice, frequency is expressed in reciprocal wavelength (as cm−1), called wavenumbers;
\fi


\chapter{wo3}

\ce{WO3} crystalline film ECD, electrochemical anodized, \cite{Ou2012}

metal oxide contact on work function and band structure.\cite{Greiner2013}

sodium tungsten bronzes, DRS from 335 nm to 1250 nm show \ce{Na_{0.3}WO3} exhibits absorption peak at about 650 nm.\cite{Brown1954} 

oxygen deficiency, electron carrier con about $5\times10^{18}$ \si{cm^3}, absorption edge at 2.7 eV attributed allowed transition. absorption in infrared region observed but not rigorously resolved: donor level in the forbidden gap near VB, or thermally excited electrons in CB. Absorption increase when oxygen vacancy is created. Resistivity increase as heated up. \cite{Berak1970}

A systematic study of optical and electrical properties of \ce{WO3} thin film.\cite{Deb1973} Film density 6.5 \si{g\per cm^3}, substantial porosity. \ce{WO3} becomes substoichiometric at 1300-1500 K, and evaporate at a composition near \ce{WO_{2.96}}. $K \propto (h\nu - E_g)^2/h\nu$, indicating an allowed band to band transition, and 3.25 eV energy gap determined thereby. Refractive index $\epsilon_{st}$ of thin film \ce{WO3} found at 2.2, lower than that of single crystal value of 2.5; $n$ increases to 4.0 at absorption edge region, with $\alpha$ as high as $10^5$ \si{cm^{-1}}. So the color center can form in a thin layer of 100 nm; its density estimated by the Smakula's Equation:
\[
Nf = 0.89\times10^17\frac{n}{(n^2+2)^2} \alpha d
\]
where $f$ osc strength, n ref index, $\alpha$ in \si{cm^{-1}}, and d is the half width of absorption band. Assuming $f$ = 1, N to be $5\times10^{19}$ \si{cm^3}. 
Narrow conduction band proposed to explain the high energy peak at 4.39 eV and 5.25 eV. Impurity band when the ele wavefuntion overlap, the color center spacing comparable to lattice constant. 


polycrystalline \ce{WO3} film reflectivity in NIR.\cite{Goldner1983} tentatively,  $E_p$ 4.07 eV was obtained by $R$ fitting using Drude model. 

T and R data on ECD. \cite{OBrien1999}

ECD wo3 NW for fast response time. \cite{Liao2006a}


exhibits electrochromic functionality. reversible, recoverable, electrochromism, 
coloration efficiency, the doping amount, 
ion mobility is the bottleneck, 
The optical properties modification in EC materials is related to the electron density difference. 


\citeauthor{Wang2009a} mentioned that amorphous \ce{WO3} can only be used in lithium-based electrolytes due to its in-compact structure and high dissolution rate in acidic electrolyte solutions. Electrochromic materials that can endure acidic electrolytes without degradation should be developed. Crystalline \ce{WO3} nanostructures with their much denser structures and small particle sizes are promising to be used as suitable electrochromic material in acidic electrolytes.

Raman assignment of \ce{WO3}
XRD assignment of \ce{WO3}

raman assign

\begin{table}[htb]
\centering
\caption{Raman fingerprints of tungsten oxides and sodium tungsten compounds}\label{tab:woram2}
\begin{tabular}{lp{3in}r}
\toprule
Phase & Raman Shift (\si{cm^{-1}}) &  Reference   \\
\midrule
\ce{WO2}  & 287(s), 334(w), 514(m), 600(w), 621(w), 785(vs) & \cite{Ma2005} \\
\ce{W18O49}  & broad bands from 750 to 780 & \cite{Guo2012} \\
             &  267(s), 778(s), 969(m) & \cite{Liu2013d} \\
m-\ce{WO3}  & 131(m), 265(m), 327(m), 715(s), 807(vs) &  \cite{Salje1975a,Daniel1987} \\
h-\ce{WO3}  & 162(m), 253(m), 320(m), 645, 690(s), 817(vs) &  \cite{Daniel1987}\\
\ce{WO3.H2O}  & 230(m), 377(w), 428(w), 650(s), 816(vs), 946(vs) &  \cite{Daniel1987} \\
\ce{WO3.2H2O}  & 235, 268(m), 380(w), 662(s), 685(vs), 960(vs) & \cite{Daniel1987} \\
\ce{Na2WO4}  & 94(w), 314(vw), 377(m), 813(m), 930(vs) &  \cite{Lima2011} \\
\ce{Na2W2O7}  & 381(w), 422(w), 763(w), 835(s), 886(m), 948(m), 957(vs) &  \cite{Knee1979} \\
\ce{Na2W4O13} & 949(m), 794(s), 777(vs), 366(w), 311(w), 272(w), 263(w) &\cite{Fomichev1992}\\
\ce{Na5W14O44} & 965(m), 943(m), 913(w), 765(vs), 695(m), 107(s) & this work\\
\bottomrule
\end{tabular}
Materials in solid state, 
vw-very weak; w-weak; m-medium; s-strong; vs-very strong
\end{table}

Nanoscale \ce{WO3} Raman.\cite{Boulova2002}

\ce{WO3} and sub-wo3 conductivity vs temp. \cite{Sahle1983}

oxygen vacancies in \ce{WO_{3-x}}.\cite{Wang2011b}  Coloration and electron conductivity changes. \citeauthor{Wang2011b} found strong dependence of WO3-x electronic properties on $V_O$ concentration and the the crystallographic direction on which O is removed. DFT band gap calculation is close to experimental value. Vacancy levels are found at 2.1eV.

The Raman spectra of \ce{WO_x} is rare because of the difficulty of preparing pure suboxides phase and the strong shielding of \ce{WS2}. Yet it does exhibit distinct Raman spectra. \cite{Tenne2005} The 870 line is attributed to \ce{W3O8}.\cite{Hardcastle1995}

Delichere1988 coloured states

\ce{W18O49} reflectivity, carrier density $1.87\times10^{22}$ \si{cm^{-3}}, quasi-free carrier reflection of Drude type, \cite{Brandt1981} 

xrd assignment
\begin{table}
\centering
\caption{XRD assignments}\label{tbl:xrd}
\begin{tabular}{llllll}
\toprule
\ce{WO3} &          &\ce{Na5W14O44} &      & \ce{Na2W4O13} & \\
2$\theta$   & hkl   & 2$\theta$   & hkl    & 2$\theta$   & hkl   \\
\midrule
 23.05   & 0 0 2 & 9.65    & 0 0 2  & 10.85   & 1 0 0 \\
 23.59   & 0 2 0 & 14.49   & 0 0 3  & 21.8    & 2 0 0 \\
 24.31   & 2 0 0 & 19.37   & 0 0 4  & 32.97   & 3 0 0 \\
 26.6    & 1 2 0 & 24.27   & 0 0 5  & 56.45   & 5 0 0 \\
 34.1    & 2 0 2 & 29.22   & 0 0 6  &         &       \\
         & 2 2 0 &         &        &         &        \\
 47.11   & 0 0 4 & 39.36   & 0 0 8  &         &        \\
 48.29   & 0 4 0 & 44.46   & 0 0 9  &         &        \\
 49.79   & 4 0 0 & 49.71   & 0 0 10 &         &        \\
         & 1 4 0 &         &        &         &        \\
         &       & 55.23   & 0 0 11 &         &        \\
         &       & 60.77   & 0 0 12 &         &        \\
\bottomrule
\end{tabular}
\end{table}



\subsection{polarons}

The concept of polaron was first proposed by Landau in 1933. In ionic or highly polar crystals, such as II-VI semiconductors, alkali halides and transition metal oxides, the Coulomb interaction between a conduction electron and the lattice ions results in a strong electron-phonon coupling. A new quasi-particle, virtual phonon, can be defined corresponding to the effect of electron pulling nearby positive ions towards it and pushing nearby negative ions away. The electron and its virtual phonons, taken together, can be treated as a new composite particle, called an electron polaron; the hole polaron is defined analogously. \cite{Devreese1996}

the electrical conductivity is given by $\sigma = n e \mu$, where $n$ is the density of free carriers, $\mu$ is their mobility and $e$ is electronic charge. The mobility is given by $\mu = e\tau/m^*$ with $\tau$ is carrier resistivity relaxation time and $m^*$ is the carrier effective mass.

\section{ECD}

carrier density estimation, the stoichiometry of tungsten bronze was calculation from the inserted charge by Faraday's Law. Na diffuses slower than H+, accumulating at the surface of wo3 film and preventing further intercalation.\cite{Dini1996}

\[
m = \frac{Q}{F} \frac{M}{z},
\]
where $m$ is the mass of substance at an electrode in gram, $Q$ is the total charge transferred; F is Faraday constant; $M$ is molar mass; $z$ is the valency number of ions. 


The degree of \ce{H^+} intercalation, $x$, is determined by integrating cathodic current, and given by
\begin{equation}
x = \frac{Q}{F}\frac{M_\ce{WO3}}{m_f}
\end{equation}
where $m_f$ is the mass of film, $M_\ce{WO3}$ is the molar mass of \ce{WO3*1/3H2O}, and F is Faraday's constant. 
The color center density, $c$, is then obtained using formula
\begin{equation}
c= \frac{x \rho N_A}{M_\ce{WO3}}
\end{equation}
where $N_A$ is Avogadro number, $\rho$ is density of film.

During the test of ECD, constant current source is often used. The total charge $Q$ is then $It$. Charge density, the diffusion of \ce{Na+}, 

Characterization of ECD (work like a thin-film batteries) includes transmission measurement and associated EC calculation, charge-discharge time, current-time curve and the fitting of obtained data.

The coloration efficiency (CE) represents the change in the optical density (OD) per unit charge density ($Q/A$ in unit of \si{\coulomb \per cm^2} during switching and can be calculated according to the formula:
\begin{equation}
CE = \frac{\Delta~OD}{(Q/A)} [cm^2/C],
\end{equation}
where OD = $log(T_{bleach}/T_{color})$. The EC and optical density depend on the wavelength and are usually higher in the near IR than in the visible region.
Using Ohm's law($U_s = IR = RQ/t_s$) with switch voltage $U_s$, resistance R and surface area A, switching time $t_s$ could be estimated as
\begin{equation}
t_s = \Delta~OD\cdot A \cdot R /(CE\cdot U_s).
\end{equation}


battery and ECD.\cite{Granqvist2012} electrolyte: PVB (poly vinyl buteral).
alternative materials and design: organic, Prussian Blue as EC materials, metal hydrides, suspended particle device, liquid crystal, electroplating,
challenges: large area nanoporosity, transparent conducting contact, electrolyte with good ionic conductivity and poor electronic conductivity, stable under UV; assembly and large scale manufacturing;
cathodic coloration:
anodic coloration:
The coloration mechanism: \ce{MO6} octahedrons lead to $e_g$ and $t_{2g}$ level and ion channeling.
ref54,60,65,66,200,209,

\ce{WO3} as cathodic and either polyaniline(PANI) or Prussian white (PW) as anodic electrochromic half cells. \cite{Heckner2002}

\begin{table}[htb]
\caption{Combinations of ECD configuration}\label{tb:ecd}
\begin{tabular}{lcccr}
\toprule
TC(both side) & electrochromic & ion conductor & counter electrode  & reference\\
\midrule
ITO &  \ce{WO3} & \ce{H^+\hyphen} polymer & PANI &\citeauthor{Heckner2002}\\
FTO &  \ce{WO3} & \ce{K^+\hyphen} polymer & PW &\cite{Heckner2002}\\
ITO & \ce{WO3} NWs & \ce{LiClO4\hyphen}PC & none & author design \\
\ce{Na_xWO3} NWs &\ce{WO3} NWs & \ce{LiClO4\hyphen}PC & none & author design\\
\bottomrule
\end{tabular}
\end{table}

\begin{table}[htb]
\centering
\caption{Comparison of MoOx ECD}\label{tab:moxecd}
\begin{tabular}{lcccr}
\toprule
$\lambda$ & $\Delta T$ & $t_c$ & $t_b$ & $CE$  \\
         (nm) & (\%)    & (s) & (s) & ($cm^2/C$)  \\
\midrule
  \\
\bottomrule
\end{tabular}
\end{table}

\begin{quote}
a viable electrochromic smart window must exhibit a cycling life time \textgreater $10^5$ cycles, corresponding to an operation life at 10 -- 20 years.
\end{quote}

\citeauthor{Sella1998} studied the optical and structural properties of RF sputtered thin film of \ce{WO3} and \ce{VO2} for electrochromic devices. Ionic conductor was built using transparent polymer electrolyte, which was prepared from a solution of 1 M \ce{LiClO4} in propylene carbonate which was mixed with methylmetharcylate (MMA). The main characteristics of polymer electrolyte were: viscosity at 25 \si{\degreeCelsius} $\approx$ 12920 Pa.s, conductivity $\approx 10^{-2}-10^{-4}$ \si{\per\ohm\per cm},non-hygroscopic if PMMA concentration \textgreater 30\%. A specific counter-electrode layer was not used since the encapsulated polymer electrolyte processes a very high ion storage capacity.\cite{Sella1998}

The device proposed was reproduced as shown in Fig.~\ref{fig:Sella1998ECD}
\begin{figure}[htb]
    \centering
    \includegraphics[angle=270,width=0.8\textwidth]{Sella1998ECD}
    \caption{citation, see original captions} \label{fig:Sella1998ECD}
\end{figure}


\chapter{moo3}

molybdena, mobility of the hexavalent \ce{Mo6+} ions in MoO3 modified by Na+, \cite{El-Shobaky1998} Solid-solid reaction between \ce{Na2O} and \ce{MoO3}: 
\[
\cee{Na2O + 2MoO3 \rightarrow[500 C] Na2Mo2O7}
\]
The solubility of sodium oxide in \ce{MoO3} lattices is controlled by the chemical affinity towards the formation of \ce{Na2Mo2O7}. Other phases, such as \ce{Na2Mo3O8}, \ce{Na2Mo3O4} were detected by XRD. 


MoO3 spreading on \ce{Al2O3}: solid-solid wetting process, favored by the decrease of total surface energy. 800 K heating, gas phase transport ruled out.\cite{Leyrer1990} 
\[
\Delta F = \gamma_M - \gamma_S + \gamma_{MS}
\]
where the subscript M, S denote material and substrate, respectively.
solid state reaction between MoO3 and Al2O3 can occur, resulting \ce{Al2(MoO4)3}, while reaction between MoO3 and \ce{SiO2} is not reported. 

MoO3 on Au (111).\cite{Song2003} spreading phase as molten MoO3. surface free energy of \ce{Al2O3} as , \ce{MoO3} as . 

crystal anisotropy affect the morphology and growth direction. \cite{Schwarz2009} \cite{Schwarz2010} 

\cite{Dubrovskii2011} 

\cite{Cao2012a}

\cite{Shin2012}

\cite{Rathi2013}

Si NW growth using Au and SiH4, the growth species incorporation rate into liquid appears larger than that into solid because of the catalytic reaction on the metal, not of the rough surface of liquid phase. The catalytic particles can lower the barrier that is present on the particle-solid interface as compared to that on the vapor-solid interface. As to the catalytic particle being liquid \cite{Harmand2005, Park2006} or solid \cite{Dick2005,Persson2004}, results from the growth of III-V NWs using different catalysts indicate it barely matter. \cite{Kolasinski2006} And the phase from which the growth material is taken is of little consequence. The essential role of catalyst appears to be in lowering the activation energy of nucleation at the growth interface. 

Cu-Si NW, VSS mechanism, in-situ TEM, \cite{Wen2009} Si growth direction show multiple ones instead of the major 111 direction in VLS method. solid catalyst be advantageous in fabricating heterostructure due to the low solubility of growth species in solid. 

ZnO NW by AuZn alloy, VSS occurs via a solid diffusion mechanism.\cite{Campos2008} the low-temperature growth of ZnO nanowires from solid Au–Zn catalysts as being an intermediate between the VLS and VS processes. The presence of a low-energy epitaxial interface between solid catalyst and the NW will favor NW growth. 

Size effect evaluated by cohesive energy theory.\cite{Qi2005} $T = T_b (1 - N/2n)$, where N surface atoms and n total atoms.
 
\begin{table}[htb]
\centering
\caption{melting of nanosolids}\label{tab:nano}
\begin{tabular}{ccp{2in}}
\toprule
Morphology & N/n &  Remarks   \\
\midrule
sphere     &  4d/D  & d atom diameter, D sphere diameter   \\ 
disk-like     &  4d(1/3h+ 2/3l)  & d atom diameter, l disk diameter, h disk height \\ 
nanowire     &  8d/3l & d atom diameter, l wire diameter   \\ 
\bottomrule
\end{tabular}
Note: atomic diameter calculated by atomic volume per mole.
\end{table}

For example, the atomic weight of Pb is 207 g/mol, and density is 11.34 g/cm3; then the molar volume is 207/11.34 = 18.2539. Assuming sphere as building unit, then $\frac{\pi}{6}D^3$ = 18.2539/N_A. Solving D as 0.3868 nm. 

for \ce{MoO3}, d as 0.46 nm. 

catalyst acts as a sink for Zn, generating concentration gradient. growth rate and diffusion coefficient estimation, MoO3 diffusing in liquid should be faster than in solid, this can explain the longer dimension in the length of belts than the height of towers. open issue, 
Na2MoO4-MoO3 phase diagram indicates that for T less than 500 C, the Na-Mo-O alloy will remain solid for all MoO3 concentration.


\ce{MoO3}, an alternative interpretation in terms of tetrahedral coordination of Mo atoms is also proposed. This is caused by the fact that four of the six surrounding O atom are at distances from 1.67 \si{\angstrom} to 1.95 \si{\angstrom}, while the remaining two are as far as 2.25 and 2.33 \si{\angstrom}. This also stress that the \ce{MoO6} octahedra are rather distorted.


% Melting points 
\begin{table}[htb]
\centering
\renewcommand*{\thetable}{S\arabic{table}}
\caption{physical constants of reactants }\label{tb:thermo}
\begin{tabular}{lccr}
\toprule
Material & MP(\si{\degreeCelsius}) & BP(\si{\degreeCelsius}) & reference\\
\midrule
\ce{NaOH}        & 318 & 1388 & handbook  \\
\ce{NaI}        & 651 & 1300 & MSDS    \\
\ce{KI}        & 681 & 1330 & MSDS   \\
\ce{Na2CO3}        & 851 & Not determined & MSDS    \\
\ce{Na2MoO4}        & 687 & Not available & handbook   \\
\ce{MoO3}    & 795 & 1155 & MSDS   \\
\ce{MoO2}    & 1100(decomp) & Not available & MSDS   \\
\bottomrule
\end{tabular}
\end{table}


In second state, the one with smaller droplet probably would exhibit faster growth rate in $\langle001\rangle$ direction due to its shortest lattice constant, while the other one endures more growth along$\langle010\rangle$ direction. We suspect that the liquid size that induce the tower growth is much larger than that of belts. From the final morphology of as-synthesized sample, we find that the above proposed two growth approaches could change into each other, that is belt growth could be initiated on the top of tower growth and vice visa. We assume this phenomena arise from the interaction between vapor supply, temperature and the shape of liquid catalyst. Or it may just come from the vapor transport of \ce{NaxMoO3}.
It had been shown by actively engineering the shape of catalysis, the growth direction of  InP NW could be switched between [111] and [100].\cite{Wang2013c} The same mechanism might exist in our case as well. For instance, when a belt is sufficient long and it could extend into the low temperature part, where the supply of MoOx vapor is reduced due to the consumption, and the shape of liquid might vary as well. Both will result in different absorption and subsequent diffusion rate along and inside the liquid catalyst. We also observed that belt growth could initiate from tower growth in reduced time growth. Actually this may be one possible mechanism of belts formation.

(The presence of \ce{Na6MoO33} phase is an indirect evidence for the formation of eutectic \ce{Na2MoO4}- \ce{MoO3} binary compounds at growth temperature. The bulk phase diagram may not accurately represent the phase transition occurring in catalyst droplet and solid interface. And it will usually lead to a significant temperature decrease of eutectic point from the bulk value. Nanoparticles do not completely melt and instead act as an active site for reactant absorption and diffusion, leading to a vapor-solid-solid growth mechanism. 

In vapor synthesis process, two growth mechanism exists: VS and VLS. VS process is widely accepted for the growth of \ce{MoO3}. Yet we caution that synthesis conditions should be scrutinized to determine the exact mechanism. \citeauthor{Li2002c} suggested a VS mechanism at 700 \si{\degreeCelsius} and VLS at 750 \si{\degreeCelsius} and higher.\cite{Li2002c} \citeauthor{Fibers2007} proposed a modified VS mechanism probably because the deposition occurs on \ce{Al2SiO5} with possible \ce{Al_{0.95}SiNa_{0.06}O_x} involved. Therefore temperature and possible impurity could potentially alter the growth mechanism. We divide the growth results into two categories: on Si substrate and on non-Si substrate, and describe them respectively. We also briefly mention using liquid exfoliation to prepare few layer \ce{MoO3}.

This rectangular shape implies the boundary plane along growth direction (long axis) is (001), in consistency with previous experimental reports\cite{Zeng1998,Li2002b} and theoretical studies.\cite{Firment1983,Cora1997} We also observed different shapes, i.e., elongated hexagonal using similar growth conditions. This is not an indication of different growth mode. It is a normal thermodynamic fluctuation. The growth rates along different crystalline direction of \ce{MoO3} are determined by the free surface energy. In fact, (201), (101) and (102) planes have all been observed as terminating planes.\cite{Zeng1998} The stacking rate of \ce{MoO6} octahedra along $a$ and $c$ axis could develop some other ratio. And the coexistence of different planes in one growth suggests the similarity of free surface energies between these surfaces. In other words, the migration barrier of adatoms on (010) plane is presumably much lower than that on other low index planes due to the Van der Waals interaction nature along [010] direction.


\citeauthor{Hardcastle1990} summarized an empirical formula to relate the Raman peaks and \ce{Mo-O} bonding lengths.\cite{Hardcastle1990} This correlation assumes general form as
\begin{equation}\label{eq:mobond}
\nu = A \exp{B\cdot R},
\end{equation}
where $A=32895$ and $B=-2.073$ are fitting parameters, R is bond distance in unit of \AA. Given a stretching frequency, the resolution for calculated bond distance is $\pm0.016$ \AA. Another empirical expression connect the bond valence $s$ and bond distance R: $s(M-O) \approx (R/X)^{-6} $, where X=1.882 when M is Mo, and 1.904 when M is W. The valence sum rule could be then used to check the state of Mo cation. It should be noticed that not all observed Raman lines could be correlated to a \ce{Mo-O} bond distance by extrapolation of Eq.~\ref{eq:mobond}. It is then regarded as a symmetry related vibrational mode, i.e., 820 \si{cm^{-1}} in \ce{MoO3}. From the correlation of various Mo compounds, a general conclusion is the lower the stretching frequency for the shortest metal-oxygen bond, the more regular is the structure.


\chapter{tmdc}


CNT chirality by TEM \cite{Zhang1993} TEM chirality of \ce{MoS2} NTs

\ce{WS2} NT transport \cite{Zhang2012c}
less grain boundary more mobility, 

\citeauthor{Ramasubramaniam2011} investigated the band gap tuning in bilayer TMDC materials by applying external $E$ field. Similar research has been done for graphene and bilayer boron nitride. Semiconductor-metal transition was suggested for \ce{MoS2} and \ce{WS2}, with difference on the CBM and VBM evolution. In \ce{MoS2}, the valence-band-splitting cause the A and B excitons in optical absorption measurement. Calculation shows that CB and VB are translated toward the Fermi level with increasing E field.  The external field localized charge along $c$ axis, but delocalized that within the plane normal towards $c$, thereby driving the semi-metal transition. It was mentioned that this transition is not anticipated in monolayer \ce{MoS2}. It was emphasized that precise band gaps might be different from the author’s results, yet the gap-tuning should be universal.\cite{Ramasubramaniam2011}

\cite{Song2013} \ce{WO3} by ALD, and sulfurized in Ar and \ce{H2S} (10:1) at 1000 C. \ce{WS2} layer No and peak intensities ratio under 633nm excitation is correlated. It was found the 2LA/A1g is less than 1 for 1L. In supporting info 532 nm Raman spectra, the 2LA/A1g is presumably larger than 1 for 1L. \ce{WS2} NT on Si NWs is also demonstrated.

\cite{Tenne2010} chemical modification of NTs. Functional ligand consists of an anchor group that attaches to the NTs surface and a tail which render them soluble in various solvents. PTAS functionalized BN nanotube lead to the formation of stable suspensions in aqueous solutions. The strong attachment is formed through $\pi-\pi$ interactions.

inorganic nanotubes review \cite{Tenne2004} , unsaturated bonds number increases as the size of MS2 sheet decreases.

Water splitting materials should process a band gap larger than 1.4 eV, considering both the NHE potential distance and practical application. The monolayer \ce{WS2} exhibits direct gap of 1.98 eV. Quantum confinement could push the gap separation farther away. \cite{wilcoxon1997} \citeauthor{Notley2013} use liquid exfoliation to prepare \ce{WS2} NPs.\cite{Notley2013} Non-ionic surfactant concentration is about 0.1\%w/w. Continuously adding surfactant during sonication improves the yield. Optimum surface tension is found at about 40 mJ/m$^2$.

G/\ce{WS2}/G stacked solar cell. For lubricant, and surface protection. Absorption $\sim 10^7 m^{-1}$. \cite{Britnell2013}

In Ref\cite{Zeng2013}, single crystal \ce{WS2} growth using \ce{I2} transport was described in supporting info.
\ce{C24H12K4O8}\footnote{http://www.chemspider.com/Chemical-Structure.24771386.html}

416 peak was prevously assumed to be a combination of LA and TA phonons at K points.
416 cm raman on WS2. B1u is pressure sensitive. \cite{Staiger2012} also studied pressure dependence.

A1g mode using resonance Raman excitation shows upshift, which is presumably caused by folding induced strains. And TEM SAED reveal 3R symmetry. We did not observed this upshift of A1g probably due to the large diameter and few layer involved.

Nanotube growth is relatively independent of substrates and FL layer is closely related to substrate. Film growth could provide some insight into the latter scenario.

\section{calculation}

Raman conditions: 0.3 mW, 532 nm, 200 sec acquisition time. photon flux = $0.3E-3\times6.242E+18/1.2398/0.532=2.84E15$

first demonstration of \ce{WS2} NT n-type FET. \cite{Levi2013}
the importance of contact, and avoiding moisture. The calculated carrier concentration is about $10^{19}cm^{-3}$, a highly doped semiconductor, possibly arising from sulfur vacancy. 

\ce{WS2} absorption coefficient $10^{-7}m^{-1}$, mean free path of photo-excited charge carriers 1 $\mu m$. the wave vector of photon is considerably small than size of BZ, therefore The wave vector of phonon in Raman scattering usually close to zero.

Multiple phonon scattering, For two identical phonons, the corresponding Raman peak in the spectrum is called an overtone of the peak from the corresponding one-phonon process. And the wave vector conservation rule is automatically filled, therefore the phonon involved is not limited to BZ center anymore.
\[
I(G) \approx \sum_k \frac{\langle f|H_M|b\rangle \langle b|H_{ep}|a\rangle \langle a|H_M|i\rangle}{(E_p - E_k^{\pi *}- E_k^{\pi}-i\gamma)(E_p - E_k^{\pi *}- E_k^{\pi}-i\gamma- \hbar\Omega_{G})}
\]


In short summary, this thesis stresses that there are several key factors in successfully synthesizing \ce{WS2} FL:
\begin{itemize}
\item \ce{WO3} powder size distribution and absolute amount (0.69 g \ce{WO3} in 10 mL acetone or IPA, inspired by the \ce{MoO3} nanoribbons usage);
\item Sulfur amount and heating method to ensure a constant sulfur-rich environment;
\item \ce{WO3} powder source to substrate distance, which is coupled to pressure and carrier gas flow in determining the transport (atmospheric, 3-5 mm);
\item temperature ramping and growth time (750-800 \si{\degreeCelsius}, 3-10 min);
\item 300 nm \ce{SiO2}-Si substrate to make OM flakes identification easier.
\end{itemize}

Four kinds of tungsten powders were used as precursor to prepare \ce{WO3} NWs, and sulfur powder was used to fabricate \ce{WO3}-\ce{WS2} core-shell NWs, as summarized in Table.~\ref{tab:ch5pre}. All reactants were employed as received without further processing. Substrates cleaning procedures have been introduced in Sec.~\ref{ch2sub}. 

% Tungsten powders size and purity
\begin{table}[htb]
\centering
\caption{List of reactants for \ce{WO3}-\ce{WS2}}\label{tab:ch5pre}
\begin{tabular}{lccr}
\toprule
Name & purity & average size & vendor info\\
\midrule
3N   &  99.9\% & 17 $\mu$m & Alfa Aesar \#39749\\
3N5   &  99.95\% & 32 $\mu$m  & Alfa Aesar \#42477\\
4N5   &  99.995\% & 3.3 $\mu$m  & Materion T-2049 \\
5N   &  99.999\% & 1.5 $\mu$m & Alfa Aesar \#12973\\
S    &   99.5\%  &  NA  & Alfa Aesar \#10785\\
\bottomrule
\end{tabular}
\end{table}

\ce{WO3} NWs were obtained by a seeded growth with tungsten powders applied onto the Si substrate. To find out the optimal conditions, tungsten powder oxidation experimental were carried out. 


\subsection{strain}

$E_{2g}$ mode is strain sensitive. 

\citeauthor{Ghorbani-Asl2013} studied the strain in tubular TMDC and found a linear dependence of Raman scattering on strain (3 \si{cm^{-1}} per percentage for $E_{2g}$mode).\cite{Ghorbani-Asl2013} 

For 2D materials, strain may be induced by elongation of an appropriate substrate, e.g. by uniform mechanical strain, or by using a material with high thermal expansion coefficient and varying the temperature. For TMD MWNT, tensile tests have been reported by various groups. However, to date, it is not perfectly clear whether inner and outer walls are stretched simultaneously, or rather the outer walls slide on the inner ones. The latter hypothesis would result in a broadening of the Raman signals, while the first one would leave the signal widths rather unaffected. In any case, there would be a shift of the Raman signals that can serve as precise scale for determining the strain.\cite{Ghorbani-Asl2013}


\citeauthor{Virsek2007} performed a Raman-TEM integrated study on multiwalled \ce{WS2} NT with diameter \textgreater 200 nm. The tubes were synthesized using chemical transport method. Up-shift of Raman is explained by strain in the walls. This shift is not observed in the specimen by sulfurization process of oxides. Applied hydrostatic pressure is isotropic,\cite{Staiger2012} while the strain is expected to anisotropic. Strain can also be relaxed by chirality.\cite{Virsek2007} 

strain effect by first-principles calculations. direct gap is only maintain in a narrow strain range (-1.3 -- 0.3 \%), \cite{Yun2012}.

Semiconducting to metallic transition in \ce{MoS2} at compressive strain of 15\% or tensile strain of 8\%; direct-to-indirect gap transition for 1L \ce{MoS2} at about 2\%. \cite{Scalise2012}

\section{heterojunction}

arix1410.8201: contact resistance at TMDC/metal interface due to Schottky barrier. \ce{MoS2} transistor is a Schottky barrier transistor with effective barrier height controlled by gate and drain biases. The carrier path for n-type \ce{MoS2} defined from source to drain. Thermal-assisted tunneling from source metal Fermi level to the channel. To reduce the contact resistance in \ce{MoS2} FET, 
\begin{itemize}
\item low $\phi$ contact metal
\item heterojunction with graphene 
\item heavily doping the source/drain region
\item substitute doping of \ce{Cl-} for \ce{S2-}, $\sigma = $
\end{itemize}

Fermi-level pinning at semi interface due to sulfur vacancies.\cite{Liu2013c}

\ce{MoO_{3-x}} as high workfunctio material aligned deeply into the VB of \ce{MoS2} as a p-contact,\cite{Battaglia2014} \cite{Chuang2014a} \ce{MoO_{3-x}} forms defect states within the band gap, 

PEI as n-type surface dopant, 

electron density in \ce{WS2} to be 6E11 \si{cm^{-2}} ref95,\cite{SikHwang2012}

scanning tunneling spectroscopy. \ce{WSe2} with Fermi level at 1.15 eV below conduction band. \cite{McDonnell2014} The use of binding energy shift in the core level as a measure of band bending at interfaces. \ce{MoS2} exhibit both p-type and n-type behavior due to local variation in stoichiometry. 0.3\% areal density was found to be able to dominate the electron Schottky barrier height. 

\ce{MoS2}, photoionization energy $5.47\pm0.15$ eV, \cite{Schlaf1999}, and  bandgap of 1.17 eV, deriving the electron affinity as $\chi_A= 4.3\pm0.15$ eV. 

interface dipole \cite{Lin2013a}
MX2 electrolyte photocurrent measurement, \ce{MoS2} gap 1.17 eV, \cite{Kautek1980}
 
The electron affinity rule, band offset of \ce{SnS2} on \ce{MoS2}, linear models, comparison of model with experimental results difficult due to the uncertainties or superposition of physical effects.\cite{Schlaf1999} interdiffusion, chemical reaction across the interface. X-ray photoemission spectroscopy and ultraviolet photoelectron spectroscopy, valence band offset from XPS and UPS measurements; conduction bands are not accessible since photoelectron spectroscopy (PES) only allows the measurement of occupied DOS.
and $\Delta C$ is calculated using $\Delta V$ and $E_g$. 

Some conventions used as following:
\begin{enumerate} 
\item Fermi level is zero energy;
\item binding energies are positive;
\end{enumerate} 

\begin{table}[htb]
\centering
\caption{WS2 and  \ce{WO3}}\label{tab:woram}
\begin{tabular}{lccccr}
\toprule
Name & Nc & Nv &  $E_g$ (eV)   & $\chi$ & $\epsilon_{st}$  \\
\midrule
\ce{WO3} &    &     &  2.7    &      & 50\\    
\ce{WS2} &    &     &  1.9~1.5   &      & \\    
\bottomrule
\end{tabular}
\end{table}


\subsection{method}
angle-resolved photo-emission/photo-electron spectroscopy. (ARPES)
\[
E_binding = \hbar\omega - E_k - \phi
\]
where $E_k$ is kinetic energy of outgoing electron, which can be measured. $\phi$ is electron work function towards vacuum. $\hbar\omega$ is incident photon energy, and is varied. The UV-PS spectra represent the DOS integrated over all wave vectors. Normal emission UPS, 

hall effect, 

photoelectrolysis quantum efficiency, 


\subsection{parameters for wo3 and ws2}

Lithium tunsten bronze, e mobility decrease with T; the decreasing mobility with higher T could be taken to indicate the principal mechanism for scattering the conduction carriers is by thermal motion of atoms in host lattice.  \ce{MxWO3} derived from a host of \ce{WO3} lattice doped with alkali metal M. \cite{Sienko1961}

\ce{WO3} valence band about 4 eV wide. \cite{Bringans1981}, \cite{DeAngelis1977} 

semiconductor-electrolyte interface, parallel capacitance C, flat band potential, Schottky-Mott plot,\cite{DiQuarto1981} 
\[
\frac{1}{C_p^2} = \frac{2}{\epsilon_0\epsilon e N_d} (U_E - U_{FB} - \frac{kT}{e})
\]
DOS $Neff$ as 1E20 \si{cm^{-3}eV^{-1}} and using 
\[
-qU_{FB} = E_F = E_{CB} + kT\log N_d/Neff
\]


quantum efficiency measurement of water photoelectrolysis. n-type oxide as anode, biased and forming a depleting region on the surface. $\alpha$ is proportional to $\eta$, and Tauc method was used to extrapolate band gap. \cite{Koffyberg1979}

12\% usage of solar spectrum, conductivity $\sigma = 6\times10^{-1}$ \si{\per\ohm\per\cm}, taking electron mobility as $\mu=10$ \si{cm^2\per Vs},\cite{Berak1970} the carrier density is 4E15 \si{cm^{-3}}.\cite{Butler1976} \ce{WO3} and SCE reference electrode in one cell, and Pt in the other. Pt cell was bubbled with Ar to remove dissolved \ce{O2}. pH changes shift the Fermi level of electrolyte; run a cell without applied potential is desirable, and \ce{WO3} is inferior to \ce{TiO2} in this respect. The bands are essentially flat, and depletion layer necessary for charge separation does not exist. 

low energy photons sample mainly the valence electron structure of oxygen, and high-energy photons see the tungsten contribution more strongly.\cite{Bullett2000}

\begin{table}[htb]
\centering
\caption{band gaps of \ce{WO3}}\label{tab:wobg}
\begin{tabular}{cccc}
\toprule
Material state & energy gap (eV) & probing technique   & Reference   \\
\midrule
amorphous film & 2.62(i), 3.52(d)   &    photoelectrolysis  & $\rho=0.75$ \si{\ohm cm}\cite{Koffyberg1979}\\ 
10 $\mu$m powder layer & 2.6   &   PL at liquid nitrogen & \cite{Paracchini1982b}\\ 
amorphous film & 3.1 & photo-current &\cite{DiQuarto1981} \\
amorphous film & 2.7 & photo-current &\cite{Butler1976} \\
(001) surface of \ce{WO3} crystal &  & UV photoemission spectroscopy &\cite{Bringans1981} \\
\bottomrule
\end{tabular}
\end{table}



\subsection{calculation results}



\chapter{to be used}

\cite{Matar2011} Using electronegativity $\chi$ and chemical hardness $\eta$ to assess electron affinity $E_a$, work function $W_f$, Fermi energy $E_f$ and band gap $E_g$.
\begin{align}
\chi &= 0.5(W_f + E_a)\\
\eta & = 0.5(W_f - E_a)
\end{align}
where I is ionization potential and $E_a$ is electron affinity.


Correlation between optical band gap and formation enthalpy; reaction occurs in order to form compounds with a larger gap.  $E_g = A \exp(0.34E_{\Delta H^0})$, and A adopts different values depending on the metal elements:
\begin{itemize}
\item A=0.8 for s and f block elements,
\item A = 1 for d block elements,
\item A = 1.35 for p block elements.
\end{itemize}

\section{MB}

MB is a heterocyclic aromatic dye which is blue colored in oxidizing environment. Upon reduction, MB is turned into colorless leuco MB. This can be used as an oxygen indicator in food industry. Photo-bleaching of MB can be also due to its leuco formation rather than total decomposition. Photocatalytic decomposition can be minimized by keeping the solution at acidic condition (pH = 4), which will limit the formation of oxidative hydroxyl radicals (E = 2.8 eV vs normal hydrogen electrode). Oxygen dissolved in the solution play a key role in conversion of LMB to MB under visible light. Purging with \ce{N2} for 20 min


\textbf{\ce{WS2}-\ce{WO3}}: 1 kW light source(Hg, or Xe lamp), photon flux, phenol (\ce{C6H5OH}, 94.1g/mol, MP 40C)concentration is 20 mg/L, hydroxyl group. The quantitative analysis of phenol was performed via a standard colorimetric method.\footnote{\url{http://omlc.ogi.edu/spectra/PhotochemCAD/html/072.html}}
\citeauthor{DiPaola1999} prepared \ce{WS2}-\ce{WO3} mixture in two methods, sulfurization of \ce{WO3} and oxidation of \ce{WS2},with the latter are more active.
\citeauthor{DiPaola1999} also concluded that the actual efficiency of mixed \ce{WS2}-\ce{WO3} catalysts depends on the ratio of each composition present of the surface of the particles, and the maximum of photoactivity is obtained with 40-50\% surface molar ratio of \ce{WS2}.

ref 25, 28 and 41.

\citeauthor{Sreedhara2013} studied the kinetics of photodegradation of methylene blue\footnote{\ce{C16H18N3SCl},319.8 g/mol, MP: 100C accompanied with decomposition \url{http://en.wikipedia.org/wiki/Methylene_blue}} dye by few layer \ce{MoO3}.
For the photodegradation method, it was stated that `` the samples were collected after the photoreaction had been centrifuged for 5 min to remove the photocatalyst before UV-Vis measurement.''


Raman \cite{Xiao2007}. Silver has the strongest SERS enhancement due to the larger imaginary part of the dielectric constant and higher thermal conductivity. Milli-Q grade water ((Milli-pore)\textgreater 18.2Mohm).

MB Raman peaks: 445, 1618, ref20. some peak splitting and shift observed on SERS, attributed to chemical adsorption. definition of Raman enhancement factor.

SERR MB on Ag. \cite{Nicolai2003}
MB: the absorption spectrum in VIS is used to infer about different adsorbed forms of MB. the formation of large aggregates. 

 can remove dissolved oxygen. \cite{Wang2014a}

solar energy harvesting representative study.\cite{Yoneyama1972} MB to LMB (\ce{C16H19N3S}) in aqueous solution upon illumination of \ce{TiO2}. The colorimetric analysis was performed in a glove box under nitrogen atmosphere. The absence of oxygen is important to prevent the oxidation of LMB to blue MB.
\[
\cee{MB^+ H2O + H^+ \rightarrow MBH3^{2+} + 1/2O2}
\]
where MB represents the uncharged center of MB molecule.

common wisdom expect that a dye incapable of injecting an electron at the excited state to CdS. MB, which process N-methyl groups in its molecular structure and does not sensitize CdS is an exemplary candidate. quantum efficiency is defined as probability of MB converted to azure B per incident photon. QE of CdS to MB decomposition is reduced in nitrogen bubbling treated solutions, indicating the necessity of oxygen. Two possible mechanisms: a) adsorbed oxygen acts as a trap for the conduction electron and prevent the accumulation of negative charge within space charge region of CdS, supported by the formation of \ce{O2^-} in excitation of CdS in aqueous suspension.\cite{Takizawa1978}

ref 16, MB aqueous solution stability. Liquid chromatogram, azure B (trimethylthionine), and thionine. Electrochemical measurement,

MB adsorption.  photocatalytic oxidation of MB by \ce{TiO2} film. photo-oxidation reaction occurs at the surface of photocatalyst. Mb molar extinction coefficient was found to be 66700 1/cm 1/M. Langmuir adsorption isotherm.\cite{Matthews1989}

\[
[MB]_{ads} = \frac{k_1 k_2 [MB]}{1 + k_1[MB]}
\]
and integrated form of Langmuir adsorption isotherm
\[
t = \frac{1}{k_1K} In\frac{[S]^0}{[S]} + \frac{1}{K}([S]^0 - [S])
\]
where $K = k_2 \phi N T_r$, with $\phi$ as quantum yield, N as total absorbed photons, and $T_r$ as rate of transport.
\[
\cee{C16H18N3SCl + 25.5O2 \rightarrow 16CO2 + 6H2O + 3HNO3 + H2SO4 +HCl}
\]
which indicates the total oxidation of $10 \mu M$ MB would exhaust the ambient oxygen concentration of initially air-equilibrated solutions (about $250 \mu M$ ). ref 28 Thus the transport of both oxygen and MB to the photocatalyst surface are anticipated to be key factors.

photoelectrochromism at \ce{TiO2}/MB interface and its control. Efficient capture of photogenerated holes by a reducing agent is crucial to the reversibility of bleach-recoloration transition. This transition is kinetically dictated by electron transfer. Holes transfer is not desired.\cite{DeTacconi1997}

256 nm band is associated to the presence of LMB. LMB formation is not favored at alkaline pH values in aqueous solution. The OH radicals are generated either with the surface hydroxyl groups on \ce{TiO2} or with water, and its high oxidizing power cause photocatalytic decomposition of the dye.

An elementary step in decomposition of MB is N-dealkylation, which is preceded by radical cation formation.\cite{Takizawa1978} This radical cation can be spectroscopically monitored by the presence of 520nm band for MB. In MB absorption spectrum, 664 and 614 nm band ratio is related to monomer and dimer relaxation.
\begin{align}
\cee{TiO2 &\rightarrow e_{CB}^- + h_{VB}^+ \\
h_{VB}^+ + red &\rightarrow ox\\
MB^+ + 2e_{CB}^ + H^+ &\rightarrow LMB}
\end{align}

Measure the ratio between 614 and 663 nm before and after adding WS2 can indicate the adsorption of monomer and dimer MB.

MB can act as sacrificial electron acceptor in the reduction to leuco form. The decomposition is favored under oxygen-rich environment. MB feature peaks at 663, 614 and 292 nm, and $\epsilon_{660}=10^5 M^{-1}cm^{-1}$. The doubly reduced form of MB, LMB has feature peak at 256 nm. The singly reduced form of MB, \ce{MB.^-} is pale yellow, with peak at 420nm.\cite{Mills1999}
\[
\cee{MB + e_{CB}^- ->[pH<7] MB.^-}
\cee{2MB.^- \rightarrow MB + LMB}
\cee{O2 + e_{CB}^- \rightarrow O2.^-}
\]

The oxidized form of MB, \ce{MB.^+} has peak at 520nm, which is stable in acidic solution, but decomposes irreversibly in slight alkaline solution(pH = 9).
thionine peaks at 600nm.
MB forms dimers in aqueous solution,
\ce{
2MB <=>[K_D] (MB)_2
}
A typical value of $K_D$ is 3970 1/M. A quadratic equation can be solved to obtain the monomer concentration:
\[
2K_D [MB]^2 + [MB] - [MB]_{total} = 0
\]
MB adsorption on metal oxides. Monomer size is less than 1.5nm.
Logarithmic acid dissociation constant $pK_a= -\log_10 \frac{[A^-][H^+]}{[HA]}$. The oxidation potential for \ce{H2O}-\ce{O2} couple is 1.23 V and 0.817 V versus NHE at pH 0 and pH 7, respectively.

%\begin{align}
%\cee{MB + SED &->[TiO2][h\nu \leq 3.2eV] LMB + SED^{2+}\\
%2LMB + O2 &\rightarrow 2MB + 2H2O}
%\end{align}


S.L. Murov, I. Carmichael, G.L. Hug, Handbook of Photochemistry, 2nd revised ed. Marcel Dekker, New York, 1993.

aerobic or anaerobic, dimerise, photominerlization, gas to liquid transfer.

Mb to LMB transition as visual time monitor. commercial colorimetric oxygen indicators. radical-bearing carbon with unpaired electrons. MB = \ce{MB^+Cl^-}.\cite{Galagan2008}

monomer MB and dimer MB kinetics.\cite{Spencer1979}



MB. \cite{Lee2003a}
\begin{align}
\cee{ 2LMB &->[\text{UV}] LMB^*\\
2LMB^* + O2 &\rightarrow 2MB^+ + 2OH^-}
\end{align}

\[
\cee{2LMB ->[\alpha] LMB^*}
\cee{2LMB ->[\text{above}] LMB^*}
\]


\section{literature to read}

\ce{WS2} 1L doping calculation. \cite{Ma2011}

electrochromic films. \cite{Yoshimura1985}
ECD \cite{Jiao2012} recent review \cite{Mortimer2011}
PEC, photoelectrode, WO3 and Si tandem structures.\cite{Coridan2013}
WO3 photoactivity MB. \cite{Watcharenwong2008}

2D wo3.\cite{Kalantar-zadeh2010a} 
WO3 plasmon \cite{Manthiram2012}

\ce{WO3} indirect gap 2.6eV, direct gap 3.4eV. \cite{Koffyberg1979}

\ce{WO3} on FTO by flame synthesis.\cite{Rao2014} \cite{Xu2006}

Seeded \ce{W_{18}O_{49}} NWs growth on W foil.\cite{Hong2006a}

\ce{Na2W4O_{13}} growth and optical properties. \cite{Oishi2001} \cite{Itoh2001}

\ce{Na2W4O_{13}} crystal phase \cite{Viswanathan1974}

\citeauthor{Salje1984} studied the transport in \ce{WO_{3-x}} ($0\leq x \leq 0.28$).\cite{Salje1984} It was found \ce{WO_{3-x}} show metallic conductivity when $x > 0.1$.

\ce{WO_{3-x}} \cite{Migas2010}

\ce{WO3} high temperature phase. \cite{Vogt1999}


\ce{WO_{3-x}} CS planes and conductivity.\cite{Sahle1983}

\ce{W-O} equilibrium diagram \cite{Wriedt1989}

\ce{W_{18}O_{49}} electrochromic devices.\cite{Liu2013d} should compare with this one \cite{Wang2008}

nucleation catalysis \cite{Turnbull1952}

\ce{WO3} NWs aggregates. \cite{Kozan2008a}

\ce{WO3} atomic layer by exfoliation and annealing \ce{WO3.H2O}. \cite{Kalantar-zadeh2010a}



charge density wave in K-doped \ce{WO3} \cite{Raj2008}

\ce{WnO_{3n-1}} NPs. \cite{Frey2001}

\ce{WO3} growth hydrothermal.\cite{Moshofsky2012}

\ce{W_{18}O_{49}} on tungsten foil by thermal growth\cite{VanHieu2012}

Cathodoluminescence \cite{Parish2007}

optical characterization of WOx film.\cite{Valyukh2010a}
homogeneous optical properties of \ce{WO3} film,\cite{Valyukh2010a} 

Tungsten bronze: \ce{AxWO3}, or \ce{AxW_{1-x}^{6+}W_x^{5+}O3}, 
Tungstates: \ce{(A2O)x.(WO3)n}, full oxidized, no free electron, transparent in visible and NIR. 

============================
E-beam penetration \cite{Kanaya2002}

optics in electron microscopy. \cite{GarciadeAbajo2010a}

Ge NW growth using Ga as catalyst. \cite{Chandrasekaran2006}

\ce{MoO3} photocatalytic \cite{Chithambararaj2013}
photocatalytic experimental setup.\cite{Hupka2006}
\ce{MoO3} pseudocapacitor  \cite{Brezesinski2010}

\ce{MoOx} few layer as hole selective contact in solar cell.\cite{Battaglia2014}
\ce{MoO_x} on n-type Si acts as a high work function metal (6.6eV), enabling a dopant-free contact and thus junction-less devices.
piranha clean of FTO. 50ms switch.\cite{Scherer2012} 

hydrogen absorption in \ce{MoO3}.\cite{Sha2009}

\ce{Na6Mo_{11}O_{36}} phase. \cite{Bramnik2004}

\ce{Na6Mo_{10}O_{33}} phase, \cite{Gatehouse1983}

\ce{MoO3} thin film. \cite{Carcia1987}

\ce{H_xMoO3} raman.\cite{Hirata1996}

MoO3 spreading \cite{Leyrer1990}

Na2Mo2O7, Na2Mo4O13 phase transition \cite{SinghMudher2005}\cite{Tangri1992}

visibility of FL \cite{Benameur2011}

exfoliation IPA \cite{Halim2013}  \cite{Zhou2011a}

\ce{MoO3} good style. \cite{Siciliano2009} \cite{Abdellaoui1997}

\ce{MoO3}  DFT study \cite{B511044K} \cite{Cora1997} \cite{Sayede2005}

\ce{MoO3} raman \cite{Lee2002}



hydrogen evolution catalysts. \cite{Merki2011}

Raman substrate dependence \cite{Buscema2013}

stability of TMS NTs \cite{Seifert2002}

2H and 1T in \ce{MoS2} \cite{Eda2012}

\ce{MoS2} FET statistical study. \cite{Liu2013i}

water splitting review. \cite{B800489G}

\ce{WS2} theory and experimental combined study. \cite{Klein2001}

2D review on oxides \cite{Osada2012}

Pb catalyzed \ce{MoS2} nanotube \cite{Brontvein2012}

\ce{WS2} Raman.\cite{Zhao2013} \cite{Sekine1980}

phonon dispersion $E_{2g}^1(M)$? \cite{Ataca2012}

MoS2 optical properties.\cite{Search1979}

FL heterostructure. \cite{Yu2013a}

\cite{Kang2013} TMDC alloy DFT.

thermoelectric TMDC \cite{Wickramaratne2014}

\ce{CH4N2S} thiourea + \ce{WOx} to \ce{WS2} \cite{Leonard-Deepak2011}

\ce{WS2} by \ce{WCl_n} and \ce{H2S}, raman (632 nm) show bulk features\cite{Tenne2008}.

direct gap of ML at corner of BZ, point K.

TMO review.\cite{Goodenough2013}

h-\ce{MoO3} \cite{Lunk2010} \cite{Zheng2009}

\ce{MoO3} (010) surface defect. \cite{Chen2001}

mass spectrometry data to extract vapor pressure of \ce{NaxMoO3}.
strain and Raman theoretical analysis.\cite{Chang2013a} 

magnetic properties of ws2.\cite{Zhang2013j} 


\section{literature read}

Under O poor condition, formation energy of Vo is -0.02 eV, almost spontaneous. \ce{WO3} hole diffusion length 150 nm; etching hydrothermal grown wo3 flakes in weak acidic condition; Mott-Schottly plot, positive slope as expected for n type, carrier density as 5.92E20 \si{cm^{-3}} \cite{Li2014g}, 

a comprehensive process of \ce{WO3} NP for ECD. \cite{Wojcik2014}

2D \ce{WO3} flakes, electron Hall mobility 12 \si{cm^2V^{-1}s^{-1}}, high charge transport due to high relative permittivity suppressing the Coulomb charge effects. \cite{Zhuiykov2014} After \ce{H+} intercalation the band gap reduced to 2.5 eV. carrier mobility 275 \si{cm^2V^{-1}s^{-1}} for exfoliated Q2D \ce{WO3-x}, commercial Si-basded device about 500 \si{cm^2V^{-1}s^{-1}}. Ohmic contact between Q2D wo3-x and Au electrode, 

\cite{Hutter2014}

\ce{W_{18}O_{49}} Raman, IR shielding.\cite{Guo2012} \cite{Guo2011}
broad peak between 750-780 cm-1.
\ce{Na2W2O7} Raman. \cite{Knee1979}
Raman fingerprint of m-\ce{WO3}, h-\ce{WO3} and \ce{WO3.nH2O} were summarized in ref\cite{Daniel1987}.
Phase transformation of \ce{Na2MoO4} and \ce{Na2WO4} by Raman scattering. \cite{Lima2011}
sodium tungstates solution compounds Raman \cite{Redkin2010}
\ce{WO2} NWs synthesis and raman \cite{Ma2005}
tungsten bronzes \cite{Wiseman1976}



To develop large-size single-crystal graphene on dielectric substrates. small carbon flow near-equilibrium CVD process. Grain size about 10 microns, precursor \ce{CH4} and \ce{H2} (ratio 2.3:50) at 1180 C. \ce{SiO2}-Si surface roughness. Although the growth substrates (quartz,\ce{SiO2}-Si and \ce{Si3N4}-\ce{SiO2}-Si ) have a complicated stereo network similar to diamond, regular hexagonal G growth is obtained, which indicates the deposition is determined by equilibrium kinetics, and this should be applicable to other 2D materials as well. I2D/IG exceeds two on \ce{SiO2}-Si subs (514.5nm), indicating monolayer G. armchair (AC) G edge grows faster than zigzag (ZZ) edge.\cite{Chen2013j}

catalytic graphitization of solid carbon sources. catalytic transformation, the source is in solid state, low temperature (less than 600C), 2nm  \ce{Al2O3} by ALD as carbon diffusion barrier. amorphous silicon (a-Si), Ni lower the activation barrier ,  tetrahedral amorphous carbon (ta-C).\cite{Weatherup2013}

low energy (50eV) ion implantation doping in G. Ions penetrate pristine G at energy larger than 100eV. Individual substitution incorporation of B into G lattice is demonstrated. 1\% doping level was obtained. \cite{Bangert2013}

\ce{CaF2} a material suitable for scattering efficiency S comparison measurement due to its large band gap ($S\times \omega_L^4$ is constant below 5eV).

heterojunction is employed to transferred photo-generated carriers. Schottky barrier conduction band electron trapping and consequent longer electron-hole pair lifetimes. Numerous studies have suggested that fine particles of transition metals or their oxides, when dispersed on the surface of a photocatalyst matrix, can act as electron traps on n-type semiconductors.\cite{Zhou2010} 

\citeauthor{Cao2014} studied the layer-dependence \ce{MoS2} electrocatalysis and propose the vertical hopping efficiency of electrons instead of the edge site numbers is a key factor for catalytic reaction.\cite{Cao2014} ref19,20

It was found that the critical step in this process is the fast conversion of the oxide nanoparticle surface into a closed monolayer of \ce{WS2}. \ce{W18O49} as an intermediate phase is observed. XRD peaks shift to monitor strain.(002) peak of nanotube shifted to lower angles, the interlayer spacing increase by about 2\% as compared to the bulk powder, likely due to the build-in strain.\cite{ZAK2009} 
$\epsilon = (a - a_0)/a_0$ =(6.4-6.16)/6.16 = 3.8\%. tensile strain ($\epsilon > 0$)


DFT doped \ce{WO3} for photocatalytic reaction.\cite{Wang2012} CBM arises from W $5d$ states and splits into $t_{2g}$ and $e_g$ states under crystal field. VBM comes from O $2p$ states, including $2p_\sigma$ (along \ce{W-O} bonds) and $2p_\pi$ (normal to \ce{W-O} bonds).



\citeauthor{Huang2006} studied the \ce{W3On} cluster with n from 7 to 10.\cite{Huang2006} It was found \ce{W3O9} clusters possess a HOMO-LUMO gap about 3.4eV. This closeness to bulk value suggests \ce{W3O9} could be viewed as the smallest molecular unit for bulk \ce{WO3}.

plasmon dispersion in 2D materials, plasmon resonances in visible regions by doping induced free carrier density. 2D plasmonics, depolarization factors, partial reduction of Mo to a lower valence state. \cite{Alsaif2014a}

WS2 photoluminescence spectra of few layer and nanotube:
NT electrical structures depend on chirality, diameter and layer No as well as strain. Theoretical calculation indicates the SWNT with diameter larger than 4nm should approach the single layer limit.\cite{Ghorbani-Asl2013}

Other chalcogenide has also been synthesized using this one-end sealed layout.\cite{Mukherjee2013}

\citeauthor{Zou2007} prepared W/\ce{WS2} core-shell NPs by reaction of tungsten and sulfur under hydrogen atmosphere.\cite{Zou2007}

CVD 1L WS2 PL.\cite{Peimyoo2013} (of NTU Yu group) PL peak at 635nm, width 40 meV, 

CVD 1L WS2.\cite{Cong2013} (of NTU Yu group) 457 nm excitation, PL at 525nm and 630nm, 

\ce{MoS2} sing-walled nanotube.\cite{Xiao2014}

1T MoS2: metallic phase a negative temperature coefficient for conductivity, XRD pattern identified. \cite{Wypych1992}

stable 1T WS2 multiwalled NT by Re doping.\cite{Enyashin2011}. 2H to 1T transition formerly known only for WS2 and MoS2 intercalated by alkali metals. 3R transition to 2H upon heating since 2H is the most stable one.

1T \ce{MoS2} Raman.\cite{Yang1991} strong peaks at 156, 226, and 330 cm-1. M point frequencies measured by neutron scattering. M point is folded into BZ zone center due to the formation of superlattice.

Electrons and Phonons in Layered Crystal Structures, edited by T. J. Wieting (Reidel, Dordrecht, Holland, 1979).

\ce{WS2} p-type or n-type.  Fermi level at the surface of semiconductor is pinned to a fixed position relative to the CBM and VBM by a sufficient density of surface states situated between CBM and VBM. \cite{Baglio1983}

Electronic structure of \ce{MoS2}.\cite{Eknapakul2014} K intercalating into bulk to create quasi-standing 1L. Large effective mass 0.6 $m_e$ found, implying low mobility. Direct gap 1.88eV is measured.

Self-assembled monolayer (SAM) on \ce{SiO2} and its effect on \ce{MoS2} 1L.\cite{Najmaei2014}

\citeauthor{Shi2013} studied the strained monolayer \ce{MoS2} and \ce{WS2}. The results show that exciton binding energy is insensitive to the strain, while optical band gap becomes smaller as strain increases. Monolayer \ce{WS2} PL maximum located at about 1.95 eV. Calculation shows the electron effective mass of \ce{WS2} is the smallest, rendering higher mobility in device.\cite{Shi2013}

\citeauthor{Kosmider2013} studied the heterojunction between two monolayers of \ce{MoS2} and \ce{WS2}. Top of VB in W layer and bottom of CB in Mo layer, forming type II structure. bilayer gap 1.2 eV.\cite{Kosmider2013}

Band structure  of \ce{MoS2} in bulk form was calculated by \citeauthor{Mattheiss1973}.The calculation result is 1.2eV (indirect gap).\cite{Mattheiss1973}

Alkali metal intercalated \ce{WS2} film was prepared.\cite{Homyonfer1997} Stage 6 superlattice formation was suggested according to X-ray diffraction, and photoresponse spectra and electron tunneling measurement were done.

decrease in dielectric screening and thereby enhanced excitonic effect.
DFT is not good at describing photoemission, GW approximation overcome this deficiency but still not enough for photoabsorption process in which ehps are created. BSE equation is used to compensate this discrepancy, WX2 exhibits larger spin-orbit splitting as compared to MX2 family.\cite{Ramasubramaniam2012}

oxygen plasma treatment on HF-etched Si (001). reaction among $e$, \ce{O^+}, \ce{O2^+}, \ce{O^-},\ce{O2}. \ce{OH}-terminated surface obtained.\cite{Habib2010}