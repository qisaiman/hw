\documentclass[12pt,oneside]{book}
 %
   \usepackage{uncc-thesis} % some format specifications define page number in up right corner
   %-------page layout--------%
% adapted from <http://www.khirevich.com/latex/page_layout/>
%\usepackage[DIV=14,BCOR=2mm,headinclude=true,footinclude=false]{typearea}

%\makeatletter
%\if@twoside % commands below work only for twoside option of \documentclass
%    \newlength{\textblockoffset}
%    \setlength{\textblockoffset}{12mm}
%    \addtolength{\hoffset}{\textblockoffset}
%    \addtolength{\evensidemargin}{-2.0\textblockoffset}
%\fi
%\makeatother

% packages used in uncc-thesis

%\RequirePackage{ifthen}
%\RequirePackage{setspace} % for double spacing
%\RequirePackage{comment}
%\RequirePackage{epsfig}
%\usepackage{sectsty} % for sectional header style. Alternative: titlesec package
% \usepackage{tocloft}
%\usepackage{geometry}

\usepackage{microtype} % better layout
%----inherent of article class-------%
%\usepackage[utf8]{inputenc} % set input encoding (not needed with XeLaTeX)

%--- for font ----
% \usepackage[T1]{fontenc}
% \usepackage{textcomp}

%\usepackage{mathptmx} % fine but not truetype
% \usepackage{newtxtext}
% \usepackage{pslatex} % not bad

\usepackage{fontspec} % to compile with LuaLatex
\setmainfont{Times New Roman} % to compile with LuaLatex

 %-- end of font adaption

\usepackage{graphicx} % support the \includegraphics command and options
% check pdftex option
\usepackage{placeins} %

\usepackage{subfig}
\usepackage{float} % for images
 \graphicspath{{./gallery/}} %--added by author

\usepackage[font=small, labelfont=bf]{caption}
%\usepackage{subcaption}


% It supplies a landscape environment, and anything inside is basically rotated.(http://en.wikibooks.org/wiki/LaTeX/Page_Layout)
%\usepackage{lscape}
\usepackage{pdflscape}
%\usepackage{rotating} % use \begin{sidewaystable}

% Helps format tables using the \toprule, \midrule, and \bottomrule commands (http://en.wikibooks.org/wiki/LaTeX/Tables#Using_booktabs)
\usepackage{booktabs}
%\usepackage{multirow} for multirow in tables
%  Helps format tables (http://en.wikibooks.org/wiki/LaTeX/Tables#Using_array)
%\usepackage{array}

%%-- Additional style---- modified by Tao Sheng 12/20/12
% for chemical formula, subscripts etc
\usepackage[version=3]{mhchem}
\usepackage{siunitx}
  \DeclareSIUnit \torr{Torr}
%%-- mathmatical symbols and equations-----
\usepackage{amsmath}
\usepackage{amssymb}
 %\numberwithin{equation}{section}
 %\numberwithin{figure}{section}
 \providecommand*{\ud}{\mathrm{d}}

%------- bibliography and citation ------
\usepackage[english]{babel}% Recommended
\usepackage{csquotes}% Recommended
\usepackage[style=numeric-comp,
		    sorting=nty,
            hyperref=true,
            url=false,
            isbn=false,
            backref=true,
            maxcitenames=2,
            maxbibnames=4,
            block=none,
            backend=bibtex,
            natbib=true]{biblatex}
% \usepackage[bibencoding=latin1]{biblatex}

\DefineBibliographyStrings{english}{%
    backrefpage  = {see p.}, % for single page number
    backrefpages = {see pp.} % for multiple page numbers
}
% suppress 'in:'
\renewbibmacro{in:}{%
  \ifentrytype{article}{}{\printtext{\bibstring{in}\intitlepunct}}}
% document preamble
% removes period at the very end of bibliographic record
\renewcommand{\finentrypunct}{}
% removes pagination (p./pp.) before page numbers
\DeclareFieldFormat{pages}{#1}


\providecommand*{\bibpath}{E:/spring2012/Ubuntu/Latex/Mendeley_Bib_lib}
\bibliography{\bibpath/arix.bib,\bibpath/ECD.bib,\bibpath/tungsten_newandgood.bib,\bibpath/ACSnano.bib,%
\bibpath/tungsten_old.bib,\bibpath/Raman.bib,\bibpath/Molybdenum.bib,%
\bibpath/tungsten_cl.bib,\bibpath/optics,\bibpath/CVD.bib,\bibpath/sodium.bib,\bibpath/VLS.bib}

%-- for works around, packages conflicts----

%-- redefine toc macros ------
\addto\captionsenglish{%
\renewcommand\chaptername{CHAPTER}%
\renewcommand\appendixname{APPENDIX}%
\renewcommand\indexname{INDEX}%
\renewcommand{\contentsname}{TABLE OF CONTENTS}%
\renewcommand{\listfigurename}{LIST OF FIGURES}%
\renewcommand{\listtablename}{LIST OF TABLES}%
}

%-- misc---
\usepackage{lipsum}
\usepackage{latexsym}
 \providecommand*{\thefootnote}{\fnsumbol{footnote}}

\usepackage{xcolor}
\usepackage{listings}
\lstset{
 frame = single,
 language = matlab,
 breaklines = true,
postbreak=\raisebox{0ex}[0ex][0ex]{\ensuremath{\color{red}\hookrightarrow\space}}
}

\usepackage{enumitem}
\setlist{nolistsep}

\setcounter{secnumdepth}{3} % show numbering of subsubsection

%% -- links ---
\usepackage{hyperref}
\hypersetup{
colorlinks,%
citecolor=black,%
filecolor=black,%
linkcolor=black,%
urlcolor=black
} % make all links black

\usepackage[acronym,toc,nonumberlist]{glossaries} % loaded after hyperref

 % additonal packages and tweaking

   \newacronym{cvd}{CVD}{chemical vapor deposition}
   \newacronym{cft}{CFT}{crystal-field theory}

   \makeglossaries

\begin{document}

  \title{\LaTeX is great}
  \author{me}
  \maketitle

    \clearpage
    \tableofcontents
    \clearpage
    \listoftables
    \clearpage
    \listoffigures

    \begin{singlespace}
    %\thispagestyle{myheadings} % before, this does not work
    \renewcommand{\glossarypreamble}{\thispagestyle{myheadings}} % after, this works.
    \printglossary[type=\acronymtype]
    \clearpage
    \end{singlespace}

 %   %\providecommand{\setflag}{\newif \ifwhole \wholefalse}
\setflag
\ifwhole\else

    \documentclass[12pt,letterpaper,oneside]{book}

    %-------page layout--------%
% adapted from <http://www.khirevich.com/latex/page_layout/>
%\usepackage[DIV=14,BCOR=2mm,headinclude=true,footinclude=false]{typearea}

%\makeatletter
%\if@twoside % commands below work only for twoside option of \documentclass
%    \newlength{\textblockoffset}
%    \setlength{\textblockoffset}{12mm}
%    \addtolength{\hoffset}{\textblockoffset}
%    \addtolength{\evensidemargin}{-2.0\textblockoffset}
%\fi
%\makeatother

% packages used in uncc-thesis

%\RequirePackage{ifthen}
%\RequirePackage{setspace} % for double spacing
%\RequirePackage{comment}
%\RequirePackage{epsfig}
%\usepackage{sectsty} % for sectional header style. Alternative: titlesec package
% \usepackage{tocloft}
%\usepackage{geometry}

\usepackage{microtype} % better layout
%----inherent of article class-------%
%\usepackage[utf8]{inputenc} % set input encoding (not needed with XeLaTeX)

%--- for font ----
% \usepackage[T1]{fontenc}
% \usepackage{textcomp}

%\usepackage{mathptmx} % fine but not truetype
% \usepackage{newtxtext}
% \usepackage{pslatex} % not bad

\usepackage{fontspec} % to compile with LuaLatex
\setmainfont{Times New Roman} % to compile with LuaLatex

 %-- end of font adaption

\usepackage{graphicx} % support the \includegraphics command and options
% check pdftex option
\usepackage{placeins} %

\usepackage{subfig}
\usepackage{float} % for images
 \graphicspath{{./gallery/}} %--added by author

\usepackage[font=small, labelfont=bf]{caption}
%\usepackage{subcaption}


% It supplies a landscape environment, and anything inside is basically rotated.(http://en.wikibooks.org/wiki/LaTeX/Page_Layout)
%\usepackage{lscape}
\usepackage{pdflscape}
%\usepackage{rotating} % use \begin{sidewaystable}

% Helps format tables using the \toprule, \midrule, and \bottomrule commands (http://en.wikibooks.org/wiki/LaTeX/Tables#Using_booktabs)
\usepackage{booktabs}
%\usepackage{multirow} for multirow in tables
%  Helps format tables (http://en.wikibooks.org/wiki/LaTeX/Tables#Using_array)
%\usepackage{array}

%%-- Additional style---- modified by Tao Sheng 12/20/12
% for chemical formula, subscripts etc
\usepackage[version=3]{mhchem}
\usepackage{siunitx}
  \DeclareSIUnit \torr{Torr}
%%-- mathmatical symbols and equations-----
\usepackage{amsmath}
\usepackage{amssymb}
 %\numberwithin{equation}{section}
 %\numberwithin{figure}{section}
 \providecommand*{\ud}{\mathrm{d}}

%------- bibliography and citation ------
\usepackage[english]{babel}% Recommended
\usepackage{csquotes}% Recommended
\usepackage[style=numeric-comp,
		    sorting=nty,
            hyperref=true,
            url=false,
            isbn=false,
            backref=true,
            maxcitenames=2,
            maxbibnames=4,
            block=none,
            backend=bibtex,
            natbib=true]{biblatex}
% \usepackage[bibencoding=latin1]{biblatex}

\DefineBibliographyStrings{english}{%
    backrefpage  = {see p.}, % for single page number
    backrefpages = {see pp.} % for multiple page numbers
}
% suppress 'in:'
\renewbibmacro{in:}{%
  \ifentrytype{article}{}{\printtext{\bibstring{in}\intitlepunct}}}
% document preamble
% removes period at the very end of bibliographic record
\renewcommand{\finentrypunct}{}
% removes pagination (p./pp.) before page numbers
\DeclareFieldFormat{pages}{#1}


\providecommand*{\bibpath}{E:/spring2012/Ubuntu/Latex/Mendeley_Bib_lib}
\bibliography{\bibpath/arix.bib,\bibpath/ECD.bib,\bibpath/tungsten_newandgood.bib,\bibpath/ACSnano.bib,%
\bibpath/tungsten_old.bib,\bibpath/Raman.bib,\bibpath/Molybdenum.bib,%
\bibpath/tungsten_cl.bib,\bibpath/optics,\bibpath/CVD.bib,\bibpath/sodium.bib,\bibpath/VLS.bib}

%-- for works around, packages conflicts----

%-- redefine toc macros ------
\addto\captionsenglish{%
\renewcommand\chaptername{CHAPTER}%
\renewcommand\appendixname{APPENDIX}%
\renewcommand\indexname{INDEX}%
\renewcommand{\contentsname}{TABLE OF CONTENTS}%
\renewcommand{\listfigurename}{LIST OF FIGURES}%
\renewcommand{\listtablename}{LIST OF TABLES}%
}

%-- misc---
\usepackage{lipsum}
\usepackage{latexsym}
 \providecommand*{\thefootnote}{\fnsumbol{footnote}}

\usepackage{xcolor}
\usepackage{listings}
\lstset{
 frame = single,
 language = matlab,
 breaklines = true,
postbreak=\raisebox{0ex}[0ex][0ex]{\ensuremath{\color{red}\hookrightarrow\space}}
}

\usepackage{enumitem}
\setlist{nolistsep}

\setcounter{secnumdepth}{3} % show numbering of subsubsection

%% -- links ---
\usepackage{hyperref}
\hypersetup{
colorlinks,%
citecolor=black,%
filecolor=black,%
linkcolor=black,%
urlcolor=black
} % make all links black

\usepackage[acronym,toc,nonumberlist]{glossaries} % loaded after hyperref


    %\input{tweak.tex}
    %\input{commando.tex}
    %\input{font}

    \begin{document}

\fi  %  comment out when assembling

\chapter{introduction: nanomaterials for energy applications}

\section{Background and Motivation}

TMO as electrochromic device and TMDC as newly 2D semiconductor, and some VLS.

Materials that human can make define the age they live in. From Stone Age to Bronze Age and Iron Age, people evolve as mastering more and more sophisticated techniques of manipulating metals, such as alloying and annealing. Obtaining extreme high purity of silicon brings us into Information Age. Future is difficult to predict. But nanotechnology is one direction that we can not ignore. According to \gls{nni}, \gls{nanotechnology}. This definition alludes that dimension comes before compositions. It is often related to the quantum confinement or surface area in nanomaterials, which we will later revisit with specific scenario.

There are three states of matter under usual conditions: solid,liquid and gas. Solids materials could be further categorized into five groups: metals, ceramics, polymers, semiconductors, and composites.\cite{William2009} This classification is based on both composition and mechanical, electrical, and thermal properties as well as the associated functionality(i.e., \gls{ceramics} are typically hard yet brittle, insulating to electricity and resistant to heating).

\section{Dissertation Outline}

The materials studied in this work/dissertation are tungsten oxides (\ce{WO3}), molybdenum oxides (\ce{MoO3}),and their chalcogenide counterparts (\ce{WS2} and \ce{MoS2}). Both tungsten (W) and molybdenum (Mo) belong to Group VIB transition metal, with outer shell electrons configuration as $4d^55s^1$ and $5d^46s^2$ respectively. Therefore we refer their oxides and chalcogenides as \gls{tmo} and \gls{tmdc}.\footnote{Obviously transition metals include many other elements, all of which have partially filled $d$-electron shell. But here we use TM to denote W and Mo exclusively.} We have synthesized \gls{tmo} and \gls{tmdc} at nanoscale, measured their crystalline structures and optical properties and demonstrated some devices assembled using as-synthesized nanomaterials. We aim to illustrate that by nanoengineering these \gls{tmo} and \gls{tmdc}, enhanced performances over their bulk states could be expected and new properties will arise. In the remaining sections of this chapter, we will discuss some general perspectives of nanomaterials, the growth apparatus and characterization methods that apply to all experiments done in this work. Then chapter 2 will focus on growth of \ce{WO3} and its derivative. We employed thermal \gls{cvd} to synthesize \ce{WO3} \gls{nw}, and we investigated the role of impurity in tungsten metallic powders, during which we observed a new state of sodium tungsten oxides: \ce{Na5W14O44} nanowires. We also found a method to potentially obtain large yield of \ce{WO3} \gls{nw}. Chapter 3 will concentrate on \ce{MoO3}. We explored two different growth mechanism of \ce{MoO3}:\gls{vs} and \gls{vls}. We discovered that alkaline oxides can be used as catalyst to grow two distinct \ce{MoO3} morphologies: nanobelts and towers. We further demonstrated the application of as-synthesized \ce{MoO3} nanomaterials in electrochromic devices.  In chapter 4 we discuss the growth of \gls{tmdc} and associated heterostructures. We synthesized \ce{WO3}-\ce{WS2} core-shell \gls{nw} and inspected the growth of \gls{fl} \ce{WS2}. Chapter 5 will conclude with an overall summary.



\section{Nanomaterials for Energy Applications}

why nano? surface-to-volume ratio, more surface area for catalytic reaction; surface energy state: tuned by dimension; quantum confinement effects: exciton size vs physical dimension. easy for dopant diffusion, thereby band structure modification; charge-separation and transport mechanism may also differ from bulk.


\section{Crystal Structures and Electronic Properties}

Solid and orderliness.




Two theories arise to describe the outer shell electrons and to correlate the structure and physical properties: \gls{cft} and band theory.\cite{Goodenough1971} \gls{cft} assumes weak interaction between neighboring atoms and localization of electron towards parent atom, whereas band theory assumes that electron is shared equally by all nuclei and therefore a many-electron problems follows. Description of a single electron in periodic potential fail to treat the electron correlations adequately, as the interaction between atoms becomes weaker.For transition metals, $s$ and $p$ electrons are well described by a collective-electron model, while the 4f or 5f electrons are tightly bound to nuclei and screened from the neighboring atoms by 5s, 5p or 6s, 6p core electrons, hence it matches well with a localized-electron model. d electrons show intermediate character.

\section{Growth Apparatus and Characterization Methods}
\subsection{CVD System}

The synthesis was accomplished in a home-made hot-wall CVD system, as visualized in Fig.~\ref{fig:ch1cvd}. The furnace is made by two semi-cylindrical ceramic fiber heaters (WATLOW inc.) with power density from 0.8 to 4.6 \si{W/cm^2}. Quartz tube (Quartz Sci inc.) of 1 inch diameter was primarily used as reaction chamber. A mechanical pump was connected to maintain the low pressure environment inside the chamber. The length of uniform heating zone is 6 inches, with cooling zone extending outward. Carriers gas argon (Ar) and reactant gas oxygen (\ce{O2}) was regulated by two mass flow controllers respectively.

\begin{figure}[htb]
\centering
\subfloat[]{\includegraphics[width=0.8\textwidth]{CVD_d346.jpg}}

\subfloat[]{\includegraphics[width=0.8\textwidth]{CVD_model.jpg}}
\caption[CVD system]{Home-built low-pressure chemical vapor system.}
\label{fig:ch1cvd}
\end{figure}

Silicon and silicon dioxides on Si wafer ($p$-Si(100),Unversity Wafer inc.) were primarily used as receiving substrates, and other substrates (i.e. Mica\footnote{ \ce{K(Al2)(Si3Al)O10(OH)2}}, glass (Fisher Scientific, microscope slide, 12-549),\footnote{Typical composition is 72.6\% \ce{SiO2}, 0.8\% \ce{B2O3}, 1.7\% \ce{Al2O3}, 4.6\% \ce{CaO}, 3.6\% \ce{MgO} and 15.2\% \ce{Na2O}} stainless steel) were occasionally adapted. The preparation procedure was illustrated in Fig. Substrates were first cut into rectangular pieces of certain size and then ultrasonically bathed (Branson 1510R-MTH) with acetone and alcohol for about 15 minutes each followed by blow-drying with nitrogen gas. After solution cleaning, the Si surface is hydrophobic. Depending on the specific experimental requirement, sometimes a hydrophilic surface is desired. We use plasma cleaning (Kurt J Lesker: Plasma-Preen 862) under 2Torr \ce{O2} for 3 minutes to render a hydrophilic Si surface. In addition, substrates can be coated with a thin layer of metal before sent into reaction chamber. This process details will be covered later in the \emph{Characterization} session. The controllable parameters of our CVD system includes central heating temperature, absolute gas flow and relative ratio of (Ar/\ce{O2}), amount of source material and the location of receiving substrates. The operation capability was summarized in Table.~\ref{tab:cvd}.

\begin{table}[htb]
\centering
\caption{CVD parameters map}\label{tab:cvd}
    \begin{tabular}{lcccr}
    \toprule
     &&&\multicolumn{2}{c}{Flow} \\
    \cmidrule(l){4-5}
             & Temperature & Pressure & Ar & \ce{O2}  \\
    \midrule
             & \si{\degreeCelsius} & mTorr & sccm & sccm\\
    \midrule
    Range      & RT-1100    & 10mTorr-1atm & 0 - 100 & 0-30  \\
    Resolution & $\pm1$  & correlated to flow & 1   & 0.1  \\
    \bottomrule
    \end{tabular}
\end{table}

The substrates were mostly positioned in downstream cooling zone since within this region, the vapor undergoes a rapid temperature gradient and precipitation occurs. The source material placed in the central heating zone is oxidized and then evaporated. The growth species, transported by carrier gas, bombard both substrate and chamber wall. Some will be adsorbed by the substrate and become adatoms while some may remain as gas molecules, waiting for another event. On the hot substrate, adatoms diffuse and do not settle down until finding a appropriate location where equilibrium is favorable.

Nucleation is a process of generating a new phase from a metastable old phase, where the Gibbs energy per molecule of the bulk of the emerging new phase is less than that of the old phase.

  General CVD knowledge, substrate preparation, and\cite{MichealK.Zuraw2003}

\subsection{Measurement}
sample/specimen preparation, data processing,

\textbf{Magnetic Sputtering} Sputtering, a process in which atoms are ejected from a solid target material by bombarding it with energetic particles, is a well established PVD process with a high degree of controllability. The high energy and controllable parameters of sputtering can result in the growth of well-structured and crystalline films. Further, sputtering can be easily implemented as a roll-to-roll process for large-scale manufacturing. It is widely utilized for deposition of \ce{WO_x} in industry.


The morphology and composition of the as-synthesized samples were analyzed by scanning electron microscopy (SEM,JEOL JSM-6480) and energy dispersive X-ray spectroscopy (EDS,Oxford Instrument INCA). Crystal structures were characterized using X-ray diffraction (XRD, PANXpert X’pert Pro MRD with Cu $K\alpha$ radiation at $\lambda$=1.5418\AA) and transmission electron microscopy (TEM, JEOL JEM-2100 \ce{LaB6} operated at 200kV). Optical measurements were performed by Micro-Raman spectroscopy (Horiba Scientific, Labram HR800 with 532 nm excitation laser) in a confocal microscope backscattering configuration with spectral resolution about 1 cm$^{-1}$. Optical absorption spectra was recorded using UV-Vis-NIR spectrophotometer(Schimadzu, UV2600Plus) in transmission mode. When necessary, the as-synthesized sample was removed from substrates by light sonication(Branson 1510R-MTH, 70W) in ethanol for 15 seconds. The dispersion was left for 12 hrs to enable the possible sedimentation, after which became transparent under the unaided eye. Then the dispersion was transferred into one 10 mm quartz cuvette (Thorlabs, W005654) for absorption measurement with another paired cuvette containing ethanol only.


%\printbibliography  %  comment out when assembling

%\ifwhole\else
   \end{document}
\fi %  comment out when assembling 
  %   \chapter{paper reading}



\section{WO3}

Tungsten bronzes was coined by Wohler in 1837.\cite{Deb2008} \ce{Na_{x}WO3}



\subsection{applications}


\citeauthor{Wang2009a} mentioned that amorphous \ce{WO3} can only be used in lithium-based electrolytes due to its in-compact structure and high dissolution rate in acidic electrolyte solutions. Electrochromic materials that can endure acidic electrolytes without degradation should be developed. Crystalline \ce{WO3} nanostructures with their much denser structures and small particle sizes are promising to be used as suitable electrochromic material in acidic electrolytes.

photocatalytic applications in solar hydrogen generation and organic pollutant degradation.

photocatalyst\cite{Macphee2010},
photoelectrochemical energy application \cite{Su2010}

Raman \cite{Xiao2007}. Silver has the strongest SERS enhancement due to the larger imaginary part of the dielectric constant and higher thermal conductivity. Milli-Q grade water ((Milli-pore)\textgreater 18.2Mohm).

MB Raman peaks: 445, 1618, ref20. some peak splitting and shift observed on SERS, attributed to chemical adsorption. definition of Raman enhancement factor (9,26).

SERR MB on Ag. \cite{Nicolai2003}
MB: the absorption spectrum in VIS is used to infer about different adsorbed forms of MB. the formation of large aggregates. I call attention to the fact that.

WO3: effective mass of bipolaron = 1.9me. for electron, for hole:
unzip nanotube. passivate BN ribbons with O and S; another player terrones psu.

\ce{WO3} catalyst.\cite{Miyauchi2013}  potential of CB e more negative than redox potential of \ce{O2}-\ce{O2^-} (-0.046 V vs NHE at pH 0). Z-scheme two photo absorption. photogenerated ele in CB of WO3 can reduce itself by formation of color centers.

electrochromic films. \cite{Yoshimura1985}

ECD \cite{Jiao2012} recent review \cite{Mortimer2011}

PEC, photoelectrode, WO3 and Si tandem structures.\cite{Coridan2013}

WO3 photoactivity MB. \cite{Watcharenwong2008}
A low recombination rate is preferred for high photocatalytic efficiency. The simultaneous migration of electrons and holes.



\subsection{properties}

\citeauthor{Chatten2005} also studied the oxygen vacancy in different phases of \ce{WO3}.\cite{Chatten2005}

2D wo3.\cite{Kalantar-zadeh2010a} 
WO3 plasmon \cite{Manthiram2012}

DFT doped \ce{WO3} for photocatalytic reaction.\cite{Wang2012} CBM arises from W $5d$ states and splits into $t_{2g}$ and $e_g$ states under crystal field. VBM comes from O $2p$ states, including $2p_\sigma$ (along \ce{W-O} bonds) and $2p_\pi$ (normal to \ce{W-O} bonds).


oxygen vacancies in \ce{WO_{3-x}}.\cite{Wang2011b}  Coloration and electron conductivity changes. \citeauthor{Wang2011b} found strong dependence of WO3-x electronic properties on $V_O$ concentration and the the crystallographic direction on which O is removed. DFT band gap calculation is close to experimental value. Vacancy levels are found at 2.1eV.

The Raman spectra of \ce{WO_x} is rare because of the difficulty of preparing pure suboxides phase and the strong shielding of \ce{WS2}. Yet it does exhibit distinct Raman spectra. \cite{Tenne2005} The 870 line is attributed to \ce{W3O8}.\cite{Hardcastle1995}


\citeauthor{Huang2006} studied the \ce{W3On} cluster with n from 7 to 10.\cite{Huang2006} It was found \ce{W3O9} clusters possess a HOMO-LUMO gap about 3.4eV. This closeness to bulk value suggests \ce{W3O9} could be viewed as the smallest molecular unit for bulk \ce{WO3}.


\ce{WO3} indirect gap 2.6eV, direct gap 3.4eV. \cite{Koffyberg1979}

Raman fingerprint of m-\ce{WO3}, h-\ce{WO3} and \ce{WO3.nH2O} were summarized in ref\cite{Daniel1987}.

\ce{WO3} on FTO by flame synthesis.\cite{Rao2014} \cite{Xu2006}

Seeded \ce{W_{18}O_{49}} NWs growth on W foil.\cite{Hong2006a}

\ce{Na2W4O_{13}} growth and optical properties. \cite{Oishi2001} \cite{Itoh2001}

\ce{Na2W4O_{13}} crystal phase \cite{Viswanathan1974}

\citeauthor{Salje1984} studied the transport in \ce{WO_{3-x}} ($0\leq x \leq 0.28$).\cite{Salje1984} It was found \ce{WO_{3-x}} show metallic conductivity when $x > 0.1$.

\ce{WO_{3-x}} \cite{Migas2010}

\ce{WO3} high temperature phase. \cite{Vogt1999}
tungsten bronzes \cite{Wiseman1976}

Phase transformation of \ce{Na2MoO4} and \ce{Na2WO4} by Raman scattering. \cite{Lima2011}

\ce{WO2} NWs synthesis and raman \cite{Ma2005}.

\ce{WO_{3-x}} CS planes and conductivity.\cite{Sahle1983}

\ce{W-O} equilibrium diagram \cite{Wriedt1989}

\ce{W_{18}O_{49}} electrochromic devices.\cite{Liu2013d} should compare with this one \cite{Wang2008}

nucleation catalysis \cite{Turnbull1952}

\ce{WO3} NWs aggregates. \cite{Kozan2008a}

optical properties of \ce{WO3} gaps\cite{Saygin-Hinczewski2008}

\ce{WO3} atomic layer by exfoliation and annealing \ce{WO3.H2O}. \cite{Kalantar-zadeh2010a}

sodium tungstates raman \cite{Redkin2010}

charge density wave in K-doped \ce{WO3} \cite{Raj2008}

\ce{W_{18}O_{49}} Raman, IR shielding.\cite{Guo2012} \cite{Guo2011}
broad peak between 750-780 cm-1.

\ce{WnO_{3n-1}} NPs. \cite{Frey2001}

\ce{WO3} growth hydrothermal.\cite{Moshofsky2012}

\ce{W_{18}O_{49}} on tungsten foil by thermal growth\cite{VanHieu2012}

Cathodoluminescence \cite{Parish2007}

optical characterization of WOx film.\cite{Valyukh2010a}

E-beam penetration \cite{Kanaya2002}

optics in electron microscopy. \cite{GarciadeAbajo2010a}


\section{MoO3}

VLS:
Ge NW growth using Ga as catalyst. \cite{Chandrasekaran2006}


plasmon dispersion in 2D materials, plasmon resonances in visible regions by doping induced free carrier density. 2D plasmonics, depolarization factors, partial reduction of Mo to a lower valence state. \cite{Alsaif2014a}

\subsection{app}
applications: electrically controlled optical shutters for heat and light modulation, smart windows associated with solar cell to provide dynamical control of incoming illumination.

piranha clean of FTO. 50ms switch.\cite{Scherer2012} 
nanoscale Kirkendall effect: the outward diffusion of metal cations are balanced by an influx of vacancies. For example, diffusion coefficient of Ni in NiO is higher than that of oxygen.

\ce{MoO3} photocatalytic \cite{Chithambararaj2013}
photocatalytic experimental setup.\cite{Hupka2006}
\ce{MoO3} pseudocapacitor  \cite{Brezesinski2010}

\ce{MoOx} few layer as hole selective contact in solar cell.\cite{Battaglia2014}
\ce{MoO_x} on n-type Si acts as a high work function metal (6.6eV), enabling a dopant-free contact and thus junction-less devices.


\subsection{properties}



\begin{table}[htb]
\caption{Combinations of ECD configuration}\label{tb:ecd}
\begin{tabular}{lcccr}
\toprule
TC(both side) & electrochromic & ion conductor & counter electrode  & reference\\
\midrule
ITO &  \ce{WO3} & \ce{H^+\hyphen} polymer & PANI &\citeauthor{Heckner2002}\\
FTO &  \ce{WO3} & \ce{K^+\hyphen} polymer & PW &\cite{Heckner2002}\\
ITO & \ce{WO3} NWs & \ce{LiClO4\hyphen}PC & none & author design \\
\ce{Na_xWO3} NWs &\ce{WO3} NWs & \ce{LiClO4\hyphen}PC & none & author design\\
\bottomrule
\end{tabular}
\end{table}



\begin{table}[htb]
\centering
\caption{Comparison of MoOx ECD}\label{tab:moxecd}
\begin{tabular}{lcccr}
\toprule
$\lambda$ & $\Delta T$ & $t_c$ & $t_b$ & $CE$  \\
         (nm) & (\%)    & (s) & (s) & ($cm^2/C$)  \\
\midrule
Range      & RT-1100    & 10mTorr-1atm & 0 - 100 & 0-30  \\
\bottomrule
\end{tabular}
\end{table}


10nm MoOx as hole extraction layer (HEL). Without HEL, Holes accumulates at QD/anode interface, causing increased recombination rate. With HEL, hole diffuse into this layer, reducing the recombination. 

The molecular unit in crystal exhibits different vibrational frequencies from that in solution or gas phases.

\ce{Na2Mo4O_{13}} phases monoclinic at RT, solid solubility of \ce{Na2MoO4} in solid \ce{MoO3} is high. vapor pressure of \ce{Na2Mo4O_{13}} over \ce{MoO3}.

melting point of \ce{Na2Mo4O_{13}}
Mp: \ce{Na2Mo2O7} 960K

\ce{MoO3} vapor pressure:

The real phase diagram is the one between \ce{Na2Mo4O_{13}} and \ce{MoO3}.
the growth temperature could be much lower than the eutectic point.

KI MP:  681
NaI MP: 661

NaOH Raman peaks lie at 3633 cm. \cite{walrafen2006} Raman scattering of \ce{Na2SiO3} exhibit major peak at 966 and 589 cm.\cite{Richet1996}

hydrogen absorption in \ce{MoO3}.\cite{Sha2009}

\ce{Na6Mo_{11}O_{36}} phase. \cite{Bramnik2004}

\ce{Na6Mo_{10}O_{33}} phase, \cite{Gatehouse1983}

\ce{MoO3} thin film. \cite{Carcia1987}

\ce{H_xMoO3} raman.\cite{Hirata1996}

MoO3 spreading \cite{Leyrer1990}

Na2Mo2O7, Na2Mo4O13 phase transition \cite{SinghMudher2005}\cite{Tangri1992}

visibility of FL \cite{Benameur2011}

exfoliation IPA \cite{Halim2013}  \cite{Zhou2011a}

\ce{MoO3} good style. \cite{Siciliano2009} \cite{Abdellaoui1997}

\ce{MoO3}  DFT study \cite{B511044K} \cite{Cora1997} \cite{Sayede2005}

\ce{MoO3} raman \cite{Lee2002}

visibility of mica thin layer on \ce{SiO2}-Si. \cite{Castellanos-gomez2011} 1.5\% contrast is almost at the threshold of human eye sensitivity.  When the thickness is below 60nm, Raman could not detect mica.

\ce{MoO3} (010) surface defect. \cite{Chen2001}

mass spectrometry data to extract vapor pressure of \ce{NaxMoO3}.

\ce{MoO3} SWNT by hydrothermal method.\cite{Hu2008a} Raman spectra is off compared to single crystal \ce{MoO3}.  Van der Waals interaction and layered structure make NT possible.

TMO review.\cite{Goodenough2013}

h-\ce{MoO3} \cite{Lunk2010} \cite{Zheng2009}



\section{TMS}

petroleum oil catalytic refinement, solid lubricants in aerospace industry.

heterojunction is employed to transferred photo-generated carriers. Schottky barrier conduction band electron trapping and consequent longer electron-hole pair lifetimes. Numerous studies have suggested that fine particles of transition metals or their oxides, when dispersed on the surface of a photocatalyst matrix, can act as electron traps on n-type semiconductors.\cite{Zhou2010} 


\citeauthor{Cao2014} studied the layer-dependence \ce{MoS2} electrocatalysis and propose the vertical hopping efficiency of electrons instead of the edge site numbers is a key factor for catalytic reaction.\cite{Cao2014} ref19,20


\subsection{strain}

$E_{2g}$ mode is strain sensitive. 

\citeauthor{Ghorbani-Asl2013} studied the strain in tubular TMDC and found a linear dependence of Raman scattering on strain (3 \si{cm^{-1}} per percentage for $E_{2g}$mode).\cite{Ghorbani-Asl2013} 

For 2D materials, strain may be induced by elongation of an appropriate substrate, e.g. by uniform mechanical strain, or by using a material with high thermal expansion coefficient and varying the temperature. For TMD MWNT, tensile tests have been reported by various groups. However, to date, it is not perfectly clear whether inner and outer walls are stretched simultaneously, or rather the outer walls slide on the inner ones. The latter hypothesis would result in a broadening of the Raman signals, while the first one would leave the signal widths rather unaffected. In any case, there would be a shift of the Raman signals that can serve as precise scale for determining the strain.\cite{Ghorbani-Asl2013}


\citeauthor{Virsek2007} performed a Raman-TEM integrated study on multiwalled \ce{WS2} NT with diameter \textgreater 200 nm. The tubes were synthesized using chemical transport method. Up-shift of Raman is explained by strain in the walls. This shift is not observed in the specimen by sulfurization process of oxides. Applied hydrostatic pressure is isotropic,\cite{Staiger2012} while the strain is expected to anisotropic. Strain can also be relaxed by chirality.\cite{Virsek2007} 

strain effect by first-principles calculations. direct gap is only maintain in a narrow strain range (-1.3 -- 0.3 \%), \cite{Yun2012}.

Semiconducting to metallic transition in \ce{MoS2} at compressive strain of 15\% or tensile strain of 8\%; direct-to-indirect gap transition for 1L \ce{MoS2} at about 2\%. \cite{Scalise2012}

strain and Raman theoretical analysis.\cite{Chang2013a} 

magnetic properties of ws2.\cite{Zhang2013j} 

growth mechanism of WS2 NT:

It was found that the critical step in this process is the fast conversion of the oxide nanoparticle surface into a closed monolayer of WS2. W18O49 as an intermediate phase is observed. XRD peaks shift to monitor strain.(002) peak of nanotube shifted to lower angles, the interlayer spacing increase by about 2\% as compared to the bulk powder, likely due to the build-in strain.\cite{ZAK2009} 
$\epsilon = (a - a_0)/a_0$ =(6.4-6.16)/6.16 = 3.8\%. tensile strain ($\epsilon > 0$)



`` Nanotubes not fully converted appeared also during short
runs with higher working pressure. HRTEM observations
revealed an amorphous phase inside some of the nanotubes’
hollow cores, generally near the nanotubes tip (Fig. 5a). The
amorphous phase occupies only a small fraction of the nanotube’s
core volume. A meniscus is found to form at the contact
point between the amorphous matter and the nanotube’s walls.
Fig. 5b displays such a meniscus in the nanotube core (marked
by arrows). The presence of this meniscus indicates that this
amorphous material solidified from a molten phase during the
cool-down period of the sample. The meniscus of the amorphous
phase suggest that the amorphous matter wets the
nanotubes’ walls. Since the WS2 nanotubes are hydrophobic,
this observation indicates that a monomolecular layer of oxide
is left on the entire hollow core of the nanotubes. The nanotube
walls near the contact area with the meniscus are quite defective,
probably due to the large differences between the thermal
expansion coefficients of the WS2 and the amorphous matter,
which induces strain during the cool-down period of the
reaction product. These observations are consistent with the
notion that the amorphous material inside the core is an oxide
phase which is hydrophilic and does not wet the hydrophobic
WS2 layers''\cite{Margolin2004}

\subsection{growth and properties}

WS2 photoluminescence spectra of few layer and nanotube:
NT electrical structures depend on chirality, diameter and layer No as well as strain. Theoretical calculation indicates the SWNT with diameter larger than 4nm should approach the single layer limit.\cite{Ghorbani-Asl2013}

Other chalcogenide has also been synthesized using this one-end sealed layout.\cite{Mukherjee2013}

\citeauthor{Zou2007} prepared W/\ce{WS2} core-shell NPs by reaction of tungsten and sulfur under hydrogen atmosphere.\cite{Zou2007}

CVD 1L WS2 PL.\cite{Peimyoo2013} (of NTU Yu group) PL peak at 635nm, width 40 meV, 

CVD 1L WS2.\cite{Cong2013} (of NTU Yu group) 457 nm excitation, PL at 525nm and 630nm, 

\ce{MoS2} sing-walled nanotube.\cite{Xiao2014}

1T MoS2: metallic phase a negative temperature coefficient for conductivity, XRD pattern identified. \cite{Wypych1992}

stable 1T WS2 multiwalled NT by Re doping.\cite{Enyashin2011}. 2H to 1T transition formerly known only for WS2 and MoS2 intercalated by alkali metals. 3R transition to 2H upon heating since 2H is the most stable one.

1T \ce{MoS2} Raman. \cite{Yang1991} strong peaks at 156, 226, and 330 cm-1. M point frequencies measured by neutron scattering. M point is folded into BZ zone center due to the formation of superlattice.

Electrons and Phonons in Layered Crystal Structures, edited by T. J. Wieting (Reidel, Dordrecht, Holland, 1979).

\ce{WS2} p-type or n-type.  Fermi level at the surface of semiconductor is pinned to a fixed position relative to the CBM and VBM by a sufficient density of surface states situated between CBM and VBM. \cite{Baglio1983}

Electronic structure of \ce{MoS2}.\cite{Eknapakul2014} K intercalating into bulk to create quasi-standing 1L. Large effective mass 0.6 $m_e$ found, implying low mobility. Direct gap 1.88eV is measured.

Self-assembled monolayer (SAM) on \ce{SiO2} and its effect on \ce{MoS2} 1L.\cite{Najmaei2014}

\ce{WS2} 1L doping calculation. \cite{Ma2011}




multipeak Lorentzian fitting. 270 to 410 cm


\citeauthor{Shi2013} studied the strained monolayer \ce{MoS2} and WS2. The results show that exciton binding energy is insensitive to the strain, while optical band gap becomes smaller as strain increases. Monolayer WS2 PL maximum located at about 1.95 eV. Calculation shows the electron effective mass of WS2 is the smallest, rendering higher mobility in device.\cite{Shi2013}

\citeauthor{Kosmider2013} studied the heterojunction between two monolayers of \ce{MoS2} and WS2. Top of VB in W layer and bottom of CB in Mo layer, forming type II structure. bilayer gap 1.2 eV.\cite{Kosmider2013}


Band structure  of \ce{MoS2} in bulk form was calculated by \citeauthor{Mattheiss1973}.The calculation result is 1.2eV (indirect gap).\cite{Mattheiss1973}

Alkali metal intercalated \ce{WS2} film was prepared.\cite{Homyonfer1997} Stage 6 superlattice formation was suggested according to X-ray diffraction, and photoresponse spectra and electron tunneling measurement were done.



quantitative Raman of MoS2 on insulating subs. intensity difference between supported and suspended was highlighted, detailed model in support info.Li2013

WO3-x (1nm) on SiO2/Si sulfurization at 750-950 degree,\cite{Elias2013}

decrease in dielectric screening and thereby enhanced excitonic effect.
DFT is not good at describing photoemission, GW approximation overcome this deficiency but still not enough for photoabsorption process in which ehps are created. BSE equation is used to compensate this discrepancy, WX2 exhibits larger spin-orbit splitting as compared to MX2 family.\cite{Ramasubramaniam2012}



arise as a result of, dispersal of Na by electron probe.



\section{CNT}

SOI:

VSS, growth kinetics,
BN nanotube capping, zigzag is more stable than armchair. \cite{Menon1999}


To develop large-size single-crystal graphene on dielectric substrates. small carbon flow near-equilibrium CVD process. Grain size about 10 microns, precursor \ce{CH4} and \ce{H2} (ratio 2.3:50) at 1180 C. \ce{SiO2}-Si surface roughness. Although the growth substrates (quartz,\ce{SiO2}-Si and \ce{Si3N4}-\ce{SiO2}-Si ) have a complicated stereo network similar to diamond, regular hexagonal G growth is obtained, which indicates the deposition is determined by equilibrium kinetics, and this should be applicable to other 2D materials as well. I2D/IG exceeds two on \ce{SiO2}-Si subs (514.5nm), indicating monolayer G. armchair (AC) G edge grows faster than zigzag (ZZ) edge.\cite{Chen2013j}

catalytic graphitization of solid carbon sources. catalytic transformation, the source is in solid state, low temperature (less than 600C), 2nm  \ce{Al2O3} by ALD as carbon diffusion barrier. amorphous silicon (a-Si), Ni lower the activation barrier ,  tetrahedral amorphous carbon (ta-C).\cite{Weatherup2013}

low energy (50eV) ion implantation doping in G. Ions penetrate pristine G at energy larger than 100eV. Individual substitutional incorporation of B into G lattice is demonstrated. 1\% doping level was obtained. \cite{Bangert2013}


CVD G on copper. Size of single crystal domain and nucleation site density.\cite{Wu2013b}

Concentration of charge carrier $n$ is related to gate voltage $V_g$ by:
\[
n = \frac{\epsilon_0 \epsilon V_g}{ed}
\]
where $\epsilon_r = \epsilon_0 \epsilon$ is dielectric constant of gate materials.

massless relativistic chiral particles, Klein paradox, 100\% tunneling and extreme high mobility.

\ce{CaF2} a material suitable for scattering efficiency S comparison measurement due to its large band gap ($S\times \omega_L^4$ is constant below 5eV).

symmetry-breaking mechanism,

low energy ion doping of graphene.\cite{Ahlgren2011}

\section{misc}

E-beam spatial coherence.\cite{Morishita2013} phase contrast transfer function, coherence estimated by the visibility of double slits interference fringes, an effective diameter in specimen plane.  Image is a result of convolution between object and lens, point source on the focal plane, after lens the EM wavefront intersect image plane at different angle $\theta = d/f$, 

nucleation and film growth \cite{Hanbucken1984}

intrinsic silicon equilibrium charge carrier concentration at RT is $n_i = p_i = 1.5 \times 10^{10} cm^{-3}$, much smaller than silicon atoms density as $5\times 10 ^{22} cm^{-3}$.

The average distance between dopant atoms is cubed root of concentration, $d = (10^{18} cm^{-3})^{-1/3} = 10nm$.

The electron mobility $\mu_n = 1500 cm^2/V\cdot sec $ at RT for Si, and hole mobility $\mu_p = 450 cm^2/V\cdot sec$ at RT.

for p-type silicon, when the conductivity $\sigma = 1 (ohm cm )^{-1}$, the doping level is
$N_A = \frac{\sigma}{q \mu_p}= 1 / (1.6E-19 \times 450) = 1.4E16 cm^{-3}$.

Built-in voltage $V_0 = \frac{kT}{q}ln(N_A N_D/n_i^2)$, depletion region width $W = \sqrt{\frac{2 \epsilon_{Si} V_0}{q}(1/N_A + 1/N_D)}$, where $\epsilon_{Si} = 11.7 \epsilon_0$. When applying external field, depletion width $W = \sqrt{\frac{2 \epsilon_{Si} (V_0 - V) }{q}(1/N_A + 1/N_D)}$

The capacitance of p-n junction is $C = A \sqrt{\frac{q \epsilon_{Si}}{2(V_0 -V)}(N_D N_A/(N_A + N_D))}$.


oxygen plasma treatment on HF-etched Si (001). reaction among $e$, \ce{O^+}, \ce{O2^+}, \ce{O^-},\ce{O2}. \ce{OH}-terminated surface obtained.\cite{Habib2010}

\subsection{MB}
MB is a heterocyclic aromatic dye which is blue colored in oxidizing environment. Upon reduction, MB is turned into colorless leuco MB. This can be used as an oxygen indicator in food industry. Photo-bleaching of MB can be also due to its leuco formation rather than total decomposition. Photocatalytic decomposition can be minimized by keeping the solution at acidic condition (PH = 4), which will limit the formation of oxidative hydroxyl radicals (E = 2.8eV vs normal hydrogen electrode). Oxygen dissolved in the solution play a key role in conversion of LMB to MB under visible light. Purging with \ce{N2} for 20mins can remove dissolved oxygen. \cite{Wang2014a}

solar energy harvesting representative study.\cite{Yoneyama1972} MB to LMB (\ce{C16H19N3S}) in aqueous solution upon illumination of \ce{TiO2}. The colorimetric analysis was performed in a glove box under nitrogen atmosphere. The absence of oxygen is important to prevent the oxidation of LMB to blue MB.
\[
\cee{MB^+ H2O + H^+ \rightarrow MBH3^{2+} + 1/2O2}
\]
where MB represents the uncharged center of MB molecule.

common wisdom expect that a dye incapable of injecting an electron at the excited state to CdS. MB, which process N-methyl groups in its molecular structure and does not sensitize CdS is an exemplary candidate. quantum efficiency is defined as probability of MB converted to azure B per incident photon. QE of CdS to MB decomposition is reduced in nitrogen bubbling treated solutions, indicating the necessity of oxygen. Two possible mechanisms: a) adsorbed oxygen acts as a trap for the conduction electron and prevent the accumulation of negative charge within space charge region of CdS, supported by the formation of \ce{O2^-} in excitation of CdS in aqueous suspension.\cite{Takizawa1978}

ref 16, MB aqueous solution stability. Liquid chromatogram, azure B (trimethylthionine), and thionine. Electrochemical measurement,

MB adsorption.  photocatalytic oxidation of MB by \ce{TiO2} film. photo-oxidation reaction occurs at the surface of photocatalyst. Mb molar extinction coefficient was found to be 66700 1/cm 1/M. Langmuir adsorption isotherm.\cite{Matthews1989}

\[
[MB]_{ads} = \frac{k_1 k_2 [MB]}{1 + k_1[MB]}
\]
and integrated form of Langmuir adsorption isotherm
\[
t = \frac{1}{k_1K} In\frac{[S]^0}{[S]} + \frac{1}{K}([S]^0 - [S])
\]
where $K = k_2 \phi N T_r$, with $\phi$ as quantum yield, N as total absorbed photons, and $T_r$ as rate of transport.
\[
\cee{C16H18N3SCl + 25.5O2 \rightarrow 16CO2 + 6H2O + 3HNO3 + H2SO4 +HCl}
\]
which indicates the total oxidation of $10 \mu M$ MB would exhaust the ambient oxygen concentration of initially air-equilibrated solutions (about $250 \mu M$ ). ref 28 Thus the transport of both oxygen and MB to the photocatalyst surface are anticipated to be key factors.

photoelectrochromism at \ce{TiO2}/MB interface and its control. Efficient capture of photogenerated holes by a reducing agent is crucial to the reversibility of bleach-recoloration transition. This transition is kinetically dictated by electron transfer. Holes transfer is not desired.\cite{DeTacconi1997}

256 nm band is associated to the presence of LMB. LMB formation is not favored at alkaline pH values in aqueous solution. The OH radicals are generated either with the surface hydroxyl groups on \ce{TiO2} or with water, and its high oxidizing power cause photocatalytic decomposition of the dye.

An elementary step in decomposition of MB is N-dealkylation, which is preceded by radical cation formation.\cite{Takizawa1978} This radical cation can be spectroscopically monitored by the presence of 520nm band for MB. In MB absorption spectrum, 664 and 614 nm band ratio is related to monomer and dimer relaxation.
\begin{align}
\cee{TiO2 &\rightarrow e_{CB}^- + h_{VB}^+ \\
h_{VB}^+ + red &\rightarrow ox\\
MB^+ + 2e_{CB}^ + H^+ &\rightarrow LMB}
\end{align}

Measure the ratio between 614 and 663 nm before and after adding WS2 can indicate the adsorption of monomer and dimer MB.

MB can act as sacrificial electron acceptor in the reduction to leuco form. The decomposition is favored under oxygen-rich environment. MB feature peaks at 663, 614 and 292 nm, and $\epsilon_{660}=10^5 M^{-1}cm^{-1}$. The doubly reduced form of MB, LMB has feature peak at 256 nm. The singly reduced form of MB, \ce{MB.^-} is pale yellow, with peak at 420nm.\cite{Mills1999}
\[
\cee{MB + e_{CB}^- ->[pH<7] MB.^-}
\cee{2MB.^- \rightarrow MB + LMB}
\cee{O2 + e_{CB}^- \rightarrow O2.^-}
\]

The oxidized form of MB, \ce{MB.^+} has peak at 520nm, which is stable in acidic solution, but decomposes irreversibly in slight alkaline solution(pH = 9).
thionine peaks at 600nm.
MB forms dimers in aqueous solution,
\ce{
2MB <=>[K_D] (MB)_2
}
A typical value of $K_D$ is 3970 1/M. A quadratic equation can be solved to obtain the monomer concentration:
\[
2K_D [MB]^2 + [MB] - [MB]_{total} = 0
\]
MB adsorption on metal oxides. Monomer size is less than 1.5nm.
Logarithmic acid dissociation constant $pK_a= -\log_10 \frac{[A^-][H^+]}{[HA]}$. The oxidation potential for \ce{H2O}-\ce{O2} couple is 1.23V and 0.817V versus NHE at pH 0 and pH 7, respectively.

%\begin{align}
%\cee{MB + SED &->[TiO2][h\nu \leq 3.2eV] LMB + SED^{2+}\\
%2LMB + O2 &\rightarrow 2MB + 2H2O}
%\end{align}


S.L. Murov, I. Carmichael, G.L. Hug, Handbook of Photochemistry, 2nd revised ed. Marcel Dekker, New York, 1993.

aerobic or anaerobic, dimerise, photominerlization, gas to liquid transfer.

Mb to LMB transition as visual time monitor. commercial colorimetric oxygen indicators. radical-bearing carbon with unpaired electrons. MB = \ce{MB^+Cl^-}.\cite{Galagan2008}

monomer MB and dimer MB kinetics.\cite{Spencer1979}



MB. \cite{Lee2003a}
\begin{align}
\cee{ 2LMB &->[\text{UV}] LMB^*\\
2LMB^* + O2 &\rightarrow 2MB^+ + 2OH^-}
\end{align}

\[
\cee{2LMB ->[\alpha] LMB^*}
\cee{2LMB ->[\text{above}] LMB^*}
\]


\section{solar cell}


\ce{TiO2} NPs for high loading of sensitizing dye. Hole conducting electrolyte with \ce{I^-} and \ce{I3^-} concentration close to $10^19 cm^3$. chemical anchoring groups.

electron injection rate. e transfer rate is several order faster than hole. 1 M = $6\times10^{20} cm^{-3}$. \ce{I^-} is known to coordinate with the sulfur atoms on NCS ligand. ref 31.

FRET: dipole-dipole coupling, energy relay dye to sentisizing dye and then to \ce{TiO2}. analogous to photosynthesis bacteria. Time-resolved PL to measure FRET $R_0$.

photocatalyst review.\cite{Mills1997} Definition: catalysis should not be used unless it can be demonstrated that the turnover number\footnote{the number of product molecules per number of active sites.} is greater than unity. Otherwise, semiconductor-assisted photoreaction is more appropriate. aerated, flush with air; nitrogen-purged. Degussa P25 \ce{TiO2} high temperature flame hydrolysis of \ce{TiCl4} in presence of hydrogen and oxygen. Oxidization of organic species is presumably obtained by \ce{Ti^{IV}OH^{.-}}, rather than direct hole transfer. carrier decay pathways. deactivation of catalyst by intermediate product.

MB natural decoloration under sunlight is found to be about 18\%.\cite{Nogueira1993} Latitude: 24 south, $3.4mW/cm^2$. natural evaporation should be prevented or corrected.

\section{dissertation}

WS2 surface hydrophilic or hydrophobic.
\[
\cee{Ti(IV)-OCH3 + h^+ -> Ti(IV)-O^+CH3}
\]

\ce{WO3} photoanode should be a n-type semiconductor, stable in acidic aqueous solution.

Swagelok TM. List of equipments,

Zheng thesis: \ce{TiO2} anodic nanotube by sputtering Ti on FTO and anodizing in F-organic electrolyte.

heat cure gasket ( ionomer surlyn 1702 Dupont), 125 C for 30s.
vacuum back filling.



\section{vocabulary}
$\delta\omega/\epsilon$ 

\subsection{pronunciation}

chamber, vias, valve, figure, energetic, managerial, inert, volatile, chromic,
laminar, palladium, platinum, photovoltaic, acronym, chirality, stoichiometric, cyclic voltammetry, quasi, pseudo, 

\section{job related paper}

OLED with \ce{MoO3} as charge generation layer. \cite{Kanno2006} The PI is stephen forrest, also a cofounder of udc oled.\footnote{http://www.umich.edu/~ocm/research.html}

stacked led, luminescence linearly increase with layer no at fix current density. hole transporting layer by MoO3, and electron transporting layer by Li.

OLED photovoltaic.\cite{Xiao2012a} functionalized squaraines as donor

excitation state management \cite{Zhang2012b}.
triplet-triplet annihilation (TTA) and singlet-triplet annihilation (STA). spin number of exciton is either 0 (singlet) or 1 (triplet).

TTA, bound electron-hole pairs, introducing a heavy metal in organic molecule to enhance the spin-orbital coupling, enabling triplet emitters.\cite{Zhang2013i} yet when operating at high current, the efficiency is decreased, and TTA is considered as a intrinsic limit.


(T. Tsujimura OLED Displays: Fundamentals and Applications John Wiley \& Sons Inc., Hoboken, New Jersey (2012))

why graphene on copper: a review article. \cite{Mattevi2011} research thrust, post CMOS fab tech. Exposure of hydrocarbon or evaporated carbon onto transition metal, the formation of graphite was surmised as a consequence of diffusion and segregation of carbon impurities from bulk to surface. carbon solubility in the metal and. 

The lack of control over layer No on Ni is partially attributed to the face that the segregation of carbon from the metal carbide upon cooling occurs rapidly within the Ni grains and heterogeneously at the grain boundaries. phase diagram of Ni and C reveals that the solubility of carbon in nickel at high T > 800 C form a solid solution. metastable phase \ce{Ni3C} promote the precipitation of C out of Ni. Carbon preferentially precipitates out at the grain boundaries of polycrystalline Ni subs so the thickness at graphite is higher than within grains. On Fe, due to the high affinity between Fe and C (\ce{Fe3C} is a stable carbide), the formation of $sp^2$ crystalline carbon film is suppressed. 

on copper (decomposition of methane gas at 1100 C), independent of heating or cooling rate. For copper, the 3d shell is filled, leading to less reactive configuration and weaker affinity towards Carbon. Cu can only form soft bonds with C via the charge transfer from the $\pi$ electron in  sp2 hybridized C to the empty 4s states of copper, as supported by the fact that copper does not form any carbide phase, and low C solubility. This low affinity and weak bond makes copper a true catalyst fro graphitic carbon formation. pre-treatment of copper foil, \ce{CuO}, \ce{Cu2O} removal by reducing annealing at 1000 C. 

epitaxial and lattice mismatch is present, 

\textbf{four-point probe}

sheet resistivity. a current source in an infinite sheet gives rise to the logarithmic potential
\[
\phi - \phi_0 = - \frac{I\rho}{2\pi}\log r,
\]
the potential for a dipole becomes
\[
\phi - \phi_0 = \frac{I\rho}{2\pi}\log r_1/r_2,
\]

When equal spacing probes are used, then potential difference between two inner points is 
\[
\delta\phi = V = \frac{I \rho}{\pi}\log2,
\]
so sheet resistivity $\rho$ is obtained as
\[
\rho = 4.5324V/I.
\]




\section{The physics of semiconductor}


quantum well devices exploit spatial quantization effects to increase the efficiency as well as alter the lasing threshold, and  novel semiconductor is made to emit light at wavelengths different from those possible when only bulk material is used.


The discontinuity in the CB and CB occurs at the interface. The corresponding potential discontinuity create a potential difference, forming a trap in which electron or hole can only have discrete energies.

a material of smaller bandgap is sandwiched between two layers of material of greater bandgap, or vice versa.

transmissivity coupled in multiple quantum well structure.

band structures=potential energy diagrams.

\textbf{junctions}
p-n homojunctions, p-n or n-n heterojunctions, metal-semi junctions, most important types: Schottky barriers, or ohmic contact.

chemical potential (Fermi level) as a measure of particle concentration.  Fermi level at 0K is equal to Fermi energy, which is defined as the energy of the topmost filled orbital. In equilibrium, the Fermi level $E_f$ is uniform throughout the material. n electron carrier, p holes carrier.

band diagram represents the electron's potential energy.

band bending: the electron energies are greater on the p side than on the n side, or the electrostatic potential is greater on the n side than on the p side, since potential $V = E/-q$.

The built-in potential for a homojunction is equal to the full band bending in equilibrium.

heterojunction: bandgap discontinuity, to solve the Poisson's equation for band bending.

Metal-Semi: metal cannot support any potential difference across it. Fermi level in Semi is pinned at the interface. Electron transfer from n-type semi to metal, leaving ionized donors behind.

ohmic contact forms if the work function of the metal is less than that of the semi. net flow of electron from metal into semi, no depletion layer forms.

metal-oxide-semi as MIS structures is capacitive in nature since no dc current flows under bias.

ch11.5 nonequilibrium conditions
quasi-fermi level $\phi_n$ or $\phi_p$.

\section{Gary semiconductor fabrication} 

photolithography a patter transferring process from mask to photoresist. clean room needed to remove dust particles, which could cause dislocation on an epitaxial film, low breakdown voltage in gate oxide, or short circuit. 

resolution, registration for effectiveness, throughput for efficiency, shadow printing where mask and wafer in direct contact: cons, dust could case permanent damage to mask, a small gap $d$ (10-50 $\mu$m ) used usually, and the minimium linewideth CD is roughly $\sqrt{\lambda d}$; and projection printing , resolution $l_m = \frac{k_1 \lambda}{NA}$, and depth of focus $\frac{k_2 \lambda}{NA^2}$. 193-nm using ArF excimer laser, and 157 nm using \ce{F2} excimer laser. 

365 nm probably used for \ce{LiNbO3} waveguide. 

resolution enhancement using phase shifting mask,  using the electric field of EM wave to chemically activate the photoresist. A $\pi$ phase change is obtained by using a transparent layer of $d = \lambda/2(n - 1)$ thickness. near-field diffraction, 

EUV 10 - 14 nm, C inner shell electron transition, 


\section{non-imaging optics}

black body radiation power density $S$ is about 1 \si{kW\per m^2}, the balanced temperature $T = \sqrt[4]{S/\sigma} = 364$ K, where $\sigma$ is Stefan-Boltzmann constant. 

ray tracing in vector form as $n^{\prime} r^{\prime} \times n = n r \times n$, invariant $na\theta$, 






















\chapter{one}
\section{level one}
Inline lists, which are sequential in nature, just like enumerated lists, but are
\begin{enumerate*}[label=\itshape\alph*\upshape)]
\item formatted within their paragraph;
\item usually labelled with letters; and
\item usually have the final item prefixed with `and' or `or',
\end{enumerate*} like this example.

\begin{hypothesis}
After 20 minutes sonication, the dispersion is milky. We measured the transmission spectra of this dispersion immediately upon sonication and after 40 hrs for gravity sedimentation.
\end{hypothesis}
\subsection{level two }

\subsubsection{level three}
After 20 minutes sonication, the dispersion is milky. We measured the transmission spectra of this dispersion immediately upon sonication and after 40 hrs for gravity sedimentation. The t0 line mainly arises from the scattering of flakes in the dispersion. There are two scattering mechanisms: Rayleigh scattering and Tyndall scattering. Rayleigh scattering, which occurs when particles are comparable to wavelength, is inversely proportional to the fourth power of wavelength; While Tyndall scattering, which occurs when the particles are larger, is inversely proportional to the square of the wavelength. To fully evaluate the true absorbance of the sample itself, one needs to decouple the scattering part from apparent absorbance. Generally we select one part of spectrum and assume it is only caused by scattering. Then a empirical dependence shown in Eq.~\ref{eq:sca2} is used as polynomial fitting model. test.

\begin{align}
Abs_{sca}  & = a\times \lambda^{n}  \label{eq:sca1}\\
\log{Abs_{sca}} & = \log{a} + n*\log{\lambda} \label{eq:sca2}
\end{align}

After least-squares fitting, the coefficients $a$ and $n$ can be used to estimated the scattering in other wavelengths. The fitting of n usually falls between -2 and -4. We examined the t40h line and found the scattering part is almost zero. Therefore we did not perform aforementioned fitting. This is not always the case. In chapter on the \ce{WS2} section, we do need to do so.

\begin{itemize}
\item s inversely proportional to the fourth
\item  inversely proportional to the fourth
\item s inversely proportional to the fourth
\end{itemize}

\begin{figure}[htb]
\centering
\subcaptionbox{\label{fig:moind}}{\includegraphics[width=0.45\linewidth]{n_MoO3}}%
\subcaptionbox{\label{fig:mocon}}{\includegraphics[width=0.45\linewidth]{MoO3_1L}}
\caption[Refractive indices of \ce{MoO3}]{(a) Refractive indices of \ce{MoO3} Dashed line is n along $a$ axis while solid line is along $c$ axis. (b) optical contrast mapping of 1L \ce{MoO3} on \ce{SiO2}-Si. The x axis is \ce{SiO2} thickness, y axis is wavelength.It is assumed that the incident light is polarized along c axis of \ce{MoO3}.}
\label{fig:mofl}
\end{figure}
%[width=0.45\textwidth]

Rayleigh scattering and Tyndall scattering. Rayleigh scattering, which occurs when particles are comparable to wavelength, is inversely proportional to the fourth power of wavelength; While Tyndall scattering, which occurs when the particles are larger, is inversely proportional to the square of the wavelength. To fully evaluate the true absorbance of the sample itself, one needs to decouple the scattering part from apparent absorbance. Generally we select one part of spectrum and assume it is only caused by scattering. Then a empirical dependence shown in Eq.~\ref{eq:sca2} is used as polynomial fitting model.
\begin{figure}[htb]
\centering
\subcaptionbox{\label{fig:moxrd}}{\includegraphics[width=0.45\linewidth]{xrd_moo3}}%
\subcaptionbox{\label{fig:moram}}{\includegraphics[width=0.45\linewidth]{xrd_moo3}}
\caption[Crystalline phase characterization of \ce{MoO3} on Si]{(a) XRD pattern and (b) Raman spectrum of typical \ce{MoO3} on Si, $\lambda_{ex} = 532nm$.}
\label{fig:mooxch}
\end{figure}


piranha clean of FTO. 50ms switch.\cite{Scherer2012} nanoscale Kirkendall effect: the outward diffusion of metal cations are balanced by an influx of vacancies. For example, diffusion coefficient of Ni in NiO is higher than that of oxygen.
\subsection{EC windows and thin film batteries}

battery and ECD.\cite{Granqvist2012} electrolyte: PVB (poly vinyl buteral).
alternative materials and design: organic, Prussian Blue as EC materials, metal hydrides, suspended particle device, liquid crystal, electroplating,
challenges: large area nanoporosity, transparent conducting contact, electrolyte with good ionic conductivity and poor electronic conductivity, stable under UV; assembly and large scale manufacturing;
cathodic coloration:
anodic coloration:
The coloration mechanism: \ce{MO6} octahedrons lead to $e_g$ and $t_{2g}$ level and ion channelling.
ref54,60,65,66,200,209,


\ce{WO3} as cathodic and either polyaniline(PANI) or Prussian white (PW) as anodic electrochromic half cells. \cite{Heckner2002}

Characterization of ECD includes transmission measurement and associated EC calculation, charge-discharge time, current-time curve and the fitting of obtained data.

\begin{quote}
a viable electrochromic smart window must exhibit a cycling life time \textgreater $10^5$ cycles, corresponding to an operation life at 10 - 20 years.
\end{quote}


\citeauthor{Sella1998} studied the optical and structural properties of RF sputtered thin film of \ce{WO3} and \ce{VO2} for electrochromic devices. Ionic conductor was built using transparent polymer electrolyte, which was prepared from a solution of 1M \ce{LiClO4} in propylene carbonate which was mixed with methylmetharcylate (MMA). The main characteristics of polymer electrolyte were: viscosity at 25 \si{\degreeCelsius} $\approx$ 12920 Pa.s, conductivity $\approx 10^{-2}-10^{-4}$\si{\per\ohm\per cm},non-hygroscopic if PMMA concentration \textgreater 30\%. A specific counter-electrode layer was not used since the encapsulated polymer electrolyte processes a very high ion storage capacity.\cite{Sella1998}


   %\glsaddallunused
    \appendix
    \chapter{math}
    \renewcommand{\bibname}{REFERENCES}
    \begin{singlespace}
    \printbibliography
    \end{singlespace}

\end{document}
