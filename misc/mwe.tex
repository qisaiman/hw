\documentclass[12pt,oneside]{book}
 %
   \usepackage{uncc-thesis} % some format specifications define page number in up right corner
   %-------page layout--------%
% adapted from <http://www.khirevich.com/latex/page_layout/>
%\usepackage[DIV=14,BCOR=2mm,headinclude=true,footinclude=false]{typearea}

%\makeatletter
%\if@twoside % commands below work only for twoside option of \documentclass
%    \newlength{\textblockoffset}
%    \setlength{\textblockoffset}{12mm}
%    \addtolength{\hoffset}{\textblockoffset}
%    \addtolength{\evensidemargin}{-2.0\textblockoffset}
%\fi
%\makeatother

% packages used in uncc-thesis

%\RequirePackage{ifthen}
%\RequirePackage{setspace} % for double spacing
%\RequirePackage{comment}
%\RequirePackage{epsfig}
%\usepackage{sectsty} % for sectional header style. Alternative: titlesec package
% \usepackage{tocloft}
%\usepackage{geometry}

\usepackage{microtype} % better layout
%----inherent of article class-------%
%\usepackage[utf8]{inputenc} % set input encoding (not needed with XeLaTeX)

%--- for font ----
% \usepackage[T1]{fontenc}
% \usepackage{textcomp}

%\usepackage{mathptmx} % fine but not truetype
% \usepackage{newtxtext}
% \usepackage{pslatex} % not bad

\usepackage{fontspec} % to compile with LuaLatex
\setmainfont{Times New Roman} % to compile with LuaLatex

 %-- end of font adaption

\usepackage{graphicx} % support the \includegraphics command and options
% check pdftex option
\usepackage{placeins} %

\usepackage{subfig}
\usepackage{float} % for images
 \graphicspath{{./gallery/}} %--added by author

\usepackage[font=small, labelfont=bf]{caption}
%\usepackage{subcaption}


% It supplies a landscape environment, and anything inside is basically rotated.(http://en.wikibooks.org/wiki/LaTeX/Page_Layout)
%\usepackage{lscape}
\usepackage{pdflscape}
%\usepackage{rotating} % use \begin{sidewaystable}

% Helps format tables using the \toprule, \midrule, and \bottomrule commands (http://en.wikibooks.org/wiki/LaTeX/Tables#Using_booktabs)
\usepackage{booktabs}
%\usepackage{multirow} for multirow in tables
%  Helps format tables (http://en.wikibooks.org/wiki/LaTeX/Tables#Using_array)
%\usepackage{array}

%%-- Additional style---- modified by Tao Sheng 12/20/12
% for chemical formula, subscripts etc
\usepackage[version=3]{mhchem}
\usepackage{siunitx}
  \DeclareSIUnit \torr{Torr}
%%-- mathmatical symbols and equations-----
\usepackage{amsmath}
\usepackage{amssymb}
 %\numberwithin{equation}{section}
 %\numberwithin{figure}{section}
 \providecommand*{\ud}{\mathrm{d}}

%------- bibliography and citation ------
\usepackage[english]{babel}% Recommended
\usepackage{csquotes}% Recommended
\usepackage[style=numeric-comp,
		    sorting=nty,
            hyperref=true,
            url=false,
            isbn=false,
            backref=true,
            maxcitenames=2,
            maxbibnames=4,
            block=none,
            backend=bibtex,
            natbib=true]{biblatex}
% \usepackage[bibencoding=latin1]{biblatex}

\DefineBibliographyStrings{english}{%
    backrefpage  = {see p.}, % for single page number
    backrefpages = {see pp.} % for multiple page numbers
}
% suppress 'in:'
\renewbibmacro{in:}{%
  \ifentrytype{article}{}{\printtext{\bibstring{in}\intitlepunct}}}
% document preamble
% removes period at the very end of bibliographic record
\renewcommand{\finentrypunct}{}
% removes pagination (p./pp.) before page numbers
\DeclareFieldFormat{pages}{#1}


\providecommand*{\bibpath}{E:/spring2012/Ubuntu/Latex/Mendeley_Bib_lib}
\bibliography{\bibpath/arix.bib,\bibpath/ECD.bib,\bibpath/tungsten_newandgood.bib,\bibpath/ACSnano.bib,%
\bibpath/tungsten_old.bib,\bibpath/Raman.bib,\bibpath/Molybdenum.bib,%
\bibpath/tungsten_cl.bib,\bibpath/optics,\bibpath/CVD.bib,\bibpath/sodium.bib,\bibpath/VLS.bib}

%-- for works around, packages conflicts----

%-- redefine toc macros ------
\addto\captionsenglish{%
\renewcommand\chaptername{CHAPTER}%
\renewcommand\appendixname{APPENDIX}%
\renewcommand\indexname{INDEX}%
\renewcommand{\contentsname}{TABLE OF CONTENTS}%
\renewcommand{\listfigurename}{LIST OF FIGURES}%
\renewcommand{\listtablename}{LIST OF TABLES}%
}

%-- misc---
\usepackage{lipsum}
\usepackage{latexsym}
 \providecommand*{\thefootnote}{\fnsumbol{footnote}}

\usepackage{xcolor}
\usepackage{listings}
\lstset{
 frame = single,
 language = matlab,
 breaklines = true,
postbreak=\raisebox{0ex}[0ex][0ex]{\ensuremath{\color{red}\hookrightarrow\space}}
}

\usepackage{enumitem}
\setlist{nolistsep}

\setcounter{secnumdepth}{3} % show numbering of subsubsection

%% -- links ---
\usepackage{hyperref}
\hypersetup{
colorlinks,%
citecolor=black,%
filecolor=black,%
linkcolor=black,%
urlcolor=black
} % make all links black

\usepackage[acronym,toc,nonumberlist]{glossaries} % loaded after hyperref

 % additonal packages and tweaking
     \newacronym{ald}{ALD}{atomic layer deposition}
   \newacronym{afm}{AFM}{atomic force microscopy}
   
   \newacronym{bse}{BSE}{backscattering electron}
   
   \newacronym{cvd}{CVD}{chemical vapor deposition}
   \newacronym{ccd}{CCD}{charge coupled device}
   \newacronym{cft}{CFT}{crystal-field theory}
   \newacronym{cnt}{CNT}{carbon nanotube}
   \newacronym{cb}{CB}{conduction band}
   \newacronym{cbm}{CBM}{conduction band minimum}
   \newacronym{cs}{CS}{crystallographic shearing}
   \newacronym{cl}{CL}{cathodoluminescence}
   \newacronym{cdw}{CDW}{charge density wave}
   \newacronym{cbed}{CBED}{convergent electron beam diffraction}
   
   \newacronym{dp}{DP}{diffraction pattern}
   \newacronym{drs}{DRS}{diffuse reflection spectroscopy}
   
   \newacronym{ecd}{ECD}{electrochromic device}
   \newacronym{ec}{EC}{electrochromic}
   \newacronym{ebeam}{E-beam}{electron beam}
   \newacronym{eels}{EELS}{electron energy loss spectroscopy}
   \newacronym{edx}{EDX}{Energy Dispersive X-ray Spectroscopy}
   
   \newacronym{fl}{FL}{few-layer}
   \newacronym{fet}{FET}{field effect transistor}
   \newacronym{fwhm}{FWHM}{full width at half maximum}
   
   \newacronym{her}{HER}{hydrogen evolution reaction}
   
   \newacronym{hrtem}{HRTEM}{high resolution TEM}
   
   \newacronym{kmt}{KMT}{Kubelka-Munk theory}
   
   \newacronym{ml}{ML}{mono-layer}
   \newacronym{mwnt}{MWNT}{multiple-walled nanotube}
   \newacronym{mfp}{MFP}{mean free path}
   
   \newacronym{np}{NP}{nanoparticle}
   \newacronym{nw}{NW}{nanowire}
   \newacronym{nhe}{NHE}{normal hydrogen electrode}
   \newacronym{nni}{NNI}{National Nanotechnology Initiative}
   
   \newacronym{pmma}{PMMA}{Poly[methylmethacrylate]}
   \newacronym{pdms}{PDMS}{polydimethylsiloxane}
   \newacronym{pl}{PL}{photoluminescence}
   \newacronym{pec}{PEC}{photoelectrochemical}
   \newacronym{pvd}{PVD}{physical vapor deposition}
   
   \newacronym{sem}{SEM}{Scanning Electron Microscopy}
   \newacronym{se}{SE}{secondary electron}
   \newacronym{sers}{SERS}{surface-enhanced Raman scattering}
   \newacronym{saed}{SAED}{selected area electron diffraction}
   \newacronym{swnt}{SWNT}{single-walled nanotube}
   \newacronym{sccm}{sccm}{standard cubic centimeters per minute}
   
   \newacronym{tmo}{TMO}{transition metal oxide}
   \newacronym{tmdc}{TMDC}{transition metal dichalcogenide}
   \newacronym{tem}{TEM}{Transmission Electron Microscopy}
   
   \newacronym{vb}{VB}{valence band}
   \newacronym{vbm}{VBM}{valence band maximum}
   \newacronym{vs}{VS}{vapor-solid}
   \newacronym{vls}{VLS}{vapor-liquid-solid}
   \newacronym{vss}{VSS}{vapor-solid-solid}
   
   \newacronym{xrd}{XRD}{X-ray Diffraction}
   \newacronym{zzz}{ZZZZZZZZZZZZZZZZ}{X-ray Diffraction}
   

   \newglossaryentry{electrochromic}{
   name=electrochromic,
   description={a phenomenon that materials be colored with positive voltage and bleached with reverse voltage}}

   \newglossaryentry{nanotechnology}{
   name=nanotechnology,
   description={is science, engineering, and technology conducted at the nanoscale, which is about 1 to 100 nanometers.}}

   \newglossaryentry{ceramics}{
   name=ceramics,
   description={compounds often including oxides, nitrides and  carbides}}

   \makeglossaries
   

\newglossarystyle{mysuper3col}{%
  \renewenvironment{theglossary}%
    {\tablehead{}\tabletail{}%
     \begin{supertabular}{rp{0.1\textwidth}p{\glsdescwidth}}}%
    {\end{supertabular}}%
  \renewcommand*{\glossaryheader}{}%
  \renewcommand*{\glsgroupheading}[1]{}%
  \renewcommand*{\glossaryentryfield}[5]{%
    \glstarget{##1}{##2} &  & ##3\\}%
  \renewcommand*{\glsgroupskip}{}%
}

%% https://github.com/bhulliger/CRF/blob/master/doc/UserManual/glossary-super.sty
%% 
   \makeglossaries

\begin{document}

\title{Controlled growth and property measurement of one-dimensional oxide nanostructures for energy applications}
  \author{Tao Sheng}
  \university{The University of North Carolina at Charlotte}
  \location{Charlotte}
  \thesisyear{\the\year}
  \degree{Doctor of Philosophy}
  \department{Physics and Optical Science}

  \committeemember{Dr. Haitao Zhang} % First sets the \advisor default.
  \committeemember{Dr. Tsinghua Her}
  \committeemember{Dr. Michael Fiddy}
  \committeemember{Dr. Stuart Smith}
  \committeemember{Dr. Yu Wang} % Maximum of 10 can be specified.

%  \advisor{Dr. Haitao Zhang}

\maketitle
% no space line between abstract and main text
\begin{abstract}
Transition metal oxides (TMOs) exhibit rich structures and useful properties, and could been used in solar energy harvesting (\emph{e.g.}, photoelectrochemistry, photocatalysis, and photovoltaics) and energy saving (\emph{e.g.}, electrochromism and photochromism) applications. Nano-engineering of these TMOs could produce a variety of nanostructures, modify the electronic and optical properties, potentially enhance the performances and extend their applications into new regimes. To fully harvest the advantages of these materials, a scalable growth method and a comprehensive understanding towards the growth-structure-property relation must be established. This Ph.D. study focused on one-dimensional (1D) tungsten trioxide (\ce{WO3}) and molybdenum trioxide (\ce{MoO3}) nanostructures. This dissertation synthesized 1D \ce{WO3} and \ce{MoO3} nanostructures using chemical vapor deposition (CVD) approach, characterized the morphologies and structures, and measured the optical properties of the as-prepared nanostructures. This study aims at accomplishing controlled and scalable growth of \ce{WO3} and \ce{MoO3} using CVD method, establishing the structure-property relations to rationalize nanostructure synthesis for energy applications, and presenting preliminary results on the devices assembled using the \ce{WO3} and \ce{MoO3} nanostructures.

\end{abstract}

\begin{dedication}
morphologies and structures, and measured the optical properties of the as-prepared nanostructures. This study aims at accomplishing controlled and scalable growth of \ce{WO3} and \ce{MoO3} using CVD method, establishing the structure-property relations to rationalize nanostructure synthesis for energy applications, and presenting preliminary results on the devices assembled using the \ce{WO3} and \ce{MoO3} nanostructures.

\end{dedication}

%\pagebreak{}

\begin{ackn}
morphologies and structures, and measured the optical properties of the as-prepared nanostructures. This study aims at accomplishing controlled and scalable growth of \ce{WO3} and \ce{MoO3} using CVD method, establishing the structure-property relations to rationalize nanostructure synthesis for energy applications, and presenting preliminary results on the devices assembled using the \ce{WO3} and \ce{MoO3} nanostructures.


\begin{minipage}[b]{0.95\textwidth}

\begin{flushright}
\vspace{1in}
\emph{Sheng, Tao}

January 2015

Charlotte, NC

\end{flushright}
\end{minipage}


\end{ackn}

%\pagebreak{}
%% --- Work around some bugs due to babel package---
% for more, see http://tex.stackexchange.com/questions/82993/how-to-change-the-name-of-document-elements-like-figure-contents-bibliogr?lq=1
\setcounter{tocdepth}{2}
% ----- end of work around--------

\clearpage
\tableofcontents
\clearpage
\listoftables
\clearpage
\listoffigures

%\thispagestyle{myheadings}
\renewcommand{\glossarypreamble}{\thispagestyle{myheadings}}
\renewcommand{\acronymname}{LIST OF ABBREVIATIONS}
\glossarystyle{long}
\printglossary[type=\acronymtype]
\addcontentsline{toc}{chapter}{\acronymname}
\clearpage

the parskip is \the\parskip, and the baselineskip is  \the\baselineskip. 


    
\chapter{test}
abc \si{\angstrom} abs

\printbibliography[category=papers]
\printbibliography[category=conferences]


\section{level one}
Inline lists, which are sequential in nature, just like enumerated lists, but are
\begin{enumerate*}[label=\itshape\alph*\upshape)]
\item formatted within their paragraph;
\item usually labelled with letters; and
\item usually have the final item prefixed with `and' or `or',
\end{enumerate*} like this example.

\begin{hypothesis}
After 20 minutes sonication, the dispersion is milky. We measured the transmission spectra of this dispersion immediately upon sonication and after 40 hrs for gravity sedimentation.
\end{hypothesis}
\subsection{level two }
\Gls{cft}, \gls{cvd}
\subsubsection{level three}
After 20 minutes sonication, the dispersion is milky. We measured the transmission spectra of this dispersion immediately upon sonication and after 40 hrs for gravity sedimentation. The t0 line mainly arises from the scattering of flakes in the dispersion. There are two scattering mechanisms: Rayleigh scattering and Tyndall scattering. Rayleigh scattering, which occurs when particles are comparable to wavelength, is inversely proportional to the fourth power of wavelength; While Tyndall scattering, which occurs when the particles are larger, is inversely proportional to the square of the wavelength. To fully evaluate the true absorbance of the sample itself, one needs to decouple the scattering part from apparent absorbance. Generally we select one part of spectrum and assume it is only caused by scattering. Then a empirical dependence shown in Eq.~\ref{eq:sca2} is used as polynomial fitting model. test.

\begin{align}
Abs_{sca}  & = a\times \lambda^{n}  \label{eq:sca1}\\
\log{Abs_{sca}} & = \log{a} + n*\log{\lambda} \label{eq:sca2}
\end{align}

After least-squares fitting, the coefficients $a$ and $n$ can be used to estimated the scattering in other wavelengths. The fitting of n usually falls between -2 and -4. We examined the t40h line and found the scattering part is almost zero. Therefore we did not perform aforementioned fitting. This is not always the case. In chapter on the \ce{WS2} section, we do need to do so.

\begin{itemize}
\item s inversely proportional to the fourth
\item  inversely proportional to the fourth
\item s inversely proportional to the fourth
\end{itemize}

\begin{figure}[htb]
\centering
\subfloat[]{\label{fig:mosem1}\includegraphics[width=0.45\textwidth]{mosemsi_a}}\hspace{0.04\textwidth}
\subfloat[]{\label{fig:mosem2}\includegraphics[width=0.45\textwidth]{mosemsi_b}}
\caption[Representative morphologies of \ce{MoO3} on Si]{(a) Low magnification and (b) high magnification SEM images of representative depositions of \ce{MoO3} on Si for non-catalytic growth.}
\label{fig:mosisem}
\end{figure}
%[width=0.45\textwidth]

Rayleigh scattering and Tyndall scattering. Rayleigh scattering, which occurs when particles are comparable to wavelength, is inversely proportional to the fourth power of wavelength; While Tyndall scattering, which occurs when the particles are larger, is inversely proportional to the square of the wavelength. To fully evaluate the true absorbance of the sample itself, one needs to decouple the scattering part from apparent absorbance. Generally we select one part of spectrum and assume it is only caused by scattering. Then a empirical dependence shown in Eq.~\ref{eq:sca2} is used as polynomial fitting model.



piranha clean of FTO. 50ms switch.\cite{Scherer2012} nanoscale Kirkendall effect: the outward diffusion of metal cations are balanced by an influx of vacancies. For example, diffusion coefficient of Ni in NiO is higher than that of oxygen.
\subsection{EC windows and thin film batteries}

battery and ECD.\cite{Granqvist2012} electrolyte: PVB (poly vinyl buteral).
alternative materials and design: organic, Prussian Blue as EC materials, metal hydrides, suspended particle device, liquid crystal, electroplating,
challenges: large area nanoporosity, transparent conducting contact, electrolyte with good ionic conductivity and poor electronic conductivity, stable under UV; assembly and large scale manufacturing;
cathodic coloration:
anodic coloration:
The coloration mechanism: \ce{MO6} octahedrons lead to $e_g$ and $t_{2g}$ level and ion channelling.
ref54,60,65,66,200,209,


\ce{WO3} as cathodic and either polyaniline(PANI) or Prussian white (PW) as anodic electrochromic half cells. \cite{Heckner2002}

Characterization of ECD includes transmission measurement and associated EC calculation, charge-discharge time, current-time curve and the fitting of obtained data.

\begin{quote}
a viable electrochromic smart window must exhibit a cycling life time \textgreater $10^5$ cycles, corresponding to an operation life at 10 - 20 years.
\end{quote}


\citeauthor{Sella1998} studied the optical and structural properties of RF sputtered thin film of \ce{WO3} and \ce{VO2} for electrochromic devices. Ionic conductor was built using transparent polymer electrolyte, which was prepared from a solution of 1M \ce{LiClO4} in propylene carbonate which was mixed with methylmetharcylate (MMA). The main characteristics of polymer electrolyte were: viscosity at 25 \si{\degreeCelsius} $\approx$ 12920 Pa.s, conductivity $\approx 10^{-2}-10^{-4}$\si{\per\ohm\per cm},non-hygroscopic if PMMA concentration \textgreater 30\%. A specific counter-electrode layer was not used since the encapsulated polymer electrolyte processes a very high ion storage capacity.\cite{Sella1998}
    
\chapter{intro}

Concentration of charge carrier $n$ is related to gate voltage $V_g$ by:
\[
n = \frac{\epsilon_0 \epsilon V_g}{ed}
\]
where $\epsilon_r = \epsilon_0 \epsilon$ is dielectric constant of gate materials.

intrinsic silicon equilibrium charge carrier concentration at RT is $n_i = p_i = 1.5 \times 10^{10} cm^{-3}$, much smaller than silicon atoms density as $5\times 10 ^{22} cm^{-3}$.

The average distance between dopant atoms is cubed root of concentration, $d = (10^{18} cm^{-3})^{-1/3} = 10nm$.

The electron mobility $\mu_n = 1500 cm^2/V\cdot sec $ at RT for Si, and hole mobility $\mu_p = 450 cm^2/V\cdot sec$ at RT.

for p-type silicon, when the conductivity $\sigma = 1 (ohm cm )^{-1}$, the doping level is
$N_A = \frac{\sigma}{q \mu_p}= 1 / (1.6E-19 \times 450) = 1.4E16 cm^{-3}$.

Built-in voltage $V_0 = \frac{kT}{q}ln(N_A N_D/n_i^2)$, depletion region width $W = \sqrt{\frac{2 \epsilon_{Si} V_0}{q}(1/N_A + 1/N_D)}$, where $\epsilon_{Si} = 11.7 \epsilon_0$. When applying external field, depletion width $W = \sqrt{\frac{2 \epsilon_{Si} (V_0 - V) }{q}(1/N_A + 1/N_D)}$

The capacitance of p-n junction is $C = A \sqrt{\frac{q \epsilon_{Si}}{2(V_0 -V)}(N_D N_A/(N_A + N_D))}$.



\iffalse
Materials that human can make define the age they live in. From Stone Age to Bronze Age and Iron Age, people evolve as mastering more and more sophisticated techniques of manipulating metals, such as alloying and annealing. Obtaining extreme high purity of silicon brings us into Information Age. Future is difficult to predict. But nanotechnology is one direction that we can not ignore. According to \gls{nni}, \gls{nanotechnology}. This definition alludes that dimension comes before compositions. It is often related to the quantum confinement or surface area in nanomaterials, which we will later revisit with specific scenario.

There are three states of matter under usual conditions: solid,liquid and gas. Solids materials could be further categorized into five groups: metals, ceramics, polymers, semiconductors, and composites.\cite{William2009} This classification is based on both composition and mechanical, electrical, and thermal properties as well as the associated functionality(i.e., \gls{ceramics} are typically hard yet brittle, insulating to electricity and resistant to heating).

\fi

\iffalse

We have synthesized \gls{tmo} and \gls{tmdc} at nanoscale, measured their crystalline structures and optical properties and demonstrated some devices assembled using as-synthesized nanomaterials. We aim to illustrate that by nanoengineering these \gls{tmo} and \gls{tmdc}, enhanced performances over their bulk states could be expected and new properties will arise. In the remaining sections of this chapter, we will discuss some general perspectives of nanomaterials, the growth apparatus and characterization methods that apply to all experiments done in this work. Then chapter 2 will focus on growth of \ce{WO3} and its derivative. We employed thermal \gls{cvd} to synthesize \ce{WO3} \gls{nw}, and we investigated the role of impurity in tungsten metallic powders, during which we observed a new state of sodium tungsten oxides: \ce{Na5W14O44} nanowires. We also found a method to potentially obtain large yield of \ce{WO3} \gls{nw}. Chapter 3 will concentrate on \ce{MoO3}. We explored two different growth mechanism of \ce{MoO3}:\gls{vs} and \gls{vls}. We discovered that alkaline oxides can be used as catalyst to grow two distinct \ce{MoO3} morphologies: nanobelts and towers. We further demonstrated the application of as-synthesized \ce{MoO3} nanomaterials in electrochromic devices.  In chapter 4 we discuss  We synthesized  and inspected the growth of \gls{fl} \ce{WS2}. Chapter 5 will conclude with an overall summary.

\fi
\section{crystal growth}

Nucleation is a process of generating a new phase from a metastable old phase, where the Gibbs energy per molecule of the bulk of the emerging new phase is less than that of the old phase.

General CVD knowledge, substrate preparation, and\cite{MichealK.Zuraw2003}

\section{band and bond}

Solid and orderliness.

Two theories arise to describe the outer shell electrons and to correlate the structure and physical properties: \gls{cft} and band theory.\cite{Goodenough1971} \gls{cft} assumes weak interaction between neighboring atoms and localization of electron towards parent atom, whereas band theory assumes that electron is shared equally by all nuclei and therefore a many-electron problems follows. Description of a single electron in periodic potential fail to treat the electron correlations adequately, as the interaction between atoms becomes weaker.For transition metals, $s$ and $p$ electrons are well described by a collective-electron model, while the 4f or 5f electrons are tightly bound to nuclei and screened from the neighboring atoms by 5s, 5p or 6s, 6p core electrons, hence it matches well with a localized-electron model. d electrons show intermediate character.


\chapter{experimental}

\section{diffuse reflection}

The KM two-flux model has been extensively used to describe the optical properties of inhomogeneous materials that consist of scattering and absorbing particles in a matrix.\cite{Vargas1997} There are some limitation in the assumptions made in original KM model, e.g., the distribution of scattering on the optical path. And revised KM models have been proposed to eliminate these limitations.\cite{Yang2004}  

\citeauthor{Morandi2005} studied the absorbance and diffuse reflection of \ce{MoO3} and \ce{WO3} thin films using FT-IR and UV-Vis spectroscopy. The observed absorption in visible and middle IR region were attributed to oxygen defects induced donor levels.\cite{Morandi2005}

The ratio between absorption and scattering, not the absolute value of each quality, determines the reflection features.


\iffalse

hemispherical reflectance, 
diffusive reflectance is a mathematical artifice without direct physical meaning. 

p332-338

the reflecting power associated with a specular-type process is often termed reflectivity, while reflectance is the analogous term for diffusely reflected radiation. This two types of reflection are present simultaneously. 

Cauchy formula, empirical equation which has subsequently been placed on a firm theoretical basis. 

intensity for a complex quantities is obtained by multiplying the complex conjugate. 

For strongly absorbing materials, the reflected light is complementary to the color of light transmitted by it, such as gold transmission is green; 

For the weakly absorbing substances of rough surface, it will appear the same color by transmission or reflection. 

penetration depth is wavelength dependent $d = \frac{In2 \lambda}{2\pi \sqrt{\sin^2\theta_i - n^2}}$, longer wavelength penetrated further than did shorter wavelength. attenuated total reflectance, 

thin film, amorphous, mean free path of electron is diminished, and porous in structure. The maxwell-garnett theory provides a satisfactory explanation, where free electron density is reduced. 

diffuse reflection: $I = I_0 \exp(-\epsilon d)$, Lambert cosine law formulated. multiple scattering by individual particles. 
$R_tot = \alpha R_sp + (1- \alpha) R_diff $, 

multiple scattering of diffuse radiation in a medium composed of closely packed particles. 

\[
\log{f(r_\infty) = \log{k} - \log{s}}
\]

It is concluded that the KM function can be used to obtain the characteristic color curve of a given sample only in the limit of small total absorption. 

by mixing the sample with an abundance of the reflectance standard, the regular reflection is extensively eliminated. 

The color of an object depends on the spectral composition of the source. The color of many light sources can be specified in terms of the temperature to which a blackbody radiator must be heated in order to achieve a color match for the source, which is known as color temperature. 

integrating sphere: the intensity at any part of the sphere, due to the reflected light is a measure of the total flux from a particular source, independent of the spatial distribution or location of the source on the sphere. 

substitution method, comparison method, 

\fi
\section{tem text}


\iffalse
An eucentric specimen stage is used, thus observation area remain fixed when tilting the specimen. The pressure in SEM chamber is on the order of $10^{-3} \sim 10^{-4}$ Pa, which is usually maintained by a diffusion pump, or turbo molecular pump when oil-free operation is needed. For a field emission electron source, a sputter ion pump becomes necessary due to the high vacuum requirement. 

The SEM instrument used in this study is JEOL JSM-6480 and EDX attachment from Oxford Instrument INCA. Typical observation conditions are listed as following:

\begin{enumerate}
\item SEM
\begin{itemize}

\item Acceleration voltage: 10 kV
\item Working distance: 10 mm
\item Scanning time: 80 s
\end{itemize}
\item EDX
\begin{itemize}

\item Acceleration voltage: 20 kV
\item Working distance: 10 mm
\item Dead time: $20\sim30$\%
\end{itemize}
\end{enumerate}


\begin{quotation}
Since the intensity of characteristic X-rays is proportional to the concentration of the corresponding element, quantitative analysis
can be performed. In actual experiment, a standard specimen containing elements with known concentrations is used. The
concentration of a certain element in an unknown specimen can be obtained by comparing the X-ray intensities of the certain element
between the standard specimen and unknown specimen. However, X-rays generated in the specimen may be absorbed in
this specimen or excite the X-rays from other elements before they are emitted in vacuum. Thus, quantitative correction is needed.
In the present EDS and WDS, correction calculation is easily made; however, a prerequisite is required for this correction. That is,
elemental distribution in an X-ray generation area is uniform, the specimen surface is flat, and the electron probe enters perpendicular to the specimen. Actually, many specimens observed with the SEM do not satisfy this prerequisite; therefore, it should be noted that a quantitative analysis result might have appreciable errors.
\end{quotation}


\fi


\iffalse
Similar to the photon-lattice interaction in XRD, the process of TEM could be understood as electron scattering events by the same crystal plane. 

\begin{itemize}
\item X-ray: characteristic X-ray for elemental analysis, Bremsstrahlung X-ray also useful for biological sample; $K_\alpha$ line from L to K transition, and $K_\beta$ from M to K transition, $L_\alpha$ line from M to L transition, Inelastic cross section, Bethe expression, X-ray energies are not identical to the ionized energy because after first emission, the atom is not in ground state until a free electron fill the last hole in the outermost shell. A cascade of transitions, Coster-Kroning transition, X-ray line shift slightly due to the chemical bonding to another atom. XEDS is not good at analyzing light elements due to the low fluorescence yield, which is strongly dependent on Z. Bremsstrahlung X-ray emission is strongly forward, 
\item SE: ejected from the conduction or valence bands, weak so only escaping if near the surface, STEM, complex cross section mechanism, 
\item Auger: Auger electron has specific energy similar to X-Ray, but is much more strongly absorbed than X-ray. Stated another way, Auger electron is hard to escape, so it is surface sensitive. 
\item CL: spatial resolution around 100 nm, 
\item collective interaction, plasmon and phonon, plasmon excitation cross section in Lorenztian form $\frac{d\sigma_\theta}{d\Omega} = \frac{1}{2\pi a_0} \frac{\theta_E}{\theta^2 + \theta_E^2}$, where $\theta_E = E_p/2E_0$, 
\end{itemize}


The diameter of E-beam in TEM is less than 5 nm in general, and can be $< 0.1$ nm at best.

correction of spherical aberration ($C_s$) and chromatic aberration ($C_c$), $C_s$ is done by , $C_c$ by energy-filtering, which is more useful for thicker specimens. $C_s$ correction permits the generation of smaller electron probes with higher currents, which significantly improves both analytical spatial resolution and sensitivity. $C_c$ correction offer the possibility to form band-gap imaging and chemical-bond imaging. The limiting apertures increase the depth of field for specimen, and the depth of focus for the image.\cite{Williams2009} 

including controlling the interactions of electron with magnetic fields and with specimen.  

XRD: X-ray scattered by electrons, electron scattered by both electrons and nuclei. Fresnel vs. Fraunhofer, high-angle scattered electron are incoherent; therefore, it can be used to form high-resolution Z-contrast image of a crystalline specimen, regardless of the orientation. Auger electron spectroscopy. EELS and XEDS constitute analytical electron microscopy (AEM). close approach the single-atom level.  energy-loss electrons cause Kikuchi lines to arise in DPs. Ionized atom enters excited state, 

penetrate electron cloud, spherical wavelets, the cross section for electrons elastically scattered into angles larger than $\theta$ is $\sigma_{nucleus}= 1.62\times10^{-24} (\frac{Z}{E_0})^2\cot^2\frac{\theta}{2}$; scattering factor $f(\theta)$ for low angle ($< \sim 3^{\circ}$). 

Point-group and space-group determination from convergent-beam patterns. crystal symmetry analysis. e-beam wavelength in metal. The high-resolution comes at the cost of poor sampling.  Human eyes and brain understand reflected light image, and not well-trained for the transmission images.

One must be just as aware of the instrument's limitations as one is of its advantages.   TEM is initially developed to overcome the image resolution imposed by light microscopes. constitute, draw analogies, resolving power, E-beam is one type of ionizing radiation, which is capable of removing the tightly bound, inner-shell electrons from attractive field of nucleus (visible light, is non-ionizing radiation to some extent). a wide range of secondary signal can be produced, the spectra exhibits characteristic peaks, which identify the elements present in the specimen. In analog to laser as a highly coherent source, . 

Both wave and particle approach, non-scattering is invisible, backscattered in large angle and secondary electrons are of interest in SEM, where they provide Z contrast and surface-sensitive, topographical images. forward scattered is of interest in TEM.  scattering events as billiard balls colliding, coherently scattered are those that remain in step, and incoherently scattered electrons have random phase relationship. Assuming single scattering events in TEM, 

The cost of TEM adds up to \$10 per eV. seize the public's imaginations. 

The cross section of tungsten, moly is 

Focused ion beam (FIB) to prepare thin foils of individual gates from one of the many millions of such on a wafer. The events of electron passing through one crystal plane, The coherent length of e-beam, collection angle, 

\fi


\iffalse
incoherent illumination $s = \frac{1.22 \lambda_{vac}}{2n\sin i}$, 
coherent illumination in microscope $s = \frac{\lambda_{vac}}{n\sin i}$,, 
phase contrast: transforming phase change in object plane into amplitude variation in image plane. 
much better than needed, human eye resolving power 0.3 mrad, 

The illumination can be considered as incoherent (adding intensity) if the object is self-luminous, or if illuminated from all direction. 

The opposite occurs in TEM probably, where the specimen is irradiated with complete transverse coherence across its area. back 
focal plane displays the Fourier transform of complex amplitude in object plane. The optical system is a spatial low-pass filter: it builds up images from only the low Fourier components present at the object, having higher values cutoff. 

The information about an object is on display in the objective lens's back focal plane in a Fourier-transform form. dark field, removing zero order; Schlieren technique, removing half of diffraction from one side of back focal plane; applications: apodizing in telescope, satellite transmitting aerial structure, 

with the aid of Fig. 
conjugate plane, 

electron lens can be made of electrostatic field, and magnetic field, and the B field can be generated from ferromagnetic materials (soft iron, 2 T) or superconducting (100 T). 

$C_s$ has dimensions of length, and is approximately equal to the focal length of objective lens (1-3 mm in TEM). 

$r_{min} \approx 0.91(C_s \lambda^3)^{1/4}$ is about 0.3 nm typically; with $C_s$ correction, $r_{min}$ can be further reduced to 0.07 nm. And human eye can resolve a distance of 0.2 mm, therefore the maximum useful magnification is about $0.2 \times 10^{-3}/0.3 \times 10^{-9} \sim 670 K$. 

And after interacting with specimen, the energy spread of transmitted E-beam becomes about 20 eV, which will limit the resolution more than does the spherical aberration. 

\textbf{coherence of E-beam}

faint points, large disk, the feature of interest is what make the materials imperfect, 

20 keV E-beam in TV. 

In XRD, both diffraction peak position and intensity are used; whereas in TEM, most time only the positions of spots are of concern. 

physical process of diffraction where the atomic planes appear to behave as mirrors for incident E-beam; 

\fi

\section{raman}


\iffalse
A Raman pattern database can be found at \url{http://wwwobs.univ-bpclermont.fr/sfmc/ramandb2/index.html}. 
In analytical practice, frequency is expressed in reciprocal wavelength (as cm−1), called wavenumbers;
\fi


\chapter{wo3}
metal oxide contact on work function and band structure.\cite{Greiner2013}

\citeauthor{Wang2009a} mentioned that amorphous \ce{WO3} can only be used in lithium-based electrolytes due to its in-compact structure and high dissolution rate in acidic electrolyte solutions. Electrochromic materials that can endure acidic electrolytes without degradation should be developed. Crystalline \ce{WO3} nanostructures with their much denser structures and small particle sizes are promising to be used as suitable electrochromic material in acidic electrolytes.

\subsection{polarons}

The concept of polaron was first proposed by Landau in 1933. In ionic or highly polar crystals, such as II-VI semiconductors, alkali halides and transition metal oxides, the Coulomb interaction between a conduction electron and the lattice ions results in a strong electron-phonon coupling. A new quasi-particle, virtual phonon, can be defined corresponding to the effect of electron pulling nearby positive ions towards it and pushing nearby negative ions away. The electron and its virtual phonons, taken together, can be treated as a new composite particle, called an electron polaron; the hole polaron is defined analogously. \cite{Devreese1996}


the electrical conductivity is given by $\sigma = n e \mu$, where $n$ is the density of free carriers, $\mu$ is their mobility and $e$ is electronic charge. The mobility is given by $\mu = e\tau/m^*$ with $\tau$ is carrier resistivity relaxation time and $m^*$ is the carrier effective mass.

\section{ECD}

Characterization of ECD (work like a thin-film batteries) includes transmission measurement and associated EC calculation, charge-discharge time, current-time curve and the fitting of obtained data.

The coloration efficiency (CE) represents the change in the optical density (OD) per unit charge density ($Q/A$, in units of \si{\cm^2\per\coulomb}) during switching and can be calculated according to the formula:
\begin{equation}
CE = \frac{\Delta~OD}{(Q/A)} [cm^2/C],
\end{equation}
where OD = $log(T_{bleach}/T_{color})$. The EC and optical density depend on the wavelength and are usually higher in the near IR than in the visible region.
Using Ohm's law($U_s = IR = RQ/t_s$) with switch voltage $U_s$, resistance R and surface area A, switching time $t_s$ could be estimated as
\begin{equation}
t_s = \Delta~OD\cdot A \cdot R /(CE\cdot U_s).
\end{equation}


The degree of \ce{H^+} intercalation, $x$, is determined by integrating cathodic current, and given by
\begin{equation}
x = \frac{Q}{F}\frac{M_\ce{WO3}}{m_f}
\end{equation}
where $m_f$ is the mass of film, $M_\ce{WO3}$ is the molar mass of \ce{WO3*1/3H2O}, and F is Faraday's constant.
The color center density, $c$, is then obtained using formula
\begin{equation}
c= \frac{x M_\ce{WO3}N_A}{\rho}
\end{equation}
where $N_A$ is Avogadro number, $\rho$ is density of film.


battery and ECD.\cite{Granqvist2012} electrolyte: PVB (poly vinyl buteral).
alternative materials and design: organic, Prussian Blue as EC materials, metal hydrides, suspended particle device, liquid crystal, electroplating,
challenges: large area nanoporosity, transparent conducting contact, electrolyte with good ionic conductivity and poor electronic conductivity, stable under UV; assembly and large scale manufacturing;
cathodic coloration:
anodic coloration:
The coloration mechanism: \ce{MO6} octahedrons lead to $e_g$ and $t_{2g}$ level and ion channelling.
ref54,60,65,66,200,209,


\ce{WO3} as cathodic and either polyaniline(PANI) or Prussian white (PW) as anodic electrochromic half cells. \cite{Heckner2002}


\begin{table}[htb]
\caption{Combinations of ECD configuration}\label{tb:ecd}
\begin{tabular}{lcccr}
\toprule
TC(both side) & electrochromic & ion conductor & counter electrode  & reference\\
\midrule
ITO &  \ce{WO3} & \ce{H^+\hyphen} polymer & PANI &\citeauthor{Heckner2002}\\
FTO &  \ce{WO3} & \ce{K^+\hyphen} polymer & PW &\cite{Heckner2002}\\
ITO & \ce{WO3} NWs & \ce{LiClO4\hyphen}PC & none & author design \\
\ce{Na_xWO3} NWs &\ce{WO3} NWs & \ce{LiClO4\hyphen}PC & none & author design\\
\bottomrule
\end{tabular}
\end{table}

\begin{table}[htb]
\centering
\caption{Comparison of MoOx ECD}\label{tab:moxecd}
\begin{tabular}{lcccr}
\toprule
$\lambda$ & $\Delta T$ & $t_c$ & $t_b$ & $CE$  \\
         (nm) & (\%)    & (s) & (s) & ($cm^2/C$)  \\
\midrule
  \\
\bottomrule
\end{tabular}
\end{table}



\begin{quote}
a viable electrochromic smart window must exhibit a cycling life time \textgreater $10^5$ cycles, corresponding to an operation life at 10 -- 20 years.
\end{quote}


\citeauthor{Sella1998} studied the optical and structural properties of RF sputtered thin film of \ce{WO3} and \ce{VO2} for electrochromic devices. Ionic conductor was built using transparent polymer electrolyte, which was prepared from a solution of 1 M \ce{LiClO4} in propylene carbonate which was mixed with methylmetharcylate (MMA). The main characteristics of polymer electrolyte were: viscosity at 25 \si{\degreeCelsius} $\approx$ 12920 Pa.s, conductivity $\approx 10^{-2}-10^{-4}$ \si{\per\ohm\per cm},non-hygroscopic if PMMA concentration \textgreater 30\%. A specific counter-electrode layer was not used since the encapsulated polymer electrolyte processes a very high ion storage capacity.\cite{Sella1998}

The device proposed was reproduced as shown in Fig.~\ref{fig:Sella1998ECD}
\begin{figure}[htb]
    \centering
    \includegraphics[angle=270,width=0.8\textwidth]{Sella1998ECD}
    \caption{citation, see original captions} \label{fig:Sella1998ECD}
\end{figure}


\chapter{moo3}
\ce{MoO3}, an alternative interpretation in terms of tetrahedral coordination of Mo atoms is also proposed. This is caused by the fact that four of the six surrounding O atom are at distances from 1.67 \si{\angstrom} to 1.95 \si{\angstrom}, while the remaining two are as far as 2.25 and 2.33 \si{\angstrom}. This also stress that the \ce{MOO6} octahedra are rather distorted.


% Melting points 
\begin{table}[htb]
\centering
\renewcommand*{\thetable}{S\arabic{table}}
\caption{physical constants of reactants }\label{tb:thermo}
\begin{tabular}{lccr}
\toprule
Material & MP(\si{\degreeCelsius}) & BP(\si{\degreeCelsius}) & reference\\
\midrule
\ce{NaOH}        & 318 & 1388 & handbook  \\
\ce{NaI}        & 651 & 1300 & MSDS    \\
\ce{KI}        & 681 & 1330 & MSDS   \\
\ce{Na2CO3}        & 851 & Not determined & MSDS    \\
\ce{Na2MoO4}        & 687 & Not available & handbook   \\
\ce{MoO3}    & 795 & 1155 & MSDS   \\
\ce{MoO2}    & 1100(decomp) & Not available & MSDS   \\
\bottomrule
\end{tabular}
\end{table}


In second state, the one with smaller droplet probably would exhibit faster growth rate in $\langle001\rangle$ direction due to its shortest lattice constant, while the other one endures more growth along$\langle010\rangle$ direction. We suspect that the liquid size that induce the tower growth is much larger than that of belts. From the final morphology of as-synthesized sample, we find that the above proposed two growth approaches could change into each other, that is belt growth could be initiated on the top of tower growth and vice visa. We assume this phenomena arise from the interaction between vapor supply, temperature and the shape of liquid catalyst. Or it may just come from the vapor transport of \ce{NaxMoO3}.
It had been shown by actively engineering the shape of catalysis, the growth direction of  InP NW could be switched between [111] and [100].\cite{Wang2013c} The same mechanism might exist in our case as well. For instance, when a belt is sufficient long and it could extend into the low temperature part, where the supply of MoOx vapor is reduced due to the consumption, and the shape of liquid might vary as well. Both will result in different absorption and subsequent diffusion rate along and inside the liquid catalyst. We also observed that belt growth could initiate from tower growth in reduced time growth. Actually this may be one possible mechanism of belts formation.

(The presence of \ce{Na6MoO33} phase is an indirect evidence for the formation of eutectic \ce{Na2MoO4}- \ce{MoO3} binary compounds at growth temperature. The bulk phase diagram may not accurately represent the phase transition occurring in catalyst droplet and solid interface. And it will usually lead to a significant temperature decrease of eutectic point from the bulk value. Nanoparticles do not completely melt and instead act as an active site for reactant absorption and diffusion, leading to a vapor-solid-solid growth mechanism. 

In vapor synthesis process, two growth mechanism exists: VS and VLS. VS process is widely accepted for the growth of \ce{MoO3}. Yet we caution that synthesis conditions should be scrutinized to determine the exact mechanism. \citeauthor{Li2002c} suggested a VS mechanism at 700 \si{\degreeCelsius} and VLS at 750 \si{\degreeCelsius} and higher.\cite{Li2002c} \citeauthor{Fibers2007} proposed a modified VS mechanism probably because the deposition occurs on \ce{Al2SiO5} with possible \ce{Al_{0.95}SiNa_{0.06}O_x} involved. Therefore temperature and possible impurity could potentially alter the growth mechanism. We divide the growth results into two categories: on Si substrate and on non-Si substrate, and describe them respectively. We also briefly mention using liquid exfoliation to prepare few layer \ce{MoO3}.

This rectangular shape implies the boundary plane along growth direction (long axis) is (001), in consistency with previous experimental reports\cite{Zeng1998,Li2002b} and theoretical studies.\cite{Firment1983,Cora1997} We also observed different shapes, i.e., elongated hexagonal using similar growth conditions. This is not an indication of different growth mode. It is a normal thermodynamic fluctuation. The growth rates along different crystalline direction of \ce{MoO3} are determined by the free surface energy. In fact, (201), (101) and (102) planes have all been observed as terminating planes.\cite{Zeng1998} The stacking rate of \ce{MoO6} octahedra along $a$ and $c$ axis could develop some other ratio. And the coexistence of different planes in one growth suggests the similarity of free surface energies between these surfaces. In other words, the migration barrier of adatoms on (010) plane is presumably much lower than that on other low index planes due to the Van der Waals interaction nature along [010] direction.


\citeauthor{Hardcastle1990} summarized an empirical formula to relate the Raman peaks and \ce{Mo-O} bonding lengths.\cite{Hardcastle1990} This correlation assumes general form as
\begin{equation}\label{eq:mobond}
\nu = A \exp{B\cdot R},
\end{equation}
where $A=32895$ and $B=-2.073$ are fitting parameters, R is bond distance in unit of \AA. Given a stretching frequency, the resolution for calculated bond distance is $\pm0.016$ \AA. Another empirical expression connect the bond valence $s$ and bond distance R: $s(M-O) \approx (R/X)^{-6} $, where X=1.882 when M is Mo, and 1.904 when M is W. The valence sum rule could be then used to check the state of Mo cation. It should be noticed that not all observed Raman lines could be correlated to a \ce{Mo-O} bond distance by extrapolation of Eq.~\ref{eq:mobond}. It is then regarded as a symmetry related vibrational mode, i.e., 820 \si{cm^{-1}} in \ce{MoO3}. From the correlation of various Mo compounds, a general conclusion is the lower the stretching frequency for the shortest metal-oxygen bond, the more regular is the structure.


\chapter{tmdc}


CNT chirality by TEM \cite{Zhang1993} TEM chirality of \ce{MoS2} NTs

\ce{WS2} NT transport \cite{Zhang2012c}
less grain boundary more mobility, 

\citeauthor{Ramasubramaniam2011} investigated the band gap tuning in bilayer TMDC materials by applying external $E$ field. Similar research has been done for graphene and bilayer boron nitride. Semiconductor-metal transition was suggested for \ce{MoS2} and \ce{WS2}, with difference on the CBM and VBM evolution. In \ce{MoS2}, the valence-band-splitting cause the A and B excitons in optical absorption measurement. Calculation shows that CB and VB are translated toward the Fermi level with increasing E field.  The external field localized charge along $c$ axis, but delocalized that within the plane normal towards $c$, thereby driving the semi-metal transition. It was mentioned that this transition is not anticipated in monolayer \ce{MoS2}. It was emphasized that precise band gaps might be different from the author’s results, yet the gap-tuning should be universal.\cite{Ramasubramaniam2011}

\cite{Song2013} \ce{WO3} by ALD, and sulfurized in Ar and \ce{H2S} (10:1) at 1000 C. \ce{WS2} layer No and peak intensities ratio under 633nm excitation is correlated. It was found the 2LA/A1g is less than 1 for 1L. In supporting info 532 nm Raman spectra, the 2LA/A1g is presumably larger than 1 for 1L. \ce{WS2} NT on Si NWs is also demonstrated.

\cite{Tenne2010} chemical modification of NTs. Functional ligand consists of an anchor group that attaches to the NTs surface and a tail which render them soluble in various solvents. PTAS functionalized BN nanotube lead to the formation of stable suspensions in aqueous solutions. The strong attachment is formed through $\pi-\pi$ interactions.

inorganic nanotubes review \cite{Tenne2004} , unsaturated bonds number increases as the size of MS2 sheet decreases.

Water splitting materials should process a band gap larger than 1.4 eV, considering both the NHE potential distance and practical application. The monolayer \ce{WS2} exhibits direct gap of 1.98 eV. Quantum confinement could push the gap separation farther away. \cite{wilcoxon1997} \citeauthor{Notley2013} use liquid exfoliation to prepare \ce{WS2} NPs.\cite{Notley2013} Non-ionic surfactant concentration is about 0.1\%w/w. Continuously adding surfactant during sonication improves the yield. Optimum surface tension is found at about 40 mJ/m$^2$.

G/\ce{WS2}/G stacked solar cell. For lubricant, and surface protection. Absorption $\sim 10^7 m^{-1}$. \cite{Britnell2013}

In Ref\cite{Zeng2013}, single crystal \ce{WS2} growth using \ce{I2} transport was described in supporting info.
\ce{C24H12K4O8}\footnote{http://www.chemspider.com/Chemical-Structure.24771386.html}



416 peak was prevously assumed to be a combination of LA and TA phonons at K points.
416 cm raman on WS2. B1u is pressure sensitive. \cite{Staiger2012} also studied pressure dependence.

A1g mode using resonance Raman excitation shows upshift, which is presumably caused by folding induced strains. And TEM SAED reveal 3R symmetry. We did not observed this upshift of A1g probably due to the large diameter and few layer involved.

Nanotube growth is relatively independent of substrates and FL layer is closely related to substrate. Film growth could provide some insight into the latter scenario.

\section{calculation}

Raman conditions: 0.3 mW, 532 nm, 200 sec acquisition time. photon flux = $0.3E-3\times6.242E+18/1.2398/0.532=2.84E15$

first demonstration of \ce{WS2} NT n-type FET. \cite{Levi2013}
the importance of contact, and avoiding moisture. 

The calculated carrier concentration is about $10^{19}cm^{-3}$, a highly doped semiconductor, possibibly arising from sulfur vacancy. 

\ce{WS2} absorption coefficient $10^{-7}m^{-1}$, mean free path of photo-excited charge carriers 1 $\mu m$. the wave vector of photon is considerably small than size of BZ, therefore The wave vector of phonon in Raman scattering usually close to zero.

Multiple phonon scattering, For two identical phonons, the corresponding Raman peak in the spectrum is called an overtone of the peak from the corresponding one-phonon process. And the wave vector conservation rule is automatically filled, therefore the phonon involved is not limited to BZ center anymore.
\[
I(G) \approx \sum_k \frac{\langle f|H_M|b\rangle \langle b|H_{ep}|a\rangle \langle a|H_M|i\rangle}{(E_p - E_k^{\pi *}- E_k^{\pi}-i\gamma)(E_p - E_k^{\pi *}- E_k^{\pi}-i\gamma- \hbar\Omega_{G})}
\]


\subsection{strain}

$E_{2g}$ mode is strain sensitive. 

\citeauthor{Ghorbani-Asl2013} studied the strain in tubular TMDC and found a linear dependence of Raman scattering on strain (3 \si{cm^{-1}} per percentage for $E_{2g}$mode).\cite{Ghorbani-Asl2013} 

For 2D materials, strain may be induced by elongation of an appropriate substrate, e.g. by uniform mechanical strain, or by using a material with high thermal expansion coefficient and varying the temperature. For TMD MWNT, tensile tests have been reported by various groups. However, to date, it is not perfectly clear whether inner and outer walls are stretched simultaneously, or rather the outer walls slide on the inner ones. The latter hypothesis would result in a broadening of the Raman signals, while the first one would leave the signal widths rather unaffected. In any case, there would be a shift of the Raman signals that can serve as precise scale for determining the strain.\cite{Ghorbani-Asl2013}


\citeauthor{Virsek2007} performed a Raman-TEM integrated study on multiwalled \ce{WS2} NT with diameter \textgreater 200 nm. The tubes were synthesized using chemical transport method. Up-shift of Raman is explained by strain in the walls. This shift is not observed in the specimen by sulfurization process of oxides. Applied hydrostatic pressure is isotropic,\cite{Staiger2012} while the strain is expected to anisotropic. Strain can also be relaxed by chirality.\cite{Virsek2007} 

strain effect by first-principles calculations. direct gap is only maintain in a narrow strain range (-1.3 -- 0.3 \%), \cite{Yun2012}.

Semiconducting to metallic transition in \ce{MoS2} at compressive strain of 15\% or tensile strain of 8\%; direct-to-indirect gap transition for 1L \ce{MoS2} at about 2\%. \cite{Scalise2012}



\chapter{to be used}

\cite{Matar2011} Using electronegativity $\chi$ and chemical hardness $\eta$ to assess electron affinity $E_a$, work function $W_f$, Fermi energy $E_f$ and band gap $E_g$.
\begin{align}
\chi &= 0.5(W_f + E_a)\\
\eta & = 0.5(W_f - E_a)
\end{align}
where I is ionization potential and $E_a$ is electron affinity.


Correlation between optical band gap and formation enthalpy; reaction occurs in order to form compounds with a larger gap.  $E_g = A \exp(0.34E_{\Delta H^0})$, and A adopts different values depending on the metal elements:
\begin{itemize}
\item A=0.8 for s and f block elements,
\item A = 1 for d block elements,
\item A = 1.35 for p block elements.
\end{itemize}

\section{MB}

MB is a heterocyclic aromatic dye which is blue colored in oxidizing environment. Upon reduction, MB is turned into colorless leuco MB. This can be used as an oxygen indicator in food industry. Photo-bleaching of MB can be also due to its leuco formation rather than total decomposition. Photocatalytic decomposition can be minimized by keeping the solution at acidic condition (pH = 4), which will limit the formation of oxidative hydroxyl radicals (E = 2.8 eV vs normal hydrogen electrode). Oxygen dissolved in the solution play a key role in conversion of LMB to MB under visible light. Purging with \ce{N2} for 20 min


\textbf{\ce{WS2}-\ce{WO3}}: 1 kW light source(Hg, or Xe lamp), photon flux, phenol (\ce{C6H5OH}, 94.1g/mol, MP 40C)concentration is 20 mg/L, hydroxyl group. The quantitative analysis of phenol was performed via a standard colorimetric method.\footnote{\url{http://omlc.ogi.edu/spectra/PhotochemCAD/html/072.html}}
\citeauthor{DiPaola1999} prepared \ce{WS2}-\ce{WO3} mixture in two methods, sulfurization of \ce{WO3} and oxidation of \ce{WS2},with the latter are more active.
\citeauthor{DiPaola1999} also concluded that the actual efficiency of mixed \ce{WS2}-\ce{WO3} catalysts depends on the ratio of each composition present of the surface of the particles, and the maximum of photoactivity is obtained with 40-50\% surface molar ratio of \ce{WS2}.

ref 25, 28 and 41.

\citeauthor{Sreedhara2013} studied the kinetics of photodegradation of methylene blue\footnote{\ce{C16H18N3SCl},319.8 g/mol, MP: 100C accompanied with decomposition \url{http://en.wikipedia.org/wiki/Methylene_blue}} dye by few layer \ce{MoO3}.
For the photodegradation method, it was stated that `` the samples were collected after the photoreaction had been centrifuged for 5 min to remove the photocatalyst before UV-Vis measurement.''


Raman \cite{Xiao2007}. Silver has the strongest SERS enhancement due to the larger imaginary part of the dielectric constant and higher thermal conductivity. Milli-Q grade water ((Milli-pore)\textgreater 18.2Mohm).

MB Raman peaks: 445, 1618, ref20. some peak splitting and shift observed on SERS, attributed to chemical adsorption. definition of Raman enhancement factor.

SERR MB on Ag. \cite{Nicolai2003}
MB: the absorption spectrum in VIS is used to infer about different adsorbed forms of MB. the formation of large aggregates. 

 can remove dissolved oxygen. \cite{Wang2014a}

solar energy harvesting representative study.\cite{Yoneyama1972} MB to LMB (\ce{C16H19N3S}) in aqueous solution upon illumination of \ce{TiO2}. The colorimetric analysis was performed in a glove box under nitrogen atmosphere. The absence of oxygen is important to prevent the oxidation of LMB to blue MB.
\[
\cee{MB^+ H2O + H^+ \rightarrow MBH3^{2+} + 1/2O2}
\]
where MB represents the uncharged center of MB molecule.

common wisdom expect that a dye incapable of injecting an electron at the excited state to CdS. MB, which process N-methyl groups in its molecular structure and does not sensitize CdS is an exemplary candidate. quantum efficiency is defined as probability of MB converted to azure B per incident photon. QE of CdS to MB decomposition is reduced in nitrogen bubbling treated solutions, indicating the necessity of oxygen. Two possible mechanisms: a) adsorbed oxygen acts as a trap for the conduction electron and prevent the accumulation of negative charge within space charge region of CdS, supported by the formation of \ce{O2^-} in excitation of CdS in aqueous suspension.\cite{Takizawa1978}

ref 16, MB aqueous solution stability. Liquid chromatogram, azure B (trimethylthionine), and thionine. Electrochemical measurement,

MB adsorption.  photocatalytic oxidation of MB by \ce{TiO2} film. photo-oxidation reaction occurs at the surface of photocatalyst. Mb molar extinction coefficient was found to be 66700 1/cm 1/M. Langmuir adsorption isotherm.\cite{Matthews1989}

\[
[MB]_{ads} = \frac{k_1 k_2 [MB]}{1 + k_1[MB]}
\]
and integrated form of Langmuir adsorption isotherm
\[
t = \frac{1}{k_1K} In\frac{[S]^0}{[S]} + \frac{1}{K}([S]^0 - [S])
\]
where $K = k_2 \phi N T_r$, with $\phi$ as quantum yield, N as total absorbed photons, and $T_r$ as rate of transport.
\[
\cee{C16H18N3SCl + 25.5O2 \rightarrow 16CO2 + 6H2O + 3HNO3 + H2SO4 +HCl}
\]
which indicates the total oxidation of $10 \mu M$ MB would exhaust the ambient oxygen concentration of initially air-equilibrated solutions (about $250 \mu M$ ). ref 28 Thus the transport of both oxygen and MB to the photocatalyst surface are anticipated to be key factors.

photoelectrochromism at \ce{TiO2}/MB interface and its control. Efficient capture of photogenerated holes by a reducing agent is crucial to the reversibility of bleach-recoloration transition. This transition is kinetically dictated by electron transfer. Holes transfer is not desired.\cite{DeTacconi1997}

256 nm band is associated to the presence of LMB. LMB formation is not favored at alkaline pH values in aqueous solution. The OH radicals are generated either with the surface hydroxyl groups on \ce{TiO2} or with water, and its high oxidizing power cause photocatalytic decomposition of the dye.

An elementary step in decomposition of MB is N-dealkylation, which is preceded by radical cation formation.\cite{Takizawa1978} This radical cation can be spectroscopically monitored by the presence of 520nm band for MB. In MB absorption spectrum, 664 and 614 nm band ratio is related to monomer and dimer relaxation.
\begin{align}
\cee{TiO2 &\rightarrow e_{CB}^- + h_{VB}^+ \\
h_{VB}^+ + red &\rightarrow ox\\
MB^+ + 2e_{CB}^ + H^+ &\rightarrow LMB}
\end{align}

Measure the ratio between 614 and 663 nm before and after adding WS2 can indicate the adsorption of monomer and dimer MB.

MB can act as sacrificial electron acceptor in the reduction to leuco form. The decomposition is favored under oxygen-rich environment. MB feature peaks at 663, 614 and 292 nm, and $\epsilon_{660}=10^5 M^{-1}cm^{-1}$. The doubly reduced form of MB, LMB has feature peak at 256 nm. The singly reduced form of MB, \ce{MB.^-} is pale yellow, with peak at 420nm.\cite{Mills1999}
\[
\cee{MB + e_{CB}^- ->[pH<7] MB.^-}
\cee{2MB.^- \rightarrow MB + LMB}
\cee{O2 + e_{CB}^- \rightarrow O2.^-}
\]

The oxidized form of MB, \ce{MB.^+} has peak at 520nm, which is stable in acidic solution, but decomposes irreversibly in slight alkaline solution(pH = 9).
thionine peaks at 600nm.
MB forms dimers in aqueous solution,
\ce{
2MB <=>[K_D] (MB)_2
}
A typical value of $K_D$ is 3970 1/M. A quadratic equation can be solved to obtain the monomer concentration:
\[
2K_D [MB]^2 + [MB] - [MB]_{total} = 0
\]
MB adsorption on metal oxides. Monomer size is less than 1.5nm.
Logarithmic acid dissociation constant $pK_a= -\log_10 \frac{[A^-][H^+]}{[HA]}$. The oxidation potential for \ce{H2O}-\ce{O2} couple is 1.23V and 0.817V versus NHE at pH 0 and pH 7, respectively.

%\begin{align}
%\cee{MB + SED &->[TiO2][h\nu \leq 3.2eV] LMB + SED^{2+}\\
%2LMB + O2 &\rightarrow 2MB + 2H2O}
%\end{align}


S.L. Murov, I. Carmichael, G.L. Hug, Handbook of Photochemistry, 2nd revised ed. Marcel Dekker, New York, 1993.

aerobic or anaerobic, dimerise, photominerlization, gas to liquid transfer.

Mb to LMB transition as visual time monitor. commercial colorimetric oxygen indicators. radical-bearing carbon with unpaired electrons. MB = \ce{MB^+Cl^-}.\cite{Galagan2008}

monomer MB and dimer MB kinetics.\cite{Spencer1979}



MB. \cite{Lee2003a}
\begin{align}
\cee{ 2LMB &->[\text{UV}] LMB^*\\
2LMB^* + O2 &\rightarrow 2MB^+ + 2OH^-}
\end{align}

\[
\cee{2LMB ->[\alpha] LMB^*}
\cee{2LMB ->[\text{above}] LMB^*}
\]


\section{literature to read}


\ce{WS2} 1L doping calculation. \cite{Ma2011}


electrochromic films. \cite{Yoshimura1985}
ECD \cite{Jiao2012} recent review \cite{Mortimer2011}
PEC, photoelectrode, WO3 and Si tandem structures.\cite{Coridan2013}
WO3 photoactivity MB. \cite{Watcharenwong2008}

2D wo3.\cite{Kalantar-zadeh2010a} 
WO3 plasmon \cite{Manthiram2012}

\ce{WO3} indirect gap 2.6eV, direct gap 3.4eV. \cite{Koffyberg1979}

Raman fingerprint of m-\ce{WO3}, h-\ce{WO3} and \ce{WO3.nH2O} were summarized in ref\cite{Daniel1987}.

\ce{WO3} on FTO by flame synthesis.\cite{Rao2014} \cite{Xu2006}

Seeded \ce{W_{18}O_{49}} NWs growth on W foil.\cite{Hong2006a}

\ce{Na2W4O_{13}} growth and optical properties. \cite{Oishi2001} \cite{Itoh2001}

\ce{Na2W4O_{13}} crystal phase \cite{Viswanathan1974}

\citeauthor{Salje1984} studied the transport in \ce{WO_{3-x}} ($0\leq x \leq 0.28$).\cite{Salje1984} It was found \ce{WO_{3-x}} show metallic conductivity when $x > 0.1$.

\ce{WO_{3-x}} \cite{Migas2010}

\ce{WO3} high temperature phase. \cite{Vogt1999}
tungsten bronzes \cite{Wiseman1976}

Phase transformation of \ce{Na2MoO4} and \ce{Na2WO4} by Raman scattering. \cite{Lima2011}

\ce{WO2} NWs synthesis and raman \cite{Ma2005}.

\ce{WO_{3-x}} CS planes and conductivity.\cite{Sahle1983}

\ce{W-O} equilibrium diagram \cite{Wriedt1989}

\ce{W_{18}O_{49}} electrochromic devices.\cite{Liu2013d} should compare with this one \cite{Wang2008}

nucleation catalysis \cite{Turnbull1952}

\ce{WO3} NWs aggregates. \cite{Kozan2008a}

optical properties of \ce{WO3} gaps\cite{Saygin-Hinczewski2008}

\ce{WO3} atomic layer by exfoliation and annealing \ce{WO3.H2O}. \cite{Kalantar-zadeh2010a}

sodium tungstates raman \cite{Redkin2010}

charge density wave in K-doped \ce{WO3} \cite{Raj2008}

\ce{W_{18}O_{49}} Raman, IR shielding.\cite{Guo2012} \cite{Guo2011}
broad peak between 750-780 cm-1.

\ce{WnO_{3n-1}} NPs. \cite{Frey2001}

\ce{WO3} growth hydrothermal.\cite{Moshofsky2012}

\ce{W_{18}O_{49}} on tungsten foil by thermal growth\cite{VanHieu2012}

Cathodoluminescence \cite{Parish2007}

optical characterization of WOx film.\cite{Valyukh2010a}

E-beam penetration \cite{Kanaya2002}

optics in electron microscopy. \cite{GarciadeAbajo2010a}

Ge NW growth using Ga as catalyst. \cite{Chandrasekaran2006}

\ce{MoO3} photocatalytic \cite{Chithambararaj2013}
photocatalytic experimental setup.\cite{Hupka2006}
\ce{MoO3} pseudocapacitor  \cite{Brezesinski2010}

\ce{MoOx} few layer as hole selective contact in solar cell.\cite{Battaglia2014}
\ce{MoO_x} on n-type Si acts as a high work function metal (6.6eV), enabling a dopant-free contact and thus junction-less devices.
piranha clean of FTO. 50ms switch.\cite{Scherer2012} 

hydrogen absorption in \ce{MoO3}.\cite{Sha2009}

\ce{Na6Mo_{11}O_{36}} phase. \cite{Bramnik2004}

\ce{Na6Mo_{10}O_{33}} phase, \cite{Gatehouse1983}

\ce{MoO3} thin film. \cite{Carcia1987}

\ce{H_xMoO3} raman.\cite{Hirata1996}

MoO3 spreading \cite{Leyrer1990}

Na2Mo2O7, Na2Mo4O13 phase transition \cite{SinghMudher2005}\cite{Tangri1992}

visibility of FL \cite{Benameur2011}

exfoliation IPA \cite{Halim2013}  \cite{Zhou2011a}

\ce{MoO3} good style. \cite{Siciliano2009} \cite{Abdellaoui1997}

\ce{MoO3}  DFT study \cite{B511044K} \cite{Cora1997} \cite{Sayede2005}

\ce{MoO3} raman \cite{Lee2002}



hydrogen evolution catalysts. \cite{Merki2011}

Raman substrate dependence \cite{Buscema2013}

stability of TMS NTs \cite{Seifert2002}

2H and 1T in \ce{MoS2} \cite{Eda2012}

\ce{MoS2} FET statistical study. \cite{Liu2013i}

water splitting review. \cite{B800489G}

\ce{WS2} theory and experimental combined study. \cite{Klein2001}

2D review on oxides \cite{Osada2012}

Pb catalyzed \ce{MoS2} nanotube \cite{Brontvein2012}

\ce{WS2} Raman.\cite{Zhao2013} \cite{Sekine1980}

phonon dispersion $E_{2g}^1(M)$? \cite{Ataca2012}

MoS2 optical properties.\cite{Search1979}

FL heterostructure. \cite{Yu2013a}

\cite{Kang2013} TMDC alloy DFT.

thermoelectric TMDC \cite{Wickramaratne2014}

\ce{CH4N2S} thiourea + \ce{WOx} to \ce{WS2} \cite{Leonard-Deepak2011}

\ce{WS2} by \ce{WCl_n} and \ce{H2S}, raman (632 nm) show bulk features\cite{Tenne2008}.

direct gap of ML at corner of BZ, point K.

TMO review.\cite{Goodenough2013}

h-\ce{MoO3} \cite{Lunk2010} \cite{Zheng2009}

\ce{MoO3} (010) surface defect. \cite{Chen2001}

mass spectrometry data to extract vapor pressure of \ce{NaxMoO3}.
strain and Raman theoretical analysis.\cite{Chang2013a} 

magnetic properties of ws2.\cite{Zhang2013j} 


\section{literature read}


To develop large-size single-crystal graphene on dielectric substrates. small carbon flow near-equilibrium CVD process. Grain size about 10 microns, precursor \ce{CH4} and \ce{H2} (ratio 2.3:50) at 1180 C. \ce{SiO2}-Si surface roughness. Although the growth substrates (quartz,\ce{SiO2}-Si and \ce{Si3N4}-\ce{SiO2}-Si ) have a complicated stereo network similar to diamond, regular hexagonal G growth is obtained, which indicates the deposition is determined by equilibrium kinetics, and this should be applicable to other 2D materials as well. I2D/IG exceeds two on \ce{SiO2}-Si subs (514.5nm), indicating monolayer G. armchair (AC) G edge grows faster than zigzag (ZZ) edge.\cite{Chen2013j}

catalytic graphitization of solid carbon sources. catalytic transformation, the source is in solid state, low temperature (less than 600C), 2nm  \ce{Al2O3} by ALD as carbon diffusion barrier. amorphous silicon (a-Si), Ni lower the activation barrier ,  tetrahedral amorphous carbon (ta-C).\cite{Weatherup2013}

low energy (50eV) ion implantation doping in G. Ions penetrate pristine G at energy larger than 100eV. Individual substitutional incorporation of B into G lattice is demonstrated. 1\% doping level was obtained. \cite{Bangert2013}

\ce{CaF2} a material suitable for scattering efficiency S comparison measurement due to its large band gap ($S\times \omega_L^4$ is constant below 5eV).

heterojunction is employed to transferred photo-generated carriers. Schottky barrier conduction band electron trapping and consequent longer electron-hole pair lifetimes. Numerous studies have suggested that fine particles of transition metals or their oxides, when dispersed on the surface of a photocatalyst matrix, can act as electron traps on n-type semiconductors.\cite{Zhou2010} 

\citeauthor{Cao2014} studied the layer-dependence \ce{MoS2} electrocatalysis and propose the vertical hopping efficiency of electrons instead of the edge site numbers is a key factor for catalytic reaction.\cite{Cao2014} ref19,20

It was found that the critical step in this process is the fast conversion of the oxide nanoparticle surface into a closed monolayer of \ce{WS2}. \ce{W18O49} as an intermediate phase is observed. XRD peaks shift to monitor strain.(002) peak of nanotube shifted to lower angles, the interlayer spacing increase by about 2\% as compared to the bulk powder, likely due to the build-in strain.\cite{ZAK2009} 
$\epsilon = (a - a_0)/a_0$ =(6.4-6.16)/6.16 = 3.8\%. tensile strain ($\epsilon > 0$)


DFT doped \ce{WO3} for photocatalytic reaction.\cite{Wang2012} CBM arises from W $5d$ states and splits into $t_{2g}$ and $e_g$ states under crystal field. VBM comes from O $2p$ states, including $2p_\sigma$ (along \ce{W-O} bonds) and $2p_\pi$ (normal to \ce{W-O} bonds).


oxygen vacancies in \ce{WO_{3-x}}.\cite{Wang2011b}  Coloration and electron conductivity changes. \citeauthor{Wang2011b} found strong dependence of WO3-x electronic properties on $V_O$ concentration and the the crystallographic direction on which O is removed. DFT band gap calculation is close to experimental value. Vacancy levels are found at 2.1eV.

The Raman spectra of \ce{WO_x} is rare because of the difficulty of preparing pure suboxides phase and the strong shielding of \ce{WS2}. Yet it does exhibit distinct Raman spectra. \cite{Tenne2005} The 870 line is attributed to \ce{W3O8}.\cite{Hardcastle1995}


\citeauthor{Huang2006} studied the \ce{W3On} cluster with n from 7 to 10.\cite{Huang2006} It was found \ce{W3O9} clusters possess a HOMO-LUMO gap about 3.4eV. This closeness to bulk value suggests \ce{W3O9} could be viewed as the smallest molecular unit for bulk \ce{WO3}.

plasmon dispersion in 2D materials, plasmon resonances in visible regions by doping induced free carrier density. 2D plasmonics, depolarization factors, partial reduction of Mo to a lower valence state. \cite{Alsaif2014a}

WS2 photoluminescence spectra of few layer and nanotube:
NT electrical structures depend on chirality, diameter and layer No as well as strain. Theoretical calculation indicates the SWNT with diameter larger than 4nm should approach the single layer limit.\cite{Ghorbani-Asl2013}

Other chalcogenide has also been synthesized using this one-end sealed layout.\cite{Mukherjee2013}

\citeauthor{Zou2007} prepared W/\ce{WS2} core-shell NPs by reaction of tungsten and sulfur under hydrogen atmosphere.\cite{Zou2007}

CVD 1L WS2 PL.\cite{Peimyoo2013} (of NTU Yu group) PL peak at 635nm, width 40 meV, 

CVD 1L WS2.\cite{Cong2013} (of NTU Yu group) 457 nm excitation, PL at 525nm and 630nm, 

\ce{MoS2} sing-walled nanotube.\cite{Xiao2014}

1T MoS2: metallic phase a negative temperature coefficient for conductivity, XRD pattern identified. \cite{Wypych1992}

stable 1T WS2 multiwalled NT by Re doping.\cite{Enyashin2011}. 2H to 1T transition formerly known only for WS2 and MoS2 intercalated by alkali metals. 3R transition to 2H upon heating since 2H is the most stable one.

1T \ce{MoS2} Raman.\cite{Yang1991} strong peaks at 156, 226, and 330 cm-1. M point frequencies measured by neutron scattering. M point is folded into BZ zone center due to the formation of superlattice.

Electrons and Phonons in Layered Crystal Structures, edited by T. J. Wieting (Reidel, Dordrecht, Holland, 1979).

\ce{WS2} p-type or n-type.  Fermi level at the surface of semiconductor is pinned to a fixed position relative to the CBM and VBM by a sufficient density of surface states situated between CBM and VBM. \cite{Baglio1983}

Electronic structure of \ce{MoS2}.\cite{Eknapakul2014} K intercalating into bulk to create quasi-standing 1L. Large effective mass 0.6 $m_e$ found, implying low mobility. Direct gap 1.88eV is measured.

Self-assembled monolayer (SAM) on \ce{SiO2} and its effect on \ce{MoS2} 1L.\cite{Najmaei2014}


\citeauthor{Shi2013} studied the strained monolayer \ce{MoS2} and \ce{WS2}. The results show that exciton binding energy is insensitive to the strain, while optical band gap becomes smaller as strain increases. Monolayer \ce{WS2} PL maximum located at about 1.95 eV. Calculation shows the electron effective mass of \ce{WS2} is the smallest, rendering higher mobility in device.\cite{Shi2013}

\citeauthor{Kosmider2013} studied the heterojunction between two monolayers of \ce{MoS2} and \ce{WS2}. Top of VB in W layer and bottom of CB in Mo layer, forming type II structure. bilayer gap 1.2 eV.\cite{Kosmider2013}

Band structure  of \ce{MoS2} in bulk form was calculated by \citeauthor{Mattheiss1973}.The calculation result is 1.2eV (indirect gap).\cite{Mattheiss1973}

Alkali metal intercalated \ce{WS2} film was prepared.\cite{Homyonfer1997} Stage 6 superlattice formation was suggested according to X-ray diffraction, and photoresponse spectra and electron tunneling measurement were done.

decrease in dielectric screening and thereby enhanced excitonic effect.
DFT is not good at describing photoemission, GW approximation overcome this deficiency but still not enough for photoabsorption process in which ehps are created. BSE equation is used to compensate this discrepancy, WX2 exhibits larger spin-orbit splitting as compared to MX2 family.\cite{Ramasubramaniam2012}

oxygen plasma treatment on HF-etched Si (001). reaction among $e$, \ce{O^+}, \ce{O2^+}, \ce{O^-},\ce{O2}. \ce{OH}-terminated surface obtained.\cite{Habib2010}
    \chapter{paper draft}
\section{wo3}

\subsection{used}

SG and FG experiments remain essentially the same as that of OT growth, except that in SG additional tungsten powders were distributed onto the receiving substrate, and in FG more than one piece of substrate was employed. The modification is schematically illustrated in Fig.~\ref{fig:wogrowsf}. More details will be provided when it comes to the discussion.
% sg fg
\begin{figure}[htb]
\centering
\includegraphics[width=0.5\textwidth]{sg_and_fg.jpg}
\caption[\ce{WO3} NW growth: SG and FG]{\ce{WO3} NW growth: SG and FG. (a) Seeded growth with additional powders on substrate (b) Flow growth with multiple substrates}
\label{fig:wogrowsf}
\end{figure}

With four kinds of W powders and three layouts, we designed the experimental matrix as illustrated in Table.~\ref{tab:matrix}. The symbol $\times$ means this combination is covered in this work, and NA means otherwise.
% Tungsten powders growth design
\begin{table}[htb]
\centering
\caption{Tungsten powders growth design}\label{tab:matrix}
\begin{tabular}{lccr}
\toprule
 & Ordinary Transport & Seeded growth & Flow growth \\
\midrule
3N   &  $\times$ & NA &  NA   \\
3N5  &  $\times$ & NA &  NA   \\
4N5  &  $\times$ & $\times$ & $\times$ \\
5N   &  $\times$ & $\times$ &  $\times$ \\
\bottomrule
\end{tabular}
\end{table}

We will first present the OT growth results in section~\ref{sec:nawox}, and discuss the other two in section~\ref{sec:sgfg}.

We found the growth using 3N source show distinctive features in comparison to the rest. This mainly arise from the higher sodium concentration in 3N source than that in others. Therefore we focus on 3N source first, and move on to other latter.


% Na5 raman fitting
\begin{figure}[htb]
\centering
\includegraphics[width=0.6\textwidth]{naxwo_ramfit}
\caption[\ce{Na5W_{14}O_{44}} Raman fitting]{Multi-peaks Lorentzian fitting on two major peaks region of \ce{Na5W_{14}O_{44}}. The peaks sum height difference is caused by different baseline value adopted in each fitting.}
\label{fig:naworamfit}
\end{figure}

% W-O bond length
\begin{table}[htb]
\centering
\caption{\ce{W-O} bond length predication}\label{tab:nawobond}
\begin{tabular}{lccr}
\toprule
peak center & length (\AA) & peak center & length (\AA) \\
\midrule
694.6 & 1.900 &  808.6 &  1.821 \\
745.4 & 1.863 &  911.5 &  1.758 \\
764.4 & 1.850 &  933.0 &  1.745 \\
778.7 & 1.840 &   943.5 & 1.740 \\
788.4 & 1.834 &   965.4 & 1.728 \\
\bottomrule
\end{tabular}
\end{table}

An empirical formula to relate the Raman peaks and \ce{W-O} bonding lengths \cite{Hardcastle1995} is
\begin{equation}\label{eq:wobond}
\nu = 25823 \exp{-1.902\cdot R}.
\end{equation}
And the standard deviation of estimating \ce{W-O} bond distance from Raman stretching wavenumber is $\pm0.034$\AA.
The observed Raman peaks of \ce{Na5W_{14}O_{44}} phase lies at 965, 943, 913, 808, 786, 778, 765, 695 and 107 cm. Multi-peaks Lorentzian fitting is preformed to precisely determine the central maximum. Good fitting is obtained, as shown in Fig.~\ref{fig:naworamfit}. We then calculated \ce{W-O} bond distance using Eq.~\ref{eq:wobond}, as illustrated in Table~\ref{tab:nawobond}. The predicted \ce{W-O} bond length comply very well with the crystallographic value of \ce{Na5W_{14}O_{44}} phase.\cite{Triantafyllou1999a} The 107 peak probably is caused by \ce{Na-O} bond.


\subsection{not used}
Both tungsten (W) and molybdenum (Mo) belong to Group VIB transition metal, with outer shell electrons configuration as $4d^55s^1$ and $5d^46s^2$, respectively. Therefore we refer their oxides and chalcogenides as \gls{tmo} and \gls{tmdc}.\footnote{Obviously transition metals include many other elements, all of which have partially filled $d$-electron shell. But here we use TM to denote W and Mo exclusively.}

Nucleation is a process of generating a new phase from a metastable old phase, where the Gibbs energy per molecule of the bulk of the emerging new phase is less than that of the old phase.

General CVD knowledge, substrate preparation, and\cite{MichealK.Zuraw2003}


The energetic sources are ion bombardment, electron beam, laser ablation, and combustion flame\cite{Rao2011}.

The sol–gel process is a well-known, intensively studied wetchemical technique that is widely used in materials synthesis. This method generally starts with a precursor solution (the ``sol") to form discrete particles or a networked gel structure. During the course of gelation (aging process), various forms of hydrolysis and polycondensation take place.
In addition, doped \ce{WO3} was also demonstrated
The composition and phase of final product highly depend on the synthesis conditions.

We do not discuss tungsten oxide hydrates (\ce{WO3.nH2O}) in this work since the product of thermal CVD approach is not plagued with this complexity. It's necessary, however, to deal with hydrated \ce{WO3} in the liquid synthesis routes, as indicated in Section.~\ref{sec:woxgrowth}.


Nonstoichiometric tungsten oxides \ce{WO_x} (i.e. \ce{WO_{2.92}}, \ce{WO_{2.87}}) are known as Magn$\acute{e}$li phases.


Theoretical computation of electronic band structures for \ce{WO_x} proves difficult due to the aforementioned phase transition. oxygen deficiency, structure change, electronic properties vary according.

the ubiquity of \ce{WO6} octahedra is essential for not only the optical properties but the ability to insert and extract ions in the EC oxides, due to the tunnels in three dimensions serving as path for transport of small ions. The intercalation of hydrogen or alkali ions into \ce{WO3} created electron donor level. By absorbing the red part of incident spectrum, electrons at donor level make transition to the conduction band, causing the blue coloration in \ce{H_xWO3}.

its one dimensional (1D) nanostructure has attained intensive research efforts in recent years due to the potential applications in advanced nano-electric and nano-optoelectronic devices.

\begin{quote}
a viable electrochromic smart window must exhibit a cycling life time \textgreater $10^5$ cycles, corresponding to an operation life at 10 - 20 years.
\end{quote}

\subsection{to be used}

W plasma oxidation.\cite{Romanyuk2005} 200nm W coating on Si (100) sub, temperature at RT, 390, and 490 C, oxygen pressure 0.5 Pa, oxidation time for 10 to 3600 s. The resultant thickness of \ce{WO3} at RT  and time of 10 s and 3600 s is found to be 0.2 nm and 11 nm respectively.

\[
 d = d_0 exp(kt)
\]

after fitting, $d_0 = 0.19777$ nm, $k = 0.00112 $ s$^{-1}$, so to oxidize 1nm W coating completely, oxygen plasma time is 1450 s at 0.5 Pa partially pressure; 2 nm for 2000 s.



\ce{WOx} and optical electric field enhancement. The enhancement arise from the structure composed of a conductive layer and an insulating layer that are laminated therein.\footnote{US patent 8601610B2} In \ce{WOx} nanorods, the oxygen deficient planes are conductive, each having atomic thickness and separated by several nm \ce{WO3}. Localized surface plasmons could possibly exist on these conductive planes. Therefore SERS applies and single molecule Raman scattering using a tungsten oxide nanorod has been demonstrated. The \ce{W_nO_{3n-1}} ($n \geq 2$) exhibit $\{ 001 \}$ CS structure. Chemical formulae corresponding to n=2, 3, 4, 5 and 6 are \ce{W2O5=WO_{2.5}}, \ce{W3O8=W_{2.67}}, \ce{W4O_{11}=WO_{2.75}}, \ce{W5O_{14}=WO_{2.8}}, and \ce{W6O_{17}=WO_{2.83}}, which indicates the existence of a oxygen deficient plane at every n row. Actually the value x in \ce{WOx} could almost continuously vary within a range of 2.5 to 3. \ce{W_{18}O_{49}=\ce{WO_{2.72}}} is an exception without $\{ 001 \}$ CS structure. Moreover, the oxygen deficient planes could extend along directions other than $\{ 001 \}$. For instance, the $\{ 102 \}$ CS planes appears in \ce{WOx} where x is within 2.93 to 2.98, and  the $\{ 103 \}$ CS planes for x within 2.87 to 2.93.\cite{Sloan1999}  \citeauthor{Shingaya2013} also synthesized \ce{WS2}-\ce{WO_x} structures and found similar Raman scattering enhancement. The x value is estimated by the Raman spectra peaks.\cite{Shingaya2013}(Data not shown in patent)

For photochemical water reduction to occur, the flat-band potential of the semiconductor (for highly doped semiconductors, this equals the bottom of the conductance band) must exceed the proton reduction potential of 0.0 V vs NHE at pH =0. \cite{Osterloh2008} flat-band potentials strongly depend on ion absorption (protonation of surface hydroxyl groups), crystallographic orientation of the exposed surface, surface defects, and surface oxidation processes.


\ce{W_{18}O_{49}} Raman, IR shielding.\cite{Guo2012} \cite{Guo2011}
broad peak between 750-780 cm-1.

\ce{WnO_{3n-1}} NPs. \cite{Frey2001}


WO3-x raman info, the encapsulated WOx core has been investigated in depth. several stable phase could occur, including \{001\} CS phases, \{103\} CS phases. no evidence of $\gamma$-\ce{W_{18}O_{49}} phase is found. The cross-section ($\sigma$) for Raman scattering and the absorption coefficient of the WS2 layers are much larger than those of the suboxide phase encapsulated inside.

WO2:168(w),189(w), 286(vs), 345(w), 423(w), 479(m), 512(m), 599(m),
617(m) cm-1, and a mode at 781(s) cm-1 which tails to higher energies (w-weak; m-medium; s-strong; vs-very strong).

W5O14: 215, 264, 325, 349, 418, 425,707, and 800 cm-1, 900 maybe

WO3: 808, 719, 275;

W3O8: 870;

no 950 peak indicates no hydrated phases.

\ce{WO_{3-x}} Raman peak at 778. \cite{Deb2007}

% wo3-x phases
\begin{table}[htb]
\centering
\caption{List of \ce{WO_{3-x}} phases}\label{tab:wo3xphase}
\begin{tabular}{lccccc}
\toprule
&&&\multicolumn{3}{c}{Lattice constants \AA} \\
\cmidrule(l){4-6}
 Symbol    & PDF  & Phase & a & b & c   \\
\midrule
\ce{W18O49}  & 00-036-0101 & monoclinic & 18.324 & 3.784 & 14.035  \\
$\delta$-\ce{WO3}   & $-50 \sim 17$  & triclinic & 7.309 & 7.522 & 7.686  \\
$\gamma$-\ce{WO3}   & $17 \sim 330$  & monoclinic I & 7.306 & 7.540 & 7.692  \\
$\beta$-\ce{WO3}    & $330 \sim 740$  & orthorhombic & 7.384 & 7.512 & 3.846  \\
$\alpha$-\ce{WO3}   & $> 740$  & tetragonal & 5.25 & NA & 3.91  \\
$h$-\ce{WO3}        &  $<400$  & hexagonal & 7.298 & NA & 3.899  \\
\bottomrule
\end{tabular}
\end{table}



\section{moo3}

\subsection{used}



\ce{MoO3}, an alternative interpretation in terms of tetrahedral coordination of Mo atoms is also proposed. This is caused by the fact that four of the six surrounding O atom are at distances from 1.67\AA to 1.95\AA, while the remaining two are as far as 2.25 and 2.33\AA. This also stress that the \ce{MOO6} octahedra are rather distorted.


\subsection{to be used}


\cite{Matar2011} Using electronegativity $\chi$ and chemical hardness $\eta$ to assess electron affinity $E_a$, work function $W_f$, Fermi energy $E_f$ and band gap $E_g$.
\begin{align}
\chi &= 0.5(W_f + E_a)\\
\eta & = 0.5(W_f - E_a)
\end{align}
where I is ionization potential and $E_a$ is electron affinity.

Correlation between optical band gap and formation enthalpy; reaction occurs in order to form compounds with a larger gap.  $E_g = A \exp(0.34E_{\Delta H^0})$, and A adopts different values depending on the metal elements:
\begin{itemize}
\item A=0.8 for s and f block elements,
\item A = 1 for d block elements,
\item A = 1.35 for p block elements.
\end{itemize}


\citeauthor{Sreedhara2013} studied the kinetics of photodegradation of methylene blue\footnote{\ce{C16H18N3SCl},319.8 g/mol, MP: 100C accompanied with decomposition \url{http://en.wikipedia.org/wiki/Methylene_blue}} dye by few layer \ce{MoO3}.
For the photodegradation method, it was stated that `` the samples were collected after the photoreaction had been centrifuged for 5 min to remove the photocatalyst before UV-Vis measurement.''


% Melting points 
\begin{table}[htb]
\centering
\renewcommand*{\thetable}{S\arabic{table}}
\caption{physical constants of reactants }\label{tb:thermo}
\begin{tabular}{lccr}
\toprule
Material & MP(\si{\degreeCelsius}) & BP(\si{\degreeCelsius}) & reference\\
\midrule
\ce{NaOH}        & 318 & 1388 & handbook  \\
\ce{NaI}        & 651 & 1300 & MSDS    \\
\ce{KI}        & 681 & 1330 & MSDS   \\
\ce{Na2CO3}        & 851 & Not determined & MSDS    \\
\ce{Na2MoO4}        & 687 & Not available & handbook   \\
\ce{MoO3}    & 795 & 1155 & MSDS   \\
\ce{MoO2}    & 1100(decomp) & Not available & MSDS   \\
\bottomrule
\end{tabular}
\end{table}



\section{ws2}


\textbf{\ce{WS2}-\ce{WO3}}: 1 kW light source(Hg, or Xe lamp), photon flux, phenol (\ce{C6H5OH}, 94.1g/mol, MP 40C)concentration is 20 mg/L, hydroxyl group. The quantitative analysis of phenol was performed via a standard colorimetric method.\footnote{\url{http://omlc.ogi.edu/spectra/PhotochemCAD/html/072.html}}
\citeauthor{DiPaola1999} prepared \ce{WS2}-\ce{WO3} mixture in two methods, sulfurization of \ce{WO3} and oxidation of \ce{WS2},with the latter are more active.
\citeauthor{DiPaola1999} also concluded that the actual efficiency of mixed \ce{WS2}-\ce{WO3} catalysts depends on the ratio of each composition present of the surface of the particles, and the maximum of photoactivity is obtained with 40-50\% surface molar ratio of \ce{WS2}.

ref 25, 28 and 41.


\subsection{used}

As the experimental setup for direct tensile tests of nanotubes is state-of-the-art,\cite{Tang2013} the application of tensile stress on 2D TMD systems is rather difficult due to the excellent lubricating properties of these materials.

\citeauthor{Zhang2013e} investigated the shear (C) and layer breathing mode (LBM) in the low frequency region of \ce{MoS2}.\cite{Zhang2013e} Even layer \ce{MS2} belong to point group D$_{6h}$ with inversion symmetry, while odd layer \ce{MS2} correspond to D$_{3h}$ without inversion symmetry. The excitation wavelength is 532nm from a diode-pumped solid-state laser. A power$\sim$0.23mW is used to avoid sample heating.

reaction mechanism of \ce{MoO3} to \ce{Mo2S}.\cite{Weber1996}

\citeauthor{Ling2014} studied the role of seeding promoters in CVD growth of FL \ce{MoS2}.\cite{Ling2014} PTAS treated substrates provided nucleation site and thus enable uniform deposition of \ce{MS2}.  This enhancement perhaps arise from the \ce{K+} ions.

\citeauthor{Splendiani2010} reported the PL in monolayer \ce{MoS2}.  Calculation indicated the indirect gap become larger when thinning, while the previous direct one almost stays as the same, the value is about 1.85eV (direct gap).\cite{Splendiani2010}


thermal decomposition of (NH4)2MoO2S2 and intermediate product MoOS2 was studied. application: hyfrodesulfurization in refinery \cite{Weber1996}

\cee{MoCl5 + 1/4S8 + 5/2H2 \rightarrow MoS2 + 5HCl} \cite{Stoffels1999}


A direct gap of $\sim 2eV$ at the corners of BZ is formed in 1L \ce{WS2}, Growth on bottom piece show the multiple domain flakes occurs at initial stage of the growth, starting from \ce{WO3} particles.
%\cite{Cong2013}
\subsection{to be used}

Exfoliated WS2 few layer PL.\cite{Zhao2012} excitonic absorption peaks A and B arising from direction transition at K point are found around 625nm (1.98eV) and 550nm, respectively, which are in agreement with results from bulk layers. The A, B excitons difference was a result of strong spin-orbital coupling. Relative PL quantum yield of WS2 between 1L and 2L is on the order of 2. The FWHM of WS2 peak is about 75 meV. wider than thermal energy at room temperature,

Electro microscopy on stacking sequences of WS2 NT.\cite{Houben2012} The probability of parallel stacking is about 30\%. a metal-semi superstructures. In NT, the layers are slightly shifted with respect to each other due to the constraints, thus the stacking is not exactly as pure phases of 2H(prismatic antiparallel), 3R(prismatic parallel) or 1T (octahedral parallel) with their perfect translational symmetry.

chevron pattern, contradictory, contradicting, Debye scattering model for XRD.

\begin{quote}
hexagonal polytype 2Hb with two molecular layers (spacegroup P63/mmc) and a rhombohedral polytype 3R with three molecular layers per unit cell (space group R3m), a high pressure polytype that is stable in plane geometry at pressures above 4 GPa. The two prismatic phases are semiconducting, and the octahedral one is metallic-like.

1T phase may be the result of a transformation from the 3R to the 2H phase by an intermediate 1T phase that is trapped by fast quenching

\end{quote}

aberration corrected TEM is used.

HRTEM on WS2 NT.\cite{Sadan2008} negative spherical-aberration imaging (NCSI). NCSI condiction were achieved at a negative spherical aberration of -20um balanced by an overfocus of +17 nm. Focal series reconstruction to retrieve the phase of electron exit plan wavefunction. Zigzag, armchair revealed.


In centrosymmetric crystals, the vibrational modes must either have even (Raman-active) or odd (IR-active) parity under inversion, which is known as rule of mutual exclusion. When this symmetry is broken, some modes may be simultaneously IR and Raman active.

inelastic neutron scattering to study the non-zone center LA mode. Zone-edge scattering can occur due to zone-folding process. The formation of superlattice could activate formerly inactive zone-edge phonons. The folding of BZ along $\Gamma-M$ would cause the M point to coincide with $\Gamma$ point, so LA(M) phonons would become Raman active in a first-order process.


\ce{SiO_x}-Si, \ce{WS2} absorption coefficient $10^{-7}m^{-1}$, mean free path of photo-excited charge carriers 1 $\mu m$. the wave vector of photon is considerably small than size of BZ, therefore The wave vector of phonon in Raman scattering usually close to zero.

Multiple phonon scattering, For two identical phonons, the corresponding Raman peak in the spectrum is called an overtone of the peak from the corresponding one-phonon process. And the wave vector conservation rule is automatically filled, therefore the phonon involved is not limited to BZ center anymore.
\[
I(G) \approx \sum_k \frac{\langle f|H_M|b\rangle \langle b|H_{ep}|a\rangle \langle a|H_M|i\rangle}{(E_p - E_k^{\pi *}- E_k^{\pi}-i\gamma)(E_p - E_k^{\pi *}- E_k^{\pi}-i\gamma- \hbar\Omega_{G})}
\]

the average distance traveled by an excited electron-hole pair before combination $l=\nu_F/\omega_D=4nm$.

Confocal Raman spectrometer:to obtain Raman spectrum in a specific depth of sample. Edge filter to cut off Rayleigh emission.


resolution $d= 1.22 \lambda/NA$,

Light Scattering in Solids II,. Springer, Berlin, 1982

influence of core WOx, Raman scattering by plasma-LO coupling to determine carrier concentration. measure resonant cross sections in absolute units.

disorder-induced light scattering, Van Hove critical points,
In resonant second-order scattering:
overtone: the same phonon,
combination: two different phonons;

\[
\frac{\ud\sigma}{\ud\Omega}= \omega_s^4 cm^6 Sr^{-1}
\]

scattering volume V in number of unit cells can be considered as one big molecule.


a single nanowire tends to minimize its surface. 2D isoperimetric quotient or circularity $C= \frac{4\pi A}{P^2}$, where A is area and P is perimeter of the cross-section.




\section{ECD}


Characterization of ECD (work like a thin-film batteries) includes transmission measurement and associated EC calculation, charge-discharge time, current-time curve and the fitting of obtained data.

The coloration efficiency (CE) represents the change in the optical density (OD) per unit charge density ($Q/A$, in units of \si{\cm^2\per\coulomb}) during switching and can be calculated according to the formula:
\begin{equation}
CE = \frac{\Delta~OD}{(Q/A)} [cm^2/C],
\end{equation}
where OD = $log(T_{bleach}/T_{color})$. The EC and optical density depend on the wavelength and are usually higher in the near IR than in the visible region.
Using Ohm's law($U_s = IR = RQ/t_s$) with switch voltage $U_s$, resistance R and surface area A, switching time $t_s$ could be estimated as
\begin{equation}
t_s = \Delta~OD\cdot A \cdot R /(CE\cdot U_s).
\end{equation}



battery and ECD.\cite{Granqvist2012} electrolyte: PVB (poly vinyl buteral).
alternative materials and design: organic, Prussian Blue as EC materials, metal hydrides, suspended particle device, liquid crystal, electroplating,
challenges: large area nanoporosity, transparent conducting contact, electrolyte with good ionic conductivity and poor electronic conductivity, stable under UV; assembly and large scale manufacturing;
cathodic coloration:
anodic coloration:
The coloration mechanism: \ce{MO6} octahedrons lead to $e_g$ and $t_{2g}$ level and ion channelling.
ref54,60,65,66,200,209,


\ce{WO3} as cathodic and either polyaniline(PANI) or Prussian white (PW) as anodic electrochromic half cells. \cite{Heckner2002}

Characterization of ECD includes transmission measurement and associated EC calculation, charge-discharge time, current-time curve and the fitting of obtained data.

\begin{quote}
a viable electrochromic smart window must exhibit a cycling life time \textgreater $10^5$ cycles, corresponding to an operation life at 10 -- 20 years.
\end{quote}


\citeauthor{Sella1998} studied the optical and structural properties of RF sputtered thin film of \ce{WO3} and \ce{VO2} for electrochromic devices. Ionic conductor was built using transparent polymer electrolyte, which was prepared from a solution of 1M \ce{LiClO4} in propylene carbonate which was mixed with methylmetharcylate (MMA). The main characteristics of polymer electrolyte were: viscosity at 25 \si{\degreeCelsius} $\approx$ 12920 Pa.s, conductivity $\approx 10^{-2}-10^{-4}$ \si{\per\ohm\per cm},non-hygroscopic if PMMA concentration \textgreater 30\%. A specific counter-electrode layer was not used since the encapsulated polymer electrolyte processes a very high ion storage capacity.\cite{Sella1998}

The device proposed was reproduced as shown in Fig.~\ref{fig:Sella1998ECD}
\begin{figure}[htb]
    \centering
    \includegraphics[angle=270,width=0.8\textwidth]{Sella1998ECD}
    \caption{citation, see original captions} \label{fig:Sella1998ECD}
\end{figure}


    \chapter{paper reading}



\section{WO3}

Tungsten bronzes was coined by Wohler in 1837.\cite{Deb2008} \ce{Na_{x}WO3}



\subsection{applications}


\citeauthor{Wang2009a} mentioned that amorphous \ce{WO3} can only be used in lithium-based electrolytes due to its in-compact structure and high dissolution rate in acidic electrolyte solutions. Electrochromic materials that can endure acidic electrolytes without degradation should be developed. Crystalline \ce{WO3} nanostructures with their much denser structures and small particle sizes are promising to be used as suitable electrochromic material in acidic electrolytes.

photocatalytic applications in solar hydrogen generation and organic pollutant degradation.

photocatalyst\cite{Macphee2010},
photoelectrochemical energy application \cite{Su2010}

Raman \cite{Xiao2007}. Silver has the strongest SERS enhancement due to the larger imaginary part of the dielectric constant and higher thermal conductivity. Milli-Q grade water ((Milli-pore)\textgreater 18.2Mohm).

MB Raman peaks: 445, 1618, ref20. some peak splitting and shift observed on SERS, attributed to chemical adsorption. definition of Raman enhancement factor (9,26).

SERR MB on Ag. \cite{Nicolai2003}
MB: the absorption spectrum in VIS is used to infer about different adsorbed forms of MB. the formation of large aggregates. I call attention to the fact that.

WO3: effective mass of bipolaron = 1.9me. for electron, for hole:
unzip nanotube. passivate BN ribbons with O and S; another player terrones psu.

\ce{WO3} catalyst.\cite{Miyauchi2013}  potential of CB e more negative than redox potential of \ce{O2}-\ce{O2^-} (-0.046 V vs NHE at pH 0). Z-scheme two photo absorption. photogenerated ele in CB of WO3 can reduce itself by formation of color centers.

electrochromic films. \cite{Yoshimura1985}

ECD \cite{Jiao2012} recent review \cite{Mortimer2011}

PEC, photoelectrode, WO3 and Si tandem structures.\cite{Coridan2013}

WO3 photoactivity MB. \cite{Watcharenwong2008}
A low recombination rate is preferred for high photocatalytic efficiency. The simultaneous migration of electrons and holes.



\subsection{properties}

\citeauthor{Chatten2005} also studied the oxygen vacancy in different phases of \ce{WO3}.\cite{Chatten2005}

2D wo3.\cite{Kalantar-zadeh2010a} 
WO3 plasmon \cite{Manthiram2012}

DFT doped \ce{WO3} for photocatalytic reaction.\cite{Wang2012} CBM arises from W $5d$ states and splits into $t_{2g}$ and $e_g$ states under crystal field. VBM comes from O $2p$ states, including $2p_\sigma$ (along \ce{W-O} bonds) and $2p_\pi$ (normal to \ce{W-O} bonds).


oxygen vacancies in \ce{WO_{3-x}}.\cite{Wang2011b}  Coloration and electron conductivity changes. \citeauthor{Wang2011b} found strong dependence of WO3-x electronic properties on $V_O$ concentration and the the crystallographic direction on which O is removed. DFT band gap calculation is close to experimental value. Vacancy levels are found at 2.1eV.

The Raman spectra of \ce{WO_x} is rare because of the difficulty of preparing pure suboxides phase and the strong shielding of \ce{WS2}. Yet it does exhibit distinct Raman spectra. \cite{Tenne2005} The 870 line is attributed to \ce{W3O8}.\cite{Hardcastle1995}


\citeauthor{Huang2006} studied the \ce{W3On} cluster with n from 7 to 10.\cite{Huang2006} It was found \ce{W3O9} clusters possess a HOMO-LUMO gap about 3.4eV. This closeness to bulk value suggests \ce{W3O9} could be viewed as the smallest molecular unit for bulk \ce{WO3}.


\ce{WO3} indirect gap 2.6eV, direct gap 3.4eV. \cite{Koffyberg1979}

Raman fingerprint of m-\ce{WO3}, h-\ce{WO3} and \ce{WO3.nH2O} were summarized in ref\cite{Daniel1987}.

\ce{WO3} on FTO by flame synthesis.\cite{Rao2014} \cite{Xu2006}

Seeded \ce{W_{18}O_{49}} NWs growth on W foil.\cite{Hong2006a}

\ce{Na2W4O_{13}} growth and optical properties. \cite{Oishi2001} \cite{Itoh2001}

\ce{Na2W4O_{13}} crystal phase \cite{Viswanathan1974}

\citeauthor{Salje1984} studied the transport in \ce{WO_{3-x}} ($0\leq x \leq 0.28$).\cite{Salje1984} It was found \ce{WO_{3-x}} show metallic conductivity when $x > 0.1$.

\ce{WO_{3-x}} \cite{Migas2010}

\ce{WO3} high temperature phase. \cite{Vogt1999}
tungsten bronzes \cite{Wiseman1976}

Phase transformation of \ce{Na2MoO4} and \ce{Na2WO4} by Raman scattering. \cite{Lima2011}

\ce{WO2} NWs synthesis and raman \cite{Ma2005}.

\ce{WO_{3-x}} CS planes and conductivity.\cite{Sahle1983}

\ce{W-O} equilibrium diagram \cite{Wriedt1989}

\ce{W_{18}O_{49}} electrochromic devices.\cite{Liu2013d} should compare with this one \cite{Wang2008}

nucleation catalysis \cite{Turnbull1952}

\ce{WO3} NWs aggregates. \cite{Kozan2008a}

optical properties of \ce{WO3} gaps\cite{Saygin-Hinczewski2008}

\ce{WO3} atomic layer by exfoliation and annealing \ce{WO3.H2O}. \cite{Kalantar-zadeh2010a}

sodium tungstates raman \cite{Redkin2010}

charge density wave in K-doped \ce{WO3} \cite{Raj2008}

\ce{W_{18}O_{49}} Raman, IR shielding.\cite{Guo2012} \cite{Guo2011}
broad peak between 750-780 cm-1.

\ce{WnO_{3n-1}} NPs. \cite{Frey2001}

\ce{WO3} growth hydrothermal.\cite{Moshofsky2012}

\ce{W_{18}O_{49}} on tungsten foil by thermal growth\cite{VanHieu2012}

Cathodoluminescence \cite{Parish2007}

optical characterization of WOx film.\cite{Valyukh2010a}

E-beam penetration \cite{Kanaya2002}

optics in electron microscopy. \cite{GarciadeAbajo2010a}


\section{MoO3}

VLS:
Ge NW growth using Ga as catalyst. \cite{Chandrasekaran2006}


plasmon dispersion in 2D materials, plasmon resonances in visible regions by doping induced free carrier density. 2D plasmonics, depolarization factors, partial reduction of Mo to a lower valence state. \cite{Alsaif2014a}

\subsection{app}
applications: electrically controlled optical shutters for heat and light modulation, smart windows associated with solar cell to provide dynamical control of incoming illumination.

piranha clean of FTO. 50ms switch.\cite{Scherer2012} 
nanoscale Kirkendall effect: the outward diffusion of metal cations are balanced by an influx of vacancies. For example, diffusion coefficient of Ni in NiO is higher than that of oxygen.

\ce{MoO3} photocatalytic \cite{Chithambararaj2013}
photocatalytic experimental setup.\cite{Hupka2006}
\ce{MoO3} pseudocapacitor  \cite{Brezesinski2010}

\ce{MoOx} few layer as hole selective contact in solar cell.\cite{Battaglia2014}
\ce{MoO_x} on n-type Si acts as a high work function metal (6.6eV), enabling a dopant-free contact and thus junction-less devices.


\subsection{properties}



\begin{table}[htb]
\caption{Combinations of ECD configuration}\label{tb:ecd}
\begin{tabular}{lcccr}
\toprule
TC(both side) & electrochromic & ion conductor & counter electrode  & reference\\
\midrule
ITO &  \ce{WO3} & \ce{H^+\hyphen} polymer & PANI &\citeauthor{Heckner2002}\\
FTO &  \ce{WO3} & \ce{K^+\hyphen} polymer & PW &\cite{Heckner2002}\\
ITO & \ce{WO3} NWs & \ce{LiClO4\hyphen}PC & none & author design \\
\ce{Na_xWO3} NWs &\ce{WO3} NWs & \ce{LiClO4\hyphen}PC & none & author design\\
\bottomrule
\end{tabular}
\end{table}



\begin{table}[htb]
\centering
\caption{Comparison of MoOx ECD}\label{tab:moxecd}
\begin{tabular}{lcccr}
\toprule
$\lambda$ & $\Delta T$ & $t_c$ & $t_b$ & $CE$  \\
         (nm) & (\%)    & (s) & (s) & ($cm^2/C$)  \\
\midrule
Range      & RT-1100    & 10mTorr-1atm & 0 - 100 & 0-30  \\
\bottomrule
\end{tabular}
\end{table}


10nm MoOx as hole extraction layer (HEL). Without HEL, Holes accumulates at QD/anode interface, causing increased recombination rate. With HEL, hole diffuse into this layer, reducing the recombination. 

The molecular unit in crystal exhibits different vibrational frequencies from that in solution or gas phases.

\ce{Na2Mo4O_{13}} phases monoclinic at RT, solid solubility of \ce{Na2MoO4} in solid \ce{MoO3} is high. vapor pressure of \ce{Na2Mo4O_{13}} over \ce{MoO3}.

melting point of \ce{Na2Mo4O_{13}}
Mp: \ce{Na2Mo2O7} 960K

\ce{MoO3} vapor pressure:

The real phase diagram is the one between \ce{Na2Mo4O_{13}} and \ce{MoO3}.
the growth temperature could be much lower than the eutectic point.

KI MP:  681
NaI MP: 661

NaOH Raman peaks lie at 3633 cm. \cite{walrafen2006} Raman scattering of \ce{Na2SiO3} exhibit major peak at 966 and 589 cm.\cite{Richet1996}

hydrogen absorption in \ce{MoO3}.\cite{Sha2009}

\ce{Na6Mo_{11}O_{36}} phase. \cite{Bramnik2004}

\ce{Na6Mo_{10}O_{33}} phase, \cite{Gatehouse1983}

\ce{MoO3} thin film. \cite{Carcia1987}

\ce{H_xMoO3} raman.\cite{Hirata1996}

MoO3 spreading \cite{Leyrer1990}

Na2Mo2O7, Na2Mo4O13 phase transition \cite{SinghMudher2005}\cite{Tangri1992}

visibility of FL \cite{Benameur2011}

exfoliation IPA \cite{Halim2013}  \cite{Zhou2011a}

\ce{MoO3} good style. \cite{Siciliano2009} \cite{Abdellaoui1997}

\ce{MoO3}  DFT study \cite{B511044K} \cite{Cora1997} \cite{Sayede2005}

\ce{MoO3} raman \cite{Lee2002}

visibility of mica thin layer on \ce{SiO2}-Si. \cite{Castellanos-gomez2011} 1.5\% contrast is almost at the threshold of human eye sensitivity.  When the thickness is below 60nm, Raman could not detect mica.

\ce{MoO3} (010) surface defect. \cite{Chen2001}

mass spectrometry data to extract vapor pressure of \ce{NaxMoO3}.

\ce{MoO3} SWNT by hydrothermal method.\cite{Hu2008a} Raman spectra is off compared to single crystal \ce{MoO3}.  Van der Waals interaction and layered structure make NT possible.

TMO review.\cite{Goodenough2013}

h-\ce{MoO3} \cite{Lunk2010} \cite{Zheng2009}



\section{TMS}

petroleum oil catalytic refinement, solid lubricants in aerospace industry.

heterojunction is employed to transferred photo-generated carriers. Schottky barrier conduction band electron trapping and consequent longer electron-hole pair lifetimes. Numerous studies have suggested that fine particles of transition metals or their oxides, when dispersed on the surface of a photocatalyst matrix, can act as electron traps on n-type semiconductors.\cite{Zhou2010} 


\citeauthor{Cao2014} studied the layer-dependence \ce{MoS2} electrocatalysis and propose the vertical hopping efficiency of electrons instead of the edge site numbers is a key factor for catalytic reaction.\cite{Cao2014} ref19,20


\subsection{strain}

$E_{2g}$ mode is strain sensitive. 

\citeauthor{Ghorbani-Asl2013} studied the strain in tubular TMDC and found a linear dependence of Raman scattering on strain (3 \si{cm^{-1}} per percentage for $E_{2g}$mode).\cite{Ghorbani-Asl2013} 

For 2D materials, strain may be induced by elongation of an appropriate substrate, e.g. by uniform mechanical strain, or by using a material with high thermal expansion coefficient and varying the temperature. For TMD MWNT, tensile tests have been reported by various groups. However, to date, it is not perfectly clear whether inner and outer walls are stretched simultaneously, or rather the outer walls slide on the inner ones. The latter hypothesis would result in a broadening of the Raman signals, while the first one would leave the signal widths rather unaffected. In any case, there would be a shift of the Raman signals that can serve as precise scale for determining the strain.\cite{Ghorbani-Asl2013}


\citeauthor{Virsek2007} performed a Raman-TEM integrated study on multiwalled \ce{WS2} NT with diameter \textgreater 200 nm. The tubes were synthesized using chemical transport method. Up-shift of Raman is explained by strain in the walls. This shift is not observed in the specimen by sulfurization process of oxides. Applied hydrostatic pressure is isotropic,\cite{Staiger2012} while the strain is expected to anisotropic. Strain can also be relaxed by chirality.\cite{Virsek2007} 

strain effect by first-principles calculations. direct gap is only maintain in a narrow strain range (-1.3 -- 0.3 \%), \cite{Yun2012}.

Semiconducting to metallic transition in \ce{MoS2} at compressive strain of 15\% or tensile strain of 8\%; direct-to-indirect gap transition for 1L \ce{MoS2} at about 2\%. \cite{Scalise2012}

strain and Raman theoretical analysis.\cite{Chang2013a} 

magnetic properties of ws2.\cite{Zhang2013j} 

growth mechanism of WS2 NT:

It was found that the critical step in this process is the fast conversion of the oxide nanoparticle surface into a closed monolayer of WS2. W18O49 as an intermediate phase is observed. XRD peaks shift to monitor strain.(002) peak of nanotube shifted to lower angles, the interlayer spacing increase by about 2\% as compared to the bulk powder, likely due to the build-in strain.\cite{ZAK2009} 
$\epsilon = (a - a_0)/a_0$ =(6.4-6.16)/6.16 = 3.8\%. tensile strain ($\epsilon > 0$)



`` Nanotubes not fully converted appeared also during short
runs with higher working pressure. HRTEM observations
revealed an amorphous phase inside some of the nanotubes’
hollow cores, generally near the nanotubes tip (Fig. 5a). The
amorphous phase occupies only a small fraction of the nanotube’s
core volume. A meniscus is found to form at the contact
point between the amorphous matter and the nanotube’s walls.
Fig. 5b displays such a meniscus in the nanotube core (marked
by arrows). The presence of this meniscus indicates that this
amorphous material solidified from a molten phase during the
cool-down period of the sample. The meniscus of the amorphous
phase suggest that the amorphous matter wets the
nanotubes’ walls. Since the WS2 nanotubes are hydrophobic,
this observation indicates that a monomolecular layer of oxide
is left on the entire hollow core of the nanotubes. The nanotube
walls near the contact area with the meniscus are quite defective,
probably due to the large differences between the thermal
expansion coefficients of the WS2 and the amorphous matter,
which induces strain during the cool-down period of the
reaction product. These observations are consistent with the
notion that the amorphous material inside the core is an oxide
phase which is hydrophilic and does not wet the hydrophobic
WS2 layers''\cite{Margolin2004}

\subsection{growth and properties}

WS2 photoluminescence spectra of few layer and nanotube:
NT electrical structures depend on chirality, diameter and layer No as well as strain. Theoretical calculation indicates the SWNT with diameter larger than 4nm should approach the single layer limit.\cite{Ghorbani-Asl2013}

Other chalcogenide has also been synthesized using this one-end sealed layout.\cite{Mukherjee2013}

\citeauthor{Zou2007} prepared W/\ce{WS2} core-shell NPs by reaction of tungsten and sulfur under hydrogen atmosphere.\cite{Zou2007}

CVD 1L WS2 PL.\cite{Peimyoo2013} (of NTU Yu group) PL peak at 635nm, width 40 meV, 

CVD 1L WS2.\cite{Cong2013} (of NTU Yu group) 457 nm excitation, PL at 525nm and 630nm, 

\ce{MoS2} sing-walled nanotube.\cite{Xiao2014}

1T MoS2: metallic phase a negative temperature coefficient for conductivity, XRD pattern identified. \cite{Wypych1992}

stable 1T WS2 multiwalled NT by Re doping.\cite{Enyashin2011}. 2H to 1T transition formerly known only for WS2 and MoS2 intercalated by alkali metals. 3R transition to 2H upon heating since 2H is the most stable one.

1T \ce{MoS2} Raman. \cite{Yang1991} strong peaks at 156, 226, and 330 cm-1. M point frequencies measured by neutron scattering. M point is folded into BZ zone center due to the formation of superlattice.

Electrons and Phonons in Layered Crystal Structures, edited by T. J. Wieting (Reidel, Dordrecht, Holland, 1979).

\ce{WS2} p-type or n-type.  Fermi level at the surface of semiconductor is pinned to a fixed position relative to the CBM and VBM by a sufficient density of surface states situated between CBM and VBM. \cite{Baglio1983}

Electronic structure of \ce{MoS2}.\cite{Eknapakul2014} K intercalating into bulk to create quasi-standing 1L. Large effective mass 0.6 $m_e$ found, implying low mobility. Direct gap 1.88eV is measured.

Self-assembled monolayer (SAM) on \ce{SiO2} and its effect on \ce{MoS2} 1L.\cite{Najmaei2014}

\ce{WS2} 1L doping calculation. \cite{Ma2011}




multipeak Lorentzian fitting. 270 to 410 cm


\citeauthor{Shi2013} studied the strained monolayer \ce{MoS2} and WS2. The results show that exciton binding energy is insensitive to the strain, while optical band gap becomes smaller as strain increases. Monolayer WS2 PL maximum located at about 1.95 eV. Calculation shows the electron effective mass of WS2 is the smallest, rendering higher mobility in device.\cite{Shi2013}

\citeauthor{Kosmider2013} studied the heterojunction between two monolayers of \ce{MoS2} and WS2. Top of VB in W layer and bottom of CB in Mo layer, forming type II structure. bilayer gap 1.2 eV.\cite{Kosmider2013}


Band structure  of \ce{MoS2} in bulk form was calculated by \citeauthor{Mattheiss1973}.The calculation result is 1.2eV (indirect gap).\cite{Mattheiss1973}

Alkali metal intercalated \ce{WS2} film was prepared.\cite{Homyonfer1997} Stage 6 superlattice formation was suggested according to X-ray diffraction, and photoresponse spectra and electron tunneling measurement were done.



quantitative Raman of MoS2 on insulating subs. intensity difference between supported and suspended was highlighted, detailed model in support info.Li2013

WO3-x (1nm) on SiO2/Si sulfurization at 750-950 degree,\cite{Elias2013}

decrease in dielectric screening and thereby enhanced excitonic effect.
DFT is not good at describing photoemission, GW approximation overcome this deficiency but still not enough for photoabsorption process in which ehps are created. BSE equation is used to compensate this discrepancy, WX2 exhibits larger spin-orbit splitting as compared to MX2 family.\cite{Ramasubramaniam2012}



arise as a result of, dispersal of Na by electron probe.



\section{CNT}

SOI:

VSS, growth kinetics,
BN nanotube capping, zigzag is more stable than armchair. \cite{Menon1999}


To develop large-size single-crystal graphene on dielectric substrates. small carbon flow near-equilibrium CVD process. Grain size about 10 microns, precursor \ce{CH4} and \ce{H2} (ratio 2.3:50) at 1180 C. \ce{SiO2}-Si surface roughness. Although the growth substrates (quartz,\ce{SiO2}-Si and \ce{Si3N4}-\ce{SiO2}-Si ) have a complicated stereo network similar to diamond, regular hexagonal G growth is obtained, which indicates the deposition is determined by equilibrium kinetics, and this should be applicable to other 2D materials as well. I2D/IG exceeds two on \ce{SiO2}-Si subs (514.5nm), indicating monolayer G. armchair (AC) G edge grows faster than zigzag (ZZ) edge.\cite{Chen2013j}

catalytic graphitization of solid carbon sources. catalytic transformation, the source is in solid state, low temperature (less than 600C), 2nm  \ce{Al2O3} by ALD as carbon diffusion barrier. amorphous silicon (a-Si), Ni lower the activation barrier ,  tetrahedral amorphous carbon (ta-C).\cite{Weatherup2013}

low energy (50eV) ion implantation doping in G. Ions penetrate pristine G at energy larger than 100eV. Individual substitutional incorporation of B into G lattice is demonstrated. 1\% doping level was obtained. \cite{Bangert2013}


CVD G on copper. Size of single crystal domain and nucleation site density.\cite{Wu2013b}

Concentration of charge carrier $n$ is related to gate voltage $V_g$ by:
\[
n = \frac{\epsilon_0 \epsilon V_g}{ed}
\]
where $\epsilon_r = \epsilon_0 \epsilon$ is dielectric constant of gate materials.

massless relativistic chiral particles, Klein paradox, 100\% tunneling and extreme high mobility.

\ce{CaF2} a material suitable for scattering efficiency S comparison measurement due to its large band gap ($S\times \omega_L^4$ is constant below 5eV).

symmetry-breaking mechanism,

low energy ion doping of graphene.\cite{Ahlgren2011}

\section{misc}

E-beam spatial coherence.\cite{Morishita2013} phase contrast transfer function, coherence estimated by the visibility of double slits interference fringes, an effective diameter in specimen plane.  Image is a result of convolution between object and lens, point source on the focal plane, after lens the EM wavefront intersect image plane at different angle $\theta = d/f$, 

nucleation and film growth \cite{Hanbucken1984}

intrinsic silicon equilibrium charge carrier concentration at RT is $n_i = p_i = 1.5 \times 10^{10} cm^{-3}$, much smaller than silicon atoms density as $5\times 10 ^{22} cm^{-3}$.

The average distance between dopant atoms is cubed root of concentration, $d = (10^{18} cm^{-3})^{-1/3} = 10nm$.

The electron mobility $\mu_n = 1500 cm^2/V\cdot sec $ at RT for Si, and hole mobility $\mu_p = 450 cm^2/V\cdot sec$ at RT.

for p-type silicon, when the conductivity $\sigma = 1 (ohm cm )^{-1}$, the doping level is
$N_A = \frac{\sigma}{q \mu_p}= 1 / (1.6E-19 \times 450) = 1.4E16 cm^{-3}$.

Built-in voltage $V_0 = \frac{kT}{q}ln(N_A N_D/n_i^2)$, depletion region width $W = \sqrt{\frac{2 \epsilon_{Si} V_0}{q}(1/N_A + 1/N_D)}$, where $\epsilon_{Si} = 11.7 \epsilon_0$. When applying external field, depletion width $W = \sqrt{\frac{2 \epsilon_{Si} (V_0 - V) }{q}(1/N_A + 1/N_D)}$

The capacitance of p-n junction is $C = A \sqrt{\frac{q \epsilon_{Si}}{2(V_0 -V)}(N_D N_A/(N_A + N_D))}$.


oxygen plasma treatment on HF-etched Si (001). reaction among $e$, \ce{O^+}, \ce{O2^+}, \ce{O^-},\ce{O2}. \ce{OH}-terminated surface obtained.\cite{Habib2010}

\subsection{MB}
MB is a heterocyclic aromatic dye which is blue colored in oxidizing environment. Upon reduction, MB is turned into colorless leuco MB. This can be used as an oxygen indicator in food industry. Photo-bleaching of MB can be also due to its leuco formation rather than total decomposition. Photocatalytic decomposition can be minimized by keeping the solution at acidic condition (PH = 4), which will limit the formation of oxidative hydroxyl radicals (E = 2.8eV vs normal hydrogen electrode). Oxygen dissolved in the solution play a key role in conversion of LMB to MB under visible light. Purging with \ce{N2} for 20mins can remove dissolved oxygen. \cite{Wang2014a}

solar energy harvesting representative study.\cite{Yoneyama1972} MB to LMB (\ce{C16H19N3S}) in aqueous solution upon illumination of \ce{TiO2}. The colorimetric analysis was performed in a glove box under nitrogen atmosphere. The absence of oxygen is important to prevent the oxidation of LMB to blue MB.
\[
\cee{MB^+ H2O + H^+ \rightarrow MBH3^{2+} + 1/2O2}
\]
where MB represents the uncharged center of MB molecule.

common wisdom expect that a dye incapable of injecting an electron at the excited state to CdS. MB, which process N-methyl groups in its molecular structure and does not sensitize CdS is an exemplary candidate. quantum efficiency is defined as probability of MB converted to azure B per incident photon. QE of CdS to MB decomposition is reduced in nitrogen bubbling treated solutions, indicating the necessity of oxygen. Two possible mechanisms: a) adsorbed oxygen acts as a trap for the conduction electron and prevent the accumulation of negative charge within space charge region of CdS, supported by the formation of \ce{O2^-} in excitation of CdS in aqueous suspension.\cite{Takizawa1978}

ref 16, MB aqueous solution stability. Liquid chromatogram, azure B (trimethylthionine), and thionine. Electrochemical measurement,

MB adsorption.  photocatalytic oxidation of MB by \ce{TiO2} film. photo-oxidation reaction occurs at the surface of photocatalyst. Mb molar extinction coefficient was found to be 66700 1/cm 1/M. Langmuir adsorption isotherm.\cite{Matthews1989}

\[
[MB]_{ads} = \frac{k_1 k_2 [MB]}{1 + k_1[MB]}
\]
and integrated form of Langmuir adsorption isotherm
\[
t = \frac{1}{k_1K} In\frac{[S]^0}{[S]} + \frac{1}{K}([S]^0 - [S])
\]
where $K = k_2 \phi N T_r$, with $\phi$ as quantum yield, N as total absorbed photons, and $T_r$ as rate of transport.
\[
\cee{C16H18N3SCl + 25.5O2 \rightarrow 16CO2 + 6H2O + 3HNO3 + H2SO4 +HCl}
\]
which indicates the total oxidation of $10 \mu M$ MB would exhaust the ambient oxygen concentration of initially air-equilibrated solutions (about $250 \mu M$ ). ref 28 Thus the transport of both oxygen and MB to the photocatalyst surface are anticipated to be key factors.

photoelectrochromism at \ce{TiO2}/MB interface and its control. Efficient capture of photogenerated holes by a reducing agent is crucial to the reversibility of bleach-recoloration transition. This transition is kinetically dictated by electron transfer. Holes transfer is not desired.\cite{DeTacconi1997}

256 nm band is associated to the presence of LMB. LMB formation is not favored at alkaline pH values in aqueous solution. The OH radicals are generated either with the surface hydroxyl groups on \ce{TiO2} or with water, and its high oxidizing power cause photocatalytic decomposition of the dye.

An elementary step in decomposition of MB is N-dealkylation, which is preceded by radical cation formation.\cite{Takizawa1978} This radical cation can be spectroscopically monitored by the presence of 520nm band for MB. In MB absorption spectrum, 664 and 614 nm band ratio is related to monomer and dimer relaxation.
\begin{align}
\cee{TiO2 &\rightarrow e_{CB}^- + h_{VB}^+ \\
h_{VB}^+ + red &\rightarrow ox\\
MB^+ + 2e_{CB}^ + H^+ &\rightarrow LMB}
\end{align}

Measure the ratio between 614 and 663 nm before and after adding WS2 can indicate the adsorption of monomer and dimer MB.

MB can act as sacrificial electron acceptor in the reduction to leuco form. The decomposition is favored under oxygen-rich environment. MB feature peaks at 663, 614 and 292 nm, and $\epsilon_{660}=10^5 M^{-1}cm^{-1}$. The doubly reduced form of MB, LMB has feature peak at 256 nm. The singly reduced form of MB, \ce{MB.^-} is pale yellow, with peak at 420nm.\cite{Mills1999}
\[
\cee{MB + e_{CB}^- ->[pH<7] MB.^-}
\cee{2MB.^- \rightarrow MB + LMB}
\cee{O2 + e_{CB}^- \rightarrow O2.^-}
\]

The oxidized form of MB, \ce{MB.^+} has peak at 520nm, which is stable in acidic solution, but decomposes irreversibly in slight alkaline solution(pH = 9).
thionine peaks at 600nm.
MB forms dimers in aqueous solution,
\ce{
2MB <=>[K_D] (MB)_2
}
A typical value of $K_D$ is 3970 1/M. A quadratic equation can be solved to obtain the monomer concentration:
\[
2K_D [MB]^2 + [MB] - [MB]_{total} = 0
\]
MB adsorption on metal oxides. Monomer size is less than 1.5nm.
Logarithmic acid dissociation constant $pK_a= -\log_10 \frac{[A^-][H^+]}{[HA]}$. The oxidation potential for \ce{H2O}-\ce{O2} couple is 1.23V and 0.817V versus NHE at pH 0 and pH 7, respectively.

%\begin{align}
%\cee{MB + SED &->[TiO2][h\nu \leq 3.2eV] LMB + SED^{2+}\\
%2LMB + O2 &\rightarrow 2MB + 2H2O}
%\end{align}


S.L. Murov, I. Carmichael, G.L. Hug, Handbook of Photochemistry, 2nd revised ed. Marcel Dekker, New York, 1993.

aerobic or anaerobic, dimerise, photominerlization, gas to liquid transfer.

Mb to LMB transition as visual time monitor. commercial colorimetric oxygen indicators. radical-bearing carbon with unpaired electrons. MB = \ce{MB^+Cl^-}.\cite{Galagan2008}

monomer MB and dimer MB kinetics.\cite{Spencer1979}



MB. \cite{Lee2003a}
\begin{align}
\cee{ 2LMB &->[\text{UV}] LMB^*\\
2LMB^* + O2 &\rightarrow 2MB^+ + 2OH^-}
\end{align}

\[
\cee{2LMB ->[\alpha] LMB^*}
\cee{2LMB ->[\text{above}] LMB^*}
\]


\section{solar cell}


\ce{TiO2} NPs for high loading of sensitizing dye. Hole conducting electrolyte with \ce{I^-} and \ce{I3^-} concentration close to $10^19 cm^3$. chemical anchoring groups.

electron injection rate. e transfer rate is several order faster than hole. 1 M = $6\times10^{20} cm^{-3}$. \ce{I^-} is known to coordinate with the sulfur atoms on NCS ligand. ref 31.

FRET: dipole-dipole coupling, energy relay dye to sentisizing dye and then to \ce{TiO2}. analogous to photosynthesis bacteria. Time-resolved PL to measure FRET $R_0$.

photocatalyst review.\cite{Mills1997} Definition: catalysis should not be used unless it can be demonstrated that the turnover number\footnote{the number of product molecules per number of active sites.} is greater than unity. Otherwise, semiconductor-assisted photoreaction is more appropriate. aerated, flush with air; nitrogen-purged. Degussa P25 \ce{TiO2} high temperature flame hydrolysis of \ce{TiCl4} in presence of hydrogen and oxygen. Oxidization of organic species is presumably obtained by \ce{Ti^{IV}OH^{.-}}, rather than direct hole transfer. carrier decay pathways. deactivation of catalyst by intermediate product.

MB natural decoloration under sunlight is found to be about 18\%.\cite{Nogueira1993} Latitude: 24 south, $3.4mW/cm^2$. natural evaporation should be prevented or corrected.

\section{dissertation}

WS2 surface hydrophilic or hydrophobic.
\[
\cee{Ti(IV)-OCH3 + h^+ -> Ti(IV)-O^+CH3}
\]

\ce{WO3} photoanode should be a n-type semiconductor, stable in acidic aqueous solution.

Swagelok TM. List of equipments,

Zheng thesis: \ce{TiO2} anodic nanotube by sputtering Ti on FTO and anodizing in F-organic electrolyte.

heat cure gasket ( ionomer surlyn 1702 Dupont), 125 C for 30s.
vacuum back filling.



\section{vocabulary}
$\delta\omega/\epsilon$ 

\subsection{pronunciation}

chamber, vias, valve, figure, energetic, managerial, inert, volatile, chromic,
laminar, palladium, platinum, photovoltaic, acronym, chirality, stoichiometric, cyclic voltammetry, quasi, pseudo, 

\section{job related paper}

OLED with \ce{MoO3} as charge generation layer. \cite{Kanno2006} The PI is stephen forrest, also a cofounder of udc oled.\footnote{http://www.umich.edu/~ocm/research.html}

stacked led, luminescence linearly increase with layer no at fix current density. hole transporting layer by MoO3, and electron transporting layer by Li.

OLED photovoltaic.\cite{Xiao2012a} functionalized squaraines as donor

excitation state management \cite{Zhang2012b}.
triplet-triplet annihilation (TTA) and singlet-triplet annihilation (STA). spin number of exciton is either 0 (singlet) or 1 (triplet).

TTA, bound electron-hole pairs, introducing a heavy metal in organic molecule to enhance the spin-orbital coupling, enabling triplet emitters.\cite{Zhang2013i} yet when operating at high current, the efficiency is decreased, and TTA is considered as a intrinsic limit.


(T. Tsujimura OLED Displays: Fundamentals and Applications John Wiley \& Sons Inc., Hoboken, New Jersey (2012))

why graphene on copper: a review article. \cite{Mattevi2011} research thrust, post CMOS fab tech. Exposure of hydrocarbon or evaporated carbon onto transition metal, the formation of graphite was surmised as a consequence of diffusion and segregation of carbon impurities from bulk to surface. carbon solubility in the metal and. 

The lack of control over layer No on Ni is partially attributed to the face that the segregation of carbon from the metal carbide upon cooling occurs rapidly within the Ni grains and heterogeneously at the grain boundaries. phase diagram of Ni and C reveals that the solubility of carbon in nickel at high T > 800 C form a solid solution. metastable phase \ce{Ni3C} promote the precipitation of C out of Ni. Carbon preferentially precipitates out at the grain boundaries of polycrystalline Ni subs so the thickness at graphite is higher than within grains. On Fe, due to the high affinity between Fe and C (\ce{Fe3C} is a stable carbide), the formation of $sp^2$ crystalline carbon film is suppressed. 

on copper (decomposition of methane gas at 1100 C), independent of heating or cooling rate. For copper, the 3d shell is filled, leading to less reactive configuration and weaker affinity towards Carbon. Cu can only form soft bonds with C via the charge transfer from the $\pi$ electron in  sp2 hybridized C to the empty 4s states of copper, as supported by the fact that copper does not form any carbide phase, and low C solubility. This low affinity and weak bond makes copper a true catalyst fro graphitic carbon formation. pre-treatment of copper foil, \ce{CuO}, \ce{Cu2O} removal by reducing annealing at 1000 C. 

epitaxial and lattice mismatch is present, 

\textbf{four-point probe}

sheet resistivity. a current source in an infinite sheet gives rise to the logarithmic potential
\[
\phi - \phi_0 = - \frac{I\rho}{2\pi}\log r,
\]
the potential for a dipole becomes
\[
\phi - \phi_0 = \frac{I\rho}{2\pi}\log r_1/r_2,
\]

When equal spacing probes are used, then potential difference between two inner points is 
\[
\delta\phi = V = \frac{I \rho}{\pi}\log2,
\]
so sheet resistivity $\rho$ is obtained as
\[
\rho = 4.5324V/I.
\]




\section{The physics of semiconductor}


quantum well devices exploit spatial quantization effects to increase the efficiency as well as alter the lasing threshold, and  novel semiconductor is made to emit light at wavelengths different from those possible when only bulk material is used.


The discontinuity in the CB and CB occurs at the interface. The corresponding potential discontinuity create a potential difference, forming a trap in which electron or hole can only have discrete energies.

a material of smaller bandgap is sandwiched between two layers of material of greater bandgap, or vice versa.

transmissivity coupled in multiple quantum well structure.

band structures=potential energy diagrams.

\textbf{junctions}
p-n homojunctions, p-n or n-n heterojunctions, metal-semi junctions, most important types: Schottky barriers, or ohmic contact.

chemical potential (Fermi level) as a measure of particle concentration.  Fermi level at 0K is equal to Fermi energy, which is defined as the energy of the topmost filled orbital. In equilibrium, the Fermi level $E_f$ is uniform throughout the material. n electron carrier, p holes carrier.

band diagram represents the electron's potential energy.

band bending: the electron energies are greater on the p side than on the n side, or the electrostatic potential is greater on the n side than on the p side, since potential $V = E/-q$.

The built-in potential for a homojunction is equal to the full band bending in equilibrium.

heterojunction: bandgap discontinuity, to solve the Poisson's equation for band bending.

Metal-Semi: metal cannot support any potential difference across it. Fermi level in Semi is pinned at the interface. Electron transfer from n-type semi to metal, leaving ionized donors behind.

ohmic contact forms if the work function of the metal is less than that of the semi. net flow of electron from metal into semi, no depletion layer forms.

metal-oxide-semi as MIS structures is capacitive in nature since no dc current flows under bias.

ch11.5 nonequilibrium conditions
quasi-fermi level $\phi_n$ or $\phi_p$.

\section{Gary semiconductor fabrication} 

photolithography a patter transferring process from mask to photoresist. clean room needed to remove dust particles, which could cause dislocation on an epitaxial film, low breakdown voltage in gate oxide, or short circuit. 

resolution, registration for effectiveness, throughput for efficiency, shadow printing where mask and wafer in direct contact: cons, dust could case permanent damage to mask, a small gap $d$ (10-50 $\mu$m ) used usually, and the minimium linewideth CD is roughly $\sqrt{\lambda d}$; and projection printing , resolution $l_m = \frac{k_1 \lambda}{NA}$, and depth of focus $\frac{k_2 \lambda}{NA^2}$. 193-nm using ArF excimer laser, and 157 nm using \ce{F2} excimer laser. 

365 nm probably used for \ce{LiNbO3} waveguide. 

resolution enhancement using phase shifting mask,  using the electric field of EM wave to chemically activate the photoresist. A $\pi$ phase change is obtained by using a transparent layer of $d = \lambda/2(n - 1)$ thickness. near-field diffraction, 

EUV 10 - 14 nm, C inner shell electron transition, 


\section{non-imaging optics}

black body radiation power density $S$ is about 1 \si{kW\per m^2}, the balanced temperature $T = \sqrt[4]{S/\sigma} = 364$ K, where $\sigma$ is Stefan-Boltzmann constant. 

ray tracing in vector form as $n^{\prime} r^{\prime} \times n = n r \times n$, invariant $na\theta$, 






















     


   %\glsaddallunused
    \appendix
    \chapter{math}
    \renewcommand{\bibname}{REFERENCES}
    \printbibliography

\end{document}

* remove all bold font, including figure captions, hypothesis env
* abstract; after skip line number is 2, title capital the first letter only
* section after skip none
* no external link in electronic version, turn off doi
* adjusting space between list of fig tbl, in tocloft afterskip command

`a ref <https://staff.washington.edu/fox/tex/uwthesis.shtml>`_
* the toc figure caption and others
* 

\usepackage[style=numeric-comp,
		    sorting=none,
		    date=year,
            hyperref=true,
            url=false,
            isbn=true,
            doi=false,
            backref=true,
            maxcitenames=2,
            maxbibnames=4,
            block=none,
            backend=biber,
            natbib=true]{biblatex}
% \usepackage[bibencoding=latin1]{biblatex}
\DeclareSourcemap{
    \maps[datatype=bibtex]{
        \map{
            \step[fieldset=abstract,null]
        }
    }
}
\AtEveryBibitem{\ifentrytype{article}{\clearfield{issn}}{}}
\AtEveryBibitem{\clearfield{month}}
\AtEveryCitekey{\clearfield{month}}

\DefineBibliographyStrings{english}{%
    backrefpage  = {see p.}, % for single page number
    backrefpages = {see pp.} % for multiple page numbers
}
% suppress 'in:'
\renewbibmacro{in:}{%
  \ifentrytype{article}{}{\printtext{\bibstring{in}\intitlepunct}}}
% document preamble
% removes period at the very end of bibliographic record
\renewcommand{\finentrypunct}{}
% removes pagination (p./pp.) before page numbers
\DeclareFieldFormat{pages}{#1}