
\chapter{test}
abc \si{\angstrom} abs

\section{level one}
Inline lists, which are sequential in nature, just like enumerated lists, but are
\begin{enumerate*}[label=\itshape\alph*\upshape)]
\item formatted within their paragraph;
\item usually labelled with letters; and
\item usually have the final item prefixed with `and' or `or',
\end{enumerate*} like this example.

\begin{hypothesis}
After 20 minutes sonication, the dispersion is milky. We measured the transmission spectra of this dispersion immediately upon sonication and after 40 hrs for gravity sedimentation.
\end{hypothesis}
\subsection{level two }
\Gls{cft}, \gls{cvd}
\subsubsection{level three}
After 20 minutes sonication, the dispersion is milky. We measured the transmission spectra of this dispersion immediately upon sonication and after 40 hrs for gravity sedimentation. The t0 line mainly arises from the scattering of flakes in the dispersion. There are two scattering mechanisms: Rayleigh scattering and Tyndall scattering. Rayleigh scattering, which occurs when particles are comparable to wavelength, is inversely proportional to the fourth power of wavelength; While Tyndall scattering, which occurs when the particles are larger, is inversely proportional to the square of the wavelength. To fully evaluate the true absorbance of the sample itself, one needs to decouple the scattering part from apparent absorbance. Generally we select one part of spectrum and assume it is only caused by scattering. Then a empirical dependence shown in Eq.~\ref{eq:sca2} is used as polynomial fitting model. test.

\begin{align}
Abs_{sca}  & = a\times \lambda^{n}  \label{eq:sca1}\\
\log{Abs_{sca}} & = \log{a} + n*\log{\lambda} \label{eq:sca2}
\end{align}

After least-squares fitting, the coefficients $a$ and $n$ can be used to estimated the scattering in other wavelengths. The fitting of n usually falls between -2 and -4. We examined the t40h line and found the scattering part is almost zero. Therefore we did not perform aforementioned fitting. This is not always the case. In chapter on the \ce{WS2} section, we do need to do so.

\begin{itemize}
\item s inversely proportional to the fourth
\item  inversely proportional to the fourth
\item s inversely proportional to the fourth
\end{itemize}

\begin{figure}[htb]
\centering
\subfloat[]{\label{fig:mosem1}\includegraphics[width=0.45\textwidth]{mosemsi_a}}\hspace{0.04\textwidth}
\subfloat[]{\label{fig:mosem2}\includegraphics[width=0.45\textwidth]{mosemsi_b}}
\caption[Representative morphologies of \ce{MoO3} on Si]{(a) Low magnification and (b) high magnification SEM images of representative depositions of \ce{MoO3} on Si for non-catalytic growth.}
\label{fig:mosisem}
\end{figure}
%[width=0.45\textwidth]

Rayleigh scattering and Tyndall scattering. Rayleigh scattering, which occurs when particles are comparable to wavelength, is inversely proportional to the fourth power of wavelength; While Tyndall scattering, which occurs when the particles are larger, is inversely proportional to the square of the wavelength. To fully evaluate the true absorbance of the sample itself, one needs to decouple the scattering part from apparent absorbance. Generally we select one part of spectrum and assume it is only caused by scattering. Then a empirical dependence shown in Eq.~\ref{eq:sca2} is used as polynomial fitting model.



piranha clean of FTO. 50ms switch.\cite{Scherer2012} nanoscale Kirkendall effect: the outward diffusion of metal cations are balanced by an influx of vacancies. For example, diffusion coefficient of Ni in NiO is higher than that of oxygen.
\subsection{EC windows and thin film batteries}

battery and ECD.\cite{Granqvist2012} electrolyte: PVB (poly vinyl buteral).
alternative materials and design: organic, Prussian Blue as EC materials, metal hydrides, suspended particle device, liquid crystal, electroplating,
challenges: large area nanoporosity, transparent conducting contact, electrolyte with good ionic conductivity and poor electronic conductivity, stable under UV; assembly and large scale manufacturing;
cathodic coloration:
anodic coloration:
The coloration mechanism: \ce{MO6} octahedrons lead to $e_g$ and $t_{2g}$ level and ion channelling.
ref54,60,65,66,200,209,


\ce{WO3} as cathodic and either polyaniline(PANI) or Prussian white (PW) as anodic electrochromic half cells. \cite{Heckner2002}

Characterization of ECD includes transmission measurement and associated EC calculation, charge-discharge time, current-time curve and the fitting of obtained data.

\begin{quote}
a viable electrochromic smart window must exhibit a cycling life time \textgreater $10^5$ cycles, corresponding to an operation life at 10 - 20 years.
\end{quote}


\citeauthor{Sella1998} studied the optical and structural properties of RF sputtered thin film of \ce{WO3} and \ce{VO2} for electrochromic devices. Ionic conductor was built using transparent polymer electrolyte, which was prepared from a solution of 1M \ce{LiClO4} in propylene carbonate which was mixed with methylmetharcylate (MMA). The main characteristics of polymer electrolyte were: viscosity at 25 \si{\degreeCelsius} $\approx$ 12920 Pa.s, conductivity $\approx 10^{-2}-10^{-4}$\si{\per\ohm\per cm},non-hygroscopic if PMMA concentration \textgreater 30\%. A specific counter-electrode layer was not used since the encapsulated polymer electrolyte processes a very high ion storage capacity.\cite{Sella1998}