\chapter{paper draft}
\section{Data Processing}

The SEM images are cropped using Adobe Photoshop for highlighting the features and fitting into text width. To obtain the dimension of specimen, scale bar was measured and compared to the size of interest using ImageJ\cite{Schneider2012} for SEM, TEM images and SAED pattern. It is worth noting that ImageJ could directly read the dm3 format generated in Gatan imaging software, which is more productive than the scale bar method.

Various spectra, be it from XRD, EDX, Raman, or UV-Vis, were redraw using Origin Pro 8.0. The peak fitting was performed using \emph{Fityk}, a nonlinear curve fitting program specialized in data processing from powder diffraction, chromatography, photoluminescence and photoelectron spectroscopy, infrared and Raman spectroscopy, \emph{etc}.\cite{Wojdyr2010}




\section{wo3}
Tungsten bronzes was coined by Wohler in 1837.\cite{Deb2008} \ce{Na_{x}WO3}

\ce{WO3} catalyst.\cite{Miyauchi2013}  potential of CB e more negative than redox potential of \ce{O2}-\ce{O2-} (-0.046 V vs NHE at pH 0). Z-scheme two photo absorption. photogenerated electron in CB of \ce{WO3} can reduce itself by formation of color centers.
photocatalytic applications in solar hydrogen generation and organic pollutant degradation.
A low recombination rate is preferred for high photocatalytic efficiency. The simultaneous migration of electrons and holes.

WO3: effective mass of bipolaron = 1.9me. for electron, for hole:
unzip nanotube. passivate BN ribbons with O and S; another player terrones psu.
applications: electrically controlled optical shutters for heat and light modulation, smart windows associated with solar cell to provide dynamical control of incoming illumination.

nanoscale Kirkendall effect: the outward diffusion of metal cations are balanced by an influx of vacancies. For example, diffusion coefficient of Ni in NiO is higher than that of oxygen.



\subsection{used}

\citeauthor{Chatten2005} also studied the oxygen vacancy in different phases of \ce{WO3}.\cite{Chatten2005}

SG and FG experiments remain essentially the same as that of OT growth, except that in SG additional tungsten powders were distributed onto the receiving substrate, and in FG more than one piece of substrate was employed. The modification is schematically illustrated in Fig.~\ref{fig:wogrowsf}. More details will be provided when it comes to the discussion.
% sg fg
\begin{figure}[htb]
\centering
\includegraphics[width=0.5\textwidth]{sg_and_fg.jpg}
\caption[\ce{WO3} NW growth: SG and FG]{\ce{WO3} NW growth: SG and FG. (a) Seeded growth with additional powders on substrate (b) Flow growth with multiple substrates}
\label{fig:wogrowsf}
\end{figure}

With four kinds of W powders and three layouts, we designed the experimental matrix as illustrated in Table.~\ref{tab:matrix}. The symbol $\times$ means this combination is covered in this work, and NA means otherwise.
% Tungsten powders growth design
\begin{table}[htb]
\centering
\caption{Tungsten powders growth design}\label{tab:matrix}
\begin{tabular}{lccr}
\toprule
 & Ordinary Transport & Seeded growth & Flow growth \\
\midrule
3N   &  $\times$ & NA &  NA   \\
3N5  &  $\times$ & NA &  NA   \\
4N5  &  $\times$ & $\times$ & $\times$ \\
5N   &  $\times$ & $\times$ &  $\times$ \\
\bottomrule
\end{tabular}
\end{table}

We will first present the OT growth results in section~\ref{sec:nawox}, and discuss the other two in section~\ref{sec:sgfg}.

We found the growth using 3N source show distinctive features in comparison to the rest. This mainly arise from the higher sodium concentration in 3N source than that in others. Therefore we focus on 3N source first, and move on to other latter.


% Na5 raman fitting
\begin{figure}[htb]
\centering
\includegraphics[width=0.6\textwidth]{naxwo_ramfit}
\caption[\ce{Na5W_{14}O_{44}} Raman fitting]{Multi-peaks Lorentzian fitting on two major peaks region of \ce{Na5W_{14}O_{44}}. The peaks sum height difference is caused by different baseline value adopted in each fitting.}
\label{fig:naworamfit}
\end{figure}

% W-O bond length
\begin{table}[htb]
\centering
\caption{\ce{W-O} bond length predication}\label{tab:nawobond}
\begin{tabular}{lccr}
\toprule
peak center & length (\AA) & peak center & length (\AA) \\
\midrule
694.6 & 1.900 &  808.6 &  1.821 \\
745.4 & 1.863 &  911.5 &  1.758 \\
764.4 & 1.850 &  933.0 &  1.745 \\
778.7 & 1.840 &   943.5 & 1.740 \\
788.4 & 1.834 &   965.4 & 1.728 \\
\bottomrule
\end{tabular}
\end{table}

An empirical formula to relate the Raman peaks and \ce{W-O} bonding lengths \cite{Hardcastle1995} is
\begin{equation}\label{eq:wobond}
\nu = 25823 \exp{-1.902\cdot R}.
\end{equation}
And the standard deviation of estimating \ce{W-O} bond distance from Raman stretching wavenumber is $\pm0.034$\AA.
The observed Raman peaks of \ce{Na5W_{14}O_{44}} phase lies at 965, 943, 913, 808, 786, 778, 765, 695 and 107 cm. Multi-peaks Lorentzian fitting is preformed to precisely determine the central maximum. Good fitting is obtained, as shown in Fig.~\ref{fig:naworamfit}. We then calculated \ce{W-O} bond distance using Eq.~\ref{eq:wobond}, as illustrated in Table~\ref{tab:nawobond}. The predicted \ce{W-O} bond length comply very well with the crystallographic value of \ce{Na5W_{14}O_{44}} phase.\cite{Triantafyllou1999a} The 107 peak probably is caused by \ce{Na-O} bond.


\subsection{not used}

This approach is essentially the same as previous studies where tungsten foils was used as substrate.The advantage of this method is W powder cost much less than a whole piece of foil, and provide more surface area than W foil in the same volume.

The FG method utilizes the flow dynamics of our CVD apparatus. We will discuss both of them in details.
We would like to discuss a little more on the dispersion of \ce{WO3} NWs, since the stability of \ce{WO3} dispersion is important for its applications. \citeauthor{Kozan2008} studied the stability of \ce{WO3} in various polar solvents such as acetone, isopropanol(IPA), ethanol, 1-methoxy-2-propanol (1M-2P) and N-dimethylformamide(DMF).\cite{Kozan2008} High power ultra-sonication was applied for about 2 minutes followed by a low power ultrasonic bath for 15 minutes. The the solvent was left to sedimentation. Samples was drawn out with pipette and transferred to clean substrate for observation. It is suggested that bundled NWs aggregates in 1M-2P seem to breakup at the outer branches to a small extent and turn into slightly more compact aggregate, providing well-dispersed stable suspension. Subsequently, this suggests we could harvest the as-grown nanowires by a solution ultra-sonication method. Generally there will be metallic tungsten in the core region of powder particles. The separation of NWs could be done via the route reported by \citeauthor{Kumar2008}, where separation by sonication in 1-methoxy 2-propanol followed by gravity sedimentation.\cite{Kozan2008}


The energetic sources are ion bombardment, electron beam, laser ablation, and combustion flame\cite{Rao2011}.

The sol–gel process is a well-known, intensively studied wet-chemical technique that is widely used in materials synthesis. This method generally starts with a precursor solution (the ``sol") to form discrete particles or a networked gel structure. During the course of gelation (aging process), various forms of hydrolysis and polycondensation take place.

In addition, doped \ce{WO3} was also demonstrated

We do not discuss tungsten oxide hydrates (\ce{WO3.nH2O}) in this work since the product of thermal CVD approach is not plagued with this complexity. It's necessary, however, to deal with hydrated \ce{WO3} in the liquid synthesis routes, as indicated in Section.~\ref{sec:woxgrowth}.


Nonstoichiometric tungsten oxides \ce{WO_x} (i.e., \ce{WO_{2.92}}, \ce{WO_{2.87}}) are known as Magn$\acute{e}$li phases.


Theoretical computation of electronic band structures for \ce{WO_x} proves difficult due to the aforementioned phase transition. oxygen deficiency, structure change, electronic properties vary according.

the ubiquity of \ce{WO6} octahedra is essential for not only the optical properties but the ability to insert and extract ions in the EC oxides, due to the tunnels in three dimensions serving as path for transport of small ions. The intercalation of hydrogen or alkali ions into \ce{WO3} created electron donor level. By absorbing the red part of incident spectrum, electrons at donor level make transition to the conduction band, causing the blue coloration in \ce{H_xWO3}.

its one dimensional (1D) nanostructure has attained intensive research efforts in recent years due to the potential applications in advanced nano-electric and nano-optoelectronic devices.

\begin{quote}
a viable electrochromic smart window must exhibit a cycling life time \textgreater $10^5$ cycles, corresponding to an operation life at 10 - 20 year.
\end{quote}

\subsection{to be used}

%wo3 band gap table
\begin{table}[htb]
\centering
\caption{Tungsten oxides band gap }\label{tab:wo3eg}
\begin{tabular}{lccr}
\toprule
Phase & Experimental (eV) & Theory (eV) &  \\
\midrule
amorphous \ce{WO3} & 3.2  & NA &    \\
monoclinic bulk \ce{WO3} &  2.6   & 1.73\cite{Migas2010a}  &    \\
tetragonal bulk \ce{WO3} &     & 0.66 \cite{Migas2010a}&    \\
\bottomrule
\end{tabular}
\end{table}

W plasma oxidation.\cite{Romanyuk2005} 200nm W coating on Si (100) sub, temperature at RT, 390, and 490 C, oxygen pressure 0.5 Pa, oxidation time for 10 to 3600 s. The resultant thickness of \ce{WO3} at RT  and time of 10 s and 3600 s is found to be 0.2 nm and 11 nm respectively.

\[
 d = d_0 exp(kt)
\]

after fitting, $d_0 = 0.19777$ nm, $k = 0.00112 $ s$^{-1}$, so to oxidize 1 nm W coating completely, oxygen plasma time is 1450 s at 0.5 Pa partially pressure; 2 nm for 2000 s.

\ce{W_{18}O_{49}} Raman, IR shielding.\cite{Guo2012} \cite{Guo2011}
broad peak between 750-780 cm-1.

\ce{WnO_{3n-1}} NPs. \cite{Frey2001}

WO3-x raman info, the encapsulated WOx core has been investigated in depth. several stable phase could occur, including \{001\} CS phases, \{103\} CS phases. no evidence of $\gamma$-\ce{W_{18}O_{49}} phase is found. The cross-section ($\sigma$) for Raman scattering and the absorption coefficient of the WS2 layers are much larger than those of the suboxide phase encapsulated inside.

WO2:168(w),189(w), 286(vs), 345(w), 423(w), 479(m), 512(m), 599(m),
617(m) cm-1, and a mode at 781(s) cm-1 which tails to higher energies (w-weak; m-medium; s-strong; vs-very strong).

W5O14: 215, 264, 325, 349, 418, 425,707, and 800 cm-1, 900 maybe

WO3: 808, 719, 275;

W3O8: 870;

no 950 peak indicates no hydrated phases.

\ce{WO_{3-x}} Raman peak at 778. \cite{Deb2007}

% wo3-x phases
\begin{table}[htb]
\centering
\caption{List of \ce{WO_{3-x}} phases}\label{tab:wo3xphase}
\begin{tabular}{lccccc}
\toprule
&&&\multicolumn{3}{c}{Lattice constants \AA} \\
\cmidrule(l){4-6}
 Symbol    & PDF  & Phase & a & b & c   \\
\midrule
\ce{W18O49}  & 00-036-0101 & monoclinic & 18.324 & 3.784 & 14.035  \\
$\delta$-\ce{WO3}   & $-50 \sim 17$  & triclinic & 7.309 & 7.522 & 7.686  \\
$\gamma$-\ce{WO3}   & $17 \sim 330$  & monoclinic I & 7.306 & 7.540 & 7.692  \\
$\beta$-\ce{WO3}    & $330 \sim 740$  & orthorhombic & 7.384 & 7.512 & 3.846  \\
$\alpha$-\ce{WO3}   & $> 740$  & tetragonal & 5.25 & NA & 3.91  \\
$h$-\ce{WO3}        &  $<400$  & hexagonal & 7.298 & NA & 3.899  \\
\bottomrule
\end{tabular}
\end{table}



\section{moo3}
\subsection{not used}

\subsection{to be used}


10nm MoOx as hole extraction layer (HEL). Without HEL, Holes accumulates at QD/anode interface, causing increased recombination rate. With HEL, hole diffuse into this layer, reducing the recombination. 

The molecular unit in crystal exhibits different vibrational frequencies from that in solution or gas phases.

\ce{Na2Mo4O_{13}} phases monoclinic at RT, solid solubility of \ce{Na2MoO4} in solid \ce{MoO3} is high. vapor pressure of \ce{Na2Mo4O_{13}} over \ce{MoO3}.

melting point of \ce{Na2Mo4O_{13}}
Mp: \ce{Na2Mo2O7} 960K

\ce{MoO3} vapor pressure:

The real phase diagram is the one between \ce{Na2Mo4O_{13}} and \ce{MoO3}.
the growth temperature could be much lower than the eutectic point.

KI MP:  681
NaI MP: 661

NaOH Raman peaks lie at 3633 cm. \cite{walrafen2006} Raman scattering of \ce{Na2SiO3} exhibit major peak at 966 and 589 cm.\cite{Richet1996}

visibility of mica thin layer on \ce{SiO2}-Si. \cite{Castellanos-gomez2011} 1.5\% contrast is almost at the threshold of human eye sensitivity.  When the thickness is below 60nm, Raman could not detect mica.

\ce{MoO3} SWNT by hydrothermal method.\cite{Hu2008a} Raman spectra is off compared to single crystal \ce{MoO3}.  Van der Waals interaction and layered structure make NT possible.


The refractive index is between $ 2.2\sim 2.4$ along $a$ axis, and $ 2.5\sim 2.8$ along $c$ axis. Comparison to the low frequency dielectric constant ($18.0\pm1$) implies a primarily ionic bonding in \ce{MoO3}.\cite{He2003}
\citeauthor{Wang2013c} changed the NW growth direction by changing the shape of droplets; \citeauthor{Biswas2013} reported enhanced aspect ratio of Ge NWs from bimetallic alloy.

We first repeated the previous growth on ITO/glass, and then found similar deposition phenomena on glass and mica substrates. We suspected that the morphology difference on the substrates employed in this study arises from the compositions difference. Both mica and glass contains alkaline elements which are absent in Si and \ce{SiO2}/Si substrates. We primarily verify this hypothesis by utilizing \ce{NaOH} treated Si substrates and found that the overall morphology almost reproduces that on ITO and mica substrates. We then propose a new growth mechanism based on the VLS process to explain the deposition of \ce{MoO3} on alkaline-ions-containing substrates.

ITO versus NaOH images.

According to the phase diagram shown in Fig.~\ref{fig:pd}, \ce{Na2MoO4} - \ce{MoO3} phase above 800 \si{\degreeCelsius} are \ce{Na2MoO4}(l) + \ce{MoO3}(l), More discussion should follow $\ldots$.


% droplet position statistics
\begin{figure}[htb]
\centering
\includegraphics[width=0.6\textwidth]{droplets_sta}
\caption[Droplet position statistics]{Droplet position statistics. On-tip and not-on-tip counts variation with respect to growth time. }
\label{fig:mo3dropsta}
\end{figure}


Nanotower is a less common morphology in the structures of nanomaterials.\cite{Kharissova2010} We have not found report on  \ce{MoO3} nanotower structures, although there are several studies on \ce{In2O3},\cite{Jean2010,Yan2007}, GaN\cite{Xiao2012} and ZnO\cite{Zhang2013c}. \citeauthor{Jean2010} investigated the growth mechanism of \ce{In2O3} nanotowers and attributed it to the periodical axial and continuous lateral growth interaction, where the former was first guided by Au-In alloy liquid, and subsequent by self-catalytic In droplets (MP 156C) while the latter due to VS process. In contrast, \citeauthor{Yan2007} observed similar \ce{In2O3} nanotower growth and proposed that the axial growth is controlled by Au-catalytic VLS process. \citeauthor{Xiao2012} reported the GaN nanotower grown on Ni-coated Si(111) substrates and explained the growth as asymmetric self-copy process based on VLS mechanism. \citeauthor{Zhang2013c} studied the growth of ZnO nanotower and provided a competitive model between axial and lateral growth controlled by Zn vapor to explain the as-synthesized structures. (When ZnO and carbon mixture was used as source instead of ZnO alone, nanorods array results, presumably due to the stable supply of Zn vapor owing to the gentle carbothermal reduction.) Therefore the fluctuation of reagents is probably responsible for tapering and periodical modification, which is absent in the coin roll style growth of \ce{MoO3} tower on Si. The occasionally observed tapering or enlarging on \ce{MoO3} tower can be attributed to the cooling down growth.

We divide the growth into two stages according to the emerging time of long belts and towers. In first stage the growth is dominated by lateral growth, as illustrated in . In the second stage, \ce{NaxMoO3} is evaporated and recondensed in the low temperature region, and some droplet will form, and when at proper locations, introduce secondary VLS growth.

In first stage, the Na-Mo-O liquid is formed and have much more higher absorption rate than that of bare substrate. The absorbed species will diffuse along the surface and inside the liquid. It was difficult to estimate which diffusion path one is faster. 


\textbf{open questions}
\begin{itemize}
\item Is the liquid consumed? if so, how? : evaporation
\item How the size of liquid is related to the morphology evolution?
\item Possible doping level of alkaline ions
\item other lateral growth mode such as Ni-W-S system
\item VLS-VS interaction, or possible solution-liquid-solid, VSS,
\item the interpretation of phase diagram
\item the overall molar amount of NaOH and MoO3 deposition
\item Young's parameters, contact angle informations, Gibbs free energy calculation
\item catalyst composition variation during growth
\item \ce{Na2Mo4O_{13}} phases, solid solubility of \ce{Na2MoO4} in solid \ce{MoO3} is high.
\item vapor pressure of \ce{Na2Mo4O_{13}} and \ce{MoO3}.
\end{itemize}


As shown in Fig, we found several solid clusters with diameter about 1 $\mu m$ on the top part of long belts. According to the \ce{Na2MoO4}-\ce{MoO3} binary compounds phase diagram, we expect the terminating solid clusters have a Na:Mo ratio close to 0.6. The experimental results match well with this predictions. EDX analysis reveal the presence of Na on the solid cluster but not at about 1$\mu m$ away from previous location.\footnote{the sensitivity of our instrument is estimated to be 1\% atomic level} Quantitative atomic ratio derived from the solid cluster is not close to the one predicted by phase diagram probably due to the high Mo content in adjacent area. However, XRD examinations on a short time growth sample revealed a possible \ce{Na2Mo4O13} phase(PDF 028-1112), providing an indirection evidence of the droplet composition. We also suspect that the growth will become different below 500 \si{\degreeCelsius} since no liquid would remain. Substrate placed at low temp end show.
We can also deduce that other sodium compounds can be employed as substrate treatment agent. This argument is confirmed by \ce{Na2CO3} and \ce{KI} assisted growth.


\section{ws2}
\subsection{Seeded Growth of Tungsten Oxide NWs}
With the oxidation experiments study in Sec.~\ref{sec:woxd}, favorable conditions for local growth of NWs were extracted to perform seeded growth. 

\begin{figure}[htb]
\centering
\subfloat[]{\label{fig:sga}\includegraphics[width=0.4\textwidth]{wox_sg_a.jpg}}\hspace{0.04\textwidth}
\subfloat[]{\label{fig:sgb}\includegraphics[width=0.4\textwidth]{wox_sg_b.jpg}}
\caption[Characterization of seeded growth \ce{WO3}: SEM]{Characterizations of seeded growth. (a) SEM graphs of \ce{WO3} NWs on \ce{SiO2/Si} substrate. (b) A high magnification view showing uniform NW growth and close-up view of one NW. }
\label{fig:woseedsem}
\end{figure}

As shown in Fig.~\ref{fig:sgb}, dense NWs array was obtained on tungsten powder seeds with individual wires of length up to 5 $\mu$m and diameter about 50 to 200 nm, according to the estimation made in the close-up view. Each tungsten powder stood as independent growth site (Fig.~\ref{fig:sga}) with island-layer growth on the substrates, a common feature without using tungsten powder as seed under current experimental conditions. It was occasionally observed that NWs growth was initiated adjacent some tungsten powders. This phenomenon was correlated to the local trap of vapor flow since it was more often found among the enclosed area by tungsten powders. It is also found that the diameter of NWs decrease as the distance between powders and upstream edge increases. This is a combination effect of lower temperature and reduced \ce{WOx} growth species supply. Similar phenomena were observed in other studies. \citeauthor{Thangala2007} reported that a decrease in NW density with increasing substrate temperature, and an increase of NW density with increasing partial pressure of oxygen.\cite{Thangala2007}

% seeded edx 
\begin{figure}[htb]
\centering
\includegraphics[width=0.5\textwidth]{wo3_seed_edx}
\caption[Composition analysis on seeded growth \ce{WO3} NWs]{EDX spectroscopy on seeded growth \ce{WO3} NWs.}
\label{fig:woedx}
\end{figure}
Energy-dispersive X-ray spectroscopy (EDX) analysis on the seeded growth \ce{WO3} NWs is shown in Fig.~\ref{fig:woedx}. Only W and O elements were detected on the NW array. The background level from 3 to 8 keV was a manifestation of the continuous components of W X-ray spectrum. 

% sg raman xrd
\begin{figure}[htb]
\centering
\subfloat[]{\label{fig:sgxrd}\includegraphics[width=0.45\textwidth]{xrd_cs_before}}\hspace{0.04\textwidth}
\subfloat[]{\label{fig:sgram}\includegraphics[width=0.45\textwidth]{wox_raman_1}}
\caption[Characterization of seeded growth \ce{WO3}: XRD and Raman]{ (a) XRD pattern of as-prepared sample indicating the \ce{WO3} phase and the presence of metallic core. (b) Raman spectrum on NWs region showing the feature of \ce{WO3}.}
\label{fig:woseedxrd}
\end{figure}

Fig.~\ref{fig:sgxrd} is the XRD spectrum of one typical sample. The peaks under circular symbol were identified to be the monoclinic \ce{WO3} phase (ICDD PDF 01-083-0950, \emph{a}=7.30084 \AA, \emph{b}=7.53889 \AA, \emph{c}=7.6896 \AA, $\beta$=90.89$^\circ$), while the peak under the triangular symbol was indexed to cubic tungsten phase (ICDD PDF 04-16-3405, \emph{a}=3.157 \AA), in agreement with the EDX analysis (Fig.~\ref{fig:woedx}). This means that during the \ce{WO3} seeded growth of 4 h heating at 1000 \si{\degreeCelsius}, the tungsten powder in downstream low temperature region (600-700 \si{\degreeCelsius}) is not entirely oxidized. Micro-Raman scattering spectroscopy was performed on the as-synthesized sample as well. During Raman examination, the laser spot was carefully focused onto the NWs on powders and several inspections on different positions were observed to ensure the reproductivity of spectra data. As shown in Fig.~\ref{fig:sgram}, five distinct bands were well resolved, with peaks located at 131, 265, 327, 711 and 803 \si{cm^{-1}}, respectively. This pattern was typical features of \ce{WO3}, consistent with previous study.\cite{Salje1975a,Dixit1986} The high background level probably arises from the metallic core.

% sg tem
\begin{figure}[htb]
\centering
\includegraphics[width=0.9\textwidth]{JAP-2column_Fig5major.jpg}
\caption[Characterization of \ce{WO3}: TEM]{TEM graphs of as-prepared NWs. (a) TEM image of one nanowire with diameter about 40 nm. (b) HRTEM images showing the spacing is 0.38 nm, corresponding to (002) plane distance.}
\label{fig:woseedtem1}
\end{figure}

TEM specimen was prepared by using carbon grid to slightly scratch the as-grown sample. Fig.~\ref{fig:woseedtem1} shows the feature of majority NWs. The growth direction is determined to be perpendicular to (002) plane. The streaking in SEAD pattern presumably arises from stacking defaults. The author also find some NWs exhibit high crystalline quality, as revealed by the TEM analysis in Fig.~\ref{fig:woseedtem2}. The NW grew normal to (002) plane with a measured lattice spacing of 3.79 \AA, which is favorably compared to the XRD peak at $23.07^\circ$ (7.7103 \AA). The sharp SEAD pattern and clear phase contrast in HRTEM demonstrated are both strong evidence of good crystallinity. This formation indicated current growth parameters have promising potential to obtain highly crystalline \ce{WO3} NWs in large scale. 

% sg tem
\begin{figure}[htb]
\centering
\includegraphics[width=0.9\textwidth]{JAP-2column_Fig5minor.jpg}
\caption[Characterization of \ce{WO3}: TEM cont]{TEM graphs of as-prepared NWs. (a) TEM image of one nanowire, the diameter is about 70 nm. (b) HRTEM images showing the spacing is 0.379 nm, corresponding to (002) plane distance.}
\label{fig:woseedtem2}
\end{figure}

In regarding to the formation of NWs on tungsten powder itself, this study assumes the driving force is related to interfacial strain between W and \ce{WOx}. Oxidation of tungsten proceed slowly at room temperature and an oxide layer of 100 \AA was found on the surface of tungsten foils.\cite{Warren1996} The tungsten powder used in current study would be covered by a thin oxide layer as well. During oxidation, different oxidation rates exist for different crystallographic orientations on the tungsten powder. Oxidation occurring at boundaries and defects are preferred thermodynamically.\cite{You2010} Compressive strain will gradually accumulate at the tungsten oxide/tungsten interface, which might limit the diffusion rate of oxygen at temperature lower than 500 \si{\degreeCelsius}.\cite{tungsten1999} At elevated temperature, cracks will primarily occur, as observed in Fig.~\ref{fig:pdtemp}. When heated up, tungsten and the oxide shell will probably relax the strain by converting into substoichiometric NWs, a similar process as suggested by \citeauthor{Klinke2005} in the chemically induced strain growth of tungsten oxide NWs.\cite{Klinke2005} It is worth noting that tungsten oxide nanowires could also formed when \ce{WO3} is reduced.\cite{Sarin1975} The elongation of \ce{WOx} is thermodynamically favorable during the conversion from metallic tungsten to tungsten oxide as well. Local evaporation-condensation process might also contribute to the formation of NWs on tungsten powder.

The enhanced yield of NWs obtained via seeded growth could be explained by a vapor-solid (VS) mechanism. External supply of growth species will condense onto the powders and substrate simultaneously, promoting the elongation of NWs on power as well as resulting island-layer growth on substrate. The local NW density in oxidation experiment was much higher than that of seeded growth. It is reasonable to presume that during seeded growth, several NWs in a small region on powder will coalescence, as evidenced by the bundled structures. At last, the author would like to point out that when low purity tungsten powder (3N) was used as source, sodium tungsten oxide nanowires were found to be dominant in the final product. The details have been published.\cite{Sheng2014} It seems surprising that when 3N powder was used as seeds, only \ce{WO3} nanowires were obtained. This result was attributed to the lower temperature and significantly reduced amount of 3N powder used in the seeded growth, compared with the conditions used in Ref.\cite{Sheng2014}. The source material in seeded growth is not limited to high purity tungsten powder. Instead, any material that could produce appropriate growth vapor could be employed, indicating the versatility of this approach.

\subsection{Optimization of W powder Oxidization}\label{sec:woxd}

Understanding the oxidation of tungsten powder is the key to obtain high yield in seeded growth. Oxidation of tungsten have been investigated under diverse conditions, such as at elevated temperature (\textgreater 1100 \si{\degreeCelsius} ) and oxygen pressure on the order of Torr,\cite{Base1965} and at temperatures ranging from 20 to 500 \si{\degreeCelsius} under atmosphere pressure.\cite{Warren1996} \ce{WO_x} NWs were readily found when tungsten (foil, wire, or powder) is oxidized under various conditions.\cite{Zhu1999,Karuppanan2007,Hsieh2010} However the study on tungsten powder oxidation behavior between intermediate temperature range and under low pressure is still rare. This work studied the oxidation of tungsten powders with diverse size within temperature range from 500 to 1000 \si{\degreeCelsius} and under several mTorr oxygen partial pressure. Using tungsten powder as seed, the author further illustrate an economic approach to obtain large yield of \ce{WO3} nanowires at relatively lower temperature than previous efforts. It is demonstrated that there is an optimal tungsten powder size under our experimental conditions for seeded growth. This will provide some insight on the role of tungsten powder as source material in CVD growth of \ce{WO_x}.

Commercial available tungsten powders with different size are usually associated with purity variation as well, as already listed in Table.~\ref{tab:ch5pre}. The dimensions of tungsten powder were obtained by measuring the average size in SEM graphs. A systematic investigation was performed on the oxidation behavior of tungsten powder to evaluate the temperature effect, size-dependence and influence of oxygen partial pressure.
% seed optimal
\begin{figure}[htb]
\centering
\includegraphics[width=0.6\textwidth]{JAP-2column_Fig1.jpg}
\caption[W powder oxidation: temperature effect]{SEM graphs of 99.9\% (3N) tungsten powder oxidization at different temperatures of a) 500 \si{\degreeCelsius}, b) 600 \si{\degreeCelsius}, c) 650 \si{\degreeCelsius}, d) 750 \si{\degreeCelsius}, showing the optimal temperature for local formation of nanowires is between 600--650 \si{\degreeCelsius}. Oxygen flow rate is 1 sccm.}
\label{fig:pdtemp}
\end{figure}

Fig.~\ref{fig:pdtemp} illustrated the effect of temperature on the morphological change and surface nanowires formation of 3N powder. At 500 \si{\degreeCelsius}, most tungsten powder retained its original shape and a layer of tiny dense NWs begun to grow. When temperature was increased to 600 \si{\degreeCelsius}, 3N powder started to crack with longer NWs on the isolated surface. Further increase of temperature lead to irregular shapes of tungsten power and aggregation of NWs, giving rise to the nanorods and bunched or bundled structures. It could be determined from the morphology variation that the optimal seeded growth temperature for 3N powder was in the range of 600 to 650 \si{\degreeCelsius}.
% seed optimal
\begin{figure}[htb]
\centering
\includegraphics[width=0.6\textwidth]{JAP-2column_Fig3.jpg}
\caption[W powder oxidation: size effect]{SEM graphs illustrating the oxidization of four different size of tungsten powders at 600~\si{\degreeCelsius} and 1 sccm oxygen flow. a) 17 $\mu m$, b) 32 $\mu m$, c) 3.3 $\mu m$, d) 1.5 $\mu m$.}
\label{fig:pdsize}
\end{figure}

Fig.~\ref{fig:pdsize} depicted the oxidation of different sizes of tungsten powder under the same experimental conditions. In contrast to the morphology of 3N powder shown in Fig.~\ref{fig:pdtemp}, 3N5 powder surface is primarily covered with sub-micro particles as well as some short tiny NWs, whereas 4N5 and 5N powder were thoroughly oxidized, showing branched flowers feature. This dramatic difference could be explained in terms of surface energy and oxygen diffusion. With smaller dimension, the increased surface-to-volume ratio and short diffusion path both lower the energy barrier of oxidation.\cite{tungsten1999} It was logical to deduce that higher temperature or increased oxygen level might favor the NWs formation on 3N5 powder. When it comes to seeded growth, however, the powder size distribution was an important factor to give uniform NWs deposition. Since the size distribution of 3N powder is more uniform than that of 3N5 powder, the author employed the former as seeds.
% seed optimal
\begin{figure}[htb]
\centering
\includegraphics[width=0.6\textwidth]{JAP-2column_Fig2.jpg}
\caption[W powder oxidation: oxygen pressure]{SEM graphs of 3N tungsten powder oxidization at 600 \si{\degreeCelsius} under different rates of oxygen flow: a) 1 sccm, b) 2 sccm, c) 3 sccm, d) 10 sccm. The oxygen partial pressures were 13 mTorr, 23 mTorr, 32 mTorr, and 82 mTorr, respectively with background pressure subtracted.}
\label{fig:pdoxy}
\end{figure}

Fig.~\ref{fig:pdoxy} depicted the morphology change of 3N powder with respect to varied oxygen partial pressure. When the oxygen flow is lower than 3 sccm, 3N powder almost stayed as the same, with cracks separating the dense layer of NWs. When oxygen flow is increased to 10 sccm, the 3N powder exemplified an enlarged version of that for 4N5 or 5N powder under 1 sccm oxygen flow. This observation again supported the surface energy explanation.

\subsection{used}

As the experimental setup for direct tensile tests of nanotubes is state-of-the-art,\cite{Tang2013} the application of tensile stress on 2D TMD systems is rather difficult due to the excellent lubricating properties of these materials.

\citeauthor{Zhang2013e} investigated the shear (C) and layer breathing mode (LBM) in the low frequency region of \ce{MoS2}.\cite{Zhang2013e} Even layer \ce{MS2} belong to point group D$_{6h}$ with inversion symmetry, while odd layer \ce{MS2} correspond to D$_{3h}$ without inversion symmetry. The excitation wavelength is 532nm from a diode-pumped solid-state laser. A power$\sim$0.23mW is used to avoid sample heating.

reaction mechanism of \ce{MoO3} to \ce{Mo2S}.\cite{Weber1996}

\citeauthor{Ling2014} studied the role of seeding promoters in CVD growth of FL \ce{MoS2}.\cite{Ling2014} PTAS treated substrates provided nucleation site and thus enable uniform deposition of \ce{MS2}.  This enhancement perhaps arise from the \ce{K+} ions.

\citeauthor{Splendiani2010} reported the PL in monolayer \ce{MoS2}.  Calculation indicated the indirect gap become larger when thinning, while the previous direct one almost stays as the same, the value is about 1.85eV (direct gap).\cite{Splendiani2010}

thermal decomposition of (NH4)2MoO2S2 and intermediate product MoOS2 was studied. application: hyfrodesulfurization in refinery \cite{Weber1996}

\cee{MoCl5 + 1/4S8 + 5/2H2 \rightarrow MoS2 + 5HCl} \cite{Stoffels1999}

A direct gap of $\sim 2eV$ at the corners of BZ is formed in 1L \ce{WS2}, Growth on bottom piece show the multiple domain flakes occurs at initial stage of the growth, starting from \ce{WO3} particles.\cite{Cong2013}


\subsection{to be used}

Exfoliated WS2 few layer PL.\cite{Zhao2012} excitonic absorption peaks A and B arising from direction transition at K point are found around 625nm (1.98eV) and 550nm, respectively, which are in agreement with results from bulk layers. The A, B excitons difference was a result of strong spin-orbital coupling. Relative PL quantum yield of WS2 between 1L and 2L is on the order of 2. The FWHM of WS2 peak is about 75 meV. wider than thermal energy at room temperature,

Electro microscopy on stacking sequences of WS2 NT.\cite{Houben2012} The probability of parallel stacking is about 30\%. a metal-semi superstructures. In NT, the layers are slightly shifted with respect to each other due to the constraints, thus the stacking is not exactly as pure phases of 2H(prismatic antiparallel), 3R(prismatic parallel) or 1T (octahedral parallel) with their perfect translational symmetry.

chevron pattern, contradictory, contradicting, Debye scattering model for XRD.

\begin{quote}
hexagonal polytype 2Hb with two molecular layers (spacegroup P63/mmc) and a rhombohedral polytype 3R with three molecular layers per unit cell (space group R3m), a high pressure polytype that is stable in plane geometry at pressures above 4 GPa. The two prismatic phases are semiconducting, and the octahedral one is metallic-like.
1T phase may be the result of a transformation from the 3R to the 2H phase by an intermediate 1T phase that is trapped by fast quenching
\end{quote}

aberration corrected TEM is used.

HRTEM on WS2 NT.\cite{Sadan2008} negative spherical-aberration imaging (NCSI). NCSI condiction were achieved at a negative spherical aberration of -20um balanced by an overfocus of +17 nm. Focal series reconstruction to retrieve the phase of electron exit plan wavefunction. Zigzag, armchair revealed.

In centrosymmetric crystals, the vibrational modes must either have even (Raman-active) or odd (IR-active) parity under inversion, which is known as rule of mutual exclusion. When this symmetry is broken, some modes may be simultaneously IR and Raman active.

inelastic neutron scattering to study the non-zone center LA mode. Zone-edge scattering can occur due to zone-folding process. The formation of superlattice could activate formerly inactive zone-edge phonons. The folding of BZ along $\Gamma-M$ would cause the M point to coincide with $\Gamma$ point, so LA(M) phonons would become Raman active in a first-order process.

the average distance traveled by an excited electron-hole pair before combination $l=\nu_F/\omega_D=4nm$.

Confocal Raman spectrometer:to obtain Raman spectrum in a specific depth of sample. Edge filter to cut off Rayleigh emission.

resolution $d= 1.22 \lambda/NA$,

Light Scattering in Solids II,. Springer, Berlin, 1982

influence of core WOx, Raman scattering by plasma-LO coupling to determine carrier concentration. measure resonant cross sections in absolute units.

disorder-induced light scattering, Van Hove critical points,
In resonant second-order scattering:
overtone: the same phonon,
combination: two different phonons;

\[
\frac{\ud\sigma}{\ud\Omega}= \omega_s^4 cm^6 Sr^{-1}
\]

scattering volume V in number of unit cells can be considered as one big molecule.

a single nanowire tends to minimize its surface. 2D isoperimetric quotient or circularity $C= \frac{4\pi A}{P^2}$, where A is area and P is perimeter of the cross-section.


