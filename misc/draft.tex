\chapter{paper draft}


The KM two-flux model has been extensively used to describe the optical properties of inhomogeneous materials that consist of scattering and absorbing particles in a matrix.\cite{Vargas1997} There are some limitation in the assumptions made in original KM model, e.g., the distribution of scattering on the optical path. And revised KM models have been proposed to eliminate these limitations.\cite{Yang2004}  

\citeauthor{Morandi2005} studied the absorbance and diffuse reflectin of \ce{MoO3} and \ce{WO3} thin films using FT-IT and UV-Vis spectroscopy. The observed absorption in visible and middle IR region were attributed to oxygen defects induced donor levels.\cite{Morandi2005}

The ratio between absorption and scattering, not the absolute value of each quality, determines the reflection features.
\section{iffalse}

\iffalse
An eucentric specimen stage is used, thus observation area remain fixed when tilting the specimen. The pressure in SEM chamber is on the order of $10^{-3} \sim 10^{-4}$ Pa, which is usually maintained by a diffusion pump, or turbo molecular pump when oil-free operation is needed. For a field emission electron source, a sputter ion pump becomes necessary due to the high vacuum requirement. 

The SEM instrument used in this study is JEOL JSM-6480 and EDX attachment from Oxford Instrument INCA. Typical observation conditions are listed as following:

\begin{enumerate}
\item SEM
\begin{itemize}

\item Acceleration voltage: 10 kV
\item Working distance: 10 mm
\item Scanning time: 80 s
\end{itemize}
\item EDX
\begin{itemize}

\item Acceleration voltage: 20 kV
\item Working distance: 10 mm
\item Dead time: $20\sim30$\%
\end{itemize}
\end{enumerate}


\begin{quotation}
Since the intensity of characteristic X-rays is proportional to the concentration of the corresponding element, quantitative analysis
can be performed. In actual experiment, a standard specimen containing elements with known concentrations is used. The
concentration of a certain element in an unknown specimen can be obtained by comparing the X-ray intensities of the certain element
between the standard specimen and unknown specimen. However, X-rays generated in the specimen may be absorbed in
this specimen or excite the X-rays from other elements before they are emitted in vacuum. Thus, quantitative correction is needed.
In the present EDS and WDS, correction calculation is easily made; however, a prerequisite is required for this correction. That is,
elemental distribution in an X-ray generation area is uniform, the specimen surface is flat, and the electron probe enters perpendicular to the specimen. Actually, many specimens observed with the SEM do not satisfy this prerequisite; therefore, it should be noted that a quantitative analysis result might have appreciable errors.
\end{quotation}


\fi


\iffalse
Similar to the photon-lattice interaction in XRD, the process of TEM could be understood as electron scattering events by the same crystal plane. 

\begin{itemize}
\item X-ray: characteristic X-ray for elemental analysis, Bremsstrahlung X-ray also useful for biological sample; $K_\alpha$ line from L to K transition, and $K_\beta$ from M to K transition, $L_\alpha$ line from M to L transition, Inelastic cross section, Bethe expression, X-ray energies are not identical to the ionized energy because after first emission, the atom is not in ground state until a free electron fill the last hole in the outermost shell. A cascade of transitions, Coster-Kroning transition, X-ray line shift slightly due to the chemical bonding to another atom. XEDS is not good at analyzing light elements due to the low fluorescence yield, which is strongly dependent on Z. Bremsstrahlung X-ray emission is strongly forward, 
\item SE: ejected from the conduction or valence bands, weak so only escaping if near the surface, STEM, complex cross section mechanism, 
\item Auger: Auger electron has specific energy similar to X-Ray, but is much more strongly absorbed than X-ray. Stated another way, Auger electron is hard to escape, so it is surface sensitive. 
\item CL: spatial resolution around 100 nm, 
\item collective interaction, plasmon and phonon, plasmon excitation cross section in Lorenztian form $\frac{d\sigma_\theta}{d\Omega} = \frac{1}{2\pi a_0} \frac{\theta_E}{\theta^2 + \theta_E^2}$, where $\theta_E = E_p/2E_0$, 
\end{itemize}


The diameter of E-beam in TEM is less than 5 nm in general, and can be $< 0.1$ nm at best.

correction of spherical aberration ($C_s$) and chromatic aberration ($C_c$), $C_s$ is done by , $C_c$ by energy-filtering, which is more useful for thicker specimens. $C_s$ correction permits the generation of smaller electron probes with higher currents, which significantly improves both analytical spatial resolution and sensitivity. $C_c$ correction offer the possibility to form band-gap imaging and chemical-bond imaging. The limiting apertures increase the depth of field for specimen, and the depth of focus for the image.\cite{Williams2009} 

including controlling the interactions of electron with magnetic fields and with specimen.  

XRD: X-ray scattered by electrons, electron scattered by both electrons and nuclei. Fresnel vs. Fraunhofer, high-angle scattered electron are incoherent; therefore, it can be used to form high-resolution Z-contrast image of a crystalline specimen, regardless of the orientation. Auger electron spectroscopy. EELS and XEDS constitute analytical electron microscopy (AEM). close approach the single-atom level. well suited to, energy-loss electrons cause Kikuchi lines to arise in DPs. Ionized atom enters excited state, 

penetrate electron cloud, spherical wavelets, the cross section for electrons elastically scattered into angles larger than $\theta$ is $\sigma_{nucleus}= 1.62\times10^{-24} (\frac{Z}{E_0})^2\cot^2\frac{\theta}{2}$; scattering factor $f(\theta)$ for low angle ($< \sim 3^{\circ}$). 

Point-group and space-group determination from convergent-beam patterns. crystal symmetry analysis. e-beam wavelength in metal. The high-resolution comes at the cost of poor sampling.  Human eyes and brain understand reflected light image, and not well-trained for the transmission images.

One must be just as aware of the instrument's limitations as one is of its advantages.   TEM is initially developed to overcome the image resolution imposed by light microscopes. constitute, draw analogies, resolving power, E-beam is one type of ionizing radiation, which is capable of removing the tightly bound, inner-shell electrons from attractive field of nucleus (visible light, is non-ionizing radiation to some extent). a wide range of secondary signal can be produced, the spectra exhibits characteristic peaks, which identify the elements present in the specimen. In analog to laser as a highly coherent source, . 

Both wave and particle approach, non-scattering is invisible, backscattered in large angle and secondary electrons are of interest in SEM, where they provide Z contrast and surface-sensitive, topographical images. forward scattered is of interest in TEM.  scattering events as billiard balls colliding, coherently scattered are those that remain in step, and incoherently scattered electrons have random phase relationship. Assuming single scattering events in TEM, 

The cost of TEM adds up to \$10 per eV. seize the public's imaginations. 

The cross section of tungsten, moly is 

Focused ion beam (FIB) to prepare thin foils of individual gates from one of the many millions of such on a wafer. The events of electron passing through one crystal plane, The coherent length of e-beam, collection angle, 

\fi


\iffalse
incoherent illumination $s = \frac{1.22 \lambda_{vac}}{2n\sin i}$, 
coherent illumination in microscope $s = \frac{\lambda_{vac}}{n\sin i}$,, 
phase contrast: transforming phase change in object plane into amplitude variation in image plane. 
much better than needed, human eye resolving power 0.3 mrad, 

The illumination can be considered as incoherent (adding intensity) if the object is self-luminous, or if illminated from all direction. 

The opposite occurs in TEM probably, where the specimen is irradiated with complete transverse coherence across its area. back 
focal plane displays the Fourier transform of complex amplitude in object plane. The optical system is a spatial low-pass filter: it builds up images from only the low Fourier components present at the object, having higher values cutoff. 

The information about an object is on display in the objective lens's back focal plane in a Fourier-transform form. dark field, removing zero order; Schlieren technique, removing half of diffraction from one side of back focal plane; applications: apodizing in telescope, satellite transmitting aerial structure, 

with the aid of Fig. 
conjugate plane, 

electron lens can be made of electrostatic field, and magnetic field, and the B field can be generated from ferromagnetic materials (soft iron, 2 T) or superconducting (100 T). 

$C_s$ has dimensions of length, and is approximately equal to the focal length of objective lens (1-3 mm in TEM). 

$r_{min} \approx 0.91(C_s \lambda^3)^{1/4}$ is about 0.3 nm typically; with $C_s$ correction, $r_{min}$ can be further reduced to 0.07 nm. And human eye can resolve a distance of 0.2 mm, therefore the maximum useful magnification is about $0.2 \times 10^{-3}/0.3 \times 10^{-9} \sim 670 K$. 

And after interacting with specimen, the energy spread of transmitted E-beam becomes about 20eV, which will limit the resolution more than does the spherical aberration. 

\textbf{coherence of E-beam}

faint points, large disk, the feature of interest is what make the materials imperfect, 

20 keV E-beam in TV. 

In XRD, both diffraction peak position and intensity are used; whereas in TEM, most time only the positions of spots are of concern. 

physical process of diffraction where the atomic planes appear to behave as mirrors for incident E-beam; 

\fi

\iffalse
A Raman pattern database can be found at \url{http://wwwobs.univ-bpclermont.fr/sfmc/ramandb2/index.html}. 
In analytical practice, frequency is expressed in reciprocal wavelength (as cm−1), called wavenumbers;
\fi


\iffalse

hemispherical reflectance, 
diffusive reflectance is a mathematical artifice without direct physical meaning. 

p332- 338

the reflecting power associated with a specular-type process is often termed reflectivity, while reflectance is the analogous term for diffusely reflected radiation. This two types of reflection are present simultaneously. 

Cauchy formula, empirical equation which has subsequently been placed on a firm theoretical basis. 

intensity for a complex quantities is obtained by multiplying the complex conjugate. 

For strongly absorbing materials, the reflected light is complementary to the color of light transmitted by it, such as gold transmission is green; 

For the weakly absorbing substances of rough surface, it will appear the same color by transmission or reflection. 

penetration depth is wavelength dependent $d = \frac{In2 \lambda}{2\pi \sqrt{\sin^2\theta_i - n^2}}$, longer wavelength penetrated further than did shorter wavelength. attenuated total reflectance, 

thin film, amorphous, mean free path of electron is diminished, and porous in structure. The maxwell-garnett theory provides a satisfactory explanation, where free electron density is reduced. 

diffuse reflection: $I = I_0 \exp(-\epsilon d)$, Lambert cosine law formulated. multiple scattering by individual particles. 
$R_tot = \alpha R_sp + (1- \alpha) R_diff $, 

multiple scattering of diffuse radiation in a medium composed of closely packed particles. 

\[
\log{f(r_\infty) = \log{k} - \log{s}}
\]

It is concluded that the KM function can be used to obtain the characteristic color curve of a given sample only in the limit of small total absorption. 

by mixing the sample with an abundance of the reflectance standard, the regular reflection is extensively eliminated. 

The color of an object depends on the spectral composition of the source. The color of many light sources can be specified in terms of the temperature to which a blackbody radiator must be heated in order to achieve a color match for the source, which is known as color temperature. 

integrating sphere: the intensity at any part of the sphere, due to the reflected light is a measure of the total flux from a particular source, independent of the spatial distribution or location of the source on the sphere. 

substitution method, comparison method, 

\fi



\section{Data Processing}

The SEM images are cropped using Adobe Photoshop for highlighting the features and fitting into text width. To obtain the dimension of specimen, scale bar was measured and compared to the size of interest using ImageJ\cite{Schneider2012} for SEM, TEM images and SAED pattern. It is worth noting that ImageJ could directly read the dm3 format generated in Gatan imaging software, which is more productive than the scale bar method.

Various spectra, be it from XRD, EDX, Raman, or UV-Vis, were redraw using Origin Pro 8.0. The peak fitting was performed using \emph{Fityk}, a nonlinear curve fitting program specialized in data processing from powder diffraction, chromatography, photoluminescence and photoelectron spectroscopy, infrared and Raman spectroscopy, \emph{etc}.\cite{Wojdyr2010}




\section{wo3}

\subsection{used}

\citeauthor{Chatten2005} also studied the oxygen vacancy in different phases of \ce{WO3}.\cite{Chatten2005}

SG and FG experiments remain essentially the same as that of OT growth, except that in SG additional tungsten powders were distributed onto the receiving substrate, and in FG more than one piece of substrate was employed. The modification is schematically illustrated in Fig.~\ref{fig:wogrowsf}. More details will be provided when it comes to the discussion.
% sg fg
\begin{figure}[htb]
\centering
\includegraphics[width=0.5\textwidth]{sg_and_fg.jpg}
\caption[\ce{WO3} NW growth: SG and FG]{\ce{WO3} NW growth: SG and FG. (a) Seeded growth with additional powders on substrate (b) Flow growth with multiple substrates}
\label{fig:wogrowsf}
\end{figure}

With four kinds of W powders and three layouts, we designed the experimental matrix as illustrated in Table.~\ref{tab:matrix}. The symbol $\times$ means this combination is covered in this work, and NA means otherwise.
% Tungsten powders growth design
\begin{table}[htb]
\centering
\caption{Tungsten powders growth design}\label{tab:matrix}
\begin{tabular}{lccr}
\toprule
 & Ordinary Transport & Seeded growth & Flow growth \\
\midrule
3N   &  $\times$ & NA &  NA   \\
3N5  &  $\times$ & NA &  NA   \\
4N5  &  $\times$ & $\times$ & $\times$ \\
5N   &  $\times$ & $\times$ &  $\times$ \\
\bottomrule
\end{tabular}
\end{table}

We will first present the OT growth results in section~\ref{sec:nawox}, and discuss the other two in section~\ref{sec:sgfg}.

We found the growth using 3N source show distinctive features in comparison to the rest. This mainly arise from the higher sodium concentration in 3N source than that in others. Therefore we focus on 3N source first, and move on to other latter.


% Na5 raman fitting
\begin{figure}[htb]
\centering
\includegraphics[width=0.6\textwidth]{naxwo_ramfit}
\caption[\ce{Na5W_{14}O_{44}} Raman fitting]{Multi-peaks Lorentzian fitting on two major peaks region of \ce{Na5W_{14}O_{44}}. The peaks sum height difference is caused by different baseline value adopted in each fitting.}
\label{fig:naworamfit}
\end{figure}

% W-O bond length
\begin{table}[htb]
\centering
\caption{\ce{W-O} bond length predication}\label{tab:nawobond}
\begin{tabular}{lccr}
\toprule
peak center & length (\AA) & peak center & length (\AA) \\
\midrule
694.6 & 1.900 &  808.6 &  1.821 \\
745.4 & 1.863 &  911.5 &  1.758 \\
764.4 & 1.850 &  933.0 &  1.745 \\
778.7 & 1.840 &   943.5 & 1.740 \\
788.4 & 1.834 &   965.4 & 1.728 \\
\bottomrule
\end{tabular}
\end{table}

An empirical formula to relate the Raman peaks and \ce{W-O} bonding lengths \cite{Hardcastle1995} is
\begin{equation}\label{eq:wobond}
\nu = 25823 \exp{-1.902\cdot R}.
\end{equation}
And the standard deviation of estimating \ce{W-O} bond distance from Raman stretching wavenumber is $\pm0.034$\AA.
The observed Raman peaks of \ce{Na5W_{14}O_{44}} phase lies at 965, 943, 913, 808, 786, 778, 765, 695 and 107 cm. Multi-peaks Lorentzian fitting is preformed to precisely determine the central maximum. Good fitting is obtained, as shown in Fig.~\ref{fig:naworamfit}. We then calculated \ce{W-O} bond distance using Eq.~\ref{eq:wobond}, as illustrated in Table~\ref{tab:nawobond}. The predicted \ce{W-O} bond length comply very well with the crystallographic value of \ce{Na5W_{14}O_{44}} phase.\cite{Triantafyllou1999a} The 107 peak probably is caused by \ce{Na-O} bond.


\subsection{not used}


\iffalse
Materials that human can make define the age they live in. From Stone Age to Bronze Age and Iron Age, people evolve as mastering more and more sophisticated techniques of manipulating metals, such as alloying and annealing. Obtaining extreme high purity of silicon brings us into Information Age. Future is difficult to predict. But nanotechnology is one direction that we can not ignore. According to \gls{nni}, \gls{nanotechnology}. This definition alludes that dimension comes before compositions. It is often related to the quantum confinement or surface area in nanomaterials, which we will later revisit with specific scenario.

There are three states of matter under usual conditions: solid,liquid and gas. Solids materials could be further categorized into five groups: metals, ceramics, polymers, semiconductors, and composites.\cite{William2009} This classification is based on both composition and mechanical, electrical, and thermal properties as well as the associated functionality(i.e., \gls{ceramics} are typically hard yet brittle, insulating to electricity and resistant to heating).

\fi

\iffalse

We have synthesized \gls{tmo} and \gls{tmdc} at nanoscale, measured their crystalline structures and optical properties and demonstrated some devices assembled using as-synthesized nanomaterials. We aim to illustrate that by nanoengineering these \gls{tmo} and \gls{tmdc}, enhanced performances over their bulk states could be expected and new properties will arise. In the remaining sections of this chapter, we will discuss some general perspectives of nanomaterials, the growth apparatus and characterization methods that apply to all experiments done in this work. Then chapter 2 will focus on growth of \ce{WO3} and its derivative. We employed thermal \gls{cvd} to synthesize \ce{WO3} \gls{nw}, and we investigated the role of impurity in tungsten metallic powders, during which we observed a new state of sodium tungsten oxides: \ce{Na5W14O44} nanowires. We also found a method to potentially obtain large yield of \ce{WO3} \gls{nw}. Chapter 3 will concentrate on \ce{MoO3}. We explored two different growth mechanism of \ce{MoO3}:\gls{vs} and \gls{vls}. We discovered that alkaline oxides can be used as catalyst to grow two distinct \ce{MoO3} morphologies: nanobelts and towers. We further demonstrated the application of as-synthesized \ce{MoO3} nanomaterials in electrochromic devices.  In chapter 4 we discuss  We synthesized  and inspected the growth of \gls{fl} \ce{WS2}. Chapter 5 will conclude with an overall summary.

\fi

Therefore we refer their oxides and chalcogenides as \gls{tmo} and \gls{tmdc}.\footnote{Obviously transition metals include many other elements, all of which have partially filled $d$-electron shell. But here we use TM to denote W and Mo exclusively.}

Nucleation is a process of generating a new phase from a metastable old phase, where the Gibbs energy per molecule of the bulk of the emerging new phase is less than that of the old phase.

General CVD knowledge, substrate preparation, and\cite{MichealK.Zuraw2003}


The energetic sources are ion bombardment, electron beam, laser ablation, and combustion flame\cite{Rao2011}.

The sol–gel process is a well-known, intensively studied wetchemical technique that is widely used in materials synthesis. This method generally starts with a precursor solution (the ``sol") to form discrete particles or a networked gel structure. During the course of gelation (aging process), various forms of hydrolysis and polycondensation take place.
In addition, doped \ce{WO3} was also demonstrated
The composition and phase of final product highly depend on the synthesis conditions.

We do not discuss tungsten oxide hydrates (\ce{WO3.nH2O}) in this work since the product of thermal CVD approach is not plagued with this complexity. It's necessary, however, to deal with hydrated \ce{WO3} in the liquid synthesis routes, as indicated in Section.~\ref{sec:woxgrowth}.


Nonstoichiometric tungsten oxides \ce{WO_x} (i.e. \ce{WO_{2.92}}, \ce{WO_{2.87}}) are known as Magn$\acute{e}$li phases.


Theoretical computation of electronic band structures for \ce{WO_x} proves difficult due to the aforementioned phase transition. oxygen deficiency, structure change, electronic properties vary according.

the ubiquity of \ce{WO6} octahedra is essential for not only the optical properties but the ability to insert and extract ions in the EC oxides, due to the tunnels in three dimensions serving as path for transport of small ions. The intercalation of hydrogen or alkali ions into \ce{WO3} created electron donor level. By absorbing the red part of incident spectrum, electrons at donor level make transition to the conduction band, causing the blue coloration in \ce{H_xWO3}.

its one dimensional (1D) nanostructure has attained intensive research efforts in recent years due to the potential applications in advanced nano-electric and nano-optoelectronic devices.

\begin{quote}
a viable electrochromic smart window must exhibit a cycling life time \textgreater $10^5$ cycles, corresponding to an operation life at 10 - 20 years.
\end{quote}

\subsection{to be used}

% band gap table
\begin{table}[htb]
\centering
\caption{Tungsten oxides band gap }\label{tab:wo3eg}
\begin{tabular}{lccr}
\toprule
Phase & Experimental (eV) & Theory (eV) &  \\
\midrule
amorphous \ce{WO3} & 3.2  & NA &    \\
monoclinic bulk \ce{WO3} &  2.6   & 1.73\cite{Migas2010a}  &    \\
tetragonal bulk \ce{WO3} &     & 0.66 \cite{Migas2010a}&    \\
\bottomrule
\end{tabular}
\end{table}

W plasma oxidation.\cite{Romanyuk2005} 200nm W coating on Si (100) sub, temperature at RT, 390, and 490 C, oxygen pressure 0.5 Pa, oxidation time for 10 to 3600 s. The resultant thickness of \ce{WO3} at RT  and time of 10 s and 3600 s is found to be 0.2 nm and 11 nm respectively.

\[
 d = d_0 exp(kt)
\]

after fitting, $d_0 = 0.19777$ nm, $k = 0.00112 $ s$^{-1}$, so to oxidize 1nm W coating completely, oxygen plasma time is 1450 s at 0.5 Pa partially pressure; 2 nm for 2000 s.



\ce{WOx} and optical electric field enhancement. The enhancement arise from the structure composed of a conductive layer and an insulating layer that are laminated therein.\footnote{US patent 8601610B2} In \ce{WOx} nanorods, the oxygen deficient planes are conductive, each having atomic thickness and separated by several nm \ce{WO3}. Localized surface plasmons could possibly exist on these conductive planes. Therefore SERS applies and single molecule Raman scattering using a tungsten oxide nanorod has been demonstrated. The \ce{W_nO_{3n-1}} ($n \geq 2$) exhibit $\{ 001 \}$ CS structure. Chemical formulae corresponding to n=2, 3, 4, 5 and 6 are \ce{W2O5=WO_{2.5}}, \ce{W3O8=W_{2.67}}, \ce{W4O_{11}=WO_{2.75}}, \ce{W5O_{14}=WO_{2.8}}, and \ce{W6O_{17}=WO_{2.83}}, which indicates the existence of a oxygen deficient plane at every n row. Actually the value x in \ce{WOx} could almost continuously vary within a range of 2.5 to 3. \ce{W_{18}O_{49}=\ce{WO_{2.72}}} is an exception without $\{ 001 \}$ CS structure. Moreover, the oxygen deficient planes could extend along directions other than $\{ 001 \}$. For instance, the $\{ 102 \}$ CS planes appears in \ce{WOx} where x is within 2.93 to 2.98, and  the $\{ 103 \}$ CS planes for x within 2.87 to 2.93.\cite{Sloan1999}  \citeauthor{Shingaya2013} also synthesized \ce{WS2}-\ce{WO_x} structures and found similar Raman scattering enhancement. The x value is estimated by the Raman spectra peaks.\cite{Shingaya2013}(Data not shown in patent)

For photochemical water reduction to occur, the flat-band potential of the semiconductor (for highly doped semiconductors, this equals the bottom of the conductance band) must exceed the proton reduction potential of 0.0 V vs NHE at pH =0. \cite{Osterloh2008} flat-band potentials strongly depend on ion absorption (protonation of surface hydroxyl groups), crystallographic orientation of the exposed surface, surface defects, and surface oxidation processes.


\ce{W_{18}O_{49}} Raman, IR shielding.\cite{Guo2012} \cite{Guo2011}
broad peak between 750-780 cm-1.

\ce{WnO_{3n-1}} NPs. \cite{Frey2001}


WO3-x raman info, the encapsulated WOx core has been investigated in depth. several stable phase could occur, including \{001\} CS phases, \{103\} CS phases. no evidence of $\gamma$-\ce{W_{18}O_{49}} phase is found. The cross-section ($\sigma$) for Raman scattering and the absorption coefficient of the WS2 layers are much larger than those of the suboxide phase encapsulated inside.

WO2:168(w),189(w), 286(vs), 345(w), 423(w), 479(m), 512(m), 599(m),
617(m) cm-1, and a mode at 781(s) cm-1 which tails to higher energies (w-weak; m-medium; s-strong; vs-very strong).

W5O14: 215, 264, 325, 349, 418, 425,707, and 800 cm-1, 900 maybe

WO3: 808, 719, 275;

W3O8: 870;

no 950 peak indicates no hydrated phases.

\ce{WO_{3-x}} Raman peak at 778. \cite{Deb2007}

% wo3-x phases
\begin{table}[htb]
\centering
\caption{List of \ce{WO_{3-x}} phases}\label{tab:wo3xphase}
\begin{tabular}{lccccc}
\toprule
&&&\multicolumn{3}{c}{Lattice constants \AA} \\
\cmidrule(l){4-6}
 Symbol    & PDF  & Phase & a & b & c   \\
\midrule
\ce{W18O49}  & 00-036-0101 & monoclinic & 18.324 & 3.784 & 14.035  \\
$\delta$-\ce{WO3}   & $-50 \sim 17$  & triclinic & 7.309 & 7.522 & 7.686  \\
$\gamma$-\ce{WO3}   & $17 \sim 330$  & monoclinic I & 7.306 & 7.540 & 7.692  \\
$\beta$-\ce{WO3}    & $330 \sim 740$  & orthorhombic & 7.384 & 7.512 & 3.846  \\
$\alpha$-\ce{WO3}   & $> 740$  & tetragonal & 5.25 & NA & 3.91  \\
$h$-\ce{WO3}        &  $<400$  & hexagonal & 7.298 & NA & 3.899  \\
\bottomrule
\end{tabular}
\end{table}



\section{moo3}

\subsection{used}



\ce{MoO3}, an alternative interpretation in terms of tetrahedral coordination of Mo atoms is also proposed. This is caused by the fact that four of the six surrounding O atom are at distances from 1.67\AA to 1.95\AA, while the remaining two are as far as 2.25 and 2.33\AA. This also stress that the \ce{MOO6} octahedra are rather distorted.


\subsection{to be used}

The refractive index is between $ 2.2\sim 2.4$ along $a$ axis, and $ 2.5\sim 2.8$ along $c$ axis. Comparison to the low frequency dielectric constant ($18.0\pm1$) implies a primarily ionic bonding in \ce{MoO3}.\cite{He2003}


\cite{Matar2011} Using electronegativity $\chi$ and chemical hardness $\eta$ to assess electron affinity $E_a$, work function $W_f$, Fermi energy $E_f$ and band gap $E_g$.
\begin{align}
\chi &= 0.5(W_f + E_a)\\
\eta & = 0.5(W_f - E_a)
\end{align}
where I is ionization potential and $E_a$ is electron affinity.

Correlation between optical band gap and formation enthalpy; reaction occurs in order to form compounds with a larger gap.  $E_g = A \exp(0.34E_{\Delta H^0})$, and A adopts different values depending on the metal elements:
\begin{itemize}
\item A=0.8 for s and f block elements,
\item A = 1 for d block elements,
\item A = 1.35 for p block elements.
\end{itemize}


\citeauthor{Sreedhara2013} studied the kinetics of photodegradation of methylene blue\footnote{\ce{C16H18N3SCl},319.8 g/mol, MP: 100C accompanied with decomposition \url{http://en.wikipedia.org/wiki/Methylene_blue}} dye by few layer \ce{MoO3}.
For the photodegradation method, it was stated that `` the samples were collected after the photoreaction had been centrifuged for 5 min to remove the photocatalyst before UV-Vis measurement.''


% Melting points 
\begin{table}[htb]
\centering
\renewcommand*{\thetable}{S\arabic{table}}
\caption{physical constants of reactants }\label{tb:thermo}
\begin{tabular}{lccr}
\toprule
Material & MP(\si{\degreeCelsius}) & BP(\si{\degreeCelsius}) & reference\\
\midrule
\ce{NaOH}        & 318 & 1388 & handbook  \\
\ce{NaI}        & 651 & 1300 & MSDS    \\
\ce{KI}        & 681 & 1330 & MSDS   \\
\ce{Na2CO3}        & 851 & Not determined & MSDS    \\
\ce{Na2MoO4}        & 687 & Not available & handbook   \\
\ce{MoO3}    & 795 & 1155 & MSDS   \\
\ce{MoO2}    & 1100(decomp) & Not available & MSDS   \\
\bottomrule
\end{tabular}
\end{table}



\section{ws2}


\textbf{\ce{WS2}-\ce{WO3}}: 1 kW light source(Hg, or Xe lamp), photon flux, phenol (\ce{C6H5OH}, 94.1g/mol, MP 40C)concentration is 20 mg/L, hydroxyl group. The quantitative analysis of phenol was performed via a standard colorimetric method.\footnote{\url{http://omlc.ogi.edu/spectra/PhotochemCAD/html/072.html}}
\citeauthor{DiPaola1999} prepared \ce{WS2}-\ce{WO3} mixture in two methods, sulfurization of \ce{WO3} and oxidation of \ce{WS2},with the latter are more active.
\citeauthor{DiPaola1999} also concluded that the actual efficiency of mixed \ce{WS2}-\ce{WO3} catalysts depends on the ratio of each composition present of the surface of the particles, and the maximum of photoactivity is obtained with 40-50\% surface molar ratio of \ce{WS2}.

ref 25, 28 and 41.

\section{literature to read}

The geometrical symmetry groups of WS2 NT.\cite{Milosevic2000} 
The basis/lattice vectors $a_1$ and $a_2$ are defined in transition metal plane with equal length $a_0 \approx 3$ \AA. This monolayered sheet can be rolled up to form a tubular structure when the chiral/translation vector $c = n_1a_1 + n_2a_2$ becomes the circumference of the tube. The diameter of a nanotube is given by 
\begin{equation}
d = \frac{\mid c \mid}{\pi} = \frac{a_0}{\pi}\sqrt{n_1^2 + n_1n_2 + n_2^2}
\end{equation}
Notice $a_1 \cdot a_2 = a_1a_2 \cos\pi/3$, where the cosine times 2 is unit. 
In analog with carbon nanotube, chiral angle $\theta$ is given by
\begin{equation}
\theta = \tan^{-1}\frac{\sqrt{3}n_2}{2n_1 + n_2}
\end{equation}

Chiral angles from the interval $[0,pi/6]$ is sufficient for all possible tubes. Tubes $(n,0)$ with zero chiral angle are named zigzag, tubes $(n,n)$ with $\pi/6$ chiral angle are armchair, whereas all others are referred as chiral ones with $\theta \in (0,pi/6)$. 



hydrogen evolution catalysts. \cite{Merki2011}

Raman substrate dependence \cite{Buscema2013}

stability of TMS NTs \cite{Seifert2002}

2H and 1T in \ce{MoS2} \cite{Eda2012}

\ce{MoS2} FET statistical study. \cite{Liu2013i}

water splitting review. \cite{B800489G}

\ce{WS2} theory and experimental combined study. \cite{Klein2001}

2D review on oxides \cite{Osada2012}

Pb catalyzed \ce{MoS2} nanotube \cite{Brontvein2012}

\ce{WS2} Raman.\cite{Zhao2013} \cite{Sekine1980}

phonon dispersion $E_{2g}^1(M)$? \cite{Ataca2012}

MoS2 optical properties.\cite{Search1979}

FL heterostructure. \cite{Yu2013a}

\cite{Kang2013} TMDC alloy DFT.

thermoelectric TMDC \cite{Wickramaratne2014}

\ce{CH4N2S} thiourea + \ce{WOx} to \ce{WS2} \cite{Leonard-Deepak2011}

\ce{WS2} by \ce{WCl_n} and \ce{H2S}, raman (632nm) show bulk features\cite{Tenne2008}.

direct gap of ML at corner of BZ, point K.

\begin{align}
\cee{WCl6 + S &\rightarrow WS2 + Cl2S2}\\
\cee{MoCl5 + S &\rightarrow MoS2 + S2Cl2} \\
\cee{S2Cl2 + NaOH &\rightarrow NaCl + S + Na2SO3 + H2O}
\end{align}

\section{literature read}

CNT chirality by TEM \cite{Zhang1993} TEM chirality of \ce{MoS2} NTs


first demonstration of \ce{WS2} NT n-type FET. \cite{Levi2013}
the importance of contact, and avoiding moisture. 

The calculated carrier concentration is about $10^{19}cm^{-3}$, a highly doped semiconductor, possibibly arising from sulfur vacancy. 

\ce{WS2} NT transport \cite{Zhang2012c}
less grain boundary more mobility, 

\citeauthor{Ramasubramaniam2011} investigated the band gap tuning in bilayer TMDC materials by applying external $E$ field. Similar research has been done for graphene and bilayer boron nitride. Semiconductor-metal transition was suggested for \ce{MoS2} and \ce{WS2}, with difference on the CBM and VBM evolution. In \ce{MoS2}, the valence-band-splitting cause the A and B excitons in optical absorption measurement. Calculation shows that CB and VB are translated toward the Fermi level with increasing E field.  The external field localized charge along $c$ axis, but delocalized that within the plane normal towards $c$, thereby driving the semi-metal transition. It was mentioned that this transition is not anticipated in monolayer \ce{MoS2}. It was emphasized that precise band gaps might be different from the author’s results, yet the gap-tuning should be universal.\cite{Ramasubramaniam2011}

\cite{Song2013} \ce{WO3} by ALD, and sulfurized in Ar and \ce{H2S} (10:1) at 1000 C. \ce{WS2} layer No and peak intensities ratio under 633nm excitation is correlated. It was found the 2LA/A1g is less than 1 for 1L. In supporting info 532 nm Raman spectra, the 2LA/A1g is presumably larger than 1 for 1L. \ce{WS2} NT on Si NWs is also demonstrated.

\cite{Tenne2010} chemical modification of NTs. Functional ligand consists of an anchor group that attaches to the NTs surface and a tail which render them soluble in various solvents. PTAS functionalized BN nanotube lead to the formation of stable suspensions in aqueous solutions. The strong attachment is formed through $\pi-\pi$ interactions.


inorganic nanotubes review \cite{Tenne2004} , unsaturated bonds number increases as the size of MS2 sheet decreases.

Water splitting materials should process a band gap larger than 1.4eV, considering both the NHE potential distance and practical application. The monolayer \ce{WS2} exhibits direct gap of 1.98 eV. Quantum confinement could push the gap separation farther away. \cite{wilcoxon1997} \citeauthor{Notley2013} use liquid exfoliation to prepare \ce{WS2} NPs.\cite{Notley2013} Non-ionic surfactant concentration is about 0.1\%w/w. Continuously adding surfactant during sonication improves the yield. Optimum surface tension is found at about 40 mJ/m$^2$.

G/\ce{WS2}/G stacked solar cell. For lubricant, and surface protection. Absorption $\sim 10^7 m^{-1}$. \cite{Britnell2013}

In Ref\cite{Zeng2013}, single crystal \ce{WS2} growth using \ce{I2} transport was described in supporting info.
\ce{C24H12K4O8}\footnote{http://www.chemspider.com/Chemical-Structure.24771386.html}



416 peak was prevously assumed to be a combination of LA and TA phonons at K points.

Raman conditions: 0.3 mW, 532 nm, 200 sec acquisition time. photon flux = $0.3E-3\times6.242E+18/1.2398/0.532=2.84E15$

416 cm raman on WS2. B1u is pressure sensitive. \cite{Staiger2012} also studied pressure dependence.

A1g mode using resonance Raman excitation shows upshift, which is presumably caused by folding induced strains. And TEM SAED reveal 3R symmetry. We did not observed this upshift of A1g probably due to the large diameter and few layer involved.

Nanotube growth is relatively independent of substrates and FL layer is closely related to substrate. Film growth could provide some insight into the latter scenario.

\fi

\subsubsection{Seeded Growth of Tungsten Oxide NWs}
With the oxidation experiments study in Sec.~\ref{sec:woxd}, favorable conditions for local growth of NWs were extracted to perform seeded growth. 

\begin{figure}[htb]
\centering
\subfloat[]{\label{fig:sga}\includegraphics[width=0.4\textwidth]{wox_sg_a.jpg}}\hspace{0.04\textwidth}
\subfloat[]{\label{fig:sgb}\includegraphics[width=0.4\textwidth]{wox_sg_b.jpg}}
\caption[Characterization of seeded growth \ce{WO3}: SEM]{Characterizations of seeded growth. (a) SEM graphs of \ce{WO3} NWs on \ce{SiO2/Si} substrate. (b) A high magnification view showing uniform NW growth and close-up view of one NW. }
\label{fig:woseedsem}
\end{figure}

As shown in Fig.~\ref{fig:sgb}, dense NWs array was obtained on tungsten powder seeds with individual wires of length up to 5 $\mu$m and diameter about 50 to 200 nm, according to the estimation made in the close-up view. Each tungsten powder stood as independent growth site (Fig.~\ref{fig:sga}) with island-layer growth on the substrates, a common feature without using tungsten powder as seed under current experimental conditions. It was occasionally observed that NWs growth was initiated adjacent some tungsten powders. This phenomenon was correlated to the local trap of vapor flow since it was more often found among the enclosed area by tungsten powders. It is also found that the diameter of NWs decrease as the distance between powders and upstream edge increases. This is a combination effect of lower temperature and reduced \ce{WOx} growth species supply. Similar phenomena were observed in other studies. \citeauthor{Thangala2007} reported that a decrease in NW density with increasing substrate temperature, and an increase of NW density with increasing partial pressure of oxygen.\cite{Thangala2007}

% seeded edx 
\begin{figure}[htb]
\centering
\includegraphics[width=0.5\textwidth]{wo3_seed_edx}
\caption[Composition analysis on seeded growth \ce{WO3} NWs]{EDX spectroscopy on seeded growth \ce{WO3} NWs.}
\label{fig:woedx}
\end{figure}
Energy-dispersive X-ray spectroscopy (EDX) analysis on the seeded growth \ce{WO3} NWs is shown in Fig.~\ref{fig:woedx}. Only W and O elements were detected on the NW array. The background level from 3 to 8 keV was a manifestation of the continuous components of W X-ray spectrum. 

% sg raman xrd
\begin{figure}[htb]
\centering
\subfloat[]{\label{fig:sgxrd}\includegraphics[width=0.45\textwidth]{xrd_cs_before}}\hspace{0.04\textwidth}
\subfloat[]{\label{fig:sgram}\includegraphics[width=0.45\textwidth]{wox_raman_1}}
\caption[Characterization of seeded growth \ce{WO3}: XRD and Raman]{ (a) XRD pattern of as-prepared sample indicating the \ce{WO3} phase and the presence of metallic core. (b) Raman spectrum on NWs region showing the feature of \ce{WO3}.}
\label{fig:woseedxrd}
\end{figure}

Fig.~\ref{fig:sgxrd} is the XRD spectrum of one typical sample. The peaks under circular symbol were identified to be the monoclinic \ce{WO3} phase (ICDD PDF 01-083-0950, \emph{a}=7.30084 \AA, \emph{b}=7.53889 \AA, \emph{c}=7.6896 \AA, $\beta$=90.89$^\circ$), while the peak under the triangular symbol was indexed to cubic tungsten phase (ICDD PDF 04-16-3405, \emph{a}=3.157 \AA), in agreement with the EDX analysis (Fig.~\ref{fig:woedx}). This means that during the \ce{WO3} seeded growth of 4 h heating at 1000 \si{\degreeCelsius}, the tungsten powder in downstream low temperature region (600 -- 700 \si{\degreeCelsius}) is not entirely oxidized. Micro-Raman scattering spectroscopy was performed on the as-synthesized sample as well. During Raman examination, the laser spot was carefully focused onto the NWs on powders and several inspections on different positions were observed to ensure the reproductivity of spectra data. As shown in Fig.~\ref{fig:sgram}, five distinct bands were well resolved, with peaks located at 131, 265, 327, 711 and 803 \si{cm^{-1}}, respectively. This pattern was typical features of \ce{WO3}, consistent with previous study.\cite{Salje1975a,Dixit1986} The high background level probably arises from the metallic core.

% sg tem
\begin{figure}[htb]
\centering
\includegraphics[width=0.9\textwidth]{JAP-2column_Fig5major.jpg}
\caption[Characterization of \ce{WO3}: TEM]{TEM graphs of as-prepared NWs. (a) TEM image of one nanowire with diameter about 40 nm. (b) HRTEM images showing the spacing is 0.38 nm, corresponding to (002) plane distance.}
\label{fig:woseedtem1}
\end{figure}

TEM specimen was prepared by using carbon grid to slightly scratch the as-grown sample. Fig.~\ref{fig:woseedtem1} shows the feature of majority NWs. The growth direction is determined to be perpendicular to (002) plane. The streaking in SEAD pattern presumably arises from stacking defaults. The author also find some NWs exhibit high crystalline quality, as revealed by the TEM analysis in Fig.~\ref{fig:woseedtem2}. The NW grew normal to (002) plane with a measured lattice spacing of 3.79 \AA, which is favorably compared to the XRD peak at $23.07^\circ$ (7.7103 \AA). The sharp SEAD pattern and clear phase contrast in HRTEM demonstrated are both strong evidence of good crystallinity. This formation indicated current growth parameters have promising potential to obtain highly crystalline \ce{WO3} NWs in large scale. 

% sg tem
\begin{figure}[htb]
\centering
\includegraphics[width=0.9\textwidth]{JAP-2column_Fig5minor.jpg}
\caption[Characterization of \ce{WO3}: TEM cont]{TEM graphs of as-prepared NWs. (a) TEM image of one nanowire, the diameter is about 70 nm. (b) HRTEM images showing the spacing is 0.379 nm, corresponding to (002) plane distance.}
\label{fig:woseedtem2}
\end{figure}

In regarding to the formation of NWs on tungsten powder itself, this study assumes the driving force is related to interfacial strain between W and \ce{WOx}. Oxidation of tungsten proceed slowly at room temperature and an oxide layer of 100 \AA was found on the surface of tungsten foils.\cite{Warren1996} The tungsten powder used in current study would be covered by a thin oxide layer as well. During oxidation, different oxidation rates exist for different crystallographic orientations on the tungsten powder. Oxidation occurring at boundaries and defects are preferred thermodynamically.\cite{You2010} Compressive strain will gradually accumulate at the tungsten oxide/tungsten interface, which might limit the diffusion rate of oxygen at temperature lower than 500 \si{\degreeCelsius}.\cite{tungsten1999} At elevated temperature, cracks will primarily occur, as observed in Fig.~\ref{fig:pdtemp}. When heated up, tungsten and the oxide shell will probably relax the strain by converting into substoichiometric NWs, a similar process as suggested by \citeauthor{Klinke2005} in the chemically induced strain growth of tungsten oxide NWs.\cite{Klinke2005} It is worth noting that tungsten oxide nanowires could also formed when \ce{WO3} is reduced.\cite{Sarin1975} The elongation of \ce{WOx} is thermodynamically favorable during the conversion from metallic tungsten to tungsten oxide as well. Local evaporation-condensation process might also contribute to the formation of NWs on tungsten powder.

The enhanced yield of NWs obtained via seeded growth could be explained by a vapor-solid (VS) mechanism. External supply of growth species will condense onto the powders and substrate simultaneously, promoting the elongation of NWs on power as well as resulting island-layer growth on substrate. The local NW density in oxidation experiment was much higher than that of seeded growth. It is reasonable to presume that during seeded growth, several NWs in a small region on powder will coalescence, as evidenced by the bundled structures. At last, the author would like to point out that when low purity tungsten powder (3N) was used as source, sodium tungsten oxide nanowires were found to be dominant in the final product. The details have been published.\cite{Sheng2014} It seems surprising that when 3N powder was used as seeds, only \ce{WO3} nanowires were obtained. This result was attributed to the lower temperature and significantly reduced amount of 3N powder used in the seeded growth, compared with the conditions used in Ref.\cite{Sheng2014}. The source material in seeded growth is not limited to high purity tungsten powder. Instead, any material that could produce appropriate growth vapor could be employed, indicating the versatility of this approach.

\subsubsection{Optimization of W powder Oxidization}\label{sec:woxd}

Understanding the oxidation of tungsten powder is the key to obtain high yield in seeded growth. Oxidation of tungsten have been investigated under diverse conditions, such as at elevated temperature (\textgreater 1100 \si{\degreeCelsius} ) and oxygen pressure on the order of Torr,\cite{Base1965} and at temperatures ranging from 20 to 500 \si{\degreeCelsius} under atmosphere pressure.\cite{Warren1996} \ce{WO_x} NWs were readily found when tungsten (foil, wire, or powder) is oxidized under various conditions.\cite{Zhu1999,Karuppanan2007,Hsieh2010} However the study on tungsten powder oxidation behavior between intermediate temperature range and under low pressure is still rare. This work studied the oxidation of tungsten powders with diverse size within temperature range from 500 to 1000 \si{\degreeCelsius} and under several mTorr oxygen partial pressure. Using tungsten powder as seed, the author further illustrate an economic approach to obtain large yield of \ce{WO3} nanowires at relatively lower temperature than previous efforts. It is demonstrated that there is an optimal tungsten powder size under our experimental conditions for seeded growth. This will provide some insight on the role of tungsten powder as source material in CVD growth of \ce{WO_x}.

Commercial available tungsten powders with different size are usually associated with purity variation as well, as already listed in Table.~\ref{tab:ch5pre}. The dimensions of tungsten powder were obtained by measuring the average size in SEM graphs. A systematic investigation was performed on the oxidation behavior of tungsten powder to evaluate the temperature effect, size-dependence and influence of oxygen partial pressure.
% seed optimal
\begin{figure}[htb]
\centering
\includegraphics[width=0.6\textwidth]{JAP-2column_Fig1.jpg}
\caption[W powder oxidation: temperature effect]{SEM graphs of 99.9\% (3N) tungsten powder oxidization at different temperatures of a) 500 \si{\degreeCelsius}, b) 600 \si{\degreeCelsius}, c) 650 \si{\degreeCelsius}, d) 750 \si{\degreeCelsius}, showing the optimal temperature for local formation of nanowires is between 600--650 \si{\degreeCelsius}. Oxygen flow rate is 1 sccm.}
\label{fig:pdtemp}
\end{figure}

Fig.~\ref{fig:pdtemp} illustrated the effect of temperature on the morphological change and surface nanowires formation of 3N powder. At 500 \si{\degreeCelsius}, most tungsten powder retained its original shape and a layer of tiny dense NWs begun to grow. When temperature was increased to 600 \si{\degreeCelsius}, 3N powder started to crack with longer NWs on the isolated surface. Further increase of temperature lead to irregular shapes of tungsten power and aggregation of NWs, giving rise to the nanorods and bunched or bundled structures. It could be determined from the morphology variation that the optimal seeded growth temperature for 3N powder was in the range of 600 to 650 \si{\degreeCelsius}.
% seed optimal
\begin{figure}[htb]
\centering
\includegraphics[width=0.6\textwidth]{JAP-2column_Fig3.jpg}
\caption[W powder oxidation: size effect]{SEM graphs illustrating the oxidization of four different size of tungsten powders at 600~\si{\degreeCelsius} and 1 sccm oxygen flow. a) 17 $\mu m$, b) 32 $\mu m$, c) 3.3 $\mu m$, d) 1.5 $\mu m$.}
\label{fig:pdsize}
\end{figure}

Fig.~\ref{fig:pdsize} depicted the oxidation of different sizes of tungsten powder under the same experimental conditions. In contrast to the morphology of 3N powder shown in Fig.~\ref{fig:pdtemp}, 3N5 powder surface is primarily covered with sub-micro particles as well as some short tiny NWs, whereas 4N5 and 5N powder were thoroughly oxidized, showing branched flowers feature. This dramatic difference could be explained in terms of surface energy and oxygen diffusion. With smaller dimension, the increased surface-to-volume ratio and short diffusion path both lower the energy barrier of oxidation.\cite{tungsten1999} It was logical to deduce that higher temperature or increased oxygen level might favor the NWs formation on 3N5 powder. When it comes to seeded growth, however, the powder size distribution was an important factor to give uniform NWs deposition. Since the size distribution of 3N powder is more uniform than that of 3N5 powder, the author employed the former as seeds.
% seed optimal
\begin{figure}[htb]
\centering
\includegraphics[width=0.6\textwidth]{JAP-2column_Fig2.jpg}
\caption[W powder oxidation: oxygen pressure]{SEM graphs of 3N tungsten powder oxidization at 600 \si{\degreeCelsius} under different rates of oxygen flow: a) 1 sccm, b) 2 sccm, c) 3 sccm, d) 10 sccm. The oxygen partial pressures were 13 mTorr, 23 mTorr, 32 mTorr, and 82 mTorr, respectively with background pressure subtracted.}
\label{fig:pdoxy}
\end{figure}

Fig.~\ref{fig:pdoxy} depicted the morphology change of 3N powder with respect to varied oxygen partial pressure. When the oxygen flow is lower than 3 sccm, 3N powder almost stayed as the same, with cracks separating the dense layer of NWs. When oxygen flow is increased to 10 sccm, the 3N powder exemplified an enlarged version of that for 4N5 or 5N powder under 1 sccm oxygen flow. This observation again supported the surface energy explanation.

\subsection{used}

As the experimental setup for direct tensile tests of nanotubes is state-of-the-art,\cite{Tang2013} the application of tensile stress on 2D TMD systems is rather difficult due to the excellent lubricating properties of these materials.

\citeauthor{Zhang2013e} investigated the shear (C) and layer breathing mode (LBM) in the low frequency region of \ce{MoS2}.\cite{Zhang2013e} Even layer \ce{MS2} belong to point group D$_{6h}$ with inversion symmetry, while odd layer \ce{MS2} correspond to D$_{3h}$ without inversion symmetry. The excitation wavelength is 532nm from a diode-pumped solid-state laser. A power$\sim$0.23mW is used to avoid sample heating.

reaction mechanism of \ce{MoO3} to \ce{Mo2S}.\cite{Weber1996}

\citeauthor{Ling2014} studied the role of seeding promoters in CVD growth of FL \ce{MoS2}.\cite{Ling2014} PTAS treated substrates provided nucleation site and thus enable uniform deposition of \ce{MS2}.  This enhancement perhaps arise from the \ce{K+} ions.

\citeauthor{Splendiani2010} reported the PL in monolayer \ce{MoS2}.  Calculation indicated the indirect gap become larger when thinning, while the previous direct one almost stays as the same, the value is about 1.85eV (direct gap).\cite{Splendiani2010}


thermal decomposition of (NH4)2MoO2S2 and intermediate product MoOS2 was studied. application: hyfrodesulfurization in refinery \cite{Weber1996}

\cee{MoCl5 + 1/4S8 + 5/2H2 \rightarrow MoS2 + 5HCl} \cite{Stoffels1999}


A direct gap of $\sim 2eV$ at the corners of BZ is formed in 1L \ce{WS2}, Growth on bottom piece show the multiple domain flakes occurs at initial stage of the growth, starting from \ce{WO3} particles.
%\cite{Cong2013}
\subsection{to be used}

Exfoliated WS2 few layer PL.\cite{Zhao2012} excitonic absorption peaks A and B arising from direction transition at K point are found around 625nm (1.98eV) and 550nm, respectively, which are in agreement with results from bulk layers. The A, B excitons difference was a result of strong spin-orbital coupling. Relative PL quantum yield of WS2 between 1L and 2L is on the order of 2. The FWHM of WS2 peak is about 75 meV. wider than thermal energy at room temperature,

Electro microscopy on stacking sequences of WS2 NT.\cite{Houben2012} The probability of parallel stacking is about 30\%. a metal-semi superstructures. In NT, the layers are slightly shifted with respect to each other due to the constraints, thus the stacking is not exactly as pure phases of 2H(prismatic antiparallel), 3R(prismatic parallel) or 1T (octahedral parallel) with their perfect translational symmetry.

chevron pattern, contradictory, contradicting, Debye scattering model for XRD.

\begin{quote}
hexagonal polytype 2Hb with two molecular layers (spacegroup P63/mmc) and a rhombohedral polytype 3R with three molecular layers per unit cell (space group R3m), a high pressure polytype that is stable in plane geometry at pressures above 4 GPa. The two prismatic phases are semiconducting, and the octahedral one is metallic-like.

1T phase may be the result of a transformation from the 3R to the 2H phase by an intermediate 1T phase that is trapped by fast quenching

\end{quote}

aberration corrected TEM is used.

HRTEM on WS2 NT.\cite{Sadan2008} negative spherical-aberration imaging (NCSI). NCSI condiction were achieved at a negative spherical aberration of -20um balanced by an overfocus of +17 nm. Focal series reconstruction to retrieve the phase of electron exit plan wavefunction. Zigzag, armchair revealed.


In centrosymmetric crystals, the vibrational modes must either have even (Raman-active) or odd (IR-active) parity under inversion, which is known as rule of mutual exclusion. When this symmetry is broken, some modes may be simultaneously IR and Raman active.

inelastic neutron scattering to study the non-zone center LA mode. Zone-edge scattering can occur due to zone-folding process. The formation of superlattice could activate formerly inactive zone-edge phonons. The folding of BZ along $\Gamma-M$ would cause the M point to coincide with $\Gamma$ point, so LA(M) phonons would become Raman active in a first-order process.


\ce{SiO_x}-Si, \ce{WS2} absorption coefficient $10^{-7}m^{-1}$, mean free path of photo-excited charge carriers 1 $\mu m$. the wave vector of photon is considerably small than size of BZ, therefore The wave vector of phonon in Raman scattering usually close to zero.

Multiple phonon scattering, For two identical phonons, the corresponding Raman peak in the spectrum is called an overtone of the peak from the corresponding one-phonon process. And the wave vector conservation rule is automatically filled, therefore the phonon involved is not limited to BZ center anymore.
\[
I(G) \approx \sum_k \frac{\langle f|H_M|b\rangle \langle b|H_{ep}|a\rangle \langle a|H_M|i\rangle}{(E_p - E_k^{\pi *}- E_k^{\pi}-i\gamma)(E_p - E_k^{\pi *}- E_k^{\pi}-i\gamma- \hbar\Omega_{G})}
\]

the average distance traveled by an excited electron-hole pair before combination $l=\nu_F/\omega_D=4nm$.

Confocal Raman spectrometer:to obtain Raman spectrum in a specific depth of sample. Edge filter to cut off Rayleigh emission.


resolution $d= 1.22 \lambda/NA$,

Light Scattering in Solids II,. Springer, Berlin, 1982

influence of core WOx, Raman scattering by plasma-LO coupling to determine carrier concentration. measure resonant cross sections in absolute units.

disorder-induced light scattering, Van Hove critical points,
In resonant second-order scattering:
overtone: the same phonon,
combination: two different phonons;

\[
\frac{\ud\sigma}{\ud\Omega}= \omega_s^4 cm^6 Sr^{-1}
\]

scattering volume V in number of unit cells can be considered as one big molecule.


a single nanowire tends to minimize its surface. 2D isoperimetric quotient or circularity $C= \frac{4\pi A}{P^2}$, where A is area and P is perimeter of the cross-section.




\section{ECD}


Characterization of ECD (work like a thin-film batteries) includes transmission measurement and associated EC calculation, charge-discharge time, current-time curve and the fitting of obtained data.

The coloration efficiency (CE) represents the change in the optical density (OD) per unit charge density ($Q/A$, in units of \si{\cm^2\per\coulomb}) during switching and can be calculated according to the formula:
\begin{equation}
CE = \frac{\Delta~OD}{(Q/A)} [cm^2/C],
\end{equation}
where OD = $log(T_{bleach}/T_{color})$. The EC and optical density depend on the wavelength and are usually higher in the near IR than in the visible region.
Using Ohm's law($U_s = IR = RQ/t_s$) with switch voltage $U_s$, resistance R and surface area A, switching time $t_s$ could be estimated as
\begin{equation}
t_s = \Delta~OD\cdot A \cdot R /(CE\cdot U_s).
\end{equation}



battery and ECD.\cite{Granqvist2012} electrolyte: PVB (poly vinyl buteral).
alternative materials and design: organic, Prussian Blue as EC materials, metal hydrides, suspended particle device, liquid crystal, electroplating,
challenges: large area nanoporosity, transparent conducting contact, electrolyte with good ionic conductivity and poor electronic conductivity, stable under UV; assembly and large scale manufacturing;
cathodic coloration:
anodic coloration:
The coloration mechanism: \ce{MO6} octahedrons lead to $e_g$ and $t_{2g}$ level and ion channelling.
ref54,60,65,66,200,209,


\ce{WO3} as cathodic and either polyaniline(PANI) or Prussian white (PW) as anodic electrochromic half cells. \cite{Heckner2002}

Characterization of ECD includes transmission measurement and associated EC calculation, charge-discharge time, current-time curve and the fitting of obtained data.

\begin{quote}
a viable electrochromic smart window must exhibit a cycling life time \textgreater $10^5$ cycles, corresponding to an operation life at 10 -- 20 years.
\end{quote}


\citeauthor{Sella1998} studied the optical and structural properties of RF sputtered thin film of \ce{WO3} and \ce{VO2} for electrochromic devices. Ionic conductor was built using transparent polymer electrolyte, which was prepared from a solution of 1M \ce{LiClO4} in propylene carbonate which was mixed with methylmetharcylate (MMA). The main characteristics of polymer electrolyte were: viscosity at 25 \si{\degreeCelsius} $\approx$ 12920 Pa.s, conductivity $\approx 10^{-2}-10^{-4}$ \si{\per\ohm\per cm},non-hygroscopic if PMMA concentration \textgreater 30\%. A specific counter-electrode layer was not used since the encapsulated polymer electrolyte processes a very high ion storage capacity.\cite{Sella1998}

The device proposed was reproduced as shown in Fig.~\ref{fig:Sella1998ECD}
\begin{figure}[htb]
    \centering
    \includegraphics[angle=270,width=0.8\textwidth]{Sella1998ECD}
    \caption{citation, see original captions} \label{fig:Sella1998ECD}
\end{figure}

\subsection{polarons}

The concept of polaron was first proposed by Landau in 1933. In ionic or highly polar crystals, such as II-VI semiconductors, alkali halides and transition metal oxides, the Coulomb interaction between a conduction electron and the lattice ions results in a strong electron-phonon coupling. A new quasi-particle, virtual phonon, can be defined corresponding to the effect of electron pulling nearby positive ions towards it and pushing nearby negative ions away. The electron and its virtual phonons, taken together, can be treated as a new composite particle, called an electron polaron; the hole polaron is defined analogously. \cite{Devreese1996}

\subsection{NW waveguide}

seriously pursued. 
\section{docend}


 
%\section{Crystal Structures and Electronic Properties}

Solid and orderliness.

Two theories arise to describe the outer shell electrons and to correlate the structure and physical properties: \gls{cft} and band theory.\cite{Goodenough1971} \gls{cft} assumes weak interaction between neighboring atoms and localization of electron towards parent atom, whereas band theory assumes that electron is shared equally by all nuclei and therefore a many-electron problems follows. Description of a single electron in periodic potential fail to treat the electron correlations adequately, as the interaction between atoms becomes weaker.For transition metals, $s$ and $p$ electrons are well described by a collective-electron model, while the 4f or 5f electrons are tightly bound to nuclei and screened from the neighboring atoms by 5s, 5p or 6s, 6p core electrons, hence it matches well with a localized-electron model. d electrons show intermediate character.


In vapor synthesis process, two growth mechanism exists: VS and VLS. VS process is widely accepted for the growth of \ce{MoO3}. Yet we caution that synthesis conditions should be scrutinized to determine the exact mechanism. \citeauthor{Li2002c} suggested a VS mechanism at 700 \si{\degreeCelsius} and VLS at 750 \si{\degreeCelsius} and higher.\cite{Li2002c} \citeauthor{Fibers2007} proposed a modified VS mechanism probably because the deposition occurs on \ce{Al2SiO5} with possible \ce{Al_{0.95}SiNa_{0.06}O_x} involved. Therefore temperature and possible impurity could potentially alter the growth mechanism. We divide the growth results into two categories: on Si substrate and on non-Si substrate, and describe them respectively. We also briefly mention using liquid exfoliation to prepare few layer \ce{MoO3}.


This rectangular shape implies the boundary plane along growth direction (long axis) is (001), in consistency with previous experimental reports\cite{Zeng1998,Li2002b} and theoretical studies.\cite{Firment1983,Cora1997} We also observed different shapes, i.e., elongated hexagonal using similar growth conditions. This is not an indication of different growth mode. It is a normal thermodynamic fluctuation. The growth rates along different crystalline direction of \ce{MoO3} are determined by the free surface energy. In fact, (201), (101) and (102) planes have all been observed as terminating planes.\cite{Zeng1998} The stacking rate of \ce{MoO6} octahedra along $a$ and $c$ axis could develop some other ratio. And the coexistence of different planes in one growth suggests the similarity of free surface energies between these surfaces. In other words, the migration barrier of adatoms on (010) plane is presumably much lower than that on other low index planes due to the Van der Waals interaction nature along [010] direction.

\citeauthor{Wang2013c} changed the NW growth direction by changing the shape of droplets; \citeauthor{Biswas2013} reported enhanced aspect ratio of Ge NWs from bimetallic alloy.




We first repeated the previous growth on ITO/glass, and then found similar deposition phenomena on glass and mica substrates. We suspected that the morphology difference on the substrates employed in this study arises from the compositions difference. Both mica and glass contains alkaline elements which are absent in Si and \ce{SiO2}/Si substrates. We primarily verify this hypothesis by utilizing \ce{NaOH} treated Si substrates and found that the overall morphology almost reproduces that on ITO and mica substrates. We then propose a new growth mechanism based on the VLS process to explain the deposition of \ce{MoO3} on alkaline-ions-containing substrates.

ITO versus NaOH images.



According to the phase diagram shown in Fig.~\ref{fig:pd}, \ce{Na2MoO4} - \ce{MoO3} phase above 800 \si{\degreeCelsius} are \ce{Na2MoO4}(l) + \ce{MoO3}(l), More discussion should follow $\ldots$.


%\subsubsection{Phase evolution probing by Raman}

\citeauthor{Hardcastle1990} summarized an empirical formula to relate the Raman peaks and \ce{Mo-O} bonding lengths.\cite{Hardcastle1990} This correlation assumes general form as
\begin{equation}\label{eq:mobond}
\nu = A \exp{B\cdot R},
\end{equation}
where $A=32895$ and $B=-2.073$ are fitting parameters, R is bond distance in unit of \AA. Given a stretching frequency, the resolution for calculated bond distance is $\pm0.016$ \AA. Another empirical expression connect the bond valence $s$ and bond distance R: $s(M-O) \approx (R/X)^{-6} $, where X=1.882 when M is Mo, and 1.904 when M is W. The valence sum rule could be then used to check the state of Mo cation. It should be noticed that not all observed Raman lines could be correlated to a \ce{Mo-O} bond distance by extrapolation of Eq.~\ref{eq:mobond}. It is then regarded as a symmetry related vibrational mode, i.e. 820 \si{cm^{-1} in \ce{MoO3}. From the correlation of various Mo compounds, a general conclusion is the lower the stretching frequency for the shortest metal-oxygen bond, the more regular is the structure.



% droplet position statistics
\begin{figure}[htb]
\centering
\includegraphics[width=0.6\textwidth]{droplets_sta}
\caption[Droplet position statistics]{Droplet position statistics. On-tip and not-on-tip counts variation with respect to growth time. }
\label{fig:mo3dropsta}
\end{figure}


Nanotower is a less common morphology in the structures of nanomaterials.\cite{Kharissova2010} We have not found report on  \ce{MoO3} nanotower structures, although there are several studies on \ce{In2O3},\cite{Jean2010,Yan2007}, GaN\cite{Xiao2012} and ZnO\cite{Zhang2013c}. \citeauthor{Jean2010} investigated the growth mechanism of \ce{In2O3} nanotowers and attributed it to the periodical axial and continuous lateral growth interaction, where the former was first guided by Au-In alloy liquid, and subsequent by self-catalytic In droplets (MP 156C) while the latter due to VS process. In contrast, \citeauthor{Yan2007} observed similar \ce{In2O3} nanotower growth and proposed that the axial growth is controlled by Au-catalytic VLS process. \citeauthor{Xiao2012} reported the GaN nanotower grown on Ni-coated Si(111) substrates and explained the growth as asymmetric self-copy process based on VLS mechanism. \citeauthor{Zhang2013c} studied the growth of ZnO nanotower and provided a competitive model between axial and lateral growth controlled by Zn vapor to explain the as-synthesized structures. (When ZnO and carbon mixture was used as source instead of ZnO alone, nanorods array results, presumably due to the stable supply of Zn vapor owing to the gentle carbothermal reduction.) Therefore the fluctuation of reagents is probably responsible for tapering and periodical modification, which is absent in the coin roll style growth of \ce{MoO3} tower on Si. The occasionally observed tapering or enlarging on \ce{MoO3} tower can be attributed to the cooling down growth.

We divide the growth into two stages according to the emerging time of long belts and towers. In first stage the growth is dominated by lateral growth, as illustrated in . In the second stage, \ce{NaxMoO3} is evaporated and recondensed in the low temperature region, and some droplet will form, and when at proper locations, introduce secondary VLS growth.

In first stage, the Na-Mo-O liquid is formed and have much more higher absorption rate than that of bare substrate. The absorbed species will diffuse along the surface and inside the liquid. It was difficult to estimate which diffusion path one is faster. 


\textbf{open questions}
\begin{itemize}
\item Is the liquid consumed? if so, how? : evaporation
\item How the size of liquid is related to the morphology evolution?
\item Possible doping level of alkaline ions
\item other lateral growth mode such as Ni-W-S system
\item VLS-VS interaction, or possible solution-liquid-solid, VSS,
\item the interpretation of phase diagram
\item the overall molar amount of NaOH and MoO3 deposition
\item Young's parameters, contact angle informations, Gibbs free energy calculation
\item catalyst composition variation during growth
\item \ce{Na2Mo4O_{13}} phases, solid solubility of \ce{Na2MoO4} in solid \ce{MoO3} is high.
\item vapor pressure of \ce{Na2Mo4O_{13}} and \ce{MoO3}.
\end{itemize}


As shown in Fig, we found several solid clusters with diameter about 1 $\mu m$ on the top part of long belts. According to the \ce{Na2MoO4}-\ce{MoO3} binary compounds phase diagram, we expect the terminating solid clusters have a Na:Mo ratio close to 0.6. The experimental results match well with this predictions. EDX analysis reveal the presence of Na on the solid cluster but not at about 1$\mu m$ away from previous location.\footnote{the sensitivity of our instrument is estimated to be 1\% atomic level} Quantitative atomic ratio derived from the solid cluster is not close to the one predicted by phase diagram probably due to the high Mo content in adjacent area. However, XRD examinations on a short time growth sample revealed a possible \ce{Na2Mo4O13} phase(PDF 028-1112), providing an indirection evidence of the droplet composition. We also suspect that the growth will become different below 500 \si{\degreeCelsius} since no liquid would remain. Substrate placed at low temp end show.
We can also deduce that other sodium compounds can be employed as substrate treatment agent. This argument is confirmed by \ce{Na2CO3} and \ce{KI} assisted growth.

