\chapter{paper draft}

The KM two-flux model has been extensively used to describe the optical properties of inhomogeneous materials that consist of scattering and absorbing particles in a matrix.\cite{Vargas1997} There are some limitation in the assumptions made in original KM model, e.g., the distribution of scattering on the optical path. And revised KM models have been proposed to eliminate these limitations.\cite{Yang2004}  

\citeauthor{Morandi2005} studied the absorbance and diffuse reflectin of \ce{MoO3} and \ce{WO3} thin films using FT-IT and UV-Vis spectroscopy. The observed absorption in visible and middle IR region were attributed to oxygen defects induced donor levels.\cite{Morandi2005}

The ratio between absorption and scattering, not the absolute value of each quality, determines the reflection features.


\section{wo3}

\subsection{used}

SG and FG experiments remain essentially the same as that of OT growth, except that in SG additional tungsten powders were distributed onto the receiving substrate, and in FG more than one piece of substrate was employed. The modification is schematically illustrated in Fig.~\ref{fig:wogrowsf}. More details will be provided when it comes to the discussion.
% sg fg
\begin{figure}[htb]
\centering
\includegraphics[width=0.5\textwidth]{sg_and_fg.jpg}
\caption[\ce{WO3} NW growth: SG and FG]{\ce{WO3} NW growth: SG and FG. (a) Seeded growth with additional powders on substrate (b) Flow growth with multiple substrates}
\label{fig:wogrowsf}
\end{figure}

With four kinds of W powders and three layouts, we designed the experimental matrix as illustrated in Table.~\ref{tab:matrix}. The symbol $\times$ means this combination is covered in this work, and NA means otherwise.
% Tungsten powders growth design
\begin{table}[htb]
\centering
\caption{Tungsten powders growth design}\label{tab:matrix}
\begin{tabular}{lccr}
\toprule
 & Ordinary Transport & Seeded growth & Flow growth \\
\midrule
3N   &  $\times$ & NA &  NA   \\
3N5  &  $\times$ & NA &  NA   \\
4N5  &  $\times$ & $\times$ & $\times$ \\
5N   &  $\times$ & $\times$ &  $\times$ \\
\bottomrule
\end{tabular}
\end{table}

We will first present the OT growth results in section~\ref{sec:nawox}, and discuss the other two in section~\ref{sec:sgfg}.

We found the growth using 3N source show distinctive features in comparison to the rest. This mainly arise from the higher sodium concentration in 3N source than that in others. Therefore we focus on 3N source first, and move on to other latter.


% Na5 raman fitting
\begin{figure}[htb]
\centering
\includegraphics[width=0.6\textwidth]{naxwo_ramfit}
\caption[\ce{Na5W_{14}O_{44}} Raman fitting]{Multi-peaks Lorentzian fitting on two major peaks region of \ce{Na5W_{14}O_{44}}. The peaks sum height difference is caused by different baseline value adopted in each fitting.}
\label{fig:naworamfit}
\end{figure}

% W-O bond length
\begin{table}[htb]
\centering
\caption{\ce{W-O} bond length predication}\label{tab:nawobond}
\begin{tabular}{lccr}
\toprule
peak center & length (\AA) & peak center & length (\AA) \\
\midrule
694.6 & 1.900 &  808.6 &  1.821 \\
745.4 & 1.863 &  911.5 &  1.758 \\
764.4 & 1.850 &  933.0 &  1.745 \\
778.7 & 1.840 &   943.5 & 1.740 \\
788.4 & 1.834 &   965.4 & 1.728 \\
\bottomrule
\end{tabular}
\end{table}

An empirical formula to relate the Raman peaks and \ce{W-O} bonding lengths \cite{Hardcastle1995} is
\begin{equation}\label{eq:wobond}
\nu = 25823 \exp{-1.902\cdot R}.
\end{equation}
And the standard deviation of estimating \ce{W-O} bond distance from Raman stretching wavenumber is $\pm0.034$\AA.
The observed Raman peaks of \ce{Na5W_{14}O_{44}} phase lies at 965, 943, 913, 808, 786, 778, 765, 695 and 107 cm. Multi-peaks Lorentzian fitting is preformed to precisely determine the central maximum. Good fitting is obtained, as shown in Fig.~\ref{fig:naworamfit}. We then calculated \ce{W-O} bond distance using Eq.~\ref{eq:wobond}, as illustrated in Table~\ref{tab:nawobond}. The predicted \ce{W-O} bond length comply very well with the crystallographic value of \ce{Na5W_{14}O_{44}} phase.\cite{Triantafyllou1999a} The 107 peak probably is caused by \ce{Na-O} bond.


\subsection{not used}
Both tungsten (W) and molybdenum (Mo) belong to Group VIB transition metal, with outer shell electrons configuration as $4d^55s^1$ and $5d^46s^2$, respectively. Therefore we refer their oxides and chalcogenides as \gls{tmo} and \gls{tmdc}.\footnote{Obviously transition metals include many other elements, all of which have partially filled $d$-electron shell. But here we use TM to denote W and Mo exclusively.}

Nucleation is a process of generating a new phase from a metastable old phase, where the Gibbs energy per molecule of the bulk of the emerging new phase is less than that of the old phase.

General CVD knowledge, substrate preparation, and\cite{MichealK.Zuraw2003}


The energetic sources are ion bombardment, electron beam, laser ablation, and combustion flame\cite{Rao2011}.

The sol–gel process is a well-known, intensively studied wetchemical technique that is widely used in materials synthesis. This method generally starts with a precursor solution (the ``sol") to form discrete particles or a networked gel structure. During the course of gelation (aging process), various forms of hydrolysis and polycondensation take place.
In addition, doped \ce{WO3} was also demonstrated
The composition and phase of final product highly depend on the synthesis conditions.

We do not discuss tungsten oxide hydrates (\ce{WO3.nH2O}) in this work since the product of thermal CVD approach is not plagued with this complexity. It's necessary, however, to deal with hydrated \ce{WO3} in the liquid synthesis routes, as indicated in Section.~\ref{sec:woxgrowth}.


Nonstoichiometric tungsten oxides \ce{WO_x} (i.e. \ce{WO_{2.92}}, \ce{WO_{2.87}}) are known as Magn$\acute{e}$li phases.


Theoretical computation of electronic band structures for \ce{WO_x} proves difficult due to the aforementioned phase transition. oxygen deficiency, structure change, electronic properties vary according.

the ubiquity of \ce{WO6} octahedra is essential for not only the optical properties but the ability to insert and extract ions in the EC oxides, due to the tunnels in three dimensions serving as path for transport of small ions. The intercalation of hydrogen or alkali ions into \ce{WO3} created electron donor level. By absorbing the red part of incident spectrum, electrons at donor level make transition to the conduction band, causing the blue coloration in \ce{H_xWO3}.

its one dimensional (1D) nanostructure has attained intensive research efforts in recent years due to the potential applications in advanced nano-electric and nano-optoelectronic devices.

\begin{quote}
a viable electrochromic smart window must exhibit a cycling life time \textgreater $10^5$ cycles, corresponding to an operation life at 10 - 20 years.
\end{quote}

\subsection{to be used}

W plasma oxidation.\cite{Romanyuk2005} 200nm W coating on Si (100) sub, temperature at RT, 390, and 490 C, oxygen pressure 0.5 Pa, oxidation time for 10 to 3600 s. The resultant thickness of \ce{WO3} at RT  and time of 10 s and 3600 s is found to be 0.2 nm and 11 nm respectively.

\[
 d = d_0 exp(kt)
\]

after fitting, $d_0 = 0.19777$ nm, $k = 0.00112 $ s$^{-1}$, so to oxidize 1nm W coating completely, oxygen plasma time is 1450 s at 0.5 Pa partially pressure; 2 nm for 2000 s.



\ce{WOx} and optical electric field enhancement. The enhancement arise from the structure composed of a conductive layer and an insulating layer that are laminated therein.\footnote{US patent 8601610B2} In \ce{WOx} nanorods, the oxygen deficient planes are conductive, each having atomic thickness and separated by several nm \ce{WO3}. Localized surface plasmons could possibly exist on these conductive planes. Therefore SERS applies and single molecule Raman scattering using a tungsten oxide nanorod has been demonstrated. The \ce{W_nO_{3n-1}} ($n \geq 2$) exhibit $\{ 001 \}$ CS structure. Chemical formulae corresponding to n=2, 3, 4, 5 and 6 are \ce{W2O5=WO_{2.5}}, \ce{W3O8=W_{2.67}}, \ce{W4O_{11}=WO_{2.75}}, \ce{W5O_{14}=WO_{2.8}}, and \ce{W6O_{17}=WO_{2.83}}, which indicates the existence of a oxygen deficient plane at every n row. Actually the value x in \ce{WOx} could almost continuously vary within a range of 2.5 to 3. \ce{W_{18}O_{49}=\ce{WO_{2.72}}} is an exception without $\{ 001 \}$ CS structure. Moreover, the oxygen deficient planes could extend along directions other than $\{ 001 \}$. For instance, the $\{ 102 \}$ CS planes appears in \ce{WOx} where x is within 2.93 to 2.98, and  the $\{ 103 \}$ CS planes for x within 2.87 to 2.93.\cite{Sloan1999}  \citeauthor{Shingaya2013} also synthesized \ce{WS2}-\ce{WO_x} structures and found similar Raman scattering enhancement. The x value is estimated by the Raman spectra peaks.\cite{Shingaya2013}(Data not shown in patent)

For photochemical water reduction to occur, the flat-band potential of the semiconductor (for highly doped semiconductors, this equals the bottom of the conductance band) must exceed the proton reduction potential of 0.0 V vs NHE at pH =0. \cite{Osterloh2008} flat-band potentials strongly depend on ion absorption (protonation of surface hydroxyl groups), crystallographic orientation of the exposed surface, surface defects, and surface oxidation processes.


\ce{W_{18}O_{49}} Raman, IR shielding.\cite{Guo2012} \cite{Guo2011}
broad peak between 750-780 cm-1.

\ce{WnO_{3n-1}} NPs. \cite{Frey2001}


WO3-x raman info, the encapsulated WOx core has been investigated in depth. several stable phase could occur, including \{001\} CS phases, \{103\} CS phases. no evidence of $\gamma$-\ce{W_{18}O_{49}} phase is found. The cross-section ($\sigma$) for Raman scattering and the absorption coefficient of the WS2 layers are much larger than those of the suboxide phase encapsulated inside.

WO2:168(w),189(w), 286(vs), 345(w), 423(w), 479(m), 512(m), 599(m),
617(m) cm-1, and a mode at 781(s) cm-1 which tails to higher energies (w-weak; m-medium; s-strong; vs-very strong).

W5O14: 215, 264, 325, 349, 418, 425,707, and 800 cm-1, 900 maybe

WO3: 808, 719, 275;

W3O8: 870;

no 950 peak indicates no hydrated phases.

\ce{WO_{3-x}} Raman peak at 778. \cite{Deb2007}

% wo3-x phases
\begin{table}[htb]
\centering
\caption{List of \ce{WO_{3-x}} phases}\label{tab:wo3xphase}
\begin{tabular}{lccccc}
\toprule
&&&\multicolumn{3}{c}{Lattice constants \AA} \\
\cmidrule(l){4-6}
 Symbol    & PDF  & Phase & a & b & c   \\
\midrule
\ce{W18O49}  & 00-036-0101 & monoclinic & 18.324 & 3.784 & 14.035  \\
$\delta$-\ce{WO3}   & $-50 \sim 17$  & triclinic & 7.309 & 7.522 & 7.686  \\
$\gamma$-\ce{WO3}   & $17 \sim 330$  & monoclinic I & 7.306 & 7.540 & 7.692  \\
$\beta$-\ce{WO3}    & $330 \sim 740$  & orthorhombic & 7.384 & 7.512 & 3.846  \\
$\alpha$-\ce{WO3}   & $> 740$  & tetragonal & 5.25 & NA & 3.91  \\
$h$-\ce{WO3}        &  $<400$  & hexagonal & 7.298 & NA & 3.899  \\
\bottomrule
\end{tabular}
\end{table}



\section{moo3}

\subsection{used}



\ce{MoO3}, an alternative interpretation in terms of tetrahedral coordination of Mo atoms is also proposed. This is caused by the fact that four of the six surrounding O atom are at distances from 1.67\AA to 1.95\AA, while the remaining two are as far as 2.25 and 2.33\AA. This also stress that the \ce{MOO6} octahedra are rather distorted.


\subsection{to be used}


\cite{Matar2011} Using electronegativity $\chi$ and chemical hardness $\eta$ to assess electron affinity $E_a$, work function $W_f$, Fermi energy $E_f$ and band gap $E_g$.
\begin{align}
\chi &= 0.5(W_f + E_a)\\
\eta & = 0.5(W_f - E_a)
\end{align}
where I is ionization potential and $E_a$ is electron affinity.

Correlation between optical band gap and formation enthalpy; reaction occurs in order to form compounds with a larger gap.  $E_g = A \exp(0.34E_{\Delta H^0})$, and A adopts different values depending on the metal elements:
\begin{itemize}
\item A=0.8 for s and f block elements,
\item A = 1 for d block elements,
\item A = 1.35 for p block elements.
\end{itemize}


\citeauthor{Sreedhara2013} studied the kinetics of photodegradation of methylene blue\footnote{\ce{C16H18N3SCl},319.8 g/mol, MP: 100C accompanied with decomposition \url{http://en.wikipedia.org/wiki/Methylene_blue}} dye by few layer \ce{MoO3}.
For the photodegradation method, it was stated that `` the samples were collected after the photoreaction had been centrifuged for 5 min to remove the photocatalyst before UV-Vis measurement.''


% Melting points 
\begin{table}[htb]
\centering
\renewcommand*{\thetable}{S\arabic{table}}
\caption{physical constants of reactants }\label{tb:thermo}
\begin{tabular}{lccr}
\toprule
Material & MP(\si{\degreeCelsius}) & BP(\si{\degreeCelsius}) & reference\\
\midrule
\ce{NaOH}        & 318 & 1388 & handbook  \\
\ce{NaI}        & 651 & 1300 & MSDS    \\
\ce{KI}        & 681 & 1330 & MSDS   \\
\ce{Na2CO3}        & 851 & Not determined & MSDS    \\
\ce{Na2MoO4}        & 687 & Not available & handbook   \\
\ce{MoO3}    & 795 & 1155 & MSDS   \\
\ce{MoO2}    & 1100(decomp) & Not available & MSDS   \\
\bottomrule
\end{tabular}
\end{table}



\section{ws2}


\textbf{\ce{WS2}-\ce{WO3}}: 1 kW light source(Hg, or Xe lamp), photon flux, phenol (\ce{C6H5OH}, 94.1g/mol, MP 40C)concentration is 20 mg/L, hydroxyl group. The quantitative analysis of phenol was performed via a standard colorimetric method.\footnote{\url{http://omlc.ogi.edu/spectra/PhotochemCAD/html/072.html}}
\citeauthor{DiPaola1999} prepared \ce{WS2}-\ce{WO3} mixture in two methods, sulfurization of \ce{WO3} and oxidation of \ce{WS2},with the latter are more active.
\citeauthor{DiPaola1999} also concluded that the actual efficiency of mixed \ce{WS2}-\ce{WO3} catalysts depends on the ratio of each composition present of the surface of the particles, and the maximum of photoactivity is obtained with 40-50\% surface molar ratio of \ce{WS2}.

ref 25, 28 and 41.


\subsection{used}

As the experimental setup for direct tensile tests of nanotubes is state-of-the-art,\cite{Tang2013} the application of tensile stress on 2D TMD systems is rather difficult due to the excellent lubricating properties of these materials.

\citeauthor{Zhang2013e} investigated the shear (C) and layer breathing mode (LBM) in the low frequency region of \ce{MoS2}.\cite{Zhang2013e} Even layer \ce{MS2} belong to point group D$_{6h}$ with inversion symmetry, while odd layer \ce{MS2} correspond to D$_{3h}$ without inversion symmetry. The excitation wavelength is 532nm from a diode-pumped solid-state laser. A power$\sim$0.23mW is used to avoid sample heating.

reaction mechanism of \ce{MoO3} to \ce{Mo2S}.\cite{Weber1996}

\citeauthor{Ling2014} studied the role of seeding promoters in CVD growth of FL \ce{MoS2}.\cite{Ling2014} PTAS treated substrates provided nucleation site and thus enable uniform deposition of \ce{MS2}.  This enhancement perhaps arise from the \ce{K+} ions.

\citeauthor{Splendiani2010} reported the PL in monolayer \ce{MoS2}.  Calculation indicated the indirect gap become larger when thinning, while the previous direct one almost stays as the same, the value is about 1.85eV (direct gap).\cite{Splendiani2010}


thermal decomposition of (NH4)2MoO2S2 and intermediate product MoOS2 was studied. application: hyfrodesulfurization in refinery \cite{Weber1996}

\cee{MoCl5 + 1/4S8 + 5/2H2 \rightarrow MoS2 + 5HCl} \cite{Stoffels1999}


A direct gap of $\sim 2eV$ at the corners of BZ is formed in 1L \ce{WS2}, Growth on bottom piece show the multiple domain flakes occurs at initial stage of the growth, starting from \ce{WO3} particles.
%\cite{Cong2013}
\subsection{to be used}

Exfoliated WS2 few layer PL.\cite{Zhao2012} excitonic absorption peaks A and B arising from direction transition at K point are found around 625nm (1.98eV) and 550nm, respectively, which are in agreement with results from bulk layers. The A, B excitons difference was a result of strong spin-orbital coupling. Relative PL quantum yield of WS2 between 1L and 2L is on the order of 2. The FWHM of WS2 peak is about 75 meV. wider than thermal energy at room temperature,

Electro microscopy on stacking sequences of WS2 NT.\cite{Houben2012} The probability of parallel stacking is about 30\%. a metal-semi superstructures. In NT, the layers are slightly shifted with respect to each other due to the constraints, thus the stacking is not exactly as pure phases of 2H(prismatic antiparallel), 3R(prismatic parallel) or 1T (octahedral parallel) with their perfect translational symmetry.

chevron pattern, contradictory, contradicting, Debye scattering model for XRD.

\begin{quote}
hexagonal polytype 2Hb with two molecular layers (spacegroup P63/mmc) and a rhombohedral polytype 3R with three molecular layers per unit cell (space group R3m), a high pressure polytype that is stable in plane geometry at pressures above 4 GPa. The two prismatic phases are semiconducting, and the octahedral one is metallic-like.

1T phase may be the result of a transformation from the 3R to the 2H phase by an intermediate 1T phase that is trapped by fast quenching

\end{quote}

aberration corrected TEM is used.

HRTEM on WS2 NT.\cite{Sadan2008} negative spherical-aberration imaging (NCSI). NCSI condiction were achieved at a negative spherical aberration of -20um balanced by an overfocus of +17 nm. Focal series reconstruction to retrieve the phase of electron exit plan wavefunction. Zigzag, armchair revealed.


In centrosymmetric crystals, the vibrational modes must either have even (Raman-active) or odd (IR-active) parity under inversion, which is known as rule of mutual exclusion. When this symmetry is broken, some modes may be simultaneously IR and Raman active.

inelastic neutron scattering to study the non-zone center LA mode. Zone-edge scattering can occur due to zone-folding process. The formation of superlattice could activate formerly inactive zone-edge phonons. The folding of BZ along $\Gamma-M$ would cause the M point to coincide with $\Gamma$ point, so LA(M) phonons would become Raman active in a first-order process.


\ce{SiO_x}-Si, \ce{WS2} absorption coefficient $10^{-7}m^{-1}$, mean free path of photo-excited charge carriers 1 $\mu m$. the wave vector of photon is considerably small than size of BZ, therefore The wave vector of phonon in Raman scattering usually close to zero.

Multiple phonon scattering, For two identical phonons, the corresponding Raman peak in the spectrum is called an overtone of the peak from the corresponding one-phonon process. And the wave vector conservation rule is automatically filled, therefore the phonon involved is not limited to BZ center anymore.
\[
I(G) \approx \sum_k \frac{\langle f|H_M|b\rangle \langle b|H_{ep}|a\rangle \langle a|H_M|i\rangle}{(E_p - E_k^{\pi *}- E_k^{\pi}-i\gamma)(E_p - E_k^{\pi *}- E_k^{\pi}-i\gamma- \hbar\Omega_{G})}
\]

the average distance traveled by an excited electron-hole pair before combination $l=\nu_F/\omega_D=4nm$.

Confocal Raman spectrometer:to obtain Raman spectrum in a specific depth of sample. Edge filter to cut off Rayleigh emission.


resolution $d= 1.22 \lambda/NA$,

Light Scattering in Solids II,. Springer, Berlin, 1982

influence of core WOx, Raman scattering by plasma-LO coupling to determine carrier concentration. measure resonant cross sections in absolute units.

disorder-induced light scattering, Van Hove critical points,
In resonant second-order scattering:
overtone: the same phonon,
combination: two different phonons;

\[
\frac{\ud\sigma}{\ud\Omega}= \omega_s^4 cm^6 Sr^{-1}
\]

scattering volume V in number of unit cells can be considered as one big molecule.


a single nanowire tends to minimize its surface. 2D isoperimetric quotient or circularity $C= \frac{4\pi A}{P^2}$, where A is area and P is perimeter of the cross-section.




\section{ECD}


Characterization of ECD (work like a thin-film batteries) includes transmission measurement and associated EC calculation, charge-discharge time, current-time curve and the fitting of obtained data.

The coloration efficiency (CE) represents the change in the optical density (OD) per unit charge density ($Q/A$, in units of \si{\cm^2\per\coulomb}) during switching and can be calculated according to the formula:
\begin{equation}
CE = \frac{\Delta~OD}{(Q/A)} [cm^2/C],
\end{equation}
where OD = $log(T_{bleach}/T_{color})$. The EC and optical density depend on the wavelength and are usually higher in the near IR than in the visible region.
Using Ohm's law($U_s = IR = RQ/t_s$) with switch voltage $U_s$, resistance R and surface area A, switching time $t_s$ could be estimated as
\begin{equation}
t_s = \Delta~OD\cdot A \cdot R /(CE\cdot U_s).
\end{equation}



battery and ECD.\cite{Granqvist2012} electrolyte: PVB (poly vinyl buteral).
alternative materials and design: organic, Prussian Blue as EC materials, metal hydrides, suspended particle device, liquid crystal, electroplating,
challenges: large area nanoporosity, transparent conducting contact, electrolyte with good ionic conductivity and poor electronic conductivity, stable under UV; assembly and large scale manufacturing;
cathodic coloration:
anodic coloration:
The coloration mechanism: \ce{MO6} octahedrons lead to $e_g$ and $t_{2g}$ level and ion channelling.
ref54,60,65,66,200,209,


\ce{WO3} as cathodic and either polyaniline(PANI) or Prussian white (PW) as anodic electrochromic half cells. \cite{Heckner2002}

Characterization of ECD includes transmission measurement and associated EC calculation, charge-discharge time, current-time curve and the fitting of obtained data.

\begin{quote}
a viable electrochromic smart window must exhibit a cycling life time \textgreater $10^5$ cycles, corresponding to an operation life at 10 -- 20 years.
\end{quote}


\citeauthor{Sella1998} studied the optical and structural properties of RF sputtered thin film of \ce{WO3} and \ce{VO2} for electrochromic devices. Ionic conductor was built using transparent polymer electrolyte, which was prepared from a solution of 1M \ce{LiClO4} in propylene carbonate which was mixed with methylmetharcylate (MMA). The main characteristics of polymer electrolyte were: viscosity at 25 \si{\degreeCelsius} $\approx$ 12920 Pa.s, conductivity $\approx 10^{-2}-10^{-4}$ \si{\per\ohm\per cm},non-hygroscopic if PMMA concentration \textgreater 30\%. A specific counter-electrode layer was not used since the encapsulated polymer electrolyte processes a very high ion storage capacity.\cite{Sella1998}

The device proposed was reproduced as shown in Fig.~\ref{fig:Sella1998ECD}
\begin{figure}[htb]
    \centering
    \includegraphics[angle=270,width=0.8\textwidth]{Sella1998ECD}
    \caption{citation, see original captions} \label{fig:Sella1998ECD}
\end{figure}

\subsection{polarons}

The concept of polaron was first proposed by Landau in 1933. In ionic or highly polar crystals, such as II-VI semiconductors, alkali halides and transition metal oxides, the Coulomb interaction between a conduction electron and the lattice ions results in a strong electron-phonon coupling. A new quasi-particle, virtual phonon, can be defined corresponding to the effect of electron pulling nearby positive ions towards it and pushing nearby negative ions away. The electron and its virtual phonons, taken together, can be treated as a new composite particle, called an electron polaron; the hole polaron is defined analogously. \cite{Devreese1996}

\subsection{NW waveguide}

seriously pursued. 