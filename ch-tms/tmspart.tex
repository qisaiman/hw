\section{Introduction}
\subsection{Properties of TMDCs}
Tungsten disulfide (\ce{WS2}) is a group VI dichalcogenide semiconductor compound. The molecular weight is 249.97 \si{g\per \mole}. Almost all natural \ce{WS2} belongs to P63/mmc space group (2H-\ce{WS2}), where $a$ is 3.153 \si{\angstrom}, $c$ is 12.323 \si{\angstrom} (PDF 04-003-4478). Two other crystal structures were found in man-made \ce{WS2}, i.e., 1T and 3R, where 3R can be prepared by bromine (Br) chemical vapor transport (CVT) method \cite{Schutte1987} and 1T was generally found in alkaline intercalated \ce{WS2}.\cite{Yang1996a, Enyashin2011}

As a reference, the lattice dimensions are summarized in Table~\ref{tab:ms2lattice}.
%lattice
\begin{table}[htb]
\centering
\caption{Lattices dimension of \ce{MS2}}\label{tab:ms2lattice}
\begin{tabular}{lccr}
\toprule
         &  & 2H-\ce{MoS2}\cite{Coehoorn1987,Ataca2012} & 2H-\ce{WS2}\cite{Albe2002,Schutte1987} \\
\midrule
Lattice constant & a(\AA) & 3.1604 & 3.171 \\
                 & c(\AA) & 12.295 & 12.359 \\
Within \ce{MS2} layer & M-3S (\AA)& 2.37  & 2.405   \\
                      & S-1S (\AA)& 3.11  & 3.14   \\
Between \ce{MS2} layer& S-3S (\AA)&   & 3.53   \\
\bottomrule
\end{tabular}
\end{table}

The geometrical symmetry groups of WS2 NT.\cite{Milosevic2000} 
The basis/lattice vectors $a_1$ and $a_2$ are defined in transition metal plane with equal length $a_0 \approx 3$ \AA. This monolayered sheet can be rolled up to form a tubular structure when the chiral/translation vector $c = n_1a_1 + n_2a_2$ becomes the circumference of the tube. The diameter of a nanotube is given by 
\begin{equation}
d = \frac{\mid c \mid}{\pi} = \frac{a_0}{\pi}\sqrt{n_1^2 + n_1n_2 + n_2^2}
\end{equation}
Notice $a_1 \cdot a_2 = a_1a_2 \cos\pi/3$, where the cosine times 2 is unit. 
In analog with carbon nanotube, chiral angle $\theta$ is given by
\begin{equation}
\theta = \tan^{-1}\frac{\sqrt{3}n_2}{2n_1 + n_2}
\end{equation}

Chiral angles from the interval $[0,pi/6]$ is sufficient for all possible tubes. Tubes $(n,0)$ with zero chiral angle are named zigzag, tubes $(n,n)$ with $\pi/6$ chiral angle are armchair, whereas all others are referred as chiral ones with $\theta \in (0,pi/6)$. 
\begin{figure}[htb]
\centering
\includegraphics[width=0.8\textwidth]{ws2_tube1}
\caption[\ce{WS2} nanotube rolling]{\ce{WS2} nanotube rolling}
\label{fig:ch5ntroll}
\end{figure}



\subsection{Synthesis Review of TMDCs}
Shortly after the discovery of \gls{cnt} in 1991,\cite{Iijima1991} its transition metal dichalcogenides (TMDC) counterparts, i.e., \ce{WS2}, was synthesized in 1992.\cite{Tenne1992} In the following decades, other TMDC and several metal oxides nanotubes were demonstrated as well. \citeauthor{Rao2003} did an excellent work to review these inorganic nanotubes.\cite{Rao2003} Hence this study just recapitulates the main categories of synthesis methods:
\begin{enumerate}
\item Reducing \ce{MO3} in \ce{H2} and \ce{H2S} atmosphere;\footnote{M = Mo, W in this work}
\item Direct decomposition of \ce{MS3};
\item Decomposition of the ammonium salt \ce{(NH4)2MS4};
\item Using CNT as templates, arc discharge, or laser ablation.
\end{enumerate}
Except the aforementioned methods, another important reaction in CVD domain is
\begin{equation}\label{eq:mclns}
\cee{MoCl5(WCl6) + S \rightarrow MoS2(WS2) + S2Cl2}.
\end{equation}
\begin{align}
\cee{WCl6 + S &\rightarrow WS2 + Cl2S2}\\
\cee{MoCl5 + S &\rightarrow MoS2 + S2Cl2} \\
\cee{S2Cl2 + NaOH &\rightarrow NaCl + S + Na2SO3 + H2O}
\end{align}
This reaction has been explored by several authors\cite{Stoffels1999} and could be used under atmospheric pressure.\cite{Li2004a}

All these methods can be applied to the growth of TMDC few layer structures, directly or with some modifications. Among all the members of TMDC nanotubes, \ce{WS2} and \ce{MoS2} NTs are probably the most well investigated.\cite{Homyonfer1997,Tenne1998,Frey1998,Frey1999,Rothschild2000,Zak2000} The reaction mechanism of \ce{WO3} with \ce{H2}/\ce{H2S} has been thoroughly studied \cite{Feldman1998} and high yield synthesis approach has been established.\cite{Margolin2004} For \ce{WO3} nanoparticles precursor, it was found that the simultaneous reduction and sulfurization is essential for encapsulation of fullerene like \ce{WS2} structures from the oxide nanoparticles. During the surfurization process, remaining oxide core is gradually reduced and transformed into an ordered superlattice of $\{ 001 \}$ \gls{cs} planes. Further reaction will consume up the \ce{WO_x} core, leaving multi-walled \ce{WS2} NTs only. For precursor of \ce{WO3} nanowires, \citeauthor{Feldman1996} proposed that TMDC nanotubes growth began with the reduced \ce{WO3} phases, in particular \ce{W_{18}O_{49}}.\cite{Feldman1996} It is worth noting that \ce{MS2} can be prepared by direct sulfurization from its oxides phase, but reaction \cee{MO3 + Se \rightarrow MSe2} will not proceed unless introducing other reducing agents, such as \ce{H2}.\cite{Tsirlina1998} This fact highlights that sulfur is rather radical at elevated temperature.

\citeauthor{Zhu2000} performed a detailed morphological and structural analysis on \ce{WS2} NTs synthesized using \ce{WO_{3-x}} NWs and \ce{H2S}.\cite{Zhu2000} Sulfur vacancy was found on the outer wall of NT. The tips exhibits various structures. Open-ended can be observed, more frequently than that in carbon nanotubes (CNTs), which is often sealed. \citeauthor{Zhu2000} suggested this open-ended tubes resulted from continuous growth on other nanotubes. And the closure configuration is rather complicated. Flat caps often dominates. \citeauthor{Zhu2000} maintained that this oxide-to-sulfides mechanism might apply to closed caps only, not to those open ended tubes.\cite{Zhu2000} TEM diffraction analysis could distinguish the chirality of NTs, as demonstrated by CNT \cite{Zhang1993} and \ce{MoS2}\cite{MARGULIS1996}. \ce{WS2} NT chirality revealed by TEM SAED shows armchair NTs dominates. \citeauthor{Sloan1999} investigated tungsten oxides structures incorporated in \ce{WS2}.\cite{Sloan1999} The encapsulated \ce{WO_x} cores often exhibit \ce{W3O8} and \ce{W5O_{14}} phases, both of which belong to the \ce{W_nO_{3n-1}} homogenous series, arising as a result of \gls{cs} planes.\cite{Miyano} Some \ce{WO_x} cores show oxygen vacancy instead of CS planes, leading to prominent streaking in SAED patterns. Although high yield growth method of \ce{WS2} has been available, the extent to which one can control the NT configuration is still limited, i.e., single-walled NT has not been routinely synthesized and the electronic properties can only be predicted theoretically. \citeauthor{Seifert2000} investigated the electronic structures of \ce{WS2} nanotube using DFT calculation.\cite{Seifert2000} It was found zigzag (n,0) NTs exhibts a direct gap at $\Gamma$ point, whose size increase monotonically with tube diameter, while armchair (n,n) NTs, unlike its metallic counterpart of (n,n) CNT, show indirect band gap increasing with tube diameter. \citeauthor{Zibouche2012} reached the same conclusion on \ce{WS2} SWNT.\cite{Zibouche2012} It was also found band gaps of armchair and zigzag NTs increase with diameter, going from values close to bulk and approaching that of  monolayer, and for a given tube diameter, $E_g$ of zigzag NTs are larger than those of armchair NTs. This difference presumably vanishes when nanotube diameter exceeds 100 nm.

Parallel with the scenario of CNT and graphene, TMDC few layer structures attract intensive research efforts recently. These unfolded TMDC nanotubes exhibit many appealing features, i.e., indirect-to-direct band gap transition,\cite{Splendiani2010} valley electronics, and become a promising candidate in energy harvesting, optoelectronics and photocatalytic activity. For scientific research purpose, exfoliation by liquid \cite{Smith2011} or mechanical methods \cite{Lee2010a} can provide sufficient materials. Yet considering the integration into current microelectronic process, CVD synthesis of thickness controllable 2D TMDC is highly desired, which stills hold the greatest potential for high yield production of \ce{MS2}. So this work focuses on reviewing the CVD based methods. This study will review literatures to date on both \ce{WS2} films and few layer structures, and highlight several typical investigations using different synthesis methods, as summarized in Table~\ref{tab:tmsgrowth}. It is important to note that the synthesis of NTs is relevantly independent of substrates, which is in sharp contrast to the scenario of few layer growth. The \ce{WS2} nucleation and interaction with substrates play dominant roles in the growth of 2D FL structures. Hence it is reasonable to focus on the \ce{WS2} films growth, which presumably could provide valuable insight into the nucleation process on various substrates.

\citeauthor{Lee1994} studied the CVD of \ce{MoS2} by \ce{MoCl5} or \ce{MoF6} and \ce{H2S} in great details. Phase diagrams for \ce{Mo-S-Cl-H} and \ce{Mo-S-F-H} system at 1 kPa were simulated.\cite{Lee1994} \citeauthor{Endler1999} also investigated the solid-phase diagram for \ce{Mo-S-Cl-H-Ar} system.\cite{Endler1999} It was found pure \ce{MoS2} was main product when molar ratio of \ce{H2S}-\ce{MoCl5} exceeds 2. More important, \ce{MoS2} basal plane orientation can be parallel to substrate when the thickness was smaller than 50 nm. \citeauthor{Ennaoui1995a} grew tungsten disulfide film using sulfurization of \ce{WO3} under \ce{N2}/\ce{H2} gas flow.\cite{Ennaoui1995a} The composition was found to be \ce{WS_{2.13}}, and the excess of sulfur lead to p-type conductivity. XRD peaks ratio is used as a measure of film orientation, which was found higher for film prepared without an intermediate Ni coating. \ce{Ni3S2} phase was found and surfactant-mediated epitaxy is proposed. \citeauthor{Regula1997} studied the Ni-W-S phase diagram and the role of Ni layer in promoting \ce{WS2} film growth from amorphous \ce{WS3}.\cite{Regula1997} \emph{In-situ} TEM analysis confirmed the formation of \ce{NiS_x} droplets and lateral growth of \ce{WS2} from these droplets.\cite{Regula1998} Recently, it has been shown that direct sulfurization of W coating (20 nm) at 750 \si{\degreeCelsius} on \ce{SiO2}-Si can also produce \ce{WS2} film.\cite{Shanmugam2012a} However, the film is of bulk in nature. Previous study of \ce{WS2} growth on various substrates by \citeauthor{Genut1992} was summarized in Table~\ref{tab:ws2subs}.\cite{Genut1992} It was found that the \ce{WS2} nucleate from an amorphous \ce{WS3} phase and the substrate has a critical role in determining both the reaction onset temperature and the texture. The adhesion of tungsten to quartz was found to be much stronger than to glass. And oxygen-containing species such as \ce{H2O} or OH tend to cause \ce{WS2} basal plane perpendicular to substrate.
% WS2 films
\begin{table}[htb]
\centering
\caption{\ce{WS2} growth on different substrates}\label{tab:ws2subs}
\begin{tabular}{lcp{1in}p{2in}}
\toprule
precursor                 & substrate &  conditions & film feature\textsuperscript{\emph{a}}  \\
\midrule
sputtered W + \ce{H2S}   & glass      & onset 400 \si{\degreeCelsius} & $\perp c$ at 500 \si{\degreeCelsius}, metastable \ce{WS3} found\\
                          & quartz      & onset 650 \si{\degreeCelsius} & $\parallel c$ below 950 \si{\degreeCelsius}\\
                          & Mo        & onset NA          & $\parallel c$ at 1000 \si{\degreeCelsius}\\
                          & W          & onset NA           & random orientation\\
\midrule
\ce{WO_x} + \ce{H2S}    & quartz      & onset 500 \si{\degreeCelsius} & predominantly $\perp c$ after 800 \si{\degreeCelsius}\\
                        & Mo        & onset NA          & $\parallel c$ dominant\\
                        & W       & onset NA          & random orientation\\
\bottomrule
\end{tabular}

\textsuperscript{\emph{a}} $\perp c$: the c axis is perpendicular to the substrate, $\parallel c$: the c axis is parallel to the substrate;

\end{table}

For \ce{MoS2}, the reaction mechanism of \ce{MoO3} to \ce{MoS2} was studied by \citeauthor{Weber1996}.\cite{Weber1996} This study presumably provided guidelines for the recent syntheses of \ce{MoS2} by sulfurization of \ce{MoO3}.\cite{Lin2012,Lee2012b,Liu2012a,Najmaei2013} Similar studies on sulfurization of \ce{WO3.H2O} to \ce{WS2} and decomposition of \ce{(NH4)2WO2S2} were also reported.\cite{VanderVlies2002,VanderVlies2002a} In combination with the knowledge from \ce{WS2}, this thesis compares those insight from several recent reports on \ce{WS2} FL. \citeauthor{Cong2013} prepared monolayer \ce{WS2} on 300 nm \ce{SiO2}-Si by sulfurization of \ce{WO3} powders in a one-end sealed tube.\cite{Cong2013} It was suggested that pre-cleaning the inner tube by IPA and DI water could effectively increase the pressure of vapor source. This observation probably arise from the reduction of \ce{WO3} assisted by IPA and water residuals, or due to the possible presence of \ce{H2S}. Intermediate phase \ce{WO_yS_{2-y}} and \ce{WS_{2+x}} is proposed in the growth mechanism. The apex of triangles could be active site of nucleation, \ce{WS_{2+x}} formation is confirmed by secondary ion mass spectrometry. Possible thick \ce{WS_{2+x}} flakes decompose subsequently, leading to ML \ce{WS2}. The substrates are thoroughly cleaned. It was mentioned that separate sulfur heating improved the PL uniformity. Sulfurization mechanism study of \ce{WO3} suggested \ce{W^{6+}} cannot be directly reduce to \ce{W^{4+}} in \ce{WS2}.\cite{VanderVlies2002,VanderVlies2002a} Tungsten oxysulfides was necessary as the intermediate phase. 

\citeauthor{Peimyoo2013} prepared \ce{WS2} on \ce{SiO2}-Si using \ce{WO3} powder and sulfur at 800 \si{\degreeCelsius}, aiming at the light emission studies to clarify several contradictory reports.\cite{Peimyoo2013} Uniform PL intensity was found on the triangular \ce{WS2} flakes, in contrast to previous edge enhanced PL.\cite{Berkdemir2013} Raman spectra (532 nm) fit includes $E_{2g}^1(M)$ mode at 343 cm$^{-1}$, according to the phonon dispersion calculation \cite{Molina-Sanchez2011} and experimental observation\cite{Zeng2013a,Zhao2013,Lee2013}. However, none of these experimental reports specifically mentioned $E_{2g}^1(M)$ mode. Tentative assignments of multi-phonon bands are summarized in supporting information of Ref\cite{Zhao2013}. As to the growth setup, similar tube furnace as in Ref \cite{VanderZande2013} and \cite{Najmaei2013} was used. To gain additional wisdom on \ce{MoS2} growth, this thesis thus summarizes the growth strategies in Ref \cite{VanderZande2013} and \cite{Najmaei2013} as following. \citeauthor{VanderZande2013} prepared \ce{MoS2} on 285 nm \ce{SiO2}-Si using \ce{MoO3} and S as precursors.\cite{VanderZande2013} In contrast to the seeding method adopted in ref\cite{Lee2013,Lee2012b}, \citeauthor{VanderZande2013} stressed the importance of carefully cleaned substrates\footnote{acetone, 2 h in \ce{H2SO4} and \ce{H2O2} (3:1) and 5 min oxygen plasma} and minimum exposure of precursor to air.\footnote{APCVD, 105 \si{\degreeCelsius} for 4 h, 700 \si{\degreeCelsius} hold for 5 min, and 10 sccm \ce{N2} within 2 in tube. rapid cooling from 570 \si{\degreeCelsius}} Dirty substrates or old precursors will lead to hexagonal, 3-point star or irregular polycrystalline structures. The growth setup is similar to those in Ref.\cite{Lee2012b}. The substrate and \ce{MoO3} source distance is critical in determining the growth density. The synthesis strategy in Ref\cite{Lee2013,Lee2012b} is briefly mentioned as well, where PTAS treated substrate is found to promote the deposition, while KCl treated substrates did not, and small carrier gas flow is preferred (1 sccm \ce{N2}). It was also noted that the \ce{MoS2} flakes morphology seems more uniform in Ref\cite{VanderZande2013} than that in Ref\cite{Lee2012b}. On the other hand, \citeauthor{Najmaei2013} also demonstrated \ce{MoS2} FL growth by sulfurization of \ce{MoO3} nanoribbons,\cite{Najmaei2013} which was applied by dispersion and meant to control the source amount. It was found the diffusion of vapor \ce{MoO_{3-x}} is rate-limiting step in \ce{MoS2} growth. This means the amount of source is critical in successful synthesis. The nucleation event is more frequently observed at substrate edges, scratches or rough surface. Step edges is then intentionally created to facilitate the nucleation. As to the growth dynamics, it was postulated the oxysulfides (\ce{MoOS2} Raman spectra found), as intermediate phase, diffuse across the bare substrates and form triangular domains upon further sulfurization. The optimal growth conditions are 800-850 \si{\degreeCelsius}, 700 Torr, and sufficient sulfur. This thesis then conjectures that the substrate cleaning in Ref.\cite{Peimyoo2013} should be similar to that in ref\cite{VanderZande2013}, yet the source-to-substrate layout and growth pressure are still needed to be optimized with respect to the home-built CVD apparatus.

\citeauthor{Zhang2013h} synthesized \ce{WS2} on sapphire (0001) using \ce{WO3} powder and sulfur as precursor under 900 \si{\degreeCelsius}.\cite{Zhang2013h} Ar slightly mixed with \ce{H2} is used to tailor the shape of \ce{WS2} flakes. It was found the source substrate distance plays an important role in determining the morphology of the as-grown flakes.\footnote{There is a lattice mismatch between \ce{Al2O3} (4.785 \AA) and \ce{WS2} (3.153 \AA)} The edge termination is not well studied. Raman spectra indicate a universal down-shift of $A_{1g}$ peak (1L from 418 \si{cm^{-1}} on \ce{SiO2}-Si to 416.4 \si{cm^{-1}} on sapphire). PL signal of \ce{WS2} on \ce{SiO2}-Si is stronger than that on sapphire. The visibility of \ce{WS2} on \ce{Al2O3} is poor. No 1L vs nL statistic is available. FET on/off ratio is about 100, indicating low mobility. The other conditions includes 880 \si{\degreeCelsius}, 90 mm source-to-substrate distance, 1 in quartz tube. The growth conditions are probably adopted from Ref.\cite{Huanga2013}.

It is worth noting the flake sizes on sapphire\cite{Zhang2013h} seems larger than that on \ce{SiO2}/Si\cite{Peimyoo2013}, either due to the nucleation barrier difference or the amount of growth vapor and growth time. The flake sizes in Ref.\cite{Cong2013} is not quite uniform, though it could be even larger than that on sapphire from Ref.\cite{Zhang2013h}. One common feature of aforementioned investigations is growth occurs on bare substrates and no catalyst or seeding promoter is employed. While there is another synthesis approach\cite{Lee2013,Ling2014} exclusively focusing on using seeding promoter. \citeauthor{Lee2013} obtained \ce{MoS2} and \ce{WS2} FL on PTAS-treated substrates (\ce{C24H12K4O8}), where tiny ($\sim 200$ nm) seeds were found under AFM. More recently, \citeauthor{Ling2014} systemically investigated the role of seeding promoter in facilitating the nucleation of \ce{MoS2} monolayer.\cite{Ling2014} It was found various aromatic molecules are effective yet inorganic nanoparticles are not. The mechanism of thin film growth depends on the surface energy and chemical potentials of the deposited layers and their substrates. Layer growth is preferred when surface adhesive force is stronger than adatoms cohesive force. Seeding promoter probably lowered the surface energy by wetting, thus provided heterogenous nucleation sites. Continuous \ce{MoS2} ML is obtained by evaporating \ce{F16CuPc} of 2 \si{\angstrom} on desired receiving substrates. In each growth, about $10^{-10}$ mol PTAS was applied onto \ce{SiO2}-Si, which was rendered hydrophilic and gentle gas blow to distribute the solution evenly.

In short summary, this thesis stresses that there are several key factors in successfully synthesizing \ce{WS2} FL:
\begin{itemize}
\item \ce{WO3} powder size distribution and absolute amount (0.69 g \ce{WO3} in 10 mL acetone or IPA, inspired by the \ce{MoO3} nanoribbons usage);
\item Sulfur amount and heating method to ensure a constant sulfur-rich environment;
\item \ce{WO3} powder source to substrate distance, which is coupled to pressure and carrier gas flow in determining the transport (atmospheric, 3-5 mm);
\item temperature ramping and growth time (750-800 \si{\degreeCelsius}, 3-10 min);
\item 300 nm \ce{SiO2}-Si substrate to make OM flakes identification easier.
\end{itemize}

% CVD TMDC
\begin{landscape}
\begin{table}[htb]
\centering
\caption{TMDC FL methods summary}\label{tab:tmsgrowth}
{\footnotesize
\begin{tabular}{lp{2.5in}p{4.5in}}
\toprule
TMDC  &  precursor & growth condition (default temperature unit \si{\degreeCelsius}) \\
\midrule
\ce{MoS2} films \cite{Lee1994,Endler1999} & \ce{MoCl5}, \ce{H2S} & 1 kPa, temperature: 400-550 \si{\degreeCelsius}, 100/10/2.5 sccm for Ar, \ce{H2S} and \ce{MoCl5} flow\\
\addlinespace[0.5em]
\ce{MoS2} FL \cite{Zhan2012} & 1-5 nm Mo films on \ce{SiO2}, Sulfur & purging, RT-550@30 min, 550-750@90 min and hold for 10 min. Mo coating on Si did not work.\\
\ce{MoS2} FL \cite{Lin2012,Wang2013} & 4 nm \ce{MoO3} coating on sapphire  & reduced to \ce{MoO2} in \ce{H2} and Ar at 500 \si{\degreeCelsius}, sulfurization at 850-1000 \si{\degreeCelsius} \\
\addlinespace[0.5em]
\ce{MoS2} FL \cite{Liu2012a} & \ce{(NH4)2MoS4} in DMF solution transport by Ar bubbler or dip-coating onto subs &  annealing under Ar or Ar + Sulfur, total pressure 0.2-2 Torr, \\
 \addlinespace[0.5em]
\ce{MoS2} FL \cite{Wu2013} & \ce{MoS2} powder & Ar flow, 900 \si{\degreeCelsius} heating, pressure 20 Torr, 650 \si{\degreeCelsius} growth\\
 \addlinespace[0.5em]
\ce{MoS2} FL \cite{Mann2013,Najmaei2013,Ji2013} Rice & \ce{MoO3} powders or ribbons, Sulfur & Ar flow, 530-850 \si{\degreeCelsius}, total pressure 0.2-2 Torr, 5-30 min, mica or \ce{SiO2}-Si\\
 \addlinespace[0.5em]
\ce{MoS2} FL \cite{Lee2012b,Ling2014} & 18mg \ce{MoO3} powders, Sulfur,various seeding promoter on sub & 5sccm Ar, 650 \si{\degreeCelsius}, 3 min growth, atmospheric pressure, quick cooling\\

 \midrule
\ce{WS2} films\cite{Ballif1999,Brunken2008} & sputtering \ce{WS_{3+x}} on 10 nm Ni  & annealing under Ar for 1 h at 850 \si{\degreeCelsius} \\
\addlinespace[0.5em]
\ce{WS2} FL \cite{Berkdemir2013} & $\sim$1nm \ce{WO3} coating on 285 nm \ce{SiO2}-Si, 500 mg sulfur & 800 \si{\degreeCelsius} for 30 min, 100 sccm Ar, atmospheric pressure in \cite{Gutierrez2012} and 450 mTorr in \cite{Elias2013}. triangular flakes obtained\\
\addlinespace[0.5em]
\ce{WS2} ML \cite{Cong2013} & 1 mg \ce{WO3} powder on \ce{SiO2}-Si covered by another sub, $d\sim3$ mm, sulfur & 750 \si{\degreeCelsius}, slow heating, hold for 5 min, one-end sealed inner tube, 100 sccm Ar\\
\addlinespace[0.5em]
\ce{WS2} ML \cite{Zhang2013h} & \ce{WO3} powders, sulfur (separate heating) & 900 \si{\degreeCelsius}, sapphire subs, 225 mTorr, Ar 80 sccm and \ce{H2} 10 sccm, growth time 60 min, adjusting precursor and sapphire distance changing the coverage of \ce{WS2}, tube diameter: 1 in. 55 $\mu$m triangular flakes\\
\addlinespace[0.5em]
\ce{WS2} ML \cite{Peimyoo2013} & 1 mg \ce{WO3} powders, sulfur  & \ce{SiO2}/Si subs. Recipe A: 200 mg S,RT-550, Sulfur begin to melt, 550-800 \si{\degreeCelsius}, 5K/min, hold 10mins, 200 sccm Ar. Recipe B: sulfur separated heated at 250 \si{\degreeCelsius}. Total pressure: maybe atmospheric, tube diameter: 2 in. 5 $\mu$m triangular flakes \\
\ce{WS2} FL \cite{Lee2013}  & 1g \ce{WO3} powders, sulfur, \ce{SiO2}-Si subs treated with PTAS \ce{C24H12K4O8} and gentle gas blow & substrate facing down, APCVD, 800 \si{\degreeCelsius}, 5 min, 5 sccm Ar, fast heating. \\
\addlinespace[0.5em]
\ce{WS2} films \cite{Shanmugam2012a}   & 20 nm W on \ce{SiO2}-Si, sulfur & 750 \si{\degreeCelsius}, 200 sccm Ar, 1Torr. Annealing at 1000 \si{\degreeCelsius}, 25 nm thick \ce{WS2} film obtained \\
\bottomrule
\end{tabular}
}
\end{table}
\end{landscape}

\subsection{Raman on WS$_2$: Bulk, FL and NTs}\label{sec:ntram}

This section will discuss previous studies using Raman to characterize \ce{MS2} films, nanotubes, and FL structures. In each morphology, one or two key points will be highlighted for study in this thesis. The previous Raman efforts on \ce{MS2} film, especially the one using resonant conditions, provided critical insight into the electronic structure and lattice dynamics of these TMDC layered materials. A comparison of Raman between TMDC nanotubes and FL structures is given. For \ce{WS2} NTs, focus is on the asymmetry of $A_{1g}$ mode and its origin. And for \ce{WS2} FL structures, layer number dependent fingerprint is summarized, and some attempts on analysis the resonant Raman profiles of \ce{WS2} are made. It is worth noting that Raman technique proves to be extreme useful in characterizing CNT and graphene, i.e., tube diameter by assigning the RBMs (radial breathing modes) and G peaks position.\cite{Bonaccorso2013} And Raman spectroscopy also qualifies as an excellent tool to monitor tensile features of TMDC in both 2D and tubular forms.\cite{Tang2013} 

Raman spectra arise from the inelastic light scattering of optical phonons. In back scattering geometry, the photon wave-vector stands as $q = 4\pi\frac{n}{\lambda}$. The refractive index $\tilde{n}$ of \ce{WS2} at 532 nm is about $4.726 - 0.737i$, corresponding to a wave-vector about $1\times10^8$ \si{m^{-1}}. Compared to the size of Brillouin zone ($\pi/a = 10^{10}$ \si{m^{-1}}), off-resonant Raman could only probe phonon at $\Gamma$ point, where the phonon momentum ($\hbar k$) is close to zero. 

Before discussing the Raman spectra features, this thesis briefly recapitulates some symmetry notations and vibration modes. \ce{MoS2} is used as an example, and those definitions apply to \ce{WS2} as well. Hexagonal \ce{MoS2} belongs to space group $D_{6h}^4$, and the repeat unit in $c$ axis contains two layers, where sulfur atoms in one layer are directly above the molybdenum atoms in adjacent layers, which is often referred as 2H-\ce{MoS2}. Group theory predicts two infrared- and four Raman-active modes for 2H-\ce{MoS2}, which are mutually exclusive when the center of inversion is present. First it should be emphasized that in few layer structures, \ce{MoS2} with odd layers belong to different space group from that of even layers. Bulk MoS$_2$ and 2L-MoS$_2$ belong to the space group P6$_3$/$_{mmc}$ (point group D$_{6h}$). There are 18 normal vibration modes. The factor group of bulk and 2L-MoS$_2$ at $\vec{\Gamma}$ is D$_{6h}$. The atoms site groups are a subgroup of the crystal factor group. The correlation of the Mo site group D$_{3h}$, S site group C$_{3v}$, and factor group D$_{6h}$ allows one to derive the following irreducible representations for the 18 normal vibration modes at $\vec{\Gamma}$: $\vec{\Gamma}$= $A_{1g}+2A_{2u}+2B_{2g}+B_{1u}+E_{1g}+2E_{1u}+2E_{2g}+E_{2u}$, where $A_{2u}$ and $E_{1u}$ are translational acoustic modes, $A_{1g}$, $E_{1g}$ and $E_{2g}$ are Raman active, $A_{2u}$ and $E_{1u}$ are infrared (IR) active. As a contrast, 1L-MoS$_2$ has $D_{3h}$ symmetry with three atoms per unit cell. The irreducible representation of $D_{3h}$ gives: $\vec{\Gamma}$= $2A_2^{''}$+$A_1^{'}$+$2E^{'}$+$E^{''}$, with $A_2^{''}$ and $E^{'}$ acoustic modes, $A_2^{''}$ IR active, $A_1^{'}$ and $E^{''}$ Raman active, and the other $E^{'}$ both Raman and IR active (See Fig.~\ref{fig:ws2ramsch}). NL-MoS$_2$ has 9N-3 optical modes: 3N-1 are vibrations along the c axis, and 3N-1 are doubly degenerate in-plane vibrations. For rigid-layer vibrations, there are N-1 layer breathing modes (LBMs) along the c axis, and N-1 doubly degenerate shear modes perpendicular to it. When N is even, there are 0 Raman active LBMs and $\frac{N}{2}$ doubly degenerate shear modes. When N is odd, there are $\frac{N-1}{2}$ LBMs and N-1 doubly degenerate shear modes.\cite{Wieting1971,Zhang2013e} This discussion can be visualized in Table~\ref{tab:tmslattice}.

% irreducible representation
\begin{table}[htb]
\centering
\caption[Lattices vibration of \ce{MS2}]{Lattices vibration of \ce{MS2}, adopted from Ref.\cite{Molina-Sanchez2011}}\label{tab:tmslattice}
\begin{tabular}{lcccc}
\toprule
 $D_{6h}$   & $D_{3h}$ & Character &  Direction & Atoms  \\
\midrule
$A_{1g}$    &  $A_1^{'}$   & Raman     & (out of plane)  & S  \\
$E_{2g}^2$  &          &           & (in plane)      & M + S  \\
$E_{2g}^1$  &  $E'$    &           & (in plane)      & M + S  \\
$E_{1g}$    &  $E''$    &           & (in plane)      & S  \\
\midrule
$A_{2u}$    &  $A_2''$  & Infrared  & (out of plane)  & M + S  \\
$E_{1u}$    &          &           & (in plane)      & M + S  \\
\midrule
$A_{2u}$    &  $A_2^{''}$   & Acoustic  & (out of plane)  & M + S  \\
$E_{1u}$    &          &           &       &    \\
\midrule
$B_{2g}^2$  &          & Inactive  & (out of plane)  & M + S  \\
$B_{2g}^1$  &          &           & (out of plane)  & M + S  \\
$B_{1u}$    &          &           & (out of plane)  & S  \\
$E_{2u}$    &          &           & (in plane)      & S  \\
\bottomrule
\end{tabular}
\end{table}
The $E$ type phonon branches correspond to in-plane normal modes, while the $A$ type phonons result from out-of-plane vibrations. $A_{1g}$ mode is an out-of-plane vibration involving only the S atoms while the $E_{2g}^1$ mode involves in-plane displacement of transition metal and S atoms. The $E_{2g}^2$ mode is a shear mode corresponding to the vibration of two rigid layers against
each other and appears at very low frequencies ($<50$ \si{cm^{-1}} \cite{Zhang2013e}). The $E_{1g}$ mode, which is an in-plane vibration of only the S atoms, is forbidden in the backscattering Raman configuration. In 2H-type TMDC, the $A_{1g}$ mode is more sensitive to electrostatic doping, while $E_{2g}^1$ mode is more sensitive to strain, in which the FWHM of the peaks are indicator of external force quantity.\cite{Zhao2013}

\begin{figure}[htb]
\centering
\includegraphics[width=0.7\textwidth]{ws2_ramsch}
\caption[\ce{MS2} vibration symmetry]{\ce{MS2} vibration symmetry in bulk and monalyer, adopted from REF\cite{Ghorbani-asl}}
\label{fig:ws2ramsch}
\end{figure}

Lattice vibration of natural \ce{MoS2} crystal was studied by \citeauthor{Wieting1971} using infrared and Raman spectroscopy.\cite{Wieting1971} It was found the $E_{1u}$ IR mode and one $E_{2g}$ Raman mode are nearly degenerate in energy. 15 optical modes are allowed assuming 6 atoms in primitive cell. Refractive indices from reflectivity measurement were $n_0$= 3.9, $n_e$ = 2.5. \citeauthor{Stacy1985} studied \ce{MoS2} and \ce{WS2} Raman spectra using lasing energy close to the absorption edges.\cite{Stacy1985} Second order scattering from phonon with nonzero momentum is used to explain the rich Raman spectra. \citeauthor{Sourisseau1991} investigated the resonant Raman profiles in 2H-\ce{WS2} using ten different excitation wavelengths.\cite{Sourisseau1991} Dramatic intensity variation at 352 \si{cm^{-1}} was observed, which is assume to be of two-phonon signal nature, and corresponds to an overtone or combination band of phonons with non-zero momenta contributing to indirect gap absorption edge. \citeauthor{Sourisseau1991} assigned this phonon with non-zero momenta as $LA(K_5)$ type. The enhancement of the total Raman cross section at excitonic resonance in which excitons serve as the intermediate state is stronger compared to that of interband resonance. The strong enhancement at excitonic resonance is attributed to the characteristics of excitons in layered materials such as large binding energy, enhanced oscillator strength, and small damping constant.\cite{Zhao2013} \citeauthor{Chung1998} grew \ce{WS2} film using \ce{W(CO)6} and \ce{H2S} precursor.\cite{Chung1998} Raman spectra ($\lambda=632 nm$) on films with non-parallel orientation revealed the presence of shoulder mode under $A_{1g}$, which is assigned to LA and TA phonon coupling. This coupling process stems from disorder-activated zone boundary phonons. A further discussion on this non-symmetric feature of $A_{1g}$ mode will be continued in section~\ref{sec:ntram}.

% WS2 Raman assignments
\begin{table}
  \centering
  \caption{\ce{WS2} symmetry assignment}  \label{tbl:ws2raman}
  \begin{tabular}{ccccc}
    \toprule
    &&\multicolumn{3}{c}{Raman Shift (\si{cm^{-1}})}\\
    \cmidrule(l){3-5}
    Symmetry                &  & \ce{WS2} ML\cite{Cong2013}  & \ce{WS2} NT \cite{JMR7990865}  & \ce{WS2} bulk \cite{Sourisseau1991} \\
    \midrule
          $E_{2g}^2(\Gamma)$ &      & 27.5\textsuperscript{\emph{a}}&             &  27.4    \\
    $LA(M)-E_{2g}^2(\Gamma)$ &      & 148.3                        &              &    \\
       TBD                   &      &                              & 153          &      \\
         $E_{2g}^1(M)-LA(M)$ &      &                              & 172          & 173  \\
    LA(M)                    &      & 174.8                        & 172          &       \\
    LA(K)                    &      & 192.4                        &              &  193 \\
    $LA(M)+E_{2g}^2(\Gamma)$ &      & 203                          &              &     \\
    $LA(K)+E_{2g}^2(\Gamma)$ &      & 213.9                        &              &  212  \\
    $A_{1g}(M)-LA(M)$        &      & 230.9                        & 230          &  233  \\
    $2LA(M)-3E_{2g}^2(\Gamma)$ &    & 264.2                        & 262          &  267  \\
    $2LA(M)-2E_{2g}^2(\Gamma)$ &    & 295.4                        & 294          &  297   \\
    $2LA(M)-E_{2g}^2(\Gamma)$ &     & 322.9                        &              &  325   \\
               $E_{2g}^1(M)$ &      & 343.1                        &              &      \\
    2LA(M)                   &      & 350.8                        & 350          &  352\\
          $E_{2g}^1(\Gamma)$ &      & 355.4                        & 350          &  356 \\
    $2LA(M)+2E_{2g}^2(M)$    &      &                              & 381          &  381   \\
     LA + TA \cite{Sourisseau1991} or $B_{1u}(\Gamma)$\cite{Staiger2012}  &      &       &   &  416 \\
          $A_{1g}(\Gamma)$   &      & 417.9                        & 416\textsuperscript{\emph{b}} &  421\\
               3LA(K)        &      & 577                          &              &      \\
       $ LA(M)+ A_{1g}(M)$   &      & 584                          & 581          &  585 \\
    4LA(M)                   &      & 704                          &              &  703\\
    \bottomrule
  \end{tabular}

  \textsuperscript{\emph{a}} Calculated from column values;
  \textsuperscript{\emph{b}} \citeauthor{JMR7990865} probably made incorrect assignment of 416 peak.\cite{JMR7990865}
\end{table}

Raman technique has also provided much insight into the few layer \ce{MS2}. \ce{MS2} layer numbers are readily identified by the Raman shift distance between $A_{1g}$ and $E_{2g}^1$ mode for \ce{MoS2}\cite{Buscema2013} and \ce{WS2}\cite{Berkdemir2013}. Yet due to the relative small shift of $A_{1g}$ mode and little shift of $E_{2g}^1$ mode in \ce{WS2} FL, the frequencies distance might not qualify as an unambiguous way to distinguish layer numbers. Yet the resonant Raman profile on \ce{WS2} exhibit unique features between the intensity of 2LA and $A_{1g}$ mode,\cite{Berkdemir2013,Zhao2013} which provide another routine to assure the monolayer presence. However, this thesis notices some discrepancy in de-convolution of \ce{WS2} resonant profile between 300 and 400 cm$^{-1}$, i.e., the presence of $E_{2g}^1(M)$ mode at about 344 cm$^{-1}$.\cite{Peimyoo2013,Cong2013,Berkdemir2013} Rigid assignments of this mode still requires further theoretical\cite{Ataca2012} and experimental efforts. In addition, \citeauthor{M2013} reported temperature dependent of 1L \ce{WS2}.\cite{M2013} When the temperature increase from 77 K to 623 K, $A_{1g}$ shift from 420 to 416.5 cm$^{-1}$. Interestingly in this report 2LA/$A_{1g}$ ratio (514 nm excitation) seems less than unit. The spectra were obtain from mechanically exfoliated \ce{WS2} lying on 300 nm \ce{SiO2}-Si substrate. Moreover, The $A_{1g}$ and $E_{2g}^1$ intensities ratio exhibit reverse behaviors under 532 and 632 nm excitation. This is caused by the different cross-section enhancement for a specific excitation condition. The A and B excitonic absorption in \ce{WS2} mainly arises from the $d_{xy}$ and $d_{x^2 - y^2}$ states to $d_{z^2}$ states of tungsten atoms. Thus, electrons excited by 633 nm laser have a character of tungsten $d_{z^2}$ orbitals aligned along the $c$ axis perpendicular to \ce{WS2} basal plane. Since $A_{1g}$ mode involves out of plane displacement along $c$ axis, $A_{1g}$ phonons could couple more strongly with $d_{z^2}$ states than that of $E_{2g}^1$ phonons. As a result, $A_{1g}$ mode is stronger than $E_{2g}^1$ mode at 633 nm resonance.\cite{Zhao2013} However, the reverse effect for 532 nm excitation could not be well explained using the above argument. This may be caused by electron-phonon coupling with other inter-band transition electrons.


\citeauthor{Dobardzic2005} calculated \ce{MoS2} \gls{swnt} phonon dispersion. The dependence of wavenumbers and their displacement on chirality and diameter were discussed. The calculation method enables studying lattice dynamics with NT diameter up to 50 nm. The chiral vector $(n_1, n_2)$ is defined within the molybdenum plane. Symmetry assignment is zigzag when $(n,0)$, armchair when $(n,n)$ and chiral when $(n_1, n_2), n_1>n_2$. \citeauthor{Dobardzic2006} theoretically presented Raman scattering of any polarization on \gls{swnt} of \ce{WS2} and their dependence on diameter (1-20 nm) and chiral angle. The author assigned 351 cm$^{-1}$ as $E_u$ for \ce{WS2} NT.\cite{Dobardzic2006} \citeauthor{Ghorbani-Asl2013} discuss the electronic and vibrational properties for large diameter \ce{WS2} NTs\cite{Ghorbani-Asl2013}. Single-walled NT is approximated by 1H monolayer and others by 2H bulk structure. It was found that large-diameter nanotubes can be approximated with layered systems as their properties should be nearly the same at the scale. Only hypothetical SWNTs, and possibly MWNTs with alternating layer compositions, may show direct band gaps. Slight mechanical deformation of the SWNTs would result in a change of the direct band gap back to the indirect one, located between $\Gamma$ and $K$ high-symmetry points, similarly to the monolayers. As for 2D materials, quantum confinement to single-walled tubes would result in direct band-gap semiconductors with $\Delta$ occurring at the $K$ point. Single-walled tubes exhibit slightly softer out-of-plane $A'$ and stronger in-plane $E'$ modes. Those results indicate that the weak interlayer interactions in MS$_2$ materials cannot be associated with the van der Waals interactions only, but most probably with Coulomb electrostatic interactions as well.

Raman signatures of \ce{WS2} nanotubes show distinct features to the spectra of their bulk counterpart. \citeauthor{JMR7990865} observed a new line at 152 cm$^{-1}$ in \ce{WS2} NT, which is absent in 2H-\ce{WS2}.\cite{JMR7990865} Another feature is a emerging shoulder on the low energy side of $A_{1g}$ mode at about 416 cm$^{-1}$. This has been attributed to a combination mode of LA + TA phonons from the $K$ point of Brillouin zone.\cite{Sourisseau1991} The shoulder mode associated with $A_{1g}$ is attributed to LA + TA. As pressure increase from 0 GPa to 18 GPa, these two bands, both shifting to higher wavenumbers, first separate and then recombine. It was assumed the compression mainly occurs in $c$ axis, so the stiffening of $A_{1g}$ is anticipated. A more prominent feature is the resonance profile broadening the shape of $E_{2g}^1$ mode, which is often assigned to 2LA mode. Yet there is different opinion on these assignments. \citeauthor{Molina-Sanchez2011} label the 350 \si{cm^{-1}} band as \ce{E_{1u}} instead of 2LA.\cite{Molina-Sanchez2011} And recent theoretical investigations suggest 416 \si{cm^{-1}} peak is inactive $B_{1u}$ mode\cite{Molina-Sanchez2011,Ataca2012}, which is the Davydov doublet with $A_{1g}$ mode. \citeauthor{Staiger2012} adopted these assignments in studying the resonance Raman profile of \ce{WS2} NTs.\cite{Staiger2012} It was found that
\begin{enumerate}
\item $B_{1u}$ mode arise from curvature and structural disorder;
\item $B_{1u}/A_{1g}$ intensity ratio strongly depends on excitation, and exceeds unity when excitation energy less than 1.9 eV; and
\item  An excitonic transition energy of NT is found have a local minimum at about 50 nm, (layer number probably \textgreater 10), and increase either way. Yet all below the bulk value.
\end{enumerate}

\citeauthor{Krause2009} also measured the resonant Raman on \ce{WS2} nanotubes and found a split within 420 \si{cm^{-1}} region, which is labelled as $D-A_{1g}$ mode in analog with the similar defect mode of graphene.\cite{Krause2009} This  $D-A_{1g}$ mode was found enhanced as diameter of \ce{WS2} NTs decrease. This thesis work will use $B_{1u}({\Gamma})$ mode to interpret this emerging line at about 416 \si{cm^{-1}} of \ce{WS2} NTs, and adopt the 350 \si{cm^{-1}} as 2LA(M). \citeauthor{Krause2009a} confirmed $B_{1u}$ mode arise from the inherent structure of \ce{WS2} nanomaterials instead of surface layer effect. It is also worth noting that $A_{1g}$ is stronger than 2LA under 632 nm yet weaker under 532 nm excitation, similar to previous discussion of FL scenarios.\cite{Krause2009a} Similar observation of Raman spectra on the \ce{WS2}-\ce{WO3} structures was found in this study, as discussed in Sec.~\ref{tms:raman}. \citeauthor{Rafailov2005} estimated the orientation dependence of resonant raman on one MWNT \ce{WS2} attached to the cantilever tip of AFM.\cite{Rafailov2005} Antenna effect lead to optical transition occurring only for polarization parallel to nanotube axis. And therefore resonance Raman intensity of SWNT varies as nanotube orientation. This dependence may provide a routine to distinguish different chiral NTs. Polarized Raman spectra (632 nm excitation) is obtained, showing $A_{1g}$ and $E_{2g}$ sharing the same polarization behavior. \citeauthor{Virsek2007} investigated the Raman scattering ($\lambda=632$ nm) of \ce{WS2} NTs.\cite{Virsek2007} The silicon peak at 520 cm$^{-1}$ is used for calibration. Up-shift of $A_{1g}$ and $E_{2g}$ modes (i.e., 420 to 423 cm$^{-1}$ at $A_{1g}$ mode) were observed, which is attributed to the strain in 3R stacking layers.
