% !TEX TS-program = pdflatex
% !TEX encoding = UTF-8 Unicode

% This is a simple template for a LaTeX document using the "article" class.
% See "book", "report", "letter" for other types of document.

\documentclass[11pt]{article} % use larger type; default would be 10pt

\usepackage[utf8]{inputenc} % set input encoding (not needed with XeLaTeX)

%%% Examples of Article customizations
% These packages are optional, depending whether you want the features they provide.
% See the LaTeX Companion or other references for full information.

%%% PAGE DIMENSIONS
\usepackage{geometry} % to change the page dimensions
\geometry{a4paper} % or letterpaper (US) or a5paper or....
% \geometry{margin=2in} % for example, change the margins to 2 inches all round
% \geometry{landscape} % set up the page for landscape
%   read geometry.pdf for detailed page layout information

\usepackage{graphicx} % support the \includegraphics command and options

% \usepackage[parfill]{parskip} % Activate to begin paragraphs with an empty line rather than an indent

%%% PACKAGES
\usepackage{booktabs} % for much better looking tables
\usepackage{array} % for better arrays (eg matrices) in maths
\usepackage{paralist} % very flexible & customisable lists (eg. enumerate/itemize, etc.)
\usepackage{verbatim} % adds environment for commenting out blocks of text & for better verbatim
\usepackage{subfig} % make it possible to include more than one captioned figure/table in a single float
% These packages are all incorporated in the memoir class to one degree or another...

%%% HEADERS & FOOTERS
\usepackage{fancyhdr} % This should be set AFTER setting up the page geometry
\pagestyle{fancy} % options: empty , plain , fancy
\renewcommand{\headrulewidth}{0pt} % customise the layout...
\lhead{}\chead{}\rhead{}
\lfoot{}\cfoot{\thepage}\rfoot{}

%%% SECTION TITLE APPEARANCE
\usepackage{sectsty}
\allsectionsfont{\sffamily\mdseries\upshape} % (See the fntguide.pdf for font help)
% (This matches ConTeXt defaults)

%%% ToC (table of contents) APPEARANCE
\usepackage[nottoc,notlof,notlot]{tocbibind} % Put the bibliography in the ToC
\usepackage[titles,subfigure]{tocloft} % Alter the style of the Table of Contents
\renewcommand{\cftsecfont}{\rmfamily\mdseries\upshape}
\renewcommand{\cftsecpagefont}{\rmfamily\mdseries\upshape} % No bold!

%%% END Article customizations
%%---added by tao sheng
\usepackage{subfig}
\usepackage{float}
\graphicspath{{../gallery/}} %--added by authior

\usepackage{xcolor}

\usepackage{amsmath}
%\numberwithin{equation}{section} % reset counters at each start of section
%\numberwithin{figure}{section}
\providecommand*{\ud}{\mathrm{d}}

\usepackage[font=small, labelfont=bf]{caption}

\usepackage{pdfpages}
\usepackage{placeins}
\usepackage{siunitx}
\usepackage[version=3]{mhchem}
%\usepackage{mhchem}
\usepackage{mathpazo}

\usepackage[affil-it]{authblk}

\usepackage[super, comma,sort&compress]{natbib}
\providecommand*{\bibpath}{E:/spring2012/Ubuntu/Latex/Mendeley_Bib_lib}


\usepackage{hyperref} % put this package at the last of all other packages
\hypersetup{
colorlinks,%
citecolor=black,%
filecolor=black,%
linkcolor=black,%
urlcolor=black
} % make all links black

%%% The "real" document content comes below...


\begin{document}
\title{Tungsten Disulfides and Tungsten Oxides Core-Shell NWs}
\author{Tao Sheng}
%\date{} % Activate to display a given date or no date (if empty),
         % otherwise the current date is printed
\affil{Department of Physics and Optical Science\\}
\author{Youfei Jiang}
\author{Haitao Zhang%
\thanks{Email address:hzhang3@uncc.edu; Corresponding author}}
\affil{Department of Mechanical Engineering and Engineering Science\\
The University of North Carolina at Charlotte\\
9201 University City Boulevard, Charlotte NC 28223}

\maketitle

\section{Introduction}
% phtotcatalytic activity



\section{Experimental}

\textbf{Synthesis}

\ce{WS2}-\ce{WO_x} core-shell NWs were synthesized using direct sulfurization of \ce{WO3} NWs (Fig.~\ref{fig0}). Tungsten trioxides NWs were synthesized by a seeded growth approach in a home-built CVD apparatus, which has been successfully employed to prepare borides\cite{Amin2009a}, titanium oxide\cite{Amin2007}, tungsten oxides\cite{Zhang2010} and \ce{MoO3}. The seeded growth details are provided in supporting information. The as-grown \ce{WO3} NWs were loaded into the center of heating furnace, and $\sim200$mg sulfur (Alfa Aesar 10785, 99.5\%) was positioned just outside the upstream edge of furnace, where the maximum temperature was about 240 \si{\degreeCelsius}. After pumping down, the reaction chamber was flushed two times to expel residual air. A cold trap filled with liquid nitrogen in downstream was used to collect possible sulfur precipitation. Then the furnace was heated to 750 \si{\degreeCelsius} in 30 minutes, held for 15 minutes, and allowed to naturally cool down to room temperature. During entire growth process, 30 sccm Ar was used as carrier flow.

\begin{figure}[htb]
\centering
\includegraphics[width=0.7\textwidth]{ws2_coreshell_setup}
\caption{Schematic illustration of sulfurization process.}
\label{fig0}
\end{figure}

\textbf{Characterization}

The morphology and composition of the as-synthesized samples were analyzed by scanning electron microscopy (SEM, JEOL JSM-6480) and the associated energy dispersive X-ray spectroscopy unit(EDS, Oxford Instrument INCA). Crystal structures were characterized using X-ray diffraction (PANXpertX’pert Pro MRD with Cu K$\alpha$ line at $\lambda$=1.5418\AA) and transmission electron microscopy (TEM, JEOL JEM-2100 \ce{LaB6} operated at 200kV). A Raman microscope setup (Horiba Scientific, Labram HR800) were used to detect Raman spectra. The spectra was acquired using 532 nm laser (0.3 mW) and 200 seconds acquisition time.

To perform TEM-Raman integrated study, the core-shell specimen was first dispersed into acetone solution, and dipped onto polydimethylsiloxane (PDMS) support. A micromanipulator was used to pick up the longer core-shell NWs and to transfer them onto TEM grids. The NWs were first examined using TEM to identify the \ce{WS2} layers configuration and the associated geometrical orientation. Then the core-shell NWs on TEM grids were examined under Raman.

To perform TEM-Raman integrated study, the core-shell nanowires need to be transferred to TEM grid first. So the core-shell specimens were dispersed into acetone, and then dipped onto polydimethylsiloxane (PDMS) substrate. Some long core-shell nanowires on PDMS substrate were picked up and transferred to TEM grids by using a home-built micromanipulator under optical microscope.

$6\mu L$ Methylene Blue\footnote{\ce{C16H18N3SCl} molar mass 319.85g/mol.} (Alfa Aesar 42771, 1\%w/v = 31.26 mM) in diluted into 20 mL DI water, giving ultimate concentration as $9.38 \mu M$. The measured absorbance spectrum is shown in supporting information, with maximum absorbance 0.42 at 664nm. Taking the \ce{MB^+} molar extinction coefficient $\epsilon$ as 95000 $M^{-1}cm^{-1}$,\cite{Cenens1988} then the measured concentration is $c = \frac{A}{\epsilon l}$ is $4.37 \mu M$.


\section{Results and Discussion}

\textbf{Seeded growth}

% sg sem
\begin{figure}[htb]
\centering
\subfloat[]{\label{fig:sga}\includegraphics[width=0.4\textwidth]{wox_sg_a.jpg}}\hspace{0.04\textwidth}
\subfloat[]{\label{fig:sgb}\includegraphics[width=0.4\textwidth]{wox_sg_b.jpg}}
\caption[Characterization of seeded growth \ce{WO3}: SEM]{Characterizations of seeded growth. (a) SEM graphs of \ce{WO3} NWs on \ce{SiO2/Si} substrate. (b) A high magnification view showing uniform NW growth and close-up view of one NW. }
\label{fig:woseedsem}
\end{figure}

As shown in Fig.~\ref{fig:sgb}, dense NWs array was obtained on tungsten powder seeds with individual wires of length up to 5 $\mu$m and diameter about 50 to 200 nm, according to the estimation made in the close-up view. Each tungsten powder stood as independent growth site (Fig.~\ref{fig:sga}) with island-layer growth on the substrates, a common feature without using tungsten powder as seed under current experimental conditions. It was occasionally observed that NWs growth was initiated adjacent some tungsten powders. We suspect that this phenomenon was correlated to the local trap of vapor flow since it was more often found among the enclosed area by tungsten powders. We also found that the diameter of NWs decrease as the distance between powders and upstream edge increases. This is a combination effect of lower temperature and reduced \ce{WOx} growth species supply. Similar phenomena were observed in other studies.\cite{Thangala2007}

% sg raman xrd
\begin{figure}[htb]
\centering
\subfloat[]{\label{fig:sgxrd}\includegraphics[width=0.45\textwidth]{wox_xrd_1}}\hspace{0.04\textwidth}
\subfloat[]{\label{fig:sgram}\includegraphics[width=0.45\textwidth]{wox_raman_1}}
\caption[Characterization of seeded growth \ce{WO3}: XRD and Raman]{ (a) XRD pattern of as-prepared sample indicating the \ce{WO3} phase and the presence of metallic core. (b) Raman spectrum on NWs region showing the feature of \ce{WO3}.}
\label{fig:woseedxrd}
\end{figure}

Fig.~\ref{fig:sgxrd} is the XRD spectrum of one typical sample. The peaks under circular symbol were identified to be the monoclinic \ce{WO3} phase (ICDD PDF 01-083-0950,\emph{a}=7.30084\AA, \emph{b}=7.53889\AA, \emph{c}=7.6896\AA, $\beta$=90.8920$^\circ$), while the peak under the triangular symbol was indexed to cubic tungsten phase (ICDD PDF 04-16-3405, \emph{a}=3.157\AA). This metallic core was anticipated considering the oxygen flow levels (Supporting info for oxidation experiment). Micro-Raman scattering spectroscopy was performed on as-synthesized sample as well. During Raman examination, the laser spot was carefully focused onto the NWs on powders and several inspections on different positions were observed to ensure the reproductivity of spectra data. As shown in Fig.~\ref{fig:sgram}, five distinct bands were well resolved, with its peaks located at 131, 265, 327, 711 and 803 \si{cm^{-1}}, respectively. This pattern closely matches previous studies on \ce{WO3} lattice dynamics.\cite{Salje1975a,Dixit1986} The high background level probably arises from the metallic core.

\textbf{core shell phase and composition}

\begin{figure}[htb]
\centering
\includegraphics[width=0.7\textwidth]{WS2_sem_edx.jpg}
\caption{ Morphology and composition analysis on as-synthesized core-shell structures. (a) SEM images showing the NWs almost retain their original shapes after sulfurization, and (b) EDX spectroscopy revealing the presence of sulfur element after sulfurization.}
\label{fig:ws2sem}
\end{figure}

%fig TEM showing CS planes
\begin{figure}[htb]
\centering
\includegraphics[width=0.7\textwidth]{ws2_coreshell_tem.jpg}
\caption{HRTEM images on one core-shell nanowires showing (a) the multilayer capping and void formation in \ce{WO_x} core, and (b) uniform \ce{WS2} layers on nanowire body and CS planes formation in core region.}
\label{fig:ws2tem}
\end{figure}
% fig Raman TEM
\begin{figure}[htb]
\centering
\includegraphics[width=0.7\textwidth]{ws2_coreshell_integ.jpg}
\caption{TEM-Raman integrated results on one core-shell nanowires. (a)-(c) \ce{WS2} layer number variation from bulk to 2L. (d) Associated Raman spectra with site1-3 corresponding to HRTEM images (a)-(c), respectively. (e) Multi-peak Lorentzian fitting on Raman spectrum site 3 with inset showing layer number dependent intensities ratio of $I_{2LA}/I_{A_{1g}}$ under 532 nm excitation.}
\label{fig:ws2ram}
\end{figure}


\subsection{Photocatalytic}

oxygen plasma treatment on HF-etched Si (001). reaction among $e$, \ce{O^+}, \ce{O2^+}, \ce{O^-},\ce{O2}. \ce{OH}-terminated surface obtained.\cite{Habib2010}


MB is a heterocyclic aromatic dye which is blue colored in oxidizing environment. Upon reduction, MB is turned into colorless leuco MB. This can be used as an oxygen indicator in food industry. Photo-bleaching of MB can be also due to its leuco formation rather than total decomposition. Photocatalytic decomposition can be minimized by keeping the solution at acidic condition (PH = 4), which will limit the formation of oxidative hydroxyl radicals (E = 2.8 eV vs normal hydrogen electrode). Oxygen dissolved in the solution play a key role in conversion of LMB to MB under visible light. Purging with \ce{N2} for 20mins can remove dissolved oxygen.\cite{Wang2014a}

\cite{Yoneyama1972} MB to LMB (\ce{C16H19N3S}) in aqueous solution upon illumination of \ce{TiO2}. The colorimetric analysis was performed in a glove box under nitrogen atmosphere. The absence of oxygen is important to prevent the oxidation of LMB to blue MB.
\[
\cee{MB^+ + H2O + H^+ \rightarrow MBH3^{2+} + 1/2O2}
\]
where MB represents the uncharged center of MB molecule.

\cite{Takizawa1978} common wisdom expect that a dye incapable of injecting an electron at the excited state to CdS. MB, which process N-methyl groups in its molecular structure and does not sensitize CdS is an exemplary candidate. quantum efficiency is defined as probability of MB converted to azure B per incident photon. QE of CdS to MB decomposition is reduced in nitrogen bubbling treated solutions, indicating the necessity of oxygen. Two possible mechanisms: a) adsorbed oxygen acts as a trap for the conduction electron and prevent the accumulation of negative charge within space charge region of CdS, supported by the formation of \ce{O2^-} in excitation of CdS in aqueous suspension.

ref 16, MB aqueous solution stability. Liquid chromatogram, azure B (trimethylthionine), and thionine. Electrochemical measurement,

\cite{Matthews1989} MB adsorption.  photocatalytic oxidation of MB by \ce{TiO2} film. photo-oxidation reaction occurs at the surface of photocatalyst. Mb molar extinction coefficient was found to be 66700 1/cm 1/M. Langmuir adsorption isotherm
\[
[MB]_{ads} = \frac{k_1 k_2 [MB]}{1 + k_1[MB]}
\]
and integrated form of Langmuir adsorption isotherm
\[
t = \frac{1}{k_1K} In\frac{[S]^0}{[S]} + \frac{1}{K}([S]^0 - [S])
\]
where $K = k_2 \phi N T_r$, with $\phi$ as quantum yield, N as total absorbed photons, and $T_r$ as rate of transport.
\[
\cee{C16H18N3SCl + 25.5O2 \rightarrow 16CO2 + 6H2O + 3HNO3 + H2SO4 +HCl}
\]
which indicates the total oxidation of $10 \mu M$ MB would exhause the ambient oxygen concentration of initially air-equilibrated solutions (about $250 \mu M$ ). ref 28 Thus the transport of both oxygen and MB to the photocatalyst surface are anticipated to be key factors.

\cite{DeTacconi1997} photoelectrochromism at \ce{TiO2}/MB interface and its control. Efficient capture of photogenerated holes by a reducing agent is crucial to the reversibility of bleach-recoloration transition. This transition is kinetically dictated by electron transfer. Holes transfer is not desired.

256 nm band is associated to the presence of LMB. LMB formation is not favored at alkaline pH values in aqueous solution. The OH radicals are generated either with the surface hydroxyl groups on \ce{TiO2} or with water, and its high oxidizing powder cause photocatalytic decomposition of the dye.

An elementary step in decomposition of MB is N-dealkylation, which is preceded by radical cation formation.\cite{Takizawa1978} This radical cation can be spectroscopically monitored by the presence of 520nm band for MB. In MB absorption spectrum, 664 and 614 nm band ratio is related to monomer and dimer relaxation.
\begin{align}
\cee{TiO2 &\rightarrow e_{CB}^- + h_{VB}^+ \\
h_{VB}^+ + red &\rightarrow ox\\
MB^+ + 2e_{CB}^ + H^+ &\rightarrow LMB}
\end{align}

Measure the ratio between 614 and 663 nm before and after adding WS2 can indicate the adsorption of monomer and dimer MB.


\cite{Mills1999} MB can act as sacrificial electron acceptor in the reduction to leuco form. The decomposition is favored under oxygen-rich environment. MB feature peaks at 663, 614 and 292 nm, and $\epsilon_{660}=10^5 M^{-1}cm^{-1}$. The doubly reduced form of MB, LMB has feature peak at 256 nm. The singly reduced form of MB, \ce{MB.^-} is pale yellow, with peak at 420nm.
\begin{align}
\cee{MB + e_{CB}^- &->[pH<7] MB^{.-} \\
2MB^{.-} &\rightarrow MB + LMB\\
O2 + e_{CB}^- &\rightarrow O2^{.-}}
\end{align}

The oxidized form of MB, \ce{MB.^+} has peak at 520nm, which is stable in acidic solution, but decomposes irreversibly in slight alkaline solution(pH = 9).
thionine peaks at 600nm.
MB forms dimers in aqueous solution,
\ce{
2MB <=>[K_D] (MB)_2
}
A typical value of $K_D$ is 3970 1/M. A quadratic equation can be solved to obtain the monomer concentration:
\[
2K_D [MB]^2 + [MB] - [MB]_{total} = 0
\]
MB adsorption on metal oxides. Monomer size is less than 1.5nm.
Logarithmic acid dissociation constant $pK_a= -\log_10 \frac{[A^-][H^+]}{[HA]}$. The oxidation potential for \ce{H2O}-\ce{O2} couple is 1.23V and 0.817V versus NHE at pH 0 and pH 7, respectively.


\ce{MB + SED ->[TiO2][h\nu \geq 3.2eV] LMB + SED^{2+}}

\ce{2LMB + O2 \rightarrow 2MB + 2H2O}

\cite{Lee2003a}
\begin{align}
\cee{ 2LMB &->[UV] LMB^*\\
2LMB^* + O2 &\rightarrow 2MB^+ + 2OH^-}
\end{align}

\cite{Breault2013} \ce{TiO2} on MB degradation rate. aliquots, adsorption and desorption equilibrium and how to measure it. Discuss the distinguish of MB and LMB by nitrogen and oxygen purging. The author also mention the pH value change is an indictor of complete mineralization. The formatin of superoxide and hydroxyl radicals has been experimentally determined by radical quenchers.
\ce{I^-} is a scavenger for VB holes, and \ce{.OH}. Adding 2mM KI will reduce the rate.

\begin{align}
\cee{ WS2-WO3 + h\nu &-> e^- + h^+\\
O2 + e^- &-> O2^{.-}\\
OH_{surface}^- + h^+ &-> .OH\\
MB^+ + O2^{.-} + .OH-> \text{degradation phases}
}
\end{align}

Energy diagram level alignment


 Transition metal anchored to \ce{TiO2} review.\cite{Ardo2009} molecular approach, chemical bonds afford large energy storage capacities. 
\begin{enumerate}
   \item photo excitation of dye sensitizer to form a molecular excited state;
   \item electron injection from HOMO to CB of semiconductor to cause e-h separation;
   \item oxidized(loss of electron) sensitized is compensated by external donor;
   \item external donor is compensated by the CB electron after performing useful works;
   \item this is a typical operation of Gratzel cell. 
\end{enumerate}

Metal-to-ligand charge transfer (MLCT) ($d \rightarrow \pi^*$) 

\clearpage
\bibliographystyle{unsrtnat}
\bibliography{\bibpath/Titanium,\bibpath/VLS,\bibpath/ECD,\bibpath/tungsten_newandgood,\bibpath/ACSnano,\bibpath/tungsten_old,\bibpath/Raman,\bibpath/Molybdenum,\bibpath/tungsten_cl}


\end{document}

\si{\degreeCelsius}
\AA


\clearpage
\begin{table}[htb]
\centering
\caption{Source of figures }\label{tab:sof}
\begin{tabular}{l|c|c|r}
\hline
No of Fig & sub index & sample date & remarks\\
\hline
1        & a & 013113 & 500 \\
1        & b & 012913 & 600 \\
1        & c & 020413 & 650 \\
1        & d & 013013 & 750 \\
\hline
2       & a & 012913 & 1sccm \\
2       & b & 020513 & 2sccm \\
2       & c & 020613 & 3sccm \\
2       & d & 021113 & 10sccm \\
\hline
3       & a & 012913 &  3N\\
3       & b & 042313 &  3N5\\
3       & c & 041013 &  4N5\\
3       & d & 041113 &  5N\\
\hline
4       & a & 021513 &  8mm-x500\\
4       & a inset & 021513 &  8mm-x10k\\
4       & b & 021513 &  13mm-x500\\
4       & b inset & 021513 &  13mm-x10ka\\
\hline
5   &   xrd & 021513 & \\
\hline
6  & raman & 060313 or 021513 & \\
\hline
7 & TEM & B10, B6 &
\end{tabular}
\end{table}


A low recombination rate is preferred for high photocatalytic efficiency. The simultaneous migration of electrons and holes.
\textbf{\ce{WS2}-\ce{WO3}}: 1 kW light source(Hg, or Xe lamp), photon flux, phenol (\ce{C6H5OH}, 94.1g/mol, MP 40C)concentration is 20 mg/L, hydroxyl group. The quantitative analysis of phenol was performed via a standard colorimetric method.\footnote{\url{http://omlc.ogi.edu/spectra/PhotochemCAD/html/072.html}}
\citeauthor{DiPaola1999} prepared \ce{WS2}-\ce{WO3} mixture in two methods, sulfurization of \ce{WO3} and oxidation of \ce{WS2},with the latter are more active.
\citeauthor{DiPaola1999} also concluded that the actual efficiency of mixed \ce{WS2}-\ce{WO3} catalysts depends on the ratio of each composition present of the surface of the particles, and the maximum of photoactivity is obtained with 40-50\% surface molar ratio of \ce{WS2}.

ref 25, 28 and 41.

\textbf{\ce{MoO3}}:

\citeauthor{Sreedhara2013} studied the kinetics of photodegradation of methylene blue\footnote{\ce{C16H18N3SCl},319.8 g/mol, MP: 100C accompanied with decomposition \url{http://en.wikipedia.org/wiki/Methylene_blue}} dye by few layer \ce{MoO3}.
For the photodegradation method, it was stated that `` the samples were collected after the photoreaction had been centrifuged for 5 min to remove the photocatalyst before UV-Vis measurement.''



