
\chapter{Transition metal dichalcogenides}

\section{literature review}


Shortly after the the discovery of carbon nanotube (CNT) in 1991, its transition metal dichalcogenides (TMDC) counterparts (\ce{WS2}) was synthesized in 1992. In the following decade, other TMDC and several metal oxides nanotubes were demonstrated as well. \citeauthor{Rao2003} did an excellent job to review these inorganic nanotubes.\cite{Rao2003} Here we just recapitulate the main categories of synthesis methods. 
\begin{enumerate}
\item Reducing \ce{MO3} in \ce{H2} and \ce{H2S} atmosphere\footnote{M = Mo, W in this work}
\item Direct decomposition of \ce{MS3}
\item Decomposition of the ammonium salt \ce{(NH4)2MS4}
\item Using CNT as templates, or arc, laser ablation
\end{enumerate}

Except the aforementioned methods, another important reaction in CVD domain is
\[
\cee{MoCl5(WCl6) + S \rightarrow MoS2(WS2) + S2Cl2}.
\]
This reaction has been explored by several authors,\cite{Stoffels1999} and could be used under atmospheric pressure.\cite{Li2004a}

All these methods can be applied to the growth of TMDC few layer structures, either directly or with some modification or combination, as we shall discuss later. Among all the members of TMDC nanotubes, \ce{WS2} NT is probably the one most well investigated.\cite{Tenne1998,Frey1998,Frey1999,Rothschild2000,Zak2000} The reaction mechanism of \ce{WO3} with \ce{H2}/\ce{H2S} has been thoroughly studied \cite{Feldman1998} and high yield synthesis approach has been established.\cite{Margolin2004}
It is worth noting that \ce{MS2} can be prepared by direct sulfurization from its oxides phase, but reaction \cee{MO3 + Se \rightarrow MSe2} will not proceed unless introducing other reducing agents, such as \ce{H2}.\cite{Tsirlina1998} This fact highlights that sulfur is rather a radical element under elevated temperature.
 
Parallel with the scenario of graphene and CNT, TMDC few layer structures attracts intensive research efforts recently. The unfolded TMDC nanotubes exhibit many appealing features, i.e., indirect-to-direct band gap transition, valley electronics, and become a promising candidate in energy harvesting, optoelectronics. 

For scientific research purpose, exfoliation by liquid or mechanics methods can provide plenty materials. Yet considering the integration into current microelectronic process, CVD synthesis of TMDC FL is desired, which hold the greatest potential for high yield production of \ce{MS2}. So in this work we focus on the CVD-based methods, as summarized in Table~\ref{tab:tmsgrowth}.


\begin{table}
  \centering
  \caption{A table with notes}  \label{tbl:notes}
  \begin{tabular}{ll}
    \toprule
    Header one                            & Header two \\
    \midrule
    Entry one\textsuperscript{\emph{a}}   & Entry two  \\
    Entry three\textsuperscript{\emph{b}} & Entry four \\
    \bottomrule
  \end{tabular}

  \textsuperscript{\emph{a}} Some text;
  \textsuperscript{\emph{b}} Some more text.
\end{table}



% CVD TMDC
\begin{landscape}
\begin{table}[htb]
\centering
\caption{TMDC FL methods summary}\label{tab:tmsgrowth}
{\footnotesize
\begin{tabular}{lp{2in}p{4in}}
\toprule
TMDC  &  precursor & growth condition  \\
\midrule
\ce{WS2} & hot W filament (above 1500 \si{\degreeCelsius}) in Ar/\ce{O2} flow  & Cubic phase, PL, resistivity measured  \\
\addlinespace[0.5em]
& W filament DC heating in \ce{NH3} or \ce{N2}/\ce{H2} flow  & multi phases, 100mg per batch, stable dispersion in both organic and aqueous solvents  \\
\bottomrule
\end{tabular}
}
\end{table}
\end{landscape}

\section{Raman spectra and Layer configuration}

\section{paragraph from word files}

Lattice vibration in hexagonal \ce{MoS2} was studied by (Wieting & Verble, 1971) using infrared and Raman spectroscopy. 15 optical modes are allowed assuming 6 atoms in primitive cell. Refractive indices was calculated as n0= 3.9, ne = 2.5. \cite{Wieting1971}

Band structure  of \ce{MoS2} in bulk form was calculated by (Mattheiss, 1973). There is some controversy about the exact value of bandgap. The calculation result is 1.2eV ( indirect gap).\cite{Mattheiss1973}

Alkali metal intercalated \ce{WS2} film was prepared.(Homyonfer et al., 1997) Stage 6 superlattice formation was suggested according to X-ray diffraction, and photoresponse spectra and electron tunneling measurement were done.

(Ballif, Regula, & Lévy, 1999) annealing of W/S/W layer , a thin Ni layer deposition, Ni-W-S phase diagram. 

Sulfurizaiton of W film, (Jäger-Waldau, Lux-Steiner, Jäger-Waldau, & Bucher, 1993) 

(Splendiani et al., 2010) reported the PL in monolayer \ce{MoS2}.  Calculation indicated the indirect gap become larger when thinning, while the previous direct one almost stays as the same, the value is about 1.85eV (direct gap).\cite{Splendiani2010}

(Ramasubramaniam, Naveh, & Towe, 2011) investigate the band gap tuning in bilayer TMD materials by applying external E field. Similar research has been done for graphene and bilayer boron nitride. Semiconductor-metal transition was suggested for \ce{MoS2} and WS2, with difference on the CBM and VBM evolution. In \ce{MoS2}, the valence-band-splitting cause the A and B excitons in optical absorption measurement. Calculation shows that CB and VB are translated toward the Fermi level with increasing E field.  The external field localized charge along C axis, but delocalized that within the plane normal towards C, thereby driving the semi-metal transition. It was mentioned that this transition is not anticipated in monolayer \ce{MoS2}. It was emphasized that precise band gaps might be different from the author’s results, yet the gap-tuning should be universal.

(Kośmider & Fernández-Rossier, 2013) studied the heterojunction between two monolayers of \ce{MoS2} and WS2. Top of VB in W layer and bottom of CB in Mo layer, forming type II structure. bilayer gap 1.2 eV.

(Shi, Pan, Zhang, & Yakobson, 2013) studied the strained monolayer \ce{MoS2} and WS2. The results show that exciton binding energy is insensitive to the strain, while optical band gap becomes smaller as strain increases. Monolayer WS2 PL maximum located at about 1.95eV. Calculation shows the electron effective mass of WS2 is the smallest, rendering higher mobility in device.

(Zhou et al., 2010) heterojunction is employed to transferred photo-generated carriers. schottky barrier conduction band electron trapping and consequent longer electron-hole pair lifetimes. Numerous studies have suggested that fine particles of transition metals or their oxides, when dispersed on the surface of a photocatalyst matrix, can act as electron traps on n-type semiconductors.

(Ueno, Saiki, et al. 1990) Van der Waals epitaxial growth of transition metal dichalcogenides (TMD) on mica. ( MoSe2, NbSe2,
Method: MBE substrate temp : 500 \si{\degreeCelsius}

CVD of MoS2 by H2S and MoF6/MoCl5; MoS2 deposition temperature: 400-550 degree. (W. Y. Lee, Besmann, and Stott 1994)

thermal decomposition of (NH4)2MoO2S2 and intermediate product MoOS2 was studied. application: hyfrodesulfurization in refinery (Weber et al. 1996) 

\cee{MoCl5 + 1/4S8 + 5/2H2 \rightarrow MoS2 + 5HCl} (Stoffels et al. 1999)

(Endler et al. 1999) solid-phase diagram for \ce{Mo-S-Cl-H-Ar} system at 1 kPa. 

(M Remskar et al. 1999) WS2 nanotube from WO3-x whisker. heating 840 degree under flow of H2/N2/H2S ,

A detailed study by (Rothschild, Sloan, and Tenne 2000) tungsten filament oxidation by water. then WS2 nanotube from WO3-x whisker. heating 840 degree under flow of H2/N2/H2S ,

IF MoS2 synthesis by MoO3 nanobelt and S.(X. L. Li and Li 2003) 

\begin{align}
\cee{WCl6 + S &rightarrow WS2 + Cl2S2}\\
\cee{MoCl5 + S &\rightarrow MoS2 + S2Cl2} \\
\cee{S2Cl2 + NaOH &\rightarrow NaCl + S + Na2SO3 + H2O}
\end{aligh}


WO3 and S by ball-milling (Z. Wu et al. 2010) 

Inorganic core –shell nanotube, WS2@MoS2 core-shell NT.(Kreizman et al. 2010) 

vdW Epitaxy of MoS2 on graphene. (Y. Shi et al. 2012)

(Zhan et al. 2012) 1-5nm thick Mo coating on SiO2/Si. Mo coating on Si did not work. 

WS2 monolayer and intense photoluminescent behavior. Sulfurization of WO3 film( 5-20 angstrom). (Gutiérrez et al. 2012)

MoO3 on sapphire reduction in H2 at 500 and sulfurizatin at 1000 degree. (Lin et al. 2012)

quantitative Raman of MoS2 on insulating subs. intensity difference between supported and suspended was highlighted, detailed model in support info.(S.-L. Li et al. 2012)

MoS2 on SiO2/Si sub pretreated with OTAS, PTCDA solution. (Y.-H. Lee et al. 2012)

WO3-x (1nm) on SiO2/Si sulfurization at 750-950 degree, (Elías et al. 2013)

source MoS2 powder, Ar On sapphire, SiO2/Si, QWP circularly polarized  light (S. Wu et al. 2013)

MoO3 powder and Sulfur powder, pregrow purge, heating 600 degree, (Mann et al. 2013)

(Wang et al. 2013)
MoS2 CVD (tsinghua univ) layer by layer sulfurization of MoO2
MoO3 powder 25mg and MnO2 powder 25 mg, extra sulfur
Ar purging  20mins, heating at 650 degree to produce MoO2 flakes
Further reaction between MoO2 and S was performed at 850-950 degree under Ar flow.

MoS2 on SiO2, sapphire, and graphite by MoCl5 and S. (Yu et al. 2013)
growth time: 10min at 850 degree, P: 2 Torr.

MoO3 nanoribbon (Na2MoO4 hydrothermal method)sulphurization, (Najmaei et al. 2013)


review of inorganic 2D materials, (Chhowalla et al., 2013) 

(Ramasubramaniam, 2012) decrease in dielectric screening and thereby enhanced excitonic effect.
DFT is not good at describing photoemission, GW approximation overcome this deficiency but still not enough for photoabsorption process in which ehps are created. BSE equation is used to compensate this discrepancy,
WX2 exhibits larger spin-orbit splitting as compared to MX2 family.



