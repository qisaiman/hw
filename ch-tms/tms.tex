
\chapter{Transition metal dichalcogenides}

In this chapter, we will first review the growth of \ce{MS2} nanotubes and few layer structures. Then we summarize the Raman spectroscopy results on NTs and FL. Finally we present our own results on synthesis and characterization.


\section{Literature review}

Shortly after the the discovery of carbon nanotube (CNT) in 1991,\cite{Iijima1991} its transition metal dichalcogenides (TMDC) counterparts (\ce{WS2}) was synthesized in 1992.\cite{Tenne1992} In the following decade, other TMDC and several metal oxides nanotubes were demonstrated as well. \citeauthor{Rao2003} did an excellent work to review these inorganic nanotubes.\cite{Rao2003} Here we just recapitulate the main categories of synthesis methods.
\begin{enumerate}
\item Reducing \ce{MO3} in \ce{H2} and \ce{H2S} atmosphere\footnote{M = Mo, W in this work}
\item Direct decomposition of \ce{MS3}
\item Decomposition of the ammonium salt \ce{(NH4)2MS4}
\item Using CNT as templates, arc discharge, or laser ablation
\end{enumerate}

Except the aforementioned methods, another important reaction in CVD domain is
\begin{equation}\label{eq:mclns}
\cee{MoCl5(WCl6) + S \rightarrow MoS2(WS2) + S2Cl2}.
\end{equation}
This reaction has been explored by several authors,\cite{Stoffels1999} and could be used under atmospheric pressure.\cite{Li2004a}

All these methods can be applied to the growth of TMDC few layer structures, either directly or with some modification, as we shall discuss later. Among all the members of TMDC nanotubes, \ce{WS2} and \ce{MoS2} NTs are probably the most well investigated.\cite{Homyonfer1997,Tenne1998,Frey1998,Frey1999,Rothschild2000,Zak2000} The reaction mechanism of \ce{WO3} with \ce{H2}/\ce{H2S} has been thoroughly studied \cite{Feldman1998} and high yield synthesis approach has been established.\cite{Margolin2004} For precursor of \ce{WO3} NPs, it was found that the simultaneous reduction and sulfurization is essential for encapsulation of fullerene like \ce{WS2} structures from the oxide nanoparticles. During the surfurization process, remaining oxide core is gradually reduced and transformed into an ordered superlattice of $\{ 001 \}$ CS planes. Further reaction will consume up the \ce{WO_x} core, leaving multi-walled \ce{WS2} NTs only. For precursor of \ce{WO3} nanowires, \citeauthor{Feldman1996} proposed that TMDC nanotubes growth began with the reduced \ce{WO3} phases, in particular \ce{W_{18}O_{49}}.\cite{Feldman1996} It is worth noting that \ce{MS2} can be prepared by direct sulfurization from its oxides phase, but reaction \cee{MO3 + Se \rightarrow MSe2} will not proceed unless introducing other reducing agents, such as \ce{H2}.\cite{Tsirlina1998} This fact highlights that sulfur is rather radical at elevated temperature.

\citeauthor{Zhu2000} performed a detailed morphological and structural analysis on \ce{WS2} NT synthesized using \ce{WO_{3-x}} NWs and \ce{H2S}.\cite{Zhu2000} Sulfur vacancy was found on the outer wall of NT. The tips exhibits various structures. Open-ended can be observed, more frequently than that in carbon nanotubes (CNTs), which is often sealed. \citeauthor{Zhu2000} suggested this open-ended tubes resulted from continuous growth on other nanotubes. And the closure configuration is rather complicated. Flat caps often dominates. \citeauthor{Zhu2000} maintained that this oxide-to-sulfides mechanism might apply to closed caps only, not to those open ended tubes.\cite{Zhu2000} TEM diffraction analysis could distinguish the chirality of NTs, as demonstrated by CNT \cite{Zhang1993} and \ce{MoS2}\cite{MARGULIS1996}. \ce{WS2} NT chirality revealed by TEM SAED show armchair NTs dominates. \citeauthor{Sloan1999} investigated tungsten oxides structures incorporated in \ce{WS2}.\cite{Sloan1999} The encapsulated \ce{WO_x} cores often exhibit \ce{W3O8} and \ce{W5O_{14}} phases, both of which belong to the \ce{W_nO_{3n-1}} homogenous series, arising as a result of crystallographic shearing (CS) planes.\cite{Miyano} Some \ce{WO_x} cores show oxygen vacancy instead of CS planes, leading to prominent streaking in SAED patterns. Although high yield growth method of \ce{WS2} has been available, the extent to which one can control the NT configuration is still limited, i.e., single-walled NT has not been routinely synthesized and the electronic properties can only be predicted theoretically. \citeauthor{Seifert2000} investigated the electronic structures of \ce{WS2} nanotube using DFT calculation.\cite{Seifert2000} It was found zigzag (n,0) NTs exhibts a direct gap at $\Gamma$ point, whose size increase monotonically with tube diameter, while armchair (n,n) NTs, unlike its metallic counterpart of (n,n) CNT, show indirect band gap increasing with tube diameter. \citeauthor{Zibouche2012} reached the same conclusion on \ce{WS2} SWNT.\cite{Zibouche2012} It was also found band gaps of armchair and zigzag NTs increase with diameter, going from values close to bulk and approaching that of  monolayer, and for a given tube diameter, $E_g$ of zigzag NTs are larger than those of armchair NTs. This difference presumably vanishes when nanotube diameter exceeds 100 nm.

Parallel with the scenario of CNT and graphene, TMDC few layer structures attracts intensive research efforts recently. These unfolded TMDC nanotubes exhibit many appealing features, i.e., indirect-to-direct band gap transition,\cite{Splendiani2010} valley electronics, and become a promising candidate in energy harvesting, optoelectronics and photocatalytic activity. For scientific research purpose, exfoliation by liquid \cite{Smith2011} or mechanical methods \cite{Lee2010a} can provide sufficient materials. Yet considering the integration into current microelectronic process, CVD synthesis of TMDC FL is highly desired, which stills hold the greatest potential for high yield production of \ce{MS2}. So in this work we focus on the CVD-based methods. We will review literature to date on both \ce{WS2} films and few layer structures, and highlight several typical investigations using different type synthesis method, as summarized in Table~\ref{tab:tmsgrowth}. We stress that the synthesis of NTs is relevantly independent of substrates, which is in sharp contrast to the scenario of few layer growth. The \ce{WS2} nucleation and interaction with substrates play dominant roles in the growth of FL structures. This lead us to focus on the \ce{WS2} films growth, which presumably could provide valuable insight into the nucleation process on various substrates.

\textbf{\ce{MS2} films} \citeauthor{Lee1994} studied the CVD of \ce{MoS2} by \ce{MoCl5} or \ce{MoF6} and \ce{H2S} in great details. Phase diagrams for \ce{Mo-S-Cl-H} and \ce{Mo-S-F-H} system at 1 kPa were simulated. \citeauthor{Endler1999} also investigated the solid-phase diagram for \ce{Mo-S-Cl-H-Ar} system.\cite{Endler1999} It was found pure \ce{MoS2} was main product when molar ratio of \ce{H2S}-\ce{MoCl5} exceeds 2. More important, \ce{MoS2} basal plane orientation can be parallel to substrate when the thickness was smaller than 50 nm. \citeauthor{Regula1997} studied the Ni-W-S phase diagram and the role of Ni layer in promoting \ce{WS2} film growth from amorphous \ce{WS3}.\cite{Regula1997} \emph{In-situ} TEM analysis confirmed the formation of \ce{NiS_x} droplets and lateral growth of \ce{WS2} from these droplets.\cite{Regula1998} Recently, it has been shown that direct sulfurization of W coating (20nm) at 750 \si{\degreeCelsius} on \ce{SiO2}-Si can also produce \ce{WS2} film.\cite{Shanmugam2012a} However, the film is of bulk in nature. This could be explained by the study of \citeauthor{Genut1992} towards \ce{WS2} growth on various substrates.\cite{Genut1992} We summarize the main results in Table~\ref{tab:ws2subs}. It was found that the \ce{WS2} nucleate from an amorphous \ce{WS3} phase and the substrate has a critical role in determining both the reaction onset temperature and the texture.
% WS2 films
\begin{table}[htb]
\centering
\caption{\ce{WS2} growth on different substrates}\label{tab:ws2subs}
\begin{tabular}{lcp{1in}p{2in}}
\toprule
precursor                 & substrate &  conditions & film feature\textsuperscript{\emph{a}}  \\
\midrule
sputtered W + \ce{H2S}   & glass      & onset 400 \si{\degreeCelsius} & $\perp c$ at 500 \si{\degreeCelsius}\\
                          & quartz      & onset 650 \si{\degreeCelsius} & $\parallel c$ below 950 \si{\degreeCelsius}\\
                          & Mo        & onset NA          & $\parallel c$ at 1000 \si{\degreeCelsius}\\
                          & W          & onset NA           & random orientation\\
\midrule
\ce{WO_x} + \ce{H2S}    & quartz      & onset 500 \si{\degreeCelsius} & predominantly $\perp c$ at 800 \si{\degreeCelsius}\\
\bottomrule
\end{tabular}

\textsuperscript{\emph{a}} $\perp c$: the c axis is perpendicular to the substrate, $\parallel c$: the c axis is parallel to the substrate;

\end{table}

\textbf{\ce{MS2} FL} For \ce{MoS2}, the reaction mechanism of \ce{MoO3} to \ce{MoS2} was studied by \citeauthor{Weber1996}.\cite{Weber1996} This study provided guidelines for the recent syntheses of \ce{MoS2} by sulfurization of \ce{MoO3}.\cite{Lin2012,Lee2012b,Liu2012a,Najmaei2013} Similar studies on sulfurization of \ce{WO3.H2O} to \ce{WS2} and decomposition of \ce{(NH4)2WO2S2} were also reported.\cite{VanderVlies2002,VanderVlies2002a} In combination with the knowledge from \ce{WS2}, we compare those insight with several recent reports on \ce{WS2} FL. \citeauthor{Cong2013} prepared monolayer \ce{WS2} on 300nm \ce{SiO2}-Si by sulfurization of \ce{WO3} powders in a one-end sealed tube.\cite{Cong2013} It was suggested that pre-cleaning the inner tube by IPA and DI water could effectively increase the pressure of vapor source. This observation probably arise from the reduction of \ce{WO3} assisted by IPA and water residuals, or due to the possible presence of \ce{H2S}. Intermediate phase \ce{WO_yS_{2-y}} and \ce{WS_{2+x}} is proposed in the growth mechanism. Possible thick \ce{WS_{2+x}} flakes decompose subsequently, leading to ML \ce{WS2}. Sulfurization mechanism study of \ce{WO3} suggested \ce{W^{6+}} cannot be directly reduce to \ce{W^{4+}} in \ce{WS2}.\cite{VanderVlies2002,VanderVlies2002a} Tungsten oxysulfides was necessary as the intermediate phase. Other chalcogenide has also been synthesized using this one-end sealed layout.\cite{Mukherjee2013}

\citeauthor{Peimyoo2013} prepared \ce{WS2} on \ce{SiO2}-Si using \ce{WO3} powder and sulfur at 800 \si{\degreeCelsius}, aiming at the light emission studies to clarify several contradictory reports.\cite{Peimyoo2013} Uniform PL intensity was found on the triangular \ce{WS2} flakes, in contrast to previous edge enhanced PL.\cite{Berkdemir2013} Raman spectra (532nm) fit includes $E_{2g}^1(M)$ mode at 343 cm$^{-1}$, according to the phonon dispersion calculation \cite{Molina-Sanchez2011} and experimental observation\cite{Zeng2013,Zhao2013,Lee2013}. However, none of these experimental reports specifically mentioned $E_{2g}^1(M)$ mode. Tentative assignments of multi-phonon bands are summarized in supporting discussion of Ref\cite{Zhao2013}. As to the growth setup, similar tube furnace as in Ref \cite{VanderZande2013} and \cite{Najmaei2013} was used. To gain additional wisdom on \ce{MoS2} growth, we thus summarize the growth strategies in Ref \cite{VanderZande2013} and \cite{Najmaei2013} as following. \citeauthor{VanderZande2013} prepared \ce{MoS2} on 285nm \ce{SiO2}-Si using \ce{MoO3} and S as precursors.\cite{VanderZande2013} In contrast to the seeding method adopted in ref\cite{Lee2013,Lee2012b}, the author stress the importance of carefully cleaned substrates\footnote{acetone, 2hrs in \ce{H2SO4} and \ce{H2O2} (3:1) and 5 mins oxygen plasma} and minimum exposure of precursor to air.\footnote{APCVD, 105 \si{\degreeCelsius} for 4hrs, 700 \si{\degreeCelsius} hold for 5mins, and 10 sccm \ce{N2} within 2 inch tube. rapid cooling from 570 \si{\degreeCelsius}} Dirty substrates or old precursors will lead to hexagonal, 3-point star or irregular polycrystalline structures. The growth setup is similar to those in Ref\cite{Lee2012b}. The substrate and \ce{MoO3} source distance is critical in determining the growth density. As a comparison, we briefly mention the synthesis strategy in Ref\cite{Lee2013,Lee2012b} as well, where PTAS treated substrate is found to promote the deposition, while KCl treated substrates did not, and small carrier gas flow is preferred (1 sccm \ce{N2}). We also notice that the \ce{MoS2} flakes morphology seems more uniform in Ref\cite{VanderZande2013} than that in Ref\cite{Lee2012b}. On the other hand, \citeauthor{Najmaei2013} also demostrated \ce{MoS2} FL growth by sulfurization of \ce{MoO3} nanoribbons.\cite{Najmaei2013} The ribbons is applied by dispersion and meant to control the source amount. It was found the diffusion of vapor \ce{MoO_{3-x}} is rate-limiting step in \ce{MoS2} growth. This means the amount of source is critical in successful synthesis. The nucleation event is more frequently observed at subs edges, scratches or rough surface. Step edges is then intentionally created to facilitate the nucleation. As to the growth dynamics, it was postulated the oxisulfides(\ce{MoOS2} Raman spectra found), as intermediate phase, diffuse across the bare substrates and form triangular domains upon further sulfurization. The optimal growth conditions are 800-850 C, 700Torr, and sufficient sulfur. We then conjecture that the substrate cleaning in ref\cite{Peimyoo2013} should be similar to that in ref\cite{VanderZande2013}, yet the source-to-substrate layout and growth pressure are still needed to be optimized with respect to our current apparatus.

\citeauthor{Zhang2013h} synthesized \ce{WS2} on sapphire (0001) using \ce{WO3} powder and sulfur as precursor under 900 \si{\degreeCelsius}.\cite{Zhang2013h} Ar slightly mixed with \ce{H2} is used to tailor the shape of \ce{WS2} flakes. It was found the source substrate distance play an important role in determining the morphology of as-grown flakes. The lattice mismatch between \ce{Al2O3} and \ce{WS2} is . The edge termination is not well studied. Raman spectra indicate a universal down-shift of $A_{1g}$ peak (1L from 418 on \ce{SiO2}-Si to 416.4 on sapphire). PL signal on \ce{SiO2}-Si is stronger than on sapphire. The visibility of \ce{WS2} on \ce{Al2O3} is poor. No 1L vs nL statistic is available. FET on/off ratio is about 100, indicating low mobility. The other conditions includes 880C, 90mm source-to-substrate distance, 1 inch quartz tube, and LBM furnace. The growth conditions are probably adopted from ref\cite{Huanga2013}(arxiv:1304.7365).

\cite{Cong2013} A direct gap of $\sim 2eV$ at the corners of BZ is formed in 1L \ce{WS2}, The subs are thoroughly cleaned. It was mentioned that separate sulfur heating improved the PL uniformity. The apex of triangles could be active site of nucleation, \ce{WS_{2+x}} formation is confirmed by secondary ion mass spectrometry. Growth on bottom piece show the multiple domain flakes occurs at initial stage of the growth, starting from \ce{WO3} particles.

It is worth noting that the growth conditions in Ref \cite{Peimyoo2013} and \cite{Zhang2013h} are basically identical except the usage of substrate. The flake size on sapphire seem larger than that on \ce{SiO2}/Si, either due to the nucleation barrier difference or the amount of growth vapor and growth time.
In studying the growth of TMDC, two issues dominate: one is the reaction of precursor, the other is the surface energy status of receiving substrates.
% CVD TMDC
\begin{landscape}
\begin{table}[htb]
\centering
\caption{TMDC FL methods summary}\label{tab:tmsgrowth}
{\footnotesize
\begin{tabular}{lp{2.5in}p{4.5in}}
\toprule
TMDC  &  precursor & growth condition (default temperature unit \si{\degreeCelsius}) \\
\midrule
\ce{MoS2} films \cite{Lee1994,Endler1999} & \ce{MoCl5}, \ce{H2S} & 1 kPa, temperature: 400-550 \si{\degreeCelsius}, 100/10/2.5 sccm for Ar, \ce{H2S} and \ce{MoCl5} flow\\
\addlinespace[0.5em]
\ce{MoS2} FL \cite{Zhan2012} & 1-5nm Mo films on \ce{SiO2}, Sulfur & purging, RT-550@30mins, 550-750@90mins and hold for 10 mins. Mo coating on Si did not work.\\
\ce{MoS2} FL \cite{Lin2012,Wang2013} & 4nm \ce{MoO3} coating on sapphire  & reduced to \ce{MoO2} in \ce{H2} and Ar at 500 \si{\degreeCelsius}, sulfurization at 850-1000 \si{\degreeCelsius} \\
\addlinespace[0.5em]
\ce{MoS2} FL \cite{Liu2012a} & \ce{(NH4)2MoS4} in DMF solution transport by Ar bubbler or dip-coating onto subs &  annealing under Ar or Ar + Sulfur, total pressure 0.2-2 Torr, \\
 \addlinespace[0.5em]
\ce{MoS2} FL \cite{Wu2013} & \ce{MoS2} powder & Ar flow, 900 \si{\degreeCelsius} heating, pressure 20 Torr, 650 \si{\degreeCelsius} growth\\
 \addlinespace[0.5em]
\ce{MoS2} FL \cite{Mann2013,Najmaei2013,Ji2013} Rice & \ce{MoO3} powders or ribbons, Sulfur & Ar flow, 530-850\si{\degreeCelsius}, total pressure 0.2-2 Torr, 5-30mins, mica or \ce{SiO2}-Si\\
 \addlinespace[0.5em]
\ce{MoS2} FL \cite{Lee2012b,Ling2014} & 18mg \ce{MoO3} powders, Sulfur,various seeding promoter on sub & 5sccm Ar, 650 \si{\degreeCelsius}, 3min growth, atmospheric pressure, quick cooling\\

 \midrule
\ce{WS2} films\cite{Ballif1999,Brunken2008} & sputtering \ce{WS_{3+x}} on 10nm Ni  & annealing under Ar for 1h at 850 \si{\degreeCelsius} \\
\addlinespace[0.5em]
\ce{WS2} FL \cite{Berkdemir2013} & $\sim$1nm \ce{WO3} coating on 285nm \ce{SiO2}-Si, 500mg sulfur & 800 \si{\degreeCelsius} for 30 mins, 100 sccm Ar, atmospheric pressure in \cite{Gutierrez2012} and 450 mTorr in \cite{Elias2013}. triangular flakes obtained\\
\addlinespace[0.5em]
\ce{WS2} ML \cite{Cong2013} & 1mg \ce{WO3} powder on \ce{SiO2}-Si covered by another sub, $d\sim3mm$, sulfur & 750 \si{\degreeCelsius}, slow heating, hold for 5mins, one-end sealed inner tube, 100 sccm Ar\\
\addlinespace[0.5em]
\ce{WS2} ML \cite{Zhang2013h} & \ce{WO3} powders, sulfur (separate heating) & 900 \si{\degreeCelsius}, sapphire subs, 225mTorr, Ar 80 sccm and \ce{H2} 10 sccm, growth time 60mins, adjusting precursor and sapphire distance changing the coverage of \ce{WS2}, tube diameter: 1 inch. 55 $\mu m$ triangular flakes\\
\addlinespace[0.5em]
\ce{WS2} ML \cite{Peimyoo2013} & 1mg \ce{WO3} powders, sulfur  & \ce{SiO2}/Si subs. Recipe A: 200mg S,RT-550, Sulfur begin to melt, 550-800, 5K/min, hold 10mins, 200 sccm Ar. Recipe B: sulfur separated heated at 250 \si{\degreeCelsius}. Total pressure: maybe atmospheric, tube diameter: 2 inch. 5 $\mu m$ triangular flakes \\
\ce{WS2} FL \cite{Lee2013}  & 1g \ce{WO3} powders, sulfur, \ce{SiO2}-Si subs treated with PTAS \ce{C24H12K4O8} and gentle gas blow & substrate facing down, APCVD, 800 \si{\degreeCelsius}, 5mins, 5 sccm Ar, fast heating. \\
\addlinespace[0.5em]
\ce{WS2} films \cite{Shanmugam2012a}   & 20nm W on \ce{SiO2}-Si, sulfur & 750 \si{\degreeCelsius}, 200 sccm Ar, 1Torr. Annealing at 1000 \si{\degreeCelsius}, 25nm thick \ce{WS2} film obtained \\
\bottomrule
\end{tabular}
}
\end{table}
\end{landscape}

In Ref\cite{Zeng2013}, single crystal \ce{WS2} growth using \ce{I2} transport was described in supporting info.

\cite{Lee2013} PTAS \ce{C24H12K4O8}\footnote{http://www.chemspider.com/Chemical-Structure.24771386.html} treated substrates. Tiny ($\sim 200$nm) seeds is found on treated surfaces. Layer growth and nucleation.




\section{Raman spectra}

\subsection{TMS bulk and FL}

Before discussing the Raman spectra features, we briefly recapitulate some symmetry notations and vibration modes. We will use \ce{MoS2} as an example, and those definitions apply to \ce{WS2} as well. Hexagonal \ce{MoS2} belongs to space group $D_{6h}^4$, and the repeat unit in $c$ axis contains two layers, where sulfur atoms in one layer are directly above the molybdenum atoms in adjacent layers, which is often referred as 2H-\ce{MoS2}. Group theory predicts two infrared- and four Raman-active modes for 2H-\ce{MoS2}, which are mutually exclusive when the center of inversion is present. First we want to emphasize that in few layer structures, \ce{MoS2} with odd layers belong to different space group from that of even layers. Bulk MoS$_2$ and 2L-MoS$_2$ belong to the space group P6$_3$/$_{mmc}$ (point group D$_{6h}$) There are 18 normal vibration modes. The factor group of bulk and 2L-MoS$_2$ at $\vec{\Gamma}$ is D$_{6h}$. The atoms site groups are a subgroup of the crystal factor group. The correlation of the Mo site group D$_{3h}$, S site group C$_{3v}$, and factor group D$_{6h}$ allows one to derive the following irreducible representations for the 18 normal vibration modes at $\vec{\Gamma}$: $\vec{\Gamma}$= $A_{1g}+2A_{2u}+2B_{2g}+B_{1u}+E_{1g}+2E_{1u}+2E_{2g}+E_{2u}$, where $A_{2u}$ and $E_{1u}$ are translational acoustic modes, $A_{1g}$, $E_{1g}$ and $E_{2g}$ are Raman active, $A_{2u}$ and $E_{1u}$ are infrared (IR) active. 1L-MoS$_2$ has $D_{3h}$ symmetry, with three atoms per unit cell. The irreducible representation of $D_{3h}$ gives: $\vec{\Gamma}$= $2A_2^{''}$+$A_1^{'}$+$2E^{'}$+$E^{''}$, with $A_2$$^{''}$ and $E^{'}$ acoustic modes, $A_2^{''}$ IR active, $A_1^{'}$ and $E^{''}$ Raman active, and the other $E^{'}$ both Raman and IR active. (See Fig.~\ref{fig:ws2ramsch}) NL-MoS$_2$ has 9N-3 optical modes: 3N-1 are vibrations along the c axis, and 3N-1 are doubly degenerate in-plane vibrations. For rigid-layer vibrations, there are N-1 layer breathing modes (LBMs) along the c axis, and N-1 doubly degenerate shear modes perpendicular to it. When N is even, there are 0 Raman active LBMs and $\frac{N}{2}$ doubly degenerate shear modes. When N is odd, there are $\frac{N-1}{2}$ LBMs and N-1 doubly degenerate shear modes.\cite{Wieting1971,Zhang2013i} This discussion can be visualized in Table~\ref{tab:tmslattice}.

% irreducible representation
\begin{table}[htb]
\centering
\caption{Lattices vibration of \ce{MS2}, adopted from ref\cite{Molina-Sanchez2011}}\label{tab:tmslattice}
\begin{tabular}{lcccc}
\toprule
 $D_{6h}$   & $D_{3h}$ & Character &  Direction & Atoms  \\
\midrule
$A_{1g}$    &  $A_1$   & Raman     & (out of plane)  & S  \\
$E_{2g}^2$  &          &           & (in plane)      & M + S  \\
$E_{2g}^1$  &  $E'$    &           & (in plane)      & M + S  \\
$E_{1g}$    &  $E''$    &           & (in plane)      & S  \\
\midrule
$A_{2u}$    &  $A_2''$  & Infrared  & (out of plane)  & M + S  \\
$E_{1u}$    &          &           & (in plane)      & M + S  \\
\midrule
$A_{2u}$    &  $A_2$   & Acoustic  & (out of plane)  & M + S  \\
$E_{1u}$    &          &           &       &    \\
\midrule
$B_{2g}^2$  &          & Inactive  & (out of plane)  & M + S  \\
$B_{2g}^1$  &          &           & (out of plane)  & M + S  \\
$B_{1u}$    &          &           & (out of plane)  & S  \\
$E_{2u}$    &          &           & (in plane)      & S  \\
\bottomrule
\end{tabular}
\end{table}

The $E$ type phonon branches correspond to the in-plane normal modes, while the $A$ type phonons result from the out-of-plane vibrations. $A_{1g}$ mode is an out-of-plane vibration involving only the S atoms while the $E_{2g}^1$ mode involves in-plane displacement of transition metal and S atoms. The $E_{2g}^2$ mode is a shear mode corresponding to the vibration of two rigid layers against
each other and appears at very low frequencies ($<50 cm^{-1}$ \cite{Zhang2013i}). The $E_{1g}$ mode, which is an in-plane vibration of only the S atoms, is forbidden in the backscattering Raman configuration. In 2H-type TMDC, the $A_{1g}$ mode is sensitive to electrostatic doping, while $E_{2g}^1$ mode is sensitive to strain, in which the FWHM of the peaks are indicator of external force quantity.

\begin{figure}[htb]
\centering
\includegraphics[width=0.7\textwidth]{ws2_ramsch}
\caption[\ce{MS2} vibration symmetry]{\ce{MS2} vibration symmetry in bulk and monalyer, adopted from REF\cite{Ghorbani-asl}}
\label{fig:ws2ramsch}
\end{figure}

Lattice vibration of natural \ce{MoS2} crystal was studied by \citeauthor{Wieting1971} using infrared and Raman spectroscopy.\cite{Wieting1971} It was found the $E_{1u}$ IR mode and one $E_{2g}$ Raman mode are nearly degenerate in energy. 15 optical modes are allowed assuming 6 atoms in primitive cell. Refractive indices from reflectivity measurement were n0= 3.9, ne = 2.5. \citeauthor{Stacy1985} studied \ce{MoS2} and \ce{WS2} Raman spectra using lasing energy close to the absorption edges.\cite{Stacy1985} Second order scattering from phonon with nonzero momentum is used to explain the rich Raman spectra. \citeauthor{Sourisseau1991} investigated the resonant Raman profiles in 2H-\ce{WS2} using ten different excitation wavelengths.\cite{Sourisseau1991} Dramatic intensity variation at 352 $cm^{-1}$ was observed, which is of two-phonon signal nature, and corresponds to an overtone or combination band of phonons with non-zero momenta contributing to indirect gap absorption edge. The authors assigned this phonon with non-zero momenta as $LA(K_5)$ type. The enhancement of the total Raman cross section at excitonic resonance in which excitons serve as the intermediate state is stronger compared to that of interband resonance. The strong enhancement at excitonic resonance is attributed to the characteristics of excitons in layered materials such as large binding energy, enhanced oscillator strength, and small damping constant.\cite{Zhao2013} \citeauthor{Chung1998} grew \ce{WS2} film using \ce{W(CO)6} and \ce{H2S} precursor.\cite{Chung1998} Raman spectra ($\lambda=632nm$) on films with non-parallel orientation revealed the presence of shoulder mode under $A_{1g}$, which is assigned to LA and TA phonon coupling. This coupling process stem from disorder-activated zone boundary phonons. We will further discuss this non-symmetric feature of $A_{1g}$ mode in section~\ref{sec:ntram}.

% WS2 Raman assignments
\begin{table}
  \centering
  \caption{\ce{WS2} Raman assignment}  \label{tbl:ws2raman}
  \begin{tabular}{ccccc}
    \toprule
    &&\multicolumn{3}{c}{Wavenumber shift ($cm^{-1}$)}\\
    \cmidrule(l){3-5}
    Symmetry                & Atoms & \ce{WS2} ML\cite{Cong2013}  & \ce{WS2} NT \cite{JMR7990865}  & \ce{WS2} bulk \cite{Sourisseau1991} \\
    \midrule
          $E_{2g}^2(\Gamma)$ &      & 27.5\textsuperscript{\emph{a}}&             &  27.4    \\
    $LA(M)-E_{2g}^2(\Gamma)$ &      & 148.3                        &              &    \\
       TBD                   &      &                              & 153          &      \\
         $E_{2g}^1(M)-LA(M)$ &      &                              & 172          & 173  \\
    LA(M)                    &      & 174.8                        & 172          &       \\
    LA(K)                    &      & 192.4                        &              &  193 \\
    $LA(M)+E_{2g}^2(\Gamma)$ &      & 203                          &              &     \\
    $LA(K)+E_{2g}^2(\Gamma)$ &      & 213.9                        &              &  212  \\
    $A_{1g}(M)-LA(M)$        &      & 230.9                        & 230          &  233  \\
    $2LA(M)-3E_{2g}^2(\Gamma)$ &    & 264.2                        & 262          &  267  \\
    $2LA(M)-2E_{2g}^2(\Gamma)$ &    & 295.4                        & 294          &  297   \\
    $2LA(M)-E_{2g}^2(\Gamma)$ &     & 322.9                        &              &  325   \\
               $E_{2g}^1(M)$ &      & 343.1                        &              &      \\
    2LA(M)                   &      & 350.8                        & 350          &  352\\
          $E_{2g}^1(\Gamma)$ &      & 355.4                        & 350          &  356 \\
    $2LA(M)+2E_{2g}^2(M)$    &      &                              & 381          &  381   \\
     LA + TA \cite{Sourisseau1991} or $B_{1u}(\Gamma)$\cite{Staiger2012}  &      &       &   &  416 \\
          $A_{1g}(\Gamma)$   &      & 417.9                        & 416\textsuperscript{\emph{b}} &  421\\
               3LA(K)        &      & 577                          &              &      \\
       $ LA(M)+ A_{1g}(M)$   &      & 584                          & 581          &  585 \\
    4LA(M)                   &      & 704                          &              &  703\\
    \bottomrule
  \end{tabular}

  \textsuperscript{\emph{a}} Calculated from column values;
  \textsuperscript{\emph{b}} \citeauthor{JMR7990865} probably made incorrect assignment of 416 peak.\cite{JMR7990865}
\end{table}

Raman technique has also provided much insight into the few layer \ce{MS2}.


\subsection{TMS NTs}\label{sec:ntram}

\citeauthor{Dobardzic2005} calculated \ce{MoS2} SWNT phonon dispersion. The dependence of wavenumbers and their displacement on chirality and diameter were discussed. The calculation method enables studying lattice dynamics with NT diameter up to 50nm. The chiral vector $(n_1, n_2)$ is defined within the molybdenum plane. Symmetry assignment is zigzag when $(n,0)$, armchair when $(n,n)$ and chiral when $(n_1, n_2), n_1>n_2$. \citeauthor{Dobardzic2006} theoretically presented Raman scattering of any polarization on SWNT of \ce{WS2} and their dependence on diameter (1-20nm) and chiral angle. The author assigned 351 $cm^{-1}$ as $E_u$ for \ce{WS2} NT.\cite{Dobardzic2006} \citeauthor{Ghorbani-asl} discuss the electronic and vibrational properties for large diameter \ce{WS2} NTs\cite{Ghorbani-asl}. Single-walled NT is approximated by 1H monolayer and others by 2H bulk structure. It was found that large-diameter nanotubes can be approximated with layered systems as their properties should be nearly the same at the scale. Only hypothetical SWNTs, and possibly MWNTs with alternating layer compositions, may show direct band gaps. Slight mechanical deformation of the SWNTs would result in a change of the direct band gap back to the indirect one, located between $\Gamma$ and $K$ high-symmetry points, similarly to the monolayers. As for 2D materials, quantum confinement to single-walled tubes would result in direct band-gap semiconductors with $\Delta$ occurring at the $K$ point. Single-walled tubes exhibit slightly softer out-of-plane $A'$ and stronger in-plane $E'$ modes. Those results indicate that the weak interlayer interactions in MS$_2$ materials cannot be associated with the van der Waals interactions only, but most probably with Coulomb electrostatic interactions as well.

Raman signatures of \ce{WS2} nanotubes show distinct features to the spectra of their bulk counterpart. \citeauthor{JMR7990865} observed a new line at 152 cm in \ce{WS2} NT, which is absent in 2H-\ce{WS2}.\cite{JMR7990865} Another feature is a emerging shoulder on the low energy side of $A_{1g}$ mode at about 416 $cm^{-1}$. This has been attributed to a combination mode of LA + TA phonons from the $K$ point of Brillouin zone.\cite{Sourisseau1991} A more prominent feature is the resonance profile broadening the shape of $E_{2g}^1$ mode, which is often assigned to 2LA mode. Yet there is some different opinion on these assignments. \citeauthor{Molina-Sanchez2011} label the 350 $cm^{-1}$ band as \ce{E_{1u}} instead of 2LA, and the 416 $cm^{-1}$ band as inactive $B_{1u}$ mode.\cite{Molina-Sanchez2011} \citeauthor{Staiger2012} adopted these assignments in studying the resonance Raman profile of \ce{WS2} NTs.\cite{Staiger2012} It was found that
\begin{enumerate}
\item $B_{1u}$ mode arise from curvature and structural disorder
\item $B_{1u}/A_{1g}$ intensity ratio strongly depends on excitation, and exceeds unity when excitation energy less than 1.9eV.
\item  An excitonic transition energy of NT is found have a local minimum at about 50nm, (layer number probably \textgreater 10), and increase either way. Yet all below the bulk value.
\end{enumerate}

\citeauthor{Krause2009} measured the resonant Raman using 632 and 532 nm excitation and found a split within 420 $cm^{-1}$ region, which is labelled as $D-A_{1g}$ mode in analog with the similar defect mode of graphene.\cite{Krause2009} This  $D-A_{1g}$ mode was found enhanced as diameter of \ce{WS2} NTs decrease. In this work, we will use $B_{1u}$ mode to interpret this emerging line at about 416 $cm^{-1}$ of \ce{WS2} NTs. \citeauthor{Krause2009a} confirmed $B_{1u}$ mode arise from the inherent structure of \ce{WS2} nanomaterials instead of surface layer effect. It is also worth noting that $A_{1g}$ is stronger than 2LA under 632nm yet weaker under 532nm, and $B_{1u}$ mode under 532 nm become less prominent than that of 632 nm.\cite{Krause2009a} This is in consistent with our observation of Raman spectra on the \ce{WS2}-\ce{WO_x} structures. \citeauthor{Rafailov2005} estimated the orientation dependence of resonant raman on one MWNT \ce{WS2} attached to the cantilever tip of AFM.\cite{Rafailov2005} Antenna effect lead to optical transition occurring only for polarization parallel to nanotube axis. And therefore resonance Raman intensity of SWNT varies as nanotube orientation. This dependence may provide a routine to distinguish different chiral NTs. Polarized Raman spectra (632nm excitation) is obtained, showing $A_{1g}$ and $E_{2g}$ sharing the same polarization behavior. \citeauthor{Virsek2007} investigated the Raman scattering ($\lambda=632nm$) of \ce{WS2} NTs.\cite{Virsek2007} The silicon peak at 520 $cm^{-1}$ is used for calibration. Up-shift of $A_{1g}$ and $E_{2g}$ modes (i.e. 420 to 423 $cm^{-1}$ at $A_{1g}$ mode) were observed, which is attributed to the strain in 3R stacking layers.


The lattices dimensions are summarized in Table~\ref{tab:ms2lattice},
%lattice
\begin{table}[htb]
\centering
\caption{Lattices dimension of \ce{MS2}}\label{tab:ms2lattice}
\begin{tabular}{lccr}
\toprule
         &  & 2H-\ce{MoS2}\cite{Coehoorn1987,Ataca2012} & 2H-\ce{WS2}\cite{Albe2002,Schutte1987} \\
\midrule
Lattice constant & a(\AA) & 3.1604 & 3.171 \\
                 & c(\AA) & 12.295 & 12.359 \\
Within \ce{MS2} layer & M-3S (\AA)& 2.37  & 2.405   \\
                      & S-1S (\AA)& 3.11  & 3.14   \\
Between \ce{MS2} layer& S-3S (\AA)&   & 3.53   \\
\bottomrule
\end{tabular}
\end{table}


\section{literature to read}

hydrogen evolution catalysts. \cite{Merki2011}

\ce{WS2} NT FET. \cite{Levi2013}

\ce{WS2} NT transport \cite{Zhang2012c}

Raman substrate dependence \cite{Buscema2013}

WO3-x NPs \cite{Frey2001}

stability of TMS NTs \cite{Seifert2002}

growth mechanism \cite{ZAK2009}

\citeauthor{Ramasubramaniam2011} investigated the band gap tuning in bilayer TMDC materials by applying external $E$ field. Similar research has been done for graphene and bilayer boron nitride. Semiconductor-metal transition was suggested for \ce{MoS2} and \ce{WS2}, with difference on the CBM and VBM evolution. In \ce{MoS2}, the valence-band-splitting cause the A and B excitons in optical absorption measurement. Calculation shows that CB and VB are translated toward the Fermi level with increasing E field.  The external field localized charge along $c$ axis, but delocalized that within the plane normal towards $c$, thereby driving the semi-metal transition. It was mentioned that this transition is not anticipated in monolayer \ce{MoS2}. It was emphasized that precise band gaps might be different from the author’s results, yet the gap-tuning should be universal.\cite{Ramasubramaniam2011}


The $A_{1g}$ and $E_{2g}^1$ intensities ratio exhibit reverse behaviors under 532 and 632 nm excitation. This is caused by the different cross-section enhancement for a specific excitation condition. The A and B excitonic absorption in \ce{WS2} mainly arise from the $d_{xy}$ and $d_{x^2 - y^2}$ states to $d_{z^2}$ states of tungsten atoms. Thus, electrons excited by 633 nm laser have a character of tungsten $d_{z^2}$ orbitals aligned along the c axis perpendicular to \ce{WS2} basal plane. Since $A_{1g}$ mode involve out of plane displacement along c axis, $A_{1g}$ phonons could couple more strongly with $d_{z^2}$ states than that of $E_{2g}^1$ phonons. As a result, $A_{1g}$ mode is stronger than $E_{2g}^1$ mode at 633 nm resonance.\cite{Zhao2013} However, the reverse effect for 532 nm excitation could not be well explained using above argument. This may be caused by electron-phonon coupling with other inter-band transition electrons. 416 peak was prevously assumed to be a combination of LA and TA phonons at K points. recent theoretical investigation suggest this peak is $B_{1u}$ mode\cite{Molina-Sanchez2011,Ataca2012}, which is the Davydov doublet with  $A_{1g}$ mode.


Raman conditions: 0.3 mW, 532 nm, 200 sec acquisition time. photon flux = $0.3E-3\times6.242E+18/1.2398/0.532=2.84E15$

nucleation and film growth \cite{Hanbucken1984}

2H and 1T in \ce{MoS2} \cite{Eda2012}

\ce{MoS2} FET statistical study. \cite{Liu2013i}

water splitting review. \cite{B800489G}

\ce{WS2} theory and experimental combined study. \cite{Klein2001}

2D review on oxides \cite{Osada2012}

Pb catalyzed \ce{MoS2} nanotube \cite{Brontvein2012}

\cite{Song2013} \ce{WO3} by ALD, and sulfurized in Ar and \ce{H2S} (10:1) at 1000 C. \ce{WS2} layer No and peak intensities ratio under 633nm excitation is correlated. It was found the 2LA/A1g is less than 1 for 1L. In supporting info 532 nm Raman spectra, the 2LA/A1g is presumably larger than 1 for 1L. \ce{WS2} NT on Si NWs is also demonstrated.

\ce{WS2} Raman.\cite{Zhao2013} \cite{Sekine1980}

phonon dispersion $E_{2g}^1(M)$? \cite{Ataca2012}

MoS2 optical properties.\cite{Search1979}

FL heterostructure. \cite{Yu2013a}

Water splitting materials should process a band gap larger than 1.4eV, considering both the NHE potential distance and practical application. The monolayer \ce{WS2} exhibits direct gap of 1.98 eV. Quantum confinement could push the gap separation farther away. \cite{wilcoxon1997} \citeauthor{Notley2013} use liquid exfoliation to prepare \ce{WS2} NPs.\cite{Notley2013} Non-ionic surfactant concentration is about 0.1\%w/w. Continuously adding surfactant during sonication improves the yield. Optimum surface tension is found at about 40 mJ/m$^2$.

\ce{WS2} thin film growth motivated by solar cell. \cite{Ennaoui1995a} grew tungsten disulfides film using sulfurization of \ce{WO3} under \ce{N2}/\ce{H2} gas flow. The composition was found to be \ce{WS_{2.13}}, and the excess of sulfur lead to p-type conductivity. XRD peaks ratio is used as a measure of film orientation, which was found higher for film prepared without an intermediate Ni coating. \ce{Ni3S2} phase was found and surfactant-mediated epitaxy is proposed.

\cite{Kang2013} TMDC alloy DFT.

thermoelectric TMDC \cite{Wickramaratne2014}

\ce{CH4N2S} thiourea + \ce{WOx} to \ce{WS2} \cite{Leonard-Deepak2011}

\ce{WS2} by \ce{WCl_n} and \ce{H2S}, raman (632nm) show bulk features\cite{Tenne2008}.

inorganic nanotubes review \cite{Tenne2004} , unsaturated bonds number increases as the size of MS2 sheet decreases.

\cite{Tenne2010} chemical modification of NTs. Functional ligand consists of an anchor group that attaches to the NTs surface and a tail which render them soluble in various solvents. PTAS functionalized BN nanotube lead to the formation of stable suspensions in aqueous solutions. The strong attachment is formed through $\pi-\pi$ interactions.

416 cm raman on WS2. B1u is pressure sensitive. \cite{Staiger2012} also studied pressure dependence.

\cite{Zou2007} prepared W/\ce{WS2} core-shell NPs by reaction of tungsten and sulfur under hydrogen atmosphere. The shoulder mode under A1g is attributed to LA+TA. As pressure increase from 0GPa to 18GPa, these two bands, both shifting to higher wavenumbers, first separate and then recombine. It was assumed the compression mainly occurs in c axis, so the stiffening of A1g is anticipated.


A1g mode using resonance Raman excitation shows upshift, which is presumably caused by folding induced strains. And TEM SEAD reveal 3R symmetry. We did not observed this upshift of A1g probably due to the large diameter and few layer involved.

Nanotube growth is relatively independent of substrates and FL layer is closely related to substrate. Film growth could provide some insight into the latter scenario.

\cite{M2013} temperature dependent of ML \ce{WS2}. Interestingly in this report 2LA/A1g (514 nm) seems less than unit. The spectra were obtain from mechanically exfoliated \ce{WS2} lying on 300 \ce{SiO2}-Si substrate.  When the temperature increase from 77K to 623K, A1g shift from 420 to 416.5 cm.




direct gap of ML at corner of BZ, point K.

\begin{align}
\cee{WCl6 + S &\rightarrow WS2 + Cl2S2}\\
\cee{MoCl5 + S &\rightarrow MoS2 + S2Cl2} \\
\cee{S2Cl2 + NaOH &\rightarrow NaCl + S + Na2SO3 + H2O}
\end{align}
