
\chapter{Transition metal dichalcogenides}

\section{literature review}


Shortly after the the discovery of carbon nanotube (CNT) in 1991,\cite{Iijima1991} its transition metal dichalcogenides (TMDC) counterparts (\ce{WS2}) was synthesized in 1992.\cite{Tenne1992} In the following decade, other TMDC and several metal oxides nanotubes were demonstrated as well. \citeauthor{Rao2003} did an excellent job to review these inorganic nanotubes.\cite{Rao2003} Here we just recapitulate the main categories of synthesis methods.
\begin{enumerate}
\item Reducing \ce{MO3} in \ce{H2} and \ce{H2S} atmosphere\footnote{M = Mo, W in this work}
\item Direct decomposition of \ce{MS3}
\item Decomposition of the ammonium salt \ce{(NH4)2MS4}
\item Using CNT as templates, arc discharge, or laser ablation
\end{enumerate}

Except the aforementioned methods, another important reaction in CVD domain is
\[
\cee{MoCl5(WCl6) + S \rightarrow MoS2(WS2) + S2Cl2}.
\]
This reaction has been explored by several authors,\cite{Stoffels1999} and could be used under atmospheric pressure.\cite{Li2004a}

All these methods can be applied to the growth of TMDC few layer structures, either directly or with some modification or combination, as we shall discuss later. Among all the members of TMDC nanotubes, \ce{WS2} and \ce{MoS2} NTs are probably the most well investigated.\cite{Homyonfer1997,Tenne1998,Frey1998,Frey1999,Rothschild2000,Zak2000} The reaction mechanism of \ce{WO3} with \ce{H2}/\ce{H2S} has been thoroughly studied \cite{Feldman1998} and high yield synthesis approach has been established.\cite{Margolin2004} For precursor of \ce{WO3} NPs, it was found that the simultaneous reduction and sulfurization is essential for encapsulation of fullerene like \ce{WS2} structures from the oxide nanoparticles. During the surfurization process, remaining oxide core is gradually reduced and transformed into an ordered superlattice of $\{ 001 \}$ CS planes. It is worth noting that \ce{MS2} can be prepared by direct sulfurization from its oxides phase, but reaction \cee{MO3 + Se \rightarrow MSe2} will not proceed unless introducing other reducing agents, such as \ce{H2}.\cite{Tsirlina1998} This fact highlights that sulfur is rather a radical element under elevated temperature.

Parallel with the scenario of CNT and graphene, TMDC few layer structures attracts intensive research efforts recently. These unfolded TMDC nanotubes exhibit many appealing features, i.e., indirect-to-direct band gap transition,\cite{Splendiani2010} valley electronics, and become a promising candidate in energy harvesting, optoelectronics.

For scientific research purpose, exfoliation by liquid \cite{Smith2011} or mechanical methods \cite{Lee2010a} can provide plenty materials. Yet considering the integration into current microelectronic process, CVD synthesis of TMDC FL is desired, which hold the greatest potential for high yield production of \ce{MS2}. So in this work we focus on the CVD-based methods, as summarized in Table~\ref{tab:tmsgrowth}.

\citeauthor{Lee1994} studied the CVD of \ce{MoS2} by \ce{MoCl5} or \ce{MoF6} and \ce{H2S} in great details. Phase diagrams for \ce{Mo-S-Cl-H} and \ce{Mo-S-F-H} system at 1 kPa were simulated. \citeauthor{Endler1999} also investigated the solid-phase diagram for \ce{Mo-S-Cl-H-Ar} system.\cite{Endler1999} It was found pure \ce{MoS2} was main product when molar ratio of \ce{H2S}-\ce{MoCl5} exceeds 2. More important, \ce{MoS2} basal plane orientation can be parallel to substrate when the thickness was smaller than 50 nm.

\citeauthor{Regula1997} studied the Ni-W-S phase diagram and the role of Ni layer in promoting \ce{WS2} film growth from amorphous \ce{WS3}.\cite{Regula1997} \emph{In-situ} TEM analysis confirmed the formation of \ce{NiS_x} droplets and lateral growth of \ce{WS2} from these droplets. \cite{Regula1998}

\citeauthor{Cong2013} prepared monolayer \ce{WS2} on 300nm \ce{SiO2}-Si by sulfurization of \ce{WO3} powders in a one-end sealed tube.\cite{Cong2013} It was suggested that pre-cleaning the inner tube by IPA and DI water could effectively increase the pressure of vapor source. This observation probably arise from the reduction of \ce{WO3} assisted by IPA and water residuals, or due to the possible presence of \ce{H2S}. Intermediate phase \ce{WO_yS_{2-y}} and \ce{WS_{2+x}} is proposed in the growth mechanism. Possible thick \ce{WS_{2+x}} flakes decompose subsequently, leading to ML \ce{WS2}. Sulfurization mechanism study of \ce{WO3} suggested \ce{W^{6+}} cannot be directly reduce to \ce{W^{4+}} in \ce{WS2}.\cite{VanderVlies2002,VanderVlies2002a} Tungsten oxysulfides was necessary as the intermediate phase. Other chalcogenide has also been synthesized using this one-end sealed layout.\cite{Mukherjee2013}

\citeauthor{Cao2014} studied the layer-dependence \ce{MoS2} electrocatalysis and propose the vertical hopping efficiency of electrons instead of the edge site numbers is a key factor for catalytic reaction.\cite{Cao2014} ref19,20

In studying the growth of TMDC, two issues dominate: one is the reaction of precursor, the other is the surface energy status of receiving substrates. For \ce{MoS2}, the reaction mechanism of \ce{MoO3} to \ce{MoS2} was studied by \citeauthor{Weber1996}.\cite{Weber1996} This study provided guideline for the recent syntheses of \ce{MoS2} by sulfurization of \ce{MoO3}. \cite{Lin2012,Lee2012b,Liu2012a}


\begin{table}
  \centering
  \caption{A table with notes}  \label{tbl:notes}
  \begin{tabular}{ll}
    \toprule
    Header one                            & Header two \\
    \midrule
    Entry one\textsuperscript{\emph{a}}   & Entry two  \\
    Entry three\textsuperscript{\emph{b}} & Entry four \\
    \bottomrule
  \end{tabular}

  \textsuperscript{\emph{a}} Some text;
  \textsuperscript{\emph{b}} Some more text.
\end{table}


% CVD TMDC
\begin{landscape}
\begin{table}[htb]
\centering
\caption{TMDC FL methods summary}\label{tab:tmsgrowth}
{\footnotesize
\begin{tabular}{lp{2.5in}p{4.5in}}
\toprule
TMDC  &  precursor & growth condition (default temperature unit \si{\degreeCelsius}) \\
\midrule
\ce{MoS2} films \cite{Lee1994,Endler1999} & \ce{MoCl5}, \ce{H2S} & 1 kPa, temperature: 400-550 \si{\degreeCelsius}, 100/10/2.5 sccm for Ar, \ce{H2S} and \ce{MoCl5} flow\\
\addlinespace[0.5em]
\ce{MoS2} FL \cite{Zhan2012} & 1-5nm Mo films on \ce{SiO2}, Sulfur & purging, RT-550@30mins, 550-750@90mins and hold for 10 mins. Mo coating on Si did not work.\\
\ce{MoS2} FL \cite{Lin2012,Wang2013} & 4nm \ce{MoO3} coating on sapphire  & reduced to \ce{MoO2} in \ce{H2} and Ar at 500 \si{\degreeCelsius}, sulfurization at 850-1000 \si{\degreeCelsius} \\
\addlinespace[0.5em]
\ce{MoS2} FL \cite{Liu2012a} & \ce{(NH4)2MoS4} in DMF solution transport by Ar bubbler or dip-coating onto subs &  annealing under Ar or Ar + Sulfur, total pressure 0.2-2 Torr, \\
 \addlinespace[0.5em]
\ce{MoS2} FL \cite{Wu2013} & \ce{MoS2} powder & Ar flow, 900 \si{\degreeCelsius} heating, pressure 20 Torr, 650 \si{\degreeCelsius} growth\\
 \addlinespace[0.5em]
\ce{MoS2} FL \cite{Mann2013,Najmaei2013,Ji2013} Rice & \ce{MoO3} powders or ribbons, Sulfur & Ar flow, 530-850\si{\degreeCelsius}, total pressure 0.2-2 Torr, 5-30mins, mica or \ce{SiO2}-Si\\
 \addlinespace[0.5em]
\ce{MoS2} FL \cite{Lee2012b,Ling2014} & 18mg \ce{MoO3} powders, Sulfur,various seeding promoter on sub & 5sccm Ar, 650 \si{\degreeCelsius}, 3min growth, atmospheric pressure, quick cooling\\

 \midrule
\ce{WS2} films\cite{Ballif1999,Brunken2008} & sputtering \ce{WS_{3+x}} on 10nm Ni  & annealing under Ar for 1h at 850 \si{\degreeCelsius} \\
\addlinespace[0.5em]
\ce{WS2} FL \cite{Berkdemir2013} & $\sim$1nm \ce{WO3} coating on 285nm \ce{SiO2}-Si, 500mg sulfur & 800 \si{\degreeCelsius} for 30 mins, 100 sccm Ar, atmospheric pressure in \cite{Gutierrez2012} and 450 mTorr in \cite{Elias2013}. triangular flakes obtained\\
\addlinespace[0.5em]
\ce{WS2} ML \cite{Cong2013} & \ce{WO3} powder on sub1 covered by sub2, sulfur & 750 \si{\degreeCelsius} hold for 5mins, one-end sealed inner tube, 100 sccm Ar\\

\bottomrule
\end{tabular}
}
\end{table}
\end{landscape}

\section{Raman spectra and Layer configuration}

\citeauthor{Zhu2000} performed a detailed morphological and structural analysis on \ce{WS2} NT synthesized using \ce{WO_{3-x}} NWs and \ce{H2S}. \cite{Zhu2000} Sulfur vacancy was found on the outer wall of NT. The tips exhibits various structures. Open-ended can be observed, more frequently than that in CNT, which is often sealed. \citeauthor{Zhu2000} suggested this open-ended tubes resulted from continuous growth on other nanotubes. And the closure configuration is rather complicated. Flat caps often dominates. \citeauthor{Feldman1996} proposed that TMDC nanotubes growth began with the reduced \ce{WO3} phases, in particular \ce{W_{18}O_{49}}.\cite{Feldman1996} \citeauthor{Zhu2000} maintained that this oxide-to-sulfides mechanism might apply to closed caps only, not to those open ended tubes.\cite{Zhu2000} TEM diffraction analysis could distinguish the chirality of NTs, as demonstrated by CNT \cite{Zhang1993} and \ce{MoS2}\cite{MARGULIS1996}. \ce{WS2} NT chirality revealed by TEM SEAD show armchair NTs dominates.

\citeauthor{Sloan1999} investigated tungsten oxides structures incorporated in \ce{WS2}.\cite{Sloan1999} The encapsulated \ce{WO_x} cores often exhibit \ce{W3O8} and \ce{W5O14} phases, both of which belong to the \ce{W_nO_{3n-1}} homogenous series, arising as a result of crystallographic shearing (CS) planes.\cite{Miyano} Some \ce{WO_x} cores show oxygen vacancy instead of CS planes, leading to prominent streaking in SEAD patterns.   and the relations between their microstructures and the growth mechanism of encapsulating shell.

\citeauthor{Rafailov2005} estimated the orientation dependence of resonant raman on one MWNT \ce{WS2} attached to the cantilever tip of AFM.\cite{Rafailov2005} Antenna effect lead to optical transition occurring only for polarization parallel to nanotube axis. And therefore resonance Raman intensity of SWNT varies as nanotube orientation. This dependence may provide a routine to distinguish different chiral NTs. 632nm excitation is used. Polarized Raman spectra is obtained, showing A1g and E2g sharing the sample polarization behavior.


hydrogen evolution catalysts. \cite{Merki2011}

\ce{WS2} NT FET. \cite{Levi2013}

\ce{WS2} NT transport \cite{Zhang2012c}

Raman substrate dependence \cite{Buscema2013}

effect of substrate on WS2 growth\cite{Genut1992}

WO3-x NPs \cite{Frey2001}

stability of TMS NTs \cite{Seifert2002}

growth mechanism \cite{ZAK2009}



\section{paragraph from word files}


\begin{align}
\cee{WCl6 + S &\rightarrow WS2 + Cl2S2}\\
\cee{MoCl5 + S &\rightarrow MoS2 + S2Cl2} \\
\cee{S2Cl2 + NaOH &\rightarrow NaCl + S + Na2SO3 + H2O}
\end{align}

\begin{figure}[htb]
\centering
\includegraphics[width=0.7\textwidth]{ws2_ramsch}
\caption[CVD system]{Chemical vapor system and \cite{Ghorbani-asl}}
\label{fig:ws2ramsch}
\end{figure}

