
\chapter{Transition metal dichalcogenides}

In this chapter, a TEM-Raman integrated study on \ce{WO3}-\ce{WS2} core-shell nanostructures is presented. We leverage a home-built micromanipulator apparatus to transfer the nanostructures between TEM grids and other desired substrates. This technique has enabled us to correlate TEM and micro-Raman measurement on individual nanowires, thereby enhancing the understanding of structure-property relations which would otherwise be blurred in conventional ensemble averaged methods. \ce{WO3}-\ce{WS2} heterostructure is chosen due to the potential applications in photocatalysis and hydrogen evolution. 

The remaining section is arranged in following order: a review on the growth of \ce{MS2} (M = Mo, W) nanotubes (NTs) and 2D few layer (FL) structures will be introduced first, followed by Raman spectroscopy summary on NTs and FL. The thesis will then focus on synthesis of \ce{WO3}-\ce{WS2} heterostructure and various characterizations, including both ensemble averaged and on individual nanowires. This chapter is concluded with a brief summary and future work.

\section{Introduction}
% \subsection{crystal structure of tmdc}
\subsection{Growth of TMDC}
Shortly after the discovery of carbon nanotube (CNT) in 1991,\cite{Iijima1991} its transition metal dichalcogenides (TMDC) counterparts, \emph{i.e.} \ce{WS2}, was synthesized in 1992.\cite{Tenne1992} In the following decades, other TMDC and several metal oxides nanotubes were demonstrated as well. \citeauthor{Rao2003} did an excellent work to review these inorganic nanotubes.\cite{Rao2003} Here the author just recapitulate the main categories of synthesis methods.
\begin{enumerate}
\item Reducing \ce{MO3} in \ce{H2} and \ce{H2S} atmosphere\footnote{M = Mo, W in this work}
\item Direct decomposition of \ce{MS3}
\item Decomposition of the ammonium salt \ce{(NH4)2MS4}
\item Using CNT as templates, arc discharge, or laser ablation
\end{enumerate}

Except the aforementioned methods, another important reaction in CVD domain is
\begin{equation}\label{eq:mclns}
\cee{MoCl5(WCl6) + S \rightarrow MoS2(WS2) + S2Cl2}.
\end{equation}
This reaction has been explored by several authors,\cite{Stoffels1999} and could be used under atmospheric pressure.\cite{Li2004a}

All these methods can be applied to the growth of TMDC few layer structures, either directly or with some modifications. Among all the members of TMDC nanotubes, \ce{WS2} and \ce{MoS2} NTs are probably the most well investigated.\cite{Homyonfer1997,Tenne1998,Frey1998,Frey1999,Rothschild2000,Zak2000} The reaction mechanism of \ce{WO3} with \ce{H2}/\ce{H2S} has been thoroughly studied \cite{Feldman1998} and high yield synthesis approach has been established.\cite{Margolin2004} For \ce{WO3} nanoparticles precursor, it was found that the simultaneous reduction and sulfurization is essential for encapsulation of fullerene like \ce{WS2} structures from the oxide nanoparticles. During the surfurization process, remaining oxide core is gradually reduced and transformed into an ordered superlattice of $\{ 001 \}$ \gls{cs} planes. Further reaction will consume up the \ce{WO_x} core, leaving multi-walled \ce{WS2} NTs only. For precursor of \ce{WO3} nanowires, \citeauthor{Feldman1996} proposed that TMDC nanotubes growth began with the reduced \ce{WO3} phases, in particular \ce{W_{18}O_{49}}.\cite{Feldman1996} It is worth noting that \ce{MS2} can be prepared by direct sulfurization from its oxides phase, but reaction \cee{MO3 + Se \rightarrow MSe2} will not proceed unless introducing other reducing agents, such as \ce{H2}.\cite{Tsirlina1998} This fact highlights that sulfur is rather radical at elevated temperature.

\citeauthor{Zhu2000} performed a detailed morphological and structural analysis on \ce{WS2} NTs synthesized using \ce{WO_{3-x}} NWs and \ce{H2S}.\cite{Zhu2000} Sulfur vacancy was found on the outer wall of NT. The tips exhibits various structures. Open-ended can be observed, more frequently than that in carbon nanotubes (CNTs), which is often sealed. \citeauthor{Zhu2000} suggested this open-ended tubes resulted from continuous growth on other nanotubes. And the closure configuration is rather complicated. Flat caps often dominates. \citeauthor{Zhu2000} maintained that this oxide-to-sulfides mechanism might apply to closed caps only, not to those open ended tubes.\cite{Zhu2000} TEM diffraction analysis could distinguish the chirality of NTs, as demonstrated by CNT \cite{Zhang1993} and \ce{MoS2}\cite{MARGULIS1996}. \ce{WS2} NT chirality revealed by TEM SAED shows armchair NTs dominates. \citeauthor{Sloan1999} investigated tungsten oxides structures incorporated in \ce{WS2}.\cite{Sloan1999} The encapsulated \ce{WO_x} cores often exhibit \ce{W3O8} and \ce{W5O_{14}} phases, both of which belong to the \ce{W_nO_{3n-1}} homogenous series, arising as a result of \gls{cs} planes.\cite{Miyano} Some \ce{WO_x} cores show oxygen vacancy instead of CS planes, leading to prominent streaking in SAED patterns. Although high yield growth method of \ce{WS2} has been available, the extent to which one can control the NT configuration is still limited, \emph{i.e.}, single-walled NT has not been routinely synthesized and the electronic properties can only be predicted theoretically. \citeauthor{Seifert2000} investigated the electronic structures of \ce{WS2} nanotube using DFT calculation.\cite{Seifert2000} It was found zigzag (n,0) NTs exhibts a direct gap at $\Gamma$ point, whose size increase monotonically with tube diameter, while armchair (n,n) NTs, unlike its metallic counterpart of (n,n) CNT, show indirect band gap increasing with tube diameter. \citeauthor{Zibouche2012} reached the same conclusion on \ce{WS2} SWNT.\cite{Zibouche2012} It was also found band gaps of armchair and zigzag NTs increase with diameter, going from values close to bulk and approaching that of  monolayer, and for a given tube diameter, $E_g$ of zigzag NTs are larger than those of armchair NTs. This difference presumably vanishes when nanotube diameter exceeds 100 nm.

Parallel with the scenario of CNT and graphene, TMDC few layer structures attract intensive research efforts recently. These unfolded TMDC nanotubes exhibit many appealing features, i.e., indirect-to-direct band gap transition,\cite{Splendiani2010} valley electronics, and become a promising candidate in energy harvesting, optoelectronics and photocatalytic activity. For scientific research purpose, exfoliation by liquid \cite{Smith2011} or mechanical methods \cite{Lee2010a} can provide sufficient materials. Yet considering the integration into current microelectronic process, CVD synthesis of thickness controllable 2D TMDC is highly desired, which stills hold the greatest potential for high yield production of \ce{MS2}. So this work focuses on reviewing the CVD based methods. The author will review literatures to date on both \ce{WS2} films and few layer structures, and highlight several typical investigations using different synthesis methods, as summarized in Table~\ref{tab:tmsgrowth}. It is important to note that the synthesis of NTs is relevantly independent of substrates, which is in sharp contrast to the scenario of few layer growth. The \ce{WS2} nucleation and interaction with substrates play dominant roles in the growth of 2D FL structures. This lead us to focus on the \ce{WS2} films growth, which presumably could provide valuable insight into the nucleation process on various substrates.

\textbf{\ce{MS2} films} \citeauthor{Lee1994} studied the CVD of \ce{MoS2} by \ce{MoCl5} or \ce{MoF6} and \ce{H2S} in great details. Phase diagrams for \ce{Mo-S-Cl-H} and \ce{Mo-S-F-H} system at 1 kPa were simulated.\cite{Lee1994} \citeauthor{Endler1999} also investigated the solid-phase diagram for \ce{Mo-S-Cl-H-Ar} system.\cite{Endler1999} It was found pure \ce{MoS2} was main product when molar ratio of \ce{H2S}-\ce{MoCl5} exceeds 2. More important, \ce{MoS2} basal plane orientation can be parallel to substrate when the thickness was smaller than 50 nm. \citeauthor{Ennaoui1995a} grew tungsten disulfide film using sulfurization of \ce{WO3} under \ce{N2}/\ce{H2} gas flow.\cite{Ennaoui1995a} The composition was found to be \ce{WS_{2.13}}, and the excess of sulfur lead to p-type conductivity. XRD peaks ratio is used as a measure of film orientation, which was found higher for film prepared without an intermediate Ni coating. \ce{Ni3S2} phase was found and surfactant-mediated epitaxy is proposed. \citeauthor{Regula1997} studied the Ni-W-S phase diagram and the role of Ni layer in promoting \ce{WS2} film growth from amorphous \ce{WS3}.\cite{Regula1997} \emph{In-situ} TEM analysis confirmed the formation of \ce{NiS_x} droplets and lateral growth of \ce{WS2} from these droplets.\cite{Regula1998} Recently, it has been shown that direct sulfurization of W coating (20nm) at 750 \si{\degreeCelsius} on \ce{SiO2}-Si can also produce \ce{WS2} film.\cite{Shanmugam2012a} However, the film is of bulk in nature. This could be explained by the study of \citeauthor{Genut1992} towards \ce{WS2} growth on various substrates,\cite{Genut1992} as summarized in Table~\ref{tab:ws2subs}. It was found that the \ce{WS2} nucleate from an amorphous \ce{WS3} phase and the substrate has a critical role in determining both the reaction onset temperature and the texture. The adhesion of tungsten to quartz was found to be much stronger than to glass. And oxygen-containing species such as \ce{H2O} or OH tend to cause \ce{WS2} basal plane perpendicular to substrate.
% WS2 films
\begin{table}[htb]
\centering
\caption{\ce{WS2} growth on different substrates}\label{tab:ws2subs}
\begin{tabular}{lcp{1in}p{2in}}
\toprule
precursor                 & substrate &  conditions & film feature\textsuperscript{\emph{a}}  \\
\midrule
sputtered W + \ce{H2S}   & glass      & onset 400 \si{\degreeCelsius} & $\perp c$ at 500 \si{\degreeCelsius}, metastable \ce{WS3} found\\
                          & quartz      & onset 650 \si{\degreeCelsius} & $\parallel c$ below 950 \si{\degreeCelsius}\\
                          & Mo        & onset NA          & $\parallel c$ at 1000 \si{\degreeCelsius}\\
                          & W          & onset NA           & random orientation\\
\midrule
\ce{WO_x} + \ce{H2S}    & quartz      & onset 500 \si{\degreeCelsius} & predominantly $\perp c$ after 800 \si{\degreeCelsius}\\
                        & Mo        & onset NA          & $\parallel c$ dominant\\
                        & W       & onset NA          & random orientation\\
\bottomrule
\end{tabular}

\textsuperscript{\emph{a}} $\perp c$: the c axis is perpendicular to the substrate, $\parallel c$: the c axis is parallel to the substrate;

\end{table}

\textbf{\ce{MS2} FL} For \ce{MoS2}, the reaction mechanism of \ce{MoO3} to \ce{MoS2} was studied by \citeauthor{Weber1996}.\cite{Weber1996} This study provided guidelines for the recent syntheses of \ce{MoS2} by sulfurization of \ce{MoO3}.\cite{Lin2012,Lee2012b,Liu2012a,Najmaei2013} Similar studies on sulfurization of \ce{WO3.H2O} to \ce{WS2} and decomposition of \ce{(NH4)2WO2S2} were also reported.\cite{VanderVlies2002,VanderVlies2002a} In combination with the knowledge from \ce{WS2}, the author compare those insight from several recent reports on \ce{WS2} FL. \citeauthor{Cong2013} prepared monolayer \ce{WS2} on 300 nm \ce{SiO2}-Si by sulfurization of \ce{WO3} powders in a one-end sealed tube.\cite{Cong2013} It was suggested that pre-cleaning the inner tube by IPA and DI water could effectively increase the pressure of vapor source. This observation probably arise from the reduction of \ce{WO3} assisted by IPA and water residuals, or due to the possible presence of \ce{H2S}. Intermediate phase \ce{WO_yS_{2-y}} and \ce{WS_{2+x}} is proposed in the growth mechanism. The apex of triangles could be active site of nucleation, \ce{WS_{2+x}} formation is confirmed by secondary ion mass spectrometry. Possible thick \ce{WS_{2+x}} flakes decompose subsequently, leading to ML \ce{WS2}. The substrates are thoroughly cleaned. It was mentioned that separate sulfur heating improved the PL uniformity. Sulfurization mechanism study of \ce{WO3} suggested \ce{W^{6+}} cannot be directly reduce to \ce{W^{4+}} in \ce{WS2}.\cite{VanderVlies2002,VanderVlies2002a} Tungsten oxysulfides was necessary as the intermediate phase. 

\citeauthor{Peimyoo2013} prepared \ce{WS2} on \ce{SiO2}-Si using \ce{WO3} powder and sulfur at 800 \si{\degreeCelsius}, aiming at the light emission studies to clarify several contradictory reports.\cite{Peimyoo2013} Uniform PL intensity was found on the triangular \ce{WS2} flakes, in contrast to previous edge enhanced PL.\cite{Berkdemir2013} Raman spectra (532nm) fit includes $E_{2g}^1(M)$ mode at 343 cm$^{-1}$, according to the phonon dispersion calculation \cite{Molina-Sanchez2011} and experimental observation\cite{Zeng2013a,Zhao2013,Lee2013}. However, none of these experimental reports specifically mentioned $E_{2g}^1(M)$ mode. Tentative assignments of multi-phonon bands are summarized in supporting information of Ref\cite{Zhao2013}. As to the growth setup, similar tube furnace as in Ref \cite{VanderZande2013} and \cite{Najmaei2013} was used. To gain additional wisdom on \ce{MoS2} growth, the author thus summarize the growth strategies in Ref \cite{VanderZande2013} and \cite{Najmaei2013} as following. \citeauthor{VanderZande2013} prepared \ce{MoS2} on 285nm \ce{SiO2}-Si using \ce{MoO3} and S as precursors.\cite{VanderZande2013} In contrast to the seeding method adopted in ref\cite{Lee2013,Lee2012b}, \citeauthor{VanderZande2013} stressed the importance of carefully cleaned substrates\footnote{acetone, 2hrs in \ce{H2SO4} and \ce{H2O2} (3:1) and 5 mins oxygen plasma} and minimum exposure of precursor to air.\footnote{APCVD, 105 \si{\degreeCelsius} for 4hrs, 700 \si{\degreeCelsius} hold for 5mins, and 10 sccm \ce{N2} within 2 inch tube. rapid cooling from 570 \si{\degreeCelsius}} Dirty substrates or old precursors will lead to hexagonal, 3-point star or irregular polycrystalline structures. The growth setup is similar to those in Ref\cite{Lee2012b}. The substrate and \ce{MoO3} source distance is critical in determining the growth density. The synthesis strategy in Ref\cite{Lee2013,Lee2012b} is briefly mentioned as well, where PTAS treated substrate is found to promote the deposition, while KCl treated substrates did not, and small carrier gas flow is preferred (1 sccm \ce{N2}). It was also noted that the \ce{MoS2} flakes morphology seems more uniform in Ref\cite{VanderZande2013} than that in Ref\cite{Lee2012b}. On the other hand, \citeauthor{Najmaei2013} also demostrated \ce{MoS2} FL growth by sulfurization of \ce{MoO3} nanoribbons.\cite{Najmaei2013} The ribbons is applied by dispersion and meant to control the source amount. It was found the diffusion of vapor \ce{MoO_{3-x}} is rate-limiting step in \ce{MoS2} growth. This means the amount of source is critical in successful synthesis. The nucleation event is more frequently observed at subs edges, scratches or rough surface. Step edges is then intentionally created to facilitate the nucleation. As to the growth dynamics, it was postulated the oxisulfides(\ce{MoOS2} Raman spectra found), as intermediate phase, diffuse across the bare substrates and form triangular domains upon further sulfurization. The optimal growth conditions are 800--850 \si{\degreeCelsius}, 700 Torr, and sufficient sulfur. This theis then conjectures that the substrate cleaning in ref\cite{Peimyoo2013} should be similar to that in ref\cite{VanderZande2013}, yet the source-to-substrate layout and growth pressure are still needed to be optimized with respect to our current apparatus.

\citeauthor{Zhang2013h} synthesized \ce{WS2} on sapphire (0001) using \ce{WO3} powder and sulfur as precursor under 900 \si{\degreeCelsius}.\cite{Zhang2013h} Ar slightly mixed with \ce{H2} is used to tailor the shape of \ce{WS2} flakes. It was found the source substrate distance plays an important role in determining the morphology of the as-grown flakes.\footnote{There is a lattice mismatch between \ce{Al2O3} (4.785 \AA) and \ce{WS2} (3.153 \AA)} The edge termination is not well studied. Raman spectra indicate a universal down-shift of $A_{1g}$ peak (1L from 418 on \ce{SiO2}-Si to 416.4 on sapphire). PL signal on \ce{SiO2}-Si is stronger than on sapphire. The visibility of \ce{WS2} on \ce{Al2O3} is poor. No 1L vs nL statistic is available. FET on/off ratio is about 100, indicating low mobility. The other conditions includes 880 \si{\degreeCelsius}, 90 mm source-to-substrate distance, 1 in quartz tube, and LBM furnace. The growth conditions are probably adopted from ref\cite{Huanga2013}.

It is worth noting the flake sizes on sapphire\cite{Zhang2013h} seems larger than that on \ce{SiO2}/Si\cite{Peimyoo2013}, either due to the nucleation barrier difference or the amount of growth vapor and growth time. The flake sizes in ref \cite{Cong2013} is not quite uniform, though it could be even larger than that on sapphire \cite{Zhang2013h}. One common feature of aforementioned investigations is growth occurs on bare substrates and no catalyst or seeding promoter is employed. While there is another synthesis approach \cite{Lee2013,Ling2014} exclusively focusing on using seeding promoter. \citeauthor{Lee2013} obtained \ce{MoS2} and \ce{WS2} FL on PTAS-treated substrates (\ce{C24H12K4O8}), where tiny ($\sim 200$ nm) seeds were found under AFM. More recently, \citeauthor{Ling2014} systemically investigated the role of seeding promoter in facilitating the nucleation of \ce{MoS2} monolayers.\cite{Ling2014} It was found various aromatic molecules are effective yet inorganic nanoparticles are not. The mechanism of thin film growth depends on the surface energy and chemical potentials of the deposited layers and their substrates. Layer growth is preferred when surface adhesive force is stronger than adatoms cohesive force. Seeding promoter probably lowered the surface energy by wetting, thus provided heterogenous nucleation sites. Continuous \ce{MoS2} monolayer is obtained by evaporating \ce{F16CuPc} of 2 \AA on desired receiving substrates. In each growth, about $10^{-10}$ mol PTAS was applied onto \ce{SiO2}-Si, which was rendered hydrophilic and gentle gas blow to distribute the solution evenly.

In short summary, this thesis stresses that there are several key factors in successfully synthesizing \ce{WS2} FL:
\begin{itemize}
\item \ce{WO3} powder size distribution and absolute amount (0.69 g \ce{WO3} in 10 mL acetone or IPA, inspired by the \ce{MoO3} nanoribbons usage)
\item Sulfur amount and heating method to ensure a constant sulfur-rich environment
\item \ce{WO3} powder source to substrate distance, which is coupled to pressure and carrier gas flow in determining the transport (atmospheric, 3-5 mm)
\item temperature ramping and growth time (750-800 \si{\degreeCelsius}, 3-10 min)
\item 300 nm-\ce{SiO2}-Si substrate to make OM flakes identification easier
\end{itemize}

% CVD TMDC
\begin{landscape}
\begin{table}[htb]
\centering
\caption{TMDC FL methods summary}\label{tab:tmsgrowth}
{\footnotesize
\begin{tabular}{lp{2.5in}p{4.5in}}
\toprule
TMDC  &  precursor & growth condition (default temperature unit \si{\degreeCelsius}) \\
\midrule
\ce{MoS2} films \cite{Lee1994,Endler1999} & \ce{MoCl5}, \ce{H2S} & 1 kPa, temperature: 400-550 \si{\degreeCelsius}, 100/10/2.5 sccm for Ar, \ce{H2S} and \ce{MoCl5} flow\\
\addlinespace[0.5em]
\ce{MoS2} FL \cite{Zhan2012} & 1-5 nm Mo films on \ce{SiO2}, Sulfur & purging, RT-550@30 min, 550-750@90 min and hold for 10 min. Mo coating on Si did not work.\\
\ce{MoS2} FL \cite{Lin2012,Wang2013} & 4 nm \ce{MoO3} coating on sapphire  & reduced to \ce{MoO2} in \ce{H2} and Ar at 500 \si{\degreeCelsius}, sulfurization at 850-1000 \si{\degreeCelsius} \\
\addlinespace[0.5em]
\ce{MoS2} FL \cite{Liu2012a} & \ce{(NH4)2MoS4} in DMF solution transport by Ar bubbler or dip-coating onto subs &  annealing under Ar or Ar + Sulfur, total pressure 0.2-2 Torr, \\
 \addlinespace[0.5em]
\ce{MoS2} FL \cite{Wu2013} & \ce{MoS2} powder & Ar flow, 900 \si{\degreeCelsius} heating, pressure 20 Torr, 650 \si{\degreeCelsius} growth\\
 \addlinespace[0.5em]
\ce{MoS2} FL \cite{Mann2013,Najmaei2013,Ji2013} Rice & \ce{MoO3} powders or ribbons, Sulfur & Ar flow, 530-850 \si{\degreeCelsius}, total pressure 0.2-2 Torr, 5-30 min, mica or \ce{SiO2}-Si\\
 \addlinespace[0.5em]
\ce{MoS2} FL \cite{Lee2012b,Ling2014} & 18mg \ce{MoO3} powders, Sulfur,various seeding promoter on sub & 5sccm Ar, 650 \si{\degreeCelsius}, 3 min growth, atmospheric pressure, quick cooling\\

 \midrule
\ce{WS2} films\cite{Ballif1999,Brunken2008} & sputtering \ce{WS_{3+x}} on 10 nm Ni  & annealing under Ar for 1 h at 850 \si{\degreeCelsius} \\
\addlinespace[0.5em]
\ce{WS2} FL \cite{Berkdemir2013} & $\sim$1nm \ce{WO3} coating on 285 nm \ce{SiO2}-Si, 500 mg sulfur & 800 \si{\degreeCelsius} for 30 min, 100 sccm Ar, atmospheric pressure in \cite{Gutierrez2012} and 450 mTorr in \cite{Elias2013}. triangular flakes obtained\\
\addlinespace[0.5em]
\ce{WS2} ML \cite{Cong2013} & 1 mg \ce{WO3} powder on \ce{SiO2}-Si covered by another sub, $d\sim3$ mm, sulfur & 750 \si{\degreeCelsius}, slow heating, hold for 5 min, one-end sealed inner tube, 100 sccm Ar\\
\addlinespace[0.5em]
\ce{WS2} ML \cite{Zhang2013h} & \ce{WO3} powders, sulfur (separate heating) & 900 \si{\degreeCelsius}, sapphire subs, 225 mTorr, Ar 80 sccm and \ce{H2} 10 sccm, growth time 60 min, adjusting precursor and sapphire distance changing the coverage of \ce{WS2}, tube diameter: 1 in. 55 $\mu$m triangular flakes\\
\addlinespace[0.5em]
\ce{WS2} ML \cite{Peimyoo2013} & 1 mg \ce{WO3} powders, sulfur  & \ce{SiO2}/Si subs. Recipe A: 200 mg S,RT-550, Sulfur begin to melt, 550-800 \si{\degreeCelsius}, 5K/min, hold 10mins, 200 sccm Ar. Recipe B: sulfur separated heated at 250 \si{\degreeCelsius}. Total pressure: maybe atmospheric, tube diameter: 2 in. 5 $\mu$m triangular flakes \\
\ce{WS2} FL \cite{Lee2013}  & 1g \ce{WO3} powders, sulfur, \ce{SiO2}-Si subs treated with PTAS \ce{C24H12K4O8} and gentle gas blow & substrate facing down, APCVD, 800 \si{\degreeCelsius}, 5 min, 5 sccm Ar, fast heating. \\
\addlinespace[0.5em]
\ce{WS2} films \cite{Shanmugam2012a}   & 20 nm W on \ce{SiO2}-Si, sulfur & 750 \si{\degreeCelsius}, 200 sccm Ar, 1Torr. Annealing at 1000 \si{\degreeCelsius}, 25 nm thick \ce{WS2} film obtained \\
\bottomrule
\end{tabular}
}
\end{table}
\end{landscape}

\subsection{Raman on bulk and FL TMDC}

This section will discuss previous studies using Raman to characterize \ce{MS2} films, nanotubes, and FL structures. In each morphology, one or two key points will be highlighted for study in this thesis. The previous Raman efforts on \ce{MS2} film, especially the one using resonant conditions, provided critical insight into the electronic structure and lattice dynamics of these TMDC layered materials. Next a comparison of Raman knowledge between TMDC nanotubes and FL structures is given. For \ce{WS2} NTs, focus is on the asymmetry of $A_{1g}$ mode and its origin. And for \ce{WS2} FL structures, layer number dependent fingerprint is summarized, and some attempts on analysis the resonant Raman profiles of \ce{WS2} are made.

It is worth noting that Raman technique proves to be extreme useful in characterizing CNT and graphene, i.e., tube diameter by assigning the RBMs (radial breathing modes) and G peaks position.\cite{Bonaccorso2013} And Raman spectroscopy also qualifies as an excellent tool to monitor tensile features of TMDC in both 2D and tubular forms.\cite{Tang2013} 

Raman spectra arise from the inelastic light scattering of optical phonons. In back scattering geometry, the phonon wave-vector stands as $q = 4\pi\frac{n}{\lambda}$. The refractive index $\tilde{n}$ of \ce{WS2} at 532 nm is about $4.726 - 0.737i$. Compared to the size of Brillouin zone ($\pi/a = 10^{10}m^{-1}$), resonant Raman could or could not probe phonon at $M$ point.

Before discussing the Raman spectra features, the author briefly recapitulate some symmetry notations and vibration modes. \ce{MoS2} is used as an example, and those definitions apply to \ce{WS2} as well. Hexagonal \ce{MoS2} belongs to space group $D_{6h}^4$, and the repeat unit in $c$ axis contains two layers, where sulfur atoms in one layer are directly above the molybdenum atoms in adjacent layers, which is often referred as 2H-\ce{MoS2}. Group theory predicts two infrared- and four Raman-active modes for 2H-\ce{MoS2}, which are mutually exclusive when the center of inversion is present. First it should be emphasized that in few layer structures, \ce{MoS2} with odd layers belong to different space group from that of even layers. Bulk MoS$_2$ and 2L-MoS$_2$ belong to the space group P6$_3$/$_{mmc}$ (point group D$_{6h}$). There are 18 normal vibration modes. The factor group of bulk and 2L-MoS$_2$ at $\vec{\Gamma}$ is D$_{6h}$. The atoms site groups are a subgroup of the crystal factor group. The correlation of the Mo site group D$_{3h}$, S site group C$_{3v}$, and factor group D$_{6h}$ allows one to derive the following irreducible representations for the 18 normal vibration modes at $\vec{\Gamma}$: $\vec{\Gamma}$= $A_{1g}+2A_{2u}+2B_{2g}+B_{1u}+E_{1g}+2E_{1u}+2E_{2g}+E_{2u}$, where $A_{2u}$ and $E_{1u}$ are translational acoustic modes, $A_{1g}$, $E_{1g}$ and $E_{2g}$ are Raman active, $A_{2u}$ and $E_{1u}$ are infrared (IR) active. As a contrast, 1L-MoS$_2$ has $D_{3h}$ symmetry with three atoms per unit cell. The irreducible representation of $D_{3h}$ gives: $\vec{\Gamma}$= $2A_2^{''}$+$A_1^{'}$+$2E^{'}$+$E^{''}$, with $A_2^{''}$ and $E^{'}$ acoustic modes, $A_2^{''}$ IR active, $A_1^{'}$ and $E^{''}$ Raman active, and the other $E^{'}$ both Raman and IR active (See Fig.~\ref{fig:ws2ramsch}). NL-MoS$_2$ has 9N-3 optical modes: 3N-1 are vibrations along the c axis, and 3N-1 are doubly degenerate in-plane vibrations. For rigid-layer vibrations, there are N-1 layer breathing modes (LBMs) along the c axis, and N-1 doubly degenerate shear modes perpendicular to it. When N is even, there are 0 Raman active LBMs and $\frac{N}{2}$ doubly degenerate shear modes. When N is odd, there are $\frac{N-1}{2}$ LBMs and N-1 doubly degenerate shear modes.\cite{Wieting1971,Zhang2013e} This discussion can be visualized in Table~\ref{tab:tmslattice}.

% irreducible representation
\begin{table}[htb]
\centering
\caption{Lattices vibration of \ce{MS2}, adopted from ref\cite{Molina-Sanchez2011}}\label{tab:tmslattice}
\begin{tabular}{lcccc}
\toprule
 $D_{6h}$   & $D_{3h}$ & Character &  Direction & Atoms  \\
\midrule
$A_{1g}$    &  $A_1^{'}$   & Raman     & (out of plane)  & S  \\
$E_{2g}^2$  &          &           & (in plane)      & M + S  \\
$E_{2g}^1$  &  $E'$    &           & (in plane)      & M + S  \\
$E_{1g}$    &  $E''$    &           & (in plane)      & S  \\
\midrule
$A_{2u}$    &  $A_2''$  & Infrared  & (out of plane)  & M + S  \\
$E_{1u}$    &          &           & (in plane)      & M + S  \\
\midrule
$A_{2u}$    &  $A_2^{''}$   & Acoustic  & (out of plane)  & M + S  \\
$E_{1u}$    &          &           &       &    \\
\midrule
$B_{2g}^2$  &          & Inactive  & (out of plane)  & M + S  \\
$B_{2g}^1$  &          &           & (out of plane)  & M + S  \\
$B_{1u}$    &          &           & (out of plane)  & S  \\
$E_{2u}$    &          &           & (in plane)      & S  \\
\bottomrule
\end{tabular}
\end{table}

The $E$ type phonon branches correspond to in-plane normal modes, while the $A$ type phonons result from out-of-plane vibrations. $A_{1g}$ mode is an out-of-plane vibration involving only the S atoms while the $E_{2g}^1$ mode involves in-plane displacement of transition metal and S atoms. The $E_{2g}^2$ mode is a shear mode corresponding to the vibration of two rigid layers against
each other and appears at very low frequencies ($<50 cm^{-1}$ \cite{Zhang2013e}). The $E_{1g}$ mode, which is an in-plane vibration of only the S atoms, is forbidden in the backscattering Raman configuration. In 2H-type TMDC, the $A_{1g}$ mode is more sensitive to electrostatic doping, while $E_{2g}^1$ mode is more sensitive to strain, in which the FWHM of the peaks are indicator of external force quantity.\cite{Zhao2013}

\begin{figure}[htb]
\centering
\includegraphics[width=0.7\textwidth]{ws2_ramsch}
\caption[\ce{MS2} vibration symmetry]{\ce{MS2} vibration symmetry in bulk and monalyer, adopted from REF\cite{Ghorbani-asl}}
\label{fig:ws2ramsch}
\end{figure}

Lattice vibration of natural \ce{MoS2} crystal was studied by \citeauthor{Wieting1971} using infrared and Raman spectroscopy.\cite{Wieting1971} It was found the $E_{1u}$ IR mode and one $E_{2g}$ Raman mode are nearly degenerate in energy. 15 optical modes are allowed assuming 6 atoms in primitive cell. Refractive indices from reflectivity measurement were $n_0$= 3.9, $n_e$ = 2.5. \citeauthor{Stacy1985} studied \ce{MoS2} and \ce{WS2} Raman spectra using lasing energy close to the absorption edges.\cite{Stacy1985} Second order scattering from phonon with nonzero momentum is used to explain the rich Raman spectra. \citeauthor{Sourisseau1991} investigated the resonant Raman profiles in 2H-\ce{WS2} using ten different excitation wavelengths.\cite{Sourisseau1991} Dramatic intensity variation at 352 \si{cm^{-1}} was observed, which is assume to be of two-phonon signal nature, and corresponds to an overtone or combination band of phonons with non-zero momenta contributing to indirect gap absorption edge. \citeauthor{Sourisseau1991} assigned this phonon with non-zero momenta as $LA(K_5)$ type. The enhancement of the total Raman cross section at excitonic resonance in which excitons serve as the intermediate state is stronger compared to that of interband resonance. The strong enhancement at excitonic resonance is attributed to the characteristics of excitons in layered materials such as large binding energy, enhanced oscillator strength, and small damping constant.\cite{Zhao2013} \citeauthor{Chung1998} grew \ce{WS2} film using \ce{W(CO)6} and \ce{H2S} precursor.\cite{Chung1998} Raman spectra ($\lambda=632 nm$) on films with non-parallel orientation revealed the presence of shoulder mode under $A_{1g}$, which is assigned to LA and TA phonon coupling. This coupling process stems from disorder-activated zone boundary phonons. A further discussion on this non-symmetric feature of $A_{1g}$ mode will be continued in section~\ref{sec:ntram}.

% WS2 Raman assignments
\begin{table}
  \centering
  \caption{\ce{WS2} symmetry assignment}  \label{tbl:ws2raman}
  \begin{tabular}{ccccc}
    \toprule
    &&\multicolumn{3}{c}{Raman Shift (\si{cm^{-1}})}\\
    \cmidrule(l){3-5}
    Symmetry                & Atoms & \ce{WS2} ML\cite{Cong2013}  & \ce{WS2} NT \cite{JMR7990865}  & \ce{WS2} bulk \cite{Sourisseau1991} \\
    \midrule
          $E_{2g}^2(\Gamma)$ &      & 27.5\textsuperscript{\emph{a}}&             &  27.4    \\
    $LA(M)-E_{2g}^2(\Gamma)$ &      & 148.3                        &              &    \\
       TBD                   &      &                              & 153          &      \\
         $E_{2g}^1(M)-LA(M)$ &      &                              & 172          & 173  \\
    LA(M)                    &      & 174.8                        & 172          &       \\
    LA(K)                    &      & 192.4                        &              &  193 \\
    $LA(M)+E_{2g}^2(\Gamma)$ &      & 203                          &              &     \\
    $LA(K)+E_{2g}^2(\Gamma)$ &      & 213.9                        &              &  212  \\
    $A_{1g}(M)-LA(M)$        &      & 230.9                        & 230          &  233  \\
    $2LA(M)-3E_{2g}^2(\Gamma)$ &    & 264.2                        & 262          &  267  \\
    $2LA(M)-2E_{2g}^2(\Gamma)$ &    & 295.4                        & 294          &  297   \\
    $2LA(M)-E_{2g}^2(\Gamma)$ &     & 322.9                        &              &  325   \\
               $E_{2g}^1(M)$ &      & 343.1                        &              &      \\
    2LA(M)                   &      & 350.8                        & 350          &  352\\
          $E_{2g}^1(\Gamma)$ &      & 355.4                        & 350          &  356 \\
    $2LA(M)+2E_{2g}^2(M)$    &      &                              & 381          &  381   \\
     LA + TA \cite{Sourisseau1991} or $B_{1u}(\Gamma)$\cite{Staiger2012}  &      &       &   &  416 \\
          $A_{1g}(\Gamma)$   &      & 417.9                        & 416\textsuperscript{\emph{b}} &  421\\
               3LA(K)        &      & 577                          &              &      \\
       $ LA(M)+ A_{1g}(M)$   &      & 584                          & 581          &  585 \\
    4LA(M)                   &      & 704                          &              &  703\\
    \bottomrule
  \end{tabular}

  \textsuperscript{\emph{a}} Calculated from column values;
  \textsuperscript{\emph{b}} \citeauthor{JMR7990865} probably made incorrect assignment of 416 peak.\cite{JMR7990865}
\end{table}

Raman technique has also provided much insight into the few layer \ce{MS2}. \ce{MS2} layer numbers are readily identified by the wavenumber distance between $A_{1g}$ and $E_{2g}^1$ mode for \ce{MoS2}\cite{Buscema2013} and \ce{WS2}\cite{Berkdemir2013}. Yet due to the relative small shift of $A_{1g}$ mode and little shift of $E_{2g}^1$ mode in \ce{WS2} FL, the frequencies distance might not qualify as an unambiguous way to distinguish layer numbers. Yet the resonant Raman profile on \ce{WS2} exhibit unique features between the intensity of 2LA and $A_{1g}$ mode,\cite{Berkdemir2013,Zhao2013} which provide another routine to assure the monolayer presence. However, the author notice some discrepancy in de-convolution of \ce{WS2} resonant profile between 300 and 400 cm$^{-1}$, i.e., the presence of $E_{2g}^1(M)$ mode at about 344 cm$^{-1}$.\cite{Peimyoo2013,Cong2013,Berkdemir2013} Rigid assignments of this mode still requires further theoretical\cite{Ataca2012} and experimental efforts. In addition, \citeauthor{M2013} reported temperature dependent of 1L \ce{WS2}.\cite{M2013} When the temperature increase from 77 K to 623 K, A1g shift from 420 to 416.5 cm$^{-1}$. Interestingly in this report 2LA/$A_{1g}$ ratio (514 nm excitation) seems less than unit. The spectra were obtain from mechanically exfoliated \ce{WS2} lying on 300 nm \ce{SiO2}-Si substrate. Moreover, The $A_{1g}$ and $E_{2g}^1$ intensities ratio exhibit reverse behaviors under 532 and 632 nm excitation. This is caused by the different cross-section enhancement for a specific excitation condition. The A and B excitonic absorption in \ce{WS2} mainly arise from the $d_{xy}$ and $d_{x^2 - y^2}$ states to $d_{z^2}$ states of tungsten atoms. Thus, electrons excited by 633 nm laser have a character of tungsten $d_{z^2}$ orbitals aligned along the $c$ axis perpendicular to \ce{WS2} basal plane. Since $A_{1g}$ mode involves out of plane displacement along $c$ axis, $A_{1g}$ phonons could couple more strongly with $d_{z^2}$ states than that of $E_{2g}^1$ phonons. As a result, $A_{1g}$ mode is stronger than $E_{2g}^1$ mode at 633 nm resonance.\cite{Zhao2013} However, the reverse effect for 532 nm excitation could not be well explained using the above argument. This may be caused by electron-phonon coupling with other inter-band transition electrons.

\subsection{Raman on TMDC NTs}\label{sec:ntram}

\citeauthor{Dobardzic2005} calculated \ce{MoS2} SWNT phonon dispersion. The dependence of wavenumbers and their displacement on chirality and diameter were discussed. The calculation method enables studying lattice dynamics with NT diameter up to 50 nm. The chiral vector $(n_1, n_2)$ is defined within the molybdenum plane. Symmetry assignment is zigzag when $(n,0)$, armchair when $(n,n)$ and chiral when $(n_1, n_2), n_1>n_2$. \citeauthor{Dobardzic2006} theoretically presented Raman scattering of any polarization on SWNT of \ce{WS2} and their dependence on diameter (1-20 nm) and chiral angle. The author assigned 351 cm$^{-1}$ as $E_u$ for \ce{WS2} NT.\cite{Dobardzic2006} \citeauthor{Ghorbani-Asl2013} discuss the electronic and vibrational properties for large diameter \ce{WS2} NTs\cite{Ghorbani-Asl2013}. Single-walled NT is approximated by 1H monolayer and others by 2H bulk structure. It was found that large-diameter nanotubes can be approximated with layered systems as their properties should be nearly the same at the scale. Only hypothetical SWNTs, and possibly MWNTs with alternating layer compositions, may show direct band gaps. Slight mechanical deformation of the SWNTs would result in a change of the direct band gap back to the indirect one, located between $\Gamma$ and $K$ high-symmetry points, similarly to the monolayers. As for 2D materials, quantum confinement to single-walled tubes would result in direct band-gap semiconductors with $\Delta$ occurring at the $K$ point. Single-walled tubes exhibit slightly softer out-of-plane $A'$ and stronger in-plane $E'$ modes. Those results indicate that the weak interlayer interactions in MS$_2$ materials cannot be associated with the van der Waals interactions only, but most probably with Coulomb electrostatic interactions as well.

Raman signatures of \ce{WS2} nanotubes show distinct features to the spectra of their bulk counterpart. \citeauthor{JMR7990865} observed a new line at 152 cm$^{-1}$ in \ce{WS2} NT, which is absent in 2H-\ce{WS2}.\cite{JMR7990865} Another feature is a emerging shoulder on the low energy side of $A_{1g}$ mode at about 416 cm$^{-1}$. This has been attributed to a combination mode of LA + TA phonons from the $K$ point of Brillouin zone.\cite{Sourisseau1991} The shoulder mode associated with $A_{1g}$ is attributed to LA + TA. As pressure increase from 0 GPa to 18 GPa, these two bands, both shifting to higher wavenumbers, first separate and then recombine. It was assumed the compression mainly occurs in $c$ axis, so the stiffening of $A_{1g}$ is anticipated. A more prominent feature is the resonance profile broadening the shape of $E_{2g}^1$ mode, which is often assigned to 2LA mode. Yet there is different opinion on these assignments. \citeauthor{Molina-Sanchez2011} label the 350 \si{cm^{-1}} band as \ce{E_{1u}} instead of 2LA.\cite{Molina-Sanchez2011} And recent theoretical investigations suggest 416 \si{cm^{-1}} peak is inactive $B_{1u}$ mode\cite{Molina-Sanchez2011,Ataca2012}, which is the Davydov doublet with $A_{1g}$ mode. \citeauthor{Staiger2012} adopted these assignments in studying the resonance Raman profile of \ce{WS2} NTs.\cite{Staiger2012} It was found that
\begin{enumerate}
\item $B_{1u}$ mode arise from curvature and structural disorder
\item $B_{1u}/A_{1g}$ intensity ratio strongly depends on excitation, and exceeds unity when excitation energy less than 1.9 eV.
\item  An excitonic transition energy of NT is found have a local minimum at about 50 nm, (layer number probably \textgreater 10), and increase either way. Yet all below the bulk value.
\end{enumerate}

\citeauthor{Krause2009} also measured the resonant Raman on \ce{WS2} nanotubes and found a split within 420 \si{cm^{-1}} region, which is labelled as $D-A_{1g}$ mode in analog with the similar defect mode of graphene.\cite{Krause2009} This  $D-A_{1g}$ mode was found enhanced as diameter of \ce{WS2} NTs decrease. This thesis work will use $B_{1u}({\Gamma})$ mode to interpret this emerging line at about 416 \si{cm^{-1}} of \ce{WS2} NTs, and adopt the 350 \si{cm^{-1}} as 2LA(M). \citeauthor{Krause2009a} confirmed $B_{1u}$ mode arise from the inherent structure of \ce{WS2} nanomaterials instead of surface layer effect. It is also worth noting that $A_{1g}$ is stronger than 2LA under 632 nm yet weaker under 532 nm excitation, similar to previous discussion of FL scenarios.\cite{Krause2009a} This is in consistent with our observation of Raman spectra on the \ce{WS2}-\ce{WO3} structures, as discussed later. \citeauthor{Rafailov2005} estimated the orientation dependence of resonant raman on one MWNT \ce{WS2} attached to the cantilever tip of AFM.\cite{Rafailov2005} Antenna effect lead to optical transition occurring only for polarization parallel to nanotube axis. And therefore resonance Raman intensity of SWNT varies as nanotube orientation. This dependence may provide a routine to distinguish different chiral NTs. Polarized Raman spectra (632 nm excitation) is obtained, showing $A_{1g}$ and $E_{2g}$ sharing the same polarization behavior. \citeauthor{Virsek2007} investigated the Raman scattering ($\lambda=632$ nm) of \ce{WS2} NTs.\cite{Virsek2007} The silicon peak at 520 cm$^{-1}$ is used for calibration. Up-shift of $A_{1g}$ and $E_{2g}$ modes (i.e. 420 to 423 cm$^{-1}$ at $A_{1g}$ mode) were observed, which is attributed to the strain in 3R stacking layers.

As a useful reference, the lattices dimensions are summarized in Table~\ref{tab:ms2lattice},
%lattice
\begin{table}[htb]
\centering
\caption{Lattices dimension of \ce{MS2}}\label{tab:ms2lattice}
\begin{tabular}{lccr}
\toprule
         &  & 2H-\ce{MoS2}\cite{Coehoorn1987,Ataca2012} & 2H-\ce{WS2}\cite{Albe2002,Schutte1987} \\
\midrule
Lattice constant & a(\AA) & 3.1604 & 3.171 \\
                 & c(\AA) & 12.295 & 12.359 \\
Within \ce{MS2} layer & M-3S (\AA)& 2.37  & 2.405   \\
                      & S-1S (\AA)& 3.11  & 3.14   \\
Between \ce{MS2} layer& S-3S (\AA)&   & 3.53   \\
\bottomrule
\end{tabular}
\end{table}


\section{Results and Discussion}

\subsection{synthesis method in this work}

Four kinds of tungsten powders were used as precursor to prepare \ce{WO3} NWs, and sulfur powder was used to fabricate \ce{WO3}-\ce{WS2} core-shell NWs, as summarized in Table.~\ref{tab:ch5pre}. All reactants were employed as received without further processing. Substrates cleaning precudures have been introduced in Sec.~\ref{ch2sub}. 

% Tungsten powders size and purity
\begin{table}[htb]
\centering
\caption{List of reactants}\label{tab:ch5pre}
\begin{tabular}{lccr}
\toprule
Name & purity & average size & vendor info\\
\midrule
3N   &  99.9\% & 17 $\mu$m & Alfa Aesar \#39749\\
3N5   &  99.95\% & 32 $\mu$m  & Alfa Aesar \#42477\\
4N5   &  99.995\% & 3.3 $\mu$m  & Materion T-2049 \\
5N   &  99.999\% & 1.5 $\mu$m & Alfa Aesar \#12973\\
S    &   99.5\%  &  NA  & Alfa Aesar \#10785\\
\bottomrule
\end{tabular}
\end{table}

\ce{WO3} NWs were obtained by a seeded growth with tungsten powders applied onto the Si substrate. To find out the optimal conditions, tungsten powder oxidation experimental were carried out. Schematic layout of each growth were illustated in Fig.~\ref{fig:ch5grow} including the sulfurization process.   

% cvd NW growth
\begin{figure}[htb]
\centering
\includegraphics[width=0.7\textwidth]{cs_sfig1_all}
\caption[\ce{WO3} NW growth: OT]{Schematic diagrams of (a) tungsten power oxidization, (b) seeded growth of \ce{WO3} NW, and (c) sulfurization process. (d) Chemical vapor deposition system temperature profile measured at ambient condition. The zero inch location is defined at the upstream edge of furnace.}
\label{fig:ch5grow}
\end{figure}

In a typical oxidation experiment shown in Fig.~\ref{fig:ch5grow}a, tungsten powders were loaded into the uniform heating zone and the sealed chamber was pumped down to an ultimate pressure of $5\sim8$ mTorr. Then oxygen flow varying between 1 sccm to 10 sccm (standard cubic centimeter per minute) was admitted from upstream inlet. With 10 sccm UHP Ar (99.999\%) as carrier gas, the overall pressure reached to about 100 mTorr. The heating temperature (500 to 750 \si{\degreeCelsius}) was ramped up in 30 min and lasted for 30 min. Then the heating power was turned off and the chamber was allowed to naturally cool down to room temperature.

In a representative seeded growth (Fig.~\ref{fig:ch5grow}b), 4N5 or 5N tungsten source powder (Table.~\ref{tab:ch5pre}) was positioned in the upstream end of a quartz boat and downstream about 2.5 in the substrate was stationed. Additional tungsten seed powders (3N) were distributed evenly onto the substrate. After pumping down, 1 sccm oxygen and 10 sccm Ar were admitted into the chamber, respectively. The heating temperature was ramped up to 1000 \si{\degreeCelsius} in 30 min and lasted for 240 min. Subsequently the apparatus was allowed to naturally cool down to room temperature. 

\ce{WO3}-\ce{WS2} core-shell NWs were synthesized using direct sulfurization of \ce{WO3} NWs. As depicted in Fig.~\ref{fig:ch5grow}c, the as-grown \ce{WO3} NWs were loaded into the center of heating furnace, and $\sim200$mg sulfur (Alfa Aesar 10785, 99.5\%) was positioned just outside the upstream edge of furnace, where the maximum temperature was about 240 \si{\degreeCelsius}. After pumping down, the reaction chamber was flushed two times to expel residual air. A cold trap filled with liquid nitrogen in downstream was used to collect possible sulfur precipitation. Then the furnace was heated to 750 \si{\degreeCelsius} in 30 min, held for 15 min, and allowed to naturally cool down to room temperature. During entire growth process, 30 sccm Ar was used as carrier flow.


\subsection{Characterizations}

\subsubsection{Optimization of W powder oxidization}\label{sec:woxd}

Understanding the oxidation of tungsten powder is the key to obtain high yield in seeded growth. Oxidation of tungsten have been investigated under diverse conditions, such as at elevated temperature (\textgreater 1100 \si{\degreeCelsius} ) and oxygen pressure on the order of Torr,\cite{Base1965} and at temperatures ranging from 20 to 500 \si{\degreeCelsius} under atmosphere pressure.\cite{Warren1996} \ce{WO_x} NWs were readily found when tungsten (foil, wire, or powder) is oxidized under various conditions.\cite{Zhu1999,Karuppanan2007,Hsieh2010} However the study on tungsten powder oxidation behavior between intermediate temperature range and under low pressure is still rare. This work studied the oxidation of tungsten powders with diverse size within temperature range from 500 to 1000 \si{\degreeCelsius} and under several mTorr oxygen partial pressure. Using tungsten powder as seed, the author further illustrate an economic approach to obtain large yield of \ce{WO3} nanowires at relatively lower temperature than previous efforts. It is demonstrated that there is an optimal tungsten powder size under our experimental conditions for seeded growth. This will provide some insight on the role of tungsten powder as source material in CVD growth of \ce{WO_x}.

Commercial available tungsten powders with different size are usually associated with purity variation as well, as already listed in Table.~\ref{tab:ch5pre}. The dimensions of tungsten powder were obtained by measuring the average size in SEM graphs. A systematic investigation was performed on the oxidation behavior of tungsten powder to evaluate the temperature effect, size-dependence and influence of oxygen partial pressure.
% seed optimal
\begin{figure}[htb]
\centering
\includegraphics[width=0.6\textwidth]{JAP-2column_Fig1.jpg}
\caption[W powder oxidation: temperature effect]{SEM graphs of 99.9\% (3N) tungsten powder oxidization at different temperatures of a) 500 \si{\degreeCelsius}, b) 600 \si{\degreeCelsius}, c) 650 \si{\degreeCelsius}, d) 750 \si{\degreeCelsius}, showing the optimal temperature for local formation of nanowires is between 600--650 \si{\degreeCelsius}. Oxygen flow rate is 1 sccm.}
\label{fig:pdtemp}
\end{figure}

Fig.~\ref{fig:pdtemp} illustrated the effect of temperature on the morphological change and surface nanowires formation of 3N powder. At 500 \si{\degreeCelsius}, most tungsten powder retained its original shape and a layer of tiny dense NWs begun to grow. When temperature was increased to 600 \si{\degreeCelsius}, 3N powder started to crack with longer NWs on the isolated surface. Further increase of temperature lead to irregular shapes of tungsten power and aggregation of NWs, giving rise to the nanorods and bunched or bundled structures. It could be determined from the morphology variation that the optimal seeded growth temperature for 3N powder was in the range of 600 to 650 \si{\degreeCelsius}.
% seed optimal
\begin{figure}[htb]
\centering
\includegraphics[width=0.6\textwidth]{JAP-2column_Fig3.jpg}
\caption[W powder oxidation: size effect]{SEM graphs illustrating the oxidization of four different size of tungsten powders at 600~\si{\degreeCelsius} and 1 sccm oxygen flow. a) 17 $\mu m$, b) 32 $\mu m$, c) 3.3 $\mu m$, d) 1.5 $\mu m$.}
\label{fig:pdsize}
\end{figure}

Fig.~\ref{fig:pdsize} depicted the oxidation of different sizes of tungsten powder under the same experimental conditions. In contrast to the morphology of 3N powder shown in Fig.~\ref{fig:pdtemp}, 3N5 powder surface is primarily covered with sub-micro particles as well as some short tiny NWs, whereas 4N5 and 5N powder were thoroughly oxidized, showing branched flowers feature. This dramatic difference could be explained in terms of surface energy and oxygen diffusion. With smaller dimension, the increased surface-to-volume ratio and short diffusion path both lower the energy barrier of oxidation.\cite{tungsten1999} It was logical to deduce that higher temperature or increased oxygen level might favor the NWs formation on 3N5 powder. When it comes to seeded growth, however, the powder size distribution was an important factor to give uniform NWs deposition. Since the size distribution of 3N powder is more uniform than that of 3N5 powder, the author employed the former as seeds.
% seed optimal
\begin{figure}[htb]
\centering
\includegraphics[width=0.6\textwidth]{JAP-2column_Fig2.jpg}
\caption[W powder oxidation: oxygen pressure]{SEM graphs of 3N tungsten powder oxidization at 600 \si{\degreeCelsius} under different rates of oxygen flow: a) 1 sccm, b) 2 sccm, c) 3 sccm, d) 10 sccm. The oxygen partial pressures were 13 mTorr, 23 mTorr, 32 mTorr, and 82 mTorr, respectively with background pressure subtracted.}
\label{fig:pdoxy}
\end{figure}

Fig.~\ref{fig:pdoxy} depicted the morphology change of 3N powder with respect to varied oxygen partial pressure. When the oxygen flow is lower than 3 sccm, 3N powder almost stayed as the same, with cracks separating the dense layer of NWs. When oxygen flow is increased to 10 sccm, the 3N powder exemplified an enlarged version of that for 4N5 or 5N powder under 1 sccm oxygen flow. This observation again supported the surface energy explanation.

\subsubsection{Seeded growth of tungsten oxide NWs}
With the oxidation experiments study in Sec.~\ref{sec:woxd}, favorable conditions for local growth of NWs were extracted to perform seeded growth. 

\begin{figure}[htb]
\centering
\subfloat[]{\label{fig:sga}\includegraphics[width=0.4\textwidth]{wox_sg_a.jpg}}\hspace{0.04\textwidth}
\subfloat[]{\label{fig:sgb}\includegraphics[width=0.4\textwidth]{wox_sg_b.jpg}}
\caption[Characterization of seeded growth \ce{WO3}: SEM]{Characterizations of seeded growth. (a) SEM graphs of \ce{WO3} NWs on \ce{SiO2/Si} substrate. (b) A high magnification view showing uniform NW growth and close-up view of one NW. }
\label{fig:woseedsem}
\end{figure}

As shown in Fig.~\ref{fig:sgb}, dense NWs array was obtained on tungsten powder seeds with individual wires of length up to 5 $\mu$m and diameter about 50 to 200 nm, according to the estimation made in the close-up view. Each tungsten powder stood as independent growth site (Fig.~\ref{fig:sga}) with island-layer growth on the substrates, a common feature without using tungsten powder as seed under current experimental conditions. It was occasionally observed that NWs growth was initiated adjacent some tungsten powders. This phenomenon was correlated to the local trap of vapor flow since it was more often found among the enclosed area by tungsten powders. It is also found that the diameter of NWs decrease as the distance between powders and upstream edge increases. This is a combination effect of lower temperature and reduced \ce{WOx} growth species supply. Similar phenomena were observed in other studies. \citeauthor{Thangala2007} reported that a decrease in NW density with increasing substrate temperature, and an increase of NW density with increasing partial pressure of oxygen.\cite{Thangala2007}

% seeded edx 
\begin{figure}[htb]
\centering
\includegraphics[width=0.5\textwidth]{wo3_seed_edx}
\caption[Composition analysis on seeded growth \ce{WO3} NWs]{EDX spectroscopy on seeded growth \ce{WO3} NWs.}
\label{fig:woedx}
\end{figure}
Energy-dispersive X-ray spectroscopy (EDX) analysis on the seeded growth \ce{WO3} NWs is shown in Fig.~\ref{fig:woedx}. Only W and O elements were detected on the NW array. The background level from 3 to 8 keV was a manifestation of the continuous components of W X-ray spectrum. 

% sg raman xrd
\begin{figure}[htb]
\centering
\subfloat[]{\label{fig:sgxrd}\includegraphics[width=0.45\textwidth]{xrd_cs_before}}\hspace{0.04\textwidth}
\subfloat[]{\label{fig:sgram}\includegraphics[width=0.45\textwidth]{wox_raman_1}}
\caption[Characterization of seeded growth \ce{WO3}: XRD and Raman]{ (a) XRD pattern of as-prepared sample indicating the \ce{WO3} phase and the presence of metallic core. (b) Raman spectrum on NWs region showing the feature of \ce{WO3}.}
\label{fig:woseedxrd}
\end{figure}

Fig.~\ref{fig:sgxrd} is the XRD spectrum of one typical sample. The peaks under circular symbol were identified to be the monoclinic \ce{WO3} phase (ICDD PDF 01-083-0950, \emph{a}=7.30084 \AA, \emph{b}=7.53889 \AA, \emph{c}=7.6896 \AA, $\beta$=90.89$^\circ$), while the peak under the triangular symbol was indexed to cubic tungsten phase (ICDD PDF 04-16-3405, \emph{a}=3.157 \AA), in agreement with the EDX analysis (Fig.~\ref{fig:woedx}). This means that during the \ce{WO3} seeded growth of 4 h heating at 1000 \si{\degreeCelsius}, the tungsten powder in downstream low temperature region (600 -- 700 \si{\degreeCelsius}) is not entirely oxidized. Micro-Raman scattering spectroscopy was performed on the as-synthesized sample as well. During Raman examination, the laser spot was carefully focused onto the NWs on powders and several inspections on different positions were observed to ensure the reproductivity of spectra data. As shown in Fig.~\ref{fig:sgram}, five distinct bands were well resolved, with peaks located at 131, 265, 327, 711 and 803 \si{cm^{-1}}, respectively. This pattern was typical features of \ce{WO3}, consistent with previous study.\cite{Salje1975a,Dixit1986} The high background level probably arises from the metallic core.

% sg tem
\begin{figure}[htb]
\centering
\includegraphics[width=0.9\textwidth]{JAP-2column_Fig5major.jpg}
\caption[Characterization of \ce{WO3}: TEM]{TEM graphs of as-prepared NWs. (a) TEM image of one nanowire with diameter about 40 nm. (b) HRTEM images showing the spacing is 0.38 nm, corresponding to (002) plane distance.}
\label{fig:woseedtem1}
\end{figure}

TEM specimen was prepared by using carbon grid to slightly scratch the as-grown sample. Fig.~\ref{fig:woseedtem1} shows the feature of majority NWs. The growth direction is determined to be perpendicular to (002) plane. The streaking in SEAD pattern presumably arises from stacking defaults. The author also find some NWs exhibit high crystalline quality, as revealed by the TEM analysis in Fig.~\ref{fig:woseedtem2}. The NW grew normal to (002) plane with a measured lattice spacing of 3.79 \AA, which is favorably compared to the XRD peak at $23.07^\circ$ (7.7103 \AA). The sharp SEAD pattern and clear phase contrast in HRTEM demonstrated are both strong evidence of good crystallinity. This formation indicated current growth parameters have promising potential to obtain highly crystalline \ce{WO3} NWs in large scale. 

% sg tem
\begin{figure}[htb]
\centering
\includegraphics[width=0.9\textwidth]{JAP-2column_Fig5minor.jpg}
\caption[Characterization of \ce{WO3}: TEM cont]{TEM graphs of as-prepared NWs. (a) TEM image of one nanowire, the diameter is about 70 nm. (b) HRTEM images showing the spacing is 0.379 nm, corresponding to (002) plane distance.}
\label{fig:woseedtem2}
\end{figure}

In regarding to the formation of NWs on tungsten powder itself, this study assumes the driving force is related to interfacial strain between W and \ce{WOx}. Oxidation of tungsten proceed slowly at room temperature and an oxide layer of 100 \AA was found on the surface of tungsten foils.\cite{Warren1996} The tungsten powder used in current study would be covered by a thin oxide layer as well. During oxidation, different oxidation rates exist for different crystallographic orientations on the tungsten powder. Oxidation occurring at boundaries and defects are preferred thermodynamically.\cite{You2010} Compressive strain will gradually accumulate at the tungsten oxide/tungsten interface, which might limit the diffusion rate of oxygen at temperature lower than 500 \si{\degreeCelsius}.\cite{tungsten1999} At elevated temperature, cracks will primarily occur, as observed in Fig.~\ref{fig:pdtemp}. When heated up, tungsten and the oxide shell will probably relax the strain by converting into substoichiometric NWs, a similar process as suggested by \citeauthor{Klinke2005} in the chemically induced strain growth of tungsten oxide NWs.\cite{Klinke2005} It is worth noting that tungsten oxide nanowires could also formed when \ce{WO3} is reduced.\cite{Sarin1975} The elongation of \ce{WOx} is thermodynamically favorable during the conversion from metallic tungsten to tungsten oxide as well. Local evaporation-condensation process might also contribute to the formation of NWs on tungsten powder.

The enhanced yield of NWs obtained via seeded growth could be explained by a vapor-solid (VS) mechanism. External supply of growth species will condense onto the powders and substrate simultaneously, promoting the elongation of NWs on power as well as resulting island-layer growth on substrate. The local NW density in oxidation experiment was much higher than that of seeded growth. It is reasonable to presume that during seeded growth, several NWs in a small region on powder will coalescence, as evidenced by the bundled structures. At last, the author would like to point out that when low purity tungsten powder (3N) was used as source, sodium tungsten oxide nanowires were found to be dominant in the final product. The details have been published.\cite{Sheng2014} It seems surprising that when 3N powder was used as seeds, only \ce{WO3} nanowires were obtained. This result was attributed to the lower temperature and significantly reduced amount of 3N powder used in the seeded growth, compared with the conditions used in Ref.\cite{Sheng2014}. The source material in seeded growth is not limited to high purity tungsten powder. Instead, any material that could produce appropriate growth vapor could be employed, indicating the versatility of this approach.
\subsubsection{Ensemble measurements on core-shell NWs}

The morphology of \ce{WO3} NWs in micron scale almost stay the same after sulfurization, as shown in the inset SEM imaging of Fig.~\ref{fig:ch5ws2sem}. The length of individual nanowire can be up to 15 $\mu$m and the diameter varies from about 40 to 300 nm. The surface of the NWs were rather smooth without other attachments. The NWs array grow slightly larger than the W seed ( $\sim 12 \mu$m), mostly due to the elongation of NW. 

\begin{figure}[htb]
\centering
\includegraphics[width=0.7\textwidth]{ws2_sem_edx}
\caption[Morphology and composition analysis on \ce{WO3}-\ce{WS2} NWs]{Morphology and composition analysis on the as-synthesized core-shell structures. (a) SEM images showing the NWs almost retain their original shapes after sulfurization, and (b) EDX spectroscopy revealing the presence of sulfur element after sulfurization.}
\label{fig:ch5ws2sem}
\end{figure}

The EDX spectrum in Fig.~\ref{fig:ch5ws2sem} revealed the presence of sulfur element, indicating the effectiveness of sulfurization. Compared to the XRD pattern on seeded growth \ce{WO3} sample, post-sulfurization pattern (Fig.~\ref{fig:ch5ws2xrd}) shows a much reduced \ce{WO3} intensity with W phase becoming relatively prominent. Still the peaks at 23.44$^{\circ}$ and 24.36$^{\circ}$ can be indexed to \ce{WO3} (020) and (200) reflection, respectively. This result as well as the SEM imaging indicates \ce{WO3} NWs could preserve some long range order after the sulfurization for 15 min at 750 \si{\degreeCelsius}. 

%fig xrd ws2 
\begin{figure}[htb]
\centering
\includegraphics[width=0.7\textwidth]{ws2_cs_xrd}
\caption[XRD on core-shell NWs]{X-ray diffraction spectra before and after sulfurization. The reference spectra for $c$-W, 2H-\ce{WS2} and $m$-\ce{WO3} are included with some major peaks labeled.}
\label{fig:ch5ws2xrd}
\end{figure}

This observation can also be used to account for the extreme small peak at 13.87$^{\circ}$ in the post-sulfurization spectrum. Compared the reference 2H-\ce{WS2} (002) peak (14.37$^{\circ}$, ICDD PDF 04-003-4478), this redshift indicates an increase of lattice spacing along $c$ axis (from 6.16 \AA to 6.38 \AA). The FWHM of this (002) peak is about 1.1$^{\circ}$, giving an estimation on the crystalline size of 7.5 nm according to Scherrer equation. 

Due to this small spacing, TEM is primarily utilized to observe the detailed morphology of \ce{WO3}-\ce{WS2} core-shell NWs. Fig.~\ref{fig:ch5ws2tem1} shows HRTEM imaging on typical tip and body region of the core-shell NW, respectively. A comparison with the pre-sulfurization specimen (Fig.~\ref{fig:woseedtem2}) readily reveals the formation of core-shell morphology after sulfurization. The measured layer spacing is 6.4 \AA, slight larger than bulk 2H-\ce{WS2} of 6.16 \AA, but in good agreement with XRD results in Fig.~\ref{fig:ch5ws2xrd}. The clear contrast in core region also prove the lattice structure of \ce{WO3} is not seriously distorted.

% tem core shell layer
\begin{figure}[htb]
\centering
\includegraphics[width=0.9\textwidth]{ws2_coreshell_tem}
\caption[TEM imaging on core-shell NWs: 1]{HRTEM imaging on one core-shell nanowire showing layer number changes.}
\label{fig:ch5ws2tem1}
\end{figure}

After examining dozens of core-shell NWs, the author found the \ce{WS2} layer number at NW tip was usually the largest, reaching more than ten, and the layer number could decrease to one along the NW body, but there exist some layer number fluctuation and shape modification. Fig.~\ref{fig:ch5ws2tem3} shows a smooth transition of \ce{WS2} layer number on the tip region of one NW. The layer number is about 7 on the tip corner, gradually decreases from 4 to 3 along the NW body, and stops at 2. Closer observation indicates the layer spacing is not quite uniform in the tip area, probably due to the complex curvature and bending. \ce{WS2} fullerene also show similar structures. Since the closing of the tip region could be of great complexity and the induced strain could modify Raman signal considerably, the TEM-Raman integrated measurement in this study will focus on the body region only, with those having smooth transition of layer number.
% coreshell tem
\begin{figure}[htb]
\centering
\includegraphics[width=0.8\textwidth]{ws2_growth3}
\caption[TEM imaging on core-shell NWs: 2]{HRTEM imaging on the tip area of one core-shell nanowire.}
\label{fig:ch5ws2tem3}
\end{figure}

As illustrated in Fig.~\ref{fig:ch5ws2tem4}, complicated \ce{WS2} encapsulation was observed on one tapered core-shell NW. This tapering probably arose from the sulfurization process, but could be also due to the parent \ce{WO3} NW. The diameter varied from about 200 nm to 15 nm, and the \ce{WS2} layer number also changed in a nonlinear manner. On some turning region, the \ce{WS2} seems penetrating into the body of primary \ce{WO3} NW. 

% coreshell tem
\begin{figure}[htb]
\centering
\includegraphics[width=0.8\textwidth]{ws2_growth4}
\caption[TEM imaging on core-shell NWs: 3]{HRTEM imaging on one tapered core-shell nanowire showing fluctuation of layer numbers}
\label{fig:ch5ws2tem4}
\end{figure}


A systematic study on the sulfurization must be performed to understand the growth dynamics. The growth mechanism of \ce{WS2} nanotubes from oxides were thoroughly studied,\cite{Feldman1998} and it is generally agreed that the sulfurization of \ce{WO3} is a diffusion limited reaction.\cite{Feldman1996} It was also suggested that the core \ce{WO3} NWs will become non-stoichiometric phases upon sulfurization.\cite{Feldman1996,ZAK2009} In this study, the author recognize this possibility but also consider two factors different from previous studies: one is the usage of sulfur as precursor instead of \ce{H2} and/or \ce{H2S} as in previous reports, which will accelerate the reduction of \ce{WO3} significantly due to the formation of \ce{H2O}; the other is the short reaction time and light degree of sulfurization, which probably will not distort the \ce{WO3} lattice significantly, as evidenced by the XRD and HRTEM analysis in this section. The sulfurization dynamic is then proposed as following in this thesis: the tips of \ce{WO3} NWs exhibit greater crystalline disorder, and \ce{WS2} layers on tips are folded at least along two axes, resulting a higher density of defects and thus a higher sulfurization rate; while the folding along NW body is quasi-one dimensional, subsequently the sulfurization rate could be lower. Besides, sulfurization could also initiated from the NW body, presumably due to the presence of surface defects. As a result, this led to the uneven \ce{WS2} layer number distribution along the core-shell NWs.

Among most of the core-shell NWs examined under TEM, the author observed no discernible consumed core region. An exception is shown in Fig.~\ref{fig:ch5ws2tem2}. It is worth mentioning that some interesting growth behavior around the void region, where a new \ce{WS2} layer pushing through the already formed \ce{WS2} ones, deviating from the diffusion growth model. 

% coreshell tem
\begin{figure}[htb]
\centering
\includegraphics[width=0.8\textwidth]{ws2_growth1}
\caption[TEM imaging on core-shell NWs: 4]{HRTEM imaging on one core-shell nanowire showing consumed core.}
\label{fig:ch5ws2tem2}
\end{figure}


\subsection{TEM-Raman integrated study} 

To perform TEM-Raman integrated study, the core-shell specimen was first dispersed into ethanol solution and then dipped onto polydimethylsiloxane (PDMS) support. A home-built micromanipulator with proble tip (tungsten, 100nm point radius, Micromaipulator) was used to pick up the core-shell NWs and transfer them onto TEM grids (Lacey C, 300 mesh, Cu, 01895-F Ted Pella, Inc) under 100x objective. The steps are visualized in Fig.~\ref{fig:ch5ws2trans}. 
% ws2 transfer
\begin{figure}[htb]
\centering
\includegraphics[width=0.5\textwidth]{ws2trans}
\caption[Manipulating core-shell NWs]{Core-shell nanowires transfer steps. }
\label{fig:ch5ws2trans}
\end{figure}

The transferred NWs were first examined using TEM to identify the \ce{WS2} layers configuration and the associated geometrical orientation. Then the core-shell NWs on TEM grids were probed under Raman in a location-resolved manner. Only those core-shell NWs exhibiting relative smooth \ce{WS2} wall number transition was used in subsequent Raman measurement (wall number and layer number were used interchangeably in this work, but 1 wall is not equal to 1 layer due to the tubular structure). To perform TEM and Raman analysis on individual NWs, only one NW was placed in each cell on the TEM grid (65 $\times$ 65 $\mu$m). And the low magnification TEM images and optical images were compared to map the NW location and orientation. The carbon film patterns served as a unique background to resolve possible ambiguity. An example is given in Fig.~\ref{fig:ch5ws2map}. 
% ws2 map
\begin{figure}[htb]
\centering
\includegraphics[width=0.6\textwidth]{ws2map}
\caption[Mapping core-shell NWs]{Core-shell nanowires mapping. }
\label{fig:ch5ws2map}
\end{figure}

Fig.~\ref{fig:ch5ws2ram}a-d depict typical \ce{WO3}-\ce{WS2} NWs used in TEM-Raman integrated study. The diameter of these core-shell NWs is around 100 nm and the length around 10 $\mu$m, allowing for up to five distinctive Raman scattering sites. Fig.~\ref{fig:ch5ws2ram}e shows the associated Raman spectra acquired on core-shell NWs using 532 nm excitation. Two strong peaks were observed between 300 and 450 \si{cm^{-1}}. The resonant nature\cite{Stacy1985} of Raman shift is recognized by the broad band at about 350 \si{cm^{-1}}.  Recent theoretical\cite{Molina-Sanchez2011} and experimental\cite{Staiger2012} studies on 2D and tubular \ce{WS2} agree this band consist of first order mode $E_{2g}^1$ and second order mode $2LA(M)$, although there is still discrepancy on the exact symmetry assignment, \emph{i.e.}, the presence of $E_{2g}^1(M)$ mode.\cite{Berkdemir2013,Peimyoo2013}. The other peak at about 420 \si{cm^{-1}} is assigned to $A_{1g}$ mode in both bulk\cite{Sekine1980} and few layer \ce{WS2}. Tubular \ce{WS2} structure can be viewed as rolling up the planar \ce{WS2} sheet, and new Raman features arise subsequently, \emph{e.g.}, a shoulder mode emerges on the low frequency side of $A_{1g}$ mode. Theoretical investigation suggests assigning this mode as $B_{1u}$, the Davydov doublet of $A_{1g}$ mode.\cite{Ataca2012} And experimental study confirmed this $B_{1u}$ mode arise from structural disorder of \ce{WS2} layers, and the intensity ratio between $B_{1u}$ and $A_{1g}$ strongly depended on excitation wavelength.\cite{Staiger2012} This study observed $B_{1u}$ mode under 532 nm is much less prominent than that of 632 nm, in consistent with previous reports.\cite{Krause2009,Krause2009a} 

% fig Raman TEM
\begin{figure}[htb]
\centering
\includegraphics[width=0.7\textwidth]{ws2_tem_ram}
\caption[TEM-Raman integrated characterization on core-shell NWs]{(a)-(d) HRTEM images on several NWs showing the \ce{WS2} wall number variation from one to about ten. (e) Raman spectra acquired from core-shell NWs supported on TEM grids, indicating a sharp contrast between the in-plane and out-of-plane vibrations. Notice that the Raman spectra primarily arise from the outer \ce{WS2} shell, and only weak Raman shift at about 800 \si{cm^{-1}} were observed at some bare core region (not shown here). (f) Layer number dependence on averaged intensity ratio between the in-plane and out-of-plane Raman bands. }
\label{fig:ch5ws2ram}
\end{figure}

Multi-peak Lorentzian fitting was also attempted on spectra shown in Fig.~\ref{fig:ch5ws2ram}e. Similar trend was found between the apparent peak ratio and the de-convolution value, as shown in Fig.~\ref{fig:ch5ws2pr}. For a qualitative purpose, the overall intensity ratio $I_{E/A}$ between in-plane E band ($E_{2g}^1+2LA$) and out-of-plane A band ($A_{1g}+B_{1u}$) is sufficient. This value is then extracted to correlate with layer number variation. Multiple Raman spectra for each layer were averaged, as shown in Fig.~\ref{fig:ch5ws2ram}f. 

% fig Raman TEM
\begin{figure}[htb]
\centering
\includegraphics[width=0.4\textwidth]{ws2_peak_ratio}
\caption[Multi-peak Lorentzian fitting on core-shell NWs]{Comparison of apparent peak ratio and de-convolution value.}
\label{fig:ch5ws2pr}
\end{figure}

The intensity ratio $I_{E/A}$ decreases monotonically with the increase of \ce{WS2} wall number, a similar trend as observed on 2D few layer \ce{WS2}. The exact value of $I_{E/A}$ shows modest difference with 2D sheet, which is attributed to the suspended configuration of current core-shell NWs specimen and the presence of \ce{WO3} core. It has been demonstrated that the optical interference from dielectric environment significantly modified the Raman scattering of layered MoS2, regardless lattice vibration modes.\cite{Li2012} The author calculated the reflectivity of three different stacking scenarios to estimate the influence of \ce{WO3} and insulating substrate, as shown in. It can be concluded that under 532 nm excitation, the \ce{WO3} core shows little influence on the Raman intensity when \ce{WS2} tube is suspended in air, whereas the \ce{SiO2}/Si substrates will dramatically enhance the Raman scattering of the core-shell NWs. It could be further predicted that current Raman spectra on core-shell NWs can be used to estimate the \ce{WS2} nanotube wall numbers. 




\section{summary} 


\section{literature to read}

The geometrical symmetry groups of WS2 NT.\cite{Milosevic2000} 
The basis/lattice vectors $a_1$ and $a_2$ are defined in transition metal plane with equal length $a_0 \approx 3$ \AA. This monolayered sheet can be rolled up to form a tubular structure when the chiral/translation vector $c = n_1a_1 + n_2a_2$ becomes the circumference of the tube. The diameter of a nanotube is given by 
\begin{equation}
d = \frac{\mid c \mid}{\pi} = \frac{a_0}{\pi}\sqrt{n_1^2 + n_1n_2 + n_2^2}
\end{equation}
Notice $a_1 \cdot a_2 = a_1a_2 \cos\pi/3$, where the cosine times 2 is unit. 
In analog with carbon nanotube, chiral angle $\theta$ is given by
\begin{equation}
\theta = \tan^{-1}\frac{\sqrt{3}n_2}{2n_1 + n_2}
\end{equation}

Chiral angles from the interval $[0,pi/6]$ is sufficient for all possible tubes. Tubes $(n,0)$ with zero chiral angle are named zigzag, tubes $(n,n)$ with $\pi/6$ chiral angle are armchair, whereas all others are referred as chiral ones with $\theta \in (0,pi/6)$. 



hydrogen evolution catalysts. \cite{Merki2011}

Raman substrate dependence \cite{Buscema2013}

stability of TMS NTs \cite{Seifert2002}

2H and 1T in \ce{MoS2} \cite{Eda2012}

\ce{MoS2} FET statistical study. \cite{Liu2013i}

water splitting review. \cite{B800489G}

\ce{WS2} theory and experimental combined study. \cite{Klein2001}

2D review on oxides \cite{Osada2012}

Pb catalyzed \ce{MoS2} nanotube \cite{Brontvein2012}

\ce{WS2} Raman.\cite{Zhao2013} \cite{Sekine1980}

phonon dispersion $E_{2g}^1(M)$? \cite{Ataca2012}

MoS2 optical properties.\cite{Search1979}

FL heterostructure. \cite{Yu2013a}

\cite{Kang2013} TMDC alloy DFT.

thermoelectric TMDC \cite{Wickramaratne2014}

\ce{CH4N2S} thiourea + \ce{WOx} to \ce{WS2} \cite{Leonard-Deepak2011}

\ce{WS2} by \ce{WCl_n} and \ce{H2S}, raman (632nm) show bulk features\cite{Tenne2008}.

direct gap of ML at corner of BZ, point K.

\begin{align}
\cee{WCl6 + S &\rightarrow WS2 + Cl2S2}\\
\cee{MoCl5 + S &\rightarrow MoS2 + S2Cl2} \\
\cee{S2Cl2 + NaOH &\rightarrow NaCl + S + Na2SO3 + H2O}
\end{align}

\section{literature read}

CNT chirality by TEM \cite{Zhang1993} TEM chirality of \ce{MoS2} NTs


first demonstration of \ce{WS2} NT n-type FET. \cite{Levi2013}
the importance of contact, and avoiding moisture. 

The calculated carrier concentration is about $10^{19}cm^{-3}$, a highly doped semiconductor, possibibly arising from sulfur vacancy. 

\ce{WS2} NT transport \cite{Zhang2012c}
less grain boundary more mobility, 

\citeauthor{Ramasubramaniam2011} investigated the band gap tuning in bilayer TMDC materials by applying external $E$ field. Similar research has been done for graphene and bilayer boron nitride. Semiconductor-metal transition was suggested for \ce{MoS2} and \ce{WS2}, with difference on the CBM and VBM evolution. In \ce{MoS2}, the valence-band-splitting cause the A and B excitons in optical absorption measurement. Calculation shows that CB and VB are translated toward the Fermi level with increasing E field.  The external field localized charge along $c$ axis, but delocalized that within the plane normal towards $c$, thereby driving the semi-metal transition. It was mentioned that this transition is not anticipated in monolayer \ce{MoS2}. It was emphasized that precise band gaps might be different from the author’s results, yet the gap-tuning should be universal.\cite{Ramasubramaniam2011}

\cite{Song2013} \ce{WO3} by ALD, and sulfurized in Ar and \ce{H2S} (10:1) at 1000 C. \ce{WS2} layer No and peak intensities ratio under 633nm excitation is correlated. It was found the 2LA/A1g is less than 1 for 1L. In supporting info 532 nm Raman spectra, the 2LA/A1g is presumably larger than 1 for 1L. \ce{WS2} NT on Si NWs is also demonstrated.

\cite{Tenne2010} chemical modification of NTs. Functional ligand consists of an anchor group that attaches to the NTs surface and a tail which render them soluble in various solvents. PTAS functionalized BN nanotube lead to the formation of stable suspensions in aqueous solutions. The strong attachment is formed through $\pi-\pi$ interactions.


inorganic nanotubes review \cite{Tenne2004} , unsaturated bonds number increases as the size of MS2 sheet decreases.

Water splitting materials should process a band gap larger than 1.4eV, considering both the NHE potential distance and practical application. The monolayer \ce{WS2} exhibits direct gap of 1.98 eV. Quantum confinement could push the gap separation farther away. \cite{wilcoxon1997} \citeauthor{Notley2013} use liquid exfoliation to prepare \ce{WS2} NPs.\cite{Notley2013} Non-ionic surfactant concentration is about 0.1\%w/w. Continuously adding surfactant during sonication improves the yield. Optimum surface tension is found at about 40 mJ/m$^2$.

G/\ce{WS2}/G stacked solar cell. For lubricant, and surface protection. Absorption $\sim 10^7 m^{-1}$. \cite{Britnell2013}

In Ref\cite{Zeng2013}, single crystal \ce{WS2} growth using \ce{I2} transport was described in supporting info.
\ce{C24H12K4O8}\footnote{http://www.chemspider.com/Chemical-Structure.24771386.html}



416 peak was prevously assumed to be a combination of LA and TA phonons at K points.

Raman conditions: 0.3 mW, 532 nm, 200 sec acquisition time. photon flux = $0.3E-3\times6.242E+18/1.2398/0.532=2.84E15$

416 cm raman on WS2. B1u is pressure sensitive. \cite{Staiger2012} also studied pressure dependence.

A1g mode using resonance Raman excitation shows upshift, which is presumably caused by folding induced strains. And TEM SAED reveal 3R symmetry. We did not observed this upshift of A1g probably due to the large diameter and few layer involved.

Nanotube growth is relatively independent of substrates and FL layer is closely related to substrate. Film growth could provide some insight into the latter scenario.

