\begin{abstract}
Transition metal oxides (TMOs) exhibit rich structures and useful properties and could be used in applications of solar energy harvesting (\emph{e.g.}, photoelectrochemistry, photocatalysis, and photovoltaics) and energy saving (\emph{e.g.}, electrochromism and photochromism). Nano-engineering of these TMOs could produce a variety of nanostructures, modify the electronic and optical properties, and potentially enhance the performances and extend their applications into new regimes. To fully harvest the advantages of these materials, a scalable growth method and a comprehensive understanding towards the growth-structure-property relation must be established. This doctoral study focused on one-dimensional (1D) nanostructures of tungsten trioxide (\ce{WO3}) and molybdenum trioxide (\ce{MoO3}). This dissertation synthesized 1D \ce{WO3} and \ce{MoO3} nanostructures using the chemical vapor deposition (CVD) approach, characterized the morphologies and structures, and measured the optical properties of the as-prepared nanostructures. 

This dissertation consists of three main parts, each centering on \ce{WO3}, \ce{MoO3}, and \ce{WO3}-\ce{WS2}, respectively. Systematical investigation of the \ce{WO3} nanostructure prepared by the CVD method with tungsten powders as precursor led to the following accomplishments:
\begin{enumerate*}[label=\itshape\alph*\upshape)]
\item the discovery of \ce{Na5W14O44} nanowires (NWs);
\item a knowledge of the delicate role of sodium in tungsten source powders; and
\item a seeded growth method to produce high quality \ce{WO3} NWs.
\end{enumerate*} Enlightened by the intricate interactions between sodium content and tungsten oxide, this dissertation extended the vapor-liquid-solid strategy in to \ce{MoO3} 1D nanostructure growth and discovered two morphologies of \ce{MoO3}: long nano-belts and micro-towers. Careful observation in extensive experiments identified the catalytic role of sodium hydroxide during the growth of \ce{MoO3} and readily proved that the catalytic behavior also exists with other alkali metal compounds. The third part of this work moved the investigation center from growth to structure and property characterization. A novel integrated structure-property (transmission electron microscopy-Raman spectroscopy, TEM-Raman) study was performed on \ce{WO3}-\ce{WS2} core-shell NWs, allowing for the observation of high resolution TEM imaging, Raman spectroscopy, and photoluminescence spectroscopy on individual nanowires. The TEM-Raman combined results further illustrated that the profiles of resonant Raman spectra exhibit strong dependence on the wall number of \ce{WS2}. This Raman fingerprints can provide a rapid approach for the optical identification of few-walled \ce{WS2} tubular structures.

In summary, this study explored methods for controlled and scalable growth of 1D \ce{WO3} and \ce{MoO3} nanostructures based on both vapor-solid and vapor-solid-solid approaches, characterized the structure-property relations to rationalize nanostructure synthesis for energy applications, and presented preliminary results for the device fabrications using the \ce{WO3} and \ce{MoO3} nanostructures.
\end{abstract}