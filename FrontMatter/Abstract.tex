\begin{abstract}

Transition metal oxides (TMOs) exhibit rich structures and useful properties, and could been used in solar energy harvesting (\emph{e.g.}, photoelectrochemistry, photocatalysis, and photovoltaics) and energy saving (\emph{e.g.}, electrochromism and photochromism) applications. Nano-engineering of these TMOs could produce a variety of nanostructures, modify the electronic and optical properties, potentially enhance the performances and extend their applications into new regimes. To fully harvest the advantages of these materials, a scalable growth method and a comprehensive understanding towards the growth-structure-property relation must be established. This Ph.D. study focused on one-dimensional (1D) tungsten trioxide (\ce{WO3}) and molybdenum trioxide (\ce{MoO3}) nanostructures. This dissertation synthesized 1D \ce{WO3} and \ce{MoO3} nanostructures using chemical vapor deposition (CVD) approach, characterized the morphologies and structures, and measured the optical properties of the as-prepared nanostructures. This study aims at accomplishing controlled and scalable growth of \ce{WO3} and \ce{MoO3} using CVD method, establishing the structure-property relations to rationalize nanostructure synthesis for energy applications, and presenting preliminary results on the devices assembled using the \ce{WO3} and \ce{MoO3} nanostructures.

\end{abstract}