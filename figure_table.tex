\subsection{misc}

\si{\degreeCelsius}

\si{\angstrom}

Inline lists, which are sequential in nature, just like enumerated lists, but are
\begin{enumerate*}[label=\itshape\alph*\upshape)]
\item formatted within their paragraph;
\item usually labelled with letters; and
\item usually have the final item prefixed with `and' or `or',
\end{enumerate*} like this example.

http://tex.stackexchange.com/questions/19675/siunitx-threeparttables-and-table-footnotes


\subsection{artwork}
% art work global settings
%  origin:

 all tick in, line color: blue or black, line width: 2; tick : 18pt;
 layout area(in): left:1, top:0.3; width 10: height 7; total
 layout area(in): left:1, top:0.8; width 9.6: height 7; total 12 by 9
 export: 3in auto
 text size: 22
% SEM
2.5in by 2in, 300dpi; gray scale

the width of band gap is a measure of chemical bond strength

\subsection{figure table units}
\si{W/in^2}
\si{\degreeCelsius}

\begin{equation}\label{eq:mclns}
\cee{MoCl5(WCl6) + S \rightarrow MoS2(WS2) + S2Cl2}.
\end{equation}

 WO3: dielectrics constants: $\epsilon_\alpha = 6.52,\epsilon_\infty > 50$. \cite{Deb2008}
\begin{figure}[htb]
\centering
\subfloat[tw2]{\label{fig:tw2}\includegraphics[width=5cm]{Tangwei2}}\hspace{0.04\textwidth}
\subfloat[tw3]{\label{fig:tw3}\includegraphics[width=5cm]{Tangwei3}}

\subfloat[tw4]{\label{fig:tw4}\includegraphics[width=10cm]{Tangwei4}}
\caption{snapshot of Late Autumn}
\label{fig:tw}
\end{figure}

\begin{figure}[htb]
\centering
\subfloat[]{\label{fig:mosem1}\includegraphics[width=0.45\textwidth]{mosemsi_a}}\hspace{0.04\textwidth}
\subfloat[]{\label{fig:mosem2}\includegraphics[width=0.45\textwidth]{mosemsi_b}}
\caption[Representative morphologies of \ce{MoO3} on Si]{(a) Low magnification and (b) high magnification SEM images of representative depositions of \ce{MoO3} on Si for non-catalytic growth.}
\label{fig:mosisem}
\end{figure}

\begin{figure}[htb]
\centering
\includegraphics[width=0.8\textwidth]{CVD_and_temp_MoO3.jpg}
\caption[CVD system]{Chemical vapor system and its temperature profile.The length of heating zone is 6 inches. On the horizontal axis of temperature, zero inch is defined at the upstream edge of furnace.The triangular labels $\blacktriangledown$ were measured points at ambient pressure.The nominal substrates temperature were estimated from interpolation data.}
\label{fig:s1}
\end{figure}

\begin{figure}[htb]
\centering
\subcaptionbox{\label{fig:moxrd}}{\includegraphics[width=0.45\linewidth]{xrd_moo3}}%
\subcaptionbox{\label{fig:moram}}{\includegraphics[width=0.45\linewidth]{xrd_moo3}}
\caption[Crystalline phase characterization of \ce{MoO3} on Si]{(a) XRD pattern and (b) Raman spectrum of typical \ce{MoO3} on Si, $\lambda_{ex} = 532nm$.}
\label{fig:mooxch}
\end{figure}

\begin{figure}[htb]
\centering
\subcaptionbox{\label{fig:moind}}{\includegraphics[width=0.45\linewidth]{n_MoO3}}%
\subcaptionbox{\label{fig:mocon}}{\includegraphics[width=0.45\linewidth]{MoO3_1L}}
\caption[Refractive indices of \ce{MoO3}]{(a) Refractive indices of \ce{MoO3} Dashed line is n along $a$ axis while solid line is along $c$ axis. (b) optical contrast mapping of 1L \ce{MoO3} on \ce{SiO2}-Si. The x axis is \ce{SiO2} thickness, y axis is wavelength.It is assumed that the incident light is polarized along c axis of \ce{MoO3}.}
\label{fig:mofl}
\end{figure}

%float package
\begin{figure}[htb]
\centering
\subcaptionbox{\label{fig:moind}}{\includegraphics{n_MoO3}}%
\subcaptionbox{\label{fig:mocon}}{\includegraphics{Graph20}}
\caption[Refractive indices of \ce{MoO3}]{(a) Refractive indices of \ce{MoO3} Dashed line is n along $a$ axis while solid line is along $c$ axis. (b) optical contrast mapping of 1L \ce{MoO3} on \ce{SiO2}-Si. The x axis is \ce{SiO2} thickness, y axis is wavelength.It is assumed that the incident light is polarized along c axis of \ce{MoO3}.}
\label{fig:mofl}
\end{figure}

\begin{align}
\cee{MB + e_{CB}^- &->[pH<7] MB^{.-} \\
2MB^{.-} &\rightarrow MB + LMB\\
O2 + e_{CB}^- &\rightarrow O2^{.-}}
\end{align}

\begin{align}
\cee{ WS2-WO3 + h\nu &-> e^- + h^+\\
O2 + e^- &-> O2^{.-}\\
OH_{surface}^- + h^+ &-> .OH\\
MB^+ + O2^{.-} + .OH-> \text{degradation phases}
}
\end{align}


\begin{align}
\cee{WO3(V) &-> W(S) + 3/2O2 \qquad T= 1150 \si{\degreeCelsius}\\
        WO3(V) &-> WO2(S) + 1/2O2}
\end{align}

\begin{align}
W + xSiO{_2} &\rightarrow WO{_x} + xSiO{_2},\\
W + SiO{_2}   &\rightarrow WO{_x} + WSi{_2},
\end{align}

\begin{align}
x\ce{M+} + x\ce{e-} +  \ce{$\alpha \hyphen$WO$_{3-y}$}= \ce{$\alpha \hyphen$M$_x$WO$_{3-y}$},
\end{align}

\begin{align}
h\nu +\ce{W^{5+}(A)} +  \ce{W^{6+}(B)} &\rightarrow \ce{W^{5+}(B)} + \ce{W^{6+}(A)} \label{eq:cl_bl1}\\
h\nu +\ce{W^{5+}(A)} +  \ce{W^{4+}(B)} &\rightarrow \ce{W^{5+}(B)} + \ce{W^{4+}(A)} \label{eq:cl_bl2}
\end{align}

\begin{table}[htb]
\centering
\caption{Source of figures }\label{tab:sof}
\begin{tabular}{lcccr}
\toprule
\multicolumn{2}{c}{Flow} \\
\cmidrule(l){4-5}
         & Temperature & Pressure & Ar & \ce{O2}  \\
\midrule
Range      & RT-1100    & 10mTorr-1atm & 0 - 100 & 0-30  \\
Resolution & $\pm1$  & correlated to flow & 1   & 0.1  \\
\bottomrule
\end{tabular}
\end{table}

% Adding notes to tables can be complicated.
\begin{table}
  \centering
  \caption{A table with notes}  \label{tbl:notes}
  \begin{tabular}{ll}
    \toprule
    Header one                            & Header two \\
    \midrule
    Entry one\textsuperscript{\emph{a}}   & Entry two  \\
    Entry three\textsuperscript{\emph{b}} & Entry four \\
    \bottomrule
  \end{tabular}

  \textsuperscript{\emph{a}} Some text;
  \textsuperscript{\emph{b}} Some more text.
\end{table}


%\newgeometry{margin=1in}
\begin{landscape}
\begin{table}[htb]
\centering
\caption{Tungsten Oxides Growth and Application}\label{tab:wox}
{\footnotesize
\begin{tabular}{lp{3.5in}p{2.5in}c}
\toprule
composition  &  methods & highlights &  reference  \\
\midrule
\ce{WO3} & hot W filament (above 1500 \si{\degreeCelsius}) in Ar/\ce{O2} flow  & Cubic phase, PL, resistivity measured & \cite{Thangala2007} \\
\addlinespace[0.5em]
& W filament DC heating in \ce{NH3} or \ce{N2}/\ce{H2} flow  & multi phases, 100mg per batch, stable dispersion in both organic and aqueous solvents & \cite{Chang2007} \\
\bottomrule
\end{tabular}
}
\end{table}
\end{landscape}
%\restoregeometry

\begin{table}[htb]
\centering
\caption{Symmetry assignment of \ce{WO3}}\label{tab:woram}
\begin{tabular}{cccc}
\toprule
Raman shift (\si{cm^{-1}}) & Symmetry &  Ref.   & Remarks   \\
\midrule
131(m)   & $B_{2g}$    &           & $T_c$    \ce{WO3} 
\bottomrule
\end{tabular}
vw-very weak; w-weak; m-medium; s-strong; vs-very strong
\end{table}


%\newgeometry{margin=1in}
%\begin{landscape}
\begin{sidewaystable}
\begin{table}[htb]
\centering \small
\caption{Tungsten Oxides Growth and Application}\label{tab:wox}
\begin{tabular}{lccr}
\toprule
composition  &  methods & highlights &  reference  \\
\midrule
\ce{WO3}     & \ce{Na2WO4.2H2O} mixed with \ce{(NH4)2Fe(SO4)2.6H2O} or%
 \ce{CoCl2.6H2O} followed by hydrothermal heating  & 3 different morphologies and photodegradation  & \cite{Rajagopal2009}  \\
Resolution & $\pm1$  & correlated to flow & 1     \\

\bottomrule
\end{tabular}
\end{table}
\end{sidewaystable}
%\end{landscape}
%\restoregeometry


\section{vocabulary}
well suited to,
seriously pursued. 
One must be just as aware of the instrument's limitations as one is of its advantages
I call attention to the fact that.
arise as a result of, dispersal of Na by electron probe.
$\delta\omega/\epsilon$ 

antidazzle window, electrochemical intercalation, coloration-bleaching kinetics, present as a minor phase, 
(i.e. )
(cf. )

put under the spotlight, barely satisfactory, decrement of,  more numerous than elsewhere, draw attention to it. not as restrictive an assumption as it may seem.

up-scaling capability, band edge smeared; major bands situated at, 
bypassing a limitation, held back by a restriction, discern,  nanoscopy, run in parallel tracks, line of thought take shape, delve into, devise, commotion, fruition, foundation was laid, one variant, ponder, passivated, halt, lack of progress, p-dope FaN in a controlled manner, transcend, theoretical description, and experimental implementation, 

stochastic, doctrine, ample evidence, suits - diamonds, clubs, hearts, and spades 

settling towards minimum energy, a free electron of momentum $p$ cab be represented by a wave of wavelength $\lambda = h/p$. to the extent to which theoretical results correspond with reality. apparent agreement of theory and experiment carries no guarantee that . another school of thought, 



bottom-up paradigm. curiosity was drawn toward. exemplify. unambiguous interpretation. pronounced, Nevertheless, highlight the necessity , As a consequence, proved beneficial. insight we have gained from our investigations. solve this conflict. critical ingredients. regime. dominate and reveal itself. 

closer inspection, consecutive, a characteristic feature of aggregation process. on the basis of present results, some conclusion could be drawn concerning the .. The observation that .... indicated .... ,discerned, discrepancy, deviating, deviation, accompanied, 

progressively consumed and converted into sulfide. ensuing purification, a matter of controversy, 

entail, prerequisite, kinetically hindered, rate-determining step, confirmed experimentally. closed cap. various synthetic methodologies.
grouping into two categories, phenomenological arguments, primarily, establish a systematic connection. 
in this letter, reason reside in, sizing histogram, vessel, implementation of , predominantly, stacking sequence.
naturally abundant, fundamentally intriguing and technologically attractive, thermodynamically preferred. 
pronounced effect, strategy for extending methodology, modify, change, alter. 

exothermic, endothermic. occurrence, impediment, batches. prevail in accordance with 

measured properties compare favorably with theoretical analysis. 
elimination of dangling bonds. conformal coating. Equation is visualized in Fig.
fluidic dynamics, entangled web, dimensional heterostructure, kinetically hindered. preceding text, more specifically, hold great potential. 
residence time, crucial aspects. temperature gradient. diverse function, distinct experiment. 

trigonal prismatic coordination, bridging coordination types. intercalation mechanism been elucidated 
hydration. not always deleterious, branched nanocrystal, orientation dictated by substrates, moscovite mica, concurrently, as described herein. applicable, suggestive, anion-terminated facets. phase segregation. 

Inorganic fullerenes, transitional metal dichalcogenides. postulate 

vast, disorder, orderliness, irregularly, solidifies, 
a constant competition between the attractive force which hole the atoms together and the irregular motions which tend to knock them apart. 
Liquid and gas are alike in their ability to flow but unlike in the volumes they occupy, while liquid and solid are alike in the volumes they occupy but unlike in their ability to flow. 

the departure of molecules from most solids at ordinary temperature is so infrequent that the evaporation is imperceptible. 
one kind of disorder and limitless orderliness. endless variety, diversity, ingredient, grows into being. They adopt positions on the surface that are forced on them by the kind of orderliness confronting them. Settling into those positions, they extend the orderliness outward, and thus the crystal grows. 

manifest their presence in various ways. impartially, small proportion, interstitial vacancy, point, line, plane disorder. the few layers of atoms near the surface will be slightly more compressed  inward than the bulk of the crystal. entangle, agitate, the increasing viscosity of liquids as their temperature goes down is a measure of the increasing difficulty with which the molecules move past one another. 

more distant past, rock, mineral, percolate, steam boiler, 
Snow flakes grow directly from moist air-in other words, from a gas. Carbon dioxide fizzes out of soft drinks when you release the pressure. dissolved or mixed, 
Gases are usually less soluble in a liquid when it is hot than when it is cold, but the reverse is usually true of solids. The opposite effects arise from the same cause: increasing temperature tend to increase their disorder. 

True solution: they are dispersed through one another molecularly, not in little chunks. 
The solubility of the substituent is limited, much as is the solubility of salt in water. retain the crystalline arrangement. 

Zinc and copper, silver and gold. 
When molecules of solute are moving about in a liquid, they cannot go as far in as short a time as they can when they are moving about in a gas: the diffusion rate is much slower. Perfume in air, 

Soldering depends on dissolving a little of the solid metal in the molten solder to form an alloy at the boundary; 

Factors that influence whether a substance will dissolve in something else: the random motion of the molecules, tending to separate and disorder them, wage a constant war with the attractive forces.  

The associations between the molecules of solute and of solvent keep forming and breaking and forming again. 

water reduces the electric forces more than almost any other liquid does, a reason why water is an effective solvent for materials whose molecular constituents bear electric charges.  A bare ion attracts a few water molecules, behaving like a new charged compound. It is this charged molecule, not the bare ion, which moves around through the solution. (hydration of an ion) 

Attached water, watery clothing. 




\subsection{pronunciation}

chamber, vias, valve, figure, energetic, managerial, inert, volatile, chromic,
laminar, palladium, platinum, photovoltaic, acronym, chirality, stoichiometric, cyclic voltammetry, quasi, pseudo, 

