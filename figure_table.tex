

\subsection{artwork}
% art work global settings
%  origin:

 all tick in, line color: blue or black, line width: 2; tick : 18pt;
 layout area(in): left:1, top:0.3; width 10: height 7; total
 layout area(in): left:1, top:0.8; width 9.6: height 7; total 12 by 9
 export: 3in auto
 text size: 22
% SEM
2.5in by 2in, 300dpi; gray scale

the width of band gap is a measure of chemical bond strength

\subsection{photocatalyst}
A low recombination rate is preferred for high photocatalytic efficiency. The simultaneous migration of electrons and holes.
\textbf{\ce{WS2}-\ce{WO3}}: 1 kW light source(Hg, or Xe lamp), photon flux, phenol (\ce{C6H5OH}, 94.1g/mol, MP 40C)concentration is 20 mg/L, hydroxyl group. The quantitative analysis of phenol was performed via a standard colorimetric method.\footnote{\url{http://omlc.ogi.edu/spectra/PhotochemCAD/html/072.html}}
\citeauthor{DiPaola1999} prepared \ce{WS2}-\ce{WO3} mixture in two methods, sulfurization of \ce{WO3} and oxidation of \ce{WS2},with the latter are more active.
\citeauthor{DiPaola1999} also concluded that the actual efficiency of mixed \ce{WS2}-\ce{WO3} catalysts depends on the ratio of each composition present of the surface of the particles, and the maximum of photoactivity is obtained with 40-50\% surface molar ratio of \ce{WS2}.

ref 25, 28 and 41.

\textbf{\ce{MoO3}}:

\citeauthor{Sreedhara2013} studied the kinetics of photodegradation of methylene blue\footnote{\ce{C16H18N3SCl},319.8 g/mol, MP: 100C accompanied with decomposition \url{http://en.wikipedia.org/wiki/Methylene_blue}} dye by few layer \ce{MoO3}.
For the photodegradation method, it was stated that `` the samples were collected after the photoreaction had been centrifuged for 5 min to remove the photocatalyst before UV-Vis measurement.''
\subsection{tms}
%%%%%%%%%%%% literature tms %%%%%%%%%

(Ramasubramaniam, Naveh, \& Towe, 2011) investigate the band gap tuning in bilayer TMD materials by applying external E field. Similar research has been done for graphene and bilayer boron nitride. Semiconductor-metal transition was suggested for \ce{MoS2} and WS2, with difference on the CBM and VBM evolution. In \ce{MoS2}, the valence-band-splitting cause the A and B excitons in optical absorption measurement. Calculation shows that CB and VB are translated toward the Fermi level with increasing E field.  The external field localized charge along C axis, but delocalized that within the plane normal towards C, thereby driving the semi-metal transition. It was mentioned that this transition is not anticipated in monolayer \ce{MoS2}. It was emphasized that precise band gaps might be different from the author’s results, yet the gap-tuning should be universal.

(Shi, Pan, Zhang, \& Yakobson, 2013) studied the strained monolayer \ce{MoS2} and WS2. The results show that exciton binding energy is insensitive to the strain, while optical band gap becomes smaller as strain increases. Monolayer WS2 PL maximum located at about 1.95eV. Calculation shows the electron effective mass of WS2 is the smallest, rendering higher mobility in device.

(Kośmider \& Fernández-Rossier, 2013) studied the heterojunction between two monolayers of \ce{MoS2} and WS2. Top of VB in W layer and bottom of CB in Mo layer, forming type II structure. bilayer gap 1.2 eV.


Lattice vibration in hexagonal \ce{MoS2} was studied by (Wieting & Verble, 1971) using infrared and Raman spectroscopy. 15 optical modes are allowed assuming 6 atoms in primitive cell. Refractive indices was calculated as n0= 3.9, ne = 2.5. \cite{Wieting1971}

Band structure  of \ce{MoS2} in bulk form was calculated by (Mattheiss, 1973). There is some controversy about the exact value of bandgap. The calculation result is 1.2eV ( indirect gap).\cite{Mattheiss1973}

Alkali metal intercalated \ce{WS2} film was prepared.(Homyonfer et al., 1997) Stage 6 superlattice formation was suggested according to X-ray diffraction, and photoresponse spectra and electron tunneling measurement were done.

Sulfurizaiton of W film, (Jäger-Waldau, Lux-Steiner, Jäger-Waldau, & Bucher, 1993)

(Splendiani et al., 2010) reported the PL in monolayer \ce{MoS2}.  Calculation indicated the indirect gap become larger when thinning, while the previous direct one almost stays as the same, the value is about 1.85eV (direct gap).\cite{Splendiani2010}

(Zhou et al., 2010) heterojunction is employed to transferred photo-generated carriers. schottky barrier conduction band electron trapping and consequent longer electron-hole pair lifetimes. Numerous studies have suggested that fine particles of transition metals or their oxides, when dispersed on the surface of a photocatalyst matrix, can act as electron traps on n-type semiconductors.

thermal decomposition of (NH4)2MoO2S2 and intermediate product MoOS2 was studied. application: hyfrodesulfurization in refinery (Weber et al. 1996) \cite{Weber1996}

\cee{MoCl5 + 1/4S8 + 5/2H2 \rightarrow MoS2 + 5HCl} (Stoffels et al. 1999)


(M Remskar et al. 1999) WS2 nanotube from WO3-x whisker. heating 840 degree under flow of H2/N2/H2S ,

A detailed study by (Rothschild, Sloan, and Tenne 2000) tungsten filament oxidation by water. then WS2 nanotube from WO3-x whisker. heating 840 degree under flow of H2/N2/H2S ,

IF MoS2 synthesis by MoO3 nanobelt and S.(X. L. Li and Li 2003)

Inorganic core –shell nanotube, WS2@MoS2 core-shell NT.(Kreizman et al. 2010)

vdW Epitaxy of MoS2 on graphene. (Y. Shi et al. 2012)

WS2 monolayer and intense photoluminescent behavior. Sulfurization of WO3 film( 5-20 angstrom). (Gutiérrez et al. 2012)

MoO3 on sapphire reduction in H2 at 500 and sulfurizatin at 1000 degree. (Lin et al. 2012)

quantitative Raman of MoS2 on insulating subs. intensity difference between supported and suspended was highlighted, detailed model in support info.(S.-L. Li et al. 2012)

MoS2 on SiO2/Si sub pretreated with OTAS, PTCDA solution. (Y.-H. Lee et al. 2012)

WO3-x (1nm) on SiO2/Si sulfurization at 750-950 degree, (Elías et al. 2013)\cite{Elias2013}

source MoS2 powder, Ar On sapphire, SiO2/Si, QWP circularly polarized  light (S. Wu et al. 2013)

(Wang et al. 2013)
MoS2 CVD (tsinghua univ) layer by layer sulfurization of MoO2
MoO3 powder 25mg and MnO2 powder 25 mg, extra sulfur
Ar purging  20mins, heating at 650 degree to produce MoO2 flakes
Further reaction between MoO2 and S was performed at 850-950 degree under Ar flow.

MoS2 on SiO2, sapphire, and graphite by MoCl5 and S. (Yu et al. 2013)
growth time: 10min at 850 degree, P: 2 Torr.


review of inorganic 2D materials, (Chhowalla et al., 2013)

(Ramasubramaniam, 2012) decrease in dielectric screening and thereby enhanced excitonic effect.
DFT is not good at describing photoemission, GW approximation overcome this deficiency but still not enough for photoabsorption process in which ehps are created. BSE equation is used to compensate this discrepancy,
WX2 exhibits larger spin-orbit splitting as compared to MX2 family.

reaction mechanism of \ce{MoO3} to \ce{Mo2S}.\cite{Weber1996}


\citeauthor{Ling2014} studied the role of seeding promoters in CVD growth of FL \ce{MoS2}.\cite{Ling2014} PTAS treated substrates provided nucleation site and thus enable uniform deposition of \ce{MS2}.  This enhancement perhaps arise from the \ce{K+} ions.

CNT chirality by TEM \cite{Zhang1993} TEM chirality of \ce{MoS2} NTs

arise as a result of, dispersal of Na by electron probe. 

%%%%%%%%%%%%%%%%%%%%%%
\subsection{figure table units}
\si{W/in^2}
\si{\degreeCelsius}

 WO3: dielectrics constants: $\epsilon_\alpha = 6.52,\epsilon_\infty > 50$. \cite{Deb2008}
\begin{figure}[htb]
\centering
\subfloat[tw2]{\label{fig:tw2}\includegraphics[width=5cm]{Tangwei2}}\hspace{0.04\textwidth}
\subfloat[tw3]{\label{fig:tw3}\includegraphics[width=5cm]{Tangwei3}}

\subfloat[tw4]{\label{fig:tw4}\includegraphics[width=10cm]{Tangwei4}}
\caption{snapshot of Late Autumn}
\label{fig:tw}
\end{figure}

\begin{figure}[htb]
\centering
\subcaptionbox{\label{fig:moind}}{\includegraphics{n_MoO3}}%
\subcaptionbox{\label{fig:mocon}}{\includegraphics{Graph20}}
\caption[Refractive indices of \ce{MoO3}]{(a) Refractive indices of \ce{MoO3} Dashed line is n along $a$ axis while solid line is along $c$ axis. (b) optical contrast mapping of 1L \ce{MoO3} on \ce{SiO2}-Si. The x axis is \ce{SiO2} thickness, y axis is wavelength.It is assumed that the incident light is polarized along c axis of \ce{MoO3}.}
\label{fig:mofl}
\end{figure}


\begin{align}
\cee{WO3(V) &-> W(S) + 3/2O2 \qquad T= 1150 \si{\degreeCelsius}\\
        WO3(V) &-> WO2(S) + 1/2O2}
\end{align}

\begin{align}
W + xSiO{_2} &\rightarrow WO{_x} + xSiO{_2},\\
W + SiO{_2}   &\rightarrow WO{_x} + WSi{_2},
\end{align}

\begin{align}
x\ce{M+} + x\ce{e-} +  \ce{$\alpha \hyphen$WO$_{3-y}$}= \ce{$\alpha \hyphen$M$_x$WO$_{3-y}$},
\end{align}

\begin{align}
h\nu +\ce{W^{5+}(A)} +  \ce{W^{6+}(B)} &\rightarrow \ce{W^{5+}(B)} + \ce{W^{6+}(A)} \label{eq:cl_bl1}\\
h\nu +\ce{W^{5+}(A)} +  \ce{W^{4+}(B)} &\rightarrow \ce{W^{5+}(B)} + \ce{W^{4+}(A)} \label{eq:cl_bl2}
\end{align}

\begin{figure}[htb]
\centering
\includegraphics[width=0.8\textwidth]{CVD_and_temp_MoO3.jpg}
\caption[CVD system]{Chemical vapor system and its temperature profile.The length of heating zone is 6 inches. On the horizontal axis of temperature, zero inch is defined at the upstream edge of furnace.The triangular labels $\blacktriangledown$ were measured points at ambient pressure.The nominal substrates temperature were estimated from interpolation data.}
\label{fig:s1}
\end{figure}

\begin{table}[htb]
\centering
\caption{Source of figures }\label{tab:sof}
\begin{tabular}{lcccr}
\toprule
\multicolumn{2}{c}{Flow} \\
\cmidrule(l){4-5}
         & Temperature & Pressure & Ar & \ce{O2}  \\
\midrule
Range      & RT-1100    & 10mTorr-1atm & 0 - 100 & 0-30  \\
Resolution & $\pm1$  & correlated to flow & 1   & 0.1  \\
\bottomrule
\end{tabular}
\end{table}

% Adding notes to tables can be complicated.
\begin{table}
  \centering
  \caption{A table with notes}  \label{tbl:notes}
  \begin{tabular}{ll}
    \toprule
    Header one                            & Header two \\
    \midrule
    Entry one\textsuperscript{\emph{a}}   & Entry two  \\
    Entry three\textsuperscript{\emph{b}} & Entry four \\
    \bottomrule
  \end{tabular}

  \textsuperscript{\emph{a}} Some text;
  \textsuperscript{\emph{b}} Some more text.
\end{table}


%\newgeometry{margin=1in}
\begin{landscape}
\begin{table}[htb]
\centering
\caption{Tungsten Oxides Growth and Application}\label{tab:wox}
{\footnotesize
\begin{tabular}{lp{3.5in}p{2.5in}c}
\toprule
composition  &  methods & highlights &  reference  \\
\midrule
\ce{WO3} & hot W filament (above 1500 \si{\degreeCelsius}) in Ar/\ce{O2} flow  & Cubic phase, PL, resistivity measured & \cite{Thangala2007} \\
\addlinespace[0.5em]
& W filament DC heating in \ce{NH3} or \ce{N2}/\ce{H2} flow  & multi phases, 100mg per batch, stable dispersion in both organic and aqueous solvents & \cite{Chang2007} \\
\bottomrule
\end{tabular}
}
\end{table}
\end{landscape}
%\restoregeometry




%\newgeometry{margin=1in}
%\begin{landscape}
\begin{sidewaystable}
\begin{table}[htb]
\centering \small
\caption{Tungsten Oxides Growth and Application}\label{tab:wox}
\begin{tabular}{lccr}
\toprule
composition  &  methods & highlights &  reference  \\
\midrule
\ce{WO3}     & \ce{Na2WO4.2H2O} mixed with \ce{(NH4)2Fe(SO4)2.6H2O} or%
 \ce{CoCl2.6H2O} followed by hydrothermal heating  & 3 different morphologies and photodegradation  & \cite{Rajagopal2009}  \\
Resolution & $\pm1$  & correlated to flow & 1     \\

\bottomrule
\end{tabular}
\end{table}
\end{sidewaystable}
%\end{landscape}
%\restoregeometry

\subsection{chromic}

the electrical conductivity is given by $\sigma = n e \mu$, where $n$ is the density of free carriers, $\mu$ is their mobility and $e$ is electronic charge. The mobility is given by $\mu = e\tau/m^*$ with $\tau$ is carrier resistivity relaxation time and $m^*$ is the carrier effective mass.


The degree of \ce{H^+} intercalation, $x$, is determined by integrating cathodic current, and given by
\begin{equation}
x = \frac{Q}{F}\frac{M_\ce{WO3}}{m_f}
\end{equation}
where $m_f$ is the mass of film, $M_\ce{WO3}$ is the molar mass of \ce{WO3*1/3H2O}, and F is Faraday's constant.
The color center density, $c$, is then obtained using formula
\begin{equation}
c= \frac{x M_\ce{WO3}N_A}{\rho}
\end{equation}
where $N_A$ is Avogadro number, $\rho$ is density of film.


In second state, the one with smaller droplet probably would exhibit faster growth rate in $\langle001\rangle$ direction due to its shortest lattice constant, while the other one endures more growth along$\langle010\rangle$ direction. We suspect that the liquid size that induce the tower growth is much larger than that of belts. From the final morphology of as-synthesized sample, we find that the above proposed two growth approaches could change into each other, that is belt growth could be initiated on the top of tower growth and vice visa. We assume this phenomena arise from the interaction between vapor supply, temperature and the shape of liquid catalyst. Or it may just come from the vapor transport of \ce{NaxMoO3}.
It had been shown by actively engineering the shape of catalysis, the growth direction of  InP NW could be switched between [111] and [100].\cite{Wang2013c} The same mechanism might exist in our case as well. For instance, when a belt is sufficient long and it could extend into the low temperature part, where the supply of MoOx vapor is reduced due to the consumption, and the shape of liquid might vary as well. Both will result in different absorption and subsequent diffusion rate along and inside the liquid catalyst. We also observed that belt growth could initiate from tower growth in reduced time growth. Actually this may be one possible mechanism of belts formation.

(The presence of \ce{Na6MoO33} phase is an indirect evidence for the formation of eutectic \ce{Na2MoO4}- \ce{MoO3} binary compounds at growth temperature. The bulk phase diagram may not accurately represent the phase transition occurring in catalyst droplet and solid interface. And it will usually lead to a significant temperature decrease of eutectic point from the bulk value. Nanoparticles do not completely melt and instead act as an active site for reactant absorption and diffusion, leading to a vapor-solid-solid growth mechanism. )

