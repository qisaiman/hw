\subsection{misc}

\si{\degreeCelsius}

\si{\angstrom}

Inline lists, which are sequential in nature, just like enumerated lists, but are
\begin{enumerate*}[label=\itshape\alph*\upshape)]
\item formatted within their paragraph;
\item usually labelled with letters; and
\item usually have the final item prefixed with `and' or `or',
\end{enumerate*} like this example.



\subsection{artwork}
% art work global settings
%  origin:

 all tick in, line color: blue or black, line width: 2; tick : 18pt;
 layout area(in): left:1, top:0.3; width 10: height 7; total
 layout area(in): left:1, top:0.8; width 9.6: height 7; total 12 by 9
 export: 3in auto
 text size: 22
% SEM
2.5in by 2in, 300dpi; gray scale

the width of band gap is a measure of chemical bond strength

\subsection{figure table units}
\si{W/in^2}
\si{\degreeCelsius}

\begin{equation}\label{eq:mclns}
\cee{MoCl5(WCl6) + S \rightarrow MoS2(WS2) + S2Cl2}.
\end{equation}

 WO3: dielectrics constants: $\epsilon_\alpha = 6.52,\epsilon_\infty > 50$. \cite{Deb2008}
\begin{figure}[htb]
\centering
\subfloat[tw2]{\label{fig:tw2}\includegraphics[width=5cm]{Tangwei2}}\hspace{0.04\textwidth}
\subfloat[tw3]{\label{fig:tw3}\includegraphics[width=5cm]{Tangwei3}}

\subfloat[tw4]{\label{fig:tw4}\includegraphics[width=10cm]{Tangwei4}}
\caption{snapshot of Late Autumn}
\label{fig:tw}
\end{figure}

\begin{figure}[htb]
\centering
\subfloat[]{\label{fig:mosem1}\includegraphics[width=0.45\textwidth]{mosemsi_a}}\hspace{0.04\textwidth}
\subfloat[]{\label{fig:mosem2}\includegraphics[width=0.45\textwidth]{mosemsi_b}}
\caption[Representative morphologies of \ce{MoO3} on Si]{(a) Low magnification and (b) high magnification SEM images of representative depositions of \ce{MoO3} on Si for non-catalytic growth.}
\label{fig:mosisem}
\end{figure}


\begin{figure}[htb]
\centering
\subcaptionbox{\label{fig:moxrd}}{\includegraphics[width=0.45\linewidth]{xrd_moo3}}%
\subcaptionbox{\label{fig:moram}}{\includegraphics[width=0.45\linewidth]{xrd_moo3}}
\caption[Crystalline phase characterization of \ce{MoO3} on Si]{(a) XRD pattern and (b) Raman spectrum of typical \ce{MoO3} on Si, $\lambda_{ex} = 532nm$.}
\label{fig:mooxch}
\end{figure}

\begin{figure}[htb]
\centering
\subcaptionbox{\label{fig:moind}}{\includegraphics[width=0.45\linewidth]{n_MoO3}}%
\subcaptionbox{\label{fig:mocon}}{\includegraphics[width=0.45\linewidth]{MoO3_1L}}
\caption[Refractive indices of \ce{MoO3}]{(a) Refractive indices of \ce{MoO3} Dashed line is n along $a$ axis while solid line is along $c$ axis. (b) optical contrast mapping of 1L \ce{MoO3} on \ce{SiO2}-Si. The x axis is \ce{SiO2} thickness, y axis is wavelength.It is assumed that the incident light is polarized along c axis of \ce{MoO3}.}
\label{fig:mofl}
\end{figure}

%float package
\begin{figure}[htb]
\centering
\subcaptionbox{\label{fig:moind}}{\includegraphics{n_MoO3}}%
\subcaptionbox{\label{fig:mocon}}{\includegraphics{Graph20}}
\caption[Refractive indices of \ce{MoO3}]{(a) Refractive indices of \ce{MoO3} Dashed line is n along $a$ axis while solid line is along $c$ axis. (b) optical contrast mapping of 1L \ce{MoO3} on \ce{SiO2}-Si. The x axis is \ce{SiO2} thickness, y axis is wavelength.It is assumed that the incident light is polarized along c axis of \ce{MoO3}.}
\label{fig:mofl}
\end{figure}

\begin{align}
\cee{MB + e_{CB}^- &->[pH<7] MB^{.-} \\
2MB^{.-} &\rightarrow MB + LMB\\
O2 + e_{CB}^- &\rightarrow O2^{.-}}
\end{align}

\begin{align}
\cee{ WS2-WO3 + h\nu &-> e^- + h^+\\
O2 + e^- &-> O2^{.-}\\
OH_{surface}^- + h^+ &-> .OH\\
MB^+ + O2^{.-} + .OH-> \text{degradation phases}
}
\end{align}


\begin{align}
\cee{WO3(V) &-> W(S) + 3/2O2 \qquad T= 1150 \si{\degreeCelsius}\\
        WO3(V) &-> WO2(S) + 1/2O2}
\end{align}

\begin{align}
W + xSiO{_2} &\rightarrow WO{_x} + xSiO{_2},\\
W + SiO{_2}   &\rightarrow WO{_x} + WSi{_2},
\end{align}

\begin{align}
x\ce{M+} + x\ce{e-} +  \ce{$\alpha \hyphen$WO$_{3-y}$}= \ce{$\alpha \hyphen$M$_x$WO$_{3-y}$},
\end{align}

\begin{align}
h\nu +\ce{W^{5+}(A)} +  \ce{W^{6+}(B)} &\rightarrow \ce{W^{5+}(B)} + \ce{W^{6+}(A)} \label{eq:cl_bl1}\\
h\nu +\ce{W^{5+}(A)} +  \ce{W^{4+}(B)} &\rightarrow \ce{W^{5+}(B)} + \ce{W^{4+}(A)} \label{eq:cl_bl2}
\end{align}

\begin{figure}[htb]
\centering
\includegraphics[width=0.8\textwidth]{CVD_and_temp_MoO3.jpg}
\caption[CVD system]{Chemical vapor system and its temperature profile.The length of heating zone is 6 inches. On the horizontal axis of temperature, zero inch is defined at the upstream edge of furnace.The triangular labels $\blacktriangledown$ were measured points at ambient pressure.The nominal substrates temperature were estimated from interpolation data.}
\label{fig:s1}
\end{figure}

\begin{table}[htb]
\centering
\caption{Source of figures }\label{tab:sof}
\begin{tabular}{lcccr}
\toprule
\multicolumn{2}{c}{Flow} \\
\cmidrule(l){4-5}
         & Temperature & Pressure & Ar & \ce{O2}  \\
\midrule
Range      & RT-1100    & 10mTorr-1atm & 0 - 100 & 0-30  \\
Resolution & $\pm1$  & correlated to flow & 1   & 0.1  \\
\bottomrule
\end{tabular}
\end{table}

% Adding notes to tables can be complicated.
\begin{table}
  \centering
  \caption{A table with notes}  \label{tbl:notes}
  \begin{tabular}{ll}
    \toprule
    Header one                            & Header two \\
    \midrule
    Entry one\textsuperscript{\emph{a}}   & Entry two  \\
    Entry three\textsuperscript{\emph{b}} & Entry four \\
    \bottomrule
  \end{tabular}

  \textsuperscript{\emph{a}} Some text;
  \textsuperscript{\emph{b}} Some more text.
\end{table}


%\newgeometry{margin=1in}
\begin{landscape}
\begin{table}[htb]
\centering
\caption{Tungsten Oxides Growth and Application}\label{tab:wox}
{\footnotesize
\begin{tabular}{lp{3.5in}p{2.5in}c}
\toprule
composition  &  methods & highlights &  reference  \\
\midrule
\ce{WO3} & hot W filament (above 1500 \si{\degreeCelsius}) in Ar/\ce{O2} flow  & Cubic phase, PL, resistivity measured & \cite{Thangala2007} \\
\addlinespace[0.5em]
& W filament DC heating in \ce{NH3} or \ce{N2}/\ce{H2} flow  & multi phases, 100mg per batch, stable dispersion in both organic and aqueous solvents & \cite{Chang2007} \\
\bottomrule
\end{tabular}
}
\end{table}
\end{landscape}
%\restoregeometry




%\newgeometry{margin=1in}
%\begin{landscape}
\begin{sidewaystable}
\begin{table}[htb]
\centering \small
\caption{Tungsten Oxides Growth and Application}\label{tab:wox}
\begin{tabular}{lccr}
\toprule
composition  &  methods & highlights &  reference  \\
\midrule
\ce{WO3}     & \ce{Na2WO4.2H2O} mixed with \ce{(NH4)2Fe(SO4)2.6H2O} or%
 \ce{CoCl2.6H2O} followed by hydrothermal heating  & 3 different morphologies and photodegradation  & \cite{Rajagopal2009}  \\
Resolution & $\pm1$  & correlated to flow & 1     \\

\bottomrule
\end{tabular}
\end{table}
\end{sidewaystable}
%\end{landscape}
%\restoregeometry


\section{vocabulary}
well suited to,
seriously pursued. 
One must be just as aware of the instrument's limitations as one is of its advantages
I call attention to the fact that.
arise as a result of, dispersal of Na by electron probe.
$\delta\omega/\epsilon$ 

\subsection{pronunciation}

chamber, vias, valve, figure, energetic, managerial, inert, volatile, chromic,
laminar, palladium, platinum, photovoltaic, acronym, chirality, stoichiometric, cyclic voltammetry, quasi, pseudo, 

