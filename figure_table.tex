% art work global settings
%  origin: 
 all tick on, line color: blue or black, line width: 2; tick : 18pt; 
 layout area(in): left:1, top:0.3; width 10: height 7;
 export: 3in auto
 text size: 22
% SEM 
2.5in by 2in, 300dpi; gray scale


the width of band gap is a measure of chemical bond strength

Fig.~\ref{fig4}(c) illustrated the XRD spectrum of one typical sample.The peaks under circular symbol were identified to be the monoclinic \ce{WO3} phase (ICDD PDF 01-083-0950,\emph{a}=7.30084\AA, \emph{b}=7.53889\AA, \emph{c}=7.6896\AA , $\beta$=90.8920$^\circ$), while the peak under the triangular symbol was indexed to cubic tungsten phase (ICDD PDF 04--16-3405,\emph{a}=3.157\AA).

\si{W/in^2}
\si{\degreeCelsius}

 WO3: dielectrics constants: $\epsilon_\alpha = 6.52,\epsilon_\infty > 50$. \cite{Deb2008}
\begin{figure}[htb]
\centering
\subfloat[tw2]{\label{fig:tw2}\includegraphics[width=5cm]{Tangwei2}}\hspace{0.04\textwidth}
\subfloat[tw3]{\label{fig:tw3}\includegraphics[width=5cm]{Tangwei3}}

\subfloat[tw4]{\label{fig:tw4}\includegraphics[width=10cm]{Tangwei4}}
\caption{snapshot of Late Autumn}
\label{fig:tw}
\end{figure}

\begin{figure}[htb]
\centering
\subcaptionbox{\label{fig:moind}}{\includegraphics{n_MoO3}}%
\subcaptionbox{\label{fig:mocon}}{\includegraphics{Graph20}}
\caption[Refractive indices of \ce{MoO3}]{(a) Refractive indices of \ce{MoO3} Dashed line is n along $a$ axis while solid line is along $c$ axis. (b) optical contrast mapping of 1L \ce{MoO3} on \ce{SiO2}-Si. The x axis is \ce{SiO2} thickness, y axis is wavelength.It is assumed that the incident light is polarized along c axis of \ce{MoO3}.}
\label{fig:mofl}
\end{figure}


\begin{align}
\cee{WO3(V) &-> W(S) + 3/2O2 \qquad T= 1150 \si{\degreeCelsius}\\
        WO3(V) &-> WO2(S) + 1/2O2}
\end{align}

\begin{align}
W + xSiO{_2} &\rightarrow WO{_x} + xSiO{_2},\\
W + SiO{_2}   &\rightarrow WO{_x} + WSi{_2},
\end{align}

\begin{align}
x\ce{M+} + x\ce{e-} +  \ce{$\alpha \hyphen$WO$_{3-y}$}= \ce{$\alpha \hyphen$M$_x$WO$_{3-y}$},
\end{align}

\begin{align}
h\nu +\ce{W^{5+}(A)} +  \ce{W^{6+}(B)} &\rightarrow \ce{W^{5+}(B)} + \ce{W^{6+}(A)} \label{eq:cl_bl1}\\
h\nu +\ce{W^{5+}(A)} +  \ce{W^{4+}(B)} &\rightarrow \ce{W^{5+}(B)} + \ce{W^{4+}(A)} \label{eq:cl_bl2}
\end{align}

\begin{figure}[htb]
\centering
\includegraphics[width=0.8\textwidth]{CVD_and_temp_MoO3.jpg}
\caption[CVD system]{Chemical vapor system and its temperature profile.The length of heating zone is 6 inches. On the horizontal axis of temperature, zero inch is defined at the upstream edge of furnace.The triangular labels $\blacktriangledown$ were measured points at ambient pressure.The nominal substrates temperature were estimated from interpolation data.}
\label{fig:s1}
\end{figure}

\begin{table}[htb]
\centering
\caption{Source of figures }\label{tab:sof}
\begin{tabular}{lcccr}
\toprule
\multicolumn{2}{c}{Flow} \\
\cmidrule(l){4-5}
         & Temperature & Pressure & Ar & \ce{O2}  \\
\midrule
Range      & RT-1100    & 10mTorr-1atm & 0 - 100 & 0-30  \\
Resolution & $\pm1$  & correlated to flow & 1   & 0.1  \\
\bottomrule
\end{tabular}
\end{table}

Adding notes to tables can be complicated.

\begin{table}
  \centering
  \caption{A table with notes}  \label{tbl:notes}
  \begin{tabular}{ll}
    \toprule
    Header one                            & Header two \\
    \midrule
    Entry one\textsuperscript{\emph{a}}   & Entry two  \\
    Entry three\textsuperscript{\emph{b}} & Entry four \\
    \bottomrule
  \end{tabular}

  \textsuperscript{\emph{a}} Some text;
  \textsuperscript{\emph{b}} Some more text.
\end{table}


%\newgeometry{margin=1in}
%\begin{landscape}
\begin{sidewaystable}
\begin{table}[htb]
\centering \small
\caption{Tungsten Oxides Growth and Application}\label{tab:wox}
\begin{tabular}{lccr}
\toprule
composition  &  methods & highlights &  reference  \\
\midrule
\ce{WO3}     & \ce{Na2WO4.2H2O} mixed with \ce{(NH4)2Fe(SO4)2.6H2O} or%
 \ce{CoCl2.6H2O} followed by hydrothermal heating  & 3 different morphologies and photodegradation  & \cite{Rajagopal2009}  \\
Resolution & $\pm1$  & correlated to flow & 1     \\

\bottomrule
\end{tabular}
\end{table}
\end{sidewaystable}
%\end{landscape}
%\restoregeometry



the electrical conductivity is given by $\sigma = n e \mu$, where $n$ is the density of free carriers, $\mu$ is their mobility and $e$ is electronic charge. The mobility is given by $\mu = e\tau/m^*$ with $\tau$ is carrier resistivity relaxation time and $m^*$ is the carrier effective mass.


The degree of \ce{H^+} intercalation, $x$, is determined by integrating cathodic current, and given by
\begin{equation}
x = \frac{Q}{F}\frac{M_\ce{WO3}}{m_f}
\end{equation}
where $m_f$ is the mass of film, $M_\ce{WO3}$ is the molar mass of \ce{WO3*1/3H2O}, and F is Faraday's constant.
The color center density, $c$, is then obtained using formula
\begin{equation}
c= \frac{x M_\ce{WO3}N_A}{\rho}
\end{equation}
where $N_A$ is Avogadro number, $\rho$ is density of film.


In second state, the one with smaller droplet probably would exhibit faster growth rate in $\langle001\rangle$ direction due to its shortest lattice constant, while the other one endures more growth along$\langle010\rangle$ direction. We suspect that the liquid size that induce the tower growth is much larger than that of belts. From the final morphology of as-synthesized sample, we find that the above proposed two growth approaches could change into each other, that is belt growth could be initiated on the top of tower growth and vice visa. We assume this phenomena arise from the interaction between vapor supply, temperature and the shape of liquid catalyst. Or it may just come from the vapor transport of \ce{NaxMoO3}.
It had been shown by actively engineering the shape of catalysis, the growth direction of  InP NW could be switched between [111] and [100].\cite{Wang2013c} The same mechanism might exist in our case as well. For instance, when a belt is sufficient long and it could extend into the low temperature part, where the supply of MoOx vapor is reduced due to the consumption, and the shape of liquid might vary as well. Both will result in different absorption and subsequent diffusion rate along and inside the liquid catalyst. We also observed that belt growth could initiate from tower growth in reduced time growth. Actually this may be one possible mechanism of belts formation.

(The presence of \ce{Na6MoO33} phase is an indirect evidence for the formation of eutectic \ce{Na2MoO4}- \ce{MoO3} binary compounds at growth temperature. The bulk phase diagram may not accurately represent the phase transition occurring in catalyst droplet and solid interface. And it will usually lead to a significant temperature decrease of eutectic point from the bulk value. Nanoparticles do not completely melt and instead act as an active site for reactant absorption and diffusion, leading to a vapor-solid-solid growth mechanism. )

