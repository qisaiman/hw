\documentclass[12pt,letterpaper]{article}
\usepackage{booktabs}
\usepackage{xcolor}
\usepackage{listings}
\lstset{
frame = single,
language = bash,
breaklines = true,
postbreak=\raisebox{0ex}[0ex][0ex]{\ensuremath{\color{red}\hookrightarrow\space}}
}


\title{Git Learning Notes}
\author{tao}
\date{\today}

\begin{document}

\maketitle
\section{New tracking project}
Minimum working sample is:
\begin{lstlisting}
Git init
Git add */*.tex
Git add *.tex
Git add readme
Git commit -m 'comments'
% on a specific file after editing
 git add xx.tex

% to Reset a staged file
Git reset HEAD xxx.tex

ssh-keygen -t rsa -C "qisaiman@gmail.com"

% checkout from github
git add remote hw https://github.com/qisaiman/hw.git
git fetch hw master or develop
git merge hw/master
git push hw master:master
% or
git push hw master:hw/master % use my master branch to update remote

% view current remote
git remote -v
% remove origin remote
git remote rm origin


%create a branch develop and switch to it
git checkout -b develop
{some editing }
git commit -m ''
git checkout master
git merge develop % into current branch
% inspect the branch status
git branch --merged % show merged branch
git branch --no-merge % show unmerged works

% delete a branch
git branch -d develop


\end{lstlisting}

\begin{lstlisting}
% initialization
touch README.md
git init
git add README.md
git commit -m "first commit"
git remote add origin git@github.com:qisaiman/cv.git
git push -u origin master

% view current remote
git remote -v
% remove origin remote
git remote rm origin

% to view current config
git config --list


\end{lstlisting}
Git protocol 9148 port.

\section{Commands}

\begin{table}[htb]
  \caption{Commands}\label{tbl:notes}
  \centering
  \begin{tabular}{lr}
    \toprule
    Command                            & Function \\
    \midrule
    gitk\textsuperscript{\emph{a}}   & Visualization versioning history  \\
    git status\textsuperscript{\emph{b}} & current working flow state \\
    ssh key   & zgcss627\\

    \bottomrule
  \end{tabular}

  \textsuperscript{\emph{a}} Some text;
  \textsuperscript{\emph{b}} Some more text.
\end{table}

\section{lynda git}

\begin{lstlisting}
git config --global % to each user level
git config  % to each project

git config --global core.editor ""
git config --list % view current config
ls -la % list hidden files
git log --since=
git log --until=
git log --grep="tms"

three tress: repository, staging, working
change set into checksum

git diff %
git diff --staged %
git diff --color-words filexxx

git mv file1 renamedfile2 % rename a file

git commit -am 'commit comments' % add and commit

git checkout -- filexxx% stay on current branch and restore previous versions from repository
git reset HEAD filexxx % unstage

git commit --amend -m "" % undo commit

git checkout SHA -- filexxx % checkout previous version

git revert SHA %

git clean % remove untracked files 

navigating commit tree
tree-ish 

parent commit: HEAD^, HEAD~1, HEAD~2,

git ls-tree master or HEAD %list commit tree 

git log --oneline 
git log --author=""
git log --since=
git log --graph --all --decorate --oneline

git show SHA 

git diff HEAD~3..HEAD % -b ignore space changes, 

%use branch to try new ideas, and isolate features
compare branches
git diff --color-words master..develop  

git branch --merged % to see if current branch include previous contents 

rename branches: 

git branch -m develop newname 

git merge --no-ff branch % no fast-forward merge

true merge  %recursive strategy

resolving conflicts: 
abort: git merge --abort
manually: 
merge tool: 

stashing: store temporally 
git stash save ""
git stash list 
git stash show -p stash@{0}
git stash apply or pop stash@{0}% pop, no longer in drawer, apply, make a copy from the drawer

git stash drop stash@{0} 
git stash clear % clear all stashes

% remotes 
git push -u % 


\end{lstlisting}


\end{document} 