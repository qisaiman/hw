\documentclass[12pt,letterpaper]{article}
\usepackage{booktabs}
\usepackage{xcolor}
\usepackage{listings}
\lstset{
frame = single,
language = bash,
breaklines = true,
postbreak=\raisebox{0ex}[0ex][0ex]{\ensuremath{\color{red}\hookrightarrow\space}}
}


\title{Git Learning Notes}
\author{tao}
\date{\today}

\begin{document}

\maketitle
\section{New tracking project}
Minimum working sample is:
\begin{lstlisting}
Git init
Git add */*.tex
Git add *.tex
Git add readme
Git commit -m 'comments'
% on a specific file after editing
 git add xx.tex

% to Reset a staged file
Git reset HEAD xxx.tex

ssh-keygen -t rsa -C "qisaiman@gmail.com"

% checkout from github 
git add remote hw https://github.com/qisaiman/hw.git
git fetch hw master or develop 
git merge hw/master
git push hw master:master
% or 
git push hw master:hw/master % use my master branch to update remote 

% view current remote
git remote -v
% remove origin remote
git remote rm origin 


%create a branch develop and switch to it
git checkout -b develop
{some editing }
git commit -m ''
git checkout master 
git merge develop % into current branch 
% inspect the branch status 
git branch --merged % show merged branch 
git branch --no-merge % show unmerged works

% delete a branch
git branch -d develop 


\end{lstlisting}

\begin{lstlisting}
% initialization
touch README.md
git init
git add README.md
git commit -m "first commit"
git remote add origin git@github.com:qisaiman/cv.git
git push -u origin master

% view current remote 
git remote -v 
% remove origin remote 
git remote rm origin 

% to view current config
git config --list 


\end{lstlisting}
Git protocol 9148 port. 

\section{Commands}

\begin{table}[htb]
  \caption{Commands}\label{tbl:notes}
  \centering
  \begin{tabular}{lr}
    \toprule
    Command                            & Function \\
    \midrule
    gitk\textsuperscript{\emph{a}}   & Visualization versioning history  \\
    git status\textsuperscript{\emph{b}} & current working flow state \\
    ssh key   & zgcss627\\
    
    \bottomrule
  \end{tabular}

  \textsuperscript{\emph{a}} Some text;
  \textsuperscript{\emph{b}} Some more text.
\end{table}


\end{document} 