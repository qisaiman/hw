% git version control added 012614
\chapter{New Vapor-Solid-Solid Growth Modes of Molybdenum Oxide}

\section{Introduction}

In this work, a new growth mechanism of molybdenum oxide (\ce{MoO3}) 1D structures was discovered, highlighted by \ce{MoO3} long nanobelts and micro-scale towers. The samples were synthesized using a group of alkaline metal based catalysts, including \ce{NaOH}, \ce{KI}, and \ce{Na2CO3}. In contrast to the sole axial growth found in the conventional catalyst-assisted growth, two different growth modes were observed for the \ce{MoO3} 1D growth here: transverse growth and axial growth. In the transverse mode, the 1D structures grow perpendicularly to the catalyst-deposition interface with catalyst particles on the side surfaces; whereas in the axial mode, the crystal grows along the catalyst-deposition axis. The growth modes were explained by a modified vapor-solid-solid (VSS) mechanism, and factors that affect the growth were explored in details. 

The remaining sections are organized as following: a brief review on previous studies of \ce{MoO3} is given first, followed is the synthesis method in this dissertation. Emphasis will be placed on morphological and crystal structure characterization of 1D \ce{MoO3} nanostructures, and the verification of proposed VSS mechanism. This chapter is concluded with a summary of VSS growth and some preliminary results of morphology controlled \ce{MoO3} nanostructures. 

\subsection{Properties and Applications of Molybdenum Oxide}

Molybdenum oxide (\ce{MoO3}) crystallizes in three phases, orthorhombic $\alpha$-\ce{MoO3}, monoclinic $\beta$-\ce{MoO3} and the metastable hexagonal h-\ce{MoO3}.\citep{Deb1968,Fibers2007} $\alpha$-\ce{MoO3} phase (hereafter \ce{MoO3}) exhibits anisotropic structure with strong bonding along [001] and [100] direction while van der Waals interaction along [010] direction.\cite{He2003} Due to this unique structure, \ce{MoO3} was found to have several important properties and a wide range of technological applications, such as electrochromism and photochromism,\cite{Yao1992} lubricants,\cite{Sheehan1996} photocatalysts,\cite{Chen2010} and gas sensor, including \ce{CO},\cite{Comini2005} \ce{NO2},\cite{Taurino2006} \ce{H2}\cite{Sha2009} and ethanol.\cite{Choopun} Moreover, other features arise when nanoscale \ce{MoO3} is specifically prepared, e.g., field emission.\citep{Li2002d,Zhou2003b}  As a layered hosting material, \ce{MoO3} can be further modified by intercalating with alkaline ions\citep{Spahr1995,Li2006b,Hu2011} and even divalent ion.\cite{Sian2005} This structural richness have enabled improved performance in Li-ion\cite{Mai2007} and sodium-ion battery.\cite{Hariharan2013} In combination with \ce{TiO2} forming a core-shell structure, these nanoparitcles are reported to lower the photon absorption energy of \ce{TiO2}.\cite{Elder2000} When combined with Ag as layered structure, transparent conducting behavior was observed.\cite{Nguyen2012} In addition, \ce{MoO3} is a good precursor for preparing other useful materials, such as \ce{MoS2} fullerene\cite{Li2003c} and few layer \ce{MoS2}.\cite{Lin2012} It will also be an important member of the van der Waals heterostructures.\cite{Geim2013}


The coordination number of Mo in \ce{MoO3} is six, so usually \ce{MoO6} octahedra are considered as the building blocks. As shown in Fig.~\ref{fig:mo3model}, the layered structure consisting of zig-zag rows of edge-sharing \ce{MoO6} octahedra, while the rows are mutually connected by corners.

\begin{figure}[htb]
\centering
\includegraphics[width=0.7\textwidth]{MoO3_model}
\caption[\ce{MoO3} crystal model]{\ce{MoO3} crystal model.}
\label{fig:mo3model}
\end{figure}

While considering the fact that four of the six surrounding O atom are at distances from 1.67 to 1.95 \AA, while the remaining two are as far as 2.25 and 2.33 \AA, \ce{MoO3} could also be considered as built up of chains of \ce{MoO4} tetrahedra connected by the sharing of two oxygen corners with two neighbouring tetrahedra in $c$ axis. The infinite chains of \ce{MoO4} tetrahedra from half-layers in the $ac$ plane. Two half-layers, which are stapled along $b$ axis, build up one \ce{MoO3} layer.\cite{Itoh2001a} This view stresses that the \ce{MoO6} octahedra are rather distorted.
%A brief summary of crystallography, band structure and dielectric function is listed in Appendix.

It is known that a series of molybdate compound forms when alkali metal ions are incorporated into the \ce{MoO3} lattice, and the structures are summarized in Table~\ref{tab:naxmow}. Due to the weak cation-oxygen bonding, alkali metal cations only introduce small perturbations into the energies of \ce{Mo-O} matrix in comparison to cations of other elements. And no mixing of vibrations of the cationic sublattices with that of \ce{Mo-O} lattices is anticipated. So the structural features of \ce{Mo-O} polyhedra are dominating factors affecting the vibration frequencies and thermodynamic values of the molybdates.\cite{Fomichev1992}

\begin{table}[htb]
\centering
\caption{Crystal structures of alkali metal molybdates and tungstates}\label{tab:naxmow}
\begin{tabular}{llr}
\toprule
formula & structure  &  \\
\midrule
\ce{A2O}:\ce{MO3}\textsuperscript{\emph{a}} & isolated tetrahedral \ce{MO4} anions& \\
\ce{A2O}:\ce{2MO3} & chain-type anions of \ce{MO4} and \ce{MO6} & \\
\ce{A2O}:\ce{3MO3} & chain-type anions of \ce{MO5} and \ce{MO6} & \\
\ce{A2O}:\ce{4MO3} & chain-type anions of \ce{MO6} & \\
\bottomrule

\textsuperscript{\emph{a}} A = Li, Na, K, Rb, Cs; M = Mo, W;
\end{tabular}
\end{table}

Raman band assignment for \ce{A2MO4} is well grounded. Alkali metal cations with octahedral coordination occur below 230 cm$^{-1}$. For \ce{A2M_nO_{3n+1}}, exact assignment is not available yet due to the complicated \ce{MO6} octahedra. Molybdenum bronze also exhibits intriguing features, for instance, \gls{cdw} states\footnote{In \gls{cdw} states, conductivity is non-Ohmic above a threshold electric field.} was found in blue bronze \ce{K_{0.3}MoO3}, which can be prepared by electrolytic reduction of \ce{K2MoO4} and \ce{MoO3} melt.\cite{Dumas1983} \ce{K_{0.3}MoO3} stay as monoclinic phase at room temperature, with lattice parameter $a=18.249$ \AA, $b=7.561$ \AA, $c=9.856$ \AA, and $\beta=117.54^{\circ}$, and exhibits a semiconductor-to-metal transition at 180 K. Electrical transport measurements yield an highly anisotropic ratio of DC conductivities of 1:10:1000. Along this high conducting axis, metallic reflection behavior is confirmed by optical measurements. Therefore, this blue bronze $\ce{K_{0.3}MoO3}$ is known as quasi-1D metal.\cite{Sing1999}

\subsection{Synthesis Review of Molybdenum Oxide}
Researchers have been exploring a variety of methods to synthesized different \ce{MoO3} nanostructures. Only a brief introduction will be given here, since these efforts has been well documented in several review articles.\cite{He2003} Most of these methods can be categorized into two groups: solution-based hydrothermal procedures \citep{Li2002b,Xia2006,Li2006a,Camacho-Bragado2006} and chemical vapor transport and deposition approaches.\citep{Zeng1998,Li2002c,Li2002d,Zhou2003b,Fibers2007,Yan2009}. Both methods have its own merits. Hydrothermal process usually need lower temperature(\textless 300 \si{\degreeCelsius}), but the duration is usually long and requires several post-growth processing steps; whereas vapor deposition demands relatively high temperature (\textgreater 500 \si{\degreeCelsius}) yet with a shorter time. Both methods are scalable for industrial applications. However the hydrothermal treatment seems to allow more  nanostructures control than does the vapor deposition method. Nanobelts,\cite{Li2002b} helical nanosheets, nanoflowers, prims-like rods\cite{Li2006a} and nanoribbons\cite{Camacho-Bragado2006} were obtained by the former one while nanoflakes,\cite{Chen2009} nanobelts\cite{Hu2009} and nanowires\citep{Zhou2003b,Chen2011b} dominated the product morphology for the latter one. The assistance of Au catalyst in vapor deposition only altered the orientation\cite{Yan2009} or served as preferred nucleation sites\cite{Cai2011} without producing new \ce{MoO3} structures. \citeauthor{Chithambararaj2013} prepared hexagonal \ce{MoO3} nanocrystal via hydrothermal method and demonstrated the photodegradation of methylene blue (MB) under visible light.\cite{Chithambararaj2013} The efficiency dependence on catalysis/dye ratio, light intensity and temperature was studied. h-\ce{MoO3} was mostly synthesized using solution methods, where \ce{NH4+} and \ce{OH-} were possible structure directing and stable agents. The band gap of h-\ce{MoO3} estimated from diffusion reflection spectrum is $2.8\sim3.0$ eV. Hexagonal phase of \ce{MoO3} is readily identified by the XRD pattern (strong peak at $2\theta=20^{\circ}$). It is worth noting that in spite of these numerous growth of \ce{MoO3}, reports on catalytic growth are still scarce. 

\section{Experimental}\label{sec:grow}

The reactants used in this study were listed in Table~\ref{tb:mosource}. All reactants were used as received without further processing.

\begin{table}[htb]
\centering
\caption{Reactants list}\label{tb:mosource}
\begin{tabular}{lcccr}
\toprule
Material & Stock No & LOT &Purity & Vendor\\
\midrule
\ce{NaOH}        & S318-500 & 070241 & 99.8\% & Fisher Scientific \\
\ce{NaI}        & 11665 & K11W054 & 99.9\% &  Alfa Aesar \\
\ce{KI}        & 42857 & H06Z051 & 99.9\% &  Alfa Aesar \\
\ce{Na2CO3}        & 33377 & 114X012 & 99.95\% &  Alfa Aesar \\
\ce{Molybdenum}        & 00932 & I07S024 & 99.9\% &  Alfa Aesar\\
\bottomrule
\end{tabular}
\end{table}

Silicon substrates was first cleaned according to the procedure in Sec.~\ref{ch2sub} on page~\pageref{ch2sub}. To apply the catalysts, 54 $\mu$L 10 mM \ce{NaOH} (or \ce{KI}, \ce{Na2CO3}) was drop-cast onto clean substrates and then naturally left dry in a convection hood. Other alkaline ions-containing substrates, such as glass (Fisher Scientific, microscope slide, 12-549) and indium tin oxide (ITO) coated glass (Delta Technologies, 25\si{\ohm}), were cleaned by the same routine except the absence of plasma cleaning and subsequent aqueous solution dipping. Mica was cleaved right before growth without other treatment.

In a typical synthesis run (Fig.~\ref{fig:mooxgrowth}), about 2 g molybdenum powders were loaded into the uniform heating zone and the sealed chamber was first pumped down to 10 mTorr. Then oxygen flow varying between 0.1 sccm to 10 sccm (standard cubic centimeter per minute) was admitted from upstream inlet. With 10 sccm UHP Ar as carrier gas, the overall pressure reached about 200 mTorr. The heating temperature was ramped up to 800 \si{\degreeCelsius} in 30 minutes and lasted for 15 to 120 minutes. Then the heating power was turned off and the chamber was allowed to naturally cool down to room temperature. The substrate was placed in downstream location where the temperature was about 350 to 650 \si{\degreeCelsius} according to open air temperature profile. Non-catalytic growth shares the same procedure except the NaOH treatment. 


\section{Results and Discussion on Molybdenum Oxide}\label{sec:result}

In this section, the growth of \ce{MoO3} on bare Si substrates was presented first, followed by catalytic growth, both of which were characterized by SEM, XRD, TEM and EDX. Then the catalytic growth mechanism was investigated using \ce{NaOH} treated Si substrate.
\subsection{Non-catalyst Growth}\label{sec:nonsi}

The oxidization of Mo starts at 500 \si{\degreeCelsius}, and \ce{MoO3} vapor pressure begins to increase rapidly above 700 \si{\degreeCelsius}.\cite{Margrave1967} The melting point of \ce{MoO3} is reported as 795 \si{\degreeCelsius}. Based on these data, the growth was designed with parameters listed in Table~\ref{tab:mooxsi}. It is worth mentioning that these parameters are selected and optimized from a wide range of combinations. When the influence of one parameter is studied, the others are fixed, following the standard practice in controlled experiments.
%growth parameter
\begin{table}[htb]
\centering
\caption{Growth conditions of \ce{MoO3} on Si}\label{tab:mooxsi}
\begin{tabular}{lcccr}
\toprule
&&&\multicolumn{2}{c}{Flow (sccm)} \\
\cmidrule(l){4-5}
 & Temperature $T_h$ (\si{\degreeCelsius}) & Pressure (mTorr) & Ar & \ce{O2}  \\
\midrule
Typical values  & 800    & 200 & 10 & 10  \\
\bottomrule
\end{tabular}
\end{table}

The growth layout is schematically shown in Fig.~\ref{fig:mooxgrowth}. Notice that on the horizontal axis of temperature, zero inch is defined at the upstream edge of furnace. To facilitate discussion and provide a tight context, the author define the position of substrate as the coordinate of its upper-stream edge along this temperature axis (i.e. the substrate location is 7 in in Fig.~\ref{fig:mooxgrowth}). The usual length of Si substrate is 1 in. When positioned at 7 in, it is then located in a temperature zone of $650 \sim 350$ \si{\degreeCelsius}.
% cvd layout
\begin{figure}[htb]
\centering
\includegraphics[width=0.7\textwidth]{CVD_and_temp_MoO3.jpg}
\caption[Growth setup of \ce{MoO3}]{Chemical vapor system and its temperature profile. The triangular labels $\blacktriangledown$ were measured points at ambient environment. The nominal substrates temperature were estimated from interpolation data.}
\label{fig:mooxgrowth}
\end{figure}

From previous works in our group,\cite{predeep2011} the author found that with $T_h = 600$ \si{\degreeCelsius} and substrate placed at 6.5 in, microflakes grew at the high temperature edge of Si, suggesting the nucleation already started.\footnote{in 020510 batch} When flow of oxygen was gradually reduced, diminished size of micro-flake was observed at the same location of substrates. This results indicated the rate-determining step is diffusion when oxygen flow was in the range of 2 sccm to 0.5 sccm.\footnote{in (31210-31510) batches} It was also found the deposition amount on a particular location of the substrate is not linearly proportional to the growth time, \emph{i.e.}, the growth at upper stream edge first increase then decrease. This is not surprising when considering the 3D temperature field in chamber requires some time to become synchronized as depicted in Fig.~\ref{fig:mooxgrowth}. Using conditions in Table~\ref{tab:mooxsi} and 2 h growth, this study observed the typical micro-flakes morphologies as shown in Fig.~\ref{fig:mosisem}. Most flakes exhibit rectangular shape, with average thickness of one micron, standard deviation $\sigma_D=0.34 \mu$m.
% sem
\begin{figure}[htb]
\centering
\subfloat[]{\label{fig:mosem1}\includegraphics[width=0.45\textwidth]{mosemsi_a}}\hspace{0.04\textwidth}
\subfloat[]{\label{fig:mosem2}\includegraphics[width=0.45\textwidth]{mosemsi_b}}
\caption[Representative morphologies of \ce{MoO3} on Si]{(a) Low magnification and (b) high magnification SEM images of representative depositions of \ce{MoO3} on Si for non-catalytic growth.}
\label{fig:mosisem}
\end{figure}

As shown in Fig.~\ref{fig:mooxch}, the crystal structure and phase of the as-synthesized specimen were examined by XRD and Raman. The XRD pattern (Fig.~\ref{fig:moxrd}) is readily indexed to the orthorhombic phase of \ce{MoO3} (ICDD PDF 05-0508, \emph{a}=3.9628 \AA, \emph{b}=13.855 \AA, \emph{c}=3.6964 \AA). The space group is $D_{2h}^{16}(Pbnm)$. This crystal structure is indeed a unique example among transition metal oxides, representing a transitional stage between tetrahedra and octahedral coordination.\cite{Itoh2001a} The strongest peak index is (110), suggesting the orientation of \ce{MoO3} is not parallel to the substrate, which is consistent with the morphology displayed in Fig.~\ref{fig:mosem1}. In contrast, a different XRD peak pattern is observed in Sec.~\ref{sec:nasi}, where the catalytic growth occurs.
% xrd raman
\begin{figure}[htb]
\centering
\subfloat[]{\label{fig:moxrd}\includegraphics[width=0.45\textwidth]{xrd_moo3}}\hspace{0.04\textwidth}
\subfloat[]{\label{fig:moram}\includegraphics[width=0.45\textwidth]{raman_moo3}}
\caption[Crystalline phase characterization of \ce{MoO3} on Si]{(a) XRD pattern and (b) Raman spectrum of typical \ce{MoO3} on Si, $\lambda_{ex} = 532$ nm.}
\label{fig:mooxch}
\end{figure}
The Raman spectrum of the as-synthesized specimen also closely matches \ce{MoO3} features in previous studies.\cite{Dixit1986,Silveira2012} During m-Raman measurement, the laser spot was carefully focused onto the plates and several inspections on different positions were observed to ensure the reproductivity of spectra data. As shown in Fig.~\ref{fig:moram}, 14 distinct bands were well resolved. The 284 \si{cm^{-1}} peak represents the wagging mode for double bond \ce{O=Mo=O}. The 337 and 380  \si{cm^{-1}} peaks are assigned to \ce{O-Mo-O} bending and scissoring modes. The 199 \si{cm^{-1}} peak and two other weaker peaks at 218 and 247 \si{cm^{-1}} represent \ce{O=Mo=O} $B_{2g}$ twist, $A_g$ chain mode and \ce{O=Mo=O} $B_{3g}$ twist mode, respectively. The 667 \si{cm^{-1}} peak is assigned to triply coordinated oxygen stretching model resulting from edge-shared oxygen in common to three octahedral. The 819 \si{cm^{-1}} peak is from doubly coordinated oxygen stretching mode arising from corner-shared oxygen between two octahedral. The 996 \si{cm^{-1}} peak comes from unshared oxygen stretching mode.\cite{Siciliano2009} The assignment is summarized in Table~\ref{tab:moram}.
% raman assign
\begin{table}[htb]
\centering
\caption{Experimental Raman peaks assignment of \ce{MoO3} on Si.\cite{Eda1992,Siciliano2009}}\label{tab:moram}
\begin{tabular}{llcll}
\toprule
this work(\si{cm^{-1}}) & Sym.       &          & Assignment &   \\
\midrule
117      & $B_{2g}$    &           & $T_c$  & RCM  \\
129      & $B_{3g}$    &           & $T_c$  & RCM  \\
158      & $A_g/B_{1g}$&           & $T_b$  & RCM  \\
199      & $B_{2g}$    & $\tau$    & \ce{O=Mo=O}  & twist  \\
218      & $A_g$       &           & $R_c$     & RCM  \\
247      & $B_{3g}$    & $\tau$    & \ce{O=Mo=O}  & twist  \\
284      & $B_{2g}$    &           & \ce{O=Mo=O}  & wag  \\
292      & $B_{3g}$     & $\delta$ & \ce{O=Mo=O}  & wag  \\
337      & $A_g,B_{1g}$ & $\delta$ & \ce{O-Mo-O} & bend  \\
380      & $B_{1g}$     & $\delta$ & \ce{O-Mo-O}  & scissor  \\
474      & $A_g$        & $\nu_{as}$ & \ce{O-Mo-O}  & stretch,bend  \\
667      & $B_{2g},B_{3g}$ & $\nu_{as}$  & \ce{O-Mo-O}  & stretch  \\
819      & $A_g$        & $\nu_{as}$  & \ce{O=Mo}  & stretch  \\
996      & $A_g$         & $\nu_{as}$  & \ce{O=Mo}  & stretch  \\
\bottomrule
\end{tabular}
\end{table}



\subsection{Catalyst-assisted Growth}\label{sec:naohsi}

All growth time of samples prepared in Sec.~\ref{sec:naohsi} were 2 h, the same as in Sec.~\ref{sec:nonsi}. The only difference in growth conditions from Sec.~\ref{sec:nonsi} is the Si substrate treatment. When NaOH is applied onto Si, the deposition morphologies change significantly. Fig.~\ref{fig:ch4sem2by3} shows a comparison of the morphology difference between the non-catalyst growth (Fig.~\ref{fig:ch4sem2by3}a) and the catalyst-assisted growth (Fig.~\ref{fig:ch4sem2by3}b-f) revealed by SEM. The non-catalyst growth morphology has been discussed from Fig.~\ref{fig:mosisem} on page~\pageref{fig:mosisem}. Several different structures emerge when catalyst is applied, such as nanobelts in Fig.~\ref{fig:ch4sem2by3}d with length up to hundreds of microns, and microtowers in Fig.~\ref{fig:ch4sem2by3}f with diameter around 10 $\mu$m. 

% sem
\begin{figure}[htb]
\centering
\includegraphics[width=0.8\textwidth]{MoO3_SEM_a.jpg}
\caption[SEM imaging on \ce{MoO3} catalytic growth]{Effect of alkaline metal-based catalysts on the morphology of \ce{MoO3} depositions. (a) SEM image of rectangular nanoplates grown without catalysts. (b-f) SEM images of different morphologies grown with NaOH catalysts: (b) nanobelts grown on top of dense array of nanoplates,(c) close-up view of the forked nanoplates from an area indicated by the square in Fig.~\ref{fig:ch4sem2by3}, (d) long nanobelts, (e) side-view of ultra-long microbelts, and (f) side-view of microtowers. Insets show detailed features of different \ce{MoO3} structures.}
\label{fig:ch4sem2by3}
\end{figure}

The crystal structures of the as-synthesized specimens were characterized with X-ray diffraction (XRD) for the non-catalyst growth (Fig.~\ref{fig:ch4tem3by3}a) and catalyst-assisted growth (Fig.~\ref{fig:ch4tem3by3}b). The calculated lattice parameters and standard database value (ICDD PDF 05-0508) matches favorably, as listed in Table~\ref{tab:ch4xrd}. No other phases were found from XRD spectra for both specimens. This result indicates that if there were any catalyst phases present in the catalyst-assisted specimen, they must be of extremely low amount as compared to that of the \ce{MoO3} phase, presumably below the detection limit. 
% xrd 
\begin{table}[htb]
\centering
\caption{Lattice information of \ce{MoO3}}\label{tab:ch4xrd}
\begin{tabular}{lccc}
\toprule
           & $a$ (\AA) & $b$ (\AA) & $c$ (\AA)   \\
\midrule
Non-catalytic growth  & 3.963    & 13.882 & 3.701  \\
Catalytic growth  & 3.967    & 13.865 & 3.701  \\
PDF 05-0508       & 3.9628    & 13.855 & 3.6964  \\
\bottomrule
\end{tabular}
\end{table}

The XRD pattern of catalyst-assisted growth shown in Fig.~\ref{fig:ch4tem3by3}b is dominated by the (0$k$0) family peaks, showing a different preferred orientation from the non-catalytic growth. The reason will be discussed in Sec.~\ref{sec:nasi}. 

% tem
\begin{figure}[htb]
\centering
\includegraphics[width=0.8\textwidth]{MoO3_TEM_XRD.jpg}
\caption[XRD and TEM on \ce{MoO3} catalytic growth]{XRD spectra of (a) a specimen grown without catalysts and (b) a specimen grown with NaOH catalysts. (c) Low-magnification TEM image and (d) HRTEM image of nanoplate grown without catalyst. Low-magnification TEM images of (e) forked nanoplate, (f) nanobelt, and (h) microtower, and (g) HRTEM image of nanobelt. Insets show SAED patterns of different \ce{MoO3} 1D structures}
\label{fig:ch4tem3by3}
\end{figure}

Fig.~\ref{fig:ch4tem3by3}c shows low magnification TEM image of a rectangular nanoplate grown without catalysts and the SAED pattern. The nanoplate was exfoliated by ultra-sonication in acetone solution to obtain high-resolution TEM (HRTEM) image, hence the nanoplate was broken into fragments. The SAED pattern in [010] axis and the HRTEM image (Fig.~\ref{fig:ch4tem3by3}d) show the nanoplate has a (010) top surface and two orthogonal edges along [100] and [001], respectively. For the catalyst-assisted growth, Fig.~\ref{fig:ch4tem3by3}e and Fig.~\ref{fig:ch4tem3by3}f-g confirm both the forked nanoplate and the nanobelt have a (010) top surface and a growth direction of [001], while Fig.~\ref{fig:ch4tem3by3}h reveals the microtower grow with (010) layers stacking along a [010] direction. Lattice measurements in Fig.~\ref{fig:ch4tem3by3}d and g indicate a planar distance of 0.38 nm for (100) planes and 0.36 nm for (001) planes. These values are slightly smaller than the standard data for $\alpha$-\ce{MoO3} and the results from XRD measurements. This deviation in lattice parameters may arise from the error in TEM measurements and the lattice distortion from the electron beam irradiation.\cite{Wang2004b} These results indicate that all the as-synthesized 1D structures have a layered structure. Raman measurement on the catalytic growth sample revealed similar \ce{MoO3} bands, as already displayed in Fig.~\ref{fig:mooxch} on page~\pageref{fig:mooxch}. 

\subsection{Growth Mechanism Study}\label{sec:nasi}

To provide a proper context, \gls{vls} process is brief introduced. \gls{vls} is first proposed by Wagner,\cite{Wagner1964} and further developed by \citeauthor{Givargizov1975}.\cite{Givargizov1975} This method has become an important strategy for synthesizing 1D nanostructures.\cite{Lieber1998} Conventionally, a liquid eutectic droplet is formed by catalyst itself or by alloying with the growth material, acting as a trap of growth species. The growth is initialized by supersaturation of the liquid alloy and subsequent precipitation at the solid-liquid interface. The choice of catalyst usually is among the several noble metals since they are physically active but chemically stable or inert in most growth scenarios. The growth materials span a wide range including group IV,\cite{Hochbaum2005} group III-V,\citep{Dalacu2013, Xiao2012, Dubrovskii2011} group II-VI,\cite{Hao2006} and some metal oxides, such as \ce{ZnO},\citep{Huang2001a,Ramgir2010} \ce{MgO},\citep{HEUER1967, Nagashima2007} \ce{SiO2},\cite{Pan2002} and \ce{TiO2},\cite{Zhuge2012} to name a few. The catalysts can be Au, Pd, Pt, Ni,\cite{Xiao2012} Ti, Ga,\cite{Pan2002} and even KI.

Understanding the interaction between liquid droplet and the solid interface allows for a rich engineering space to fine tune the geometry and structures of as-grown nanostructures. For instances, the diameter of Si nanowires (NWs) can be controlled by laser ablated catalyst\cite{Morales1998} or well-defined Au nanoparticles.\cite{Cui2001b} Under proper conditions, axial modulation can produce nanowire superlattices from group IV and III-V materials.\citep{Gudiksen2002,Bjork2002}  Radial composition modulation have also been demonstrated by selectively suppressing VLS process, providing a robust routine for homogeneous or heterogeneous core-shell structures.\cite{Lauhon2002a}  

Previous study in our group has found that \ce{MoO3} deposition on ITO glass exhibits new morphologies that are not observed on Si substrates, yet the mechanism is not well investigated.\cite{predeep2011} Based on the \gls{vls} process, it is known that a eutectic state is the key to promote preferred growth direction. A list of composition difference between glass and Si substrate shows the extra elements of Na, K, Ca, \emph{etc}. Since the concentration of Na is highest among these extra elements, it is most likely to induce the new morphology growth. A literature search indeed shows that there exists a Na-Mo-O phase diagram as shown in Fig.~\ref{fig:ch4pd}.\cite{Hoermann1929} Notice the phases of the Na-Mo-O system can be written as \ce{Na2O.nMoO3}, where n = 1, 2, 3, and 4. 

%phase diagram
\begin{figure}[htb]
\centering
\includegraphics[width=0.6\textwidth]{moo3_pdsmall}
\caption[Phase diagram of Na-Mo-O system]{Phase diagram of \ce{Na2MoO4} and \ce{MoO3} reproduced from Ref.~\cite{Hoermann1929}}
\label{fig:ch4pd}
\end{figure}

Based on this phase diagram and the growth conditions adopted in this study, it is predicted that the overall reactions occur in the following sequences. First, Mo powder is oxidized and \ce{MoO3} vapor is produced and transported to downstream by Ar carrier gas. Then NaOH reacts with incoming \ce{MoO3} according to Formula~\ref{eq:pd}. 
\begin{subequations}\label{eq:pd}
\begin{align}
\cee{NaOH(l) + MoO3(g) &\rightarrow Na2MoO4(l) + H2O(g)\\
Na2MoO4 + nMoO3   &\rightarrow Na2O.(n + 1)MoO3}
\end{align}
\end{subequations}
The melting point of \ce{Na2MoO4} is 687 \si{\degreeCelsius}. Based on the open air temperature profile (Fig.~\ref{fig:mooxgrowth}), the substrate is located in temperature zone between 650 and 350 \si{\degreeCelsius}. The actual temperature distribution during growth is different from ambient measurement due to the pressure change and thermal conduction along the Si substrate. It is difficult to acquire the accurate values along the substrate; however, by considering the flow rate, the actual temperature will be higher than predicted by open air measurement.\cite{Subannajui2010} The melting of \ce{Na2MoO4} is highly possible when taking into account of its size as well.\cite{Bruggemann1997} The continuous supply of \ce{MoO3} growth species will push the phase evolution towards high \ce{MoO3} molar ratio end, with $n$ in \ce{Na2O.nMoO3} increasing from 1 to 4. Each site of \ce{Na2MoO4} will continue absorb incoming \ce{MoO3}, forming a swelling droplet. \ce{MoO3} will precipitate from these seeds when a critical supersaturation concentration is reached, which is about 75\% according to Fig.~\ref{fig:ch4pd}.\footnote{This ratio considerably exceeds that of the usual metal catalyst scenario. This high solubility accounts for several observations in current experiments: the dimensions of individual deposit is much larger than that of initial droplet, the concentration of sodium in final products is extremely low.}    Based on this aforementioned analysis, the author propose a hypothesis:
%theorem environment
\begin{hypothesis}\label{hypo1}
The reaction on NaOH-Si substrate follow the phase diagram, and \ce{Na2Mo4O13} serves as the catalyst promoting various 1D \ce{MoO3} growth. 
\end{hypothesis}

In a typical 2 h growth, the \ce{MoO3} weight increase is about 20 mg, which is about $1.4\times 10^{-4}$ mol. This amount far exceeds the applied NaOH of $5\times 10^{-7}$ mol.\footnote{Sec.~\ref{sec:grow} on page.~\pageref{sec:grow}} To capture the early stage growth, the supply of \ce{MoO3} must be reduced in a controlled way. This can be accomplished by tuning two factors: oxygen flow rate and growth time. In this thesis, two series of controlled growth were used to probe the evolution of \ce{MoO3} deposition on NaOH-Si substrate. The first series regulated \ce{O2} flow at 0.1, 1, 3 and 10 sccm with fixed growth time of 15 min. The second series controls growth time at 15, 30, 60, and 90 min with fixed \ce{O2} flow of 10 sccm. Two series share one common growth of 10 sccm \ce{O2} for 15 min. 

The oxygen regulated growths were characterized using SEM, EDX and Raman, as shown in Fig.~\ref{fig:ch4oxy}. When \ce{O2} flow $\leq 1$ sccm, the deposition show micron size plate growth (Fig.~\ref{fig:ch4oxy}a and d). As oxygen flow becomes higher, the deposition amount increases dramatically, with a morphology of 1D forked plate decorated by triangular shape particles.
% sem
\begin{figure}[htb]
\centering
\includegraphics[width=0.8\textwidth]{MoO3_EDX_Raman}
\caption[Growth evolution of \ce{MoO3}: first stage]{Evolution of the early stage for catalyst-assisted \ce{MoO3} growth: SEM images, micro-Raman spectra, and EDX spectra of the specimens grown for 15 min with different \ce{O2} flows of (a-c) 0.1 sccm, (d-f) 1 sccm, (g-i) 3 sccm, and (j-l) 10 sccm. }
\label{fig:ch4oxy}
\end{figure}

Micro-Raman scattering was used to probe the phase change during the oxygen regulated growth. The Raman spectra of sodium molybdates is featured with multiple strong peaks located between 800 and 1000 \si{cm^{-1}}, which we refer as high-frequency region hereafter. \ce{MoO3} phase shows two peaks in this region with the strongest one at about 820 \si{cm^{-1}}; \ce{Na2Mo2O7} phase exhibits four peaks in this region with the strongest one moving to 937 \si{cm^{-1}}, whereas \ce{Na2Mo4O13} phase shows even more peaks in high-frequency region, with a doublet at 962 and 970 \si{cm^{-1}}, respectively.\cite{Schofield2005,Saraiva2011} As predicted by the phase diagram (Fig.~\ref{fig:ch4pd}), Raman spectra identified \ce{Na2Mo2O7} phase in the 0.1 sccm growth (Fig.~\ref{fig:ch4oxy}b), and \ce{Na2Mo4O13} phase on the particles in the rest three growths (Fig.~\ref{fig:ch4oxy}e, h, and k). This observation supports the previous hypothesis about the reaction sequence on NaOH-Si substrate (Formula~\ref{eq:pd}). Element analysis performed on the Raman sites also confirm the presence of Na, Mo and O in particles growth, and only Mo and O in the plate growth (Fig.~\ref{fig:ch4oxy}c, f, i, and l). It is worth noting that the absence of Na signal in the forked \ce{MoO3} plate indicates the doping level would be lower than the detection limit of 0.01 wt.\% if the Na doping ever occurs. It should be pointed out the existence of \ce{Na2Mo3O10} phase is in doubt so far, and no Raman fingerprint is reported.\cite{Fomichev1992}  

The morphologies of second series experiment at constant \ce{O2} flow and varied growth time were shown in Fig.~\ref{fig:ch4ev}. The 15 min and 30 min growth show that dense arrays of \ce{MoO3} forked nanoplates with a length of hundreds of microns covering the substrate areas at different temperatures. With the increase of the growth time, these nanoplates keep growing and become larger, longer, and denser. Long 1D \ce{MoO3} structures start to grow on top of the nanoplate arrays. For 60 min growth, long nanobelts appear on top of the nanoplates. For the growth time of 90 min, the nanobelts grow longer, and some grow into microbelts. Microtowers start to grow as well.

\begin{figure}[htb]
\centering
\includegraphics[width=0.8\textwidth]{mooe_sem_ev}
\caption[Growth evolution of \ce{MoO3}: second stage]{Growth evolution of the second stage of catalyst-assisted growth of \ce{MoO3} 1D structures: SEM images of the deposition at different growth temperatures with growth time of row 1: 15 min, row 2: 30 min, row 3: 60 min, and row 4: 90 min. The estimated temperature for each column from left to right is 525 \si{\degreeCelsius}, 445 \si{\degreeCelsius}, 390 \si{\degreeCelsius} and 340 \si{\degreeCelsius}, respectively. }
\label{fig:ch4ev}
\end{figure}


To reveal the growth mechanisms for the long 1D structures in the second stage, specimens were carefully examined to locate the catalyst particles. Small catalyst particles ranging from several hundred nanometers to several microns were found at different locations of these long 1D structures. As shown in Fig.~\ref{fig:ch4edx1} and Fig.~\ref{fig:ch4edx2}, EDX element analysis confirmed the presence of Na, Mo, and O in these particles, indicating the particles are sodium molybdate catalysts.  

\begin{figure}[htb]
\centering
\includegraphics[width=0.5\textwidth]{MoO3_SFig_EDX}
\caption[Growth evolution of \ce{MoO3}: second stage]{SEM images and EDS spectra identifying the location, morphology, and composition of the catalyst particles on the nanobelt structures: (a-b) a catalyst particle (I) in the middle of a nanobelt (II), (c-d) two catalyst particles (I and II) leading the growth of two perpendicular nanobelts. }
\label{fig:ch4edx1}
\end{figure}


\begin{figure}[htb]
\centering
\includegraphics[width=0.5\textwidth]{MoO3_SFig_EDX_tower}
\caption[Growth evolution of \ce{MoO3}: second stage]{SEM images and EDS spectra for identifying the location, morphology, and composition of the catalysts on the microtower structures: (a-b) a tiny catalyst particle (I) on top of a small cone shaped microtower (II), and (c-d) a catalyst particle (III) on a large and short microtower (IV).}
\label{fig:ch4edx2}
\end{figure}
It could be of difficulty to observe the sodium molybdate particles on the as-synthesized belts and towers due to the following two factors, one is the consumption of catalysts, such as evaporation; the other is the unpredictable location of the catalyst. Unlike the conventional \gls{vls} growth where the catalysts were predominantly found at the tip of 1D nanostructures, the catalyst in current study were mostly found at the side wall of long nanobelts, and the location could be on the tip, or several microns away from tip. As for the microtowers, it is even rare to observe Na element.\footnote{try m-Raman at different depths} 

Some observations under similar controlled growth conditions is also listed as following to support the proposed growth model: 

\begin{enumerate}
\item When no Mo powder is used, NaOH still holds onto substrate with some locations exhibiting etching.
\item When no Mo powder is used while growth is performed in an old tube or oxygen flow is limited to a extreme small value, circular droplet-like solid were found and Mo, Na and O were detected, suggesting the existence of VLS mechanism. Using Mo coated substrate and subsequent NaOH treatment, similar droplets formation was repeated. No discernible XRD pattern emerged, illustrating the amorphous nature of liquid catalyst.
\item 1 min growth shows plates in high temperature end and irregular shapes in low temperature end.
\item 15 min growth shows \ce{MoO3} plates parallel to substrate, with some belts emerging and growing out of the substrate plane. This is presumably due to the growth direction mismatch between different plates. Morphology will also depend on local NaOH concentration. NaOH trace mark turns into connected plates at the original locations, showing the high concentration mediated the morphology evolution. The possible catalyst solid was sitting on the top of plate layer. Also some belt-like growth is visible on reaction chamber walls, suggesting the evaporation and transport of \ce{Na_xMoO3} vapor phase;
\item 30 min growth: belts growth begin (500 microns long, grow rate is estimated to be 0.5 $\mu$ m/s), tower structures show up as well, but much less than belt. Two kinds of belts found, tapering and non-tapering. The former one seems to have droplets mostly associated with sidewalls of the top part.
\item In reduced time or less oxygen supply growth, Na-containing site was found located onto the sidewall of belts instead of the usual expected tip. XRD pattern indicated the existence of \ce{Na_xMoO3} phase. And \ce{MoO3} XRD pattern is dominated by (0k0) peaks, indicating a 2D layer-by-layer growth with lateral rate exceeding that of axial direction.
\item 2 h growth: there are crossed tapering belts at downstream boundary, with more than one droplet located along the interface of two perpendicular belts, EDX confirmed high Na content.
\item Using NaOH treated hydrophobic substrates, dense tower array growth was found along the edge areas or molten periphery of droplets, where 5 min growth already show the initial stage of tower with diameter about 1 $\mu$ m, which increase to 10 $\mu$ m in prolonged growth.
\end{enumerate}


The growth of \ce{MoO3} on NaOH-Si can be divided into two stages delimited by the emerging of long nanobelts and microtowers. A growth model is proposed to account for the observed morphologies of \ce{MoO3} long nanobelt and tower-like structures. 
\begin{hypothesis}\label{hypo2}
A transverse growth mode exists with growth direction perpendicular to the catalyst-deposition interface, promoting \ce{MoO3} nanoplate and nanobelt growth. A axial growth mode akin to conventional VLS process exists, and is responsible for \ce{MoO3} tower-like growth.  
\end{hypothesis}

The first stage is dominated by transverse growth. The atomic steps in proposed growth model are depicted in the upper panel of Fig.~\ref{fig:ch4vls}. It has been verified that \ce{Na2Mo4O13} forms during the early stage of growth, and can act as catalysts. Although the exact phase (liquid or solid) of \ce{Na2Mo4O13} is not known yet in current study, it can be deduced that the \ce{Na2Mo4O13} exists as an active site for the adsorption of incoming \ce{MoO3} growth species. When the catalyst is smaller than some critical size, there is no preferred growth direction. This changed when the precipitated solid exceeds certain critical dimension, and the morphology evolution is then determined by the relative size of already formed solid and the associated catalyst. Several pathways could occur after the accommodation of \ce{MoO3} vapor and supersaturation. \ce{MoO3} could precipitate at catalyst-deposition interface, or propagate along catalyst-environment boundary and spread on the already formed \ce{MoO3} surface. Meanwhile, the \gls{vs} process is not suppressed and \ce{MoO3} adatoms could directly condense on \ce{MoO3} surface.

\begin{figure}[htb]
\centering
\includegraphics[width=0.6\textwidth]{vls_mode2}
\caption[Growth model of \ce{MoO3}]{Schematic drawing on the growth steps of the proposed growth model in hypothesis~\ref{hypo2}.}
\label{fig:ch4vls}
\end{figure}

Some unique features of \ce{MoO3} must be taken into consideration to properly evaluate the dominate pathway(s). It has been reported that \ce{MoO3} can spontaneously spread over the surface of supports (\emph{e.g.}, \ce{Al2O3}, \ce{SiO2}, and Au) to form a monolayer or submonolayer at a temperature (257 \si{\degreeCelsius}) well below the melting point.\cite{Leyrer1990} This strong spreading behavior can be explained by the solid-solid wetting process, in which the driving force is the decrease in total surface free energy.\cite{Leyrer1988} Although the detailed transport mechanisms of the spreading are still in debate, it is suggested that the high mobility of the Mo oxide species on substrates and \ce{MoO3} islands can promote the spreading and the ambient gases could further enhance the spreading.\cite{Gunther2000, Song2003} Based on above analysis and the final morphology of \ce{MoO3}, the dominance of accommodation and spreading pathway in higher temperature part of the growth sample is most possible in this study. Hereby the transverse growth mode is attributed to the strong spreading capability of \ce{MoO3}. The final morphology of observed nanoplate and nanobelt is a combined result of \gls{vs} and \gls{vls} growth. 

On the other hand, the tower-like structures in the second stage follow the axial growth mode in conventional VLS process. The molybdates can be evaporated from the higher temperature end and re-condense onto the lower temperature region. As illustrated in the lower panel of Fig.~\ref{fig:ch4vls}, this could initiate either belt growth or tower growth depending on the locations. The microtowers in this study primarily grow at the low temperature end of the substrate, where the spreading capability of the Mo oxide species is possibly limited. Thus the transverse growth pathway is probably suppressed and axial growth becomes dominant.

The transport of molybdates in proposed growth model is verified by a side-by-side experiment, as schematically depicted in Fig.~\ref{fig:ch4sbs}a. Two substrates, one with NaOH catalyst and one bare substrate, were loaded side-by-side at the same location for the growth. To avoid any possible catalyst transports through surface diffusion, the two substrates were separated with several millimeters gap in between. 

\begin{figure}[htb]
\centering
\includegraphics[width=0.8\textwidth]{MoO3_SFig_sidebyside}
\caption[side by side growth of \ce{MoO3}]{(a) Schematic drawing of the side-by-side growth with one NaOH catalyst treated substrate and one bare substrate. SEM images of the growth on the untreated substrates at different growth times: (b-e) 30 min and (f-i) 120 min. For 30 min growth: (b) low magnification SEM image showing labeled growth areas with different morphologies, (c) rectangular nanoplates grown without catalyst, (d) nanobelts grown with catalyst, and (e) forked nanoplates grown with catalyst. For 120 min growth: (f) rectangular nanoplates without catalyst, (g) forked nanoplates with catalyst, (h) microbelts with catalyst, and (i) microtowers with catalyst.}
\label{fig:ch4sbs}
\end{figure}

After the growth, besides the non-catalyst rectangular plates, catalyst induced 1D structures were also found on the bare substrate (Fig.~\ref{fig:ch4sbs}b-i). This result clearly confirms that the catalyst can be evaporated and transferred from the NaOH treated substrate to the untreated one. The catalyst vapors can nucleate and form catalyst particles on the \ce{MoO3} deposition promoting the VLS growth of different 1D structures. It is worth mentioning that catalyst particles were not observed on all the 1D structures. This fact can be explained by following reasons:
\begin{enumerate*}[label=\itshape\alph*\upshape)]
\item Catalyst particles can nucleate at any locations along the 1D structures. Some particles may be invisible hiding on the backside of the 1D structures;
\item Because of the evaporation, the catalyst particles may become too small to be detected. Some particles could even disappear terminating the catalyzed growth, which may explain the growth of the microtowers with a flat top.
\end{enumerate*} The evaporation and nucleation of catalyst not only promote the growth of the 1D structures in the second stage growth, they also shape the morphologies of these 1D structures. Many 1D structures in the second stage were found to have a tapered shape, such as triangular nanobelts (Fig.~\ref{fig:ch4sem2by3}b and d), microtowers with tapered tips (Fig.~\ref{fig:ch4sem2by3}f). Several factors can contribute to the tapering of the 1D structures. First, the reduction in catalyst size may induce the tapering. Due to the evaporation, the size of the catalyst particles can shrink during the growth resulting in the tapered growth. The \gls{vs} growth on the side surfaces is another possible mechanism, which promotes the radial growth of the 1D structures forming the tapered shape. Another important factor is the gradually reduced growth during the cooling process. During the cooling, the \ce{MoO3} vapor supply reduced and the growth temperature gradually decrease. Hence the growth rate slows down producing the tapered tips on the top of some 1D structures.

\subsection{Growth on Alkaline Metal Containing Substrate}
In analog with the \ce{Na2MoO4}-\ce{MoO3} binary phase diagram, other alkaline metals, such as Li and K, could also form A-Mo-O alloy at elevated temperature. This fact allows for the usage of a variety of alkaline metal based material as catalyst to promote \gls{vls} or \gls{vss} growth of \ce{MoO3}. Some experiments have been performed in this study, including growth using KI and \ce{Na2CO3} treated Si substrate, glass and mica. Fig.~\ref{fig:ch4al1}a-b show the long belt growth on KI treated Si substrate and towers on ITO glass, respectively. Compared to NaOH-Si growth, the belt on KI appears with rough edge, whereas the towers on ITO glass exhibit stacking features. 

\begin{figure}[htb]
\centering
\includegraphics[width=0.6\textwidth]{Fig7_MoO3_Small}
\caption[Alternative catalysts for \ce{MoO3} growth 1]{SEM images of \ce{MoO3} 1D structures. (a) nanobelts grown with KI and (b) microtowers grown on ITO glass}
\label{fig:ch4al1}
\end{figure}

Fig.~\ref{fig:ch4al2}a-c illustrate the forked plate growth on \ce{Na2CO3} treated Si, sword-shape belts on glass, and nanoblets growth on mica, respectively. No long belt or tower was observed on the sample with \ce{Na2CO3}. Long belts found on mica (\ce{K(Al2)(Si3Al)O10(OH)2}) probably arise from potassium-catalyzed \gls{vls} mechanism.\cite{Hu2011} More detailed results on glass will be presented in Sec.~\ref{sec:glass}. 

\begin{figure}[htb]
\centering
\includegraphics[width=0.8\textwidth]{SFig5_MoO3_Small}
\caption[Alternative catalysts for \ce{MoO3} growth 2]{SEM images of MoO3 1D structures: (a) forked nanoplates grown with \ce{Na2CO3}, (b) triangular microbelts grown on glass, and (c) nanobelts and microbelts grown on mica.}
\label{fig:ch4al2}
\end{figure}

The various growth morphologies found using alkaline metal based materials as catalysts demonstrate the proposed \gls{vss} mechanism and transverse mode is a general phenomenon, and broad engineering space could be explored to control the dimension of \ce{MoO3}. 

\subsection{Growth on Glass}\label{sec:glass}
The mass transport of sodium ions in silica glass is first estimated to evaluate the amount of catalyst available during the growth of \ce{MoO3}. The driving force could be external field or concentration gradient. And the dynamics is governed by diffusion equation
\begin{align}
\frac{\partial C}{\partial t} = D \frac{\partial^2 C}{\partial x^2},
\end{align}
where $C$ is concentration in unit of \si{mol\per cm^3}, and $D$ is diffusion coefficient in unit of \si{cm^2\per\second}. Diffusion coefficient highly depends on the overall environment in which the ions reside. Typical values of $D$ for Na ions are listed in Table~\ref{tab:mona}

\begin{table}[htb]
\centering
\caption[Na diffusion coefficient]{Na diffusion coefficient (\si{cm^2\per\second}), R is molar gas constant (8.315 \si{\joule\per mol\per K}), and T is temperature in K}\label{tab:mona}
\begin{tabular}{lcr}
\toprule
 Composition & Value  & Reference  \\
\midrule
Quartz      & $3.8\times10^{-2}\exp(\frac{-24500}{RT})$  & \cite{Rybach1967a}  \\
 \addlinespace[0.5em]
Sodalite      & $6.6\exp(\frac{-42500}{RT})$  & \cite{Sippel1963}  \\
 \addlinespace[0.5em]
Obsidian     & $4.4\times10^{-2}\exp(\frac{-22900}{RT})$  & \cite{Sippel1963}  \\
 \addlinespace[0.5em]
silicate glass & $3.1\times10^{-8}$ at 420\si{\degreeCelsius} & \cite{Jbara1995} \\
 \addlinespace[0.5em]
\ce{SiO2} glass & $1.1\times10^{-7}$ at 670\si{\degreeCelsius} &  \cite{FRISCHAT1968}\\
\bottomrule
\end{tabular}
\end{table}

It is worth mentioning that the gradient of Gibbs free enthalpy overruled the concentration of a particular species, and higher \ce{OH^-} contents tend to reduce Na diffusivity.\cite{Materials2012}  The substrates used in this work are glass and ITO/glass. Related parameters is as following:
\begin{itemize}
\item dimension: $25\times10\times1$ mm
\item density: 2.567 \si{g\per cm^3}
\item Na concentration: 0.01 \si{mol\per cm^3}
\item $D$: $1.1\times10^{-7}$ \si{cm^2\per\second}
\end{itemize}
Then the net flux $J$ is estimated as
\begin{align}
J = -D \frac{\Delta C}{\Delta x} = 2.0\times 10^{-8} \si{mol\per cm^2\per\second}.
\end{align}

In typical growth of 2 h, the net amount of Na diffusing out of substrate is then $1.44\times10^{-4}$mol, about 10\% of total Na ions in the glass substrates. The diffusion length $\sqrt{Dt}$ is 190 $\mu$ m. Notice this result overestimates since the flux will reduce gradually. Meanwhile the molar amount of \ce{NaOH} applied on Si substrate is about $5\times10^{-7}$mol. Considering the adjusted ratio, Na contents should be much more when using glass substrate than when using NaOH-Si substrate. Another important factor is the evaporation of sodium molybdates. However, few report could be found in literature. Only one was reported by \citeauthor{Kazenas2010}.\cite{Kazenas2010} The partial pressure of \ce{Na2MoO4} is calculated to be 0.2 mTorr at 800 \si{\degreeCelsius} and $3.5\times 10^{-3}$ mTorr at 670 \si{\degreeCelsius}.\footnote{$\log P(atm)= -13794/T + 6.19$} 

This thesis studied the effect of temperature, oxygen partial pressure and growth time towards \ce{MoO3} deposition on glass substrates. Each factor was divided into two levels, with $T$ at 700 and 800 \si{\degreeCelsius}, \ce{O2} flow at 1 and 10 sccm, and growth time of 15 and 60 min. And the morphological variations revealed by SEM imaging were summarized in Table~\ref{tab:mo3glass}.
% glass growth matrix
\begin{table}[htb]
\centering
\caption{\ce{MoO3} growth on glass}\label{tab:mo3glass}
\begin{tabular}{lccp{3in}}
\toprule
\multicolumn{3}{c}{Growth Conditions} \\
\cmidrule(l){1-3}
 Temperature & \ce{O2} & Time & Morphology  \\
\midrule
700    &  10   & 15  &   about 50 $\mu m$ belts, some towers \\
700   &  1   & 60  &   irregular shape, not too much growth\\
700    &  10   & 60  &   droplet almost all on tip of belts, many towers \\
800    &  1   & 15  &   tens $\mu m$ belt with sharp edge(100 nm thick, and 1micro wide) dominates\\
800   &  1   & 60  &   exhibit ITO features, droplets on tips of belt, many tower like structures\\
800    &  10   & 15  &   rounded edge flakes dominate, modest density tapering belt, and some possible tower\\
800    &  10   & 60  &   ITO/glass features, aggregated belts, some towers\\
\bottomrule
\end{tabular}
\end{table}

The 800-1-15 combination produced dense \ce{MoO3} nanoplate with length about 10 $\mu$m, width $300\sim500$ nm and thickness less than 100 nm, as illustrated in Fig.~\ref{fig:ch4glass}. 
%
\begin{figure}[htb]
\centering
\includegraphics[width=0.7\textwidth]{moo3_glass}
\caption[\ce{MoO3} nanoplates on glass]{Optimized \ce{MoO3} nanoplates growth on glass.}
\label{fig:ch4glass}
\end{figure}
These \ce{MoO3} nanoplates are of high crystalline quality, and could potentially show enhanced photoelectrochemical and electrochromic activities. 

\subsection{Optical Properties of Molybdenum Oxide}

\gls{fl} materials attracts intensive research efforts recently. This study also made some efforts using liquid exfoliation method to prepare \gls{fl} \ce{MoO3}. The exfoliated \ce{MoO3} was characterized using UV-Vis and TEM. The optical contrast of \gls{fl} \ce{MoO3} on \ce{SiO2}-Si is also calculated, which could serve as a guideline to identify \gls{fl} \ce{MoO3} using optical microscope.

The exfoliation solution is 30\% isopropanol (IPA) in DI water, which has been proved to work well on other layered compounds.\cite{Halim2013} After 20 min sonication, the dispersion appears milky. The transmission spectra of this dispersion immediately upon sonication and after 40 h of gravity sedimentation were measured and compared in Fig.~\ref{fig:moabs}. The t0 line mainly arise from the scattering of flakes in the dispersion. There are two scattering mechanisms: Rayleigh scattering and Tyndall scattering. Rayleigh scattering, which occurs when particles are comparable to wavelength, is inversely proportional to the fourth power of wavelength; while Tyndall scattering, which occurs when the particles are larger, is inversely proportional to the square of the wavelength. To fully evaluate the true absorbance of the sample itself, one needs to decouple the scattering part from apparent absorbance. Generally, one part of spectrum away from absorption edge is selected and assumed it is only caused by scattering. Then an empirical dependence shown in Eq.~\ref{eq:sca2} is used in the polynomial fitting model.

\begin{align}
Abs_{sca}  & = a\times \lambda^{n}  \label{eq:sca1}\\
\log{Abs_{sca}} & = \log{a} + n*\log{\lambda} \label{eq:sca2}
\end{align}

After least-squares fitting, the coefficients $a$ and $n$ can be used to estimated the scattering in other wavelengths. The fitting of $n$ usually falls between -2 and -4. The author examined the t40h line in Fig.~\ref{fig:moabs} and found the scattering part is almost zero. Therefore no scattering fitting was performed.

\begin{figure}[htb]
\centering
\subfloat[]{\label{fig:moabs}\includegraphics[width=0.45\textwidth]{MoO3-abs}}\hspace{0.04\textwidth}
\subfloat[]{\label{fig:mobg}\includegraphics[width=0.45\textwidth]{MoO3-bg}}
\caption[UV-Vis spectra of exfoliated \ce{MoO3}]{(a) UV-Vis absorbance spectra of exfoliated \ce{MoO3}. Dashed line (t0) represents measurement immediately after sonication, and solid line (t40h) is obtained after sitting for 40 h. The dot dashed line is reference spectrum of the 30\% IPA. (b) Band gap estimation}
\label{fig:mouv}
\end{figure}

According to the corrected absorbance spectrum, optical bandgap can be deduced, which is known as Tauc Plot.\cite{Tauc1972} The procedure is recapitulated here briefly. Eq.~\ref{eq:tauc} is given to empirically relate optical band gap to absorption.

\begin{equation}\label{eq:tauc}
 (h\nu \alpha)^{\frac{1}{n}} = B(h\nu - E_g^{opt}),
\end{equation}
where B is a proportional coefficient, $\alpha$ is absorption coefficient\footnote{The absorbance $A = 0.4343\alpha*l$, where l is the optical path} and $E_g^{opt}$ is defined as optical bandgap. The value n is correlated to the nature of optical transition.
\begin{itemize}
\item For direct allowed transition, n = 1/2;
\item For direct forbidden transition, n = 3/2;
\item For indirect allowed transition, n = 2;
\item And for indirect forbidden transition, n = 3.
\end{itemize}

When the transition nature of sample is not well known, one should apply each to check which one provide the best fit. Depending on the synthesis methods and \ce{MoO3} morphologies, the transition phenomena might vary. On the \ce{MoO3} films, \citeauthor{Bouzidi2003} assumed a direct gap (n = 1/2)\cite{Bouzidi2003} while \citeauthor{Szekeres2002} used an indirect gap.\cite{Szekeres2002} The author use n = 2 in this work, plot $\sqrt{(h\nu \alpha)}$ versus photon energy, and extrapolate the linear part onto $h\nu$ axis. The intercept is used as $E_g^{opt}$ then. The estimated energy gap and some results from previous studies is summarized in Table.~\ref{tab:mobg}.

\begin{table}[htb]
\centering
\caption{Optical band gaps of \ce{MoO3}}\label{tab:mobg}
\begin{tabular}{lccr}
\toprule
&\multicolumn{2}{c}{Band gap} \\
\cmidrule(l){2-3}
Reference & value(eV) & orientation & material states\\
\midrule
\cite{Deb1968}   & 2.96  & $E\parallel c_0$ & single crystal\\
\cite{Deb1968}   & 2.80  & $E\perp c_0$ & single crystal \\
\cite{Julien1995} & 2.8$\sim$ 3.2 & NA & films\\
 this work  & 3.1  & NA & nanoflakes\\
\bottomrule
\end{tabular}
\end{table}

An alternative approach is a) define $Y = \alpha E=\alpha h\nu$, and obtain $Y' = \frac{\partial Y}{\partial E}$, then b) plot $Y'/Y$ versus $h\nu$. The band gap value can be obtained from the intercept of extrapolated line on photo energy axis.\cite{Choopun} This approach is not difficult to comprehend as long as we notice from Eq.~\ref{eq:tauc}, $Y = B (h\nu - E_g^{opt})^n$ and $Y' = B n(h\nu - E_g^{opt})^{n-1}$. The advantage is one can also estimate n without using assumption.

The dielectric constants\footnote{Not a constant at all, instead, it has complicated dependence on photon energy} ($\epsilon = \epsilon_1 + i*\epsilon_2$) of \ce{MoO3} have been studied in several reports.\cite{Deb1968,Sabhapathi1995,Miyata1996,Abdellaoui1997,Mondragon1999} The author extracted $\epsilon_1$ from Ref.~\cite{Itoh2001a} due to the wide photon energy range utilized there. Notice that $\epsilon_2$ is approximately zero when $h\nu$ is less than 4 eV. Since this study is primarily concerned with the optical contrast of few-layered \ce{MoO3} in visible wavelength region ($h\nu < 3$ eV), the refractive indices $n$ is calculated from $n = \sqrt{\epsilon_1}$. Due to the orthorhombic phase, \ce{MoO3} exhibit birefringence, that is there are two refractive indices, $n_c$ and $n_a$, as shown in Fig.~\ref{fig:moind}. The refractive index is between $ 2.2\sim 2.4$ along $a$ axis, and $ 2.5\sim 2.8$ along $c$ axis.

\begin{figure}[htb]
\centering
\subfloat[]{\label{fig:moind}\includegraphics[width=0.45\textwidth]{n_MoO3}}\hspace{0.04\textwidth}
\subfloat[]{\label{fig:mocon}\includegraphics[width=0.45\textwidth]{MoO3_1L}}
\caption[Refractive indices of \ce{MoO3}]{(a) Refractive indices of \ce{MoO3} Dashed line is n along $a$ axis while solid line is along $c$ axis. (b) optical contrast mapping of 1L \ce{MoO3} on \ce{SiO2}-Si. The x axis is \ce{SiO2} thickness, y axis is wavelength.It is assumed that the incident light is polarized along c axis of \ce{MoO3}.}
\label{fig:mofl}
\end{figure}

Fig.~\ref{fig:mocon} illustrates the optical contrast mapping of 1L \ce{MoO3} on \ce{SiO2}-Si when the incident light is polarized along $c$ axis of \ce{MoO3}. One can see that there is a pseudo-periodical trend with either wavelength or \ce{SiO2} thickness fixed. The positive extreme occur at around 80 nm and 240 nm \ce{SiO2} layer, slightly less than the optimum for observing graphene. The refractive indices of Si and \ce{SiO2} within visible wavelengths and MATLAB source code for optical contrast calculation is included in Appendix~\ref{app:matlab}, with which one can readily evaluate the few layers \ce{MoO3} scenarios and the circumstance when incident light is polarized along $a$ axis.

The dimension and phase of exfoliated \ce{MoO3} is examined using TEM. Fig.~\ref{fig:motem} shows the size of exfoliated \ce{MoO3} is about 500 nm. High-resolution TEM (HRTEM) image was taken to assess the \ce{MoO3} crystallinity. The clearly resolved lattices indicate high crystal quality of these exfoliated \ce{MoO3}.
\begin{figure}[htb]
\centering
\includegraphics[width=0.8\textwidth]{exf_moo3.jpg}
\caption[TEM images of exfoliated \ce{MoO3}]{(a) TEM image and (b) HRTEM image of FL \ce{MoO3} obtained by liquid sonication method.}
\label{fig:motem}
\end{figure}
Nevertheless, identifying these few layer \ce{MoO3} under optical microscope is still difficult due to both the small flake size and low contrast.\footnote{Human eye threshold contrast (contrast sensitivity)$^{-1}$ is about 10\%.} One need to improve the liquid exfoliation process to obtain larger dimension and high yield. This is, however, beyond the scope of current study.


\section{Summary}

In this chapter, 
