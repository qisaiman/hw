% git version control added 012614
\chapter{Molybdenum Oxides}

\section{Introduction}

Molybdenum oxides \ce{MoO3} crystallizes in three phases, orthorhombic $\alpha$-\ce{MoO3}, monoclinic $\beta$-\ce{MoO3} and the metastable hexagonal h-\ce{MoO3}.\citep{Deb1968,Fibers2007} $\alpha$-\ce{MoO3} phase (hereafter \ce{MoO3})exhibits anisotropic structure with strong bonding along [001] and [100] direction while Van der Waals interaction along [010] direction.\cite{He2003} Due to this unique structure, \ce{MoO3} was found to have several important properties and a wide range of technological applications, such as electrochromism and photochromism,\cite{Yao1992} lubricants,\cite{Sheehan1996} photocatalysts,\cite{Chen2010} and gas sensor, including \ce{CO},\cite{Comini2005} \ce{NO2},\cite{Taurino2006} \ce{H2}\cite{Sha2009} and ethanol.\cite{Choopun} Moreover, other features arise when nanoscale \ce{MoO3} is specifically prepared, for instance, field emission.\citep{Li2002d,Zhou2003b}  As a layered hosting material, \ce{MoO3} can be further modified by intercalating with alkaline ions\citep{Spahr1995,Li2006b,Hu2011} and even divalent ion\cite{Sian2005}. This structural richness have enabled improved performance in Li-ion\cite{Mai2007} and sodium-ion battery.\cite{Hariharan2013} In combination with \ce{TiO2} forming a core-shell structure, these nanoparitcles are reported to lower the photoabsorption energy of \ce{TiO2}.\cite{Elder2000} When combined with Ag as layered structure, transparent conducting behavior was observed.\cite{Nguyen2012} In addition, \ce{MoO3} is a good precursor for preparing other useful materials, such as \ce{MoS2} fullerenes\cite{Li2003c}, and few layer \ce{MoS2}.\cite{Lin2012} It will also be an important member of the Van der Waals heterostructures.\cite{Geim2013}

In this thesis, we developed a new method to prepare \ce{MoO3} by using alkaline oxides as catalysts. This newly discovered a vapor-liquid-solid growth mechanism produced two new \ce{MoO3} structures: long nanoribbons and microscale towers. We then investigated the phase and crystalline structures of these new morphologies, and demonstrated the application in photocatalyst and electrochromic devices.


\section{Synthesis and Properties of Molybdenum Oxides}

\subsection{Growth of molybdenum oxides}

Researchers have been exploring a variety of methods to synthesized different \ce{MoO3} nanostructures. We will not try to exhaust these efforts since it has been well summarized in several review articles.\cite{He2003} So we only gave a brief introduction here. Most of these methods can be categorized into two groups: solution-based hydrothermal procedures \citep{Li2002b,Xia2006,Li2006a,Camacho-Bragado2006} and chemical vapor transport and deposition approaches. \citep{Zeng1998,Li2002c,Li2002d,Zhou2003b,Fibers2007,Yan2009}. Both methods have its own merits. Hydrothermal process usually need lower temperature(\textless 300 \si{\degreeCelsius}), but the duration is usually long and requires several post-growth processing steps; while vapor deposition demands relatively high temperature (\textgreater 500 \si{\degreeCelsius}) yet with a shorter time. Both methods are scalable for industrial applications. However the hydrothermal treatment seems to allow more  nanostructures control than do the vapor deposition method. Nanobelts,\cite{Li2002b} helical nanosheets, nanoflowers, prims-like rods\cite{Li2006a} and nanoribbons\cite{Camacho-Bragado2006} were obtained by the former one while nanoflakes,\cite{Chen2009} nanobelts\cite{Hu2009} and nanowires\citep{Zhou2003b,Chen2011b} dominated the product morphology for the latter one. The assistance of Au catalyst in vapor deposition only altered the orientation\cite{Yan2009} or served as preferred nucleation sites\cite{Cai2011} without producing new \ce{MoO3} structures. \citeauthor{Chithambararaj2013} prepared hexagonal \ce{MoO3} nanocrystal via hydrothermal method and demonstrated the photodegradation of MB under visible light.\cite{Chithambararaj2013} The efficiency dependence on catalysis/dye ratio, light intensity and temperature was studied. h-\ce{MoO3} was mostly synthesized using solution methods, where \ce{NH4+} and \ce{OH-} were possible structure directing and stable agents. The band gap of h-\ce{MoO3} estimated from diffusion reflection spectrum is $2.8\sim3.0eV$. Hexagonal phase of \ce{MoO3} is readily identified by the XRD pattern (strong peak at $2\theta=20^{\circ}$).

In spite of these numerous growth of \ce{MoO3}, we point out that reports on VLS mechanism and other than Si substrates are still scarce. We will explore these relative new domains and present our results in Sec.~\ref{sec:result}.

\subsection{Crystal structures and chemical features of molybdenum oxides}

The coordination number of Mo in \ce{MoO3} is six, so usually \ce{MoO6} octahedra are considered as the building blocks. As shown in Fig, the layered structure consisting of zig-zag rows of edge-sharing \ce{MoO6} octahedra, while the rows are mutually connected by corners.

While considering the fact that four of the six surrounding O atom are at distances from 1.67\AA to 1.95\AA, while the remaining two are as far as 2.25 and 2.33\AA, \ce{MoO3} could be considered as built up of chains of \ce{MoO4} tetrahedra connected by the sharing of two oxygen corners with two neighbouring tetrahedra in $c$ axis. The infinite chains of \ce{MoO4} tetrahedra from half-layers in the $ac$ plane. Two half-layers, which are stapled along $b$ axis, build up one \ce{MoO3} layer.\cite{Itoh2001a} This also stress that the \ce{MoO6} octahedra are rather distorted.

Due to the weak cation-oxygen bonding, alkali metal cations only introduce small perturbations into the energies of Mo-O or W-O ensembles in comparison to cations of other elements. And no mixing of vibrations of the cationic sublattices with that of \ce{Mo-O} or \ce{W-O} lattices. So the structural features of \ce{Mo-O} or \ce{W-O} polyhedra are dominating factors affecting the vibration frequencies and thermodynamic values of the molybdates and tungstates.\cite{Fomichev1992}

\begin{table}[htb]
\centering
\caption{Crystal structures of alkali metal molybdates and tungstates}\label{tab:naxmow}
\begin{tabular}{llr}
\toprule
formula & structure  & remarks  \\
\midrule
\ce{A2O}:\ce{MO3}\textsuperscript{\emph{a}} & isolated tetrahedral \ce{MO4} anions& \\
\ce{A2O}:\ce{2MO3} & chain-type anions of \ce{MO4} and \ce{MO6} & \\
\ce{A2O}:\ce{3MO3} & chain-type anions of \ce{MO5} and \ce{MO6} & \\
\ce{A2O}:\ce{4MO3} & chain-type anions of \ce{MO6} & \\
\bottomrule

\textsuperscript{\emph{a}} A = Li, Na, K, Rb, Cs; M = Mo, W;
\end{tabular}
\end{table}

Raman band assignment for \ce{A2MO4} is well grounded. Alkali metal cations with octahedral coordination occur below 230 $cm^{-1}$. For \ce{A2M_nO_{3n+1}}, exact assignment is not available yet due to the complicated \ce{MO6} octahedra.

Molybdenum bronze also exhibits intriguing features, for instance, charge-density-wave states\footnote{In CDW states, conductivity is non-Ohmic above a threshold electric field} was found in blue bronze \ce{K_{0.3}MoO3}, which can be prepared by electrolytic reduction of \ce{K2MoO4} and \ce{MoO3} melt.\cite{Dumas1983} \ce{K_{0.3}MoO3} is side=centered monoclinic phase, with lattice parameter $a=18.249$\AA, $b=7.561$\AA, $c=9.856$\AA, and $\beta=117.54$ at RT. \ce{K0.3MoO3} presents a semiconductor-to-metal transition at 180 K. Electrical transport measurements yield an highly anisotropic ratio of DC conductivities of 1:10:1000. Along this high conducting axis, metallic reflection behavior is confirmed by optical measurements. Therefore, this blue bronze $\ce{K_{0.3}MoO3}$ is known as quasi-1D metal.\cite{Sing1999}

\subsection{Synthesis methods in this theis}

The reactants used in this study were listed in Table.~\ref{tb:mosource}. All reactants were used as received without further processing.

\begin{table}[htb]
\centering
\caption{Reactants list}\label{tb:mosource}
\begin{tabular}{lcccr}
\toprule
Material & Stock No & LOT &Purity & Vendor\\
\midrule
\ce{NaOH}        & S318-500 & 070241 & 99.8\% & Fisher Scientific \\
\ce{NaI}        & 11665 & K11W054 & 99.9\% &  Alfa Aesar \\
\ce{KI}        & 42857 & H06Z051 & 99.9\% &  Alfa Aesar \\
\ce{Na2CO3}        & 33377 & 114X012 & 99.95\% &  Alfa Aesar \\
\ce{Molybdenum}        & 00932 & I07S024 & 99.9\% &  Alfa Aesar\\
\bottomrule
\end{tabular}
\end{table}

\textbf{Substrate treatment.} Silicon substrates(p-$\langle100\rangle$, University Wafers) were first cut into 10mm by 25mm pieces and ultrasonically cleaned with acetone and ethanol for 15 minutes(Branson 1510R-MTH), each followed by blow-drying with nitrogen gas. Then the substrates were selectively treated in an oxygen plasma cleaning apparatus(Kurt J Lesker: Plasma-Preen 862, 2Torr) for 3 minutes to increase the wettability. The substrates without plasma cleaning were rendered as hydrophobic. Finally, the substrate was drop-cast 54 $\mu$L 10 mM \ce{NaOH} (or \ce{KI}, \ce{Na2CO3}) and naturally left dry in a convection hood with a flask cover above. Other alkaline ions-containing substrates, such as glass (Fisher Scientific, microscope slide, 12-549) and indium tin oxide (ITO) coated glass(Delta Technologies, 25\si{\ohm}), were cleaned by the same routine except the absence of plasma cleaning and subsequent aqueous solution dipping. Mica was cleaved right before growth without other treatment.

\textbf{Synthesis.} In a typical synthesis (Fig.~\ref{fig:mooxgrowth}), molybdenum powders were loaded into the uniform heating zone and the sealed chamber was first pumped down to 10 mTorr. Then oxygen flow varying between 0.1 sccm to 10 sccm (standard cubic centimeter per minute) was admitted from upstream inlet. With 10 sccm UHP Ar as carrier gas, the overall pressure reached about 200 mTorr. The heating temperature was ramped up to 800 \si{\degreeCelsius} in 30 minutes and lasted for 15 to 120 minutes. Then the heating power was turned off and the chamber was allowed to naturally cool down to room temperature. The receiving substrate was placed in downstream location where the temperature was about 400 to 670 \si{\degreeCelsius} according to open air temperature profile. To observe the formation of droplets, clean silicon substrate was coated with 9 nm molybdenum film using a magnetron sputter(Desk IV TSC, Denton Vacuum), followed by \ce{NaOH} dipping as described before. Then the substrate was placed into chamber for growth, where all the parameters remained unchanged except the absence of Mo powder source.

\textbf{Materials characterization.} The morphology and composition of the as-synthesized samples were analyzed by scanning electron microscopy (SEM,JEOL JSM-6480) and energy dispersive X-ray spectroscopy (EDS,Oxford Instrument INCA). Crystal structures were characterized using X-ray diffraction (XRD, PANXpert X’pert Pro MRD with Cu $K\alpha$ radiation at $\lambda$=1.5418\AA) and transmission electron microscopy (TEM, JEOL JEM-2100 \ce{LaB6} operated at 200kV). Optical measurements were performed by Micro-Raman spectroscopy (Horiba Scientific, Labram HR800 with 532 nm excitation laser) in a confocal microscope backscattering configuration with spectral resolution about 1 cm$^{-1}$. Optical absorption spectra was recorded using UV-Vis-NIR spectrophotometer(Schimadzu, UV2600Plus) in transmission mode. The as-synthesized sample was removed from substrates by light sonication(Branson 1510R-MTH, 70W) in ethanol for 15 seconds. The dispersion was left for 12 hrs to enable the possible sedimentation, after which became transparent under the unaided eye. Then the dispersion was transferred into one 10 mm quartz cuvette (Thorlabs, W005654) for absorption measurement with another paired cuvette containing ethanol only.


\section{Discussion on molybdenum oxides}\label{sec:result}

In vapor synthesis process, two growth mechanism exists: VS and VLS. VS process is widely accepted for the growth of \ce{MoO3}. Yet we caution that synthesis conditions should be scrutinized to determine the exact mechanism. \citeauthor{Li2002c} suggested a VS mechanism at 700 \si{\degreeCelsius} and VLS at 750 \si{\degreeCelsius} and higher.\cite{Li2002c} \citeauthor{Fibers2007} proposed a modified VS mechanism probably because the deposition occurs on \ce{Al2SiO5} with possible \ce{Al_{0.95}SiNa_{0.06}O_x} involved. Therefore temperature and possible impurity could potentially alter the growth mechanism. We divide the growth results into two categories: on Si substrate and on non-Si substrate, and describe them respectively. We also briefly mention using liquid exfoliation to prepare few layer \ce{MoO3}.

\subsection{Growth on Si substrates}

Mo began to be oxidized into \ce{MoO3} at 500 \si{\degreeCelsius}, \ce{MoO3} vapor pressure began to increase rapidly above 700 \si{\degreeCelsius}. \cite{Margrave1967} And the melting point of \ce{MoO3} is reported as 795 \si{\degreeCelsius}. With these facts in mind, we design the experimental parameters as listed in Table.~\ref{tab:mooxsi}. We should point out that these parameters are selected and optimized from a wide range of combinations. And when we discuss the influence of one parameter with different values, one should assume other parameters still adapt those listed in Table.~\ref{tab:mooxsi}.

\begin{table}[htb]
\centering
\caption{Growth conditions of \ce{MoO3} on Si}\label{tab:mooxsi}
\begin{tabular}{lcccr}
\toprule
&&&\multicolumn{2}{c}{Flow (sccm)} \\
\cmidrule(l){4-5}
 & Temperature $T_h$ (\si{\degreeCelsius}) & Pressure  & Ar & \ce{O2}  \\
\midrule
typical values  & 800    & 200mTorr & 10 & 10  \\
\end{tabular}
\end{table}

The growth layout is schematically shown in Fig.~\ref{fig:mooxgrowth}. Notice that on the horizontal axis of temperature, zero inch is defined at the upstream edge of furnace. To facilitate discussion and provide a tight context, we define the position of substrate as the coordinate of its upper-stream edge along this temperature axis (i.e. the substrate location is 7 inch in Fig.~\ref{fig:mooxgrowth}). The usual length of Si substrate is 1 inch. When positioned at 7 inch, it is then located in a temperature range of $650 \sim 350$ \si{\degreeCelsius}.

\begin{figure}[htb]
\centering
\includegraphics[width=0.8\textwidth]{CVD_and_temp_MoO3.jpg}
\caption[Growth setup of \ce{MoO3}]{Chemical vapor system and its temperature profile. The triangular labels $\blacktriangledown$ were measured points at ambient pressure. The nominal substrates temperature were estimated from interpolation data.}
\label{fig:mooxgrowth}
\end{figure}

From previous works,\cite{predeep2011} we found that with $T_h = 600$ and substrate placed at 6.5 inches, microflakes was found at the high temperature edge of Si, suggesting the nucleation already started.\footnote{020510 batch} When flow of oxygen was gradually reduced, diminished size of micro-flake was observed at the same location of substrates. This results indicated the rate-determining step is diffusion when oxygen flow was in the range of 2 sccm to 0.5 sccm.\footnote{(31210-31510 batches)} We also found the deposition amount on a particular location of the substrate is not linearly proportional to the growth time, i.e., the growth at upper stream edge first increase then decrease. This is not surprising when considering the temperature field requires some time to become synchronized as depicted in Fig.~\ref{fig:mooxgrowth}. Using conditions in Table.~\ref{tab:mooxsi} and 1 hr growth, we observed the typical micro-flakes morphologies as shown in Fig.~\ref{fig:mosisem}. Most flakes exhibit rectangular shape, with average thickness of one micron, standard deviation $\sigma_D=0.34\mu m$.

\begin{figure}[htb]
\centering
\subfloat[]{\label{fig:mosem1}\includegraphics[width=0.45\textwidth]{mosemsi_a}}\hspace{0.04\textwidth}
\subfloat[]{\label{fig:mosem2}\includegraphics[width=0.45\textwidth]{mosemsi_b}}
\caption[Representative morphologies of \ce{MoO3} on Si]{SEM images of representative depositions of \ce{MoO3} on Si. (a)Low magnification. (b) High magnification. }
\label{fig:mosisem}
\end{figure}

This rectangular shape implies the boundary plane along growth direction (long axis) is (001), in consistency with previous experimental reports\cite{Zeng1998,Li2002b} and theoretical studies.\cite{Firment1983,Cora1997} We also observed different shapes, i.e., elongated hexagonal using similar growth conditions. This is not an indication of different growth mode. It is a normal thermodynamic fluctuation. The growth rates along different crystalline direction of \ce{MoO3} are determined by the free surface energy. In fact, (201), (101) and (102) planes have all been observed as terminating planes.\cite{Zeng1998} The stacking rate of \ce{MoO6} octahedra along $a$ and $c$ axis could develop some other ratio. And the coexistence of different planes in one growth suggests the similarity of free surface energies between these surfaces. In other words, the migration barrier of adatoms on (010) plane is presumably much lower than that on other low index planes due to the Van der Waals interaction nature along [010] direction.

\begin{figure}[htb]
\centering
\subfloat[]{\label{fig:moxrd}\includegraphics[width=0.45\textwidth]{xrd_moo3}}\hspace{0.04\textwidth}
\subfloat[]{\label{fig:moram}\includegraphics[width=0.45\textwidth]{raman_moo3}}
\caption[Crystalline phase characterization of \ce{MoO3} on Si]{(a) XRD pattern and (b) Raman spectrum of typical \ce{MoO3} on Si, $\lambda_{ex} = 532nm$.}
\label{fig:mooxch}
\end{figure}

As shown in Fig.~\ref{fig:mooxch}, the crystal structures and phase of as-synthesized samples were examined by XRD and Raman. The XRD pattern (Fig.~\ref{fig:moxrd}) is readily indexed to the orthorhombic phase of \ce{MoO3} (ICDD PDF 05-0508, \emph{a}=3.9628\AA, \emph{b}=13.855\AA, \emph{c}=3.6964\AA). The space group is $D_{2h}^{16}(Pbnm)$. This crystal structure is indeed a unique example among transition metal oxides, representing a transitional stage between tetrahedra and octahedral coordination.\cite{Itoh2001a} The strongest peak index is (110), suggesting the orientation of \ce{MoO3} is not parallel to the substrate, which is consistent with the morphology displayed in Fig.~\ref{fig:mosem1}. In contrast, we shall see a different scenario in Section.~\ref{sec:nasi}, where the orientation is changed.

\begin{table}[htb]
\centering
\caption{Experimental Raman peaks assignment of \ce{MoO3} on Si.\cite{Eda1992,Siciliano2009}}\label{tab:moram}
\begin{tabular}{llcll}
\toprule
this work(\si{cm^{-1}}) & Sym.       &          & Assignment &   \\
\midrule
117      & $B_{2g}$    &           & $T_c$  & RCM  \\
129      & $B_{3g}$    &           & $T_c$  & RCM  \\
158      & $A_g/B_{1g}$&           & $T_b$  & RCM  \\
199      & $B_{2g}$    & $\tau$    & \ce{O=Mo=O}  & twist  \\
218      & $A_g$       &           & $R_c$     & RCM  \\
247      & $B_{3g}$    & $\tau$    & \ce{O=Mo=O}  & twist  \\
284      & $B_{2g}$    &           & \ce{O=Mo=O}  & wag  \\
292      & $B_{3g}$     & $\delta$ & \ce{O=Mo=O}  & wag  \\
337      & $A_g,B_{1g}$ & $\delta$ & \ce{O-Mo-O} & bend  \\
380      & $B_{1g}$     & $\delta$ & \ce{O-Mo-O}  & scissor  \\
474      & $A_g$        & $\nu_{as}$ & \ce{O-Mo-O}  & stretch,bend  \\
667      & $B_{2g},B_{3g}$ & $\nu_{as}$  & \ce{O-Mo-O}  & stretch  \\
819      & $A_g$        & $\nu_{as}$  & \ce{O=Mo}  & stretch  \\
996      & $A_g$         & $\nu_{as}$  & \ce{O=Mo}  & stretch  \\
\bottomrule
\end{tabular}
\end{table}

The Raman spectrum of as-synthesized specimens also closely matches \ce{MoO3} features in previous studies.\cite{Dixit1986,Silveira2012} During m-Raman measurement, the laser spot was carefully focused onto the plates and several inspections on different positions were observed to ensure the reproductivity of spectra data. As shown in Fig.~\ref{fig:moram}, 14 distinct bands were well resolved. The 284 \si{cm^{-1}} peak represents the wagging mode for double bond \ce{O=Mo=O}. The 337 and 380  \si{cm^{-1}} peaks are assigned to \ce{O-Mo-O} bending and scissoring modes. The 199 \si{cm^{-1}} peak and two other weaker peaks at 218 and 247 \si{cm^{-1}} represent \ce{O=Mo=O} $B_{2g}$ twist, $A_g$ chain mode and \ce{O=Mo=O} $B_{3g}$ twist mode, respectively. 667 \si{cm^{-1}} peak is assigned to triply coordinated oxygen stretching model resulting from edge-shared oxygen in common to three octahedral. The 819 \si{cm^{-1}} peak is from doubly coordinated oxygen stretching mode arising from corner-shared oxygen between two octahedral. The 996 \si{cm^{-1}} peak comes from unshared oxygen stretching mode.\cite{Siciliano2009} The assignment is summarized in Table.~\ref{tab:moram}.


\subsection{Liquid exfoliation on as-grown molybdenum oxides}

Few layer (FL) materials attracts intensive research efforts recently. We studied using liquid exfoliation method to prepare FL \ce{MoO3}. We then characterized the absorption feature and crystalline structures of the exfoliated \ce{MoO3} using UV-Vis and TEM. Then we calculate the optical contrast of FL \ce{MoO3} on \ce{SiO2}-Si, which serves as a guideline to identify FL \ce{MoO3} using optical microscope.

The exfoliation solution is 30\% isopropanol (IPA) in DI water, which has been proved to work well on other layered compounds.\cite{Halim2013} After 20 minutes sonication, the dispersion is milky. We measured the transmission spectra of this dispersion immediately upon sonication and after 40 hrs for gravity sedimentation, as illustrated in Fig.~\ref{fig:moabs}. The t0 line mainly arises from the scattering of flakes in the dispersion. There are two scattering mechanisms: Rayleigh scattering and Tyndall scattering. Rayleigh scattering, which occurs when particles are comparable to wavelength, is inversely proportional to the fourth power of wavelength; While Tyndall scattering, which occurs when the particles are larger, is inversely proportional to the square of the wavelength. To fully evaluate the true absorbance of the sample itself, one needs to decouple the scattering part from apparent absorbance. Generally we select one part of spectrum and assume it is only caused by scattering. Then a empirical dependence shown in Eq.~\ref{eq:sca2} is used as polynomial fitting model.

\begin{align}
Abs_{sca}  & = a\times \lambda^{n}  \label{eq:sca1}\\
\log{Abs_{sca}} & = \log{a} + n*\log{\lambda} \label{eq:sca2}
\end{align}

After least-squares fitting, the coefficients $a$ and $n$ can be used to estimated the scattering in other wavelengths. The fitting of $n$ usually falls between -2 and -4. We examined the t40h line in Fig.~\ref{fig:moabs} and found the scattering part is almost zero. Therefore we did not perform aforementioned fitting. This is not always the case. In Chapter on the \ce{WS2} section, we do need to do so.

\begin{figure}[htb]
\centering
\subfloat[]{\label{fig:moabs}\includegraphics[width=0.45\textwidth]{MoO3-abs}}\hspace{0.04\textwidth}
\subfloat[]{\label{fig:mobg}\includegraphics[width=0.45\textwidth]{MoO3-bg}}
\caption[UV-Vis spectra of exfoliated \ce{MoO3}]{(a) UV-Vis absorbance spectra of exfoliated \ce{MoO3}. Dashed line (t0) represents measurement immediately after sonication, and solid line (t40h) is obtained after sitting for 40 hours. The dot dashed line is reference spectrum of the 30\% IPA. (b) Band gap estimation}
\label{fig:mouv}
\end{figure}

According to the corrected absorbance spectrum, optical band gap can be deduced, which is known as Tauc Plot.\cite{Tauc1972} The procedure is recapitulated here. Eq.~\ref{eq:tauc} is given to empirically relate optical band gap to absorption.

\begin{equation}\label{eq:tauc}
 (h\nu \alpha)^{\frac{1}{n}} = B(h\nu - E_g^{opt}),
\end{equation}
where B is a proportional coefficient, $\alpha$ is absorption\footnote{The absorbance $A = 0.4343\alpha*l$, where l is the optical path} and $E_g^{opt}$ is defined as optical band gap. The value n is correlated to the nature of optical transition.
\begin{itemize}
\item For direct allowed transition, n = 1/2;
\item For direct forbidden transition, n = 3/2;
\item For indirect allowed transition, n = 2;
\item And for indirect forbidden transition, n = 3.
\end{itemize}

When the transition nature of sample is not well known, one should apply each to check which one provide the best fit. Depending on the synthesis methods and \ce{MoO3} morphologies, the transition phenomena might vary. On the \ce{MoO3} films, \citeauthor{Bouzidi2003} assumed a direct gap (n = 1/2)\cite{Bouzidi2003} while \citeauthor{Szekeres2002} not.\cite{Szekeres2002} We use n = 2 in this work, plot $\sqrt{(h\nu \alpha)}$ versus photon energy, and extrapolate the linear part onto $h\nu$ axis. The intercept is used as $E_g^{opt}$ then. Our evaluation and some previous studies is summarized in Table.~\ref{tab:mobg}.

\begin{table}[htb]
\centering
\caption{Optical band gaps of \ce{MoO3}}\label{tab:mobg}
\begin{tabular}{lccr}
\toprule
&\multicolumn{2}{c}{Band gap} \\
\cmidrule(l){2-3}
Reference & value(eV) & orientation & material states\\
\midrule
\cite{Deb1968}   & 2.96  & $E\parallel c_0$ & single crystal\\
\cite{Deb1968}   & 2.80  & $E\perp c_0$ & single crystal \\
\cite{Julien1995} & 2.8$\sim$ 3.2 & NA & films\\
 this work  & 3.1  & NA & nanoflakes\\
\bottomrule
\end{tabular}
\end{table}

An alternative apporach is a) define $Y = \alpha E=\alpha h\nu$, and obtain $Y' = \frac{\partial Y}{\partial E}$, then b) plot $Y'/Y$ versus $h\nu$. The band gap value can be obtained from the intercept of extrapolated line on photo energy axis.\cite{Choopun} This approach is not difficult to comprehend as long as we notice from Eq.~\ref{eq:tauc}, $Y = B (h\nu - E_g^{opt})^n$ and $Y' = B n(h\nu - E_g^{opt})^{n-1}$. The advantage is one can also estimate n without using assumption.

The dielectric constants\footnote{Not a constant at all, instead, it has complicated dependence on photon energy} ($\epsilon = \epsilon_1 + i*\epsilon_2$) of \ce{MoO3} have been studied in several reports.\cite{Deb1968,Sabhapathi1995,Miyata1996,Abdellaoui1997,Mondragon1999} We extracted $\epsilon_1$ from Ref\cite{Itoh2001a} due to the wide photon energy range utilized. We also notice that $\epsilon_2$ is approximately zero when $h\nu$ is less than 4 eV. Since we are primarily concerned with the optical contrast of FL \ce{MoO3} in visible wavelength region($h\nu < 3$ eV), we obtained refractive indices $n$ from $n = \sqrt{\epsilon_1}$. Due to the orthorhombic phase, \ce{MoO3} exhibit birefringence, that is there are two refractive indices, $n_c$ and $n_a$, as shown in Fig.~\ref{fig:moind}. The refractive index is between $ 2.2\sim 2.4$ along a axis, and $ 2.5\sim 2.8$ along c axis. Comparison to the low frequency dielectric constant ($18.0\pm1$) implies a primarily ionic bonding in \ce{MoO3}.\cite{He2003}

\begin{figure}[htb]
\centering
\subfloat[]{\label{fig:moind}\includegraphics[width=0.45\textwidth]{n_MoO3}}\hspace{0.04\textwidth}
\subfloat[]{\label{fig:mocon}\includegraphics[width=0.45\textwidth]{MoO3_1L}}
\caption[Refractive indices of \ce{MoO3}]{(a) Refractive indices of \ce{MoO3} Dashed line is n along $a$ axis while solid line is along $c$ axis. (b) optical contrast mapping of 1L \ce{MoO3} on \ce{SiO2}-Si. The x axis is \ce{SiO2} thickness, y axis is wavelength.It is assumed that the incident light is polarized along c axis of \ce{MoO3}.}
\label{fig:mofl}
\end{figure}

Fig.~\ref{fig:mocon} illustrates the optical contrast mapping of 1L \ce{MoO3} on \ce{SiO2}-Si when the incident light is polarized along c axis of \ce{MoO3}. One can see that there is a pseudo-periodical trend with either wavelength or \ce{SiO2} thickness fixed. The positive extreme occur at around 80nm and 240nm \ce{SiO2} layer, slightly less than the optimum for observing graphene. The refractive indices of Si and \ce{SiO2} within visible wavelengths and MATLAB source code for optical contrast calculation is included in Appendix, with which one can readily evaluate the few layers \ce{MoO3} scenarios and the circumstance when incident light is polarized along a axis.

The dimension and phase of exfoliated \ce{MoO3} is examined using TEM. Fig.~\ref{fig:motem} shows the size of exfoliated \ce{MoO3} is about 500 nm. High-resolution TEM (HRTEM) image was taken to assess the \ce{MoO3} crystallinity. The clearly resolved lattices indicate high crystal quality of these exfoliated \ce{MoO3}.

\begin{figure}[htb]
\centering
\includegraphics[width=0.8\textwidth]{exf_moo3.jpg}
\caption[TEM images of exfoliated \ce{MoO3}]{(a) TEM image and (b) HRTEM image of FL \ce{MoO3} obtained by liquid sonication method.}
\label{fig:motem}
\end{figure}

Nevertheless, identifying these few layer \ce{MoO3} under optical microscope is still difficult due to both the small flake size and low contrast.\footnote{Human eye threshold contrast (contrast sensitivity)$^{-1}$ is about 10\%.} One need to improve the liquid exfoliation process to obtain larger dimension and high yield. This is however beyond the scope of current study.

\subsection{Growth on non-Si substrates}\label{sec:nasi}

In previous works, \ce{MoO3} deposition on ITO/glass exhibits new morphologies that were not observed on Si substrates, yet the mechanism is not well investigated.\cite{predeep2011} We first repeated the previous growth on ITO/glass, and then found similar deposition phenomena on glass and mica substrates. We suspected that the morphology difference on the substrates employed in this study arises from the compositions difference. Both mica and glass contains alkaline elements which are absent in Si and \ce{SiO2}/Si substrates. We primarily verify this hypothesis by utilizing \ce{NaOH} treated Si substrates and found that the overall morphology almost reproduces that on ITO and mica substrates. We then propose a new growth mechanism based on the VLS process to explain the deposition of \ce{MoO3} on alkaline-ions-containing substrates.

ITO versus NaOH images.


\subsubsection{Na transport}
We estimated the mass transport of sodium ions in silica glass. The driving force could be external field or concentration gradient. And the dynamics is governed by diffusion equation

\begin{align}
\frac{\partial C}{\partial t} = D \frac{\partial^2 C}{\partial x^2},
\end{align}

where C is concentration in unit of \si{mol\per cm^3}, and D is diffusion coefficient in unit of \si{cm^2\per\second}. Diffusion coefficient highly depends on the overall environments in which the ions resides. Typical values of D for Na ions are listed in Table.~\ref{tab:mona}

\begin{table}[htb]
\centering
\caption{Na diffusion coefficient ($cm^2/sec$), R is molar gas constant (8.315 \si{\joule\per mol\per K}), and T is temperature in K}\label{tab:mona}
\begin{tabular}{lcr}
\toprule
 Composition & Value  & Reference  \\
\midrule
Quartz      & $3.8\times10^{-2}exp(\frac{-24500}{RT})$  & \cite{Rybach1967a}  \\
Sodalite      & $6.6exp(\frac{-42500}{RT})$  & \cite{Sippel1963}  \\
Obsidian     & $4.4\times10^{-2}exp(\frac{-22900}{RT})$  & \cite{Sippel1963}  \\
silicate glass & $3.1\times10^{-8}$ at 420\si{\degreeCelsius} & \cite{Jbara1995} \\
\ce{SiO2} glass & $1.1\times10^{-7}$ at 670\si{\degreeCelsius} &  \cite{FRISCHAT1968}\\
\bottomrule
\end{tabular}
\end{table}

We also would like to mention the gradient of Gibbs free enthalpy overruled the concentration of a particular species, and higher \ce{OH^-} contents tend to reduce Na diffusivity.\cite{Materials2012}  The substrates used in this work are glass and ITO/glass. Related parameters is as following:
\begin{itemize}
\item dimension: $25mm\times10mm\times1mm$
\item density: 2.567 \si{g\per cm^3}
\item Na concentration: 0.01 \si{mol\per cm^3}
\item D: $1.1\times10^{-7}$ \si{cm^2\per\second}
\end{itemize}

Then the net flux $J$ is estimated as
\begin{align}
J = -D \frac{\Delta C}{\Delta x} = 2.0e^{-8} mol/cm^2/sec,
\end{align}

for a typical 2hrs growth, the net amount of Na diffusing out of substrate is then 144$\mu mol$, about 10\% of total Na ions. The diffusion length $\sqrt{Dt}$ is 190 $\mu m$. We noticed this result overestimate since the flux will reduce gradually. Meanwhile the molar amount of \ce{NaOH} applied on Si substrate is about 1$\mu mol$. Considering the adjusted ratio, Na contents should be much more when using glass substrate than when using NaOH-Si substrate.

Another important factor is the evaporation of sodium molybdates. However, few report could be found in literature. We only find one by \citeauthor{Kazenas2010}.\cite{Kazenas2010} The partial pressure\footnote{$logP(atm)= -13794/T + 6.19$} of \ce{Na2MoO4} is calculated to be 0.2 mTorr at 800 \si{\degreeCelsius} and $3.5\times 10^{-3}$ mTorr at 670 \si{\degreeCelsius}.

\textbf{tower-like structure}

Nanotower is a less common morphology in the structures of nanomaterials.\cite{Kharissova2010} We have not found report on  \ce{MoO3} nanotower structures, although there are several studies on \ce{In2O3},\cite{Jean2010,Yan2007}, GaN\cite{Xiao2012} and ZnO\cite{Zhang2013c}. \citeauthor{Jean2010} investigated the growth mechanism of \ce{In2O3} nanotowers and attributed it to the periodical axial and continuous lateral growth interaction, where the former was first guided by Au-In alloy liquid, and subsequent by self-catalytic In droplets (MP 156C) while the latter due to VS process. In contrast, \citeauthor{Yan2007} observed similar \ce{In2O3} nanotower growth and proposed that the axial growth is controlled by Au-catalytic VLS process. \citeauthor{Xiao2012} reported the GaN nanotower grown on Ni-coated Si(111) substrates and explained the growth as asymmetric self-copy process based on VLS mechanism. \citeauthor{Zhang2013c} studied the growth of ZnO nanotower and provided a competitive model between axial and lateral growth controlled by Zn vapor to explain the as-synthesized structures. (When ZnO and carbon mixture was used as source instead of ZnO alone, nanorods array results, presumably due to the stable supply of Zn vapor owing to the gentle carbothermal reduction.) Therefore the fluctuation of reagents is probably responsible for tapering and periodical modification, which is absent in the coin roll style growth of \ce{MoO3} tower on Si. The occasionally observed tapering or enlarging on \ce{MoO3} tower can be attributed to the cooling down growth.  An oxygen regulated growth will be conducted to reveal the role of vapor pressure during the formation of tower.

long belts  is also found on Mica(\ce{K(Al2)(Si3Al)O10(OH)2}), the formation probably arise from Potassium-catalyzed VLS mechanism.\cite{Hu2011}

\subsubsection{time stepwise growth}

With time stepwise growth, the following results were obtained:

\begin{enumerate}
\item When no Mo powder is used, NaOH still hold onto substrate with some locations exhibiting etching.
\item When no Mo powder is used while growth is performed in an old tube or oxygen flow is limited to a extreme small value, circular droplet-like solid were found and Mo, Na and O were detected, suggesting the existence of VLS mechanism. Using Moly coated substrate and subsequent NaOH treatment, similar droplets formation was repeated. No discernible XRD pattern emerged, illustrating the amorphous nature of liquid catalyst.
\item 1mins growth shows plates in high temp end and irregular shapes in low temp end.
\item 15mins growth shows \ce{MoO3} plates parallel to substrate, with some belts emerging and growing out of the substrate plane. This is presumably due to the growth direction mismatch between different plates. Morphology will also depend on local NaOH concentration. NaOH trace mark turns into connected plates at the original locations, showing the high concentration mediated the morphology evolution. The possible catalyst solid was sitting on the top of plate layer. Also some belt-like growth is visible on reaction chamber walls, suggesting the evaporation and transport of \ce{NaxMoO3} vapor phase;
\item 30mins growth:  belts growth begin (500 microns long, grow rate is estimated to be 0.5 $\mu m/s$), tower structures show up as well, but much less than belt. Two kinds of belts found, tapering and non-tapering. The former one seems to have droplets mostly associated with sidewalls of the top part.
\item In reduced time or less oxygen supply growth, Na-containing site was found located onto the sidewall of belts instead of the usual expected tip. XRD pattern indicated the existence of \ce{NaxMoO3} phase. And MoO3 XRD pattern is dominated by (0k0) peaks, indicating a 2D layer-by-layer growth with lateral rate exceeding that of axial direction.
\item 2hr growth: there are crossed tapering belts at downstream boundary, with more than one droplet located along the interface of two perpendicular belts, EDX confirmed high Na content.
\item The tower is paled along $\langle010\rangle$ direction, and the belt growth direction is $\langle001\rangle$.
\item Using NaOH treated hydrophobic substrates, dense tower array growth was found along the edge areas or molten periphery of droplets, where 5 mins growth already show the initial stage of tower with diameter about 1 $\mu m$, which increase to 10 $\mu m$ in prolonged growth.
\end{enumerate}

We divide the growth into two stages according to the emerging time of long belts and towers. In first stage the growth is dominated by lateral growth, as illustrated in . In the second stage, \ce{NaxMoO3} is evaporated and recondensed in the low temperature region, and some droplet will form, and when at proper locations, introduce secondary VLS growth.

In first stage, the Na-Mo-O liquid is formed and have much more higher absorption rate than that of bare substrate. The absorbed species will diffuse along the surface and inside the liquid. It was difficult to estimate which diffusion path one is faster. When the size of liquid is smaller than some critical size, there is no preferred growth direction. This changed when the precipitated solid exceeds certain critical dimension, and the morphology evolution is then determined by the relative size of already formed solid and its liquid catalyst.


The overall reactions probably occur in the following sequences.
First, Mo powder is oxidized and \ce{MoO3} vapor is produced and transported to downstream by Ar carrier gas. Then,
\begin{align}
\cee{NaOH(l) + MoO3(g) -> Na2MoO4(l) + H2O(g)}
\end{align}

The melting point of \ce{Na2MoO4} is 687 \si{\degreeCelsius}. Based on the open air temperature profile (Fig.~\ref{fig:mooxgrowth}), the substrate is locate in temperature zone between 650 and 400 \si{\degreeCelsius}. The temperature distribution is different due to both the sealed condition and thermal conduction along the Si substrate. It will be difficult to acquire the accurate values along the substrate. However by considering the flow rate, the actual temperature will be higher than predicted by open air measurement.\cite{Subannajui2010} The melting of \ce{Na2MoO4} is highly possible when taking into account of its size as well. \cite{Bruggemann1997}

\begin{figure}[htb]
\centering
\includegraphics[width=0.6\textwidth]{Na2MoO4_Phase}
\caption[Phase diagram of Na-Mo-O system]{Phase diagram of \ce{Na2MoO4} and \ce{MoO3} reproduced from ref.~\cite{Hoermann1929}}
\label{fig:pd}
\end{figure}

According to the phase diagram shown in Fig.~\ref{fig:pd}, \ce{Na2MoO4} - \ce{MoO3} phase above 800 \si{\degreeCelsius} are \ce{Na2MoO4}(l) + \ce{MoO3}(l), More discussion should follow $\ldots$.

Each site of \ce{Na2MoO4} will continue absorb incoming \ce{MoO3}, forming a swelling droplet. \ce{MoO3} will precipitate from these seeds when a critical supersaturation concentration is reached, which is about 75\% according to Fig.~\ref{fig:pd}.\footnote{This ratio considerably exceeds that of the usual metal catalyst scenario. This high solubility accounts for several observations in our experiments: the dimensions of individual deposit is much larger than that of initial droplet, the concentration of sodium in final products is extremely low.}

As shown in Fig, we found several solid clusters with diameter about 1 $\mu m$ on the top part of long belts. According to the \ce{Na2MoO4}-\ce{MoO3} binary compounds phase diagram, we expect the terminating solid clusters have a Na:Mo ratio close to 0.6. The experimental results match well with this predictions. EDX analysis reveal the presence of Na on the solid cluster but not at about 1$\mu m$ away from previous location.\footnote{the sensitivity of our instrument is estimated to be 1\% atomic level} Quantitative atomic ratio derived from the solid cluster is not close to the one predicted by phase diagram probably due to the high Mo content in adjacent area. However, XRD examinations on a short time growth sample revealed a possible \ce{Na2Mo4O13} phase(PDF 028-1112), providing an indirection evidence of the droplet composition.

We also suspect that the growth will become different below 500 \si{\degreeCelsius} since no liquid would remain. Substrate placed at low temp end show.

We can also deduce that other sodium compounds can be employed as substrate treatment agent. This argument is confirmed by \ce{Na2CO3} and \ce{KI} assisted growth.


\textbf{open questions}
\begin{itemize}
\item Is the liquid consumed? if so, how? : evaporation
\item How the size of liquid is related to the morphology evolution?
\item Possible doping level of alkaline ions
\item other lateral growth mode such as Ni-W-S system
\item VLS-VS interaction, or possible solution-liquid-solid, VSS,
\item the interpretation of phase diagram
\item the overall molar amount of NaOH and MoO3 deposition
\item Young's parameters, contact angle informations, Gibbs free energy calculation
\item catalyst composition variation during growth
\item \ce{Na2Mo4O_{13}} phases, solid solubility of \ce{Na2MoO4} in solid \ce{MoO3} is high.
\item vapor pressure of \ce{Na2Mo4O_{13}} and \ce{MoO3}.
\end{itemize}

\subsubsection{Phase evolution probing by Raman}

\citeauthor{Hardcastle1990} summarized an empirical formula to relate the Raman peaks and \ce{Mo-O} bonding lengths.\cite{Hardcastle1990} This correlation assumes general form as
\begin{equation}\label{eq:mobond}
\nu = A \exp{B\cdot R},
\end{equation}
where $A=32895$ and $B=-2.073$ are fitting parameters, R is bond distance in unit of \AA. Given a stretching frequency, the resolution for calculated bond distance is $\pm0.016$\AA. Another empirical expression connect the bond valence $s$ and bond distance R: $s(M-O) \approx (R/X)^{-6} $, where X=1.882 when M is Mo, and 1.904 when M is W. The valence sum rule could be then used to check the state of Mo cation. It should be noticed that not all observed Raman lines could be correlated to a \ce{Mo-O} bond distance by extrapolation of Eq.~\ref{eq:mobond}. It is then regarded as a symmetry related vibrational mode, i.e. $820 cm^{-1}$ in \ce{MoO3}. From the correlation of various Mo compounds, a general conclusion is the lower the stretching frequency for the shortest metal-oxygen bond, the more regular is the structure.


\textbf{Fundamental of VLS}

Vapor-Liquid-Solid (VLS) process is first proposed by Wagner,\cite{Wagner1964} and further developed by \citeauthor{Givargizov1975}.\cite{Givargizov1975} This method has become an important strategy for synthesizing one-dimensional nanostructures.\cite{Lieber1998} Conventionally, a liquid eutectic droplet is formed by catalyst itself or by alloying with the growth material, acting as a trap of growth species. The growth is initialized by supersaturation of the liquid alloy and subsequent precipitation at the solid-liquid interface. The choice of catalyst usually is among the several noble metals since they are physically active but chemically stable or inert in most growth scenarios. The growth materials span a wide range including group IV,\cite{Hochbaum2005} group III-V,\citep{Dalacu2013, Xiao2012, Dubrovskii2011} group II-VI,\cite{Hao2006} and some metal oxides, such as \ce{ZnO},\citep{Huang2001a,Ramgir2010} \ce{MgO},\citep{HEUER1967, Nagashima2007} \ce{SiO2},\cite{Pan2002} and \ce{TiO2},\cite{Zhuge2012} to name a few. The catalysts can be Au, Pd, Pt, Ni,\cite{Xiao2012} Ti, Ga,\cite{Pan2002} and even KI.

Understanding the interaction between liquid droplet and the solid interface allows for a rich engineering space to fine tune the geometry and structures of as-grown nanostructures. For instances, the diameter of Si nanowires (NWs) can be controlled by laser ablated catalyst\cite{Morales1998} or well-defined Au nanoparticles.\cite{Cui2001b} Under proper conditions,  axial modulation can produce nanowire superlattices from group IV and III-V materials.\citep{Gudiksen2002,Bjork2002}  Radial composition modulation have also been demonstrated by selectively suppressing VLS process, providing a robust routine for homogeneous or heterogeneous core-shell structures.\cite{Lauhon2002a}  \citeauthor{Wang2013c} changed the NW growth direction by changing the shape of droplets; \citeauthor{Biswas2013} reported enhanced aspect ratio of Ge NWs from bimetallic alloy.


\subsection{glass growth}


We studied the effect of temperature, oxygen partial pressure and growth time towards \ce{MoO3} deposition on glass substrates. Each factor is divided into two levels, with T at 700 and 800, \ce{O2} at 1 and 10 sccm, and growth time of 15 and 60 mins. We will assign $+$ and $-$ to higher and lower levels, respectively. And the morphological variation is summarized in Table~\ref{tab:mo3glass}.
% glass growth matrix
\begin{table}[htb]
\centering
\caption{\ce{MoO3} growth on glass}\label{tab:mo3glass}
\begin{tabular}{lccp{3in}}
\toprule
\multicolumn{3}{c}{Growth Conditions} \\
\cmidrule(l){1-3}
 Temperature & \ce{O2} & Time & Morphology  \\
\midrule
700    &  10   & 15  &   about 50 $\mu m$ belts, some towers \\
700   &  1   & 60  &   irregular shape, not too much growth\\
700    &  10   & 60  &   droplet almost all on tip of belts, many towers \\
800    &  1   & 15  &   tens $\mu m$ belt with sharp edge(100 nm thick, and 1micro wide) dominates\\
800   &  1   & 60  &   exhibit ITO features, droplets on tips of belt, many tower like structures\\
800    &  10   & 15  &   rounded edge flakes dominate, modest density tapering belt, and some possible tower\\
800    &  10   & 60  &   ITO/glass features, aggregated belts, some towers\\
\bottomrule
\end{tabular}
\end{table}

% droplet position statistics
\begin{figure}[htb]
\centering
\includegraphics[width=0.6\textwidth]{droplets_sta}
\caption[Droplet position statistics]{Droplet position statistics. On-tip and not-on-tip counts variation with respect to growth time. }
\label{fig:mo3dropsta}
\end{figure}


