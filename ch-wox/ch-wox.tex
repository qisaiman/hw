%\providecommand{\setflag}{\newif \ifwhole \wholefalse}
\setflag
\ifwhole\else

    \documentclass[12pt,letterpaper,oneside]{book}

    %-------page layout--------%
% adapted from <http://www.khirevich.com/latex/page_layout/>
%\usepackage[DIV=14,BCOR=2mm,headinclude=true,footinclude=false]{typearea}

%\makeatletter
%\if@twoside % commands below work only for twoside option of \documentclass
%    \newlength{\textblockoffset}
%    \setlength{\textblockoffset}{12mm}
%    \addtolength{\hoffset}{\textblockoffset}
%    \addtolength{\evensidemargin}{-2.0\textblockoffset}
%\fi
%\makeatother

% packages used in uncc-thesis

%\RequirePackage{ifthen}
%\RequirePackage{setspace} % for double spacing
%\RequirePackage{comment}
%\RequirePackage{epsfig}
%\usepackage{sectsty} % for sectional header style. Alternative: titlesec package
% \usepackage{tocloft}
%\usepackage{geometry}

\usepackage{microtype} % better layout
%----inherent of article class-------%
%\usepackage[utf8]{inputenc} % set input encoding (not needed with XeLaTeX)

%--- for font ----
% \usepackage[T1]{fontenc}
% \usepackage{textcomp}

%\usepackage{mathptmx} % fine but not truetype
% \usepackage{newtxtext}
% \usepackage{pslatex} % not bad

\usepackage{fontspec} % to compile with LuaLatex
\setmainfont{Times New Roman} % to compile with LuaLatex

 %-- end of font adaption

\usepackage{graphicx} % support the \includegraphics command and options
% check pdftex option
\usepackage{placeins} %

\usepackage{subfig}
\usepackage{float} % for images
 \graphicspath{{./gallery/}} %--added by author

\usepackage[font=small, labelfont=bf]{caption}
%\usepackage{subcaption}


% It supplies a landscape environment, and anything inside is basically rotated.(http://en.wikibooks.org/wiki/LaTeX/Page_Layout)
%\usepackage{lscape}
\usepackage{pdflscape}
%\usepackage{rotating} % use \begin{sidewaystable}

% Helps format tables using the \toprule, \midrule, and \bottomrule commands (http://en.wikibooks.org/wiki/LaTeX/Tables#Using_booktabs)
\usepackage{booktabs}
%\usepackage{multirow} for multirow in tables
%  Helps format tables (http://en.wikibooks.org/wiki/LaTeX/Tables#Using_array)
%\usepackage{array}

%%-- Additional style---- modified by Tao Sheng 12/20/12
% for chemical formula, subscripts etc
\usepackage[version=3]{mhchem}
\usepackage{siunitx}
  \DeclareSIUnit \torr{Torr}
%%-- mathmatical symbols and equations-----
\usepackage{amsmath}
\usepackage{amssymb}
 %\numberwithin{equation}{section}
 %\numberwithin{figure}{section}
 \providecommand*{\ud}{\mathrm{d}}

%------- bibliography and citation ------
\usepackage[english]{babel}% Recommended
\usepackage{csquotes}% Recommended
\usepackage[style=numeric-comp,
		    sorting=nty,
            hyperref=true,
            url=false,
            isbn=false,
            backref=true,
            maxcitenames=2,
            maxbibnames=4,
            block=none,
            backend=bibtex,
            natbib=true]{biblatex}
% \usepackage[bibencoding=latin1]{biblatex}

\DefineBibliographyStrings{english}{%
    backrefpage  = {see p.}, % for single page number
    backrefpages = {see pp.} % for multiple page numbers
}
% suppress 'in:'
\renewbibmacro{in:}{%
  \ifentrytype{article}{}{\printtext{\bibstring{in}\intitlepunct}}}
% document preamble
% removes period at the very end of bibliographic record
\renewcommand{\finentrypunct}{}
% removes pagination (p./pp.) before page numbers
\DeclareFieldFormat{pages}{#1}


\providecommand*{\bibpath}{E:/spring2012/Ubuntu/Latex/Mendeley_Bib_lib}
\bibliography{\bibpath/arix.bib,\bibpath/ECD.bib,\bibpath/tungsten_newandgood.bib,\bibpath/ACSnano.bib,%
\bibpath/tungsten_old.bib,\bibpath/Raman.bib,\bibpath/Molybdenum.bib,%
\bibpath/tungsten_cl.bib,\bibpath/optics,\bibpath/CVD.bib,\bibpath/sodium.bib,\bibpath/VLS.bib}

%-- for works around, packages conflicts----

%-- redefine toc macros ------
\addto\captionsenglish{%
\renewcommand\chaptername{CHAPTER}%
\renewcommand\appendixname{APPENDIX}%
\renewcommand\indexname{INDEX}%
\renewcommand{\contentsname}{TABLE OF CONTENTS}%
\renewcommand{\listfigurename}{LIST OF FIGURES}%
\renewcommand{\listtablename}{LIST OF TABLES}%
}

%-- misc---
\usepackage{lipsum}
\usepackage{latexsym}
 \providecommand*{\thefootnote}{\fnsumbol{footnote}}

\usepackage{xcolor}
\usepackage{listings}
\lstset{
 frame = single,
 language = matlab,
 breaklines = true,
postbreak=\raisebox{0ex}[0ex][0ex]{\ensuremath{\color{red}\hookrightarrow\space}}
}

\usepackage{enumitem}
\setlist{nolistsep}

\setcounter{secnumdepth}{3} % show numbering of subsubsection

%% -- links ---
\usepackage{hyperref}
\hypersetup{
colorlinks,%
citecolor=black,%
filecolor=black,%
linkcolor=black,%
urlcolor=black
} % make all links black

\usepackage[acronym,toc,nonumberlist]{glossaries} % loaded after hyperref


    %\input{tweak.tex}
    %\input{commando.tex}
    %\input{font}

    \begin{document}

\fi  %  comment out when assembling

\chapter{IMPURITY EFFECTS ON TUNGSTEN OXIDE NANOWIRES}


\section{Introduction}

Tungsten oxide (\ce{WO_x}) is an important functional materials with distinctive properties and technology applications. Intense research interest was rekindled by the discovery of \gls{ec} effect in \citeyear{Granqvist1993}.\cite{Granqvist1993}  Nano-engineering \ce{WO_x} bring more possibility and flexibility to its already rich characteristics, therefore resulting research efforts spanning multiple fields of scientific community. Besides smart window based on EC effect, \ce{WO_x} is also well investigated for several other significant applications: photoelectrochemical cell for solar energy conversion and storage, photocatalysis for hydrogen evolution reaction, chemical and biological sensing based on gasochromic effect. In addition, a few young fields are not thoroughly explored: field emission, optical storage, thermoelectrcity, ferroelectricity, and superconductivity.

In this chapter, we will first review the literature to date for both \ce{WO_x} and \ce{Na_xWO3}, then divide the summaries into a) technological application, b) crystal structure and electronic properties and c) synthesis approaches. Then we present our growth methods and results. And we conclude with a brief summary.

\subsection{Rich and useful technological appeals}

Electrochromic effect of tungsten oxide means the coloration (deep blue) and bleaching (transparent) states of \ce{WO3} are reversibly switched upon forward and backward voltage, and the coloration or bleached states remain after disconnecting the voltage. On basis of this property, \gls{ecd} of tremendous energy saving potential is conceptualized and some products have already been commercialized (i.e. smart window\footnote{``We did a case study in five cities, and the average savings in commercial buildings are about 25 percent of the heating, ventilation, and air-conditioning energy use annually," says CEO of View, Inc.}).

In the past decades, tons of works have been devoted to understanding the chromogenic phenomena in \ce{WO3}. We highlight several key results as reviewed by ,
\begin{itemize}
    \item coloration and bleaching can also be stimulated by other routes, such as by UV irradiation, thermal treating, heating in vacuum, reducing atmosphere etc
    \item no electrical coloration occurs in vacuum, but other routes still works
    \item coloration is associated with a proportional increase in conductivity
    \item coloration spectrum is essentially similar in all cases except small variation in peak position and FWHM
    \item coloration is structure sensitive and most efficient in amorphous films
\end{itemize}

Sofar there is no unifying model that could reconcile these contradictory experimental observations. Among all the developed models, polaronic absorption is the most widely accepted mechanism for coloration. In polaron model\footnote{When a free electron travels through a polar solid, it creates a local lattice displacement (longitudinal optical phonon clouds) due to the coulombic interaction with neighboring ions. This local distortion and the electron together is equivalent to a new elementary exciton of the crystal, and is named as polaron}, the intercalation of \ce{M^+} (M = H, \ce{Li}, or Na) ions into \ce{WO3} films is accompanied with the formation of small polarons ($r_p$ comparable to unit cell size) and formal reduction of some \ce{W^{6+}} sites to \ce{W^{5+}}, as depicted in Eq.~\ref{eq:ec}. During the intercalation process, the \ce{M^+} ions enter into these vacant sites.\cite{Hepel2008} Coloration occurs when the polaron band .

\begin{align}\label{eq:ec}
x\ce{M+} + x\ce{e-} +  \ce{$\alpha \hyphen$WO$_{3-y}$}= \ce{$\alpha \hyphen$M$_x$WO$_{3-y}$},
\end{align}

And the polaron binding energy ($E_p$) is given by
\begin{align}
E_p = - \frac{e^2}{2r_p} (\epsilon_\infty^{-1} - \epsilon_{st}^{-1}),
\end{align}

where $\epsilon_\infty$ and $\epsilon_{st}$ are optical and static dielectric constants respectively, and polaron radius $r_p$, which specifies how far the lattice distortion extends, is related to polaron density $N_p$ by $r_p = \frac{1}{2}\sqrt[3]{\pi/6N_p}$. However, the polaron model has difficulty in estimating $r_p$. The asymmetric optical absorption spectrum are often characteristic of large polarons, and dielectric constants ($\epsilon_\infty = 6.52,\epsilon_{st} > 50$,\cite{Deb2008}) suggests the formation of bipolaron. Moreover, polaron model does not take oxygen vacancy into account, which plays a vital role in the nonstoichiometric tungsten oxides. For instance, it is observed that \ce{WO_{3-y}} films are metallic and conductive for $y > 0.5$, blue and conductive for $y = 0.3 \sim 0.5$, and transparent and resistive when $y < 0.3$, regardless of the preparation methods.\cite{Chatten2005}

Therefore another model in analogy to the F-color center is proposed. Color center model assumes the presence of oxygen vacancy $V_O^0$ is associated with \ce{W^{4+}} or 2\ce{W^{5+}} states. This defect level is expected to be inside or near the valence band. When one electron is removed from this level, $V_O^0$ is converted to $V_O^+$.The positively charged vacancy exerts coulombic repulsion to the nearest W-ions, which results in a displacement of the neighboring W-ions and an upward shift of the defect level into the bandgap, thereby creating a color center. The optical transition from $V_O^+$ to $V_O^{2+}$ (a state within the conduction band)
contributes to coloration.\cite{Deb2008}

So the polaron and color center models both agree on that \ce{W^{5+}} and its transition is responsible for the coloration, but disagree on how this \ce{W^{5+}} state is created (foreign ion reduction in polaron model and oxygen vacancy in color center model) and the corresponding energy levels. A modified polaron model is proposed to include \ce{W^{4+}} states in host lattice. Coloration mechanism is represented by Eq.~\ref{eq:cl_bl1} and \ref{eq:cl_bl2}, which described the polaron hopping from one site to another.\cite{Chatten2005}

\begin{align}
h\nu +\ce{W^{5+}(A)} +  \ce{W^{6+}(B)} &\rightarrow \ce{W^{5+}(B)} + \ce{W^{6+}(A) + E_{phonon}} \label{eq:cl_bl1}\\
h\nu +\ce{W^{5+}(A)} +  \ce{W^{4+}(B)} &\rightarrow \ce{W^{5+}(B)} + \ce{W^{4+}(A) + E_{phonon}} \label{eq:cl_bl2}
\end{align}

Similar scenarios occurs in the gasochromic effect. Two models, double injection and color center, arise to account for the coloration upon exposure to certain gases. Both consents to Eq.~\ref{eq:cl_bl1}. But there is a disagreement on the final states. Double injection supports the formation of tungsten bronze \ce{H_xWO3} while color center insists on the inward diffusion of oxygen vacancy and outward diffusion of water molecules. Both have been substantiated experimentally. Therefore the exact mechanism in \ce{WO_x} still requires further investigation. The author believes the resolution to a large extent depends on the phase transitions, as discussed in Section.~\ref{sec:wonawo}.

photocatalytic applications in solar hydrogen generation and organic pollutant degradation.

photocatalyst\cite{Macphee2010}, photoelectrochemical energy application \cite{Su2010}

Photocatalytic activity occurs when a semiconductor in aqueous solution is illuminated by photons of energy larger than the band gap, then electron-hole pairs generate free radicals, (i.e. \ce{OH.}) which enable further reactions. For water splitting using solar energy, the band gap should be within $2.0 \sim 3.0 $ eV, and CB edge should be more negative than reduction potential of \ce{H^+/H2}, whereas the VB top should be more positive than the oxidation potential of \ce{H2O/O2)}.\cite{Wang2012} The valence band holes (\ce{h^+}) oxidize water to oxygen and conduction band electrons propel hydrogen generation, as depicted in Eq.~\ref{eq:hervb} and \ref{eq:hercb}.

\begin{align}
4\ce{h^+} +  \ce{H2O} &\rightarrow \ce{O2} + 4\ce{H^+} \label{eq:hervb}\\
4\ce{H^+} +  4e^- &\rightarrow 2\ce{H2} \label{eq:hercb}
\end{align}

As shown in Fig.~\ref{fig:woxnhe}, \ce{WO3} has CB edge positioned slightly more positive than reduction potential of \ce{H^+/H2}(versus \gls{nhe}), and VB edge much more positive than the oxidation potential of \ce{H2O/O2}. So the photo-cleavage of water cannot be accomplished by \ce{WO3} alone. Nevertheless, a tandem cell approach by \ce{WO3} film and dye-sensitized \ce{TiO2} has been demonstrated with an efficiency of 4.5\%.\cite{Michael1999} It is worth noting that green plants also have two photosynthetic systems connected in series, one for oxidation of water into oxygen and the other for fixation of carbon dioxide.

% wo3 vs NHE
\begin{figure}[htb]
\centering
\includegraphics[width=0.7\textwidth]{woxnhe.jpg}
\caption[Bands positions of \ce{WO3} versus NHE]{Bands positions of \ce{WO3} in contact with aqueous electolyte at pH 1. adapted from Ref\cite{Gratzel2001}}
\label{fig:woxnhe}
\end{figure}

Moreover, favorable oxygen evolution of \ce{WO3} brings good performance in degradation of organic compounds\cite{Hepel2001,Luo2001,Watcharenwong2008}. The formation of long-lived holes is recognized as a key requirement.\cite{Pesci2011} Besides, \ce{WO3} is remarkably stable in acid, making it a significant candidate for treating water pollutant caused by organic acids.\cite{Monllor-Satoca2006}



the ubiquity of \ce{WO6} octahedra is essential for not only the optical properties but the ability to insert and extract ions in the EC oxides, due to the tunnels in three dimensions serving as path for transport of small ions.
The intercalation of hydrogen or alkali ions into \ce{WO3} created electron donor level. By absorbing the red part of incident spectrum, electrons at donor level make transition to the conduction band, causing the blue coloration in \ce{H_xWO3}.

\citeauthor{Wang2009a} mentioned that amorphous \ce{WO3} can only be used in lithium-based electrolytes due to its
in-compact structure and high dissolution rate in acidic electrolyte solutions. Electrochromic materials that can endure acidic electrolytes without degradation should be developed. Crystalline \ce{WO3} nanostructures with their much denser structures and small particle sizes are promising to be used as suitable electrochromic material in acidic electrolytes.

Characterization of ECD (work like a thin-film batteries) includes transmission measurement and associated EC calculation, charge-discharge time, current-time curve and the fitting of obtained data.

The coloration efficiency (CE) represents the change in the optical density (OD) per unit charge density ($Q/A$, in units of \si{\cm^2\per\coulomb}) during switching and can be calculated according to the formula:
\begin{equation}
CE = \frac{\Delta~OD}{(Q/A)} [cm^2/C],
\end{equation}
where OD = $log(T_{bleach}/T_{color})$. The EC and optical density depend on the wavelength and are usually higher in the near IR than in the visible region.
Using Ohm's law($U_s = IR = RQ/t_s$) with switch voltage $U_s$, resistance R and surface area A, switching time $t_s$ could be estimated as
\begin{equation}
t_s = \Delta~OD\cdot A \cdot R /(CE\cdot U_s).
\end{equation}

its one dimensional (1D) nanostructure has attained intensive research efforts in recent years due to the potential applications in advanced nano-electric and nano-optoelectronic devices.

\begin{quote}
a viable electrochromic smart window must exhibit a cycling life time \textgreater $10^5$ cycles, corresponding to an operation life at 10 - 20 years.
\end{quote}






\subsection{Crystal structures and properties of tungsten oxides and sodium tungsten oxides}\label{sec:wonawo}

\textbf{\ce{WO3}} Tungsten trioxides crystalize in multiple phases. The basic building block is \ce{WO6} octahedra.\footnote{Tungsten, with its electronic configuration as \ce{(Xe)4f^{14}5d^{4}6s^{2}}, has empty 5d and 6s orbitals in its +6 oxidation state.} \ce{WO3} crystal structure consists of \ce{WO6} octahedra joined at their corners, which may be considered as a perovskite structure of \ce{CaTiO3} with all \ce{Ca^{2+}} sites vacant. A representative lattice structures is illustrated in Fig.~\ref{fig:wo3oct}. These distorted \ce{WO6} octahedra adapts different tilting angles in different phases and edge-sharing octahedra also arises. The temperature-dependent phase transition and corresponding lattice constants in bulk form is summarized in Table.~\ref{tab:wo3phase}.\cite{Zheng2011} And a firm assignation of space group to monoclinic phases are still in debate.\cite{Chatten2005} It is noticed that the lattice parameters obtained via \emph{ab initio} calculation closely match the experimental values.\cite{Migas2010a} The phase transition scenarios in nanostructured \ce{WO3} are supposed to be much more sophisticated. Within Gibbs-Thomson frame work, one can generally expect lower transition temperature than their bulk counterparts due to enhanced surface energy. Temperature-dependent Raman spectroscopy provided support for this deduction.\cite{Boulova2002}

% wo3 phases
\begin{table}[htb]
\centering
\caption{\ce{WO3} phases}\label{tab:wo3phase}
\begin{tabular}{lccccc}
\toprule
&&&\multicolumn{3}{c}{Lattice constants \AA} \\
\cmidrule(l){4-6}
 Symbol    & Temperature (\si{\degreeCelsius}) & Phase & a & b & c   \\
\midrule
$\epsilon$-\ce{WO3} & $ -140 \sim -50$  & monoclinic II & 7.378 & 7.378 & 7.664  \\
$\delta$-\ce{WO3} & $-50 \sim 17$  & triclinic & 7.309 & 7.522 & 7.686  \\
$\gamma$-\ce{WO3} & $17 \sim 330$  & monoclinic I & 7.306 & 7.540 & 7.692  \\
$\beta$-\ce{WO3} & $330 \sim 740$  & orthorhombic & 7.384 & 7.512 & 3.846  \\
$\alpha$-\ce{WO3} & $> 740$  & tetragonal & 5.25 & NA & 3.91  \\
$h$-\ce{WO3} &  $<400$  & hexagonal & 7.298 & NA & 3.899  \\
\bottomrule
\end{tabular}
\end{table}

\begin{figure}[htb]
\centering
\includegraphics[width=0.4\textwidth]{octwo3.jpg}
\caption[Octahedra model of \ce{WO3}]{Octahedra model of \ce{WO3}}
\label{fig:wo3oct}
\end{figure}


\ce{WO3} is wide gap n-type semiconductor with valence band top featuring $2p$ states of oxygen and conduction band bottom arising primarily from $5d$ states of tungsten with some mixing of oxygen $2p$ states.\cite{Gillet2004} \citeauthor{Migas2010a} maintained there is essentially identical band dispersion near the gap region in case of $\epsilon$-\ce{WO3}, $\delta$-\ce{WO3}, $\gamma$-\ce{WO3} and $\beta$-\ce{WO3}.\cite{Migas2010a} When there is oxygen vacancy, the Fermi level moves into the conduction band and the gap shrinks by about 0.5 eV. \citeauthor{Migas2010a} also pointed out the flat bands at VBM and CBM could lead to poor transport of holes and electrons, thus may compromising the function in photoelectrochemical cells.

 \citeauthor{Chatten2005} also studied the oxygen vacancy in different phases of \ce{WO3}.\cite{Chatten2005}

Nonstoichiometric tungsten oxides \ce{WO_x} (i.e. \ce{WO_{2.92}}, \ce{WO_{2.87}}) are known as Magn$\acute{e}$li phases.

We do not discuss tungsten oxide hydrates (\ce{WO3.nH2O}) in this work since the product of thermal CVD approach is not plagued with this complexity. It's necessary, however, to deal with hydrated \ce{WO3} in the liquid synthesis routes, as indicated in Section.~\ref{sec:woxgrowth}.

Theoretical computation of electronic band structures for \ce{WO_x} proves difficult due to the aforementioned phase transition.

 oxygen deficiency, structure change, electronic properties vary according.

% band gap table
\begin{table}[htb]
\centering
\caption{Tungsten oxides band gap }\label{tab:wo3eg}
\begin{tabular}{lccr}
\toprule
Phase & Experimental (eV) & Theory (eV) & Remarks  \\
\midrule
amorphous \ce{WO3} & 3.2  & NA &    \\
monoclinic bulk \ce{WO3} &  2.6   & 1.73\cite{Migas2010a}  &    \\
tetragonal bulk \ce{WO3} &     & 0.66 \cite{Migas2010a}&    \\
nano-\ce{WO3} & 2.6$\sim$3.2  & NA &    \\
nano-\ce{WO_{3-x}} & NA  & NA &    \\
\bottomrule
\end{tabular}
\end{table}

Tungsten bronzes was coined by Wohler in 1837.\cite{Deb2008} \ce{Na_{x}WO3}

\subsection{Synthesis strategies}\label{sec:woxgrowth}

We will first give a brief review on the synthesis of \ce{WO3}. As the chemical formula suggested, the most straightforward way is heating metallic tungsten in various forms (i.e. powders,\cite{Zhou2005a,Cao2009,Hsieh2010} foils and wires). Actually \ce{WO_x}\footnote{$x$ is between 2 and 3.} NWs was observed when directly heating W wires.\cite{Gu2002a} Due to the extreme high melting point of W, it requires very high temperature (say, above 1100 \si{\degreeCelsius}) to produce large amount of growth species. Therefore scalable growth seems difficult. The DC current heating proved to be a route with potential large yield. Tungsten wire or filament is connected to a voltage source, and substrate is positioned in proximity to the heated W wire.\cite{Lingfei2006,Thangala2007,Chang2007}

Tungsten oxides powder can also been used as precursor to prepare \ce{WO_{x}} \glspl{nw}.\cite{Huang2008a,Wang2009} Hydrogen-containing agents (i.e. water, \ce{H2},or methane\cite{Klinke2005}) were often involved into the reaction.\cite{Baek2007,Karuppanan2007} The potential benefit is \ce{WO3} has a much lower melting point ($\approx 1470$\si{\degreeCelsius}) than that of W.

Direct oxidation and \ce{WO3} conversion could be grouped as addition approach. In contrast, decomposition method is also feasible.
\citeauthor{Pol2005} obtained \ce{W18O49} nanorods by thermal dissociation of \ce{WO(OMe)4} at 700 \si{\degreeCelsius}, and \ce{WO3} by the annealing at 500 \si{\degreeCelsius} at air atmosphere.\cite{Pol2005}

Spray pyrolysis is a typical aerosol-assisted CVD technique. Typical process flow is: the precursor solution is pumped to an atomizer, and then sprayed through the carrier gas as a fine mist of very small droplets onto heated substrates. Subsequently the droplets undergo evaporation, solute condensation, and thermal decomposition, which then result in film formation.\cite{Zheng2011}

Main reactions include: \cee{W + O2 -> WO_{3-x}} \cee{W + H2O -> W_{3-x} + H2}

the energetic sources are ion bombardment, electron beam, laser ablation, and combustion flame\cite{Rao2011}.

The sol–gel process is a well-known, intensively studied wetchemical technique that is widely used in materials synthesis. This method generally starts with a precursor solution (the ``sol") to form discrete particles or a networked gel structure. During the course of gelation (aging process), various forms of hydrolysis and polycondensation take place.

Hydrothermal method has been an important route to synthesize a diversity of nanomaterials (i.e. ). Preparation of \ce{WO_{3-x}} has also been demonstrated hydrothermally.\cite{Lee2003,Gu2007} Usually, the precursor (\ce{H2WO4}) is mixed with other reactants (sulfides or certain organic acid) in solution and pH value is adjusted as another control degree of freedom. Then the solution is transferred into a sealed container (i.e. Teflon autoclave) and maintained at elevated temperature ($120 \sim 300$ \si{\degreeCelsius}) for tens of hours. Finally the product is separated from the solution and dried, layered \ce{WO3.nH2O} flakes are usual products.

In addition, doped \ce{WO3} was also demonstrated

 The composition and phase of final product highly depend on the synthesis conditions.

%\newgeometry{margin=1in}
\begin{landscape}
\begin{table}[htb]
\centering
\caption{Tungsten Oxides Growth and Application}\label{tab:wox}
{\footnotesize
\begin{tabular}{lp{3.5in}p{2.5in}c}
\toprule
composition  &  methods & highlights &  reference  \\
\midrule
\ce{WO3} & hot W filament (above 1500 \si{\degreeCelsius}) in Ar/\ce{O2} flow  & Cubic phase, PL, resistivity measured & \cite{Thangala2007} \\
\addlinespace[0.5em]
& W filament DC heating in \ce{NH3} or \ce{N2}/\ce{H2} flow  & multi phases, 100mg per batch, stable dispersion in both organic and aqueous solvents & \cite{Chang2007} \\
\addlinespace[0.5em]
& W powder heating Ar/\ce{O2} flow  & triclinic phase, CL at 370,415nm, UV-Vis, 3eV & \cite{Hsieh2010} \\
\cmidrule(l){2-4}
& \ce{WOx} film in \ce{H2} and \ce{CH4} flow  & monoclinic phase, tungsten carbide is the key & \cite{Klinke2005} \\
\cmidrule(l){2-4}
& \ce{Na2WO4.2H2O} by hydrothermal heating to test for line break & 3 different morphologies and photodegradation & \cite{Rajagopal2009}  \\
& \ce{H2WO4.2H2O}, \ce{H2O2} and poly(vinyl alcohol) etc by solvothermal  & multi phases deposition of FTO with varied bandgaps  & \cite{Su2010}  \\

\midrule
\ce{W18O49} & \ce{KOH} etching of W tips followed by heating in Ar flow  & oxygen from leakage, VS process& \textcite{Gu2002a} \\
\addlinespace[0.5em]
& \ce{NW(CO)6}, \ce{Me3NO.2H2O} and oleylamine by hydrothermal  &  RT PL at 350nm and 440nm & \cite{Lee2003}  \\
\bottomrule
\end{tabular}
}
\end{table}
\end{landscape}
%\restoregeometry



\section{Experimental}

Metallic tungsten powders were utilized as source materials in this work. Four kinds of W powder were used in total, as summarized in Table~\ref{tab:powder}. Two growth configurations were explored, namely ordinary transport and seeded growth.
% Tungsten powders size and purity
\begin{table}[htb]
\centering
\caption{Tungsten powder sizes and purities}\label{tab:powder}
\begin{tabular}{lccr}
\toprule
Name & purity & average size & vendor info\\
\midrule
3N   &  99.9\% & 17 $\mu$m & Alfa Aesar \#39749\\
3N5   &  99.95\% & 32 $\mu$m  & Alfa Aesar \#42477\\
4N5   &  99.995\% & 3.3 $\mu$m  & Materion T-2049 \\
5N   &  99.999\% & 1.5 $\mu$m & Alfa Aesar \#12973\\
\bottomrule
\end{tabular}
\end{table}

In a typical ordinary transport experiment (Fig.~\ref{fig:wogrow}), about 2 g tungsten powder was positioned in the upstream end of a quartz boat and downstream about 2.5 inch the substrate was stationed. 
% cvd NW growth
\begin{figure}[htb]
\centering
\includegraphics[width=0.6\textwidth]{CVD_and_temp.jpg}
\caption[\ce{WO3} NW CVD growth]{\ce{WO3} NW CVD growth and the temperature profile. The nominal NWs growth temperature were estimated according to the interpolation data. Zero inch location is defined at the upstream edge of furnace.}
\label{fig:wogrow}
\end{figure}
The substrates were p-type, boron-doped and [100]-orientated silicon with about 1 $\mu$m thermally-grown silicon oxide layer on the polished side. The cleaning procedures have been elucidated on page~\pageref{ch2sub}. The boat was then loaded into reaction chamber in a way that the powder source was placed in the center of heating zone, and the substrate upstream end was aligned at 6.5 inch.\footnote{As defined in Fig.~\ref{fig:wogrow}} After pumping down, 1 \gls{sccm} \ce{O2} and 10 sccm Ar were admitted into the chamber, respectively, after which the overall pressure read approximately 110 mTorr. The temperature ramped up to 1000 \si{\degreeCelsius} in 30 min and lasted for 4 h, and subsequently the apparatus was allowed to naturally cool down to room temperature.


\section{Results and Discussion on 1D Nanowire Growth}
\subsection{Tungsten Oxides: Impurities and Residual Effects}\label{sec:nawox}

The growth dynamics of each source were investigated in ordinary transport configuration. In each experiment, the temperature, flow rate and source and substrate locations were controlled. Another parameter the author found important is the growth number\footnote{Defined as the order of growth performed in the same reaction chamber} in the same reaction chamber. Usually one will conduct several runs in the same chamber, and different results may arise with respect to the growth number. With this degree of freedom included, the primary features of the deposition and the associated parameters were summarized in Table~\ref{tab:wot}.
% \ce{WO_x} ordinary transport results
\begin{table}[htb]
\centering
\caption{\ce{WO_x} ordinary transport results}\label{tab:wot}
\begin{tabular}{lp{2in}p{2in}r}
\toprule
\multicolumn{2}{c}{Growth Number} \\
\cmidrule(l){2-3}
 Source   & First & Second and more & Sodium(ppm)   \\
\midrule
3N      & Morphology from high to low temperature: islands, dense layer and a few NWs & Dense wires become more and more, sodium tungsten oxides phases dominate & 20  \\
3N5     & islands in high temperature end and layers in low temperature end & NA &      NA\\
4N5     & dense islands, some tiny wires in low temperature end & NA & 0.065 \\
5N      & similar to those of 4N5  & NA & 0.05\\
\bottomrule
\end{tabular}
\end{table}

The change of metallic W source after growth carries important information on the growth. So before examining the deposition on substrates, the SEM images of all powders before and after growth are presented in Fig.~\ref{fig:pdbefore} and Fig.~\ref{fig:pdafter}, respectively.
% powder before growth
\begin{figure}[htb]
\centering
\includegraphics[width=0.6\textwidth]{pd_before.jpg}
\caption[SEM images of W powder before growth]{SEM images of W powder before growth for (a) 3N, (b) 3N5, (c) 4N5 and (d) 5N, respectively.}
\label{fig:pdbefore}
\end{figure}
Tungsten metal crystallizes in body center cubic (BCC) phase. Overall, the facets were still discernible. The 4N5 and 5N powders share similar appearance and size distribution (Fig.~\ref{fig:pdbefore}c-d). In contrast, 3N powders are almost half the size of 3N5 ones, and have more uniform size distribution as well. Absolute size and its distribution play a key role in its usage as seed, as discussed in Section~\ref{sec:sgfg}.
% powder after growth
\begin{figure}[htb]
\centering
\includegraphics[width=0.6\textwidth]{pd_after.jpg}
\caption[SEM images of W powder after growth]{SEM images of W powder after growth for (a) 3N, (b) 3N5, (c) 4N5 and (d) 5N. }
\label{fig:pdafter}
\end{figure}
After growth, 4N5 and 5N powder (Fig.~\ref{fig:pdafter}c-d) both develop the round edges, while 3N and 3N5 powders (Fig.~\ref{fig:pdafter}a-b) become bundled rods. This difference mainly arises from the average size, which will dictate the degree of oxidation. Another possible reason is the presence of foreign elements in each W powders. As indicated by the certificate of analysis from corresponding vendors, carbon, nickel, iron and sodium can amount up to tens of ppm in low purity W sources. 

The sodium contents in 3N powders fundamentally change the deposition in ordinary transport growth. Typical morphologies of the as-synthesized nanostructures grown on a \ce{SiO2}-Si substrate were shown in Fig.~\ref{fig:nawoxsemedx}a. The nanostructures were \glspl{nw} with lengths up to several hundred microns and diameters about 40 to 500 nm. The deposition of nanowires was generally located on the substrates with the growth temperature ranging from 660 to 420 \si{\degreeCelsius}. Enlarged view at higher magnification revealed some nanowires were cylindrical and some were bundled belt. Rectangular microplate structures grew at lower temperature end of substrate, as displayed in Fig.~\ref{fig:nawoxsemedx}b and inset. The chemical compositions of the deposition were analyzed using \gls{edx}. A representative \gls{edx} spectrum in Fig.~\ref{fig:nawoxsemedx}c shows the existence of W, Na, O, and Si signals in the specimen, where the Si signal is from the substrate. The crystal structures of the as-synthesized specimens were examined using XRD. The diffraction peaks were carefully indexed and the deposition was identified as two phases of sodium tungsten oxides and one phase of tungsten oxide as indicated on Fig.~\ref{fig:nawoxsemedx}d. The two sodium tungsten oxide phases are the triclinic \ce{Na5W_{14}O_{44}} phase (ICDD PDF 04-012-4449, \emph{a}=7.2740 \si{\angstrom}, \emph{b}=7.2911 \si{\angstrom}, \emph{c}=18.5510 \si{\angstrom}, $\alpha$=96.37$^\circ$, $\beta$=90.89$^\circ$, $\gamma$=119.65$^\circ$) and the triclinic \ce{Na2W4O_{13}} phase (ICDD PDF 04-012-7108, \emph{a}=11.1630 \si{\angstrom}, \emph{b}=3.8940 \si{\angstrom}, \emph{c}=8.2550 \si{\angstrom}, $\alpha$=90.60$^\circ$, $\beta$=131.36$^\circ$, $\gamma$=79.70$^\circ$). The tungsten oxide is the monoclinic \ce{WO3} phase (ICDD PDF 01-083-0950, \emph{a}=7.30084 \si{\angstrom}, \emph{b}=7.53889 \si{\angstrom}, \emph{c}=7.6896 \si{\angstrom}, $\beta$=90.89$^\circ$).

% nawox sem edx
\begin{figure}[htb]
\centering
\includegraphics[width=0.7\textwidth]{nawox_semedx.jpg}
\caption[SEM and EDX on \ce{Na_xWO3}]{SEM images of (a) dense array of the as-synthesized ultra-long nanowires on \ce{SiO2}-Si substrate and (b) rectangular microplates grow among the nanowires. (c) EDS and (d) XRD spectra showing the chemical compositions and phases of the deposition.}
\label{fig:nawoxsemedx}
\end{figure}

XRD measurements revealed the overall structure of the specimen. To obtain detailed information on the crystallinity, composition, and growth direction of the nanostructures, TEM analyses with imaging, electron diffraction, and \gls{edx} were performed on more than 20 nanowires. The majority of the nanowires were identified as the \ce{Na5W_{14}O_{44}} phase (Fig.~\ref{fig:nawoxtem}a-c). The HRTEM image in Fig.~\ref{fig:nawoxtem}b reveals that the nanowire exhibits a single-crystalline structure. The corresponding diffraction pattern in the inset of Fig.~\ref{fig:nawoxtem}b was recorded in $[\bar{1}10]$ axis. Based on the analyses on a series of diffraction patterns and HRTEM images, the nanowire was confirmed as the triclinic \ce{Na5W_{14}O_{44}}. The growth direction of \ce{Na5W_{14}O_{44}} NWs was determined to be parallel to the (001) plane. EDX measurement (Fig.~\ref{fig:nawoxtem}c) shows the nanowire consisted of W, Na, and O. No other elements were detected.\footnote{The Cu and C signals come from the supporting Cu grid with lacey carbon and the Cr signal from the tip of JEOL double tilt holder.} Small fraction of nanowires were found to be the monoclinic \ce{WO3} phase (Fig.~\ref{fig:nawoxtem}d-f). The SAED shown in the inset of Fig.~\ref{fig:nawoxtem}e was recorded in [100] zone. The growth direction of the \ce{WO3} nanowire was determined to be perpendicular to the (002) plane with a d-spacing of 0.38 nm. EDX spectrum in Fig.~\ref{fig:nawoxtem}f shows the nanowire consisted of W and O without Na or other impurities. The rectangular microplate structures shown in Fig.~\ref{fig:nawoxsemedx}b were also collected for TEM examination. These microplate structures were identified to be the triclinic \ce{Na2W4O_{13}} phase. The SAED in Fig.~\ref{fig:nawoxtem}h was recorded in [101] axis. No high-quality HRTEM images were acquired for the \ce{Na2W4O_{13}} plate due to its thickness. The long edge of the plate was identified as parallel to the (010) plane and the short edge parallel to the ($\bar{1}01$) plane. From XRD spectra, the long edge of \ce{Na2W4O_{13}} plate can be figured out as the $\vec{c}$ axis and the short edge is along the $\vec{b}$ axis. Fig.~\ref{fig:nawoxtem}i shows the plate structure also consisted of W, Na, and O and no other impurities were detected. Normalized to the highest W peak, the Na peak intensity in Fig.~\ref{fig:nawoxtem}i is higher than the one in Fig.~\ref{fig:nawoxtem}c. This result indicates a higher Na:W ratio for the \ce{Na2W4O_{13}} plate structure than that of \ce{Na5W_{14}O_{44}} nanowires, in consistent with the compositions of these two phases.
% tem
\begin{figure}[htb]
\centering
\includegraphics[width=0.8\textwidth]{nawox_tem.jpg}
\caption[TEM analyses on sodium tungsten oxide sample]{TEM images at low-magnification, HRTEM images, and EDS spectra of a \ce{Na5W_{14}O_{44}} NW (a, b, c), a \ce{WO3} NW (d, e, f), and a \ce{Na2W4O_{13}} microplate (g, h, i).}
\label{fig:nawoxtem}
\end{figure}

Micro-Raman spectroscopy was also carried out at room temperature in ambient atmosphere to confirm the phases of the microplates and ultra-long nanowires. Fig.~\ref{fig:nawoxram}a shows the m-Raman spectrum of the microplates. The Raman lines at 949, 794-777, 366-263 \si{cm^{-1}} closely matches the reported \ce{Na2W4O_{13}} Raman shift frequencies.\cite{Fomichev1992} The Raman spectrum from the ultra-long nanowires is shown in Fig.~\ref{fig:nawoxram}b. The spectrum shows major peaks in 100-150 \si{cm^{-1}}, 650-900 \si{cm^{-1}}, and 900-100 \si{cm^{-1}} regions. As compared to the reported peak positions (shown by reference lines in Fig.~\ref{fig:nawoxram}b), the nanowires consist of only little or no amount of \ce{WO3} and \ce{Na2W4O_{13}} phases. Major peaks at 107, 695, 765, 913, 943, and 965 \si{cm^{-1}} cannot be assigned to reference in literatures. According to XRD and TEM analyses, the nanowires are mainly \ce{Na5W_{14}O_{44}}; therefore these peaks should belong to the \ce{Na5W_{14}O_{44}} phase. So far no Raman spectrum has been reported for this phase, hence this is probably the first observed Raman pattern for \ce{Na5W_{14}O_{44}}.
% nawox raman
\begin{figure}[htb]
\centering
\includegraphics[width=0.7\textwidth]{nawox_raman_series}
\caption[Raman spectra on \ce{Na_xWO3}]{Raman spectra of (a) microplates matching the reported peak positions of the \ce{Na2W4O_{13}} phase, and (b) nanowires with comparisons to the reported major peak positions of \ce{WO3} (shown by black line) and \ce{Na2W4O_{13}} (shown by red line) phases.}
\label{fig:nawoxram}
\end{figure}
Based on the correlation of Raman spectra and crystallographic data of a variety of tungsten bronze, an empirical formula to relate the Raman peaks and \ce{W-O} bonding lengths has been found as
\begin{equation}\label{eq:wobond}
\nu = 25823 \exp(-1.902\cdot R)
\end{equation}
where $R$ is tungsten-oxygen bond length in \si{\angstrom}, and $\nu$ is Raman stretching wavenumber in \si{cm^{-1}}.\cite{Hardcastle1995} The standard deviation of estimating \ce{W-O} bond distance from Raman stretching wavenumber is $\pm 0.034$ \si{\angstrom}. The observed Raman peaks of \ce{Na5W_{14}O_{44}} phase lies at 965, 943, 913, 808, 786, 778, 765, 695 and 107 \si{cm^{-1}}. Multi-peak Lorentzian fitting is preformed to precisely determine the central maximum. Good fitting is obtained as shown in Fig.~\ref{fig:naworamfit}. The calculated \ce{W-O} bond distances using Eq.~\ref{eq:wobond} are then listed in Table~\ref{tab:nawobond}. The predicted \ce{W-O} bond lengths agree very well with the crystallographic value of \ce{Na5W_{14}O_{44}} phase.\cite{Triantafyllou1999a} The 107 \si{cm^{-1}} peak probably is caused by \ce{Na-O} bond.
% Na5 raman fitting
\begin{figure}[htb]
\centering
\includegraphics[width=0.6\textwidth]{naxwo_ramfit}
\caption[Multi-peak Lorentzian fitting on \ce{Na5W_{14}O_{44}}]{Multi-peak Lorentzian fitting on two major peaks region of \ce{Na5W_{14}O_{44}}. The peaks sum height difference is caused by different baseline value adopted in each fitting.}
\label{fig:naworamfit}
\end{figure}

% W-O bond length
\begin{table}[htb]
\centering
\caption{\ce{W-O} bond length predication}\label{tab:nawobond}
\begin{tabular}{cccc}
\toprule
Fitted center (\si{cm^{-1}}) & length (\AA) & Fitted center (\si{cm^{-1}}) & length (\AA) \\
\midrule
694.6 & 1.900 &  808.6 &  1.821 \\
745.4 & 1.863 &  911.5 &  1.758 \\
764.4 & 1.850 &  933.0 &  1.745 \\
778.7 & 1.840 &   943.5 & 1.740 \\
788.4 & 1.834 &   965.4 & 1.728 \\
\bottomrule
\end{tabular}
\end{table}
For a comprehensive comparison purpose, the Raman fingerprints of various tungsten oxides and tungstate is shown in Table.~\ref{tab:woram2}. 
\begin{table}[htb]
\centering
\caption{Raman fingerprints of tungsten oxides and sodium tungsten oxides}\label{tab:woram2}
\begin{tabular}{lp{3in}r}
\toprule
Phase & Raman Shift (\si{cm^{-1}}) &  Reference   \\
\midrule
\ce{WO2}  & 287(s), 334(w), 514(m), 600(w), 621(w), 785(vs) & \cite{Ma2005} \\
\ce{W18O49}  & broad bands from 750 to 780 & \cite{Guo2012} \\
             &  267(s), 778(s), 969(m) & \cite{Liu2013d} \\
m-\ce{WO3}  & 131(m), 265(m), 327(m), 715(s), 807(vs) &  \cite{Salje1975a,Daniel1987} \\
h-\ce{WO3}  & 162(m), 253(m), 320(m), 645, 690(s), 817(vs) &  \cite{Daniel1987}\\
\ce{WO3.H2O}  & 230(m), 377(w), 428(w), 650(s), 816(vs), 946(vs) &  \cite{Daniel1987} \\
\ce{WO3.2H2O}  & 235, 268(m), 380(w), 662(s), 685(vs), 960(vs) & \cite{Daniel1987} \\
\ce{Na2WO4}  & 94(w), 314(vw), 377(m), 813(m), 930(vs) &  \cite{Lima2011} \\
\ce{Na2W2O7}  & 381(w), 422(w), 763(w), 835(s), 886(m), 948(m), 957(vs) &  \cite{Knee1979} \\
\ce{Na2W4O13} & 263(w), 272(w), 311(w), 366(w), 777(vs), 794(s), 949(m) &\cite{Fomichev1992}\\
\ce{Na5W14O44} & 107(s),695(m), 765(vs), 913(w), 943(m), 965(m)& this work\\
\bottomrule
\end{tabular}

Materials in solid state, 
vw-very weak; w-weak; m-medium; s-strong; vs-very strong
\end{table}

Heated at elevated temperatures, the W precursor, quartz tube, and quartz boat were three possible sources for Na. The sodium concentration in 3N source was 20 ppm as provided by vendor.\footnote{Due to the non-uniform distribution of foreign elements, detection of sodium by EDX is possible although the concentration is below the limit of typical EDX capacity.} The Na concentrations in quartz tube and quartz boat (GE 214 quartz, Wilmad Labglass) were 1 ppm and 0.7 ppm, respectively. To verify the major source of Na content coming from 3N powder, control experiments were employed by using W source with higher purities (4N5 or 5N). These ultra-high purity W sources produced pure tungsten oxide deposition and no Na content was detected by EDX. These results verified the Na content was mainly from the 3N W powders. In the OT growth experiments, the W source was first slowly oxidized at 1000 \si{\degreeCelsius} with 1 sccm \ce{O2} and the oxidized source was then evaporated producing \ce{WO3} and other sub-stoichiometric tungsten oxide vapors. The tungsten oxide vapors produced this way were limited by the slow oxidation, as indicated by
\begin{enumerate*}[label=\itshape\alph*\upshape)]
\item only small amount of the tungsten source was consumed during the growth; and
\item only the top layer of the tungsten source was oxidized showing a dark blue color after growth.
\end{enumerate*} Due to its low concentration, the exact composition of the Na content in the tungsten source is difficult to probe. However, the evaporation rate of the Na content is expected to be high when heated at 1000 \si{\degreeCelsius}. This assumption is supported by the observation that no Na was found in the deposition along the whole substrate in control experiments using the used 3N W source. Put it in another way, this means that the majority of Na content was already evaporated in previous experiment. Hence, despite of its low concentration in the source, the total amount of Na-based vapors produced during the growth was significant, resulting in the dominance of sodium tungsten oxide phases in the nanowire growth region.

\begin{figure}[htb]
\centering
\includegraphics[width=0.8\textwidth]{nawox_edx_series.jpg}
\caption[Photograph of a series of specimens grown]{(a) Photograph of a series of specimens grown in the same reaction chamber with the increase of growth number showing different growth zones and corresponding morphology changes. (b) Change of Na/W ratios with the increase of growth number at different growth zones.}
\label{fig:nawoxser}
\end{figure}

As shown by the photograph in Fig.~\ref{fig:nawoxser}a, for a series of growth performed in the same reaction chamber, the morphology of the nanowires changes with the increase of growth number. Three growth zones of nanowires can be identified on basis of the coverage and density of the nanowires, as delimited by the vertical guide line. The growth Zone I was located at high temperature end ranging from 660 to 520 \si{\degreeCelsius}, Zone II was from 520 to 470 \si{\degreeCelsius}, and Zone III from 470 to 420 \si{\degreeCelsius}. Detailed morphological changes are depicted in Fig.~\ref{fig:nawoxsemall}.
\begin{figure}[htb]
\centering
\includegraphics[width=0.8\textwidth]{nawox_sem_series.jpg}
\caption[SEM images of morphology evolution]{SEM images of morphology evolution of the NWs at different growth zones showing morphology changes with the increase of growth number. The insets show detailed structures of the NWs.}
\label{fig:nawoxsemall}
\end{figure}
For the first growth (Fig.~\ref{fig:nawoxsemall}a-c) in a new clean reaction chamber, there were only a little nanowires scattered in Zone I (Fig.~\ref{fig:nawoxsemall}a), ultra-long nanowires were mostly found in Zone II with higher density and coverage (Fig.~\ref{fig:nawoxsemall}b), and in Zone III the coverage, density, and length of nanowires gradually reduced as the growth location moved downstream (Fig.~\ref{fig:nawoxsemall}c). With growth number increased to 3 (Fig.~\ref{fig:nawoxsemall}d-f) and 5 (Fig.~\ref{fig:nawoxsemall}g-i), the coverage, density, and length of the nanowires increased in all three zones. Zone I saw the most significant morphology changes as shown by Fig.~\ref{fig:nawoxser}a and Fig.~\ref{fig:nawoxsemall}g. After the fifth growth in the same reaction chamber, the deposition of ultra-long nanowires almost covered all three zones. For Zones I and II, no significant morphology change was observed with further increase of the growth number up to 8. However, for Zone III the microplate structures (shown in inset of Fig.~\ref{fig:nawoxsemall}i) kept increasing significantly with the growth number. EDX study revealed the variation of Na concentration in the deposition with the growth number. Fig.~\ref{fig:nawoxser}b shows the variation of the Na/W ratios with the increase of growth number at different nanowire growth zones. Generally, the Na/W ratios at all growth zones increased with the increase of growth number. The large fluctuation of Na/W ratio at Zone III was presumably due to the uneven distribution of Na content with different morphologies at Zone III.

\begin{figure}[htb]
\centering
\includegraphics[width=0.9\textwidth]{nawox_sch.jpg}
\caption[Schematic drawings of residual \ce{Na_xWO3} growth]{Schematic drawings showing (a) growth with tungsten source in a new tube, (b) enhanced growth with both residue deposition and tungsten source, and (c) growth with residue deposition only.}
\label{fig:nawoxsch}
\end{figure}

A mechanism of residue deposition enhanced growth was proposed to explain the morphology evolution of the nanowires with the growth number. Since the CVD setup employed in this thesis is a hot-wall system, deposition formed on the substrate surface as well as on the inner wall of the reaction chamber in the growth area. Residue deposition was found on the quartz tube which can be distinguished by the color change on the quartz tube after each growth. With the increase of growth number for the same reaction chamber, the residue deposition on the inner wall of the tube also increased. If a clean quartz tube free of any residue deposition is used as the reaction chamber, the vapors are supplied only from the source material forming deposition in the growth area with lower temperature (as shown by the schematics in Fig.~\ref{fig:nawoxsch}a). When a quartz tube with residue deposition from previous depositions is used repeatedly as the reaction chamber, the residue deposition heated at growth temperature will also produce vapor locally. Therefore, the vapors from both the source materials and the local residue deposition will result in an enhanced growth of nanowires (Fig.~\ref{fig:nawoxsch}b). To prove this hypothesis, control experiments were performed utilizing a quartz tube used multiple times previously (e.g. a growth number of 9). The experiments were carried out without any source materials while other growth parameters remain the same. Similar nanowires were found on the substrate in the nanowire growth zones. Without the source materials, the residue deposition from the tube surface was the only possible source that could produce vapors forming nanowires on the substrate (Fig.~\ref{fig:nawoxsch}c). This result confirmed the nanowire growth could be enhanced by the presence of the residue deposition and explained the morphology changes of the nanowires with the increase of growth number.
%xrd na5
\begin{figure}[htb]
\centering
\includegraphics[width=0.8\textwidth]{wox_xrd_series}
\caption[XRD spectra of the as-synthesized specimen with the increase of growth number from 1 to 5]{XRD spectra of the as-synthesized specimen with the increase of growth number from 1 to 5.}
\label{fig:nawoxxrd}
\end{figure}
The general trend of Na/W ratios with the growth number can also be explained by the aforementioned effect of residue deposition. The sodium contents from the residue deposition will join the new Na contents from the W source forming deposition. Fig.~\ref{fig:nawoxxrd} displays the XRD spectra of a series of specimens with the growth number increases from 1 to 5. The XRD spectra reveal the relative concentration dynamics of the three phases as the growth number increases. Compared to the \ce{WO3} phase, the \ce{Na5W_{14}O_{44}} and the \ce{Na2W4O_{13}} phases increase significantly as the growth number increase from 1 to 5. And in the 5th growth, \ce{Na2W4O_{13}} become the dominated phase compared to the other two phases. These results are consistent with the SEM observation (Fig.~\ref{fig:nawoxsemall}) and the composition change from the EDX measurements (Fig.~\ref{fig:nawoxser}b). Details XRD indices assignment is summarized in Table~\ref{tbl:wo3xrd}.

\begin{table}
\centering
\caption{Assignment of XRD peak indices to different phases}\label{tbl:wo3xrd}
\begin{tabular}{llllll}
\toprule
\ce{WO3} &          &\ce{Na5W14O44} &      & \ce{Na2W4O13} & \\
2$\theta$   & hkl   & 2$\theta$   & hkl    & 2$\theta$   & hkl   \\
\midrule
 23.05   & 0 0 2 & 9.65    & 0 0 2  & 10.85   & 1 0 0 \\
 23.59   & 0 2 0 & 14.49   & 0 0 3  & 21.8    & 2 0 0 \\
 24.31   & 2 0 0 & 19.37   & 0 0 4  & 32.97   & 3 0 0 \\
 26.60   & 1 2 0 & 24.27   & 0 0 5  & 56.45   & 5 0 0 \\
 34.10   & 2 0 2 & 29.22   & 0 0 6  &         &       \\
 47.11   & 0 0 4 & 39.36   & 0 0 8  &         &        \\
 48.29   & 0 4 0 & 44.46   & 0 0 9  &         &        \\
 49.79   & 4 0 0 & 49.71   & 0 0 10 &         &        \\
         &       & 55.23   & 0 0 11 &         &        \\
         &       & 60.77   & 0 0 12 &         &        \\
\bottomrule
\end{tabular}
\end{table}

The \ce{Na5W14O44} phase was further confirmed by a XRD scanning from 4 to 10 degree in $2\theta/\Omega$ configuration. As shown in Fig.~\ref{fig:naxrdlow}, two peaks at 4.84$^{\circ}$ and 9.65$^{\circ}$ were well resolved, corresponding a lattice spacing of 18.25 and 9.16 \si{\angstrom}, respectively. This compares favorably with both HRTEM analysis (Fig.~\ref{fig:nawoxtem}e) and ICDD PDF 04-012-4449 database reference. 
%061212 sample
\begin{figure}[htb]
\centering
\includegraphics[width=0.5\textwidth]{naxwo_xrd_low}
\caption{Low angle scan of XRD pattern on the as-grown sodium tungsten oxide sample}
\label{fig:naxrdlow}
\end{figure}

The optical transmission and reflection properties of the as-grown \ce{Na5W14O44} NWs were evaluated using UV-Vis spectroscopy, as shown in Fig.~\ref{fig:ch3naxuvvis}. The sample was prepared on a quartz substrate, which is transparent from 300 to 1300 nm. SEM images (not shown) found dense NWs array growth similar to those on \ce{SiO2}-Si substrates, indicating the residue effect is not sensitive to the substrate. 
%051412 sample
\begin{figure}[htb]
\centering
\subfloat[]{\label{fig:nauv}\includegraphics[width=0.45\textwidth]{uvvis_naxwo3}}\hspace{0.04\textwidth}
\subfloat[]{\label{fig:nadrs}\includegraphics[width=0.45\textwidth]{uvvis_naxwo2}}
\caption[UV-Vis and \gls{drs} of \ce{Na5W14O44} NWs on quartz]{UV-Vis and DRS of \ce{Na5W14O44} NWs on quartz: (a) Optical transmission and (b) diffuse reflection spectra.}
\label{fig:ch3naxuvvis}
\end{figure}
The transmission (Fig.~\ref{fig:nauv}) was about 70\% above 400nm, and began to drop significantly for shorter wavelength. This variation was also identified in the diffuse reflection spectrum in Fig.~\ref{fig:nadrs}, where the reflection reached to a maximum of 16\% at 390 nm. One difference is the DRS increased from 1300 nm to 400nm, presumably due to the inverse scattering intensity with NW dimension of 300 nm. The calculated absorption was shown in Fig.~\ref{fig:nauv}, suggesting an energy gap edge at about 380 nm. This is consistent with the observed white sample color; however, the conductivity of \ce{Na5W14O44} has not been investigated experimentally. Based on the electron sum rule, the sodium tungsten oxide can be cast into \ce{Na5W_1^{5+}W_{13}^{6+}O44}, which indicates the W atom state is quite close to that in \ce{WO3}. More investigation is needed to vigorously determined the electronic transport properties of \ce{Na5W14O44} NWs. 

Typical morphologies of ordinary transport growth using other powders are shown in Fig.~\ref{fig:wox3n5} and Fig.~\ref{fig:wox4n5}, respectively. The common feature of these samples is the dominance of islands growth in upstream end and layers growth in downstream end. The exact dimension still shows some minor difference, particularly in low temperature region. The deposition with 4N5 and 5N have much smaller grain size than that with 3N5 powder.
% 3N5 powder growth
\begin{figure}[htb]
\centering
\includegraphics[width=0.9\textwidth]{wox3n5.jpg}
\caption[SEM images of \ce{WO3} growth using 3N5 powder]{SEM images of \ce{WO3} growth using 3N5 powder showing the morphology variation from (a) high temperature, (b) intermediate temperature, to (c) low temperature.}
\label{fig:wox3n5}
\end{figure}

% 4N5 powder growth
\begin{figure}[htb]
\centering
\includegraphics[width=0.9\textwidth]{wox4n5.jpg}
\caption[SEM images of \ce{WO3} growth using 4N5 powder]{SEM images of \ce{WO3} growth using 4N5 powder showing the morphology variation from (a) high temperature and (b) intermediate temperature, to (c) low temperature. Growth with 5N powders have almost the same morphology with that of 4N5, and is not shown hereby.}
\label{fig:wox4n5}
\end{figure}
It is worth pointing out that there are some NWs growth in low temperature end associated with 4N5 growth(Fig.~\ref{fig:wox4n5}c), although not in large scale.

\subsection{Tungsten Oxides: Seeded Growth}\label{sec:sgfg}

As mentioned in Sec.~\ref{sec:nawox}, growth using higher purity tungsten powders primarily produces thin film deposition, and growth using 3N powder is plagued with the impurity effect producing sodium tungsten oxide NW instead of tungsten oxide NW. New growth method was thus explored to obtain \ce{WO3} NWs. In this study, a seeded growth approach was developed and proved to be a successful alternative way. In the seeded growth, additional tungsten powders was used on the substrate serving as local seed, on which dense \ce{WO3} NW array grew. 

Understanding the oxidation of tungsten powder is the key to obtain high yield in seeded growth. Oxidation of tungsten have been investigated under diverse conditions, such as at elevated temperature (\textgreater 1100 \si{\degreeCelsius}) and oxygen pressure on the order of Torr,\cite{Base1965} and at temperatures ranging from 20 to 500 \si{\degreeCelsius} under atmosphere pressure.\cite{Warren1996} \ce{WO_x} NWs were readily found when tungsten (foil, wire, or powder) is oxidized under various conditions.\cite{Zhu1999,Karuppanan2007,Hsieh2010} However the study on tungsten powder oxidation behavior between intermediate temperature range and under low pressure is still rare. This thesis studied the oxidation of tungsten powders with diverse size within temperature range from 500 to 1000 \si{\degreeCelsius} and under several mTorr oxygen partial pressure. It was illustrated that using tungsten powder as seed is an economic approach to obtain high yield of \ce{WO3} nanowires at relatively lower temperature. It was also demonstrated that there was an optimal tungsten powder size under current experimental conditions for seeded growth. This observation will provide some insight on the role of tungsten powder as source material in CVD growth of \ce{WO_x}.

Commercial available tungsten powders with different size are usually associated with purity variation as well. Four kinds of tungsten powders were used as precursor to prepare \ce{WO3} NWs, as already summarized in Table~\ref{tab:powder}. The dimensions of tungsten powder were obtained by measuring the average size in SEM graphs. A systematic investigation was performed on the oxidation behavior of tungsten powder to evaluate the temperature effect, size-dependence and influence of oxygen partial pressure.

In a typical oxidation experiment, tungsten powders were loaded into the uniform heating zone and the sealed chamber was pumped down to an ultimate pressure of $5\sim8$ mTorr. Then oxygen flow varying between 1 sccm to 10 sccm was admitted from upstream inlet. With 10 sccm UHP Ar (99.999\%) as carrier gas, the overall pressure reached to about 100 mTorr. The heating temperature (500 to 750 \si{\degreeCelsius}) was ramped up in 30 min and lasted for 30 min. Then the heating power was turned off and the chamber was allowed to naturally cool down to room temperature.

% seed optimal
\begin{figure}[htb]
\centering
\includegraphics[width=0.7\textwidth]{JAP-2column_Fig1.jpg}
\caption[W powder oxidation: temperature effect]{W powder oxidation: temperature effect. SEM graphs of 99.9\% (3N) tungsten powder oxidization at different temperatures of a) 500 \si{\degreeCelsius}, b) 600 \si{\degreeCelsius}, c) 650 \si{\degreeCelsius}, d) 750 \si{\degreeCelsius}, showing the optimal temperature for local formation of nanowires is between 600-650 \si{\degreeCelsius}. Oxygen flow rate is 1 sccm.}
\label{fig:pdtemp}
\end{figure}

Fig.~\ref{fig:pdtemp} illustrated the effect of temperature on the morphological change and surface nanowires formation of 3N powder. At 500 \si{\degreeCelsius}, most tungsten powder retained its original shape and a layer of tiny dense NWs begun to grow. When temperature was increased to 600 \si{\degreeCelsius}, 3N powder started to crack with longer NWs on the isolated surface. Further increase of temperature lead to irregular shapes of tungsten power and aggregation of NWs, giving rise to the nanorods and bunched or bundled structures. It could be determined from the morphology variation that the optimal seeded growth temperature for 3N powder was in the range of 600 to 650 \si{\degreeCelsius}.
% seed optimal
\begin{figure}[htb]
\centering
\includegraphics[width=0.7\textwidth]{JAP-2column_Fig3.jpg}
\caption[W powder oxidation: size effect]{W powder oxidation: size effect. SEM graphs illustrating the oxidization of four different size of tungsten powders at 600~\si{\degreeCelsius} and 1 sccm oxygen flow. a) 17 $\mu$m, b) 32 $\mu$m, c) 3.3 $\mu$m, d) 1.5 $\mu$m.}
\label{fig:pdsize}
\end{figure}

Fig.~\ref{fig:pdsize} depicted the oxidation of different sizes of tungsten powder under the same experimental conditions. In contrast to the morphology of 3N powder shown in Fig.~\ref{fig:pdtemp}, 3N5 powder surface is primarily covered with sub-micron particles as well as some short tiny NWs; whereas 4N5 and 5N powder were thoroughly oxidized, showing branched flowers feature. This dramatic difference could be explained in terms of surface energy and oxygen diffusion. With smaller dimension, the increased surface-to-volume ratio and short diffusion path both lower the energy barrier of oxidation.\cite{tungsten1999} It was logical to deduce that higher temperature or increased oxygen level might favor the NWs formation on 3N5 powder. When it comes to seeded growth, however, the powder size distribution was an important factor to give uniform NWs deposition. Since the size distribution of 3N powder is more uniform than that of 3N5 powder, this study used the former as seeds.
% seed optimal
\begin{figure}[htb]
\centering
\includegraphics[width=0.7\textwidth]{JAP-2column_Fig2.jpg}
\caption[W powder oxidation: oxygen pressure]{W powder oxidation: oxygen pressure. SEM graphs of 3N tungsten powder oxidization at 600 \si{\degreeCelsius} under different rates of oxygen flow: a) 1 sccm, b) 2 sccm, c) 3 sccm, d) 10 sccm. The oxygen partial pressures were 13 mTorr, 23 mTorr, 32 mTorr, and 82 mTorr, respectively with background pressure subtracted.}
\label{fig:pdoxy}
\end{figure}

Fig.~\ref{fig:pdoxy} depicted the morphology change of 3N powder with respect to varied oxygen partial pressure. When the oxygen flow is lower than 3 sccm, 3N powder almost stayed as the same, with cracks separating the dense layer of NWs. When oxygen flow is increased to 10 sccm, the 3N powder exemplified an enlarged version of that for 4N5 or 5N powder under 1 sccm oxygen flow. This observation again supported the surface energy explanation.

With all above oxidation experiments, favorable conditions for local growth of NWs were extracted to perform seeded growth. The seeded growth was performed by placing high purity tungsten powders (4N5 or 5N) in heating zone and receiving substrate with 3N tungsten powders in downstream location where the temperature was about 600 \si{\degreeCelsius}. The temperature profile and source and substrate locations remain essentially the same as in Fig.~\ref{fig:wogrow} except the presence of 3N powder on substrate. In all experiments, tungsten powders were uniformly distributed by sliding two pieces of substrates and the growth time was kept at 4 h.
% sg sem
\begin{figure}[htb]
\centering
\subfloat[]{\label{fig:sga}\includegraphics[width=0.4\textwidth]{wox_sg_a.jpg}}\hspace{0.04\textwidth}
\subfloat[]{\label{fig:sgb}\includegraphics[width=0.4\textwidth]{wox_sg_b.jpg}}
\caption[SEM characterization of \ce{WO3} seeded growth]{SEM characterization of \ce{WO3} seeded growth. (a) SEM graphs of \ce{WO3} NWs on \ce{SiO2}/Si substrate. (b) A high magnification view showing uniform NW growth and close-up view of one NW.}
\label{fig:woseedsem}
\end{figure}
As shown in Fig.~\ref{fig:sgb}, dense NWs array was obtained on tungsten powder seeds with individual wires of length up to 10 $\mu$m and diameter about 50 to 200 nm, according to the measurements made in the close-up view. Each tungsten powder stood as independent growth site (Fig.~\ref{fig:sga}) with island-layer growth on the substrates, a common feature without using tungsten powder as seed under current experimental conditions. It was occasionally observed that NWs growth was initiated adjacent some tungsten powders. This phenomenon was correlated to the local trap of vapor flow since it was more often found among the enclosed area by tungsten powders. It is also found that the diameter of NWs decrease as the distance between powders and upstream edge increases. This is a combination effect of lower temperature and reduced \ce{WOx} growth species supply. Similar phenomena were observed in other studies. \citeauthor{Thangala2007} reported that a decrease in NW density with increasing substrate temperature, and an increase of NW density with increasing partial pressure of oxygen.\cite{Thangala2007}

% seeded edx 
\begin{figure}[htb]
\centering
\includegraphics[width=0.5\textwidth]{wo3_seed_edx}
\caption[Composition analysis on seeded growth \ce{WO3} NWs]{Composition analysis on seeded growth \ce{WO3} NWs. EDX spectroscopy on seeded growth \ce{WO3} NWs.}
\label{fig:woedx}
\end{figure}
\gls{edx} analysis on the seeded growth \ce{WO3} NWs is shown in Fig.~\ref{fig:woedx}. Only W and O elements were detected on the NW array. The background level from 3 to 8 keV was a manifestation of the continuous components of W X-ray spectrum. 

% sg raman xrd
\begin{figure}[htb]
\centering
\subfloat[]{\label{fig:sgxrd}\includegraphics[width=0.45\textwidth]{xrd_cs_before}}\hspace{0.04\textwidth}
\subfloat[]{\label{fig:sgram}\includegraphics[width=0.45\textwidth]{wox_raman_1}}
\caption[Characterization of seeded growth \ce{WO3}: XRD and Raman]{Characterization of seeded growth \ce{WO3}: XRD and Raman. (a) XRD pattern of as-prepared sample indicating the \ce{WO3} phase and the presence of metallic core. (b) Raman spectrum on NWs region showing the feature of \ce{WO3}.}
\label{fig:woseedxrd}
\end{figure}

Fig.~\ref{fig:sgxrd} is the XRD spectrum of one typical sample. The peaks under circular symbol were identified to be the monoclinic \ce{WO3} phase (ICDD PDF 01-083-0950, \emph{a}=7.30084 \AA, \emph{b}=7.53889 \AA, \emph{c}=7.6896 \AA, $\beta$=90.89$^\circ$), while the peak under the triangular symbol was indexed to cubic tungsten phase (ICDD PDF 04-16-3405, \emph{a}=3.157 \AA), in agreement with the EDX analysis (Fig.~\ref{fig:woedx}). This means that during the \ce{WO3} seeded growth of 4 h heating at 1000 \si{\degreeCelsius}, the tungsten powder in downstream low temperature region (600-700 \si{\degreeCelsius}) is not entirely oxidized. Micro-Raman scattering spectroscopy was performed on the as-synthesized sample as well. During Raman examination, the laser spot was carefully focused onto the NWs on powders and several inspections on different positions were observed to ensure the reproductivity of spectra data. As shown in Fig.~\ref{fig:sgram}, five distinct bands were well resolved, with peaks located at 131, 265, 327, 711 and 803 \si{cm^{-1}}, respectively. This pattern was typical features of \ce{WO3}, consistent with previous study.\cite{Salje1975a,Dixit1986} The high background level probably arises from the metallic core.
% sg tem
\begin{figure}[htb]
\centering
\includegraphics[width=0.9\textwidth]{JAP-2column_Fig5major.jpg}
\caption[Characterization of \ce{WO3}: TEM]{TEM Characterization of \ce{WO3}: (a) TEM image of one nanowire with diameter about 40 nm, and (b) HRTEM images showing the spacing is 0.38 nm, corresponding to (002) plane distance.}
\label{fig:woseedtem1}
\end{figure}

TEM specimen was prepared by using carbon grid to slightly scratch the as-grown sample. Fig.~\ref{fig:woseedtem1} shows the feature of majority NWs. The growth direction is determined to be perpendicular to (002) plane. The streaking in SEAD pattern presumably arises from stacking defaults during \ce{WO3} NW growth. This study also found some NWs exhibit high crystalline quality, as revealed by the TEM analysis in Fig.~\ref{fig:woseedtem2}. The NW grew normal to (002) plane with a measured lattice spacing of 3.79 \AA, which is favorably compared to the XRD peak at $23.07^\circ$ (7.7103 \AA). The sharp SEAD pattern and clear phase contrast in HRTEM demonstrated are both strong evidence of good crystallinity. This formation indicated current growth parameters have promising potential to obtain highly crystalline \ce{WO3} NWs in large scale. 

% sg tem
\begin{figure}[htb]
\centering
\includegraphics[width=0.9\textwidth]{JAP-2column_Fig5minor.jpg}
\caption[Characterization of \ce{WO3}: TEM cont]{TEM Characterization of \ce{WO3}: (a) TEM image of one nanowire, the diameter is about 70 nm, and (b) HRTEM images showing the spacing is 0.379 nm, corresponding to (002) plane distance.}
\label{fig:woseedtem2}
\end{figure}

In regarding to the formation of NWs on tungsten powder itself, this study assumes the driving force is related to interfacial strain between W and \ce{WOx}. Oxidation of tungsten proceed slowly at room temperature and an oxide layer of 100 \si{\angstrom} was found on the surface of tungsten foils.\cite{Warren1996} The tungsten powder used in current study would be covered by a thin oxide layer as well. During oxidation, different oxidation rates exist for different crystallographic orientations on the tungsten powder. Oxidation occurring at boundaries and defects are preferred thermodynamically.\cite{You2010} Compressive strain will gradually accumulate at the tungsten oxide/tungsten interface, which might limit the diffusion rate of oxygen at temperature lower than 500 \si{\degreeCelsius}.\cite{tungsten1999} At elevated temperature, cracks will primarily occur, as observed in Fig.~\ref{fig:pdtemp}. When heated up, tungsten and the oxide shell will probably relax the strain by converting into sub-stoichiometric NWs, a similar process as suggested by \citeauthor{Klinke2005} in the chemically induced strain growth of tungsten oxide NWs.\cite{Klinke2005} It is worth noting that tungsten oxide nanowires could also formed when \ce{WO3} is reduced.\cite{Sarin1975} The elongation of \ce{WO_x} is thermodynamically favorable during the conversion from metallic tungsten to tungsten oxide as well. Local evaporation-condensation process might also contribute to the formation of NWs on tungsten powder.

The enhanced yield of NWs obtained via seeded growth could be explained by a vapor-solid (VS) mechanism. External supply of growth species will condense onto the powders and substrate simultaneously, promoting the elongation of NWs on power as well as resulting island-layer growth on substrate. The local NW density in oxidation experiment was much higher than that of seeded growth. It is reasonable to presume that during seeded growth, several NWs in a small region on powder will coalescence, as evidenced by the bundled structures. At last, the author would like to point out that when low purity tungsten powder (3N) was used as source, sodium tungsten oxide nanowires were found to be dominant in the final product. The details have been published.\cite{Sheng2014} It seems surprising that when 3N powder was used as seeds, only \ce{WO3} nanowires were obtained. This result was attributed to the lower temperature and significantly reduced amount of 3N powder used in the seeded growth, compared with the conditions used in Ref.\cite{Sheng2014}. The source material in seeded growth is not limited to high purity tungsten powder. Instead, any material that could produce appropriate growth vapor could be employed, indicating the versatility of this approach.

\subsection{Tungsten Oxides: Other Methods}

Another interesting approach for \ce{WO3} NW growth were also found in this study, producing \ce{WO3} NWs of high crystalline quality. The method was supposed to be related to the vapor flow dynamic in the deposition region. As shown in Sec.~\ref{sec:nawox}, ordinary transport growth using high purity W power mostly leads to thin film growth on one substrate of 1 in long. However, it was found that when two shorter substrates were closely placed together, the \ce{WO3} NW yield was significantly enhanced in the second substrate located in lower temperature region. For instance, substrate 1 (Sub1) is located at 6.75 to 7 inch (Fig.~\ref{fig:wogrow} horizontal axis on page~\pageref{fig:wogrow}), and substrate 2 (sub2) is placed in close vicinity of downstream end of sub1. W source moves to 5 inch, and 0.3 sccm \ce{O2} is used to reduce the \ce{WO_x} vapor pressure. Deposition on sub1 is characterized by thin film, similar to those shown in Fig.~\ref{fig:wox4n5}b. However, deposition on sub2 (Fig.~\ref{fig:fg1}) is in sharp contrast to that shown in Fig.~\ref{fig:wox4n5}c, although they are both located in the low temperature region of reaction chamber. The NW density is dramatically increased. The NWs have length about 1 to 2 $\mu$m and diameter about 200 nm. Close-up view (inset of Fig.~\ref{fig:fg2}) shows the NWs have clear facets, implying its high crystalline quality.
% fg sem
\begin{figure}[htb]
\centering
\subfloat[]{\label{fig:fg1}\includegraphics[width=0.45\textwidth]{wox_flow1}}\hspace{0.04\textwidth}
\subfloat[]{\label{fig:fg2}\includegraphics[width=0.45\textwidth]{wox_flow2}}
\caption[Characterization of flow growth \ce{WO3}: SEM]{Characterization of flow growth \ce{WO3}: SEM. (a) Low magnification SEM image showing dense array of NWs (b) High magnification SEM image of NWs grew out of layer and close-up view on individual wire.}
\label{fig:wogrowsf}
\end{figure}
The growth mechanism here can be explained by aforementioned VS process as well. Both local flow and temperature play an vital role in the formation of dense NW array. Splitting one substrate into two presumably introduces some fluctuation in the vapor flow. It is well known that in lamellar flow region, there is a velocity boundary above the side wall. In spite of the high crystalline quality, the \ce{WO3} is not long enough and yield is limited by the evaporation of source materials; therefore this approach is not seriously pursued in this study. 

%to be added, NaOH WO3 results. 

\section{Summary}

A systematical investigation of \ce{WO3} nanowire growth using CVD method with tungsten powders as precursor was performed. Four kinds of tungsten powder sources were used, and the sodium impurity effect was throughly studied. \ce{Na5W14O44} NW was first reported from this study; the crystal structure was investigated in details, and HRTEM images were obtained. Besides, the Raman vibrational spectroscopy of \ce{Na5W14O44} was also revealed. The presence of sodium tungsten oxide phase indicates that the CVD growth of \ce{WO3} is extremely sensitive to Na contents. The insight of \ce{WO3} growth kinetics we have gained from this study could help establish a systematic connection between the interaction of transition metal oxide and alkali metal ions, thereby turning this deleterious effect into a beneficial approach of controlled growth. In fact, favorable results have been observed on both \ce{WO3} and \ce{MoO3} 1D nanostructures syntheses using NaOH treated Si substrates. Details will be presented in Chapter 4 and 6. Moreover, two alternative approaches of \ce{WO3} NWs were also attempted, and the seeded growth method proved effective for large scale growth. 



