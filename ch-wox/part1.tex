
\section{Introduction}

Tungsten oxide (\ce{WO_x}) is an important functional materials with distinctive properties and technology applications. Intense research interest was rekindled by the discovery of \gls{ec} effect in \citeyear{Granqvist1993}.\cite{Granqvist1993}  Nano-engineering \ce{WO_x} bring more possibility and flexibility to its already rich characteristics, therefore resulting research efforts spanning multiple fields of scientific community. Besides smart window based on EC effect, \ce{WO_x} is also well investigated for several other significant applications: photoelectrochemical cell for solar energy conversion and storage, photocatalysis for hydrogen evolution reaction, chemical and biological sensing based on gasochromic effect. In addition, a few young fields are not thoroughly explored: field emission, optical storage, thermoelectrcity, ferroelectricity, and superconductivity.

In this chapter, we will first review the literature to date for both \ce{WO_x} and \ce{Na_xWO3}, then divide the summaries into a) technological application, b) crystal structure and electronic properties and c) synthesis approaches. Then we present our growth methods and results. And we conclude with a brief summary.

\subsection{Rich and useful technological appeals}

Electrochromic effect of tungsten oxide means the coloration (deep blue) and bleaching (transparent) states of \ce{WO3} are reversibly switched upon forward and backward voltage, and the coloration or bleached states remain after disconnecting the voltage. On basis of this property, \gls{ecd} of tremendous energy saving potential is conceptualized and some products have already been commercialized (i.e. smart window\footnote{``We did a case study in five cities, and the average savings in commercial buildings are about 25 percent of the heating, ventilation, and air-conditioning energy use annually," says CEO of View, Inc.}).

In the past decades, tons of works have been devoted to understanding the chromogenic phenomena in \ce{WO3}. We highlight several key results as reviewed by ,
\begin{itemize}
    \item coloration and bleaching can also be stimulated by other routes, such as by UV irradiation, thermal treating, heating in vacuum, reducing atmosphere etc
    \item no electrical coloration occurs in vacuum, but other routes still works
    \item coloration is associated with a proportional increase in conductivity
    \item coloration spectrum is essentially similar in all cases except small variation in peak position and FWHM
    \item coloration is structure sensitive and most efficient in amorphous films
\end{itemize}

Sofar there is no unifying model that could reconcile these contradictory experimental observations. Among all the developed models, polaronic absorption is the most widely accepted mechanism for coloration. In polaron model\footnote{When a free electron travels through a polar solid, it creates a local lattice displacement (longitudinal optical phonon clouds) due to the coulombic interaction with neighboring ions. This local distortion and the electron together is equivalent to a new elementary exciton of the crystal, and is named as polaron}, the intercalation of \ce{M^+} (M = H, \ce{Li}, or Na) ions into \ce{WO3} films is accompanied with the formation of small polarons ($r_p$ comparable to unit cell size) and formal reduction of some \ce{W^{6+}} sites to \ce{W^{5+}}, as depicted in Eq.~\ref{eq:ec}. During the intercalation process, the \ce{M^+} ions enter into these vacant sites.\cite{Hepel2008} Coloration occurs when the polaron band .

\begin{align}\label{eq:ec}
x\ce{M+} + x\ce{e-} +  \ce{$\alpha \hyphen$WO$_{3-y}$}= \ce{$\alpha \hyphen$M$_x$WO$_{3-y}$},
\end{align}

And the polaron binding energy ($E_p$) is given by
\begin{align}
E_p = - \frac{e^2}{2r_p} (\epsilon_\infty^{-1} - \epsilon_{st}^{-1}),
\end{align}

where $\epsilon_\infty$ and $\epsilon_{st}$ are optical and static dielectric constants respectively, and polaron radius $r_p$, which specifies how far the lattice distortion extends, is related to polaron density $N_p$ by $r_p = \frac{1}{2}\sqrt[3]{\pi/6N_p}$. However, the polaron model has difficulty in estimating $r_p$. The asymmetric optical absorption spectrum are often characteristic of large polarons, and dielectric constants ($\epsilon_\infty = 6.52,\epsilon_{st} > 50$,\cite{Deb2008}) suggests the formation of bipolaron. Moreover, polaron model does not take oxygen vacancy into account, which plays a vital role in the nonstoichiometric tungsten oxides. For instance, it is observed that \ce{WO_{3-y}} films are metallic and conductive for $y > 0.5$, blue and conductive for $y = 0.3 \sim 0.5$, and transparent and resistive when $y < 0.3$, regardless of the preparation methods.\cite{Chatten2005}

Therefore another model in analogy to the F-color center is proposed. Color center model assumes the presence of oxygen vacancy $V_O^0$ is associated with \ce{W^{4+}} or 2\ce{W^{5+}} states. This defect level is expected to be inside or near the valence band. When one electron is removed from this level, $V_O^0$ is converted to $V_O^+$.The positively charged vacancy exerts coulombic repulsion to the nearest W-ions, which results in a displacement of the neighboring W-ions and an upward shift of the defect level into the bandgap, thereby creating a color center. The optical transition from $V_O^+$ to $V_O^{2+}$ (a state within the conduction band)
contributes to coloration.\cite{Deb2008}

So the polaron and color center models both agree on that \ce{W^{5+}} and its transition is responsible for the coloration, but disagree on how this \ce{W^{5+}} state is created (foreign ion reduction in polaron model and oxygen vacancy in color center model) and the corresponding energy levels. A modified polaron model is proposed to include \ce{W^{4+}} states in host lattice. Coloration mechanism is represented by Eq.~\ref{eq:cl_bl1} and \ref{eq:cl_bl2}, which described the polaron hopping from one site to another.\cite{Chatten2005}

\begin{align}
h\nu +\ce{W^{5+}(A)} +  \ce{W^{6+}(B)} &\rightarrow \ce{W^{5+}(B)} + \ce{W^{6+}(A) + E_{phonon}} \label{eq:cl_bl1}\\
h\nu +\ce{W^{5+}(A)} +  \ce{W^{4+}(B)} &\rightarrow \ce{W^{5+}(B)} + \ce{W^{4+}(A) + E_{phonon}} \label{eq:cl_bl2}
\end{align}

Similar scenarios occurs in the gasochromic effect. Two models, double injection and color center, arise to account for the coloration upon exposure to certain gases. Both consents to Eq.~\ref{eq:cl_bl1}. But there is a disagreement on the final states. Double injection supports the formation of tungsten bronze \ce{H_xWO3} while color center insists on the inward diffusion of oxygen vacancy and outward diffusion of water molecules. Both have been substantiated experimentally. Therefore the exact mechanism in \ce{WO_x} still requires further investigation. The author believes the resolution to a large extent depends on the phase transitions, as discussed in Section.~\ref{sec:wonawo}.

photocatalytic applications in solar hydrogen generation and organic pollutant degradation.

photocatalyst\cite{Macphee2010}, photoelectrochemical energy application \cite{Su2010}

Photocatalytic activity occurs when a semiconductor in aqueous solution is illuminated by photons of energy larger than the band gap, then electron-hole pairs generate free radicals, (i.e. \ce{OH.}) which enable further reactions. For water splitting using solar energy, the band gap should be within $2.0 \sim 3.0 $ eV, and CB edge should be more negative than reduction potential of \ce{H^+/H2}, whereas the VB top should be more positive than the oxidation potential of \ce{H2O/O2)}.\cite{Wang2012} The valence band holes (\ce{h^+}) oxidize water to oxygen and conduction band electrons propel hydrogen generation, as depicted in Eq.~\ref{eq:hervb} and \ref{eq:hercb}.

\begin{align}
4\ce{h^+} +  \ce{H2O} &\rightarrow \ce{O2} + 4\ce{H^+} \label{eq:hervb}\\
4\ce{H^+} +  4e^- &\rightarrow 2\ce{H2} \label{eq:hercb}
\end{align}

As shown in Fig.~\ref{fig:woxnhe}, \ce{WO3} has CB edge positioned slightly more positive than reduction potential of \ce{H^+/H2}(versus \gls{nhe}), and VB edge much more positive than the oxidation potential of \ce{H2O/O2}. So the photo-cleavage of water cannot be accomplished by \ce{WO3} alone. Nevertheless, a tandem cell approach by \ce{WO3} film and dye-sensitized \ce{TiO2} has been demonstrated with an efficiency of 4.5\%.\cite{Michael1999} It is worth noting that green plants also have two photosynthetic systems connected in series, one for oxidation of water into oxygen and the other for fixation of carbon dioxide.

% wo3 vs NHE
\begin{figure}[htb]
\centering
\includegraphics[width=0.7\textwidth]{woxnhe.jpg}
\caption[Bands positions of \ce{WO3} versus NHE]{Bands positions of \ce{WO3} in contact with aqueous electolyte at pH 1. adapted from Ref\cite{Gratzel2001}}
\label{fig:woxnhe}
\end{figure}

Moreover, favorable oxygen evolution of \ce{WO3} brings good performance in degradation of organic compounds\cite{Hepel2001,Luo2001,Watcharenwong2008}. The formation of long-lived holes is recognized as a key requirement.\cite{Pesci2011} Besides, \ce{WO3} is remarkably stable in acid, making it a significant candidate for treating water pollutant caused by organic acids.\cite{Monllor-Satoca2006}



the ubiquity of \ce{WO6} octahedra is essential for not only the optical properties but the ability to insert and extract ions in the EC oxides, due to the tunnels in three dimensions serving as path for transport of small ions.
The intercalation of hydrogen or alkali ions into \ce{WO3} created electron donor level. By absorbing the red part of incident spectrum, electrons at donor level make transition to the conduction band, causing the blue coloration in \ce{H_xWO3}.

\citeauthor{Wang2009a} mentioned that amorphous \ce{WO3} can only be used in lithium-based electrolytes due to its
in-compact structure and high dissolution rate in acidic electrolyte solutions. Electrochromic materials that can endure acidic electrolytes without degradation should be developed. Crystalline \ce{WO3} nanostructures with their much denser structures and small particle sizes are promising to be used as suitable electrochromic material in acidic electrolytes.

Characterization of ECD (work like a thin-film batteries) includes transmission measurement and associated EC calculation, charge-discharge time, current-time curve and the fitting of obtained data.

The coloration efficiency (CE) represents the change in the optical density (OD) per unit charge density ($Q/A$, in units of \si{\cm^2\per\coulomb}) during switching and can be calculated according to the formula:
\begin{equation}
CE = \frac{\Delta~OD}{(Q/A)} [cm^2/C],
\end{equation}
where OD = $log(T_{bleach}/T_{color})$. The EC and optical density depend on the wavelength and are usually higher in the near IR than in the visible region.
Using Ohm's law($U_s = IR = RQ/t_s$) with switch voltage $U_s$, resistance R and surface area A, switching time $t_s$ could be estimated as
\begin{equation}
t_s = \Delta~OD\cdot A \cdot R /(CE\cdot U_s).
\end{equation}

its one dimensional (1D) nanostructure has attained intensive research efforts in recent years due to the potential applications in advanced nano-electric and nano-optoelectronic devices.

\begin{quote}
a viable electrochromic smart window must exhibit a cycling life time \textgreater $10^5$ cycles, corresponding to an operation life at 10 - 20 years.
\end{quote}






\subsection{Crystal structures and properties of tungsten oxides and sodium tungsten oxides}\label{sec:wonawo}

\textbf{\ce{WO3}} Tungsten trioxides crystalize in multiple phases. The basic building block is \ce{WO6} octahedra.\footnote{Tungsten, with its electronic configuration as \ce{(Xe)4f^{14}5d^{4}6s^{2}}, has empty 5d and 6s orbitals in its +6 oxidation state.} \ce{WO3} crystal structure consists of \ce{WO6} octahedra joined at their corners, which may be considered as a perovskite structure of \ce{CaTiO3} with all \ce{Ca^{2+}} sites vacant. A representative lattice structures is illustrated in Fig.~\ref{fig:wo3oct}. These distorted \ce{WO6} octahedra adapts different tilting angles in different phases and edge-sharing octahedra also arises. The temperature-dependent phase transition and corresponding lattice constants in bulk form is summarized in Table.~\ref{tab:wo3phase}.\cite{Zheng2011} And a firm assignation of space group to monoclinic phases are still in debate.\cite{Chatten2005} It is noticed that the lattice parameters obtained via \emph{ab initio} calculation closely match the experimental values.\cite{Migas2010a} The phase transition scenarios in nanostructured \ce{WO3} are supposed to be much more sophisticated. Within Gibbs-Thomson frame work, one can generally expect lower transition temperature than their bulk counterparts due to enhanced surface energy. Temperature-dependent Raman spectroscopy provided support for this deduction.\cite{Boulova2002}

% wo3 phases
\begin{table}[htb]
\centering
\caption{\ce{WO3} phases}\label{tab:wo3phase}
\begin{tabular}{lccccc}
\toprule
&&&\multicolumn{3}{c}{Lattice constants \AA} \\
\cmidrule(l){4-6}
 Symbol    & Temperature (\si{\degreeCelsius}) & Phase & a & b & c   \\
\midrule
$\epsilon$-\ce{WO3} & $ -140 \sim -50$  & monoclinic II & 7.378 & 7.378 & 7.664  \\
$\delta$-\ce{WO3} & $-50 \sim 17$  & triclinic & 7.309 & 7.522 & 7.686  \\
$\gamma$-\ce{WO3} & $17 \sim 330$  & monoclinic I & 7.306 & 7.540 & 7.692  \\
$\beta$-\ce{WO3} & $330 \sim 740$  & orthorhombic & 7.384 & 7.512 & 3.846  \\
$\alpha$-\ce{WO3} & $> 740$  & tetragonal & 5.25 & NA & 3.91  \\
$h$-\ce{WO3} &  $<400$  & hexagonal & 7.298 & NA & 3.899  \\
\bottomrule
\end{tabular}
\end{table}

\begin{figure}[htb]
\centering
\includegraphics[width=0.4\textwidth]{octwo3.jpg}
\caption[Octahedra model of \ce{WO3}]{Octahedra model of \ce{WO3}}
\label{fig:wo3oct}
\end{figure}


\ce{WO3} is wide gap n-type semiconductor with valence band top featuring $2p$ states of oxygen and conduction band bottom arising primarily from $5d$ states of tungsten with some mixing of oxygen $2p$ states.\cite{Gillet2004} \citeauthor{Migas2010a} maintained there is essentially identical band dispersion near the gap region in case of $\epsilon$-\ce{WO3}, $\delta$-\ce{WO3}, $\gamma$-\ce{WO3} and $\beta$-\ce{WO3}.\cite{Migas2010a} When there is oxygen vacancy, the Fermi level moves into the conduction band and the gap shrinks by about 0.5 eV. \citeauthor{Migas2010a} also pointed out the flat bands at VBM and CBM could lead to poor transport of holes and electrons, thus may compromising the function in photoelectrochemical cells.

 \citeauthor{Chatten2005} also studied the oxygen vacancy in different phases of \ce{WO3}.\cite{Chatten2005}

Nonstoichiometric tungsten oxides \ce{WO_x} (i.e. \ce{WO_{2.92}}, \ce{WO_{2.87}}) are known as Magn$\acute{e}$li phases.

We do not discuss tungsten oxide hydrates (\ce{WO3.nH2O}) in this work since the product of thermal CVD approach is not plagued with this complexity. It's necessary, however, to deal with hydrated \ce{WO3} in the liquid synthesis routes, as indicated in Section.~\ref{sec:woxgrowth}.

Theoretical computation of electronic band structures for \ce{WO_x} proves difficult due to the aforementioned phase transition.

 oxygen deficiency, structure change, electronic properties vary according.

% band gap table
\begin{table}[htb]
\centering
\caption{Tungsten oxides band gap }\label{tab:wo3eg}
\begin{tabular}{lccr}
\toprule
Phase & Experimental (eV) & Theory (eV) & Remarks  \\
\midrule
amorphous \ce{WO3} & 3.2  & NA &    \\
monoclinic bulk \ce{WO3} &  2.6   & 1.73\cite{Migas2010a}  &    \\
tetragonal bulk \ce{WO3} &     & 0.66 \cite{Migas2010a}&    \\
nano-\ce{WO3} & 2.6$\sim$3.2  & NA &    \\
nano-\ce{WO_{3-x}} & NA  & NA &    \\
\bottomrule
\end{tabular}
\end{table}

Tungsten bronzes was coined by Wohler in 1837.\cite{Deb2008} \ce{Na_{x}WO3}

\subsection{Synthesis strategies}\label{sec:woxgrowth}

We will first give a brief review on the synthesis of \ce{WO3}. As the chemical formula suggested, the most straightforward way is heating metallic tungsten in various forms (i.e. powders,\cite{Zhou2005a,Cao2009,Hsieh2010} foils and wires). Actually \ce{WO_x}\footnote{$x$ is between 2 and 3.} NWs was observed when directly heating W wires.\cite{Gu2002a} Due to the extreme high melting point of W, it requires very high temperature (say, above 1100 \si{\degreeCelsius}) to produce large amount of growth species. Therefore scalable growth seems difficult. The DC current heating proved to be a route with potential large yield. Tungsten wire or filament is connected to a voltage source, and substrate is positioned in proximity to the heated W wire.\cite{Lingfei2006,Thangala2007,Chang2007}

Tungsten oxides powder can also been used as precursor to prepare \ce{WO_{x}} \glspl{nw}.\cite{Huang2008a,Wang2009} Hydrogen-containing agents (i.e. water, \ce{H2},or methane\cite{Klinke2005}) were often involved into the reaction.\cite{Baek2007,Karuppanan2007} The potential benefit is \ce{WO3} has a much lower melting point ($\approx 1470$\si{\degreeCelsius}) than that of W.

Direct oxidation and \ce{WO3} conversion could be grouped as addition approach. In contrast, decomposition method is also feasible.
\citeauthor{Pol2005} obtained \ce{W18O49} nanorods by thermal dissociation of \ce{WO(OMe)4} at 700 \si{\degreeCelsius}, and \ce{WO3} by the annealing at 500 \si{\degreeCelsius} at air atmosphere.\cite{Pol2005}

Spray pyrolysis is a typical aerosol-assisted CVD technique. Typical process flow is: the precursor solution is pumped to an atomizer, and then sprayed through the carrier gas as a fine mist of very small droplets onto heated substrates. Subsequently the droplets undergo evaporation, solute condensation, and thermal decomposition, which then result in film formation.\cite{Zheng2011}

Main reactions include: \cee{W + O2 -> WO_{3-x}} \cee{W + H2O -> W_{3-x} + H2}

the energetic sources are ion bombardment, electron beam, laser ablation, and combustion flame\cite{Rao2011}.

The sol–gel process is a well-known, intensively studied wetchemical technique that is widely used in materials synthesis. This method generally starts with a precursor solution (the ``sol") to form discrete particles or a networked gel structure. During the course of gelation (aging process), various forms of hydrolysis and polycondensation take place.

Hydrothermal method has been an important route to synthesize a diversity of nanomaterials (i.e. ). Preparation of \ce{WO_{3-x}} has also been demonstrated hydrothermally.\cite{Lee2003,Gu2007} Usually, the precursor (\ce{H2WO4}) is mixed with other reactants (sulfides or certain organic acid) in solution and pH value is adjusted as another control degree of freedom. Then the solution is transferred into a sealed container (i.e. Teflon autoclave) and maintained at elevated temperature ($120 \sim 300$ \si{\degreeCelsius}) for tens of hours. Finally the product is separated from the solution and dried, layered \ce{WO3.nH2O} flakes are usual products.

In addition, doped \ce{WO3} was also demonstrated

 The composition and phase of final product highly depend on the synthesis conditions.

%\newgeometry{margin=1in}
\begin{landscape}
\begin{table}[htb]
\centering
\caption{Tungsten Oxides Growth and Application}\label{tab:wox}
{\footnotesize
\begin{tabular}{lp{3.5in}p{2.5in}c}
\toprule
composition  &  methods & highlights &  reference  \\
\midrule
\ce{WO3} & hot W filament (above 1500 \si{\degreeCelsius}) in Ar/\ce{O2} flow  & Cubic phase, PL, resistivity measured & \cite{Thangala2007} \\
\addlinespace[0.5em]
& W filament DC heating in \ce{NH3} or \ce{N2}/\ce{H2} flow  & multi phases, 100mg per batch, stable dispersion in both organic and aqueous solvents & \cite{Chang2007} \\
\addlinespace[0.5em]
& W powder heating Ar/\ce{O2} flow  & triclinic phase, CL at 370,415nm, UV-Vis, 3eV & \cite{Hsieh2010} \\
\cmidrule(l){2-4}
& \ce{WOx} film in \ce{H2} and \ce{CH4} flow  & monoclinic phase, tungsten carbide is the key & \cite{Klinke2005} \\
\cmidrule(l){2-4}
& \ce{Na2WO4.2H2O} by hydrothermal heating to test for line break & 3 different morphologies and photodegradation & \cite{Rajagopal2009}  \\
& \ce{H2WO4.2H2O}, \ce{H2O2} and poly(vinyl alcohol) etc by solvothermal  & multi phases deposition of FTO with varied bandgaps  & \cite{Su2010}  \\

\midrule
\ce{W18O49} & \ce{KOH} etching of W tips followed by heating in Ar flow  & oxygen from leakage, VS process& \textcite{Gu2002a} \\
\addlinespace[0.5em]
& \ce{NW(CO)6}, \ce{Me3NO.2H2O} and oleylamine by hydrothermal  &  RT PL at 350nm and 440nm & \cite{Lee2003}  \\
\bottomrule
\end{tabular}
}
\end{table}
\end{landscape}
%\restoregeometry

