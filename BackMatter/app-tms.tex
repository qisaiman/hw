\chapter{bulk tungsten disulfide}



Tungsten disulfide (\ce{WS2}) is a group VI dichalcogenide semiconductor compound. The molecular weight is 249.97 \si{g\per \mole}. Almost all natural \ce{WS2} belongs to P63/mmc space group (2H-\ce{WS2}), where $a$ is 3.153 \AA, $c$ is 12.323 \AA (PDF 04-003-4478). Two other crystal structures were found in man-made \ce{WS2}, i.e. 1T and 3R, where 3R can be prepared by bromine chemical vapor transport (CVT) method \cite{Schutte1987} and 1T was generally found in alkaline intercalated \ce{WS2}.\cite{Yang1996a, Enyashin2011}


\section{Crystallography and thermodynamics}

The refined crystallographic data, including bond length, angle is listed in Fig.~\ref{app:bond}. 

\begin{figure}[htb]
\centering
\includegraphics[width=0.9\textwidth]{tms_crys}
\caption{Adopted from Ref.\cite{Schutte1987}}
\label{app:bond}
\end{figure}

The melt point of \ce{WS2} is 1523 K (decompose). A phase diagram between W and S is shown in Fig.~\ref{app:pd}. 

\begin{figure}[htb]
\centering
\includegraphics[width=0.6\textwidth]{ws_phase}
\caption{Adopted from Ref.\cite{Tenne1995,Tenne1998}}
\label{app:pd}
\end{figure}

\section{Band structures}

A theoretical calculation is shown in Fig.~\ref{app:band} 
\begin{figure}[htb]
\centering
\includegraphics[width=0.7\textwidth]{ws2_bandcal}
\caption{Adopted from Ref.\cite{Kuc2011}}
\label{app:band}
\end{figure}



\section{Dielectric function}


$\epsilon(h\nu)$ is shown in Fig.~\ref{app:nk}. 
\begin{figure}[htb]
\centering
\includegraphics[width=0.5\textwidth]{ws2_nk}
\caption{Adopted from Ref.\cite{Hughes1976}}
\label{app:nk}
\end{figure}

The $n$ and $\kappa$ can be retrieved using other software, such as Tracer 2.0.\footnote{\url{https://sites.google.com/site/kalypsosimulation/Home/data-analysis-software-1}}




