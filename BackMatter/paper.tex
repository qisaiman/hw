\chapter{paper reading}

\section{WO3}

\ce{WOx} and optical electric field enhancement. The enhancement arise from the structure composed of a conductive layer and an insulating layer that are laminated therein.\footnote{US patent 8601610B2} In \ce{WOx} nanorods, the oxygen deficient planes are conductive, each having atomic thickness and separated by several nm \ce{WO3}. Localized surface plasmons could possibly exist on these conductive planes. Therefore SERS applies and single molecule Raman scattering using a tungsten oxide nanorod has been demonstrated. The \ce{W_n_O{3n-1}} ($n \geq 2$) exhibit $\{ 001 \}$ CS structure. Chemical formulae corresponding to n=2, 3, 4, 5 and 6 are \ce{W2O5=WO_{2.5}}, \ce{W3O8=W_{2.67}}, \ce{W4O_{11}=WO_{2.75}}, \ce{W5O_{14}=WO_{2.8}}, and \ce{W6O_{17}=WO_{2.83}}, which indicates the existence of a oxygen deficient plane at every n row. Actually the value x in \ce{WOx} could almost continuously vary within a range of 2.5 to 3. \ce{W_{18}O_{49}=\ce{WO_{2.72}}} is an exception without $\{ 001 \}$ CS structure. Moreover, the oxygen deficient planes could extend along directions other than $\{ 001 \}$. For instance, the $\{ 102 \}$ CS planes appears in \ce{WOx} where x is within 2.93 to 2.98, and  the $\{ 103 \}$ CS planes for x within 2.87 to 2.93. \citeauthor{Shingaya2013} also synthesized \ce{WS2}-\ce{WO_x} structures and found similar Raman scattering enhancement. The x value is estimated by the Raman spectra peaks.\cite{Shingaya2013}(Data not shown in patent)

The Raman spectra of \ce{WO_x} is rare because of the difficulty of preparing pure suboxides phase and the strong shielding of \ce{WS2}. Yet it does exhibit distinct Raman spectra. \cite{Tenne2005} The 870 line is attributed to \ce{W3O8}.\cite{Hardcastle1995}

unzip nanotube. passivate BN ribbons with O and S; another player terrones psu.

\citeauthor{Huang2006} studied the \ce{W3On} cluster with n from 7 to 10.\cite{Huang2006} It was found \ce{W3O9} clusters possess a HOMO-LUMO gap about 3.4eV. This closeness to bulk value suggests \ce{W3O9} could be viewed as the smallest molecular unit for bulk \ce{WO3}. 

\ce{WO3} indirect gap 2.6eV, direct gap 3.4eV. \cite{Koffyberg1979}

Fingerprint of m-\ce{WO3}, h-\ce{WO3} and \ce{WO3,nH2O} were summarized in ref\cite{Daniel1987}.

 \ce{WO3} on FTO by flame synthesis.\cite{Rao2014} \cite{Xu2006}
 
 Seeded \ce{W_{18}O_{49}} NWs growth on W foil.\cite{Hong2006a} 
 
 \ce{Na2W4O_{13}} growth and optical properties. \cite{Oishi2001} \cite{Itoh2001} 
 
 For photochemical water reduction to occur, the flat-band potential of the semiconductor (for highly doped semiconductors, this equals the bottom of the conductance band) must exceed the proton reduction potential of 0.0 V vs NHE at pH =0. \cite{Osterloh2008} flat-band potentials strongly depend on ion absorption (protonation of surface hydroxyl groups), crystallographic orientation of the exposed surface, surface defects, and surface oxidation processes.

\citeauthor{Salje1984} studied the transport in \ce{WO_{3-x}} ($0\leq x \leq 0.28$).\cite{Salje1984} It was found \ce{WO_{3-x}} show metallic conductivity when $x > 0.1$. 

 \ce{WO3} high temperature phase. \cite{Vogt1999}
 
 tungsten bronzes \cite{Wiseman1976} 
 
 Ge NW growth using Ga as catalyst. \cite{Chandrasekaran2006}
 
 Phase transformation of \ce{Na2MoO4} and \ce{Na2WO4} by Raman scattering. \cite{Lima2011}
 
 \ce{WO2} NWs synthesis and raman \cite{Ma2005}. 
 
 \ce{WO_{3-x}} CS planes and conductivity.\cite{Sahle1983}

\section{MoO3}

\citeauthor{Hardcastle1990} summarized an empirical formula to relate the Raman peaks and \ce{Mo-O} bonding lengths.\cite{Hardcastle1990} This correlation assumes general form as
\begin{equation}\label{eq:mobond}
\nu = A \exp{B\cdot R},
\end{equation}
where $A=32895$ and $B=-2.073$ are fitting parameters, R is bond distance in unit of \AA. Given a stretching frequency, the resolution for calculated bond distance is $\pm0.016$\AA. The counterpart for \ce{W-O} bond \cite{Hardcastle1995} is 
\begin{equation}\label{eq:wobond}
\nu = 25823 \exp{-1.902\cdot R},
\end{equation}
with standard deviation of estimating \ce{W-O} bond distance from Raman stretching wavenumber is $\pm0.034$\AA. Another empirical expression connect the bond valence $s$ and bond distance R: $s(M-O) \approx (R/X)^{-6} $, where X=1.882 when M is Mo, and 1.904 when M is W. The valence sum rule could be then used to check the state of Mo cation. It should be noticed that not all observed Raman lines could be correlated to a \ce{Mo-O} bond distance by extrapolation of Eq.~\ref{eq:mobond}. It is then regarded as a symmetry related vibrational mode, i.e. $820 cm^{-1}$ in \ce{MoO3}. From the correlation of various Mo compounds, a general conclusion is the lower the stretching frequency for the shortest metal-oxygen bond, the more regular is the structure.

The observed raman lines of \ce{Na5W_{14}O_{44}} phase lies at 965, 943, 913, 808, 786, 778, 765, 695 and 107 cm. The calculated \ce{W-O} bond distance using Eq.~\ref{eq:wobond} fit very well with the crystallographic value of \ce{Na5W_{14}O_{44}}. \cite{Triantafyllou1999a} The 107 peak probably arise from \ce{Na-O} bond. 

The molecular unit in crystal exhibits different vibrational frequencies from that in solution or gas phases.

\ce{Na2Mo4O_{13}} phases monoclinic at RT, solid solubility of \ce{Na2MoO4} in solid \ce{MoO3} is high. vapor pressure of \ce{Na2Mo4O_{13}} over \ce{MoO3}.

melting point of \ce{Na2Mo4O_{13}}
Mp: \ce{Na2Mo2O7} 960K

\ce{MoO3} vapor pressure:

The real phase diagram is the one between \ce{Na2Mo4O_{13}} and \ce{MoO3}.

the growth temperature could be much lower than the eutectic point.

KI MP:  681 
NaI MP: 661

NaOH Raman peaks lie at 3633 cm. \cite{walrafen2006} Raman scattering of \ce{Na2SiO3} exhibit major peak at 966 and 589 cm.\cite{Richet1996} 

\section{TMS}


multipeak Lorentzian fitting. 270 to 410 cm

In 2H type TMDC, the $A_{1g}$ mode is sensitive to electrostatic doping, while $E_{2g}^1$ mode is sensitive to strain, in which the FWHM of the peaks are indicator of external force quantity.

Raman technique could provide valuable insight into the structures of CNT, i.e., tube diameter by assigning the RBMs ( radial breathing modes) and G peaks position.\cite{Bonaccorso2013}

\citeauthor{Virsek2007} investigated the Raman scattering ($\lambda=632nm$) of \ce{WS2} NTs.\cite{Virsek2007} The silicon peak at 520 cm is used for calibration. Up-shift of A1g and E2g modes were observed, attributing to the strain in 3R stacking layers.  

\citeauthor{Dobardzic2005} calculated \ce{MoS2} SWNT phonon dispersion. The dependence of wavenumbers and their displacement on chirality and diameter were discussed. The calculation method enable studying lattice dynamics with NT diameter up to 50nm. The chiral vector $(n_1, n_2)$ is defined within the Moly plane. Symmetry assignment is zigzag when $(n,0)$, armchair when $(n,n)$ and chiral when $(n_1, n_2), n_1>n_2$.


\begin{enumerate}
\item WS2 NT phase, 2H or 3R or 1T; 

\end{enumerate}

\section{misc}

SOI:

VSS, growth kinetics, 