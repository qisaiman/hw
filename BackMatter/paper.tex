\chapter{paper reading}

\section{WO3}

\ce{WOx} and optical electric field enhancement. The enhancement arise from the structure composed of a conductive layer and an insulating layer that are laminated therein.\footnote{US patent 8601610B2} In \ce{WOx} nanorods, the oxygen deficient planes are conductive, each having atomic thickness and separated by several nm \ce{WO3}. Localized surface plasmons could possibly exist on these conductive planes. Therefore SERS applies and single molecule Raman scattering using a tungsten oxide nanorod has been demonstrated. The \ce{W_n_O{3n-1}} ($n \geq 2$) exhibit $\{ 001 \}$ CS structure. Chemical formulae corresponding to n=2, 3, 4, 5 and 6 are \ce{W2O5=WO_{2.5}}, \ce{W3O8=W_{2.67}}, \ce{W4O_{11}=WO_{2.75}}, \ce{W5O_{14}=WO_{2.8}}, and \ce{W6O_{17}=WO_{2.83}}, which indicates the existence of a oxygen deficient plane at every n row. Actually the value x in \ce{WOx} could almost continuously vary within a range of 2.5 to 3. \ce{W_{18}O_{49}=\ce{WO_{2.72}}} is an exception without $\{ 001 \}$ CS structure. Moreover, the oxygen deficient planes could extend along directions other than $\{ 001 \}$. For instance, the $\{ 102 \}$ CS planes appears in \ce{WOx} where x is within 2.93 to 2.98, and  the $\{ 103 \}$ CS planes for x within 2.87 to 2.93. \citeauthor{Shingaya2013} also synthesized \ce{WS2}-\ce{WO_x} structures and found similar Raman scattering enhancement. The x value is estimated by the Raman spectra peaks.\cite{Shingaya2013}(Data not shown in patent)

The Raman spectra of \ce{WO_x} is rare because of the difficulty of preparing pure suboxides phase and the strong shielding of \ce{WS2}. Yet it does exhibit distinct Raman spectra. \cite{Tenne2005} The 870 line is attributed to \ce{W3O8}.\cite{Hardcastle1995}

unzip nanotube. passivate BN ribbons with O and S; another player terrones psu.

\citeauthor{Huang2006} studied the \ce{W3On} cluster with n from 7 to 10.\cite{Huang2006} It was found \ce{W3O9} clusters possess a HOMO-LUMO gap about 3.4eV. This closeness to bulk value suggests \ce{W3O9} could be viewed as the smallest molecular unit for bulk \ce{WO3}.

\ce{WO3} indirect gap 2.6eV, direct gap 3.4eV. \cite{Koffyberg1979}

Fingerprint of m-\ce{WO3}, h-\ce{WO3} and \ce{WO3,nH2O} were summarized in ref\cite{Daniel1987}.

 \ce{WO3} on FTO by flame synthesis.\cite{Rao2014} \cite{Xu2006}

 Seeded \ce{W_{18}O_{49}} NWs growth on W foil.\cite{Hong2006a}

 \ce{Na2W4O_{13}} growth and optical properties. \cite{Oishi2001} \cite{Itoh2001}

 \ce{Na2W4O_{13}} crystal phase \cite{Viswanathan1974}

 For photochemical water reduction to occur, the flat-band potential of the semiconductor (for highly doped semiconductors, this equals the bottom of the conductance band) must exceed the proton reduction potential of 0.0 V vs NHE at pH =0. \cite{Osterloh2008} flat-band potentials strongly depend on ion absorption (protonation of surface hydroxyl groups), crystallographic orientation of the exposed surface, surface defects, and surface oxidation processes.

\citeauthor{Salje1984} studied the transport in \ce{WO_{3-x}} ($0\leq x \leq 0.28$).\cite{Salje1984} It was found \ce{WO_{3-x}} show metallic conductivity when $x > 0.1$. 

 \ce{WO_{3-x}} \cite{Migas2010}

 \ce{WO3} high temperature phase. \cite{Vogt1999}

 tungsten bronzes \cite{Wiseman1976}

 Ge NW growth using Ga as catalyst. \cite{Chandrasekaran2006}

 Phase transformation of \ce{Na2MoO4} and \ce{Na2WO4} by Raman scattering. \cite{Lima2011}

 \ce{WO2} NWs synthesis and raman \cite{Ma2005}.

 \ce{WO_{3-x}} CS planes and conductivity.\cite{Sahle1983}

 \ce{W-O} equilibrium diagram \cite{Wriedt1989}

 \ce{W_{18}O_{49}} electrochromic devices.\cite{Liu2013d} should compare with this one \cite{Wang2008}

  nucleation catalysis \cite{Turnbull1952}

  \ce{WO3} NWs aggregates. \cite{Kozan2008a}

  electrochromic films. \cite{Yoshimura1985}

  optical properties of \ce{WO3} gaps\cite{Saygin-Hinczewski2008}

  \ce{WO3} atomic layer by exfoliation and annealing \ce{WO3.H2O}. \cite{Kalantar-zadeh2010a}

  sodium tungstates raman \cite{Redkin2010}

  charge density wave in K-doped \ce{WO3} \cite{Raj2008}

  \ce{W_{18}O_{49}} Raman \cite{Guo2012} \cite{Guo2011}

  ECD \cite{Jiao2012} recent review \cite{Mortimer2011}

  \ce{WnO_{3n-1}} NPs. \cite{Frey2001}

  \ce{WO3} growth hydrothermal.\cite{Moshofsky2012}

  \ce{W_{18}O_{49}} on tungsten foil by thermal growth\cite{VanHieu2012}

  \ce{WO_{3-x}} Raman peak at 778. \cite{Deb2007}

  Cathodoluminescence \cite{Parish2007}

  optical characterization of WOx film.\cite{Valyukh2010a}

  E-beam penetration \cite{Kanaya2002}

  PEC, photoelectrode, WO3 and Si tandem structures.\cite{Coridan2013}

  WO3 photoactivity MB. \cite{Watcharenwong2008}

  optics in electron microscopy. \cite{GarciadeAbajo2010a}

  A low recombination rate is preferred for high photocatalytic efficiency. The simultaneous migration of electrons and holes.
\textbf{\ce{WS2}-\ce{WO3}}: 1 kW light source(Hg, or Xe lamp), photon flux, phenol (\ce{C6H5OH}, 94.1g/mol, MP 40C)concentration is 20 mg/L, hydroxyl group. The quantitative analysis of phenol was performed via a standard colorimetric method.\footnote{\url{http://omlc.ogi.edu/spectra/PhotochemCAD/html/072.html}}
\citeauthor{DiPaola1999} prepared \ce{WS2}-\ce{WO3} mixture in two methods, sulfurization of \ce{WO3} and oxidation of \ce{WS2},with the latter are more active.
\citeauthor{DiPaola1999} also concluded that the actual efficiency of mixed \ce{WS2}-\ce{WO3} catalysts depends on the ratio of each composition present of the surface of the particles, and the maximum of photoactivity is obtained with 40-50\% surface molar ratio of \ce{WS2}.

ref 25, 28 and 41.

\textbf{\ce{MoO3}}:

\citeauthor{Sreedhara2013} studied the kinetics of photodegradation of methylene blue\footnote{\ce{C16H18N3SCl},319.8 g/mol, MP: 100C accompanied with decomposition \url{http://en.wikipedia.org/wiki/Methylene_blue}} dye by few layer \ce{MoO3}.
For the photodegradation method, it was stated that `` the samples were collected after the photoreaction had been centrifuged for 5 min to remove the photocatalyst before UV-Vis measurement.''


\section{MoO3}

\citeauthor{Hardcastle1990} summarized an empirical formula to relate the Raman peaks and \ce{Mo-O} bonding lengths.\cite{Hardcastle1990} This correlation assumes general form as
\begin{equation}\label{eq:mobond}
\nu = A \exp{B\cdot R},
\end{equation}
where $A=32895$ and $B=-2.073$ are fitting parameters, R is bond distance in unit of \AA. Given a stretching frequency, the resolution for calculated bond distance is $\pm0.016$\AA. The counterpart for \ce{W-O} bond \cite{Hardcastle1995} is
\begin{equation}\label{eq:wobond}
\nu = 25823 \exp{-1.902\cdot R},
\end{equation}
with standard deviation of estimating \ce{W-O} bond distance from Raman stretching wavenumber is $\pm0.034$\AA. Another empirical expression connect the bond valence $s$ and bond distance R: $s(M-O) \approx (R/X)^{-6} $, where X=1.882 when M is Mo, and 1.904 when M is W. The valence sum rule could be then used to check the state of Mo cation. It should be noticed that not all observed Raman lines could be correlated to a \ce{Mo-O} bond distance by extrapolation of Eq.~\ref{eq:mobond}. It is then regarded as a symmetry related vibrational mode, i.e. $820 cm^{-1}$ in \ce{MoO3}. From the correlation of various Mo compounds, a general conclusion is the lower the stretching frequency for the shortest metal-oxygen bond, the more regular is the structure.

The observed raman lines of \ce{Na5W_{14}O_{44}} phase lies at 965, 943, 913, 808, 786, 778, 765, 695 and 107 cm. The calculated \ce{W-O} bond distance using Eq.~\ref{eq:wobond} fit very well with the crystallographic value of \ce{Na5W_{14}O_{44}}. \cite{Triantafyllou1999a} The 107 peak probably arise from \ce{Na-O} bond.

The molecular unit in crystal exhibits different vibrational frequencies from that in solution or gas phases.

\ce{Na2Mo4O_{13}} phases monoclinic at RT, solid solubility of \ce{Na2MoO4} in solid \ce{MoO3} is high. vapor pressure of \ce{Na2Mo4O_{13}} over \ce{MoO3}.

melting point of \ce{Na2Mo4O_{13}}
Mp: \ce{Na2Mo2O7} 960K

\ce{MoO3} vapor pressure:

The real phase diagram is the one between \ce{Na2Mo4O_{13}} and \ce{MoO3}.

the growth temperature could be much lower than the eutectic point.

KI MP:  681
NaI MP: 661

NaOH Raman peaks lie at 3633 cm. \cite{walrafen2006} Raman scattering of \ce{Na2SiO3} exhibit major peak at 966 and 589 cm.\cite{Richet1996}

\section{TMS}


multipeak Lorentzian fitting. 270 to 410 cm

In 2H type TMDC, the $A_{1g}$ mode is sensitive to electrostatic doping, while $E_{2g}^1$ mode is sensitive to strain, in which the FWHM of the peaks are indicator of external force quantity.

Raman technique could provide valuable insight into the structures of CNT, i.e., tube diameter by assigning the RBMs (radial breathing modes) and G peaks position.\cite{Bonaccorso2013}

\begin{enumerate}
\item WS2 NT phase, 2H or 3R or 1T;

\end{enumerate}


(Ramasubramaniam, Naveh, \& Towe, 2011) investigate the band gap tuning in bilayer TMD materials by applying external E field. Similar research has been done for graphene and bilayer boron nitride. Semiconductor-metal transition was suggested for \ce{MoS2} and WS2, with difference on the CBM and VBM evolution. In \ce{MoS2}, the valence-band-splitting cause the A and B excitons in optical absorption measurement. Calculation shows that CB and VB are translated toward the Fermi level with increasing E field.  The external field localized charge along C axis, but delocalized that within the plane normal towards C, thereby driving the semi-metal transition. It was mentioned that this transition is not anticipated in monolayer \ce{MoS2}. It was emphasized that precise band gaps might be different from the author’s results, yet the gap-tuning should be universal.

(Shi, Pan, Zhang, \& Yakobson, 2013) studied the strained monolayer \ce{MoS2} and WS2. The results show that exciton binding energy is insensitive to the strain, while optical band gap becomes smaller as strain increases. Monolayer WS2 PL maximum located at about 1.95eV. Calculation shows the electron effective mass of WS2 is the smallest, rendering higher mobility in device.

(Kośmider \& Fernández-Rossier, 2013) studied the heterojunction between two monolayers of \ce{MoS2} and WS2. Top of VB in W layer and bottom of CB in Mo layer, forming type II structure. bilayer gap 1.2 eV.


Band structure  of \ce{MoS2} in bulk form was calculated by (Mattheiss, 1973). There is some controversy about the exact value of bandgap. The calculation result is 1.2eV ( indirect gap).\cite{Mattheiss1973}

Alkali metal intercalated \ce{WS2} film was prepared.(Homyonfer et al., 1997) Stage 6 superlattice formation was suggested according to X-ray diffraction, and photoresponse spectra and electron tunneling measurement were done.

Sulfurizaiton of W film, (Jäger-Waldau, Lux-Steiner, Jäger-Waldau, & Bucher, 1993)

(Splendiani et al., 2010) reported the PL in monolayer \ce{MoS2}.  Calculation indicated the indirect gap become larger when thinning, while the previous direct one almost stays as the same, the value is about 1.85eV (direct gap).\cite{Splendiani2010}

(Zhou et al., 2010) heterojunction is employed to transferred photo-generated carriers. schottky barrier conduction band electron trapping and consequent longer electron-hole pair lifetimes. Numerous studies have suggested that fine particles of transition metals or their oxides, when dispersed on the surface of a photocatalyst matrix, can act as electron traps on n-type semiconductors.

thermal decomposition of (NH4)2MoO2S2 and intermediate product MoOS2 was studied. application: hyfrodesulfurization in refinery (Weber et al. 1996) \cite{Weber1996}

\cee{MoCl5 + 1/4S8 + 5/2H2 \rightarrow MoS2 + 5HCl} (Stoffels et al. 1999)


(M Remskar et al. 1999) WS2 nanotube from WO3-x whisker. heating 840 degree under flow of H2/N2/H2S ,

A detailed study by (Rothschild, Sloan, and Tenne 2000) tungsten filament oxidation by water. then WS2 nanotube from WO3-x whisker. heating 840 degree under flow of H2/N2/H2S ,

IF MoS2 synthesis by MoO3 nanobelt and S.(X. L. Li and Li 2003)

Inorganic core –shell nanotube, WS2@MoS2 core-shell NT.(Kreizman et al. 2010)

vdW Epitaxy of MoS2 on graphene. (Y. Shi et al. 2012)

WS2 monolayer and intense photoluminescent behavior. Sulfurization of WO3 film( 5-20 angstrom). (Gutiérrez et al. 2012)

MoO3 on sapphire reduction in H2 at 500 and sulfurizatin at 1000 degree. (Lin et al. 2012)

quantitative Raman of MoS2 on insulating subs. intensity difference between supported and suspended was highlighted, detailed model in support info.(S.-L. Li et al. 2012)

MoS2 on SiO2/Si sub pretreated with OTAS, PTCDA solution. (Y.-H. Lee et al. 2012)

WO3-x (1nm) on SiO2/Si sulfurization at 750-950 degree, (Elías et al. 2013)\cite{Elias2013}

source MoS2 powder, Ar On sapphire, SiO2/Si, QWP circularly polarized  light (S. Wu et al. 2013)

(Wang et al. 2013)
MoS2 CVD (tsinghua univ) layer by layer sulfurization of MoO2
MoO3 powder 25mg and MnO2 powder 25 mg, extra sulfur
Ar purging  20mins, heating at 650 degree to produce MoO2 flakes
Further reaction between MoO2 and S was performed at 850-950 degree under Ar flow.

MoS2 on SiO2, sapphire, and graphite by MoCl5 and S. (Yu et al. 2013)
growth time: 10min at 850 degree, P: 2 Torr.


review of inorganic 2D materials, (Chhowalla et al., 2013)

(Ramasubramaniam, 2012) decrease in dielectric screening and thereby enhanced excitonic effect.
DFT is not good at describing photoemission, GW approximation overcome this deficiency but still not enough for photoabsorption process in which ehps are created. BSE equation is used to compensate this discrepancy,
WX2 exhibits larger spin-orbit splitting as compared to MX2 family.

reaction mechanism of \ce{MoO3} to \ce{Mo2S}.\cite{Weber1996}


\citeauthor{Ling2014} studied the role of seeding promoters in CVD growth of FL \ce{MoS2}.\cite{Ling2014} PTAS treated substrates provided nucleation site and thus enable uniform deposition of \ce{MS2}.  This enhancement perhaps arise from the \ce{K+} ions.

CNT chirality by TEM \cite{Zhang1993} TEM chirality of \ce{MoS2} NTs

arise as a result of, dispersal of Na by electron probe.

%%%%%%%%%%%%%%%%%%%%%%

\section{misc}

SOI:

VSS, growth kinetics, 