\chapter{CVD gas dynamics}

Reynolds number is define for laminar flow as
\[
Re = D \cdot \frac{ V_c \rho}{\eta} = D \cdot \frac{ V_c }{\nu}.
\]

D is inner diameter of pipe, $V_c$ is velocity, $\rho$ is mass density and $\eta$ is dynamic viscosity $\nu$ is and kinematic viscosity. 

Reynolds number is the square of the ratio of the system size to the diffusion length for momentum: Re is the Peclet number for momentum. We can now see that small Re corresponds to diffusion lengths comparable to system size: just as before, gradients (in this case in velocity) are small, and the system behaves as a single unit. When Re is very large (flows with Re \textgreater 10,000 or even \textgreater 1,000,000 are common in the real world, though fortunately not in CVD reactors) diffusion of momentum only proceeds locally within very tiny cells, and huge gradients are possible: i.e. recirculating and turbulent flows. Laminar flow is considered to exist if the Re numbers are significantly below 1000.

chamber volume estimation. We use 30 sccm Ar to fill the sealed chamber. At ambient condition, 1 sccm will fill a one liter vessel from zero pressure to 1 atm in a thousand minutes. And in 87 sec, the pressure increase to 80 Torr. So we estimate the chamber volume to be 410 mL.

kinematic viscosity $\nu$ in unit of $m^2/s$. Dynamic viscosity $\eta = \rho*\nu$ in unit of kg/m/s. The relationship between Pascal second and centipoise is 1 Pas = 1 $Ns/m^2$=1 kg/(ms)=103 cP.

Ideal gas equation
\[
PV = n R T;
\]
where R = 8.3 J/mole K.

\begin{itemize}
\item 1 mole = 22.4 liters at "STP" (0 C, 1 atmosphere)
\item 1 liter = 0.045 moles at STP
\item 1 cm3 = 4.5E-5 moles @ STP
\item 1 cm3 = 6.4E-8 moles @ 1 Torr, 23 C
\item 1 atm = 760 Torr = 101,000 Pa
\item 1 Pa = 7.6 mTorr
\item 1 SLPM = 7.4E-4 moles/second
\item 1 sccm = 7.4E-7 moles/second
\end{itemize}

Maxwellian velocity distribution is given by
\[
\frac{\partial N}{\partial \nu} = N 4\pi \nu^2 (\frac{m}{2\pi k T})^{3/2} exp(-\frac{m\nu^2}{2k T})
\]
with mean velocity $c_m = \sqrt{\frac{8 RT}{\pi M}}$.

pumping ability in unit of $cm^3/s$ is estimated to be.

\begin{table}[htb]
\centering
\caption{CVD parameters}\label{tab:cvd2}
    \begin{tabular}{lccr}
    \toprule
    parameter       & value    & Ar     & \ce{O2}  \\
    \midrule
    Inner Dia       & 25.4mm    &       &      \\
    pressure (mTorr)&           & 100   & 10   \\
    pressure (Pa)   &           & 13.33 & 1.333 \\
    Molar weight (g/mol) &      & 40     & 32  \\
    chamber volume  &           & $4.1e^{-4}m^3$  & 410ml \\
    molar con       & $mol/cm^3$ & $5.35e^{-9}$  & $5.35e^{-10}$ \\
    molecular con   & $molecules/cm^3$  & $3.22e^{15}$ & $3.22e^{14}$ \\
    average velocity (m/s) & 300K & 398   & 445  \\
    average velocity (m/s) & 1200K& 796   & 890  \\
    Flux  &$ molecules/cm^2/s$ & 3.21e19 & 3.58e18 \\
    $\eta$  kg/(ms) & 300K      &  $2.272e^{-5}$  & $2.063e^{-5}$   \\
    Reynolds number &          & 95.2      & 9.4   \\
    MFP  ($\mu m$)   &         &           &     \\
    \bottomrule
    \end{tabular}
\end{table}
The molar concentration at 1200 K will be only 25\% of that at RT. 

Knudsen equation is 
\[
J = ($\alpha$)3.51\times10^{22}\frac{P_{torr}}{\sqrt{MT}},
\]
Condensation rate $J(\text{atoms}/\text{cm}^2 \text{sec})$, M in grams/mol. 
This set the upper bound on the rate of deposition. Usually an accommodation coefficient $\alpha$ is used to account for the adsorption. 

Mean free path of gas in vacuum:
\[
\lambda_p = \frac{5\times10^{-3}}{P} \approx 0.05,
\]
where $\lambda_p$ is in centimeter, and P is pressure in Torr. Gas impingement flux $\Phi$, is a measure of the frequency with which gas molecules impinge on or collide with a surface.
\[
\Phi = 3.513\times10^{22}\frac{P}{\sqrt{M T}},
\]
Gas flow could be divided into three regimes: molecular flow, intermediate flow and viscous flow.
Knudsen number is
\[
K_n = D/\lambda_p \approx 50,
\]
where D is characteristic dimension of the CVD system. 