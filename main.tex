\documentclass[12pt,oneside]{book}
 %
   \usepackage{uncc-thesis} % uncc format specification seems this style should be loaded before other packages

   %-------page layout--------%
% adapted from <http://www.khirevich.com/latex/page_layout/>
%\usepackage[DIV=14,BCOR=2mm,headinclude=true,footinclude=false]{typearea}

%\makeatletter
%\if@twoside % commands below work only for twoside option of \documentclass
%    \newlength{\textblockoffset}
%    \setlength{\textblockoffset}{12mm}
%    \addtolength{\hoffset}{\textblockoffset}
%    \addtolength{\evensidemargin}{-2.0\textblockoffset}
%\fi
%\makeatother

% packages used in uncc-thesis

%\RequirePackage{ifthen}
%\RequirePackage{setspace} % for double spacing
%\RequirePackage{comment}
%\RequirePackage{epsfig}
%\usepackage{sectsty} % for sectional header style. Alternative: titlesec package
%\usepackage{tocloft}
%\usepackage{geometry}

\usepackage{microtype} % better layout
%%-- mathmatical symbols and equations-----
\usepackage{amsmath,amsthm}
\providecommand*{\ud}{\mathrm{d}}
\theoremstyle{definition}
\newtheorem{hypothesis}{Hypothesis}

\usepackage{ifluatex}
% Note: When switching the compiler, delete the .aux file.
\ifluatex

\usepackage{fontspec} % to compile with LuaLatex
\setmainfont{Times New Roman} % to compile with LuaLatex
\usepackage{unicode-math}
\setmathfont{Asana-Math.otf} % latinmodern-math or xits-math
% truetype math font
\usepackage{polyglossia}
\setdefaultlanguage{english}
\usepackage{csquotes}% Recommended


\else

%----inherent of article class-------%
\usepackage[utf8]{inputenc} % set input encoding (not needed with XeLaTeX)
%\DeclareUnicodeCharacter{C290}{\mathbb{R}}
%0090 not found
%\usepackage{pslatex} % not bad
%\usepackage[T1]{fontenc}
\usepackage{charter} % nice font
%\usepackage[expert]{mathdesign}
\usepackage{amssymb}
 %\numberwithin{equation}{section}
 %\numberwithin{figure}{section}
\usepackage{csquotes}% Recommended
\usepackage[english]{babel}% Recommended

\fi


%--- for font ----
% \usepackage[T1]{fontenc}
% \usepackage{textcomp}

%\usepackage{mathptmx} % fine but not truetype
% \usepackage{newtxtext}
 %-- end of font adaption

\usepackage{graphicx} % support the \includegraphics command and options
% check pdftex option
\usepackage{placeins} % for float and \FloatBarrier command

\usepackage{subfig} % inconsistent with subcaption package
\usepackage{float} % for images
 \graphicspath{{./gallery/}} %--added by author

\usepackage[font=small]{caption}
%\usepackage[font=small, labelfont=bf]{caption}
%\usepackage{subcaption} % for multi floats in figure or minipage environments

% It supplies a landscape environment, and anything inside is basically rotated.(http://en.wikibooks.org/wiki/LaTeX/Page_Layout)
%\usepackage{lscape}
\usepackage{pdflscape}
%\usepackage{rotating} % use \begin{sidewaystable}

% Helps format tables using the \toprule, \midrule, and \bottomrule commands (http://en.wikibooks.org/wiki/LaTeX/Tables#Using_booktabs)
\usepackage{booktabs}
%\usepackage{multirow} for multirow in tables
%  Helps format tables (http://en.wikibooks.org/wiki/LaTeX/Tables#Using_array)
%\usepackage{array}

%%-- Additional style---- modified by Tao Sheng 12/20/12
% for chemical formula, subscripts etc
\usepackage[version=3]{mhchem}
\usepackage{siunitx}
\DeclareSIUnit \torr{Torr}


%------- bibliography and citation ------


\usepackage[style=numeric-comp,
		    sorting=none,
            hyperref=true,
            url=false,
            isbn=false,
            doi=false,
            backref=true,
            maxcitenames=2,
            maxbibnames=4,
            block=none,
            backend=biber,
            natbib=true]{biblatex}
%bibencoding=latin1 error.
% safeinputenc : still error 
% bibencoding=ascii, error 
\DeclareSourcemap{
    \maps[datatype=bibtex]{
        \map{
            \step[fieldset=abstract,null]
        }
    }
}
\DefineBibliographyStrings{english}{%
    backrefpage  = {see p.}, % for single page number
    backrefpages = {see pp.} % for multiple page numbers
}
% suppress 'in:'
\renewbibmacro{in:}{%
  \ifentrytype{article}{}{\printtext{\bibstring{in}\intitlepunct}}}
% document preamble
% removes period at the very end of bibliographic record
\renewcommand{\finentrypunct}{}
% removes pagination (p./pp.) before page numbers
\DeclareFieldFormat{pages}{#1}


\providecommand*{\bibpath}{E:/spring2012/Ubuntu/Latex/Mendeley_Bib_lib}
\bibliography{\bibpath/ECD.bib,\bibpath/tungsten_newandgood.bib,\bibpath/ACSnano.bib,%
\bibpath/tungsten_old.bib,\bibpath/Raman.bib,\bibpath/Molybdenum.bib,%
\bibpath/tungsten_cl.bib,\bibpath/optics,\bibpath/CVD.bib,\bibpath/sodium.bib,\bibpath/VLS.bib,\bibpath/thesis.bib,\bibpath/arix.bib}

%-- for works around, packages conflicts----

%-- redefine toc macros ------of babel package
%
\addto\captionsenglish{%
\renewcommand\chaptername{CHAPTER}%
\renewcommand\appendixname{APPENDIX}%
\renewcommand\indexname{INDEX}%
\renewcommand{\contentsname}{TABLE OF CONTENTS}%
\renewcommand{\listfigurename}{LIST OF FIGURES}%
\renewcommand{\listtablename}{LIST OF TABLES}%
}


%-- misc---
\usepackage{lipsum}
\usepackage{latexsym}
 \providecommand*{\thefootnote}{\fnsumbol{footnote}}

\usepackage{xcolor}
\usepackage{listings}
\lstset{
 frame = single,
 language = matlab,
 breaklines = true,
postbreak=\raisebox{0ex}[0ex][0ex]{\ensuremath{\color{red}\hookrightarrow\space}}
}

\usepackage[inline]{enumitem}
\setlist{nolistsep} % or \setlist{noitemsep}
%\setlist[1]{\labelindent=\parindent}
\usepackage{mdframed}

%% -- links ---
\usepackage{hyperref}
\hypersetup{
colorlinks,%
citecolor=black,%
filecolor=black,%
linkcolor=black,%
urlcolor=black
} % make all links black
% to set pdf in a preset view
% pdfstartview:
\hypersetup{pdfstartview={XYZ null null 1.00}}
% http://tex.stackexchange.com/questions/128879/defined-acronymname-is-not-used-in-toc-but-the-standard-acronyms-why
% solve the acronym name in TOC issue
% using addcontentsline{toc}{chapter}{\acronymname} command
% to add List of Abbreviation into TOC, instead of using the toc option provided by glossaries package
% probably due to the renewcommand for changing list of abbre name is not respected
% when the glossaries packge add the index to TOC; 
% it always tries to use acronym, which is probably the default name
% However, this workaround may have pagenumbering issues for long list of abbre
%\usepackage[acronym,nonumberlist]{glossaries} % loaded after hyperref
\usepackage[acronym,toc,nonumberlist,nopostdot,nogroupskip]{glossaries} % loaded after hyperref

%% http://tex.stackexchange.com/questions/30812/two-column-glossary
%% http://tex.stackexchange.com/questions/197222/formatting-for-glossaries

 % additonal packages and tweaking
      \newacronym{cvd}{CVD}{chemical vapor deposition}
   \newacronym{ecd}{ECD}{electrochromic device}
   \newacronym{ec}{EC}{electrochromic}
   \newacronym{nhe}{NHE}{normal hydrogen electrode}
   \newacronym{pec}{PEC}{photoelectrochemical}
   \newacronym{her}{HER}{hydrogen evolution reaction}
   \newacronym{sccm}{sccm}{standard cubic centimeters per minute}
   \newacronym{np}{NP}{nanoparticle}
   \newacronym{nw}{NW}{nanowire}
   \newacronym{fl}{FL}{few-layer}
   \newacronym{cft}{CFT}{crystal-field theory}
   \newacronym{nni}{NNI}{National Nanotechnology Initiative}
   \newacronym{tmo}{TMO}{transition metal oxides}
   \newacronym{tmdc}{TMDC}{transition metal dichalcogenids}
   \newacronym{sem}{SEM}{Scanning Electron Microscopy}
   \newacronym{sead}{SEAD}{Selective-area electron diffraction}
   \newacronym{tem}{TEM}{Transmission Electron Microscopy}
   \newacronym{vs}{VS}{Vapor-Solid}
   \newacronym{pmma}{PMMA}{Poly[methylmethacrylate]}
   \newacronym{vls}{VLS}{Vapor-Liquid-Solid}
   \newacronym{xrd}{XRD}{X-ray Diffraction}
   \newacronym{edx}{EDX}{Energy Dispersive X-ray Spectroscopy}

   \newglossaryentry{electrochromic}{
   name=electrochromic,
   description={a phenomenon that materials be colored with positive voltage and bleached with reverse voltage}}

   \newglossaryentry{nanotechnology}{
   name=nanotechnology,
   description={is science, engineering, and technology conducted at the nanoscale, which is about 1 to 100 nanometers.}}

   \newglossaryentry{ceramics}{
   name=ceramics,
   description={compounds often including oxides, nitrides and  carbides}}



   \makeglossaries

   \begin{document}

%       
  \title{Controlled Growth and Properties Measurement of One-Dimensional Nanostructures: Nanowires and Hybrids}
  \author{Tao Sheng}
  \university{The University of North Carolina at Charlotte}
  \location{Charlotte}
  \thesisyear{\the\year}
  \degree{Doctor of Philosophy}
  \department{Physics and Optical Science}

  \committeemember{Dr. Haitao Zhang} % First sets the \advisor default.
  \committeemember{Dr. Tsinghua Her}
  \committeemember{Dr. Michael Fiddy}
  \committeemember{Dr. Stuart Smith}
  \committeemember{Last  person} % Maximum of 10 can be specified.

%  \advisor{Dr. Haitao Zhang}

\maketitle

\begin{abstract}
Transition metal oxides (TMOs) exhibit rich structures and useful properties and could be used in applications of solar energy harvesting (\emph{e.g.}, photoelectrochemistry, photocatalysis, and photovoltaics) and energy saving (\emph{e.g.}, electrochromism and photochromism). Nano-engineering of these TMOs could produce a variety of nanostructures, modify the electronic and optical properties, and potentially enhance the performances and extend their applications into new regimes. To fully harvest the advantages of these materials, a scalable growth method and a comprehensive understanding towards the growth-structure-property relation must be established. This doctoral study focused on one-dimensional (1D) nanostructures of tungsten trioxide (\ce{WO3}) and molybdenum trioxide (\ce{MoO3}). This dissertation synthesized 1D \ce{WO3} and \ce{MoO3} nanostructures using the chemical vapor deposition (CVD) approach, characterized the morphologies and structures, and measured the optical properties of the as-prepared nanostructures. 

This dissertation consists of three main parts, each centering on \ce{WO3}, \ce{MoO3}, and \ce{WO3}-\ce{WS2}, respectively. Systematical investigation of the \ce{WO3} nanostructure prepared by the CVD method with tungsten powders as precursor led to the following accomplishments:
\begin{enumerate*}[label=\itshape\alph*\upshape)]
\item the discovery of \ce{Na5W14O44} nanowires (NWs);
\item a knowledge of the delicate role of sodium in tungsten source powders; and
\item a seeded growth method to produce high quality \ce{WO3} NWs.
\end{enumerate*} Enlightened by the intricate interactions between sodium content and tungsten oxide, this dissertation extended the vapor-liquid-solid strategy in to \ce{MoO3} 1D nanostructure growth and discovered two morphologies of \ce{MoO3}: long nano-belts and micro-towers. Careful observation in extensive experiments identified the catalytic role of sodium hydroxide during the growth of \ce{MoO3} and readily proved that the catalytic behavior also exists with other alkali metal compounds. The third part of this work moved the investigation center from growth to structure and property characterization. A novel integrated structure-property (transmission electron microscopy-Raman spectroscopy, TEM-Raman) study was performed on \ce{WO3}-\ce{WS2} core-shell NWs, allowing for the observation of high resolution TEM imaging, Raman spectroscopy, and photoluminescence spectroscopy on individual nanowires. The TEM-Raman combined results further illustrated that the profiles of resonant Raman spectra exhibit strong dependence on the wall number of \ce{WS2}. This Raman fingerprints can provide a rapid approach for the optical identification of few-walled \ce{WS2} tubular structures.

In summary, this study explored methods for controlled and scalable growth of 1D \ce{WO3} and \ce{MoO3} nanostructures based on both vapor-solid and vapor-solid-solid approaches, characterized the structure-property relations to rationalize nanostructure synthesis for energy applications, and presented preliminary results for the device fabrications using the \ce{WO3} and \ce{MoO3} nanostructures.
\end{abstract}


\begin{dedication}
This mis my appear in the named sectional unit, otherwise each glossary will
appear in a chapter, if chapters exist, otherwise in a section. Unnumbered
sectional units will be used by default.
\end{dedication}

%\pagebreak{}

\begin{ackn}
This work is supported by National Science Foundation (DMR-1006547) and the start-up fund from the Department of Mechanical Engineering and Engineering Science(MEES) and Charlotte Research Institute at the University of North Carolina at Charlotte (UNC Charlotte). We appreciate the Department of MEES and the Center for Optoelectronics and Optical Communications at UNC Charlotte for the support of multiuser facilities.
\end{ackn}
%\pagebreak{}
%% --- Work around some bugs due to babel package---
% for more, see http://tex.stackexchange.com/questions/82993/how-to-change-the-name-of-document-elements-like-figure-contents-bibliogr?lq=1
\setcounter{tocdepth}{1}
% ----- end of work around--------

\clearpage
\tableofcontents
\clearpage
\listoftables
\clearpage
\listoffigures
\begin{singlespace}
%\thispagestyle{myheadings}
\renewcommand{\glossarypreamble}{\thispagestyle{myheadings}}
\renewcommand{\acronymname}{LIST OF ABBREVIATIONS}
\glossarystyle{long}
\printglossary[type=\acronymtype]
\clearpage
\end{singlespace}
%\listofabbre


\pagebreak{} 

% %\providecommand{\setflag}{\newif \ifwhole \wholefalse}
\setflag
\ifwhole\else

    \documentclass[12pt,letterpaper,oneside]{book}

    %-------page layout--------%
% adapted from <http://www.khirevich.com/latex/page_layout/>
%\usepackage[DIV=14,BCOR=2mm,headinclude=true,footinclude=false]{typearea}

%\makeatletter
%\if@twoside % commands below work only for twoside option of \documentclass
%    \newlength{\textblockoffset}
%    \setlength{\textblockoffset}{12mm}
%    \addtolength{\hoffset}{\textblockoffset}
%    \addtolength{\evensidemargin}{-2.0\textblockoffset}
%\fi
%\makeatother

% packages used in uncc-thesis

%\RequirePackage{ifthen}
%\RequirePackage{setspace} % for double spacing
%\RequirePackage{comment}
%\RequirePackage{epsfig}
%\usepackage{sectsty} % for sectional header style. Alternative: titlesec package
% \usepackage{tocloft}
%\usepackage{geometry}

\usepackage{microtype} % better layout
%----inherent of article class-------%
%\usepackage[utf8]{inputenc} % set input encoding (not needed with XeLaTeX)

%--- for font ----
% \usepackage[T1]{fontenc}
% \usepackage{textcomp}

%\usepackage{mathptmx} % fine but not truetype
% \usepackage{newtxtext}
% \usepackage{pslatex} % not bad

\usepackage{fontspec} % to compile with LuaLatex
\setmainfont{Times New Roman} % to compile with LuaLatex

 %-- end of font adaption

\usepackage{graphicx} % support the \includegraphics command and options
% check pdftex option
\usepackage{placeins} %

\usepackage{subfig}
\usepackage{float} % for images
 \graphicspath{{./gallery/}} %--added by author

\usepackage[font=small, labelfont=bf]{caption}
%\usepackage{subcaption}


% It supplies a landscape environment, and anything inside is basically rotated.(http://en.wikibooks.org/wiki/LaTeX/Page_Layout)
%\usepackage{lscape}
\usepackage{pdflscape}
%\usepackage{rotating} % use \begin{sidewaystable}

% Helps format tables using the \toprule, \midrule, and \bottomrule commands (http://en.wikibooks.org/wiki/LaTeX/Tables#Using_booktabs)
\usepackage{booktabs}
%\usepackage{multirow} for multirow in tables
%  Helps format tables (http://en.wikibooks.org/wiki/LaTeX/Tables#Using_array)
%\usepackage{array}

%%-- Additional style---- modified by Tao Sheng 12/20/12
% for chemical formula, subscripts etc
\usepackage[version=3]{mhchem}
\usepackage{siunitx}
  \DeclareSIUnit \torr{Torr}
%%-- mathmatical symbols and equations-----
\usepackage{amsmath}
\usepackage{amssymb}
 %\numberwithin{equation}{section}
 %\numberwithin{figure}{section}
 \providecommand*{\ud}{\mathrm{d}}

%------- bibliography and citation ------
\usepackage[english]{babel}% Recommended
\usepackage{csquotes}% Recommended
\usepackage[style=numeric-comp,
		    sorting=nty,
            hyperref=true,
            url=false,
            isbn=false,
            backref=true,
            maxcitenames=2,
            maxbibnames=4,
            block=none,
            backend=bibtex,
            natbib=true]{biblatex}
% \usepackage[bibencoding=latin1]{biblatex}

\DefineBibliographyStrings{english}{%
    backrefpage  = {see p.}, % for single page number
    backrefpages = {see pp.} % for multiple page numbers
}
% suppress 'in:'
\renewbibmacro{in:}{%
  \ifentrytype{article}{}{\printtext{\bibstring{in}\intitlepunct}}}
% document preamble
% removes period at the very end of bibliographic record
\renewcommand{\finentrypunct}{}
% removes pagination (p./pp.) before page numbers
\DeclareFieldFormat{pages}{#1}


\providecommand*{\bibpath}{E:/spring2012/Ubuntu/Latex/Mendeley_Bib_lib}
\bibliography{\bibpath/arix.bib,\bibpath/ECD.bib,\bibpath/tungsten_newandgood.bib,\bibpath/ACSnano.bib,%
\bibpath/tungsten_old.bib,\bibpath/Raman.bib,\bibpath/Molybdenum.bib,%
\bibpath/tungsten_cl.bib,\bibpath/optics,\bibpath/CVD.bib,\bibpath/sodium.bib,\bibpath/VLS.bib}

%-- for works around, packages conflicts----

%-- redefine toc macros ------
\addto\captionsenglish{%
\renewcommand\chaptername{CHAPTER}%
\renewcommand\appendixname{APPENDIX}%
\renewcommand\indexname{INDEX}%
\renewcommand{\contentsname}{TABLE OF CONTENTS}%
\renewcommand{\listfigurename}{LIST OF FIGURES}%
\renewcommand{\listtablename}{LIST OF TABLES}%
}

%-- misc---
\usepackage{lipsum}
\usepackage{latexsym}
 \providecommand*{\thefootnote}{\fnsumbol{footnote}}

\usepackage{xcolor}
\usepackage{listings}
\lstset{
 frame = single,
 language = matlab,
 breaklines = true,
postbreak=\raisebox{0ex}[0ex][0ex]{\ensuremath{\color{red}\hookrightarrow\space}}
}

\usepackage{enumitem}
\setlist{nolistsep}

\setcounter{secnumdepth}{3} % show numbering of subsubsection

%% -- links ---
\usepackage{hyperref}
\hypersetup{
colorlinks,%
citecolor=black,%
filecolor=black,%
linkcolor=black,%
urlcolor=black
} % make all links black

\usepackage[acronym,toc,nonumberlist]{glossaries} % loaded after hyperref


    %\input{tweak.tex}
    %\input{commando.tex}
    %\input{font}

    \begin{document}

\fi  %  comment out when assembling

\chapter{introduction: nanomaterials for energy applications}

\section{Background and Motivation}

TMO as electrochromic device and TMDC as newly 2D semiconductor, and some VLS.

Materials that human can make define the age they live in. From Stone Age to Bronze Age and Iron Age, people evolve as mastering more and more sophisticated techniques of manipulating metals, such as alloying and annealing. Obtaining extreme high purity of silicon brings us into Information Age. Future is difficult to predict. But nanotechnology is one direction that we can not ignore. According to \gls{nni}, \gls{nanotechnology}. This definition alludes that dimension comes before compositions. It is often related to the quantum confinement or surface area in nanomaterials, which we will later revisit with specific scenario.

There are three states of matter under usual conditions: solid,liquid and gas. Solids materials could be further categorized into five groups: metals, ceramics, polymers, semiconductors, and composites.\cite{William2009} This classification is based on both composition and mechanical, electrical, and thermal properties as well as the associated functionality(i.e., \gls{ceramics} are typically hard yet brittle, insulating to electricity and resistant to heating).

\section{Dissertation Outline}

The materials studied in this work/dissertation are tungsten oxides (\ce{WO3}), molybdenum oxides (\ce{MoO3}),and their chalcogenide counterparts (\ce{WS2} and \ce{MoS2}). Both tungsten (W) and molybdenum (Mo) belong to Group VIB transition metal, with outer shell electrons configuration as $4d^55s^1$ and $5d^46s^2$ respectively. Therefore we refer their oxides and chalcogenides as \gls{tmo} and \gls{tmdc}.\footnote{Obviously transition metals include many other elements, all of which have partially filled $d$-electron shell. But here we use TM to denote W and Mo exclusively.} We have synthesized \gls{tmo} and \gls{tmdc} at nanoscale, measured their crystalline structures and optical properties and demonstrated some devices assembled using as-synthesized nanomaterials. We aim to illustrate that by nanoengineering these \gls{tmo} and \gls{tmdc}, enhanced performances over their bulk states could be expected and new properties will arise. In the remaining sections of this chapter, we will discuss some general perspectives of nanomaterials, the growth apparatus and characterization methods that apply to all experiments done in this work. Then chapter 2 will focus on growth of \ce{WO3} and its derivative. We employed thermal \gls{cvd} to synthesize \ce{WO3} \gls{nw}, and we investigated the role of impurity in tungsten metallic powders, during which we observed a new state of sodium tungsten oxides: \ce{Na5W14O44} nanowires. We also found a method to potentially obtain large yield of \ce{WO3} \gls{nw}. Chapter 3 will concentrate on \ce{MoO3}. We explored two different growth mechanism of \ce{MoO3}:\gls{vs} and \gls{vls}. We discovered that alkaline oxides can be used as catalyst to grow two distinct \ce{MoO3} morphologies: nanobelts and towers. We further demonstrated the application of as-synthesized \ce{MoO3} nanomaterials in electrochromic devices.  In chapter 4 we discuss the growth of \gls{tmdc} and associated heterostructures. We synthesized \ce{WO3}-\ce{WS2} core-shell \gls{nw} and inspected the growth of \gls{fl} \ce{WS2}. Chapter 5 will conclude with an overall summary.



\section{Nanomaterials for Energy Applications}

why nano? surface-to-volume ratio, more surface area for catalytic reaction; surface energy state: tuned by dimension; quantum confinement effects: exciton size vs physical dimension. easy for dopant diffusion, thereby band structure modification; charge-separation and transport mechanism may also differ from bulk.


\section{Crystal Structures and Electronic Properties}

Solid and orderliness.




Two theories arise to describe the outer shell electrons and to correlate the structure and physical properties: \gls{cft} and band theory.\cite{Goodenough1971} \gls{cft} assumes weak interaction between neighboring atoms and localization of electron towards parent atom, whereas band theory assumes that electron is shared equally by all nuclei and therefore a many-electron problems follows. Description of a single electron in periodic potential fail to treat the electron correlations adequately, as the interaction between atoms becomes weaker.For transition metals, $s$ and $p$ electrons are well described by a collective-electron model, while the 4f or 5f electrons are tightly bound to nuclei and screened from the neighboring atoms by 5s, 5p or 6s, 6p core electrons, hence it matches well with a localized-electron model. d electrons show intermediate character.

\section{Growth Apparatus and Characterization Methods}
\subsection{CVD System}

The synthesis was accomplished in a home-made hot-wall CVD system, as visualized in Fig.~\ref{fig:ch1cvd}. The furnace is made by two semi-cylindrical ceramic fiber heaters (WATLOW inc.) with power density from 0.8 to 4.6 \si{W/cm^2}. Quartz tube (Quartz Sci inc.) of 1 inch diameter was primarily used as reaction chamber. A mechanical pump was connected to maintain the low pressure environment inside the chamber. The length of uniform heating zone is 6 inches, with cooling zone extending outward. Carriers gas argon (Ar) and reactant gas oxygen (\ce{O2}) was regulated by two mass flow controllers respectively.

\begin{figure}[htb]
\centering
\subfloat[]{\includegraphics[width=0.8\textwidth]{CVD_d346.jpg}}

\subfloat[]{\includegraphics[width=0.8\textwidth]{CVD_model.jpg}}
\caption[CVD system]{Home-built low-pressure chemical vapor system.}
\label{fig:ch1cvd}
\end{figure}

Silicon and silicon dioxides on Si wafer ($p$-Si(100),Unversity Wafer inc.) were primarily used as receiving substrates, and other substrates (i.e. Mica\footnote{ \ce{K(Al2)(Si3Al)O10(OH)2}}, glass (Fisher Scientific, microscope slide, 12-549),\footnote{Typical composition is 72.6\% \ce{SiO2}, 0.8\% \ce{B2O3}, 1.7\% \ce{Al2O3}, 4.6\% \ce{CaO}, 3.6\% \ce{MgO} and 15.2\% \ce{Na2O}} stainless steel) were occasionally adapted. The preparation procedure was illustrated in Fig. Substrates were first cut into rectangular pieces of certain size and then ultrasonically bathed (Branson 1510R-MTH) with acetone and alcohol for about 15 minutes each followed by blow-drying with nitrogen gas. After solution cleaning, the Si surface is hydrophobic. Depending on the specific experimental requirement, sometimes a hydrophilic surface is desired. We use plasma cleaning (Kurt J Lesker: Plasma-Preen 862) under 2Torr \ce{O2} for 3 minutes to render a hydrophilic Si surface. In addition, substrates can be coated with a thin layer of metal before sent into reaction chamber. This process details will be covered later in the \emph{Characterization} session. The controllable parameters of our CVD system includes central heating temperature, absolute gas flow and relative ratio of (Ar/\ce{O2}), amount of source material and the location of receiving substrates. The operation capability was summarized in Table.~\ref{tab:cvd}.

\begin{table}[htb]
\centering
\caption{CVD parameters map}\label{tab:cvd}
    \begin{tabular}{lcccr}
    \toprule
     &&&\multicolumn{2}{c}{Flow} \\
    \cmidrule(l){4-5}
             & Temperature & Pressure & Ar & \ce{O2}  \\
    \midrule
             & \si{\degreeCelsius} & mTorr & sccm & sccm\\
    \midrule
    Range      & RT-1100    & 10mTorr-1atm & 0 - 100 & 0-30  \\
    Resolution & $\pm1$  & correlated to flow & 1   & 0.1  \\
    \bottomrule
    \end{tabular}
\end{table}

The substrates were mostly positioned in downstream cooling zone since within this region, the vapor undergoes a rapid temperature gradient and precipitation occurs. The source material placed in the central heating zone is oxidized and then evaporated. The growth species, transported by carrier gas, bombard both substrate and chamber wall. Some will be adsorbed by the substrate and become adatoms while some may remain as gas molecules, waiting for another event. On the hot substrate, adatoms diffuse and do not settle down until finding a appropriate location where equilibrium is favorable.

Nucleation is a process of generating a new phase from a metastable old phase, where the Gibbs energy per molecule of the bulk of the emerging new phase is less than that of the old phase.

  General CVD knowledge, substrate preparation, and\cite{MichealK.Zuraw2003}

\subsection{Measurement}
sample/specimen preparation, data processing,

\textbf{Magnetic Sputtering} Sputtering, a process in which atoms are ejected from a solid target material by bombarding it with energetic particles, is a well established PVD process with a high degree of controllability. The high energy and controllable parameters of sputtering can result in the growth of well-structured and crystalline films. Further, sputtering can be easily implemented as a roll-to-roll process for large-scale manufacturing. It is widely utilized for deposition of \ce{WO_x} in industry.


The morphology and composition of the as-synthesized samples were analyzed by scanning electron microscopy (SEM,JEOL JSM-6480) and energy dispersive X-ray spectroscopy (EDS,Oxford Instrument INCA). Crystal structures were characterized using X-ray diffraction (XRD, PANXpert X’pert Pro MRD with Cu $K\alpha$ radiation at $\lambda$=1.5418\AA) and transmission electron microscopy (TEM, JEOL JEM-2100 \ce{LaB6} operated at 200kV). Optical measurements were performed by Micro-Raman spectroscopy (Horiba Scientific, Labram HR800 with 532 nm excitation laser) in a confocal microscope backscattering configuration with spectral resolution about 1 cm$^{-1}$. Optical absorption spectra was recorded using UV-Vis-NIR spectrophotometer(Schimadzu, UV2600Plus) in transmission mode. When necessary, the as-synthesized sample was removed from substrates by light sonication(Branson 1510R-MTH, 70W) in ethanol for 15 seconds. The dispersion was left for 12 hrs to enable the possible sedimentation, after which became transparent under the unaided eye. Then the dispersion was transferred into one 10 mm quartz cuvette (Thorlabs, W005654) for absorption measurement with another paired cuvette containing ethanol only.


%\printbibliography  %  comment out when assembling

%\ifwhole\else
   \end{document}
\fi %  comment out when assembling 
 %
\chapter{experimental tools}

In this study, the nanostructures were grown in a home-built \gls{cvd} system and characterized using both electron microscopy and optical spectroscopy methods, including \gls{sem}, \gls{xrd}, \gls{edx}, \gls{tem}, Raman and UV-Vis spectroscopy. A detailed description is given to the \gls{cvd} system. Other sample and substrate preparation methods are also briefly introduced. For mature processes and techniques, the contents are limited to the extent where experienced material researchers could repeat the experiments performed here. Some details are mentioned in particular to perform the nanostrucutres characterization in a proper way. 

\section{Synthesis Equipments and Processes}
\subsection{Home-built Chemical Vapor Deposition System}
The syntheses were accomplished in a home-made hot-wall \gls{cvd} system. The reaction chamber and a schematic drawing of whole system can be visualized in Fig.~\ref{fig:ch2cvd}. The reaction chamber consists of fused silica tube and a thermal furnace. The furnace is made by two semi-cylindrical ceramic fiber heaters (WATLOW, Inc.) with power density from 0.8 to 4.6 \si{W cm^{-2}}. Quartz tube (Quartz Sci, Inc.) of 1 inch diameter is inserted into the furnace serving as the reaction chamber. A mechanical pump is connected to maintain the low pressure environment inside the chamber. The length of heating zone is 6 inch, with 1 inch cooling zone extending outward on each side. Carriers gas argon (Ar) and reactant gas oxygen (\ce{O2}) are regulated by mass flow controllers.

\begin{figure}[htb]
\centering
\subfloat[]{\includegraphics[width=0.7\textwidth]{CVD_d346.jpg}}

\subfloat[]{\includegraphics[width=0.9\textwidth]{CVD_model.jpg}}
\caption[Home-built low-pressure chemical vapor deposition system]{Home-built low-pressure chemical vapor deposition system. (a) Photograph of reaction chamber. (b) Sketch of CVD system, where A1: quartz tube; B1: bubbler; C1-2: gas cylinders; E1: mechanical pump; H1-2: ceramic heater; MFC: flow controller; P1: pressure gauge; V1-4: needle valves; V5-7: butterfly valves.}
\label{fig:ch2cvd}
\end{figure}

The controllable parameters of this \gls{cvd} system include the central heating temperature, the absolute gas flow, amount of source materials, and the location of substrates. The operation capability is summarized in Table~\ref{tab:cvd}.
\begin{table}[htb]
\centering
\caption{Home-built CVD parameters}\label{tab:cvd}
    \begin{tabular}{lcccr}
    \toprule
     &&&\multicolumn{2}{c}{Flow} \\
    \cmidrule(l){4-5}
             & Temperature & Pressure & Ar & \ce{O2}  \\
    \midrule
             & \si{\degreeCelsius} & mTorr & sccm & sccm\\
    \midrule
    Range      & RT - 1100    & 10 mTorr -1 atm & 0 - 100 & 0 - 30  \\
    Resolution & $\pm1$  & correlated to flow & 1   & 0.1  \\
    \bottomrule
    \end{tabular}
\end{table}
The heating temperature $T$ profile measured at different settings in ambient environment is shown in Fig.~\ref{fig:ch2temp}. There exists a uniform heating zone in the center area, spanning over about 2 inch; then $T$ descends gradually within the 6 inch heating zone and decreases rapidly at the rest part. In this study, a zero position referred as 0 inch is defined at the upstream edge of reaction chamber covered by the furnace.
\begin{figure}[htb]
\centering
\includegraphics[width=0.7\textwidth]{temp_profile.pdf}
\caption[Temperature profile of home-built CVD in ambient environment]{Temperature profile of home-built CVD in ambient environment.}
\label{fig:ch2temp}
\end{figure}
In typical growth, the substrates are positioned just outside the downstream heating zone, since within this region the vapor of growth species undergoes a rapid temperature gradient, and precipitation occurs. After the chamber is heated to target temperatures, the source material (tungsten and molybdenum powders) placed in the furnace center is oxidized and evaporated. The growth species, transported by carrier gas, bombard both substrate and chamber wall. Some will be adsorbed by the substrate and become adatoms, while some may remain as gas molecules, waiting for another event. On the hot substrate, adatoms diffuse and do not settle down until finding an appropriate location where equilibrium is favored. Detailed growth conditions will be introduced in the corresponding chapters. 
\subsection{Substrates Treatments}\label{ch2sub}

Silicon (Si) and 1 $\mu$m silicon dioxides on Si wafer (\ce{SiO2}-Si) (University Wafer Inc.) were primarily used, and other substrates were also employed, i.e. Mica\footnote{\ce{K(Al2)(Si3Al)O10(OH)2}}, glass (Fisher Scientific, microscope slide, 12-549),\footnote{Typical composition is 72.6\% \ce{SiO2}, 0.8\% \ce{B2O3}, 1.7\% \ce{Al2O3}, 4.6\% \ce{CaO}, 3.6\% \ce{MgO} and 15.2\% \ce{Na2O}} and stainless steel. Si and \ce{SiO2}-Si substrates were first cut into rectangular pieces of certain size and then ultrasonically cleaned (Branson 1510R-MTH) with acetone and alcohol for about 15 min, each followed by blow-drying with nitrogen gas. After solution cleaning, the substrate surface is usually hydrophobic. Depending on the specific experimental requirement, a hydrophilic surface may be required. A plasma cleaner (Kurt J Lesker, Plasma-Preen 862) was used to render a hydrophilic Si surface, where reaction among $e$, \ce{O+}, \ce{O2+}, \ce{O^-},\ce{O2} occurs, and \ce{OH}-terminated surface followed.\cite{Habib2010} The treatment proceeds at 2 Torr \ce{O2} for 3 min. Glass substrate cleaning procedure was the same as that of Si. Mica substrates were cleaved right before the growth without extra cleaning steps. In addition, substrates can be coated with a thin layer of metal before growth. This process will be covered in Sec.~\ref{sec:mag}.

\subsection{Magnetron Sputtering}\label{sec:mag}
Sputtering, a process in which atoms are ejected from a solid target material by bombarding it with energetic particles, is a well established \gls{pvd} process with a high degree of controllability. The high energy and controllable parameters of sputtering can result in the growth of well-structured and crystalline films. Further, sputtering can be easily implemented as a roll-to-roll process for large-scale manufacturing. It is widely utilized for deposition of \ce{WO_x} in industry.

In this work, magnetron sputtering (Denton Vacuum Desk IV) was used to coat thin layer of metals onto cleaned substrates. As shown in Fig.~\ref{fig:ch2magsp}, the metal target is the cathode, and the specimen is anode. A dc voltage is applied to ionizes the argon gas, and a permanent magnet behind the target focuses the gas plasma onto the target and also deflect electrons from the sample. These ions are then accelerated to bombard the target foils, knocking out active atoms which subsequently deposit onto the substrates. 
\begin{figure}[htb]
\centering
\includegraphics[width=0.6\textwidth]{sputter.jpg}
\caption[Schematic drawing of magnetron sputtering system]{Schematic drawing of magnetron sputtering system. Reproduced from Ref.\cite{Goldstein2007}}
\label{fig:ch2magsp}
\end{figure}
\ce{Ar+} ions are accelerated to target foils. The pressure of \ce{Ar+} has two functions: to sputter off the target ions and to influence the \gls{mfp}. These collisions between target ions and \ce{Ar+} lead to almost all arrival angles, thus uniform coverage. The sputtering yield is number of target ions released per ions hitting on target. If \ce{Ar+} ion energy is less than 100 eV, the yield is zero; if larger than 10 keV, implantation of \ce{Ar+} into target foils occurs. Usually, yield between 1 and 2 is desired. The \gls{mfp} in sputtering is on order of 3 cm (assuming pressure of 30 mTorr). With source-to-substrate distance as 5 cm, average number of collisions is about 5/3$\sim$2. 

A common purpose of sputter coating is to facilitate SEM observation. Pure gold target can be used for low-magnification studies; a platinum target will deliver satisfactory result for medium-to-high resolution studies. For high-magnification studies, a target of gold-palladium alloy is a good choice because the epitaxial growth is limited.\cite{Goldstein2007} 

\section{Characterization Tools}
\subsection{Scanning Electron Microscopy (SEM)}\label{sec:sem}

The SEM instrument used in this study is JEOL JSM-6480. The \gls{ebeam} source is a tungsten filament heated at about 2800 K.\cite{jeolsem} The acceleration voltage can be up to 30 kV. After acceleration, electron beam strikes the sample mounted on specimen stage producing rich signals, as illustrated in Fig.~\ref{fig:ch2sem}. 
\begin{figure}[htb]
\centering
\includegraphics[width=0.5\textwidth]{sem_sch.png}
\caption[Schematic drawing of electron-matter interactions in SEM]{Schematic drawing of electron-matter interactions in SEM.\cite{Good2000} Some processes are not shown, such as Auger emission.}
\label{fig:ch2sem}
\end{figure}
Depending on the energy and scattering angle, the out-coming electrons from specimen can be grouped into \gls{bse} and \gls{se}, respectively. \gls{se} is generated from the emission of valence band electrons. And due to the limited energy ($<50$ eV), the emission from deep region is mostly absorbed again; only those electrons that are at the vicinity of specimen surface can escape. Hence, the \gls{se} image is topographical sensitive. The emission efficiency is strongly affected by the geometry (tilting angle of specimen surface) and voltage potential (charging effect), so does the contrast in \gls{se} image. On the other hand, backscattered electron could have large energy close to that of incident \gls{ebeam}. The scattering cross section strongly depends on atomic number Z, thus allowing for the compositional analysis. The X-ray emission stems from inner shell transition of the constitutive atoms, and cathodoluminescence arises from band gap transition. 

In this study, secondary electron images were mainly used to reveal the morphology of the samples. Unless otherwise noticed, the SEM images were acquired using 10 kV acceleration voltage, 10 mm working distance, and 30 spot size.

Charge effect could arise on the samples grown in this study, leading to anomalous contrast on the images. Several measures could be taken to alleviate this adverse influence, including adjusting the acceleration voltage, tilting the specimen, and coating a thin layer of metal. Sometimes when an area is scanned for a long time at high magnification, the surrounding region could appear brighter after switching to a lower magnification. This is a sample contamination, due to the \gls{ebeam} induced polymerization and deposition of hydrocarbon molecules in the vicinity of that smaller area. 

\subsection{Energy Dispersive X-ray Spectroscopy (EDX)}\label{sec:edx}

Energy dispersive X-ray spectroscopy provides compositional information attaining to the studied sample because of the characteristic X-ray emission from each constitutive element. EDX unit is usually attached to another electron microscopy, SEM or TEM. In this study, the SEM-attached EDX was primarily used. The X-ray detector (INCA 7573-M, 10 \si{mm^2} Si cooled by liquid nitrogen) is equipped with a super thin window enabling light element detection, such as boron. Typical working parameters of EDX detection in this study include 20 kV acceleration voltage, 10 mm working distance, and 60 s analysis time.

To obtain quantitative results on EDX spectra, special attentions should be paid to the following aspects:
\begin{enumerate*}[label=\itshape\alph*\upshape)]
\item elemental distribution in an X-ray generation area is uniform;
\item the specimen surface is flat;
\item the electron beam enters perpendicular to the specimen; and
\item a reference sample consisting of identical elements and known ratio is used. 
\end{enumerate*} Even for qualitative analysis, ambiguity exists when the $K$ lines of lighter elements overlap with $L$ lines of heavier elements; for instance, Na-$K_\alpha$ (1.041 keV) interferes with $L_\alpha$ lines of Zn (1.009 keV) and Cu (0.928 keV). Overall spectra and growth conditions should be weighted against possible wrong assignment. 

\subsection{X-ray Diffraction (XRD)}
Crystal structures of the as-synthesized specimens in this study were characterized using PANXpert Pro MRD with Cu K$\alpha_{avg}$ radiation at $\lambda$=1.5418 \si{\angstrom}. The high energy photon flux ($E = 1239.8/0.15148= 8184.5$ eV) was generated by using accelerated electron beam from tungsten filament to bombard Cu target. The $\theta-2\theta$ configuration was primarily used, as shown in Fig.~\ref{fig:ch2theta}. 
\begin{figure}[htb]
\centering
\includegraphics[width=0.5\textwidth]{xrd_2theta.png}
\caption[Bragg-Brentano geometry used in XRD configuration]{Bragg-Brentano geometry used in XRD configuration.\cite{Good2000}}
\label{fig:ch2theta}
\end{figure}
When striking onto the crystal, these photons are scattered by the lattices. Since the wavelength of X-ray is similar to the lattice spacing, the scattering events interact coherently. Only at specific angles constructive interference occurs, and a diffraction peak registers when the detector is scanned across that angle. The as-synthesized specimens were mounted onto diffractometer in a way that the X-ray will illuminate interested region during the whole scanning. Typical settings used were summarized in Table~\ref{tab:ch2xrd}.
\begin{table}[htb]
\centering
\caption{XRD settings used in this study}\label{tab:ch2xrd}
\begin{tabular}{lp{1.5in}lp{1.5in}}
\toprule
Name & Value & Name & Value  \\
\midrule
Voltage   & 45 kV & Current & 40 mA \\
Divergence slit & 1/32$^\circ$(alignment) 1/2$^\circ$(scanning) & Receiving slit& Parallel collimator \\
Soller slit & 0.04 rad & Collimator & Parallel plate 0.27 rad \\
Scan range & $10 \sim 65 ^\circ$ & Step size & 0.02$^\circ$ \\
\bottomrule
\end{tabular}
\end{table}
The 1D nanostructures in this study usually grew with preferred orientation; therefore, they cannot be treated as powders. The interpretation of XRD patterns could become difficult somehow. Two possible approaches were suggested to bypass this difficulty: one is using grazing angle XRD which could reveal in-plane domain sizes and orientations;\cite{Tersigni2011,Goorsky2002} the other is combining XRD with alternative techniques, such as Raman and TEM. The latter approach was pursued in this work. 

\subsection{Transmission Electron Microscopy (TEM)}

TEM qualifies as a powerful and versatile tool for material characterization. Akin to the electron-materials interaction introduced in Sec.~\ref{sec:sem}, various signals are generated when the thin specimen is radiated by electron beam. Besides the \gls{saed} pattern and \gls{hrtem} imaging, other TEM based techniques include \gls{cbed}, EDX, \gls{eels} and \gls{cl}, etc; therefore, TEM is essentially a signal-generating and detecting tool.\cite{Williams2009} Fundamental physics of electron is needed to obtain a comprehensive understanding of these TEM related techniques, and only a brief summary is given here.

Electron scattering is dictated by Coulomb interaction, and can be characterized by the following four aspects: the cross section (scattering probability), differential cross section (scattering angle), mean free path, and inelastic or elastic scattering. The cross section $\sigma$ in \si{m^2}, which when divided by area of targeting atom, represents a probability that scattering event will occur. Differential cross section for an isolated atom is 
\[
\frac{d\sigma}{d\Omega} = \frac{1}{2\pi \sin\theta} \frac{d\sigma}{d\theta}
\] 
where $2\theta$ is scattering angle in radians, $\Omega$ is solid angle in steradians (sr), and $\theta$ is half of the scattering angle. The total cross section is just $\sigma$ times atom numbers in unit volume. The mean free path is the inverse of $\sigma_{total}$, that is $\lambda_{MFP} = 1/\sigma_{total}$. 

The spatial distribution of scattering is observed as contrast in images, and angular distribution is viewed in form of diffraction patterns. The positions of diffracted E-beams are determined by size and shape of unit cell, and intensities are governed by distribution, number, and types of atoms in the specimen. 

JEOL JEM-2100 TEM with a \ce{LaB6} filament operated at 200 kV was used in this work. A numerical comparison between the scattering events in XRD and TEM is provided to visualize the difference in these two techniques. In both processes, the constructive interference is described by $2d\sin\theta = \lambda$, where $d$ is crystal plane distance, $\lambda$ is incident wavelength (photon or electron), and $\theta$, again, is half of the scattering angle. Table~\ref{tab:ch2tem} demonstrates the origin of difference in these two techniques.
\begin{table}[htb]
\centering
\caption{Scattering angle difference of Si(111) in XRD and TEM}\label{tab:ch2tem}
\begin{tabular}{lccr}
\toprule
Si(111) spacing & Source & Wavelength & scattering angle($2\theta$) \\
\midrule
3.135 \AA & Cu K$\alpha$ & 1.541 \AA & 28.45$^\circ$  \\
3.135 \AA & 200 kV & 0.0251 \si{\angstrom} & 0.45$^\circ$  \\
\bottomrule
\end{tabular}
\end{table}
This small deviation angle determines the way of secondary signal usage in TEM. In analog to the diffraction-limited resolution in conventional lens optics, the ultimate resolved point distance in TEM should be on the order of electron wavelength used, which is 0.025 \si{\angstrom} at 200 kV. However, it is generally more difficult to manipulate electron with magnetic field than do photon with dielectric lens. The current resolving power in TEM is on the order of 1.0 \si{\angstrom} without correction of the spherical aberration. The nominal resolution of JEM-2100 is 2.3 \si{\angstrom}. On the other hand, it is much easier to fine tune the power (focal length) of lens in TEM than in conventional optics. This capacity enables other observation approaches that would be rather difficult, if not impossible, to be realized in lens optics, such as \gls{saed} pattern. 

In this study, a JEOL double tilt holder was used to rotate the crystal orientation. TEM images and \gls{saed} patterns were acquired using a \gls{ccd} (Gatan, Inc.). EDX spectroscopy was captured by similar Oxford Instrument INCA attachment introduced in Sec.~\ref{sec:edx}. To make the best out of TEM analysis, one also needs to be aware of the limitations of TEM. Two aspects were pointed out here, namely the small sampling volume and the image interpreting. The former one is inherent and can only be partially overcome by combining TEM with other ensemble characterization tools. As to the latter one, it should be noted that the specimen is usually in focus from top to bottom surface; therefore the images, \glspl{dp} and spectra are all averaged through the thickness of specimen. In other words, single TEM image has no depth sensitivity.\cite{Williams2009}  

\section{Optical Measurements}
\subsection{Raman Spectroscopy}

Raman spectroscopy, a common vibrational spectroscopy, is based on the inelastic scattering of a monochromatic excitation source with photon energy variation range of 100 to 4000 \si{cm^{-1}}. This energy difference between incident photon and inelastically scattered photon is closely related to a series of vibrational modes of the sample under observation. Being nondestructive and requiring minimal preparation, Raman spectroscopy is an excellent tools to assess lattice dynamics and to fingerprint species.\cite{McCreery2000} For instances, characteristic Raman shift can be used to determine material compositions; changes in Raman peak (\gls{fwhm}, frequency shift) are often related to strain; polarization Raman spectra can be used to derive crystal orientation and symmetry. 

Some common symbols for symmetry representations are as following:
\begin{enumerate}
\item $A$ representation indicates that the functions are symmetric with respect to rotation about the principal axis of rotation.
\item $B$ representations are asymmetric with respect to rotation about the principal axis.
\item $E$ representations are doubly degenerate.
\item $T$ representations are triply degenerate.
\item Subscripts u and g indicate asymmetric (\emph{ungerade}) or symmetric (\emph{gerade}) with respect to a center of inversion.
\end{enumerate}

In this study, Raman measurement was performed using a confocal micro-Raman system (Horiba Scentific, Labram HR800) with excitation wavelengths at 441, 532, and 632 nm, where the corresponding photon energy is 2.81 eV, 2.33 eV, and 1.96 eV, respectively. For nanowire samples, the laser powers were kept between 0.2 and 0.3 mW, and typical acquisition time was 100 s to avoid possible thermal damage. The spectral resolution is about 1 \si{cm^{-1}}, and the depth resolution about 2 $\mu$m.

\subsection{UV-Vis Spectroscopy}
UV-Vis-NIR spectrophotometer is a useful tool to obtain optical properties from various samples. Generally speaking, the UV spectrum ranges from 100 nm to 400 nm; visible spectrum spans from 400 nm to 750 nm; and near-IR ranges from 0.75 to 3 $\mu$m. Absorption from atmospheric \ce{CO2} becomes significant below 200 nm; therefore 100 to 200 nm region is usually not measured unless some vacuum technique is applied.The cost of fabricating a spectrometer covering the whole NIR region can be quite high. The instrument used in this dissertation (Schimadzu, UV2600Plus) can measure from 220 nm to 1350 nm (5.6 eV to 0.92 eV) with an integrating sphere.

The absorption of materials can be described by Lambert-Beer law: $I = I_0 \exp(-A)=I_0\exp(-\alpha(\lambda) x)$, where $\alpha$ is the absorption coefficient in unit of \si{cm^{-1}}, and $x$ is the optical path in cm. When the sample is in liquid form, it is more convenient to use $A =\epsilon_\lambda C x$, where $C$ is the molar concentration (M = \si{\mole \per L}), and $\epsilon_\lambda$ is the extinction coefficient/molar absorptivity (\si{M^{-1} cm^{-1}}). For example, $\epsilon$ for methylene blue (MB) is $10^5$ \si{M^{-1}cm^{-1}} at 660 nm.\cite{Mills1999} So by measuring the maximum absorbance of sample with known composition, its concentration can be estimated accordingly. 

In the transmission mode, the reflection from materials phase boundary has been compensated with the usage of paired container and solvent. The scattering, however, cannot be eliminated this way. Usually scattering is negligible in molecular disperse media, yet should be considered in colloids or solids when the incident wavelength is comparable to the particle dimension. The size of nanostructures studied in this work is estimated to be 200 to 400 nm after sonication treatment; therefore, scattering effect should be included in the absorbance spectrum. 

In fact, there have been ongoing efforts to retrieve particle size profile from the dynamic light scattering, a technique also known as photon correlation spectroscopy. The experimental principle is as following: the sample in liquid solution is illuminated by a laser source and the fluctuations of the scattered light are detected at a known scattering angle $\theta$ by a fast photon detector. The fluctuations arise from the Brownian motion of small particles, equivalent to a random variation of scatter distance. By auto-correlation analysis, the decay rate of scattering intensity correlation can be related to the diffusion coefficient of small particles in the solvent.\cite{Maret1987} This approach, however, will not be pursued in this work. Instead, the scattering effect of the nanostructured samples is partially circumvented by using diffuse reflection technique. 

Diffuse reflectance spectroscopy (DRS) serves as another tool to estimate the sample band gap using \gls{kmt}.\cite{Tandon1970} The original KM theory was proposed in 1931\cite{Kubelka1931} and has been popular in the color-related industry, such as painting, pigments, and paper. This method is useful for the analysis of samples of difficultly soluble substances and for samples that will react with the solvents upon being dissolved. It is also particularly applicable when single crystals of the material could not be obtained. Some key points of \gls{drs} were recapitulated in this thesis.

The sample is under isotropic diffuse illumination. And the upward flux $J$ and downward flux $I$ are characterized by the KM scattering and absorption coefficients denoted as $S$ and $K$, respectively; that is
\begin{align}
\ud I &= - (K + S)I \ud z + SJ\ud z,\\
\ud J &= + (K + S)J \ud z - SI\ud z.
\end{align}
Quantities $S$ and $K$, in unit of percentage of light scattered and absorbed per unit vertical length, have no direct physical meaning. For example, $S$ and $K$ depend on illumination geometry: diffuse or collimated. In the limit of infinite thick sample, the KM equation is given by
\begin{equation}
f(R_\infty) = \frac{(1-R_\infty)^2}{2R_\infty} = K/S,
\end{equation}
where $R_\infty$ is relative reflectance between sample and standard. In case of dilute species, one can approximate the KM function as $f(R) \propto \frac{\epsilon c}{s}$. When the diffuse scattering condition is fulfilled, $K$ can be related to absorption coefficient $\alpha$ by $K = 2\alpha$. If the scattering coefficient $S$ is fixed with respect to $\lambda$, one can then obtain
\begin{equation}
(f(R_\infty) h \nu)^n \propto (hv - E_g),
\end{equation}
where $n$ depends on the nature of electronic transition. The diffuse reflection spectroscopy is more proper to characterize nanomaterials because compared with UV-Vis transmission measurement for sample dispersed in liquid media, it takes the scattering effect into consideration. Favorable results have been obtained on the comparison of band gaps using KM model and other methods.\cite{Tandon1970,Morales2007} 

\begin{figure}[htb]
\centering
\includegraphics[width=0.7\textwidth]{UV-Vis_drs}
\caption[Schematic drawing of UV-Vis and DRS measurement]{Schematic drawing of (a) UV-Vis and (b) DRS measurement setup adopted from UV2600Plus manual.}
\label{fig:ch2uvvis}
\end{figure}

In this thesis, UV-Vis and \gls{drs} were recorded using Schimadzu, UV2600Plus, with schematic measurement setup shown in Fig.~\ref{fig:ch2uvvis}. For UV-Vis measurement in transmission mode, the as-synthesized sample was removed from substrates by light sonication (Branson 1510R-MTH, 70W) in ethanol or DI water for 15 second. The dispersion was left for 12 h to enable the possible sedimentation. Then, the dispersion was transferred into one 10 mm quartz cuvette (Thorlabs, W005654) for absorption measurement with another paired cuvette containing pairing liquid only. To carry out diffuse reflectance measurement, an integrating sphere (ISR2600Plus) was used with an incident angle of 0 degree. The baseline reflectance was first recorded with the standard white plate (barium sulfate powder, \ce{BaSO4}) placed at the exit window on the sample path side, then the target sample was set in for relative reflectance measurement. The \ce{BaSO4} powder should be replaced periodically, otherwise the surface might turn yellow. To replace the \ce{BaSO4} powder, one should supply new powder into the sample holder for several time, each followed by a compact pressing using the glass rod. Chemical wrapping paper can be inserted between the glass rod and \ce{BaSO4} powder to prevent the powder sticking. Nanostructured sample on transparent substrates, i.e. quartz, could be mounted directly with the growth substrate, or ground and spread evenly onto \ce{BaSO4} surface.\cite{isr2011}

  % %\providecommand{\setflag}{\newif \ifwhole \wholefalse}
\setflag
\ifwhole\else

    \documentclass[12pt,letterpaper,oneside]{book}

    %-------page layout--------%
% adapted from <http://www.khirevich.com/latex/page_layout/>
%\usepackage[DIV=14,BCOR=2mm,headinclude=true,footinclude=false]{typearea}

%\makeatletter
%\if@twoside % commands below work only for twoside option of \documentclass
%    \newlength{\textblockoffset}
%    \setlength{\textblockoffset}{12mm}
%    \addtolength{\hoffset}{\textblockoffset}
%    \addtolength{\evensidemargin}{-2.0\textblockoffset}
%\fi
%\makeatother

% packages used in uncc-thesis

%\RequirePackage{ifthen}
%\RequirePackage{setspace} % for double spacing
%\RequirePackage{comment}
%\RequirePackage{epsfig}
%\usepackage{sectsty} % for sectional header style. Alternative: titlesec package
% \usepackage{tocloft}
%\usepackage{geometry}

\usepackage{microtype} % better layout
%----inherent of article class-------%
%\usepackage[utf8]{inputenc} % set input encoding (not needed with XeLaTeX)

%--- for font ----
% \usepackage[T1]{fontenc}
% \usepackage{textcomp}

%\usepackage{mathptmx} % fine but not truetype
% \usepackage{newtxtext}
% \usepackage{pslatex} % not bad

\usepackage{fontspec} % to compile with LuaLatex
\setmainfont{Times New Roman} % to compile with LuaLatex

 %-- end of font adaption

\usepackage{graphicx} % support the \includegraphics command and options
% check pdftex option
\usepackage{placeins} %

\usepackage{subfig}
\usepackage{float} % for images
 \graphicspath{{./gallery/}} %--added by author

\usepackage[font=small, labelfont=bf]{caption}
%\usepackage{subcaption}


% It supplies a landscape environment, and anything inside is basically rotated.(http://en.wikibooks.org/wiki/LaTeX/Page_Layout)
%\usepackage{lscape}
\usepackage{pdflscape}
%\usepackage{rotating} % use \begin{sidewaystable}

% Helps format tables using the \toprule, \midrule, and \bottomrule commands (http://en.wikibooks.org/wiki/LaTeX/Tables#Using_booktabs)
\usepackage{booktabs}
%\usepackage{multirow} for multirow in tables
%  Helps format tables (http://en.wikibooks.org/wiki/LaTeX/Tables#Using_array)
%\usepackage{array}

%%-- Additional style---- modified by Tao Sheng 12/20/12
% for chemical formula, subscripts etc
\usepackage[version=3]{mhchem}
\usepackage{siunitx}
  \DeclareSIUnit \torr{Torr}
%%-- mathmatical symbols and equations-----
\usepackage{amsmath}
\usepackage{amssymb}
 %\numberwithin{equation}{section}
 %\numberwithin{figure}{section}
 \providecommand*{\ud}{\mathrm{d}}

%------- bibliography and citation ------
\usepackage[english]{babel}% Recommended
\usepackage{csquotes}% Recommended
\usepackage[style=numeric-comp,
		    sorting=nty,
            hyperref=true,
            url=false,
            isbn=false,
            backref=true,
            maxcitenames=2,
            maxbibnames=4,
            block=none,
            backend=bibtex,
            natbib=true]{biblatex}
% \usepackage[bibencoding=latin1]{biblatex}

\DefineBibliographyStrings{english}{%
    backrefpage  = {see p.}, % for single page number
    backrefpages = {see pp.} % for multiple page numbers
}
% suppress 'in:'
\renewbibmacro{in:}{%
  \ifentrytype{article}{}{\printtext{\bibstring{in}\intitlepunct}}}
% document preamble
% removes period at the very end of bibliographic record
\renewcommand{\finentrypunct}{}
% removes pagination (p./pp.) before page numbers
\DeclareFieldFormat{pages}{#1}


\providecommand*{\bibpath}{E:/spring2012/Ubuntu/Latex/Mendeley_Bib_lib}
\bibliography{\bibpath/arix.bib,\bibpath/ECD.bib,\bibpath/tungsten_newandgood.bib,\bibpath/ACSnano.bib,%
\bibpath/tungsten_old.bib,\bibpath/Raman.bib,\bibpath/Molybdenum.bib,%
\bibpath/tungsten_cl.bib,\bibpath/optics,\bibpath/CVD.bib,\bibpath/sodium.bib,\bibpath/VLS.bib}

%-- for works around, packages conflicts----

%-- redefine toc macros ------
\addto\captionsenglish{%
\renewcommand\chaptername{CHAPTER}%
\renewcommand\appendixname{APPENDIX}%
\renewcommand\indexname{INDEX}%
\renewcommand{\contentsname}{TABLE OF CONTENTS}%
\renewcommand{\listfigurename}{LIST OF FIGURES}%
\renewcommand{\listtablename}{LIST OF TABLES}%
}

%-- misc---
\usepackage{lipsum}
\usepackage{latexsym}
 \providecommand*{\thefootnote}{\fnsumbol{footnote}}

\usepackage{xcolor}
\usepackage{listings}
\lstset{
 frame = single,
 language = matlab,
 breaklines = true,
postbreak=\raisebox{0ex}[0ex][0ex]{\ensuremath{\color{red}\hookrightarrow\space}}
}

\usepackage{enumitem}
\setlist{nolistsep}

\setcounter{secnumdepth}{3} % show numbering of subsubsection

%% -- links ---
\usepackage{hyperref}
\hypersetup{
colorlinks,%
citecolor=black,%
filecolor=black,%
linkcolor=black,%
urlcolor=black
} % make all links black

\usepackage[acronym,toc,nonumberlist]{glossaries} % loaded after hyperref


    %\input{tweak.tex}
    %\input{commando.tex}
    %\input{font}

    \begin{document}

\fi  %  comment out when assembling

\chapter{IMPURITY EFFECTS ON TUNGSTEN OXIDE NANOWIRES}


\section{Introduction}

Tungsten oxide (\ce{WO_x}) is an important functional materials with distinctive properties and technology applications. Intense research interest was rekindled by the discovery of \gls{ec} effect in \citeyear{Granqvist1993}.\cite{Granqvist1993}  Nano-engineering \ce{WO_x} bring more possibility and flexibility to its already rich characteristics, therefore resulting research efforts spanning multiple fields of scientific community. Besides smart window based on EC effect, \ce{WO_x} is also well investigated for several other significant applications: photoelectrochemical cell for solar energy conversion and storage, photocatalysis for hydrogen evolution reaction, chemical and biological sensing based on gasochromic effect. In addition, a few young fields are not thoroughly explored: field emission, optical storage, thermoelectrcity, ferroelectricity, and superconductivity.

In this chapter, we will first review the literature to date for both \ce{WO_x} and \ce{Na_xWO3}, then divide the summaries into a) technological application, b) crystal structure and electronic properties and c) synthesis approaches. Then we present our growth methods and results. And we conclude with a brief summary.

\subsection{Rich and useful technological appeals}

Electrochromic effect of tungsten oxide means the coloration (deep blue) and bleaching (transparent) states of \ce{WO3} are reversibly switched upon forward and backward voltage, and the coloration or bleached states remain after disconnecting the voltage. On basis of this property, \gls{ecd} of tremendous energy saving potential is conceptualized and some products have already been commercialized (i.e. smart window\footnote{``We did a case study in five cities, and the average savings in commercial buildings are about 25 percent of the heating, ventilation, and air-conditioning energy use annually," says CEO of View, Inc.}).

In the past decades, tons of works have been devoted to understanding the chromogenic phenomena in \ce{WO3}. We highlight several key results as reviewed by ,
\begin{itemize}
    \item coloration and bleaching can also be stimulated by other routes, such as by UV irradiation, thermal treating, heating in vacuum, reducing atmosphere etc
    \item no electrical coloration occurs in vacuum, but other routes still works
    \item coloration is associated with a proportional increase in conductivity
    \item coloration spectrum is essentially similar in all cases except small variation in peak position and FWHM
    \item coloration is structure sensitive and most efficient in amorphous films
\end{itemize}

Sofar there is no unifying model that could reconcile these contradictory experimental observations. Among all the developed models, polaronic absorption is the most widely accepted mechanism for coloration. In polaron model\footnote{When a free electron travels through a polar solid, it creates a local lattice displacement (longitudinal optical phonon clouds) due to the coulombic interaction with neighboring ions. This local distortion and the electron together is equivalent to a new elementary exciton of the crystal, and is named as polaron}, the intercalation of \ce{M^+} (M = H, \ce{Li}, or Na) ions into \ce{WO3} films is accompanied with the formation of small polarons ($r_p$ comparable to unit cell size) and formal reduction of some \ce{W^{6+}} sites to \ce{W^{5+}}, as depicted in Eq.~\ref{eq:ec}. During the intercalation process, the \ce{M^+} ions enter into these vacant sites.\cite{Hepel2008} Coloration occurs when the polaron band .

\begin{align}\label{eq:ec}
x\ce{M+} + x\ce{e-} +  \ce{$\alpha \hyphen$WO$_{3-y}$}= \ce{$\alpha \hyphen$M$_x$WO$_{3-y}$},
\end{align}

And the polaron binding energy ($E_p$) is given by
\begin{align}
E_p = - \frac{e^2}{2r_p} (\epsilon_\infty^{-1} - \epsilon_{st}^{-1}),
\end{align}

where $\epsilon_\infty$ and $\epsilon_{st}$ are optical and static dielectric constants respectively, and polaron radius $r_p$, which specifies how far the lattice distortion extends, is related to polaron density $N_p$ by $r_p = \frac{1}{2}\sqrt[3]{\pi/6N_p}$. However, the polaron model has difficulty in estimating $r_p$. The asymmetric optical absorption spectrum are often characteristic of large polarons, and dielectric constants ($\epsilon_\infty = 6.52,\epsilon_{st} > 50$,\cite{Deb2008}) suggests the formation of bipolaron. Moreover, polaron model does not take oxygen vacancy into account, which plays a vital role in the nonstoichiometric tungsten oxides. For instance, it is observed that \ce{WO_{3-y}} films are metallic and conductive for $y > 0.5$, blue and conductive for $y = 0.3 \sim 0.5$, and transparent and resistive when $y < 0.3$, regardless of the preparation methods.\cite{Chatten2005}

Therefore another model in analogy to the F-color center is proposed. Color center model assumes the presence of oxygen vacancy $V_O^0$ is associated with \ce{W^{4+}} or 2\ce{W^{5+}} states. This defect level is expected to be inside or near the valence band. When one electron is removed from this level, $V_O^0$ is converted to $V_O^+$.The positively charged vacancy exerts coulombic repulsion to the nearest W-ions, which results in a displacement of the neighboring W-ions and an upward shift of the defect level into the bandgap, thereby creating a color center. The optical transition from $V_O^+$ to $V_O^{2+}$ (a state within the conduction band)
contributes to coloration.\cite{Deb2008}

So the polaron and color center models both agree on that \ce{W^{5+}} and its transition is responsible for the coloration, but disagree on how this \ce{W^{5+}} state is created (foreign ion reduction in polaron model and oxygen vacancy in color center model) and the corresponding energy levels. A modified polaron model is proposed to include \ce{W^{4+}} states in host lattice. Coloration mechanism is represented by Eq.~\ref{eq:cl_bl1} and \ref{eq:cl_bl2}, which described the polaron hopping from one site to another.\cite{Chatten2005}

\begin{align}
h\nu +\ce{W^{5+}(A)} +  \ce{W^{6+}(B)} &\rightarrow \ce{W^{5+}(B)} + \ce{W^{6+}(A) + E_{phonon}} \label{eq:cl_bl1}\\
h\nu +\ce{W^{5+}(A)} +  \ce{W^{4+}(B)} &\rightarrow \ce{W^{5+}(B)} + \ce{W^{4+}(A) + E_{phonon}} \label{eq:cl_bl2}
\end{align}

Similar scenarios occurs in the gasochromic effect. Two models, double injection and color center, arise to account for the coloration upon exposure to certain gases. Both consents to Eq.~\ref{eq:cl_bl1}. But there is a disagreement on the final states. Double injection supports the formation of tungsten bronze \ce{H_xWO3} while color center insists on the inward diffusion of oxygen vacancy and outward diffusion of water molecules. Both have been substantiated experimentally. Therefore the exact mechanism in \ce{WO_x} still requires further investigation. The author believes the resolution to a large extent depends on the phase transitions, as discussed in Section.~\ref{sec:wonawo}.

photocatalytic applications in solar hydrogen generation and organic pollutant degradation.

photocatalyst\cite{Macphee2010}, photoelectrochemical energy application \cite{Su2010}

Photocatalytic activity occurs when a semiconductor in aqueous solution is illuminated by photons of energy larger than the band gap, then electron-hole pairs generate free radicals, (i.e. \ce{OH.}) which enable further reactions. For water splitting using solar energy, the band gap should be within $2.0 \sim 3.0 $ eV, and CB edge should be more negative than reduction potential of \ce{H^+/H2}, whereas the VB top should be more positive than the oxidation potential of \ce{H2O/O2)}.\cite{Wang2012} The valence band holes (\ce{h^+}) oxidize water to oxygen and conduction band electrons propel hydrogen generation, as depicted in Eq.~\ref{eq:hervb} and \ref{eq:hercb}.

\begin{align}
4\ce{h^+} +  \ce{H2O} &\rightarrow \ce{O2} + 4\ce{H^+} \label{eq:hervb}\\
4\ce{H^+} +  4e^- &\rightarrow 2\ce{H2} \label{eq:hercb}
\end{align}

As shown in Fig.~\ref{fig:woxnhe}, \ce{WO3} has CB edge positioned slightly more positive than reduction potential of \ce{H^+/H2}(versus \gls{nhe}), and VB edge much more positive than the oxidation potential of \ce{H2O/O2}. So the photo-cleavage of water cannot be accomplished by \ce{WO3} alone. Nevertheless, a tandem cell approach by \ce{WO3} film and dye-sensitized \ce{TiO2} has been demonstrated with an efficiency of 4.5\%.\cite{Michael1999} It is worth noting that green plants also have two photosynthetic systems connected in series, one for oxidation of water into oxygen and the other for fixation of carbon dioxide.

% wo3 vs NHE
\begin{figure}[htb]
\centering
\includegraphics[width=0.7\textwidth]{woxnhe.jpg}
\caption[Bands positions of \ce{WO3} versus NHE]{Bands positions of \ce{WO3} in contact with aqueous electolyte at pH 1. adapted from Ref\cite{Gratzel2001}}
\label{fig:woxnhe}
\end{figure}

Moreover, favorable oxygen evolution of \ce{WO3} brings good performance in degradation of organic compounds\cite{Hepel2001,Luo2001,Watcharenwong2008}. The formation of long-lived holes is recognized as a key requirement.\cite{Pesci2011} Besides, \ce{WO3} is remarkably stable in acid, making it a significant candidate for treating water pollutant caused by organic acids.\cite{Monllor-Satoca2006}



the ubiquity of \ce{WO6} octahedra is essential for not only the optical properties but the ability to insert and extract ions in the EC oxides, due to the tunnels in three dimensions serving as path for transport of small ions.
The intercalation of hydrogen or alkali ions into \ce{WO3} created electron donor level. By absorbing the red part of incident spectrum, electrons at donor level make transition to the conduction band, causing the blue coloration in \ce{H_xWO3}.

\citeauthor{Wang2009a} mentioned that amorphous \ce{WO3} can only be used in lithium-based electrolytes due to its
in-compact structure and high dissolution rate in acidic electrolyte solutions. Electrochromic materials that can endure acidic electrolytes without degradation should be developed. Crystalline \ce{WO3} nanostructures with their much denser structures and small particle sizes are promising to be used as suitable electrochromic material in acidic electrolytes.

Characterization of ECD (work like a thin-film batteries) includes transmission measurement and associated EC calculation, charge-discharge time, current-time curve and the fitting of obtained data.

The coloration efficiency (CE) represents the change in the optical density (OD) per unit charge density ($Q/A$, in units of \si{\cm^2\per\coulomb}) during switching and can be calculated according to the formula:
\begin{equation}
CE = \frac{\Delta~OD}{(Q/A)} [cm^2/C],
\end{equation}
where OD = $log(T_{bleach}/T_{color})$. The EC and optical density depend on the wavelength and are usually higher in the near IR than in the visible region.
Using Ohm's law($U_s = IR = RQ/t_s$) with switch voltage $U_s$, resistance R and surface area A, switching time $t_s$ could be estimated as
\begin{equation}
t_s = \Delta~OD\cdot A \cdot R /(CE\cdot U_s).
\end{equation}

its one dimensional (1D) nanostructure has attained intensive research efforts in recent years due to the potential applications in advanced nano-electric and nano-optoelectronic devices.

\begin{quote}
a viable electrochromic smart window must exhibit a cycling life time \textgreater $10^5$ cycles, corresponding to an operation life at 10 - 20 years.
\end{quote}






\subsection{Crystal structures and properties of tungsten oxides and sodium tungsten oxides}\label{sec:wonawo}

\textbf{\ce{WO3}} Tungsten trioxides crystalize in multiple phases. The basic building block is \ce{WO6} octahedra.\footnote{Tungsten, with its electronic configuration as \ce{(Xe)4f^{14}5d^{4}6s^{2}}, has empty 5d and 6s orbitals in its +6 oxidation state.} \ce{WO3} crystal structure consists of \ce{WO6} octahedra joined at their corners, which may be considered as a perovskite structure of \ce{CaTiO3} with all \ce{Ca^{2+}} sites vacant. A representative lattice structures is illustrated in Fig.~\ref{fig:wo3oct}. These distorted \ce{WO6} octahedra adapts different tilting angles in different phases and edge-sharing octahedra also arises. The temperature-dependent phase transition and corresponding lattice constants in bulk form is summarized in Table.~\ref{tab:wo3phase}.\cite{Zheng2011} And a firm assignation of space group to monoclinic phases are still in debate.\cite{Chatten2005} It is noticed that the lattice parameters obtained via \emph{ab initio} calculation closely match the experimental values.\cite{Migas2010a} The phase transition scenarios in nanostructured \ce{WO3} are supposed to be much more sophisticated. Within Gibbs-Thomson frame work, one can generally expect lower transition temperature than their bulk counterparts due to enhanced surface energy. Temperature-dependent Raman spectroscopy provided support for this deduction.\cite{Boulova2002}

% wo3 phases
\begin{table}[htb]
\centering
\caption{\ce{WO3} phases}\label{tab:wo3phase}
\begin{tabular}{lccccc}
\toprule
&&&\multicolumn{3}{c}{Lattice constants \AA} \\
\cmidrule(l){4-6}
 Symbol    & Temperature (\si{\degreeCelsius}) & Phase & a & b & c   \\
\midrule
$\epsilon$-\ce{WO3} & $ -140 \sim -50$  & monoclinic II & 7.378 & 7.378 & 7.664  \\
$\delta$-\ce{WO3} & $-50 \sim 17$  & triclinic & 7.309 & 7.522 & 7.686  \\
$\gamma$-\ce{WO3} & $17 \sim 330$  & monoclinic I & 7.306 & 7.540 & 7.692  \\
$\beta$-\ce{WO3} & $330 \sim 740$  & orthorhombic & 7.384 & 7.512 & 3.846  \\
$\alpha$-\ce{WO3} & $> 740$  & tetragonal & 5.25 & NA & 3.91  \\
$h$-\ce{WO3} &  $<400$  & hexagonal & 7.298 & NA & 3.899  \\
\bottomrule
\end{tabular}
\end{table}

\begin{figure}[htb]
\centering
\includegraphics[width=0.4\textwidth]{octwo3.jpg}
\caption[Octahedra model of \ce{WO3}]{Octahedra model of \ce{WO3}}
\label{fig:wo3oct}
\end{figure}


\ce{WO3} is wide gap n-type semiconductor with valence band top featuring $2p$ states of oxygen and conduction band bottom arising primarily from $5d$ states of tungsten with some mixing of oxygen $2p$ states.\cite{Gillet2004} \citeauthor{Migas2010a} maintained there is essentially identical band dispersion near the gap region in case of $\epsilon$-\ce{WO3}, $\delta$-\ce{WO3}, $\gamma$-\ce{WO3} and $\beta$-\ce{WO3}.\cite{Migas2010a} When there is oxygen vacancy, the Fermi level moves into the conduction band and the gap shrinks by about 0.5 eV. \citeauthor{Migas2010a} also pointed out the flat bands at VBM and CBM could lead to poor transport of holes and electrons, thus may compromising the function in photoelectrochemical cells.

 \citeauthor{Chatten2005} also studied the oxygen vacancy in different phases of \ce{WO3}.\cite{Chatten2005}

Nonstoichiometric tungsten oxides \ce{WO_x} (i.e. \ce{WO_{2.92}}, \ce{WO_{2.87}}) are known as Magn$\acute{e}$li phases.

We do not discuss tungsten oxide hydrates (\ce{WO3.nH2O}) in this work since the product of thermal CVD approach is not plagued with this complexity. It's necessary, however, to deal with hydrated \ce{WO3} in the liquid synthesis routes, as indicated in Section.~\ref{sec:woxgrowth}.

Theoretical computation of electronic band structures for \ce{WO_x} proves difficult due to the aforementioned phase transition.

 oxygen deficiency, structure change, electronic properties vary according.

% band gap table
\begin{table}[htb]
\centering
\caption{Tungsten oxides band gap }\label{tab:wo3eg}
\begin{tabular}{lccr}
\toprule
Phase & Experimental (eV) & Theory (eV) & Remarks  \\
\midrule
amorphous \ce{WO3} & 3.2  & NA &    \\
monoclinic bulk \ce{WO3} &  2.6   & 1.73\cite{Migas2010a}  &    \\
tetragonal bulk \ce{WO3} &     & 0.66 \cite{Migas2010a}&    \\
nano-\ce{WO3} & 2.6$\sim$3.2  & NA &    \\
nano-\ce{WO_{3-x}} & NA  & NA &    \\
\bottomrule
\end{tabular}
\end{table}

Tungsten bronzes was coined by Wohler in 1837.\cite{Deb2008} \ce{Na_{x}WO3}

\subsection{Synthesis strategies}\label{sec:woxgrowth}

We will first give a brief review on the synthesis of \ce{WO3}. As the chemical formula suggested, the most straightforward way is heating metallic tungsten in various forms (i.e. powders,\cite{Zhou2005a,Cao2009,Hsieh2010} foils and wires). Actually \ce{WO_x}\footnote{$x$ is between 2 and 3.} NWs was observed when directly heating W wires.\cite{Gu2002a} Due to the extreme high melting point of W, it requires very high temperature (say, above 1100 \si{\degreeCelsius}) to produce large amount of growth species. Therefore scalable growth seems difficult. The DC current heating proved to be a route with potential large yield. Tungsten wire or filament is connected to a voltage source, and substrate is positioned in proximity to the heated W wire.\cite{Lingfei2006,Thangala2007,Chang2007}

Tungsten oxides powder can also been used as precursor to prepare \ce{WO_{x}} \glspl{nw}.\cite{Huang2008a,Wang2009} Hydrogen-containing agents (i.e. water, \ce{H2},or methane\cite{Klinke2005}) were often involved into the reaction.\cite{Baek2007,Karuppanan2007} The potential benefit is \ce{WO3} has a much lower melting point ($\approx 1470$\si{\degreeCelsius}) than that of W.

Direct oxidation and \ce{WO3} conversion could be grouped as addition approach. In contrast, decomposition method is also feasible.
\citeauthor{Pol2005} obtained \ce{W18O49} nanorods by thermal dissociation of \ce{WO(OMe)4} at 700 \si{\degreeCelsius}, and \ce{WO3} by the annealing at 500 \si{\degreeCelsius} at air atmosphere.\cite{Pol2005}

Spray pyrolysis is a typical aerosol-assisted CVD technique. Typical process flow is: the precursor solution is pumped to an atomizer, and then sprayed through the carrier gas as a fine mist of very small droplets onto heated substrates. Subsequently the droplets undergo evaporation, solute condensation, and thermal decomposition, which then result in film formation.\cite{Zheng2011}

Main reactions include: \cee{W + O2 -> WO_{3-x}} \cee{W + H2O -> W_{3-x} + H2}

the energetic sources are ion bombardment, electron beam, laser ablation, and combustion flame\cite{Rao2011}.

The sol–gel process is a well-known, intensively studied wetchemical technique that is widely used in materials synthesis. This method generally starts with a precursor solution (the ``sol") to form discrete particles or a networked gel structure. During the course of gelation (aging process), various forms of hydrolysis and polycondensation take place.

Hydrothermal method has been an important route to synthesize a diversity of nanomaterials (i.e. ). Preparation of \ce{WO_{3-x}} has also been demonstrated hydrothermally.\cite{Lee2003,Gu2007} Usually, the precursor (\ce{H2WO4}) is mixed with other reactants (sulfides or certain organic acid) in solution and pH value is adjusted as another control degree of freedom. Then the solution is transferred into a sealed container (i.e. Teflon autoclave) and maintained at elevated temperature ($120 \sim 300$ \si{\degreeCelsius}) for tens of hours. Finally the product is separated from the solution and dried, layered \ce{WO3.nH2O} flakes are usual products.

In addition, doped \ce{WO3} was also demonstrated

 The composition and phase of final product highly depend on the synthesis conditions.

%\newgeometry{margin=1in}
\begin{landscape}
\begin{table}[htb]
\centering
\caption{Tungsten Oxides Growth and Application}\label{tab:wox}
{\footnotesize
\begin{tabular}{lp{3.5in}p{2.5in}c}
\toprule
composition  &  methods & highlights &  reference  \\
\midrule
\ce{WO3} & hot W filament (above 1500 \si{\degreeCelsius}) in Ar/\ce{O2} flow  & Cubic phase, PL, resistivity measured & \cite{Thangala2007} \\
\addlinespace[0.5em]
& W filament DC heating in \ce{NH3} or \ce{N2}/\ce{H2} flow  & multi phases, 100mg per batch, stable dispersion in both organic and aqueous solvents & \cite{Chang2007} \\
\addlinespace[0.5em]
& W powder heating Ar/\ce{O2} flow  & triclinic phase, CL at 370,415nm, UV-Vis, 3eV & \cite{Hsieh2010} \\
\cmidrule(l){2-4}
& \ce{WOx} film in \ce{H2} and \ce{CH4} flow  & monoclinic phase, tungsten carbide is the key & \cite{Klinke2005} \\
\cmidrule(l){2-4}
& \ce{Na2WO4.2H2O} by hydrothermal heating to test for line break & 3 different morphologies and photodegradation & \cite{Rajagopal2009}  \\
& \ce{H2WO4.2H2O}, \ce{H2O2} and poly(vinyl alcohol) etc by solvothermal  & multi phases deposition of FTO with varied bandgaps  & \cite{Su2010}  \\

\midrule
\ce{W18O49} & \ce{KOH} etching of W tips followed by heating in Ar flow  & oxygen from leakage, VS process& \textcite{Gu2002a} \\
\addlinespace[0.5em]
& \ce{NW(CO)6}, \ce{Me3NO.2H2O} and oleylamine by hydrothermal  &  RT PL at 350nm and 440nm & \cite{Lee2003}  \\
\bottomrule
\end{tabular}
}
\end{table}
\end{landscape}
%\restoregeometry



\section{Experimental}

Metallic tungsten powders were utilized as source materials in this work. Four kinds of W powder were used in total, as summarized in Table~\ref{tab:powder}. Two growth configurations were explored, namely ordinary transport and seeded growth.
% Tungsten powders size and purity
\begin{table}[htb]
\centering
\caption{Tungsten powder sizes and purities}\label{tab:powder}
\begin{tabular}{lccr}
\toprule
Name & purity & average size & vendor info\\
\midrule
3N   &  99.9\% & 17 $\mu$m & Alfa Aesar \#39749\\
3N5   &  99.95\% & 32 $\mu$m  & Alfa Aesar \#42477\\
4N5   &  99.995\% & 3.3 $\mu$m  & Materion T-2049 \\
5N   &  99.999\% & 1.5 $\mu$m & Alfa Aesar \#12973\\
\bottomrule
\end{tabular}
\end{table}

In a typical ordinary transport experiment (Fig.~\ref{fig:wogrow}), about 2 g tungsten powder was positioned in the upstream end of a quartz boat and downstream about 2.5 inch the substrate was stationed. 
% cvd NW growth
\begin{figure}[htb]
\centering
\includegraphics[width=0.6\textwidth]{CVD_and_temp.jpg}
\caption[\ce{WO3} NW CVD growth]{\ce{WO3} NW CVD growth and the temperature profile. The nominal NWs growth temperature were estimated according to the interpolation data. Zero inch location is defined at the upstream edge of furnace.}
\label{fig:wogrow}
\end{figure}
The substrates were p-type, boron-doped and [100]-orientated silicon with about 1 $\mu$m thermally-grown silicon oxide layer on the polished side. The cleaning procedures have been elucidated on page~\pageref{ch2sub}. The boat was then loaded into reaction chamber in a way that the powder source was placed in the center of heating zone, and the substrate upstream end was aligned at 6.5 inch.\footnote{As defined in Fig.~\ref{fig:wogrow}} After pumping down, 1 \gls{sccm} \ce{O2} and 10 sccm Ar were admitted into the chamber, respectively, after which the overall pressure read approximately 110 mTorr. The temperature ramped up to 1000 \si{\degreeCelsius} in 30 min and lasted for 4 h, and subsequently the apparatus was allowed to naturally cool down to room temperature.


\section{Results and Discussion on 1D Nanowire Growth}
\subsection{Tungsten Oxides: Impurities and Residual Effects}\label{sec:nawox}

The growth dynamics of each source were investigated in ordinary transport configuration. In each experiment, the temperature, flow rate and source and substrate locations were controlled. Another parameter the author found important is the growth number\footnote{Defined as the order of growth performed in the same reaction chamber} in the same reaction chamber. Usually one will conduct several runs in the same chamber, and different results may arise with respect to the growth number. With this degree of freedom included, the primary features of the deposition and the associated parameters were summarized in Table~\ref{tab:wot}.
% \ce{WO_x} ordinary transport results
\begin{table}[htb]
\centering
\caption{\ce{WO_x} ordinary transport results}\label{tab:wot}
\begin{tabular}{lp{2in}p{2in}r}
\toprule
\multicolumn{2}{c}{Growth Number} \\
\cmidrule(l){2-3}
 Source   & First & Second and more & Sodium(ppm)   \\
\midrule
3N      & Morphology from high to low temperature: islands, dense layer and a few NWs & Dense wires become more and more, sodium tungsten oxides phases dominate & 20  \\
3N5     & islands in high temperature end and layers in low temperature end & NA &      NA\\
4N5     & dense islands, some tiny wires in low temperature end & NA & 0.065 \\
5N      & similar to those of 4N5  & NA & 0.05\\
\bottomrule
\end{tabular}
\end{table}

The change of metallic W source after growth carries important information on the growth. So before examining the deposition on substrates, the SEM images of all powders before and after growth are presented in Fig.~\ref{fig:pdbefore} and Fig.~\ref{fig:pdafter}, respectively.
% powder before growth
\begin{figure}[htb]
\centering
\includegraphics[width=0.6\textwidth]{pd_before.jpg}
\caption[SEM images of W powder before growth]{SEM images of W powder before growth for (a) 3N, (b) 3N5, (c) 4N5 and (d) 5N, respectively.}
\label{fig:pdbefore}
\end{figure}
Tungsten metal crystallizes in body center cubic (BCC) phase. Overall, the facets were still discernible. The 4N5 and 5N powders share similar appearance and size distribution (Fig.~\ref{fig:pdbefore}c-d). In contrast, 3N powders are almost half the size of 3N5 ones, and have more uniform size distribution as well. Absolute size and its distribution play a key role in its usage as seed, as discussed in Section~\ref{sec:sgfg}.
% powder after growth
\begin{figure}[htb]
\centering
\includegraphics[width=0.6\textwidth]{pd_after.jpg}
\caption[SEM images of W powder after growth]{SEM images of W powder after growth for (a) 3N, (b) 3N5, (c) 4N5 and (d) 5N. }
\label{fig:pdafter}
\end{figure}
After growth, 4N5 and 5N powder (Fig.~\ref{fig:pdafter}c-d) both develop the round edges, while 3N and 3N5 powders (Fig.~\ref{fig:pdafter}a-b) become bundled rods. This difference mainly arises from the average size, which will dictate the degree of oxidation. Another possible reason is the presence of foreign elements in each W powders. As indicated by the certificate of analysis from corresponding vendors, carbon, nickel, iron and sodium can amount up to tens of ppm in low purity W sources. 

The sodium contents in 3N powders fundamentally change the deposition in ordinary transport growth. Typical morphologies of the as-synthesized nanostructures grown on a \ce{SiO2}-Si substrate were shown in Fig.~\ref{fig:nawoxsemedx}a. The nanostructures were \glspl{nw} with lengths up to several hundred microns and diameters about 40 to 500 nm. The deposition of nanowires was generally located on the substrates with the growth temperature ranging from 660 to 420 \si{\degreeCelsius}. Enlarged view at higher magnification revealed some nanowires were cylindrical and some were bundled belt. Rectangular microplate structures grew at lower temperature end of substrate, as displayed in Fig.~\ref{fig:nawoxsemedx}b and inset. The chemical compositions of the deposition were analyzed using \gls{edx}. A representative \gls{edx} spectrum in Fig.~\ref{fig:nawoxsemedx}c shows the existence of W, Na, O, and Si signals in the specimen, where the Si signal is from the substrate. The crystal structures of the as-synthesized specimens were examined using XRD. The diffraction peaks were carefully indexed and the deposition was identified as two phases of sodium tungsten oxides and one phase of tungsten oxide as indicated on Fig.~\ref{fig:nawoxsemedx}d. The two sodium tungsten oxide phases are the triclinic \ce{Na5W_{14}O_{44}} phase (ICDD PDF 04-012-4449, \emph{a}=7.2740 \si{\angstrom}, \emph{b}=7.2911 \si{\angstrom}, \emph{c}=18.5510 \si{\angstrom}, $\alpha$=96.37$^\circ$, $\beta$=90.89$^\circ$, $\gamma$=119.65$^\circ$) and the triclinic \ce{Na2W4O_{13}} phase (ICDD PDF 04-012-7108, \emph{a}=11.1630 \si{\angstrom}, \emph{b}=3.8940 \si{\angstrom}, \emph{c}=8.2550 \si{\angstrom}, $\alpha$=90.60$^\circ$, $\beta$=131.36$^\circ$, $\gamma$=79.70$^\circ$). The tungsten oxide is the monoclinic \ce{WO3} phase (ICDD PDF 01-083-0950, \emph{a}=7.30084 \si{\angstrom}, \emph{b}=7.53889 \si{\angstrom}, \emph{c}=7.6896 \si{\angstrom}, $\beta$=90.89$^\circ$).

% nawox sem edx
\begin{figure}[htb]
\centering
\includegraphics[width=0.7\textwidth]{nawox_semedx.jpg}
\caption[SEM and EDX on \ce{Na_xWO3}]{SEM images of (a) dense array of the as-synthesized ultra-long nanowires on \ce{SiO2}-Si substrate and (b) rectangular microplates grow among the nanowires. (c) EDS and (d) XRD spectra showing the chemical compositions and phases of the deposition.}
\label{fig:nawoxsemedx}
\end{figure}

XRD measurements revealed the overall structure of the specimen. To obtain detailed information on the crystallinity, composition, and growth direction of the nanostructures, TEM analyses with imaging, electron diffraction, and \gls{edx} were performed on more than 20 nanowires. The majority of the nanowires were identified as the \ce{Na5W_{14}O_{44}} phase (Fig.~\ref{fig:nawoxtem}a-c). The HRTEM image in Fig.~\ref{fig:nawoxtem}b reveals that the nanowire exhibits a single-crystalline structure. The corresponding diffraction pattern in the inset of Fig.~\ref{fig:nawoxtem}b was recorded in $[\bar{1}10]$ axis. Based on the analyses on a series of diffraction patterns and HRTEM images, the nanowire was confirmed as the triclinic \ce{Na5W_{14}O_{44}}. The growth direction of \ce{Na5W_{14}O_{44}} NWs was determined to be parallel to the (001) plane. EDX measurement (Fig.~\ref{fig:nawoxtem}c) shows the nanowire consisted of W, Na, and O. No other elements were detected.\footnote{The Cu and C signals come from the supporting Cu grid with lacey carbon and the Cr signal from the tip of JEOL double tilt holder.} Small fraction of nanowires were found to be the monoclinic \ce{WO3} phase (Fig.~\ref{fig:nawoxtem}d-f). The SAED shown in the inset of Fig.~\ref{fig:nawoxtem}e was recorded in [100] zone. The growth direction of the \ce{WO3} nanowire was determined to be perpendicular to the (002) plane with a d-spacing of 0.38 nm. EDX spectrum in Fig.~\ref{fig:nawoxtem}f shows the nanowire consisted of W and O without Na or other impurities. The rectangular microplate structures shown in Fig.~\ref{fig:nawoxsemedx}b were also collected for TEM examination. These microplate structures were identified to be the triclinic \ce{Na2W4O_{13}} phase. The SAED in Fig.~\ref{fig:nawoxtem}h was recorded in [101] axis. No high-quality HRTEM images were acquired for the \ce{Na2W4O_{13}} plate due to its thickness. The long edge of the plate was identified as parallel to the (010) plane and the short edge parallel to the ($\bar{1}01$) plane. From XRD spectra, the long edge of \ce{Na2W4O_{13}} plate can be figured out as the $\vec{c}$ axis and the short edge is along the $\vec{b}$ axis. Fig.~\ref{fig:nawoxtem}i shows the plate structure also consisted of W, Na, and O and no other impurities were detected. Normalized to the highest W peak, the Na peak intensity in Fig.~\ref{fig:nawoxtem}i is higher than the one in Fig.~\ref{fig:nawoxtem}c. This result indicates a higher Na:W ratio for the \ce{Na2W4O_{13}} plate structure than that of \ce{Na5W_{14}O_{44}} nanowires, in consistent with the compositions of these two phases.
% tem
\begin{figure}[htb]
\centering
\includegraphics[width=0.8\textwidth]{nawox_tem.jpg}
\caption[TEM analyses on sodium tungsten oxide sample]{TEM images at low-magnification, HRTEM images, and EDS spectra of a \ce{Na5W_{14}O_{44}} NW (a, b, c), a \ce{WO3} NW (d, e, f), and a \ce{Na2W4O_{13}} microplate (g, h, i).}
\label{fig:nawoxtem}
\end{figure}

Micro-Raman spectroscopy was also carried out at room temperature in ambient atmosphere to confirm the phases of the microplates and ultra-long nanowires. Fig.~\ref{fig:nawoxram}a shows the m-Raman spectrum of the microplates. The Raman lines at 949, 794-777, 366-263 \si{cm^{-1}} closely matches the reported \ce{Na2W4O_{13}} Raman shift frequencies.\cite{Fomichev1992} The Raman spectrum from the ultra-long nanowires is shown in Fig.~\ref{fig:nawoxram}b. The spectrum shows major peaks in 100-150 \si{cm^{-1}}, 650-900 \si{cm^{-1}}, and 900-100 \si{cm^{-1}} regions. As compared to the reported peak positions (shown by reference lines in Fig.~\ref{fig:nawoxram}b), the nanowires consist of only little or no amount of \ce{WO3} and \ce{Na2W4O_{13}} phases. Major peaks at 107, 695, 765, 913, 943, and 965 \si{cm^{-1}} cannot be assigned to reference in literatures. According to XRD and TEM analyses, the nanowires are mainly \ce{Na5W_{14}O_{44}}; therefore these peaks should belong to the \ce{Na5W_{14}O_{44}} phase. So far no Raman spectrum has been reported for this phase, hence this is probably the first observed Raman pattern for \ce{Na5W_{14}O_{44}}.
% nawox raman
\begin{figure}[htb]
\centering
\includegraphics[width=0.7\textwidth]{nawox_raman_series}
\caption[Raman spectra on \ce{Na_xWO3}]{Raman spectra of (a) microplates matching the reported peak positions of the \ce{Na2W4O_{13}} phase, and (b) nanowires with comparisons to the reported major peak positions of \ce{WO3} (shown by black line) and \ce{Na2W4O_{13}} (shown by red line) phases.}
\label{fig:nawoxram}
\end{figure}
Based on the correlation of Raman spectra and crystallographic data of a variety of tungsten bronze, an empirical formula to relate the Raman peaks and \ce{W-O} bonding lengths has been found as
\begin{equation}\label{eq:wobond}
\nu = 25823 \exp(-1.902\cdot R)
\end{equation}
where $R$ is tungsten-oxygen bond length in \si{\angstrom}, and $\nu$ is Raman stretching wavenumber in \si{cm^{-1}}.\cite{Hardcastle1995} The standard deviation of estimating \ce{W-O} bond distance from Raman stretching wavenumber is $\pm 0.034$ \si{\angstrom}. The observed Raman peaks of \ce{Na5W_{14}O_{44}} phase lies at 965, 943, 913, 808, 786, 778, 765, 695 and 107 \si{cm^{-1}}. Multi-peak Lorentzian fitting is preformed to precisely determine the central maximum. Good fitting is obtained as shown in Fig.~\ref{fig:naworamfit}. The calculated \ce{W-O} bond distances using Eq.~\ref{eq:wobond} are then listed in Table~\ref{tab:nawobond}. The predicted \ce{W-O} bond lengths agree very well with the crystallographic value of \ce{Na5W_{14}O_{44}} phase.\cite{Triantafyllou1999a} The 107 \si{cm^{-1}} peak probably is caused by \ce{Na-O} bond.
% Na5 raman fitting
\begin{figure}[htb]
\centering
\includegraphics[width=0.6\textwidth]{naxwo_ramfit}
\caption[Multi-peak Lorentzian fitting on \ce{Na5W_{14}O_{44}}]{Multi-peak Lorentzian fitting on two major peaks region of \ce{Na5W_{14}O_{44}}. The peaks sum height difference is caused by different baseline value adopted in each fitting.}
\label{fig:naworamfit}
\end{figure}

% W-O bond length
\begin{table}[htb]
\centering
\caption{\ce{W-O} bond length predication}\label{tab:nawobond}
\begin{tabular}{cccc}
\toprule
Fitted center (\si{cm^{-1}}) & length (\AA) & Fitted center (\si{cm^{-1}}) & length (\AA) \\
\midrule
694.6 & 1.900 &  808.6 &  1.821 \\
745.4 & 1.863 &  911.5 &  1.758 \\
764.4 & 1.850 &  933.0 &  1.745 \\
778.7 & 1.840 &   943.5 & 1.740 \\
788.4 & 1.834 &   965.4 & 1.728 \\
\bottomrule
\end{tabular}
\end{table}
For a comprehensive comparison purpose, the Raman fingerprints of various tungsten oxides and tungstate is shown in Table.~\ref{tab:woram2}. 
\begin{table}[htb]
\centering
\caption{Raman fingerprints of tungsten oxides and sodium tungsten oxides}\label{tab:woram2}
\begin{tabular}{lp{3in}r}
\toprule
Phase & Raman Shift (\si{cm^{-1}}) &  Reference   \\
\midrule
\ce{WO2}  & 287(s), 334(w), 514(m), 600(w), 621(w), 785(vs) & \cite{Ma2005} \\
\ce{W18O49}  & broad bands from 750 to 780 & \cite{Guo2012} \\
             &  267(s), 778(s), 969(m) & \cite{Liu2013d} \\
m-\ce{WO3}  & 131(m), 265(m), 327(m), 715(s), 807(vs) &  \cite{Salje1975a,Daniel1987} \\
h-\ce{WO3}  & 162(m), 253(m), 320(m), 645, 690(s), 817(vs) &  \cite{Daniel1987}\\
\ce{WO3.H2O}  & 230(m), 377(w), 428(w), 650(s), 816(vs), 946(vs) &  \cite{Daniel1987} \\
\ce{WO3.2H2O}  & 235, 268(m), 380(w), 662(s), 685(vs), 960(vs) & \cite{Daniel1987} \\
\ce{Na2WO4}  & 94(w), 314(vw), 377(m), 813(m), 930(vs) &  \cite{Lima2011} \\
\ce{Na2W2O7}  & 381(w), 422(w), 763(w), 835(s), 886(m), 948(m), 957(vs) &  \cite{Knee1979} \\
\ce{Na2W4O13} & 263(w), 272(w), 311(w), 366(w), 777(vs), 794(s), 949(m) &\cite{Fomichev1992}\\
\ce{Na5W14O44} & 107(s),695(m), 765(vs), 913(w), 943(m), 965(m)& this work\\
\bottomrule
\end{tabular}

Materials in solid state, 
vw-very weak; w-weak; m-medium; s-strong; vs-very strong
\end{table}

Heated at elevated temperatures, the W precursor, quartz tube, and quartz boat were three possible sources for Na. The sodium concentration in 3N source was 20 ppm as provided by vendor.\footnote{Due to the non-uniform distribution of foreign elements, detection of sodium by EDX is possible although the concentration is below the limit of typical EDX capacity.} The Na concentrations in quartz tube and quartz boat (GE 214 quartz, Wilmad Labglass) were 1 ppm and 0.7 ppm, respectively. To verify the major source of Na content coming from 3N powder, control experiments were employed by using W source with higher purities (4N5 or 5N). These ultra-high purity W sources produced pure tungsten oxide deposition and no Na content was detected by EDX. These results verified the Na content was mainly from the 3N W powders. In the OT growth experiments, the W source was first slowly oxidized at 1000 \si{\degreeCelsius} with 1 sccm \ce{O2} and the oxidized source was then evaporated producing \ce{WO3} and other sub-stoichiometric tungsten oxide vapors. The tungsten oxide vapors produced this way were limited by the slow oxidation, as indicated by
\begin{enumerate*}[label=\itshape\alph*\upshape)]
\item only small amount of the tungsten source was consumed during the growth; and
\item only the top layer of the tungsten source was oxidized showing a dark blue color after growth.
\end{enumerate*} Due to its low concentration, the exact composition of the Na content in the tungsten source is difficult to probe. However, the evaporation rate of the Na content is expected to be high when heated at 1000 \si{\degreeCelsius}. This assumption is supported by the observation that no Na was found in the deposition along the whole substrate in control experiments using the used 3N W source. Put it in another way, this means that the majority of Na content was already evaporated in previous experiment. Hence, despite of its low concentration in the source, the total amount of Na-based vapors produced during the growth was significant, resulting in the dominance of sodium tungsten oxide phases in the nanowire growth region.

\begin{figure}[htb]
\centering
\includegraphics[width=0.8\textwidth]{nawox_edx_series.jpg}
\caption[Photograph of a series of specimens grown]{(a) Photograph of a series of specimens grown in the same reaction chamber with the increase of growth number showing different growth zones and corresponding morphology changes. (b) Change of Na/W ratios with the increase of growth number at different growth zones.}
\label{fig:nawoxser}
\end{figure}

As shown by the photograph in Fig.~\ref{fig:nawoxser}a, for a series of growth performed in the same reaction chamber, the morphology of the nanowires changes with the increase of growth number. Three growth zones of nanowires can be identified on basis of the coverage and density of the nanowires, as delimited by the vertical guide line. The growth Zone I was located at high temperature end ranging from 660 to 520 \si{\degreeCelsius}, Zone II was from 520 to 470 \si{\degreeCelsius}, and Zone III from 470 to 420 \si{\degreeCelsius}. Detailed morphological changes are depicted in Fig.~\ref{fig:nawoxsemall}.
\begin{figure}[htb]
\centering
\includegraphics[width=0.8\textwidth]{nawox_sem_series.jpg}
\caption[SEM images of morphology evolution]{SEM images of morphology evolution of the NWs at different growth zones showing morphology changes with the increase of growth number. The insets show detailed structures of the NWs.}
\label{fig:nawoxsemall}
\end{figure}
For the first growth (Fig.~\ref{fig:nawoxsemall}a-c) in a new clean reaction chamber, there were only a little nanowires scattered in Zone I (Fig.~\ref{fig:nawoxsemall}a), ultra-long nanowires were mostly found in Zone II with higher density and coverage (Fig.~\ref{fig:nawoxsemall}b), and in Zone III the coverage, density, and length of nanowires gradually reduced as the growth location moved downstream (Fig.~\ref{fig:nawoxsemall}c). With growth number increased to 3 (Fig.~\ref{fig:nawoxsemall}d-f) and 5 (Fig.~\ref{fig:nawoxsemall}g-i), the coverage, density, and length of the nanowires increased in all three zones. Zone I saw the most significant morphology changes as shown by Fig.~\ref{fig:nawoxser}a and Fig.~\ref{fig:nawoxsemall}g. After the fifth growth in the same reaction chamber, the deposition of ultra-long nanowires almost covered all three zones. For Zones I and II, no significant morphology change was observed with further increase of the growth number up to 8. However, for Zone III the microplate structures (shown in inset of Fig.~\ref{fig:nawoxsemall}i) kept increasing significantly with the growth number. EDX study revealed the variation of Na concentration in the deposition with the growth number. Fig.~\ref{fig:nawoxser}b shows the variation of the Na/W ratios with the increase of growth number at different nanowire growth zones. Generally, the Na/W ratios at all growth zones increased with the increase of growth number. The large fluctuation of Na/W ratio at Zone III was presumably due to the uneven distribution of Na content with different morphologies at Zone III.

\begin{figure}[htb]
\centering
\includegraphics[width=0.9\textwidth]{nawox_sch.jpg}
\caption[Schematic drawings of residual \ce{Na_xWO3} growth]{Schematic drawings showing (a) growth with tungsten source in a new tube, (b) enhanced growth with both residue deposition and tungsten source, and (c) growth with residue deposition only.}
\label{fig:nawoxsch}
\end{figure}

A mechanism of residue deposition enhanced growth was proposed to explain the morphology evolution of the nanowires with the growth number. Since the CVD setup employed in this thesis is a hot-wall system, deposition formed on the substrate surface as well as on the inner wall of the reaction chamber in the growth area. Residue deposition was found on the quartz tube which can be distinguished by the color change on the quartz tube after each growth. With the increase of growth number for the same reaction chamber, the residue deposition on the inner wall of the tube also increased. If a clean quartz tube free of any residue deposition is used as the reaction chamber, the vapors are supplied only from the source material forming deposition in the growth area with lower temperature (as shown by the schematics in Fig.~\ref{fig:nawoxsch}a). When a quartz tube with residue deposition from previous depositions is used repeatedly as the reaction chamber, the residue deposition heated at growth temperature will also produce vapor locally. Therefore, the vapors from both the source materials and the local residue deposition will result in an enhanced growth of nanowires (Fig.~\ref{fig:nawoxsch}b). To prove this hypothesis, control experiments were performed utilizing a quartz tube used multiple times previously (e.g. a growth number of 9). The experiments were carried out without any source materials while other growth parameters remain the same. Similar nanowires were found on the substrate in the nanowire growth zones. Without the source materials, the residue deposition from the tube surface was the only possible source that could produce vapors forming nanowires on the substrate (Fig.~\ref{fig:nawoxsch}c). This result confirmed the nanowire growth could be enhanced by the presence of the residue deposition and explained the morphology changes of the nanowires with the increase of growth number.
%xrd na5
\begin{figure}[htb]
\centering
\includegraphics[width=0.8\textwidth]{wox_xrd_series}
\caption[XRD spectra of the as-synthesized specimen with the increase of growth number from 1 to 5]{XRD spectra of the as-synthesized specimen with the increase of growth number from 1 to 5.}
\label{fig:nawoxxrd}
\end{figure}
The general trend of Na/W ratios with the growth number can also be explained by the aforementioned effect of residue deposition. The sodium contents from the residue deposition will join the new Na contents from the W source forming deposition. Fig.~\ref{fig:nawoxxrd} displays the XRD spectra of a series of specimens with the growth number increases from 1 to 5. The XRD spectra reveal the relative concentration dynamics of the three phases as the growth number increases. Compared to the \ce{WO3} phase, the \ce{Na5W_{14}O_{44}} and the \ce{Na2W4O_{13}} phases increase significantly as the growth number increase from 1 to 5. And in the 5th growth, \ce{Na2W4O_{13}} become the dominated phase compared to the other two phases. These results are consistent with the SEM observation (Fig.~\ref{fig:nawoxsemall}) and the composition change from the EDX measurements (Fig.~\ref{fig:nawoxser}b). Details XRD indices assignment is summarized in Table~\ref{tbl:wo3xrd}.

\begin{table}
\centering
\caption{Assignment of XRD peak indices to different phases}\label{tbl:wo3xrd}
\begin{tabular}{llllll}
\toprule
\ce{WO3} &          &\ce{Na5W14O44} &      & \ce{Na2W4O13} & \\
2$\theta$   & hkl   & 2$\theta$   & hkl    & 2$\theta$   & hkl   \\
\midrule
 23.05   & 0 0 2 & 9.65    & 0 0 2  & 10.85   & 1 0 0 \\
 23.59   & 0 2 0 & 14.49   & 0 0 3  & 21.8    & 2 0 0 \\
 24.31   & 2 0 0 & 19.37   & 0 0 4  & 32.97   & 3 0 0 \\
 26.60   & 1 2 0 & 24.27   & 0 0 5  & 56.45   & 5 0 0 \\
 34.10   & 2 0 2 & 29.22   & 0 0 6  &         &       \\
 47.11   & 0 0 4 & 39.36   & 0 0 8  &         &        \\
 48.29   & 0 4 0 & 44.46   & 0 0 9  &         &        \\
 49.79   & 4 0 0 & 49.71   & 0 0 10 &         &        \\
         &       & 55.23   & 0 0 11 &         &        \\
         &       & 60.77   & 0 0 12 &         &        \\
\bottomrule
\end{tabular}
\end{table}

The \ce{Na5W14O44} phase was further confirmed by a XRD scanning from 4 to 10 degree in $2\theta/\Omega$ configuration. As shown in Fig.~\ref{fig:naxrdlow}, two peaks at 4.84$^{\circ}$ and 9.65$^{\circ}$ were well resolved, corresponding a lattice spacing of 18.25 and 9.16 \si{\angstrom}, respectively. This compares favorably with both HRTEM analysis (Fig.~\ref{fig:nawoxtem}e) and ICDD PDF 04-012-4449 database reference. 
%061212 sample
\begin{figure}[htb]
\centering
\includegraphics[width=0.5\textwidth]{naxwo_xrd_low}
\caption{Low angle scan of XRD pattern on the as-grown sodium tungsten oxide sample}
\label{fig:naxrdlow}
\end{figure}

The optical transmission and reflection properties of the as-grown \ce{Na5W14O44} NWs were evaluated using UV-Vis spectroscopy, as shown in Fig.~\ref{fig:ch3naxuvvis}. The sample was prepared on a quartz substrate, which is transparent from 300 to 1300 nm. SEM images (not shown) found dense NWs array growth similar to those on \ce{SiO2}-Si substrates, indicating the residue effect is not sensitive to the substrate. 
%051412 sample
\begin{figure}[htb]
\centering
\subfloat[]{\label{fig:nauv}\includegraphics[width=0.45\textwidth]{uvvis_naxwo3}}\hspace{0.04\textwidth}
\subfloat[]{\label{fig:nadrs}\includegraphics[width=0.45\textwidth]{uvvis_naxwo2}}
\caption[UV-Vis and \gls{drs} of \ce{Na5W14O44} NWs on quartz]{UV-Vis and DRS of \ce{Na5W14O44} NWs on quartz: (a) Optical transmission and (b) diffuse reflection spectra.}
\label{fig:ch3naxuvvis}
\end{figure}
The transmission (Fig.~\ref{fig:nauv}) was about 70\% above 400nm, and began to drop significantly for shorter wavelength. This variation was also identified in the diffuse reflection spectrum in Fig.~\ref{fig:nadrs}, where the reflection reached to a maximum of 16\% at 390 nm. One difference is the DRS increased from 1300 nm to 400nm, presumably due to the inverse scattering intensity with NW dimension of 300 nm. The calculated absorption was shown in Fig.~\ref{fig:nauv}, suggesting an energy gap edge at about 380 nm. This is consistent with the observed white sample color; however, the conductivity of \ce{Na5W14O44} has not been investigated experimentally. Based on the electron sum rule, the sodium tungsten oxide can be cast into \ce{Na5W_1^{5+}W_{13}^{6+}O44}, which indicates the W atom state is quite close to that in \ce{WO3}. More investigation is needed to vigorously determined the electronic transport properties of \ce{Na5W14O44} NWs. 

Typical morphologies of ordinary transport growth using other powders are shown in Fig.~\ref{fig:wox3n5} and Fig.~\ref{fig:wox4n5}, respectively. The common feature of these samples is the dominance of islands growth in upstream end and layers growth in downstream end. The exact dimension still shows some minor difference, particularly in low temperature region. The deposition with 4N5 and 5N have much smaller grain size than that with 3N5 powder.
% 3N5 powder growth
\begin{figure}[htb]
\centering
\includegraphics[width=0.9\textwidth]{wox3n5.jpg}
\caption[SEM images of \ce{WO3} growth using 3N5 powder]{SEM images of \ce{WO3} growth using 3N5 powder showing the morphology variation from (a) high temperature, (b) intermediate temperature, to (c) low temperature.}
\label{fig:wox3n5}
\end{figure}

% 4N5 powder growth
\begin{figure}[htb]
\centering
\includegraphics[width=0.9\textwidth]{wox4n5.jpg}
\caption[SEM images of \ce{WO3} growth using 4N5 powder]{SEM images of \ce{WO3} growth using 4N5 powder showing the morphology variation from (a) high temperature and (b) intermediate temperature, to (c) low temperature. Growth with 5N powders have almost the same morphology with that of 4N5, and is not shown hereby.}
\label{fig:wox4n5}
\end{figure}
It is worth pointing out that there are some NWs growth in low temperature end associated with 4N5 growth(Fig.~\ref{fig:wox4n5}c), although not in large scale.

\subsection{Tungsten Oxides: Seeded Growth}\label{sec:sgfg}

As mentioned in Sec.~\ref{sec:nawox}, growth using higher purity tungsten powders primarily produces thin film deposition, and growth using 3N powder is plagued with the impurity effect producing sodium tungsten oxide NW instead of tungsten oxide NW. New growth method was thus explored to obtain \ce{WO3} NWs. In this study, a seeded growth approach was developed and proved to be a successful alternative way. In the seeded growth, additional tungsten powders was used on the substrate serving as local seed, on which dense \ce{WO3} NW array grew. 

Understanding the oxidation of tungsten powder is the key to obtain high yield in seeded growth. Oxidation of tungsten have been investigated under diverse conditions, such as at elevated temperature (\textgreater 1100 \si{\degreeCelsius}) and oxygen pressure on the order of Torr,\cite{Base1965} and at temperatures ranging from 20 to 500 \si{\degreeCelsius} under atmosphere pressure.\cite{Warren1996} \ce{WO_x} NWs were readily found when tungsten (foil, wire, or powder) is oxidized under various conditions.\cite{Zhu1999,Karuppanan2007,Hsieh2010} However the study on tungsten powder oxidation behavior between intermediate temperature range and under low pressure is still rare. This thesis studied the oxidation of tungsten powders with diverse size within temperature range from 500 to 1000 \si{\degreeCelsius} and under several mTorr oxygen partial pressure. It was illustrated that using tungsten powder as seed is an economic approach to obtain high yield of \ce{WO3} nanowires at relatively lower temperature. It was also demonstrated that there was an optimal tungsten powder size under current experimental conditions for seeded growth. This observation will provide some insight on the role of tungsten powder as source material in CVD growth of \ce{WO_x}.

Commercial available tungsten powders with different size are usually associated with purity variation as well. Four kinds of tungsten powders were used as precursor to prepare \ce{WO3} NWs, as already summarized in Table~\ref{tab:powder}. The dimensions of tungsten powder were obtained by measuring the average size in SEM graphs. A systematic investigation was performed on the oxidation behavior of tungsten powder to evaluate the temperature effect, size-dependence and influence of oxygen partial pressure.

In a typical oxidation experiment, tungsten powders were loaded into the uniform heating zone and the sealed chamber was pumped down to an ultimate pressure of $5\sim8$ mTorr. Then oxygen flow varying between 1 sccm to 10 sccm was admitted from upstream inlet. With 10 sccm UHP Ar (99.999\%) as carrier gas, the overall pressure reached to about 100 mTorr. The heating temperature (500 to 750 \si{\degreeCelsius}) was ramped up in 30 min and lasted for 30 min. Then the heating power was turned off and the chamber was allowed to naturally cool down to room temperature.

% seed optimal
\begin{figure}[htb]
\centering
\includegraphics[width=0.7\textwidth]{JAP-2column_Fig1.jpg}
\caption[W powder oxidation: temperature effect]{W powder oxidation: temperature effect. SEM graphs of 99.9\% (3N) tungsten powder oxidization at different temperatures of a) 500 \si{\degreeCelsius}, b) 600 \si{\degreeCelsius}, c) 650 \si{\degreeCelsius}, d) 750 \si{\degreeCelsius}, showing the optimal temperature for local formation of nanowires is between 600-650 \si{\degreeCelsius}. Oxygen flow rate is 1 sccm.}
\label{fig:pdtemp}
\end{figure}

Fig.~\ref{fig:pdtemp} illustrated the effect of temperature on the morphological change and surface nanowires formation of 3N powder. At 500 \si{\degreeCelsius}, most tungsten powder retained its original shape and a layer of tiny dense NWs begun to grow. When temperature was increased to 600 \si{\degreeCelsius}, 3N powder started to crack with longer NWs on the isolated surface. Further increase of temperature lead to irregular shapes of tungsten power and aggregation of NWs, giving rise to the nanorods and bunched or bundled structures. It could be determined from the morphology variation that the optimal seeded growth temperature for 3N powder was in the range of 600 to 650 \si{\degreeCelsius}.
% seed optimal
\begin{figure}[htb]
\centering
\includegraphics[width=0.7\textwidth]{JAP-2column_Fig3.jpg}
\caption[W powder oxidation: size effect]{W powder oxidation: size effect. SEM graphs illustrating the oxidization of four different size of tungsten powders at 600~\si{\degreeCelsius} and 1 sccm oxygen flow. a) 17 $\mu$m, b) 32 $\mu$m, c) 3.3 $\mu$m, d) 1.5 $\mu$m.}
\label{fig:pdsize}
\end{figure}

Fig.~\ref{fig:pdsize} depicted the oxidation of different sizes of tungsten powder under the same experimental conditions. In contrast to the morphology of 3N powder shown in Fig.~\ref{fig:pdtemp}, 3N5 powder surface is primarily covered with sub-micron particles as well as some short tiny NWs; whereas 4N5 and 5N powder were thoroughly oxidized, showing branched flowers feature. This dramatic difference could be explained in terms of surface energy and oxygen diffusion. With smaller dimension, the increased surface-to-volume ratio and short diffusion path both lower the energy barrier of oxidation.\cite{tungsten1999} It was logical to deduce that higher temperature or increased oxygen level might favor the NWs formation on 3N5 powder. When it comes to seeded growth, however, the powder size distribution was an important factor to give uniform NWs deposition. Since the size distribution of 3N powder is more uniform than that of 3N5 powder, this study used the former as seeds.
% seed optimal
\begin{figure}[htb]
\centering
\includegraphics[width=0.7\textwidth]{JAP-2column_Fig2.jpg}
\caption[W powder oxidation: oxygen pressure]{W powder oxidation: oxygen pressure. SEM graphs of 3N tungsten powder oxidization at 600 \si{\degreeCelsius} under different rates of oxygen flow: a) 1 sccm, b) 2 sccm, c) 3 sccm, d) 10 sccm. The oxygen partial pressures were 13 mTorr, 23 mTorr, 32 mTorr, and 82 mTorr, respectively with background pressure subtracted.}
\label{fig:pdoxy}
\end{figure}

Fig.~\ref{fig:pdoxy} depicted the morphology change of 3N powder with respect to varied oxygen partial pressure. When the oxygen flow is lower than 3 sccm, 3N powder almost stayed as the same, with cracks separating the dense layer of NWs. When oxygen flow is increased to 10 sccm, the 3N powder exemplified an enlarged version of that for 4N5 or 5N powder under 1 sccm oxygen flow. This observation again supported the surface energy explanation.

With all above oxidation experiments, favorable conditions for local growth of NWs were extracted to perform seeded growth. The seeded growth was performed by placing high purity tungsten powders (4N5 or 5N) in heating zone and receiving substrate with 3N tungsten powders in downstream location where the temperature was about 600 \si{\degreeCelsius}. The temperature profile and source and substrate locations remain essentially the same as in Fig.~\ref{fig:wogrow} except the presence of 3N powder on substrate. In all experiments, tungsten powders were uniformly distributed by sliding two pieces of substrates and the growth time was kept at 4 h.
% sg sem
\begin{figure}[htb]
\centering
\subfloat[]{\label{fig:sga}\includegraphics[width=0.4\textwidth]{wox_sg_a.jpg}}\hspace{0.04\textwidth}
\subfloat[]{\label{fig:sgb}\includegraphics[width=0.4\textwidth]{wox_sg_b.jpg}}
\caption[SEM characterization of \ce{WO3} seeded growth]{SEM characterization of \ce{WO3} seeded growth. (a) SEM graphs of \ce{WO3} NWs on \ce{SiO2}/Si substrate. (b) A high magnification view showing uniform NW growth and close-up view of one NW.}
\label{fig:woseedsem}
\end{figure}
As shown in Fig.~\ref{fig:sgb}, dense NWs array was obtained on tungsten powder seeds with individual wires of length up to 10 $\mu$m and diameter about 50 to 200 nm, according to the measurements made in the close-up view. Each tungsten powder stood as independent growth site (Fig.~\ref{fig:sga}) with island-layer growth on the substrates, a common feature without using tungsten powder as seed under current experimental conditions. It was occasionally observed that NWs growth was initiated adjacent some tungsten powders. This phenomenon was correlated to the local trap of vapor flow since it was more often found among the enclosed area by tungsten powders. It is also found that the diameter of NWs decrease as the distance between powders and upstream edge increases. This is a combination effect of lower temperature and reduced \ce{WOx} growth species supply. Similar phenomena were observed in other studies. \citeauthor{Thangala2007} reported that a decrease in NW density with increasing substrate temperature, and an increase of NW density with increasing partial pressure of oxygen.\cite{Thangala2007}

% seeded edx 
\begin{figure}[htb]
\centering
\includegraphics[width=0.5\textwidth]{wo3_seed_edx}
\caption[Composition analysis on seeded growth \ce{WO3} NWs]{Composition analysis on seeded growth \ce{WO3} NWs. EDX spectroscopy on seeded growth \ce{WO3} NWs.}
\label{fig:woedx}
\end{figure}
\gls{edx} analysis on the seeded growth \ce{WO3} NWs is shown in Fig.~\ref{fig:woedx}. Only W and O elements were detected on the NW array. The background level from 3 to 8 keV was a manifestation of the continuous components of W X-ray spectrum. 

% sg raman xrd
\begin{figure}[htb]
\centering
\subfloat[]{\label{fig:sgxrd}\includegraphics[width=0.45\textwidth]{xrd_cs_before}}\hspace{0.04\textwidth}
\subfloat[]{\label{fig:sgram}\includegraphics[width=0.45\textwidth]{wox_raman_1}}
\caption[Characterization of seeded growth \ce{WO3}: XRD and Raman]{Characterization of seeded growth \ce{WO3}: XRD and Raman. (a) XRD pattern of as-prepared sample indicating the \ce{WO3} phase and the presence of metallic core. (b) Raman spectrum on NWs region showing the feature of \ce{WO3}.}
\label{fig:woseedxrd}
\end{figure}

Fig.~\ref{fig:sgxrd} is the XRD spectrum of one typical sample. The peaks under circular symbol were identified to be the monoclinic \ce{WO3} phase (ICDD PDF 01-083-0950, \emph{a}=7.30084 \AA, \emph{b}=7.53889 \AA, \emph{c}=7.6896 \AA, $\beta$=90.89$^\circ$), while the peak under the triangular symbol was indexed to cubic tungsten phase (ICDD PDF 04-16-3405, \emph{a}=3.157 \AA), in agreement with the EDX analysis (Fig.~\ref{fig:woedx}). This means that during the \ce{WO3} seeded growth of 4 h heating at 1000 \si{\degreeCelsius}, the tungsten powder in downstream low temperature region (600-700 \si{\degreeCelsius}) is not entirely oxidized. Micro-Raman scattering spectroscopy was performed on the as-synthesized sample as well. During Raman examination, the laser spot was carefully focused onto the NWs on powders and several inspections on different positions were observed to ensure the reproductivity of spectra data. As shown in Fig.~\ref{fig:sgram}, five distinct bands were well resolved, with peaks located at 131, 265, 327, 711 and 803 \si{cm^{-1}}, respectively. This pattern was typical features of \ce{WO3}, consistent with previous study.\cite{Salje1975a,Dixit1986} The high background level probably arises from the metallic core.
% sg tem
\begin{figure}[htb]
\centering
\includegraphics[width=0.9\textwidth]{JAP-2column_Fig5major.jpg}
\caption[Characterization of \ce{WO3}: TEM]{TEM Characterization of \ce{WO3}: (a) TEM image of one nanowire with diameter about 40 nm, and (b) HRTEM images showing the spacing is 0.38 nm, corresponding to (002) plane distance.}
\label{fig:woseedtem1}
\end{figure}

TEM specimen was prepared by using carbon grid to slightly scratch the as-grown sample. Fig.~\ref{fig:woseedtem1} shows the feature of majority NWs. The growth direction is determined to be perpendicular to (002) plane. The streaking in SEAD pattern presumably arises from stacking defaults during \ce{WO3} NW growth. This study also found some NWs exhibit high crystalline quality, as revealed by the TEM analysis in Fig.~\ref{fig:woseedtem2}. The NW grew normal to (002) plane with a measured lattice spacing of 3.79 \AA, which is favorably compared to the XRD peak at $23.07^\circ$ (7.7103 \AA). The sharp SEAD pattern and clear phase contrast in HRTEM demonstrated are both strong evidence of good crystallinity. This formation indicated current growth parameters have promising potential to obtain highly crystalline \ce{WO3} NWs in large scale. 

% sg tem
\begin{figure}[htb]
\centering
\includegraphics[width=0.9\textwidth]{JAP-2column_Fig5minor.jpg}
\caption[Characterization of \ce{WO3}: TEM cont]{TEM Characterization of \ce{WO3}: (a) TEM image of one nanowire, the diameter is about 70 nm, and (b) HRTEM images showing the spacing is 0.379 nm, corresponding to (002) plane distance.}
\label{fig:woseedtem2}
\end{figure}

In regarding to the formation of NWs on tungsten powder itself, this study assumes the driving force is related to interfacial strain between W and \ce{WOx}. Oxidation of tungsten proceed slowly at room temperature and an oxide layer of 100 \si{\angstrom} was found on the surface of tungsten foils.\cite{Warren1996} The tungsten powder used in current study would be covered by a thin oxide layer as well. During oxidation, different oxidation rates exist for different crystallographic orientations on the tungsten powder. Oxidation occurring at boundaries and defects are preferred thermodynamically.\cite{You2010} Compressive strain will gradually accumulate at the tungsten oxide/tungsten interface, which might limit the diffusion rate of oxygen at temperature lower than 500 \si{\degreeCelsius}.\cite{tungsten1999} At elevated temperature, cracks will primarily occur, as observed in Fig.~\ref{fig:pdtemp}. When heated up, tungsten and the oxide shell will probably relax the strain by converting into sub-stoichiometric NWs, a similar process as suggested by \citeauthor{Klinke2005} in the chemically induced strain growth of tungsten oxide NWs.\cite{Klinke2005} It is worth noting that tungsten oxide nanowires could also formed when \ce{WO3} is reduced.\cite{Sarin1975} The elongation of \ce{WO_x} is thermodynamically favorable during the conversion from metallic tungsten to tungsten oxide as well. Local evaporation-condensation process might also contribute to the formation of NWs on tungsten powder.

The enhanced yield of NWs obtained via seeded growth could be explained by a vapor-solid (VS) mechanism. External supply of growth species will condense onto the powders and substrate simultaneously, promoting the elongation of NWs on power as well as resulting island-layer growth on substrate. The local NW density in oxidation experiment was much higher than that of seeded growth. It is reasonable to presume that during seeded growth, several NWs in a small region on powder will coalescence, as evidenced by the bundled structures. At last, the author would like to point out that when low purity tungsten powder (3N) was used as source, sodium tungsten oxide nanowires were found to be dominant in the final product. The details have been published.\cite{Sheng2014} It seems surprising that when 3N powder was used as seeds, only \ce{WO3} nanowires were obtained. This result was attributed to the lower temperature and significantly reduced amount of 3N powder used in the seeded growth, compared with the conditions used in Ref.\cite{Sheng2014}. The source material in seeded growth is not limited to high purity tungsten powder. Instead, any material that could produce appropriate growth vapor could be employed, indicating the versatility of this approach.

\subsection{Tungsten Oxides: Other Methods}

Another interesting approach for \ce{WO3} NW growth were also found in this study, producing \ce{WO3} NWs of high crystalline quality. The method was supposed to be related to the vapor flow dynamic in the deposition region. 
% fg sem
\begin{figure}[htb]
\centering
\subfloat[]{\label{fig:fg1}\includegraphics[width=0.45\textwidth]{wox_flow1}}\hspace{0.04\textwidth}
\subfloat[]{\label{fig:fg2}\includegraphics[width=0.45\textwidth]{wox_flow2}}
\caption[Characterization of flow growth \ce{WO3}: SEM]{Characterization of flow growth \ce{WO3}: SEM. (a) Low magnification SEM image showing dense array of NWs (b) High magnification SEM image of NWs grew out of layer and close-up view on individual wire.}
\label{fig:wogrowsf}
\end{figure}
As shown in Sec.~\ref{sec:nawox}, ordinary transport growth using high purity W power mostly leads to thin film growth on one substrate of 1 inch long. However, it was found that when two shorter substrates were closely placed together, the \ce{WO3} NW yield was significantly enhanced in the second substrate located in lower temperature region. For instance, substrate 1 (Sub1) is located at 6.75 to 7 inch (Fig.~\ref{fig:wogrow} horizontal axis on page~\pageref{fig:wogrow}), and substrate 2 (sub2) is placed in close vicinity of downstream end of sub1. W source moves to 5 inch, and 0.3 sccm \ce{O2} is used to reduce the \ce{WO_x} vapor pressure. Deposition on sub1 is characterized by thin film, similar to those shown in Fig.~\ref{fig:wox4n5}b. However, deposition on sub2 (Fig.~\ref{fig:fg1}) is in sharp contrast to that shown in Fig.~\ref{fig:wox4n5}c, although they are both located in the low temperature region of reaction chamber. The NW density is dramatically increased. The NWs have length about 1 to 2 $\mu$m and diameter about 200 nm. Close-up view (inset of Fig.~\ref{fig:fg2}) shows the NWs have clear facets, implying its high crystalline quality.

The growth mechanism here can be explained by aforementioned VS process as well. Both local flow and temperature play an vital role in the formation of dense NW array. Splitting one substrate into two presumably introduces some fluctuation in the vapor flow. It is well known that in lamellar flow region, there is a velocity boundary above the side wall. In spite of the high crystalline quality, the \ce{WO3} is not long enough and yield is limited by the evaporation of source materials; therefore this approach is not seriously pursued in this study. 

%to be added, NaOH WO3 results. 

\section{Summary}

A systematical investigation of \ce{WO3} nanowire growth using CVD method with tungsten powders as precursor was performed. Four kinds of tungsten powder sources were used, and the sodium impurity effect was throughly studied. \ce{Na5W14O44} NW was first reported from this study; the crystal structure was investigated in details, and HRTEM images were obtained. Besides, the Raman vibrational spectroscopy of \ce{Na5W14O44} was also revealed. The presence of sodium tungsten oxide phase indicates that the CVD growth of \ce{WO3} is extremely sensitive to Na contents. The insight of \ce{WO3} growth kinetics we have gained from this study could help establish a systematic connection between the interaction of transition metal oxide and alkali metal ions, thereby turning this deleterious effect into a beneficial approach of controlled growth. In fact, favorable results have been observed on both \ce{WO3} and \ce{MoO3} 1D nanostructures syntheses using NaOH treated Si substrates. Details will be presented in Chapter 4 and 6. Moreover, two alternative approaches of \ce{WO3} NWs were also attempted, and the seeded growth method proved effective for large scale growth. 




%  % git version control added 012614
\chapter{Vapor-Solid-Solid Growth of MOLYBDENUM OXIDE}

\section{Introduction}

In this work, a new growth mechanism of molybdenum oxide (\ce{MoO3}) 1D structures was discovered, manifest as \ce{MoO3} long nanobelts and micro-towers. The samples were synthesized using a group of alkali metal based catalysts, including \ce{NaOH}, \ce{KI}, and \ce{Na2CO3}. In contrast to the sole axial growth found in the conventional catalyst-assisted process, two different growth modes were observed for the 1D \ce{MoO3} deposition: transverse and axial growth. In the transverse mode, the 1D structures grew parallel to the catalyst-deposition interface with catalyst particles on the side surfaces; whereas in the axial mode, the crystal grew perpendicularly to the catalyst-deposition interface. The growth modes were explained by a modified \gls{vss} mechanism, and the growth kinetics were explored in detail. 

The remaining sections are organized as following: a brief review on previous studies of \ce{MoO3} is given first, followed by the synthesis method in this dissertation. Emphasis will be placed on morphological and crystal structure characterization of 1D \ce{MoO3} nanostructures and the verification of proposed VSS mechanism. This chapter concludes with a summary of VSS growth and preliminary results on morphology controlled \ce{MoO3} nanostructures. 

\subsection{Properties and Applications of Molybdenum Oxide}

Molybdenum oxide (\ce{MoO3}) crystallizes in three phases: orthorhombic $\alpha$-\ce{MoO3}, monoclinic $\beta$-\ce{MoO3}, and the metastable hexagonal h-\ce{MoO3}.\citep{Deb1968,Fibers2007} $\alpha$-\ce{MoO3} phase (hereafter \ce{MoO3}) exhibits anisotropic structure with strong bonding along [001] and [100] direction while van der Waals interaction along [010] direction.\cite{He2003} Due to this unique structure, \ce{MoO3} has been found to have several important properties and a wide range of technological applications, such as electrochromism and photochromism,\cite{Yao1992} lubricants,\cite{Sheehan1996} photocatalysts,\cite{Chen2010} and gas sensor, including \ce{CO},\cite{Comini2005} \ce{NO2},\cite{Taurino2006} \ce{H2},\cite{Sha2009} and ethanol.\cite{Choopun} Moreover, other features arise when nanoscale \ce{MoO3} is specifically prepared, e.g. field emission.\citep{Li2002d,Zhou2003b}  As a layered hosting material, \ce{MoO3} can be further modified by intercalating with alkali ions\citep{Spahr1995,Li2006b,Hu2011} and even divalent ion.\cite{Sian2005} This structural richness has enabled improved performance in Li-ion\cite{Mai2007} and sodium-ion batteries.\cite{Hariharan2013} In combination with \ce{TiO2} forming a core-shell structure, these nanoparitcles were reported to lower the photon absorption energy of \ce{TiO2}.\cite{Elder2000} When \ce{MoO3} was combined with Ag as layered structure, transparent conducting behavior was observed.\cite{Nguyen2012} In addition, \ce{MoO3} has been a good precursor for preparing other useful materials, such as \ce{MoS2} fullerene\cite{Li2003c} and few layer \ce{MoS2}.\cite{Lin2012} Recently, 2D atomic layers attract intensive research efforts, and \ce{MoO3} can be used in fabricating the van der Waals heterostructures.\cite{Geim2013}

The coordination number of Mo in \ce{MoO3} is six; thus \ce{MoO6} octahedra are often considered as the building block. As shown in Fig.~\ref{fig:mo3model}, the layered structure consists of zig-zag rows of edge-sharing \ce{MoO6} octahedra, and the rows are mutually connected by corners.
\begin{figure}[htb]
\centering
\includegraphics[width=0.5\textwidth]{MoO3_model}
\caption[\ce{MoO3} crystal model]{\ce{MoO3} crystal model.}
\label{fig:mo3model}
\end{figure}
Another point of view emerges when one considers the fact that four of the six surrounding O atoms are at distances from 1.67 to 1.95 \AA, while the remaining two are as far as 2.25 and 2.33 \AA, \ce{MoO3} could also be considered as chains of \ce{MoO4} tetrahedra connected by the sharing of two oxygen corners with two neighbouring tetrahedra in $c$ axis. The infinite chains of \ce{MoO4} tetrahedra form half-layers in the $ac$ plane. Two half-layers, which are stapled along $b$ axis, build up one \ce{MoO3} layer.\cite{Itoh2001a} This view stresses that the \ce{MoO6} octahedra are rather distorted.
%A brief summary of crystallography, band structure and dielectric function is listed in Appendix.

It is known that a series of molybdates form when alkali metal ions are incorporated into the \ce{MoO3} lattice, and the structures are summarized in Table~\ref{tab:naxmow}. Due to the weak cation-oxygen bonding, alkali metal cations only introduce small perturbations into the energies of \ce{Mo-O} matrix in comparison to cations of other elements. And no mixing of vibrations of the cationic sublattices with that of \ce{Mo-O} lattices is anticipated. So the structural features of \ce{Mo-O} polyhedra are dominating factors affecting the vibration frequencies and thermodynamic values of the molybdates.\cite{Fomichev1992}

\begin{table}[htb]
\centering
\caption{Crystal structures of alkali metal molybdates and tungstates}\label{tab:naxmow}
\begin{tabular}{llr}
\toprule
formula & structure  &  \\
\midrule
\ce{A2O}:\ce{MO3}\textsuperscript{\emph{a}} & isolated tetrahedral \ce{MO4} anions& \\
\ce{A2O}:\ce{2MO3} & chain-type anions of \ce{MO4} and \ce{MO6} & \\
\ce{A2O}:\ce{3MO3} & chain-type anions of \ce{MO5} and \ce{MO6} & \\
\ce{A2O}:\ce{4MO3} & chain-type anions of \ce{MO6} & \\
\bottomrule

\textsuperscript{\emph{a}} A = Li, Na, K, Rb, Cs; M = Mo, W;
\end{tabular}
\end{table}

Molybdenum bronze also exhibits intriguing features; for instance, \gls{cdw} states\footnote{In \gls{cdw} states, conductivity is non-Ohmic above a threshold electric field.} was found in blue bronze \ce{K_{0.3}MoO3}, which can be prepared by electrolytic reduction of \ce{K2MoO4} and \ce{MoO3} melt.\cite{Dumas1983} \ce{K_{0.3}MoO3} stays as monoclinic phase at room temperature with lattice parameter $a=18.249$ \AA, $b=7.561$ \AA, $c=9.856$ \AA, and $\beta=117.54^{\circ}$ and exhibits a semiconductor-to-metal transition at 180 K. Electrical transport measurements yield an highly anisotropic ratio of DC conductivities of 1:10:1000. Along this high conducting axis, metallic reflection behavior is confirmed by optical measurements. Therefore, this blue bronze $\ce{K_{0.3}MoO3}$ is also known as quasi-1D metal.\cite{Sing1999}

\subsection{Review of Synthesis Methods of Molybdenum Oxide}

Researchers have been exploring a variety of methods to synthesize different \ce{MoO3} nanostructures. Only a brief introduction will be given here, since these efforts has been well documented in several review articles.\cite{He2003} Most of these methods can be categorized into two groups: solution-based hydrothermal procedures \citep{Li2002b,Xia2006,Li2006a,Camacho-Bragado2006} and chemical vapor deposition approaches.\citep{Zeng1998,Li2002c,Li2002d,Zhou2003b,Fibers2007,Yan2009} Both methods have their own merits. Hydrothermal process usually needs lower temperature(\textless 300 \si{\degreeCelsius}), but the reaction often requires tens of hours, and the sample requires several post-growth processing steps; whereas vapor deposition demands relatively high temperature (\textgreater 500 \si{\degreeCelsius}) with a few hours reaction time. Both methods are scalable for industrial applications. However, the hydrothermal treatment seems to allow more  nanostructures control than does the vapor deposition method. Nanobelts,\cite{Li2002b} helical nanosheets, nanoflowers, nanorods,\cite{Li2006a} and nanoribbons\cite{Camacho-Bragado2006} were obtained by the former one while nanoflakes,\cite{Chen2009} nanobelts,\cite{Hu2009} and nanowires\citep{Zhou2003b,Chen2011b} dominated the product morphology for the latter one. The assistance of Au catalyst in vapor deposition only altered the orientation\cite{Yan2009} or served as preferred nucleation sites\cite{Cai2011} without producing new \ce{MoO3} structures. \citeauthor{Chithambararaj2013} prepared hexagonal \ce{MoO3} nanocrystal via hydrothermal method and demonstrated the photodegradation of methylene blue (MB) under visible light.\cite{Chithambararaj2013} The efficiency dependence on catalysis/dye ratio, light intensity, and temperature was studied. h-\ce{MoO3} was mostly synthesized using solution methods, where \ce{NH4+} and \ce{OH-} were possible structure directing and stable agents. The band gap of h-\ce{MoO3} estimated from diffusion reflection spectrum is $2.8\sim3.0$ eV. Hexagonal phase of \ce{MoO3} is readily identified by the XRD pattern (strong peak at $2\theta=20^{\circ}$). It is worth noting that in spite of these numerous growth of \ce{MoO3}, reports on catalytic growth are still scarce. 

\section{Experimental}\label{sec:grow}

The reactants used in this study were listed in Table~\ref{tb:mosource}. All reactants were used as received without further processing.
\begin{table}[htb]
\centering
\caption{Reactants list of \ce{MoO3} growth}\label{tb:mosource}
\begin{tabular}{lcccr}
\toprule
Material & Stock No & LOT &Purity & Vendor\\
\midrule
\ce{NaOH}     & S318-500 & 070241 & 99.8\% & Fisher Scientific \\
\ce{NaI}      & 11665 & K11W054 & 99.9\% &  Alfa Aesar \\
\ce{KI}        & 42857 & H06Z051 & 99.9\% &  Alfa Aesar \\
\ce{Na2CO3}    & 33377 & 114X012 & 99.95\% &  Alfa Aesar \\
\ce{Molybdenum}& 00932 & I07S024 & 99.9\% &  Alfa Aesar\\
\bottomrule
\end{tabular}
\end{table}

Silicon substrate was first cleaned according to the procedures stated in Sec.~\ref{ch2sub} on page~\pageref{ch2sub} and directly used in non-catalyst-assisted experiment. In catalyst-assisted growth, hydrophilic Si substrates were first drop-cast with \ce{NaOH} (or \ce{KI}, \ce{Na2CO3}) solution (54 $\mu$L 10 mM) and then air dried in a chemical fume hood before experiment. Other alkali ions-containing substrates, such as glass (Fisher Scientific, microscope slide, 12-549) and indium tin oxide (ITO) coated glass (Delta Technologies, 25\si{\ohm}), were cleaned by the same routine except the absence of plasma cleaning and subsequent aqueous solution dipping. Mica was cleaved right before growth without other treatment. 
\begin{table}[htb]
\centering
\caption{Physical constants of reactants }\label{tb:mothermo}
\begin{tabular}{lccr}
\toprule
Material & MP(\si{\degreeCelsius}) & BP(\si{\degreeCelsius}) & Reference\\
\midrule
\ce{NaOH}        & 318 & 1388 & \cite{crc1977}  \\
\ce{NaI}         & 651 & 1300 & \cite{crc1977}    \\
\ce{KI}          & 681 & 1330 & \cite{crc1977}   \\
\ce{Na2CO3}      & 851 & Not determined & \cite{crc1977}    \\
\ce{Na2MoO4}     & 687 & Not available & \cite{crc1977}   \\
\ce{MoO3}    & 795 & 1155 & \cite{crc1977}   \\
\ce{MoO2}    & 1100(decomp) & Not available & MSDS   \\
\bottomrule
\end{tabular}
\end{table}
In a typical catalytic synthesis (Fig.~\ref{fig:mooxgrowth}), about 2 g molybdenum powders were loaded into the uniform heating zone, and the sealed chamber was first pumped down to 10 mTorr. 
% cvd layout
\begin{figure}[htb]
\centering
\includegraphics[width=0.6\textwidth]{CVD_and_temp_MoO3.jpg}
\caption[CVD growth setup of \ce{MoO3}]{CVD growth setup of \ce{MoO3}. The triangular labels $\blacktriangledown$ were measured points at ambient environment.}
\label{fig:mooxgrowth}
\end{figure}
Oxygen flow, varying between 0.1 sccm to 10 sccm, was then admitted from the upstream inlet. With 10 sccm UHP Ar as carrier gas, the overall pressure reached about 200 mTorr. The heating temperature was ramped up to 800 \si{\degreeCelsius} in 30 min and lasted for 15 to 120 min. At last, the heating power was turned off, and the chamber was allowed to naturally cool to room temperature. The substrate was placed in downstream location where the temperature was about 350 to 650 \si{\degreeCelsius} according to open air temperature profile. Non-catalytic growth shared the same procedure except the NaOH treatment on Si substrates. 

\section{Results and Discussion on Molybdenum Oxide}\label{sec:result}
In this section, the growth of \ce{MoO3} on bare Si substrates was presented first, followed by catalytic growth, both of which were characterized by SEM, XRD, TEM and EDX. Then the catalytic growth mechanism was investigated using \ce{NaOH} treated Si substrate. The growth mechanism was probed using two series of control experiments. Then, a VSS growth model was proposed and verified. In the end, this growth model was applied onto other substrates with alkali metal ions, and various \ce{MoO3} nanostructures were produced.

\subsection{Non-catalytic Growth}\label{sec:nonsi}

For \ce{MoO3} non-catalytic growth, this study observed micro-flakes morphology, as shown in Fig.~\ref{fig:mosisem}. Most flakes exhibited rectangular shape with average thickness of one micrometer and standard deviation $\sigma_D$ of $0.34$ $\mu$m.
% sem in (31210-31510) batches
\begin{figure}[htb]
\centering
\subfloat[]{\label{fig:mosem1}\includegraphics[width=0.4\textwidth]{mosemsi_a}}\hspace{0.04\textwidth}
\subfloat[]{\label{fig:mosem2}\includegraphics[width=0.4\textwidth]{mosemsi_b}}
\caption[Representative morphologies of \ce{MoO3} on Si]{Representative morphologies of \ce{MoO3} on Si: (a) low magnification and (b) high magnification SEM images of representative depositions of \ce{MoO3} on Si for non-catalytic growth.}
\label{fig:mosisem}
\end{figure}

As shown in Fig.~\ref{fig:mooxch}, the crystal structure and phase of the as-synthesized specimen were examined by XRD and Raman. The XRD pattern (Fig.~\ref{fig:moxrd}) was readily indexed to the orthorhombic phase of \ce{MoO3} (ICDD PDF 05-0508, \emph{a}=3.9628 \AA, \emph{b}=13.855 \AA, and \emph{c}=3.6964 \AA). The space group is $D_{2h}^{16}(Pbnm)$. This crystal structure is indeed a unique example among transition metal oxides, representing a transitional stage between tetrahedra and octahedral coordination.\cite{Itoh2001a} The strongest peak index was (110), suggesting the orientation of \ce{MoO3} was not parallel to the substrate, in consistent with the morphology displayed in Fig.~\ref{fig:mosem1}. In contrast, different dominating orientation was observed in the XRD pattern from catalytic growth (Sec.~\ref{sec:nasi}).
% xrd raman
% 081113 sample
\begin{figure}[htb]
\centering
\subfloat[]{\label{fig:moxrd}\includegraphics[width=0.45\textwidth]{xrd_moo3}}\hspace{0.04\textwidth}
\subfloat[]{\label{fig:moram}\includegraphics[width=0.45\textwidth]{raman_moo3}}
\caption[Crystalline phase characterization of \ce{MoO3} on Si]{Crystalline phase characterization of \ce{MoO3} on Si: (a) XRD pattern and (b) Raman spectrum of typical \ce{MoO3} on Si, $\lambda_{ex} = 532$ nm.}
\label{fig:mooxch}
\end{figure}
The Raman spectrum of the as-synthesized specimen also closely matched \ce{MoO3} features observed in previous studies.\cite{Dixit1986,Silveira2012} During m-Raman measurement, the laser spot was carefully focused onto the \ce{MoO3} flakes, and several inspections on different positions were observed to ensure the reproductivity of spectra data. As shown in Fig.~\ref{fig:moram}, 14 distinct bands were well resolved. The 284 \si{cm^{-1}} peak represented the wagging mode for double bond \ce{O=Mo=O}. The 337 and 380  \si{cm^{-1}} peaks were assigned to \ce{O-Mo-O} bending and scissoring modes, respectively. The 199 \si{cm^{-1}} peak and two other weaker peaks at 218 and 247 \si{cm^{-1}} represented \ce{O=Mo=O} $B_{2g}$ twist, $A_g$ chain mode, and \ce{O=Mo=O} $B_{3g}$ twist mode, respectively. The 667 \si{cm^{-1}} peak was assigned to triply coordinated oxygen stretching model resulting from edge-shared oxygen in common to three octahedral. The 819 \si{cm^{-1}} peak arose from doubly coordinated oxygen stretching mode between two octahedral. The 996 \si{cm^{-1}} peak was from unshared oxygen stretching mode.\cite{Siciliano2009} The Raman symmetry assignment is summarized in Table~\ref{tab:moram}.
% raman assign
\begin{table}[htb]
\centering
\caption[Experimental Raman peaks assignment of \ce{MoO3} on Si]{Experimental Raman peaks assignment of \ce{MoO3} on Si.\cite{Eda1992,Siciliano2009}}\label{tab:moram}
\begin{tabular}{llcll}
\toprule
this work(\si{cm^{-1}}) & Sym.  &    & Assignment &   \\
\midrule
117      & $B_{2g}$    &           & $T_c$  & RCM\textsuperscript{\emph{a}}  \\
129      & $B_{3g}$    &           & $T_c$  & RCM  \\
158      & $A_g/B_{1g}$&           & $T_b$  & RCM  \\
199      & $B_{2g}$    & $\tau$    & \ce{O=Mo=O}  & twist  \\
218      & $A_g$       &           & $R_c$     & RCM  \\
247      & $B_{3g}$    & $\tau$    & \ce{O=Mo=O}  & twist  \\
284      & $B_{2g}$    &           & \ce{O=Mo=O}  & wag  \\
292      & $B_{3g}$     & $\delta$ & \ce{O=Mo=O}  & wag  \\
337      & $A_g,B_{1g}$ & $\delta$ & \ce{O-Mo-O} & bend  \\
380      & $B_{1g}$     & $\delta$ & \ce{O-Mo-O}  & scissor  \\
474      & $A_g$        & $\nu_{as}$ & \ce{O-Mo-O}  & stretch,bend  \\
667      & $B_{2g},B_{3g}$ & $\nu_{as}$  & \ce{O-Mo-O}  & stretch  \\
819      & $A_g$        & $\nu_{as}$  & \ce{O=Mo}  & stretch  \\
996      & $A_g$         & $\nu_{as}$  & \ce{O=Mo}  & stretch  \\
\bottomrule
\end{tabular}

\textsuperscript{\emph{a}} R chain mode;
\end{table}

\subsection{Catalyst-assisted Growth}\label{sec:naohsi}

All samples in Sec.~\ref{sec:naohsi} were grown for 2 h, the same as that in Sec.~\ref{sec:nonsi}. The only difference in growth conditions from Sec.~\ref{sec:nonsi} was the Si substrate treatment. When NaOH was applied onto Si, the deposition morphologies were changed significantly. Fig.~\ref{fig:ch4sem2by3} show a comparison of the morphology difference between the non-catalyst growth (Fig.~\ref{fig:ch4sem2by3}a) and the catalyst-assisted growth (Fig.~\ref{fig:ch4sem2by3}b-f) revealed by SEM imaging. The non-catalyst growth morphology has been discussed from Fig.~\ref{fig:mosisem} on page~\pageref{fig:mosisem}. Several different structures emerged when catalyst was applied, such as nanobelts in Fig.~\ref{fig:ch4sem2by3}d with length up to hundreds of microns and microtowers in Fig.~\ref{fig:ch4sem2by3}f with diameter around 10 $\mu$m. A detailed description of these \ce{MoO3} nanostructures has been published and can be found in Ref.\cite{Sheng2015}.

% sem
\begin{figure}[htb]
\centering
\includegraphics[width=0.9\textwidth]{MoO3_SEM_a.jpg}
\caption[Effect of alkali metal based catalysts on \ce{MoO3} depositions]{Effect of alkali metal based catalysts on the morphology of \ce{MoO3} depositions: (a) SEM image of rectangular nanoplates grown without catalysts. (b-f) SEM images of different morphologies grown with NaOH catalysts: (b) nanobelts grown on top of dense array of nanoplates,(c) close-up view of the forked nanoplates from an area indicated by the square in Fig.~\ref{fig:ch4sem2by3}, (d) long nanobelts, (e) side-view of ultra-long microbelts, and (f) side-view of microtowers. Insets show detailed features of different \ce{MoO3} structures.}
\label{fig:ch4sem2by3}
\end{figure}

The crystal structures of the as-synthesized specimens were characterized with \gls{xrd} for the non-catalyst growth (Fig.~\ref{fig:ch4tem3by3}a) and catalyst-assisted growth (Fig.~\ref{fig:ch4tem3by3}b). The calculated lattice parameters and standard database value (ICDD PDF 05-0508) matched favorably, as listed in Table~\ref{tab:ch4xrd}. No other phases were found from XRD spectra for both specimens. This result indicated that if there were any catalyst phases present in the catalyst-assisted specimen, they must be of extremely low amount as compared to that of the \ce{MoO3} phase, presumably below the detection limit. 
% xrd 
\begin{table}[htb]
\centering
\caption{Lattice parameters of \ce{MoO3} derived from XRD}\label{tab:ch4xrd}
\begin{tabular}{lccc}
\toprule
           & $a$ (\AA) & $b$ (\AA) & $c$ (\AA)   \\
\midrule
Non-catalytic growth  & 3.963    & 13.882 & 3.701  \\
Catalytic growth  & 3.967    & 13.865 & 3.701  \\
PDF 05-0508       & 3.9628    & 13.855 & 3.6964  \\
\bottomrule
\end{tabular}
\end{table}
The XRD pattern of catalyst-assisted growth shown in Fig.~\ref{fig:ch4tem3by3}b was dominated by the (0$k$0) family peaks, showing a different preferred orientation from the non-catalytic growth. The reason will be discussed in Sec.~\ref{sec:nasi}. 
% tem
\begin{figure}[htb]
\centering
\includegraphics[width=0.8\textwidth]{MoO3_TEM_XRD.jpg}
\caption[XRD and TEM on \ce{MoO3} growths with and without catalyst]{XRD and TEM on \ce{MoO3} growths with and without catalyst. XRD spectra of (a) a specimen grown without catalysts and (b) a specimen grown with NaOH catalysts. (c) Low-magnification TEM image and (d) HRTEM image of nanoplate grown without catalyst. Low-magnification TEM images of (e) forked nanoplate, (f) nanobelt, and (h) microtower, and (g) HRTEM image of nanobelt. Insets show SAED patterns of different \ce{MoO3} 1D structures}
\label{fig:ch4tem3by3}
\end{figure}
Fig.~\ref{fig:ch4tem3by3}c show low magnification TEM image of a rectangular nanoplate grown without catalysts and the SAED pattern. The nanoplate was exfoliated by ultra-sonication in acetone solution to obtain high-resolution TEM (HRTEM) image; hence, the nanoplate was broken into fragments. The SAED pattern in [010] axis and the HRTEM image (Fig.~\ref{fig:ch4tem3by3}d) show the nanoplate had a (010) top surface and two orthogonal edges along [100] and [001], respectively. For the catalyst-assisted growth, Fig.~\ref{fig:ch4tem3by3}e and Fig.~\ref{fig:ch4tem3by3}f-g confirmed both the forked nanoplate and the nanobelt had a (010) top surface and a growth direction of [001]; while Fig.~\ref{fig:ch4tem3by3}h revealed the microtower grew with (010) layers stacking along a [010] direction. Lattice measurements in Fig.~\ref{fig:ch4tem3by3}d and g indicated a planar distance of 0.38 nm for (100) planes and 0.36 nm for (001) planes. These values were slightly smaller than the standard data for $\alpha$-\ce{MoO3} and the results from XRD measurements. This deviation in lattice parameters may arise from the error in TEM measurements and/or the lattice distortion of \ce{MoO3} during the electron beam irradiation.\cite{Wang2004b} These results indicated that all the as-synthesized 1D structures had a layered structure. Raman measurement on the catalytic growth sample revealed similar \ce{MoO3} bands, as already displayed in Fig.~\ref{fig:mooxch} on page~\pageref{fig:mooxch}. 

\subsection{Growth Mechanism Study}\label{sec:nasi}

The growth process of the 1D \ce{MoO3} nanoflakes in non-catalytic experiment can be explained by the \gls{vs} mechanism, as proposed in other vapor phase depositions of \ce{MoO3} nanostructures.\cite{Brezesinski2010, Kalantar-zadeh2010} It has been shown that \ce{MoO3} growth on the NaOH treated Si substrates exhibited dramatic differences. Previous study in our group had found that \ce{MoO3} deposition on ITO glass exhibited new morphologies that were not observed on Si substrates under the same growth conditions, yet the mechanism was not well investigated.\cite{predeep2011} A list of composition difference between glass and Si substrate show the additional elements of Na, K, and Ca, \emph{etc} in glass. Since the concentration of Na is highest among these extra elements, it is most likely to induce the new morphology growth. Therefore, a \gls{vls} mechanism may exist for the \ce{MoO3} growth when alkali metal ions are involved. To provide a proper context, \gls{vls} process is briefly introduced here. 

\gls{vls} was first proposed by Wagner\cite{Wagner1964} and further developed by \citeauthor{Givargizov1975}.\cite{Givargizov1975} This method has become an important strategy for synthesizing 1D nanostructures.\cite{Lieber1998} Conventionally, a liquid eutectic droplet is formed by catalyst itself or by alloying with the growth material, acting as a trap of growth species. The growth is initialized by supersaturation of the liquid alloy, and subsequently precipitation occurs at the solid-liquid interface. The choice of catalyst usually is among the several noble metals, since they are physically active but chemically stable or inert in most growth scenarios. The growth materials span a wide range, including group IV,\cite{Hochbaum2005} group III-V,\citep{Dalacu2013, Xiao2012, Dubrovskii2011} group II-VI,\cite{Hao2006} and some metal oxides, such as \ce{ZnO},\citep{Huang2001a,Ramgir2010} \ce{MgO},\citep{HEUER1967, Nagashima2007} \ce{SiO2},\cite{Pan2002} and \ce{TiO2},\cite{Zhuge2012} to name a few. The catalysts can be Au, Pd, Pt, Ni,\cite{Xiao2012} Ti, Ga,\cite{Pan2002} and even KI. Understanding the interaction between liquid droplet and the solid interface allows for a rich engineering space to fine tune the geometry and structures of the as-grown nanostructures. For instances, the diameter of Si nanowires can be controlled by laser ablated catalyst\cite{Morales1998} or well-defined Au nanoparticles.\cite{Cui2001b} Under proper conditions, axial modulation can produce nanowire superlattices from group IV and III-V materials.\citep{Gudiksen2002,Bjork2002} Radial composition modulation has also been demonstrated by selectively suppressing VLS process, providing a robust routine for homogeneous or heterogeneous core-shell structures.\cite{Lauhon2002a}  

Back to the \ce{MoO3} and alkali metal ions, a literature search found that there exists a \ce{Na2MoO4}-\ce{MoO3} phase diagram, as shown in Fig.~\ref{fig:ch4pd}.\cite{Hoermann1929} Note that the phases of the Na-Mo-O system can be written as \ce{Na2O.nMoO3}, where n = 1, 2, 3, and 4. 
%phase diagram
\begin{figure}[htb]
\centering
\includegraphics[width=0.5\textwidth]{moo3_pdsmall}
\caption[Phase diagram of \ce{Na2MoO4} and \ce{MoO3}]{Phase diagram of \ce{Na2MoO4}-\ce{MoO3} reproduced from Ref.~\cite{Hoermann1929}}
\label{fig:ch4pd}
\end{figure}
Based on this phase diagram, the physical constants of reactants in Table~\ref{tb:mothermo}, and the growth conditions adopted in this study, it was predicted that the overall reactions occurred in the following sequences. First, Mo powder was oxidized, and \ce{MoO3} vapor was produced and transported to downstream by Ar carrier gas. Then, NaOH reacted with the incoming \ce{MoO3} vapor according to Formula~\ref{eq:pd}. 
\begin{subequations}\label{eq:pd}
\begin{align}
\cee{NaOH(l) + MoO3(g) &\rightarrow Na2MoO4 + H2O(g)\\
Na2MoO4 + nMoO3   &\rightarrow Na2O.(n + 1)MoO3}
\end{align}
\end{subequations}
The melting point of \ce{Na2MoO4} is 687 \si{\degreeCelsius}. Based on the open air temperature profile (Fig.~\ref{fig:mooxgrowth}), the substrate was located in temperature zone between 650 and 350 \si{\degreeCelsius}. The actual temperature distribution during growth was different from ambient measurement due to the pressure change and thermal conduction along the Si substrate. It was difficult to acquire the accurate temperatures along the substrate; however, by considering the flow rate, the actual temperature should be higher than that predicted by open air measurement.\cite{Subannajui2010} The continuous supply of \ce{MoO3} growth species pushed the phase evolution towards high \ce{MoO3} molar ratio end, with $n$ in \ce{Na2O.nMoO3} increasing from 1 to 4. Each site of \ce{Na2MoO4} continued absorb incoming \ce{MoO3}, forming a swelling particle. \ce{MoO3} precipitated from these seeds when a critical supersaturation concentration was reached, which was about 75\% in unit of mole according to Fig.~\ref{fig:ch4pd}. This ratio considerably exceeded that of the usual metal catalyst scenario, such as the Au-Si system. This high solubility accounted for several observations in current experiments: the dimensions of individual deposit was much larger than that of initial particle and the concentration of sodium in final products was extremely low. Based on this aforementioned analysis, the author proposed following hypothesis:

\emph{The reaction on NaOH-Si substrate follows the phase diagram, and \ce{Na2Mo4O13} serves as the catalyst promoting various 1D \ce{MoO3} growth.}

In a typical 2 h growth, the \ce{MoO3} weight increase was about 20 mg, which was $1.4\times 10^{-4}$ mol. This amount far exceeded the applied NaOH of $5\times 10^{-7}$ mol used in the procedure of Sec.~\ref{sec:grow} on page~\pageref{sec:grow}. To capture the early stage growth, the supply of \ce{MoO3} must be reduced in a controlled way. This can be accomplished by tuning two factors: oxygen flow rate and growth time. In this thesis, two series of controlled growth were used to probe the evolution of \ce{MoO3} deposition on NaOH-Si substrate. The first series regulated \ce{O2} flow at 0.1, 1, 3, and 10 sccm with fixed growth time of 15 min. The second series controlled the growth time at 15, 30, 60, and 90 min with fixed \ce{O2} flow of 10 sccm. Two series shared one common growth of 10 sccm \ce{O2} for 15 min. 

The oxygen regulated growths were characterized using SEM, EDX, and Raman, as shown in Fig.~\ref{fig:ch4oxy}. When \ce{O2} flow was $\leq 1$ sccm, the deposition show micron size plate growth (Fig.~\ref{fig:ch4oxy}a and d). As oxygen flow became higher, the deposition amount increased dramatically, with a morphology of 1D forked plate decorated by triangular shape particles.
% sem
\begin{figure}[htb]
\centering
\includegraphics[width=0.9\textwidth]{MoO3_EDX_Raman}
\caption[Evolution of the early stage for catalyst-assisted \ce{MoO3} growth]{Evolution of the early stage for catalyst-assisted \ce{MoO3} growth: SEM images, micro-Raman spectra, and EDX spectra of the specimens grown for 15 min with different \ce{O2} flows of (a-c) 0.1 sccm, (d-f) 1 sccm, (g-i) 3 sccm, and (j-l) 10 sccm. }
\label{fig:ch4oxy}
\end{figure}
Micro-Raman scattering was used to probe the phase change during the oxygen regulated growth. The Raman spectra of sodium molybdates was featured with multiple strong peaks located between 800 and 1000 \si{cm^{-1}}, which was referred as high-frequency region hereafter. \ce{MoO3} phase show two peaks in this region with the strongest one at about 820 \si{cm^{-1}}; \ce{Na2Mo2O7} phase exhibited four peaks in this region with the strongest one moving to 937 \si{cm^{-1}}, whereas \ce{Na2Mo4O13} phase show even more peaks in the high-frequency region with a doublet at 962 and 970 \si{cm^{-1}}, respectively.\cite{Schofield2005,Saraiva2011} As predicted by the phase diagram (Fig.~\ref{fig:ch4pd}), Raman spectra identified \ce{Na2Mo2O7} phase in the 0.1 sccm growth (Fig.~\ref{fig:ch4oxy}b), and \ce{Na2Mo4O13} phase on the particles in the other three growths (Fig.~\ref{fig:ch4oxy}e, h, and k). This observation supported the previous hypothesis about the reaction sequences on NaOH-Si substrate (Formula~\ref{eq:pd}). Element analysis performed on the Raman sites also confirmed the presence of Na, Mo, and O in particles growth and only Mo and O in the plate growth (Fig.~\ref{fig:ch4oxy}c, f, i, and l). It is worth noting that the absence of Na signal in the forked \ce{MoO3} plate indicated the doping level would be lower than the detection limit of 0.01 wt.\% if the Na doping ever occurred. It should also be pointed out the existence of \ce{Na2Mo3O10} phase is in doubt so far, and no Raman fingerprint is reported.\cite{Fomichev1992} Table~\ref{tb:moo3ram} compared the Raman spectra of sodium molybdates and \ce{MoO3} between present work and literature reports.
% moo3ram
\begin{landscape}
\begin{table}[htb]
\centering
\caption[Raman shift of sodium molybdates and molybdenum oxide]{Comparison of Raman shift (\si{cm^{-1}}) of sodium molybdates and molybdenum oxide}\label{tb:moo3ram}
\begin{tabular}{lllllllll}
\toprule
        & \ce{Na2Mo2O7}    &    &    & \ce{Na2Mo2O7}    &       &       & \ce{MoO3}   &    \\
\midrule
\cite{Saraiva2011} & \cite{Fomichev1992} & this work& \cite{Fomichev1992} & \cite{Schofield2005} & this work   & \cite{Eda1992} & \cite{Siciliano2009}   &  this work   \\
\midrule
 939    & 939/935 & 937         &        & 995    & 995         & 995    & 996       & 996   \\
 921    & 920/918 & 919         & 970    & 971    & 970         & 819    & 819       & 820   \\
 873    & 875/869 & 873         & 960    & 963    & 962         & 666    & 666       & 667   \\
 833    & 836/828 & 833         & 940    &        &             & 471    & 473       &       \\
 821    & 818/815 &             & 925    &        & 926         & 378    & 378       & 380   \\
 768    & 774/765 &             & 910    & 918    & 915         & 366    & 365       & 364   \\
 741    & 741/739 &             & 895    & 899    & 898         & 338    & 337       & 338   \\
 524    & 526/525 &             & 870    &        &             & 291    & 291       & 291   \\
 467    & 464/460 &             & 835    & 841    & 841         & 283    & 283       & 284   \\
 413    & 411/411 &             &        & 820    & 819         & 246    & 244       & 245   \\
 369    & 369/368 & 366         & 745    &        &             & 217    & 218       & 216   \\
 339    & 340/336 & 340         & 650    & 666    &             & 197    & 197       & 199   \\
 295    & 296/294 &             &        & 620    & 619         & 159    & 158       & 156   \\
 249    & 245/245 &             & 438    &        &             & 129    & 129       &       \\
 226    & 228/227 & 226         &        &        & 417         & 117    & 117       &       \\
        & 215/215 &             & 405    &        &             & 100    & 98        &       \\
 200    & 199/199 & 198         &        & 396    & 393         & 84     & 84        &       \\
 176    &         &             & 382    & 381    &             &        &           &       \\
 137    & 140/138 &             &        & 368    &             &        &           &       \\
 120    & 120/120 &             & 348    &        &             &        &           &       \\
 84     & 84/84   &             & 335    & 339    & 336         &        &           &       \\
        & 45/44   &             & 320    & 320    & 318         &        &           &       \\
 36     &         &             & 295    & 293    &             &        &           &       \\
        &         &             & 272    &        & 267         &        &           &       \\
        &         &             & 255    &        &             &        &           &       \\
        &         &             &        & 246    &             &        &           &       \\
        &         &             & 220    & 225    & 224         &        &           &       \\
        &         &             &        & 202    &             &        &           &       \\
        &         &             &        & 159    & 155         &        &           &       \\
\bottomrule
\end{tabular}
\end{table}
\end{landscape}
The morphologies of second series experiment at constant \ce{O2} flow and varied growth time were shown in Fig.~\ref{fig:ch4ev}. The 15 min and 30 min growth show that dense arrays of \ce{MoO3} forked nanoplates with a length of hundreds of micrometers covering the substrate areas at different temperatures. With the increase of the growth time, these nanoplates kept growing and became larger, longer, and denser. Long 1D \ce{MoO3} structures started to grow on top of the nanoplate arrays. For 60 min growth, long nanobelts appeared on top of the nanoplates. For the growth time of 90 min, the nanobelts grew longer, and some grew into microbelts. Microtowers started to grow as well.

\begin{figure}[htb]
\centering
\includegraphics[width=0.9\textwidth]{mooe_sem_ev}
\caption[Evolution of the second stage on \ce{MoO3} growth]{Growth evolution of the second stage of catalyst-assisted growth of \ce{MoO3} 1D structures: SEM images of the deposition at different growth temperatures with growth time of row 1: 15 min, row 2: 30 min, row 3: 60 min, and row 4: 90 min.}
\label{fig:ch4ev}
\end{figure}
To reveal the growth mechanisms for the long 1D structures in the second stage, specimens were carefully examined to locate the catalyst particles. Small catalyst particles ranging from several hundred nanometers to several micrometers were found at different locations of these long 1D structures. As shown in Fig.~\ref{fig:ch4edx1} and Fig.~\ref{fig:ch4edx2}, EDX element analysis confirmed the presence of Na, Mo, and O in these particles, suggesting the particles were sodium molybdate catalysts.  

\begin{figure}[htb]
\centering
\includegraphics[width=0.5\textwidth]{MoO3_SFig_EDX}
\caption[Identifying the catalyst particles on the nanobelt structures]{SEM images and EDS spectra identifying the location, morphology, and composition of the catalyst particles on the nanobelt structures: (a-b) a catalyst particle (I) in the middle of a nanobelt (II), (c-d) two catalyst particles (I and II) leading the growth of two perpendicular nanobelts.}
\label{fig:ch4edx1}
\end{figure}

\begin{figure}[htb]
\centering
\includegraphics[width=0.5\textwidth]{MoO3_SFig_EDX_tower}
\caption[Identifying the catalysts on the microtower structures]{SEM images and EDS spectra for identifying the location, morphology, and composition of the catalysts on the microtower structures: (a-b) a tiny catalyst particle (I) on top of a small cone shaped microtower (II), and (c-d) a catalyst particle (III) on a large and short microtower (IV).}
\label{fig:ch4edx2}
\end{figure}
It could be of difficulty to observe the sodium molybdate particles on the as-synthesized belts and towers due to the following two factors, one is the consumption of catalysts, such as evaporation; the other is the unpredictable location of the catalyst. Unlike the conventional \gls{vls} growth where the catalysts were predominantly found at the tip of 1D nanostructures, the catalyst in this study were mostly found at the side wall of long nanobelts, and the location could be on the tip or several microns away from the tip. As for the microtowers, it was even rare to observe Na element. Some observations under similar controlled growth conditions was also listed as following to support the proposed growth model: 
\begin{enumerate}
\item When no Mo powder was used, NaOH still held onto substrate with some locations exhibiting etching.
\item 1 min growth show plates in high temperature end and irregular shapes in low temperature end.
\item 15 min growth show \ce{MoO3} plates parallel to substrate, with some belts emerging and growing out of the substrate plane. This was presumably due to the growth direction mismatch between different plates. Morphology will also depend on local NaOH concentration. NaOH trace mark turned into connected plates at the original locations, showing the high concentration mediated the morphology evolution. The possible catalyst solid was sitting on the top of plate layer. Some belt-like growth was visible on reaction chamber walls, suggesting the evaporation and transport of \ce{Na_xMoO3} vapor phase;
\item Using NaOH treated hydrophobic substrates, dense tower array growth was found along the edge areas or molten periphery of particles, where 5 min growth already show the initial stage of tower with diameter about 1 $\mu$ m, which increased to 10 $\mu$ m with prolonged growth.
\end{enumerate}

The growth of \ce{MoO3} on NaOH-Si can be divided into two stages delimited by the emerging of long nanobelts and microtowers. A growth model was proposed to account for the observed morphologies of \ce{MoO3} long nanobelt and tower-like structures.
 
\emph{
A transverse growth mode exists with growth direction perpendicular to the catalyst-deposition interface, promoting \ce{MoO3} nanoplate and nanobelt growth. A axial growth mode akin to conventional VLS process exists, and is responsible for \ce{MoO3} tower-like growth.
}

The first stage was dominated by transverse growth. The atomic steps in proposed growth model were depicted in the upper panel of Fig.~\ref{fig:ch4vls}. It has been verified that \ce{Na2Mo4O13} formed during the early stage of growth and acted as catalysts. Although the exact phase (liquid or solid) of \ce{Na2Mo4O13} was not known yet in current study, it can be deduced that the \ce{Na2Mo4O13} existed as an active site for the adsorption of incoming \ce{MoO3} growth species. When the catalyst was smaller than some critical size, there is no preferred growth direction. This changed when the precipitated solid exceeded certain critical dimension, and the morphology evolution was then determined by the relative size of already formed solid and the associated catalyst. Several pathways could occur after the accommodation of \ce{MoO3} vapor and supersaturation. \ce{MoO3} could precipitate at catalyst-deposition interface, or propagate along catalyst-environment boundary and spread on the already formed \ce{MoO3} surface. Meanwhile, the \gls{vs} process is not suppressed and \ce{MoO3} adatoms could directly condense on \ce{MoO3} surface.

\begin{figure}[htb]
\centering
\includegraphics[width=0.7\textwidth]{vls_mode2}
\caption[Schematic drawing on the growth steps of the proposed growth mode]{Schematic drawing on the growth steps of the proposed growth mode.}
\label{fig:ch4vls}
\end{figure}

Some unique features of \ce{MoO3} must be taken into consideration to properly evaluate the dominate pathway(s). It has been reported that \ce{MoO3} can spontaneously spread over the surface of supports (\emph{e.g.} \ce{Al2O3}, \ce{SiO2}, and Au) to form a monolayer or submonolayer at a temperature (257 \si{\degreeCelsius}) well below the melting point.\cite{Leyrer1990} This strong spreading behavior can be explained by the solid-solid wetting process, in which the driving force was the decrease in total surface free energy.\cite{Leyrer1988} Although the detailed transport mechanisms of the spreading are still in debate, it is suggested that the high mobility of the Mo oxide species on substrates and \ce{MoO3} islands can promote the spreading and the ambient gases could further enhance the spreading.\cite{Gunther2000, Song2003} Based on above analysis and the final morphology of \ce{MoO3}, the dominance of accommodation and spreading pathway in higher temperature part of the growth sample was most possible in this study. Hereby the transverse growth mode was attributed to the strong spreading capability of \ce{MoO3}. The final morphology of observed nanoplate and nanobelt was a combined result of \gls{vs} and \gls{vls} growth. On the other hand, the tower-like structures in the second stage followed the axial growth mode in conventional VLS process. The molybdates can be evaporated from the higher temperature end and re-condense onto the lower temperature region. As illustrated in the lower panel of Fig.~\ref{fig:ch4vls}, this could initiate either belt growth or tower growth depending on the locations. The microtowers in this study primarily grew at the low temperature end of the substrate, where the spreading capability of the Mo oxide species was possibly limited. Thus the transverse growth pathway was probably suppressed and axial growth became dominant. The above analysis was substantiated by the multiple observation revealed by SEM imaging, as shown in Fig.~\ref{fig:ch4msch}.
\begin{figure}[htb]
\centering
\includegraphics[width=0.6\textwidth]{moo3_sche}
\caption[SEM images for catalytic \ce{MoO3} growth]{SEM images for catalytic \ce{MoO3} growth: SEM images showing catalyst particles at different locations: (a) tip of a forked nanplate, (b) middle of a nanobelt, and (c) tip of another thin nanobelt; SEM images of microtowers with different morphologies: (d) cylindrical shape with tapered tip, (e) flat top, and (f) cone shape. }
\label{fig:ch4msch}
\end{figure}

The transport of molybdates in proposed growth model was verified by a side-by-side experiment, as schematically depicted in Fig.~\ref{fig:ch4sbs}a. Two substrates, one with NaOH catalyst and one bare substrate, were loaded side-by-side at the same location for the growth. To avoid any possible catalyst transports through surface diffusion, the two substrates were separated with several millimeters gap in between. 
\begin{figure}[htb]
\centering
\includegraphics[width=0.8\textwidth]{MoO3_SFig_sidebyside}
\caption[Side-by-side growth of \ce{MoO3}]{Side-by-side growth of \ce{MoO3}. (a) Schematic drawing of the side-by-side growth with one NaOH catalyst treated substrate and one bare substrate. SEM images of the growth on the untreated substrates at different growth times: (b-e) 30 min and (f-i) 120 min. For 30 min growth: (b) low magnification SEM image showing labeled growth areas with different morphologies, (c) rectangular nanoplates grown without catalyst, (d) nanobelts grown with catalyst, and (e) forked nanoplates grown with catalyst. For 120 min growth: (f) rectangular nanoplates without catalyst, (g) forked nanoplates with catalyst, (h) microbelts with catalyst, and (i) microtowers with catalyst.}
\label{fig:ch4sbs}
\end{figure}
After the growth, besides the non-catalyst rectangular plates, catalyst induced 1D \ce{MoO3} structures were also found on the bare substrate (Fig.~\ref{fig:ch4sbs}b-i). This result clearly confirmed that the catalyst can be evaporated and transferred from the NaOH-treated substrate onto the bare one. The catalyst vapors can nucleate and form catalyst particles on the \ce{MoO3} deposition promoting the VSS growth of different 1D structures. It is worth mentioning that catalyst particles were not observed on all the 1D structures. This fact can be explained by following reasons:
\begin{enumerate*}[label=\itshape\alph*\upshape)]
\item Catalyst particles can nucleate at any locations along the 1D structures. Some particles may be invisible hiding on the backside of the 1D structures;
\item Because of the evaporation, the catalyst particles may become too small to be detected. Some particles could even have disappeared terminating the catalyzed growth, which may explain the growth of the microtowers with a flat top.
\end{enumerate*} The evaporation and nucleation of catalyst not only promoted the growth of the 1D structures in the second stage growth, they also shaped the morphologies of these 1D structures. Many 1D structures in the second stage were found to have a tapered shape, such as triangular nanobelts (Fig.~\ref{fig:ch4sem2by3}b and d), microtowers with tapered tips (Fig.~\ref{fig:ch4sem2by3}f). Several factors can contribute to the tapering of the 1D structures. First, the reduction in catalyst size may induce the tapering. Due to the evaporation, the size of the catalyst particles can shrink during the growth resulting in the tapered growth. The \gls{vs} growth on the side surfaces was another possible mechanism, which promoted the radial growth of the 1D structures forming the tapered shape. Another important factor was the gradually reduced growth during the cooling process. During the cooling, the \ce{MoO3} vapor supply reduced, and the growth temperature gradually decreased. Hence, the growth rate slowed down producing the tapered tips on the top of some 1D structures.

\subsection{Growth on Alkali Metal Containing Substrates}\label{sec:glass}
In analogy to the \ce{Na2MoO4}-\ce{MoO3} binary phase diagram, other alkali metals, such as Li and K, could also form A-Mo-O alloy at elevated temperature. This has potential for using a variety of alkali metal based material as catalysta to promote \gls{vls} or \gls{vss} growth of \ce{MoO3}. Two groups of experiments have been performed in this study; the first one used \ce{KI} and \ce{Na2CO3} treated Si substrate and the second one used alkali metal ions containing substrate, including ITO/glass, glass, and mica. Fig.~\ref{fig:ch4al1}a-b show the long belt growth on KI treated Si substrate and the forked plate growth on \ce{Na2CO3} treated Si, respectively. Compared to NaOH-Si growth, the belt on KI appeared with rough edge; whereas no long belt or tower was observed on the sample with \ce{Na2CO3}. 

\begin{figure}[htb]
\centering
\includegraphics[width=0.6\textwidth]{moo3_alt2}
\caption[Alternative catalysts for \ce{MoO3} growth]{Alternative catalysts for \ce{MoO3} growth. SEM images of \ce{MoO3} 1D structures: (a) nanobelts grown with KI on Si and (b) the forked plate growth with \ce{Na2CO3} on Si.}
\label{fig:ch4al1}
\end{figure}

Fig.~\ref{fig:ch4al2}a-c illustrated microtowers grown on ITO glass, sword-shape belts on glass, and nanobelts growth on mica, respectively. The towers on ITO glass exhibited stacking features. The sword-shape belts on glass arose from a combination of VSS and VS mechanism. The long belts found on mica (\ce{K(Al2)(Si3Al)O10(OH)2}) probably stemmed from potassium-catalyzed \gls{vls} mechanism.\cite{Hu2011}

\begin{figure}[htb]
\centering
\includegraphics[width=0.8\textwidth]{moo3_alt1}
\caption[Alternative catalysts for \ce{MoO3} growth cont]{Alternative catalysts for \ce{MoO3} growth. SEM images of MoO3 1D structures: (a) microtowers grown on ITO glass, (b) triangular microbelts grown on glass, and (c) nanobelts and microbelts grown on mica.}
\label{fig:ch4al2}
\end{figure}
A rudimentary analysis on the \ce{Na+} diffusion in glass was performed to understand the growth dynamics of \ce{MoO3} on the alkali ions containing substrates. Because the alkali ions in glass can diffuse out at elevated temperature, the proposed \gls{vss} mechanism will modify the \ce{MoO3} deposition. Therefore, the mass transport of sodium ions in silica glass was first estimated to evaluate the amount of catalyst available during the growth of \ce{MoO3}. The driving force could be external field or concentration gradient, and the dynamics was governed by diffusion equation
\begin{align}\label{eq:nadiff}
\frac{\partial C}{\partial t} = D \frac{\partial^2 C}{\partial x^2},
\end{align}
where $C$ is concentration in unit of \si{mol\per cm^3}, and $D$ is diffusion coefficient in unit of \si{cm^2\per\second}. Diffusion coefficient highly depends on the overall environment in which the ions reside. Typical values of $D$ for Na ions were listed in Table~\ref{tab:mona}

\begin{table}[htb]
\centering
\caption[Mass transport of sodium ions in silica glass]{Mass transport of sodium ions in silica glass, with diffusion coefficient in unit of \si{cm^2\per\second}}\label{tab:mona}
\begin{tabular}{lcr}
\toprule
 Composition & Na diffusion coefficient  & Reference  \\
\midrule
Quartz      & $3.8\times10^{-2}\exp(\frac{-24500}{RT})$  & \cite{Rybach1967a}  \\
 \addlinespace[0.5em]
Sodalite      & $6.6\exp(\frac{-42500}{RT})$  & \cite{Sippel1963}  \\
 \addlinespace[0.5em]
Obsidian     & $4.4\times10^{-2}\exp(\frac{-22900}{RT})$  & \cite{Sippel1963}  \\
 \addlinespace[0.5em]
silicate glass & $3.1\times10^{-8}$ at 420\si{\degreeCelsius} & \cite{Jbara1995} \\
 \addlinespace[0.5em]
\ce{SiO2} glass & $1.1\times10^{-7}$ at 670\si{\degreeCelsius} &  \cite{FRISCHAT1968}\\
\bottomrule
\end{tabular}

R is molar gas constant (8.315 \si{\joule\per mol\per K}), and T is temperature in K.
\end{table}

The substrates used in this work were glass and ITO/glass; the relevant parameters were as following:
\begin{itemize}
\item dimension: $25\times10\times1$ mm
\item density: 2.567 \si{g\per cm^3}
\item Na concentration: 0.01 \si{mol\per cm^3}
\item $D$: $1.1\times10^{-7}$ \si{cm^2\per\second}
\end{itemize}
Substituting these values into Eq.~\ref{eq:nadiff}, the net flux $J$ was estimated to be
\begin{align}
J = -D \frac{\Delta C}{\Delta x} = 2.0\times 10^{-8} \si{mol cm^{-2}s^{-1}}.
\end{align}
In typical growth of 2 h, the net amount of Na diffusing out of substrate was then $1.44\times10^{-4}$ mol, about 10\% of total Na ions in the glass substrates. The diffusion length $\sqrt{Dt}$ was 190 $\mu$m. Note this result overestimated since the flux will reduce gradually. Meanwhile the molar amount of \ce{NaOH} applied on Si substrate was about $5\times10^{-7}$ mol. Considering the adjusted ratio, Na contents should be much more when using glass substrate than using NaOH-Si substrate. Another important factor was the evaporation of sodium molybdates; however, only one relevant report by \citeauthor{Kazenas2010} could be found in literature.\cite{Kazenas2010} The partial pressure of \ce{Na2MoO4} was calculated to be 0.2 mTorr at 800 \si{\degreeCelsius} and $3.5\times 10^{-3}$ mTorr at 670 \si{\degreeCelsius}.\footnote{$\log P(atm)= -13794/T + 6.19$} 

\subsection{Optical Properties of Molybdenum Oxide}

%
\gls{fl} materials attracts intensive research efforts recently. This study also made some efforts using liquid exfoliation method to prepare \gls{fl} \ce{MoO3}. The exfoliated \ce{MoO3} was characterized using UV-Vis and TEM. The optical contrast of \gls{fl} \ce{MoO3} on \ce{SiO2}-Si is also calculated, which could serve as a guideline to identify \gls{fl} \ce{MoO3} using optical microscope.

The exfoliation solution is 30\% isopropanol (IPA) in DI water, which has been proved to work well on other layered compounds.\cite{Halim2013} After 20 min sonication, the dispersion appears milky. The transmission spectra of this dispersion immediately upon sonication and after 40 h of gravity sedimentation were measured and compared in Fig.~\ref{fig:moabs}. The t0 line mainly arise from the scattering of flakes in the dispersion. There are two scattering mechanisms: Rayleigh scattering and Tyndall scattering. Rayleigh scattering, which occurs when particles are comparable to wavelength, is inversely proportional to the fourth power of wavelength; while Tyndall scattering, which occurs when the particles are larger, is inversely proportional to the square of the wavelength. To fully evaluate the true absorbance of the sample itself, one needs to decouple the scattering part from apparent absorbance. Generally, one part of spectrum away from absorption edge is selected and assumed it is only caused by scattering. Then an empirical dependence shown in Eq.~\ref{eq:sca2} is used in the polynomial fitting model.

\begin{align}
Abs_{sca}  & = a\times \lambda^{n}  \label{eq:sca1}\\
\log{Abs_{sca}} & = \log{a} + n*\log{\lambda} \label{eq:sca2}
\end{align}

After least-squares fitting, the coefficients $a$ and $n$ can be used to estimated the scattering in other wavelengths. The fitting of $n$ usually falls between -2 and -4. The author examined the t40h line in Fig.~\ref{fig:moabs} and found the scattering part is almost zero. Therefore no scattering fitting was performed.

\begin{figure}[htb]
\centering
\subfloat[]{\label{fig:moabs}\includegraphics[width=0.45\textwidth]{MoO3-abs}}\hspace{0.04\textwidth}
\subfloat[]{\label{fig:mobg}\includegraphics[width=0.45\textwidth]{MoO3-bg}}
\caption[UV-Vis spectra of exfoliated \ce{MoO3}]{UV-Vis spectra of exfoliated \ce{MoO3}. (a) UV-Vis absorbance spectra of exfoliated \ce{MoO3}. Dashed line (t0) represents measurement immediately after sonication, and solid line (t40h) is obtained after sitting for 40 h. The dot dashed line is reference spectrum of the 30\% IPA. (b) Band gap estimation}
\label{fig:mouv}
\end{figure}

According to the corrected absorbance spectrum, optical bandgap can be deduced, which is known as Tauc Plot.\cite{Tauc1972} The procedure is recapitulated here briefly. Eq.~\ref{eq:tauc} is given to empirically relate optical band gap to absorption.

\begin{equation}\label{eq:tauc}
 (h\nu \alpha)^{\frac{1}{n}} = B(h\nu - E_g^{opt}),
\end{equation}
where B is a proportional coefficient, $\alpha$ is absorption coefficient\footnote{The absorbance $A = 0.4343\alpha*l$, where l is the optical path} and $E_g^{opt}$ is defined as optical bandgap. The value n is correlated to the nature of optical transition.
\begin{itemize}
\item For direct allowed transition, n = 1/2;
\item For direct forbidden transition, n = 3/2;
\item For indirect allowed transition, n = 2;
\item And for indirect forbidden transition, n = 3.
\end{itemize}

When the transition nature of sample is not well known, one should apply each to check which one provide the best fit. Depending on the synthesis methods and \ce{MoO3} morphologies, the transition phenomena might vary. On the \ce{MoO3} films, \citeauthor{Bouzidi2003} assumed a direct gap (n = 1/2)\cite{Bouzidi2003} while \citeauthor{Szekeres2002} used an indirect gap.\cite{Szekeres2002} The author use n = 2 in this work, plot $\sqrt{(h\nu \alpha)}$ versus photon energy, and extrapolate the linear part onto $h\nu$ axis. The intercept is used as $E_g^{opt}$ then. The estimated energy gap and some results from previous studies is summarized in Table.~\ref{tab:mobg}.

\begin{table}[htb]
\centering
\caption{Optical band gaps of \ce{MoO3}}\label{tab:mobg}
\begin{tabular}{lccr}
\toprule
&\multicolumn{2}{c}{Band gap} \\
\cmidrule(l){2-3}
Reference & value(eV) & orientation & material states\\
\midrule
\cite{Deb1968}   & 2.96  & $E\parallel c_0$ & single crystal\\
\cite{Deb1968}   & 2.80  & $E\perp c_0$ & single crystal \\
\cite{Julien1995} & 2.8$\sim$ 3.2 & NA & films\\
 this work  & 3.1  & NA & nanoflakes\\
\bottomrule
\end{tabular}
\end{table}

An alternative approach is a) define $Y = \alpha E=\alpha h\nu$, and obtain $Y' = \frac{\partial Y}{\partial E}$, then b) plot $Y'/Y$ versus $h\nu$. The band gap value can be obtained from the intercept of extrapolated line on photo energy axis.\cite{Choopun} This approach is not difficult to comprehend as long as we notice from Eq.~\ref{eq:tauc}, $Y = B (h\nu - E_g^{opt})^n$ and $Y' = B n(h\nu - E_g^{opt})^{n-1}$. The advantage is one can also estimate n without using assumption.

The dielectric constants\footnote{Not a constant at all, instead, it has complicated dependence on photon energy} ($\epsilon = \epsilon_1 + i*\epsilon_2$) of \ce{MoO3} have been studied in several reports.\cite{Deb1968,Sabhapathi1995,Miyata1996,Abdellaoui1997,Mondragon1999} The author extracted $\epsilon_1$ from Ref.~\cite{Itoh2001a} due to the wide photon energy range utilized there. Notice that $\epsilon_2$ is approximately zero when $h\nu$ is less than 4 eV. Since this study is primarily concerned with the optical contrast of few-layered \ce{MoO3} in visible wavelength region ($h\nu < 3$ eV), the refractive indices $n$ is calculated from $n = \sqrt{\epsilon_1}$. Due to the orthorhombic phase, \ce{MoO3} exhibit birefringence, that is there are two refractive indices, $n_c$ and $n_a$, as shown in Fig.~\ref{fig:moind}. The refractive index is between $ 2.2\sim 2.4$ along $a$ axis, and $ 2.5\sim 2.8$ along $c$ axis.

\begin{figure}[htb]
\centering
\subfloat[]{\label{fig:moind}\includegraphics[width=0.45\textwidth]{n_MoO3}}\hspace{0.04\textwidth}
\subfloat[]{\label{fig:mocon}\includegraphics[width=0.45\textwidth]{MoO3_1L}}
\caption[Refractive indices of \ce{MoO3}]{Refractive indices of \ce{MoO3}. (a) Refractive indices of \ce{MoO3} Dashed line is n along $a$ axis while solid line is along $c$ axis. (b) optical contrast mapping of 1L \ce{MoO3} on \ce{SiO2}-Si. The $x$ axis is \ce{SiO2} thickness, $y$ axis is wavelength. It is assumed that the incident light is polarized along $c$ axis of \ce{MoO3}.}
\label{fig:mofl}
\end{figure}

Fig.~\ref{fig:mocon} illustrates the optical contrast mapping of 1L \ce{MoO3} on \ce{SiO2}-Si when the incident light is polarized along $c$ axis of \ce{MoO3}. One can see that there is a pseudo-periodical trend with either wavelength or \ce{SiO2} thickness fixed. The positive extreme occur at around 80 nm and 240 nm \ce{SiO2} layer, slightly less than the optimum for observing graphene. The refractive indices of Si and \ce{SiO2} within visible wavelengths and MATLAB source code for optical contrast calculation is included in Appendix~\ref{app:matlab}, with which one can readily evaluate the few layers \ce{MoO3} scenarios and the circumstance when incident light is polarized along $a$ axis.

The dimension and phase of exfoliated \ce{MoO3} is examined using TEM. Fig.~\ref{fig:motem} shows the size of exfoliated \ce{MoO3} is about 500 nm. High-resolution TEM (HRTEM) image was taken to assess the \ce{MoO3} crystallinity. The clearly resolved lattices indicate high crystal quality of these exfoliated \ce{MoO3}.
\begin{figure}[htb]
\centering
\includegraphics[width=0.8\textwidth]{exf_moo3.jpg}
\caption[TEM images of exfoliated \ce{MoO3}]{TEM images of exfoliated \ce{MoO3}. (a) TEM image and (b) HRTEM image of FL \ce{MoO3} obtained by liquid sonication method.}
\label{fig:motem}
\end{figure}
Nevertheless, identifying these few layer \ce{MoO3} under optical microscope is still difficult due to both the small flake size and low contrast.\footnote{Human eye threshold contrast (contrast sensitivity)$^{-1}$ is about 10\%.} One need to improve the liquid exfoliation process to obtain larger dimension and high yield. This is, however, beyond the scope of current study.
The optical properties of the as-grown specimen were evaluated using \gls{drs} method. Fig.~\ref{fig:modrs} show the hemisphere diffuse reflection spectrum from 800 nm to 250 nm, where the diffuse reflection $R_{dff}$ kept increasing from 27\% to 33\% when incident wavelength scanned from 800 nm to about 400 nm; then, $R_{dff}$ dropped sharply to 15\% between 400 nm and 300 nm. The KMT function fitting was used to estimated the energy gap, as displayed in Fig.~\ref{fig:mokmt}. Extrapolation indicated a gap of 3.2 eV, corresponding to absorption edge at 387 nm. This matched well with the DRS data.
\begin{figure}[htb]
\centering
\subfloat[]{\label{fig:modrs}\includegraphics[width=0.45\textwidth]{moo3_DRS}}\hspace{0.04\textwidth}
\subfloat[]{\label{fig:mokmt}\includegraphics[width=0.45\textwidth]{moo3_kmt}}
\caption[DRS spectrum of \ce{MoO3} nanostructures]{DRS spectrum of \ce{MoO3} nanostructures: (a) diffuse reflection spectrum, and (b) the associated KMT function.}
\label{fig:moo3drs}
\end{figure}
Another reflection peak was found at 260 nm, illustrating there is more complex electronic band structures for \ce{MoO3} above 4 eV. As a comparison, the estimated energy gap and results from previous studies is summarized in Table~\ref{tab:mobg}.
\begin{table}[htb]
\centering
\caption{Optical band gaps of \ce{MoO3}}\label{tab:mobg}
\begin{tabular}{lccr}
\toprule
&\multicolumn{2}{c}{Band gap} \\
\cmidrule(l){2-3}
Reference & value(eV) & orientation & material states\\
\midrule
\cite{Deb1968}   & 2.96  & $E\parallel c_0$ & single crystal\\
\cite{Deb1968}   & 2.80  & $E\perp c_0$ & single crystal \\
\cite{Julien1995} & 2.8$\sim$ 3.2 & NA & films\\
 this work  & 3.2  & NA & nanoflakes\\
\bottomrule
\end{tabular}
\end{table}

\section{Summary}

In this chapter, a new growth model of \ce{MoO3} using alkali metal ions as catalysts was discovered and investigated in depth. The growth kinetics were closely inspected using a series of controlled experiments. A modified vapor-solid-solid mechanism was proposed and favorably verified. Two growth modes, i.e., transverse growth and axial growth, were highlighted, which not only improved the yield but also allowed for versatile morphology control of 1D \ce{MoO3} nanostructures. This methodology was further extended onto glass and ITO glass substrates, paving the way to a more integrated process of using \ce{MoO3} nanostructures onto various solar energy saving and harvesting applications. 
 % no tex needed, otherwise there is error
 %   \chapter{TUNGSTEN OXIDE-TUNGSTEN DISULFIDE NANOWIRES}

In this chapter, a TEM-Raman integrated study on \ce{WO3}-\ce{WS2} core-shell nanostructures is presented. This study leverages a home-built micromanipulator apparatus to transfer the nanostructures from growth substrates to TEM grids. This transferring technique has allowed for correlating TEM and micro-Raman measurement on individual \ce{WO3}-\ce{WS2} core-shell nanowires and enhancing the understanding of structure-property relations, which would otherwise be blurred in conventional ensemble averaged methods. \ce{WO3}-\ce{WS2} heterostructure was chosen due to its potential applications in photocatalysis and hydrogen evolution reaction.\cite{DiPaola1999,Chen2011b} 

The remaining sections are arranged in the following order: a review on the growth of \ce{MS2} (M = Mo, W) \glspl{nt} and 2D \gls{fl} structures will be introduced first, followed by Raman spectroscopy summary on \ce{WS2} NTs and FL. The thesis will then focus on synthesis of \ce{WO3}-\ce{WS2} heterostructure and various characterizations on both ensemble and individual nanowires. This chapter is concluded with a brief summary and future work.

\section{Introduction}
% \subsection{crystal structure of tmdc}
\subsection{Growth of TMDC}
Shortly after the discovery of carbon nanotube (CNT) in 1991,\cite{Iijima1991} its transition metal dichalcogenides (TMDC) counterparts, \emph{i.e.} \ce{WS2}, was synthesized in 1992.\cite{Tenne1992} In the following decades, other TMDC and several metal oxides nanotubes were demonstrated as well. \citeauthor{Rao2003} did an excellent work to review these inorganic nanotubes.\cite{Rao2003} Here the author just recapitulate the main categories of synthesis methods.
\begin{enumerate}
\item Reducing \ce{MO3} in \ce{H2} and \ce{H2S} atmosphere\footnote{M = Mo, W in this work}
\item Direct decomposition of \ce{MS3}
\item Decomposition of the ammonium salt \ce{(NH4)2MS4}
\item Using CNT as templates, arc discharge, or laser ablation
\end{enumerate}

Except the aforementioned methods, another important reaction in CVD domain is
\begin{equation}\label{eq:mclns}
\cee{MoCl5(WCl6) + S \rightarrow MoS2(WS2) + S2Cl2}.
\end{equation}
This reaction has been explored by several authors,\cite{Stoffels1999} and could be used under atmospheric pressure.\cite{Li2004a}

All these methods can be applied to the growth of TMDC few layer structures, either directly or with some modifications. Among all the members of TMDC nanotubes, \ce{WS2} and \ce{MoS2} NTs are probably the most well investigated.\cite{Homyonfer1997,Tenne1998,Frey1998,Frey1999,Rothschild2000,Zak2000} The reaction mechanism of \ce{WO3} with \ce{H2}/\ce{H2S} has been thoroughly studied \cite{Feldman1998} and high yield synthesis approach has been established.\cite{Margolin2004} For \ce{WO3} nanoparticles precursor, it was found that the simultaneous reduction and sulfurization is essential for encapsulation of fullerene like \ce{WS2} structures from the oxide nanoparticles. During the surfurization process, remaining oxide core is gradually reduced and transformed into an ordered superlattice of $\{ 001 \}$ \gls{cs} planes. Further reaction will consume up the \ce{WO_x} core, leaving multi-walled \ce{WS2} NTs only. For precursor of \ce{WO3} nanowires, \citeauthor{Feldman1996} proposed that TMDC nanotubes growth began with the reduced \ce{WO3} phases, in particular \ce{W_{18}O_{49}}.\cite{Feldman1996} It is worth noting that \ce{MS2} can be prepared by direct sulfurization from its oxides phase, but reaction \cee{MO3 + Se \rightarrow MSe2} will not proceed unless introducing other reducing agents, such as \ce{H2}.\cite{Tsirlina1998} This fact highlights that sulfur is rather radical at elevated temperature.

\citeauthor{Zhu2000} performed a detailed morphological and structural analysis on \ce{WS2} NTs synthesized using \ce{WO_{3-x}} NWs and \ce{H2S}.\cite{Zhu2000} Sulfur vacancy was found on the outer wall of NT. The tips exhibits various structures. Open-ended can be observed, more frequently than that in carbon nanotubes (CNTs), which is often sealed. \citeauthor{Zhu2000} suggested this open-ended tubes resulted from continuous growth on other nanotubes. And the closure configuration is rather complicated. Flat caps often dominates. \citeauthor{Zhu2000} maintained that this oxide-to-sulfides mechanism might apply to closed caps only, not to those open ended tubes.\cite{Zhu2000} TEM diffraction analysis could distinguish the chirality of NTs, as demonstrated by CNT \cite{Zhang1993} and \ce{MoS2}\cite{MARGULIS1996}. \ce{WS2} NT chirality revealed by TEM SAED shows armchair NTs dominates. \citeauthor{Sloan1999} investigated tungsten oxides structures incorporated in \ce{WS2}.\cite{Sloan1999} The encapsulated \ce{WO_x} cores often exhibit \ce{W3O8} and \ce{W5O_{14}} phases, both of which belong to the \ce{W_nO_{3n-1}} homogenous series, arising as a result of \gls{cs} planes.\cite{Miyano} Some \ce{WO_x} cores show oxygen vacancy instead of CS planes, leading to prominent streaking in SAED patterns. Although high yield growth method of \ce{WS2} has been available, the extent to which one can control the NT configuration is still limited, \emph{i.e.}, single-walled NT has not been routinely synthesized and the electronic properties can only be predicted theoretically. \citeauthor{Seifert2000} investigated the electronic structures of \ce{WS2} nanotube using DFT calculation.\cite{Seifert2000} It was found zigzag (n,0) NTs exhibts a direct gap at $\Gamma$ point, whose size increase monotonically with tube diameter, while armchair (n,n) NTs, unlike its metallic counterpart of (n,n) CNT, show indirect band gap increasing with tube diameter. \citeauthor{Zibouche2012} reached the same conclusion on \ce{WS2} SWNT.\cite{Zibouche2012} It was also found band gaps of armchair and zigzag NTs increase with diameter, going from values close to bulk and approaching that of  monolayer, and for a given tube diameter, $E_g$ of zigzag NTs are larger than those of armchair NTs. This difference presumably vanishes when nanotube diameter exceeds 100 nm.

Parallel with the scenario of CNT and graphene, TMDC few layer structures attract intensive research efforts recently. These unfolded TMDC nanotubes exhibit many appealing features, i.e., indirect-to-direct band gap transition,\cite{Splendiani2010} valley electronics, and become a promising candidate in energy harvesting, optoelectronics and photocatalytic activity. For scientific research purpose, exfoliation by liquid \cite{Smith2011} or mechanical methods \cite{Lee2010a} can provide sufficient materials. Yet considering the integration into current microelectronic process, CVD synthesis of thickness controllable 2D TMDC is highly desired, which stills hold the greatest potential for high yield production of \ce{MS2}. So this work focuses on reviewing the CVD based methods. The author will review literatures to date on both \ce{WS2} films and few layer structures, and highlight several typical investigations using different synthesis methods, as summarized in Table~\ref{tab:tmsgrowth}. It is important to note that the synthesis of NTs is relevantly independent of substrates, which is in sharp contrast to the scenario of few layer growth. The \ce{WS2} nucleation and interaction with substrates play dominant roles in the growth of 2D FL structures. This lead us to focus on the \ce{WS2} films growth, which presumably could provide valuable insight into the nucleation process on various substrates.

\textbf{\ce{MS2} films} \citeauthor{Lee1994} studied the CVD of \ce{MoS2} by \ce{MoCl5} or \ce{MoF6} and \ce{H2S} in great details. Phase diagrams for \ce{Mo-S-Cl-H} and \ce{Mo-S-F-H} system at 1 kPa were simulated.\cite{Lee1994} \citeauthor{Endler1999} also investigated the solid-phase diagram for \ce{Mo-S-Cl-H-Ar} system.\cite{Endler1999} It was found pure \ce{MoS2} was main product when molar ratio of \ce{H2S}-\ce{MoCl5} exceeds 2. More important, \ce{MoS2} basal plane orientation can be parallel to substrate when the thickness was smaller than 50 nm. \citeauthor{Ennaoui1995a} grew tungsten disulfide film using sulfurization of \ce{WO3} under \ce{N2}/\ce{H2} gas flow.\cite{Ennaoui1995a} The composition was found to be \ce{WS_{2.13}}, and the excess of sulfur lead to p-type conductivity. XRD peaks ratio is used as a measure of film orientation, which was found higher for film prepared without an intermediate Ni coating. \ce{Ni3S2} phase was found and surfactant-mediated epitaxy is proposed. \citeauthor{Regula1997} studied the Ni-W-S phase diagram and the role of Ni layer in promoting \ce{WS2} film growth from amorphous \ce{WS3}.\cite{Regula1997} \emph{In-situ} TEM analysis confirmed the formation of \ce{NiS_x} droplets and lateral growth of \ce{WS2} from these droplets.\cite{Regula1998} Recently, it has been shown that direct sulfurization of W coating (20nm) at 750 \si{\degreeCelsius} on \ce{SiO2}-Si can also produce \ce{WS2} film.\cite{Shanmugam2012a} However, the film is of bulk in nature. This could be explained by the study of \citeauthor{Genut1992} towards \ce{WS2} growth on various substrates,\cite{Genut1992} as summarized in Table~\ref{tab:ws2subs}. It was found that the \ce{WS2} nucleate from an amorphous \ce{WS3} phase and the substrate has a critical role in determining both the reaction onset temperature and the texture. The adhesion of tungsten to quartz was found to be much stronger than to glass. And oxygen-containing species such as \ce{H2O} or OH tend to cause \ce{WS2} basal plane perpendicular to substrate.
% WS2 films
\begin{table}[htb]
\centering
\caption{\ce{WS2} growth on different substrates}\label{tab:ws2subs}
\begin{tabular}{lcp{1in}p{2in}}
\toprule
precursor                 & substrate &  conditions & film feature\textsuperscript{\emph{a}}  \\
\midrule
sputtered W + \ce{H2S}   & glass      & onset 400 \si{\degreeCelsius} & $\perp c$ at 500 \si{\degreeCelsius}, metastable \ce{WS3} found\\
                          & quartz      & onset 650 \si{\degreeCelsius} & $\parallel c$ below 950 \si{\degreeCelsius}\\
                          & Mo        & onset NA          & $\parallel c$ at 1000 \si{\degreeCelsius}\\
                          & W          & onset NA           & random orientation\\
\midrule
\ce{WO_x} + \ce{H2S}    & quartz      & onset 500 \si{\degreeCelsius} & predominantly $\perp c$ after 800 \si{\degreeCelsius}\\
                        & Mo        & onset NA          & $\parallel c$ dominant\\
                        & W       & onset NA          & random orientation\\
\bottomrule
\end{tabular}

\textsuperscript{\emph{a}} $\perp c$: the c axis is perpendicular to the substrate, $\parallel c$: the c axis is parallel to the substrate;

\end{table}

\textbf{\ce{MS2} FL} For \ce{MoS2}, the reaction mechanism of \ce{MoO3} to \ce{MoS2} was studied by \citeauthor{Weber1996}.\cite{Weber1996} This study provided guidelines for the recent syntheses of \ce{MoS2} by sulfurization of \ce{MoO3}.\cite{Lin2012,Lee2012b,Liu2012a,Najmaei2013} Similar studies on sulfurization of \ce{WO3.H2O} to \ce{WS2} and decomposition of \ce{(NH4)2WO2S2} were also reported.\cite{VanderVlies2002,VanderVlies2002a} In combination with the knowledge from \ce{WS2}, the author compare those insight from several recent reports on \ce{WS2} FL. \citeauthor{Cong2013} prepared monolayer \ce{WS2} on 300 nm \ce{SiO2}-Si by sulfurization of \ce{WO3} powders in a one-end sealed tube.\cite{Cong2013} It was suggested that pre-cleaning the inner tube by IPA and DI water could effectively increase the pressure of vapor source. This observation probably arise from the reduction of \ce{WO3} assisted by IPA and water residuals, or due to the possible presence of \ce{H2S}. Intermediate phase \ce{WO_yS_{2-y}} and \ce{WS_{2+x}} is proposed in the growth mechanism. The apex of triangles could be active site of nucleation, \ce{WS_{2+x}} formation is confirmed by secondary ion mass spectrometry. Possible thick \ce{WS_{2+x}} flakes decompose subsequently, leading to ML \ce{WS2}. The substrates are thoroughly cleaned. It was mentioned that separate sulfur heating improved the PL uniformity. Sulfurization mechanism study of \ce{WO3} suggested \ce{W^{6+}} cannot be directly reduce to \ce{W^{4+}} in \ce{WS2}.\cite{VanderVlies2002,VanderVlies2002a} Tungsten oxysulfides was necessary as the intermediate phase. 

\citeauthor{Peimyoo2013} prepared \ce{WS2} on \ce{SiO2}-Si using \ce{WO3} powder and sulfur at 800 \si{\degreeCelsius}, aiming at the light emission studies to clarify several contradictory reports.\cite{Peimyoo2013} Uniform PL intensity was found on the triangular \ce{WS2} flakes, in contrast to previous edge enhanced PL.\cite{Berkdemir2013} Raman spectra (532nm) fit includes $E_{2g}^1(M)$ mode at 343 cm$^{-1}$, according to the phonon dispersion calculation \cite{Molina-Sanchez2011} and experimental observation\cite{Zeng2013a,Zhao2013,Lee2013}. However, none of these experimental reports specifically mentioned $E_{2g}^1(M)$ mode. Tentative assignments of multi-phonon bands are summarized in supporting information of Ref\cite{Zhao2013}. As to the growth setup, similar tube furnace as in Ref \cite{VanderZande2013} and \cite{Najmaei2013} was used. To gain additional wisdom on \ce{MoS2} growth, the author thus summarize the growth strategies in Ref \cite{VanderZande2013} and \cite{Najmaei2013} as following. \citeauthor{VanderZande2013} prepared \ce{MoS2} on 285nm \ce{SiO2}-Si using \ce{MoO3} and S as precursors.\cite{VanderZande2013} In contrast to the seeding method adopted in ref\cite{Lee2013,Lee2012b}, \citeauthor{VanderZande2013} stressed the importance of carefully cleaned substrates\footnote{acetone, 2hrs in \ce{H2SO4} and \ce{H2O2} (3:1) and 5 mins oxygen plasma} and minimum exposure of precursor to air.\footnote{APCVD, 105 \si{\degreeCelsius} for 4hrs, 700 \si{\degreeCelsius} hold for 5mins, and 10 sccm \ce{N2} within 2 inch tube. rapid cooling from 570 \si{\degreeCelsius}} Dirty substrates or old precursors will lead to hexagonal, 3-point star or irregular polycrystalline structures. The growth setup is similar to those in Ref\cite{Lee2012b}. The substrate and \ce{MoO3} source distance is critical in determining the growth density. The synthesis strategy in Ref\cite{Lee2013,Lee2012b} is briefly mentioned as well, where PTAS treated substrate is found to promote the deposition, while KCl treated substrates did not, and small carrier gas flow is preferred (1 sccm \ce{N2}). It was also noted that the \ce{MoS2} flakes morphology seems more uniform in Ref\cite{VanderZande2013} than that in Ref\cite{Lee2012b}. On the other hand, \citeauthor{Najmaei2013} also demostrated \ce{MoS2} FL growth by sulfurization of \ce{MoO3} nanoribbons.\cite{Najmaei2013} The ribbons is applied by dispersion and meant to control the source amount. It was found the diffusion of vapor \ce{MoO_{3-x}} is rate-limiting step in \ce{MoS2} growth. This means the amount of source is critical in successful synthesis. The nucleation event is more frequently observed at subs edges, scratches or rough surface. Step edges is then intentionally created to facilitate the nucleation. As to the growth dynamics, it was postulated the oxisulfides(\ce{MoOS2} Raman spectra found), as intermediate phase, diffuse across the bare substrates and form triangular domains upon further sulfurization. The optimal growth conditions are 800--850 \si{\degreeCelsius}, 700 Torr, and sufficient sulfur. This theis then conjectures that the substrate cleaning in ref\cite{Peimyoo2013} should be similar to that in ref\cite{VanderZande2013}, yet the source-to-substrate layout and growth pressure are still needed to be optimized with respect to our current apparatus.

\citeauthor{Zhang2013h} synthesized \ce{WS2} on sapphire (0001) using \ce{WO3} powder and sulfur as precursor under 900 \si{\degreeCelsius}.\cite{Zhang2013h} Ar slightly mixed with \ce{H2} is used to tailor the shape of \ce{WS2} flakes. It was found the source substrate distance plays an important role in determining the morphology of the as-grown flakes.\footnote{There is a lattice mismatch between \ce{Al2O3} (4.785 \AA) and \ce{WS2} (3.153 \AA)} The edge termination is not well studied. Raman spectra indicate a universal down-shift of $A_{1g}$ peak (1L from 418 on \ce{SiO2}-Si to 416.4 on sapphire). PL signal on \ce{SiO2}-Si is stronger than on sapphire. The visibility of \ce{WS2} on \ce{Al2O3} is poor. No 1L vs nL statistic is available. FET on/off ratio is about 100, indicating low mobility. The other conditions includes 880 \si{\degreeCelsius}, 90 mm source-to-substrate distance, 1 in quartz tube, and LBM furnace. The growth conditions are probably adopted from ref\cite{Huanga2013}.

It is worth noting the flake sizes on sapphire\cite{Zhang2013h} seems larger than that on \ce{SiO2}/Si\cite{Peimyoo2013}, either due to the nucleation barrier difference or the amount of growth vapor and growth time. The flake sizes in ref \cite{Cong2013} is not quite uniform, though it could be even larger than that on sapphire \cite{Zhang2013h}. One common feature of aforementioned investigations is growth occurs on bare substrates and no catalyst or seeding promoter is employed. While there is another synthesis approach \cite{Lee2013,Ling2014} exclusively focusing on using seeding promoter. \citeauthor{Lee2013} obtained \ce{MoS2} and \ce{WS2} FL on PTAS-treated substrates (\ce{C24H12K4O8}), where tiny ($\sim 200$ nm) seeds were found under AFM. More recently, \citeauthor{Ling2014} systemically investigated the role of seeding promoter in facilitating the nucleation of \ce{MoS2} monolayers.\cite{Ling2014} It was found various aromatic molecules are effective yet inorganic nanoparticles are not. The mechanism of thin film growth depends on the surface energy and chemical potentials of the deposited layers and their substrates. Layer growth is preferred when surface adhesive force is stronger than adatoms cohesive force. Seeding promoter probably lowered the surface energy by wetting, thus provided heterogenous nucleation sites. Continuous \ce{MoS2} monolayer is obtained by evaporating \ce{F16CuPc} of 2 \AA on desired receiving substrates. In each growth, about $10^{-10}$ mol PTAS was applied onto \ce{SiO2}-Si, which was rendered hydrophilic and gentle gas blow to distribute the solution evenly.

In short summary, this thesis stresses that there are several key factors in successfully synthesizing \ce{WS2} FL:
\begin{itemize}
\item \ce{WO3} powder size distribution and absolute amount (0.69 g \ce{WO3} in 10 mL acetone or IPA, inspired by the \ce{MoO3} nanoribbons usage)
\item Sulfur amount and heating method to ensure a constant sulfur-rich environment
\item \ce{WO3} powder source to substrate distance, which is coupled to pressure and carrier gas flow in determining the transport (atmospheric, 3-5 mm)
\item temperature ramping and growth time (750-800 \si{\degreeCelsius}, 3-10 min)
\item 300 nm-\ce{SiO2}-Si substrate to make OM flakes identification easier
\end{itemize}

% CVD TMDC
\begin{landscape}
\begin{table}[htb]
\centering
\caption{TMDC FL methods summary}\label{tab:tmsgrowth}
{\footnotesize
\begin{tabular}{lp{2.5in}p{4.5in}}
\toprule
TMDC  &  precursor & growth condition (default temperature unit \si{\degreeCelsius}) \\
\midrule
\ce{MoS2} films \cite{Lee1994,Endler1999} & \ce{MoCl5}, \ce{H2S} & 1 kPa, temperature: 400-550 \si{\degreeCelsius}, 100/10/2.5 sccm for Ar, \ce{H2S} and \ce{MoCl5} flow\\
\addlinespace[0.5em]
\ce{MoS2} FL \cite{Zhan2012} & 1-5 nm Mo films on \ce{SiO2}, Sulfur & purging, RT-550@30 min, 550-750@90 min and hold for 10 min. Mo coating on Si did not work.\\
\ce{MoS2} FL \cite{Lin2012,Wang2013} & 4 nm \ce{MoO3} coating on sapphire  & reduced to \ce{MoO2} in \ce{H2} and Ar at 500 \si{\degreeCelsius}, sulfurization at 850-1000 \si{\degreeCelsius} \\
\addlinespace[0.5em]
\ce{MoS2} FL \cite{Liu2012a} & \ce{(NH4)2MoS4} in DMF solution transport by Ar bubbler or dip-coating onto subs &  annealing under Ar or Ar + Sulfur, total pressure 0.2-2 Torr, \\
 \addlinespace[0.5em]
\ce{MoS2} FL \cite{Wu2013} & \ce{MoS2} powder & Ar flow, 900 \si{\degreeCelsius} heating, pressure 20 Torr, 650 \si{\degreeCelsius} growth\\
 \addlinespace[0.5em]
\ce{MoS2} FL \cite{Mann2013,Najmaei2013,Ji2013} Rice & \ce{MoO3} powders or ribbons, Sulfur & Ar flow, 530-850 \si{\degreeCelsius}, total pressure 0.2-2 Torr, 5-30 min, mica or \ce{SiO2}-Si\\
 \addlinespace[0.5em]
\ce{MoS2} FL \cite{Lee2012b,Ling2014} & 18mg \ce{MoO3} powders, Sulfur,various seeding promoter on sub & 5sccm Ar, 650 \si{\degreeCelsius}, 3 min growth, atmospheric pressure, quick cooling\\

 \midrule
\ce{WS2} films\cite{Ballif1999,Brunken2008} & sputtering \ce{WS_{3+x}} on 10 nm Ni  & annealing under Ar for 1 h at 850 \si{\degreeCelsius} \\
\addlinespace[0.5em]
\ce{WS2} FL \cite{Berkdemir2013} & $\sim$1nm \ce{WO3} coating on 285 nm \ce{SiO2}-Si, 500 mg sulfur & 800 \si{\degreeCelsius} for 30 min, 100 sccm Ar, atmospheric pressure in \cite{Gutierrez2012} and 450 mTorr in \cite{Elias2013}. triangular flakes obtained\\
\addlinespace[0.5em]
\ce{WS2} ML \cite{Cong2013} & 1 mg \ce{WO3} powder on \ce{SiO2}-Si covered by another sub, $d\sim3$ mm, sulfur & 750 \si{\degreeCelsius}, slow heating, hold for 5 min, one-end sealed inner tube, 100 sccm Ar\\
\addlinespace[0.5em]
\ce{WS2} ML \cite{Zhang2013h} & \ce{WO3} powders, sulfur (separate heating) & 900 \si{\degreeCelsius}, sapphire subs, 225 mTorr, Ar 80 sccm and \ce{H2} 10 sccm, growth time 60 min, adjusting precursor and sapphire distance changing the coverage of \ce{WS2}, tube diameter: 1 in. 55 $\mu$m triangular flakes\\
\addlinespace[0.5em]
\ce{WS2} ML \cite{Peimyoo2013} & 1 mg \ce{WO3} powders, sulfur  & \ce{SiO2}/Si subs. Recipe A: 200 mg S,RT-550, Sulfur begin to melt, 550-800 \si{\degreeCelsius}, 5K/min, hold 10mins, 200 sccm Ar. Recipe B: sulfur separated heated at 250 \si{\degreeCelsius}. Total pressure: maybe atmospheric, tube diameter: 2 in. 5 $\mu$m triangular flakes \\
\ce{WS2} FL \cite{Lee2013}  & 1g \ce{WO3} powders, sulfur, \ce{SiO2}-Si subs treated with PTAS \ce{C24H12K4O8} and gentle gas blow & substrate facing down, APCVD, 800 \si{\degreeCelsius}, 5 min, 5 sccm Ar, fast heating. \\
\addlinespace[0.5em]
\ce{WS2} films \cite{Shanmugam2012a}   & 20 nm W on \ce{SiO2}-Si, sulfur & 750 \si{\degreeCelsius}, 200 sccm Ar, 1Torr. Annealing at 1000 \si{\degreeCelsius}, 25 nm thick \ce{WS2} film obtained \\
\bottomrule
\end{tabular}
}
\end{table}
\end{landscape}

\subsection{Raman on bulk and FL TMDC}

This section will discuss previous studies using Raman to characterize \ce{MS2} films, nanotubes, and FL structures. In each morphology, one or two key points will be highlighted for study in this thesis. The previous Raman efforts on \ce{MS2} film, especially the one using resonant conditions, provided critical insight into the electronic structure and lattice dynamics of these TMDC layered materials. Next a comparison of Raman knowledge between TMDC nanotubes and FL structures is given. For \ce{WS2} NTs, focus is on the asymmetry of $A_{1g}$ mode and its origin. And for \ce{WS2} FL structures, layer number dependent fingerprint is summarized, and some attempts on analysis the resonant Raman profiles of \ce{WS2} are made.

It is worth noting that Raman technique proves to be extreme useful in characterizing CNT and graphene, i.e., tube diameter by assigning the RBMs (radial breathing modes) and G peaks position.\cite{Bonaccorso2013} And Raman spectroscopy also qualifies as an excellent tool to monitor tensile features of TMDC in both 2D and tubular forms.\cite{Tang2013} 

Raman spectra arise from the inelastic light scattering of optical phonons. In back scattering geometry, the phonon wave-vector stands as $q = 4\pi\frac{n}{\lambda}$. The refractive index $\tilde{n}$ of \ce{WS2} at 532 nm is about $4.726 - 0.737i$. Compared to the size of Brillouin zone ($\pi/a = 10^{10}m^{-1}$), resonant Raman could or could not probe phonon at $M$ point.

Before discussing the Raman spectra features, the author briefly recapitulate some symmetry notations and vibration modes. \ce{MoS2} is used as an example, and those definitions apply to \ce{WS2} as well. Hexagonal \ce{MoS2} belongs to space group $D_{6h}^4$, and the repeat unit in $c$ axis contains two layers, where sulfur atoms in one layer are directly above the molybdenum atoms in adjacent layers, which is often referred as 2H-\ce{MoS2}. Group theory predicts two infrared- and four Raman-active modes for 2H-\ce{MoS2}, which are mutually exclusive when the center of inversion is present. First it should be emphasized that in few layer structures, \ce{MoS2} with odd layers belong to different space group from that of even layers. Bulk MoS$_2$ and 2L-MoS$_2$ belong to the space group P6$_3$/$_{mmc}$ (point group D$_{6h}$). There are 18 normal vibration modes. The factor group of bulk and 2L-MoS$_2$ at $\vec{\Gamma}$ is D$_{6h}$. The atoms site groups are a subgroup of the crystal factor group. The correlation of the Mo site group D$_{3h}$, S site group C$_{3v}$, and factor group D$_{6h}$ allows one to derive the following irreducible representations for the 18 normal vibration modes at $\vec{\Gamma}$: $\vec{\Gamma}$= $A_{1g}+2A_{2u}+2B_{2g}+B_{1u}+E_{1g}+2E_{1u}+2E_{2g}+E_{2u}$, where $A_{2u}$ and $E_{1u}$ are translational acoustic modes, $A_{1g}$, $E_{1g}$ and $E_{2g}$ are Raman active, $A_{2u}$ and $E_{1u}$ are infrared (IR) active. As a contrast, 1L-MoS$_2$ has $D_{3h}$ symmetry with three atoms per unit cell. The irreducible representation of $D_{3h}$ gives: $\vec{\Gamma}$= $2A_2^{''}$+$A_1^{'}$+$2E^{'}$+$E^{''}$, with $A_2^{''}$ and $E^{'}$ acoustic modes, $A_2^{''}$ IR active, $A_1^{'}$ and $E^{''}$ Raman active, and the other $E^{'}$ both Raman and IR active (See Fig.~\ref{fig:ws2ramsch}). NL-MoS$_2$ has 9N-3 optical modes: 3N-1 are vibrations along the c axis, and 3N-1 are doubly degenerate in-plane vibrations. For rigid-layer vibrations, there are N-1 layer breathing modes (LBMs) along the c axis, and N-1 doubly degenerate shear modes perpendicular to it. When N is even, there are 0 Raman active LBMs and $\frac{N}{2}$ doubly degenerate shear modes. When N is odd, there are $\frac{N-1}{2}$ LBMs and N-1 doubly degenerate shear modes.\cite{Wieting1971,Zhang2013e} This discussion can be visualized in Table~\ref{tab:tmslattice}.

% irreducible representation
\begin{table}[htb]
\centering
\caption{Lattices vibration of \ce{MS2}, adopted from ref\cite{Molina-Sanchez2011}}\label{tab:tmslattice}
\begin{tabular}{lcccc}
\toprule
 $D_{6h}$   & $D_{3h}$ & Character &  Direction & Atoms  \\
\midrule
$A_{1g}$    &  $A_1^{'}$   & Raman     & (out of plane)  & S  \\
$E_{2g}^2$  &          &           & (in plane)      & M + S  \\
$E_{2g}^1$  &  $E'$    &           & (in plane)      & M + S  \\
$E_{1g}$    &  $E''$    &           & (in plane)      & S  \\
\midrule
$A_{2u}$    &  $A_2''$  & Infrared  & (out of plane)  & M + S  \\
$E_{1u}$    &          &           & (in plane)      & M + S  \\
\midrule
$A_{2u}$    &  $A_2^{''}$   & Acoustic  & (out of plane)  & M + S  \\
$E_{1u}$    &          &           &       &    \\
\midrule
$B_{2g}^2$  &          & Inactive  & (out of plane)  & M + S  \\
$B_{2g}^1$  &          &           & (out of plane)  & M + S  \\
$B_{1u}$    &          &           & (out of plane)  & S  \\
$E_{2u}$    &          &           & (in plane)      & S  \\
\bottomrule
\end{tabular}
\end{table}

The $E$ type phonon branches correspond to in-plane normal modes, while the $A$ type phonons result from out-of-plane vibrations. $A_{1g}$ mode is an out-of-plane vibration involving only the S atoms while the $E_{2g}^1$ mode involves in-plane displacement of transition metal and S atoms. The $E_{2g}^2$ mode is a shear mode corresponding to the vibration of two rigid layers against
each other and appears at very low frequencies ($<50 cm^{-1}$ \cite{Zhang2013e}). The $E_{1g}$ mode, which is an in-plane vibration of only the S atoms, is forbidden in the backscattering Raman configuration. In 2H-type TMDC, the $A_{1g}$ mode is more sensitive to electrostatic doping, while $E_{2g}^1$ mode is more sensitive to strain, in which the FWHM of the peaks are indicator of external force quantity.\cite{Zhao2013}

\begin{figure}[htb]
\centering
\includegraphics[width=0.7\textwidth]{ws2_ramsch}
\caption[\ce{MS2} vibration symmetry]{\ce{MS2} vibration symmetry in bulk and monalyer, adopted from REF\cite{Ghorbani-asl}}
\label{fig:ws2ramsch}
\end{figure}

Lattice vibration of natural \ce{MoS2} crystal was studied by \citeauthor{Wieting1971} using infrared and Raman spectroscopy.\cite{Wieting1971} It was found the $E_{1u}$ IR mode and one $E_{2g}$ Raman mode are nearly degenerate in energy. 15 optical modes are allowed assuming 6 atoms in primitive cell. Refractive indices from reflectivity measurement were $n_0$= 3.9, $n_e$ = 2.5. \citeauthor{Stacy1985} studied \ce{MoS2} and \ce{WS2} Raman spectra using lasing energy close to the absorption edges.\cite{Stacy1985} Second order scattering from phonon with nonzero momentum is used to explain the rich Raman spectra. \citeauthor{Sourisseau1991} investigated the resonant Raman profiles in 2H-\ce{WS2} using ten different excitation wavelengths.\cite{Sourisseau1991} Dramatic intensity variation at 352 \si{cm^{-1}} was observed, which is assume to be of two-phonon signal nature, and corresponds to an overtone or combination band of phonons with non-zero momenta contributing to indirect gap absorption edge. \citeauthor{Sourisseau1991} assigned this phonon with non-zero momenta as $LA(K_5)$ type. The enhancement of the total Raman cross section at excitonic resonance in which excitons serve as the intermediate state is stronger compared to that of interband resonance. The strong enhancement at excitonic resonance is attributed to the characteristics of excitons in layered materials such as large binding energy, enhanced oscillator strength, and small damping constant.\cite{Zhao2013} \citeauthor{Chung1998} grew \ce{WS2} film using \ce{W(CO)6} and \ce{H2S} precursor.\cite{Chung1998} Raman spectra ($\lambda=632 nm$) on films with non-parallel orientation revealed the presence of shoulder mode under $A_{1g}$, which is assigned to LA and TA phonon coupling. This coupling process stems from disorder-activated zone boundary phonons. A further discussion on this non-symmetric feature of $A_{1g}$ mode will be continued in section~\ref{sec:ntram}.

% WS2 Raman assignments
\begin{table}
  \centering
  \caption{\ce{WS2} symmetry assignment}  \label{tbl:ws2raman}
  \begin{tabular}{ccccc}
    \toprule
    &&\multicolumn{3}{c}{Raman Shift (\si{cm^{-1}})}\\
    \cmidrule(l){3-5}
    Symmetry                & Atoms & \ce{WS2} ML\cite{Cong2013}  & \ce{WS2} NT \cite{JMR7990865}  & \ce{WS2} bulk \cite{Sourisseau1991} \\
    \midrule
          $E_{2g}^2(\Gamma)$ &      & 27.5\textsuperscript{\emph{a}}&             &  27.4    \\
    $LA(M)-E_{2g}^2(\Gamma)$ &      & 148.3                        &              &    \\
       TBD                   &      &                              & 153          &      \\
         $E_{2g}^1(M)-LA(M)$ &      &                              & 172          & 173  \\
    LA(M)                    &      & 174.8                        & 172          &       \\
    LA(K)                    &      & 192.4                        &              &  193 \\
    $LA(M)+E_{2g}^2(\Gamma)$ &      & 203                          &              &     \\
    $LA(K)+E_{2g}^2(\Gamma)$ &      & 213.9                        &              &  212  \\
    $A_{1g}(M)-LA(M)$        &      & 230.9                        & 230          &  233  \\
    $2LA(M)-3E_{2g}^2(\Gamma)$ &    & 264.2                        & 262          &  267  \\
    $2LA(M)-2E_{2g}^2(\Gamma)$ &    & 295.4                        & 294          &  297   \\
    $2LA(M)-E_{2g}^2(\Gamma)$ &     & 322.9                        &              &  325   \\
               $E_{2g}^1(M)$ &      & 343.1                        &              &      \\
    2LA(M)                   &      & 350.8                        & 350          &  352\\
          $E_{2g}^1(\Gamma)$ &      & 355.4                        & 350          &  356 \\
    $2LA(M)+2E_{2g}^2(M)$    &      &                              & 381          &  381   \\
     LA + TA \cite{Sourisseau1991} or $B_{1u}(\Gamma)$\cite{Staiger2012}  &      &       &   &  416 \\
          $A_{1g}(\Gamma)$   &      & 417.9                        & 416\textsuperscript{\emph{b}} &  421\\
               3LA(K)        &      & 577                          &              &      \\
       $ LA(M)+ A_{1g}(M)$   &      & 584                          & 581          &  585 \\
    4LA(M)                   &      & 704                          &              &  703\\
    \bottomrule
  \end{tabular}

  \textsuperscript{\emph{a}} Calculated from column values;
  \textsuperscript{\emph{b}} \citeauthor{JMR7990865} probably made incorrect assignment of 416 peak.\cite{JMR7990865}
\end{table}

Raman technique has also provided much insight into the few layer \ce{MS2}. \ce{MS2} layer numbers are readily identified by the wavenumber distance between $A_{1g}$ and $E_{2g}^1$ mode for \ce{MoS2}\cite{Buscema2013} and \ce{WS2}\cite{Berkdemir2013}. Yet due to the relative small shift of $A_{1g}$ mode and little shift of $E_{2g}^1$ mode in \ce{WS2} FL, the frequencies distance might not qualify as an unambiguous way to distinguish layer numbers. Yet the resonant Raman profile on \ce{WS2} exhibit unique features between the intensity of 2LA and $A_{1g}$ mode,\cite{Berkdemir2013,Zhao2013} which provide another routine to assure the monolayer presence. However, the author notice some discrepancy in de-convolution of \ce{WS2} resonant profile between 300 and 400 cm$^{-1}$, i.e., the presence of $E_{2g}^1(M)$ mode at about 344 cm$^{-1}$.\cite{Peimyoo2013,Cong2013,Berkdemir2013} Rigid assignments of this mode still requires further theoretical\cite{Ataca2012} and experimental efforts. In addition, \citeauthor{M2013} reported temperature dependent of 1L \ce{WS2}.\cite{M2013} When the temperature increase from 77 K to 623 K, A1g shift from 420 to 416.5 cm$^{-1}$. Interestingly in this report 2LA/$A_{1g}$ ratio (514 nm excitation) seems less than unit. The spectra were obtain from mechanically exfoliated \ce{WS2} lying on 300 nm \ce{SiO2}-Si substrate. Moreover, The $A_{1g}$ and $E_{2g}^1$ intensities ratio exhibit reverse behaviors under 532 and 632 nm excitation. This is caused by the different cross-section enhancement for a specific excitation condition. The A and B excitonic absorption in \ce{WS2} mainly arise from the $d_{xy}$ and $d_{x^2 - y^2}$ states to $d_{z^2}$ states of tungsten atoms. Thus, electrons excited by 633 nm laser have a character of tungsten $d_{z^2}$ orbitals aligned along the $c$ axis perpendicular to \ce{WS2} basal plane. Since $A_{1g}$ mode involves out of plane displacement along $c$ axis, $A_{1g}$ phonons could couple more strongly with $d_{z^2}$ states than that of $E_{2g}^1$ phonons. As a result, $A_{1g}$ mode is stronger than $E_{2g}^1$ mode at 633 nm resonance.\cite{Zhao2013} However, the reverse effect for 532 nm excitation could not be well explained using the above argument. This may be caused by electron-phonon coupling with other inter-band transition electrons.

\subsection{Raman on TMDC NTs}\label{sec:ntram}

\citeauthor{Dobardzic2005} calculated \ce{MoS2} SWNT phonon dispersion. The dependence of wavenumbers and their displacement on chirality and diameter were discussed. The calculation method enables studying lattice dynamics with NT diameter up to 50 nm. The chiral vector $(n_1, n_2)$ is defined within the molybdenum plane. Symmetry assignment is zigzag when $(n,0)$, armchair when $(n,n)$ and chiral when $(n_1, n_2), n_1>n_2$. \citeauthor{Dobardzic2006} theoretically presented Raman scattering of any polarization on SWNT of \ce{WS2} and their dependence on diameter (1-20 nm) and chiral angle. The author assigned 351 cm$^{-1}$ as $E_u$ for \ce{WS2} NT.\cite{Dobardzic2006} \citeauthor{Ghorbani-Asl2013} discuss the electronic and vibrational properties for large diameter \ce{WS2} NTs\cite{Ghorbani-Asl2013}. Single-walled NT is approximated by 1H monolayer and others by 2H bulk structure. It was found that large-diameter nanotubes can be approximated with layered systems as their properties should be nearly the same at the scale. Only hypothetical SWNTs, and possibly MWNTs with alternating layer compositions, may show direct band gaps. Slight mechanical deformation of the SWNTs would result in a change of the direct band gap back to the indirect one, located between $\Gamma$ and $K$ high-symmetry points, similarly to the monolayers. As for 2D materials, quantum confinement to single-walled tubes would result in direct band-gap semiconductors with $\Delta$ occurring at the $K$ point. Single-walled tubes exhibit slightly softer out-of-plane $A'$ and stronger in-plane $E'$ modes. Those results indicate that the weak interlayer interactions in MS$_2$ materials cannot be associated with the van der Waals interactions only, but most probably with Coulomb electrostatic interactions as well.

Raman signatures of \ce{WS2} nanotubes show distinct features to the spectra of their bulk counterpart. \citeauthor{JMR7990865} observed a new line at 152 cm$^{-1}$ in \ce{WS2} NT, which is absent in 2H-\ce{WS2}.\cite{JMR7990865} Another feature is a emerging shoulder on the low energy side of $A_{1g}$ mode at about 416 cm$^{-1}$. This has been attributed to a combination mode of LA + TA phonons from the $K$ point of Brillouin zone.\cite{Sourisseau1991} The shoulder mode associated with $A_{1g}$ is attributed to LA + TA. As pressure increase from 0 GPa to 18 GPa, these two bands, both shifting to higher wavenumbers, first separate and then recombine. It was assumed the compression mainly occurs in $c$ axis, so the stiffening of $A_{1g}$ is anticipated. A more prominent feature is the resonance profile broadening the shape of $E_{2g}^1$ mode, which is often assigned to 2LA mode. Yet there is different opinion on these assignments. \citeauthor{Molina-Sanchez2011} label the 350 \si{cm^{-1}} band as \ce{E_{1u}} instead of 2LA.\cite{Molina-Sanchez2011} And recent theoretical investigations suggest 416 \si{cm^{-1}} peak is inactive $B_{1u}$ mode\cite{Molina-Sanchez2011,Ataca2012}, which is the Davydov doublet with $A_{1g}$ mode. \citeauthor{Staiger2012} adopted these assignments in studying the resonance Raman profile of \ce{WS2} NTs.\cite{Staiger2012} It was found that
\begin{enumerate}
\item $B_{1u}$ mode arise from curvature and structural disorder
\item $B_{1u}/A_{1g}$ intensity ratio strongly depends on excitation, and exceeds unity when excitation energy less than 1.9 eV.
\item  An excitonic transition energy of NT is found have a local minimum at about 50 nm, (layer number probably \textgreater 10), and increase either way. Yet all below the bulk value.
\end{enumerate}

\citeauthor{Krause2009} also measured the resonant Raman on \ce{WS2} nanotubes and found a split within 420 \si{cm^{-1}} region, which is labelled as $D-A_{1g}$ mode in analog with the similar defect mode of graphene.\cite{Krause2009} This  $D-A_{1g}$ mode was found enhanced as diameter of \ce{WS2} NTs decrease. This thesis work will use $B_{1u}({\Gamma})$ mode to interpret this emerging line at about 416 \si{cm^{-1}} of \ce{WS2} NTs, and adopt the 350 \si{cm^{-1}} as 2LA(M). \citeauthor{Krause2009a} confirmed $B_{1u}$ mode arise from the inherent structure of \ce{WS2} nanomaterials instead of surface layer effect. It is also worth noting that $A_{1g}$ is stronger than 2LA under 632 nm yet weaker under 532 nm excitation, similar to previous discussion of FL scenarios.\cite{Krause2009a} This is in consistent with our observation of Raman spectra on the \ce{WS2}-\ce{WO3} structures, as discussed later. \citeauthor{Rafailov2005} estimated the orientation dependence of resonant raman on one MWNT \ce{WS2} attached to the cantilever tip of AFM.\cite{Rafailov2005} Antenna effect lead to optical transition occurring only for polarization parallel to nanotube axis. And therefore resonance Raman intensity of SWNT varies as nanotube orientation. This dependence may provide a routine to distinguish different chiral NTs. Polarized Raman spectra (632 nm excitation) is obtained, showing $A_{1g}$ and $E_{2g}$ sharing the same polarization behavior. \citeauthor{Virsek2007} investigated the Raman scattering ($\lambda=632$ nm) of \ce{WS2} NTs.\cite{Virsek2007} The silicon peak at 520 cm$^{-1}$ is used for calibration. Up-shift of $A_{1g}$ and $E_{2g}$ modes (i.e. 420 to 423 cm$^{-1}$ at $A_{1g}$ mode) were observed, which is attributed to the strain in 3R stacking layers.

As a useful reference, the lattices dimensions are summarized in Table~\ref{tab:ms2lattice},
%lattice
\begin{table}[htb]
\centering
\caption{Lattices dimension of \ce{MS2}}\label{tab:ms2lattice}
\begin{tabular}{lccr}
\toprule
         &  & 2H-\ce{MoS2}\cite{Coehoorn1987,Ataca2012} & 2H-\ce{WS2}\cite{Albe2002,Schutte1987} \\
\midrule
Lattice constant & a(\AA) & 3.1604 & 3.171 \\
                 & c(\AA) & 12.295 & 12.359 \\
Within \ce{MS2} layer & M-3S (\AA)& 2.37  & 2.405   \\
                      & S-1S (\AA)& 3.11  & 3.14   \\
Between \ce{MS2} layer& S-3S (\AA)&   & 3.53   \\
\bottomrule
\end{tabular}
\end{table}


\section{Experimental}

The \ce{WO3}-\ce{WS2} core-shell nanowires were fabricated using a sulfurization process based on the \ce{WO3} NWs prepared from the seeded growth method in Chapter~3, Sec.~\ref{sec:sgfg}. For the sake of completeness, schematic layout of the two-step growth is illustrated in Fig.~\ref{fig:ch5grow}, including the seeded growth and subsequent sulfurization process.
% cvd NW growth
\begin{figure}[htb]
\centering
\includegraphics[width=0.5\textwidth]{cs_growth_lbd}
\caption[Schematic diagrams of two-step synthesis process]{Schematic diagrams of (a) seeded growth of \ce{WO3} NW and (b) sulfurization process. }
\label{fig:ch5grow}
\end{figure}
In a representative seeded growth (Fig.~\ref{fig:ch5grow}a), 4N5 or 5N tungsten source powders as defined in Table~\ref{tab:powder} were positioned in the upstream end of a quartz boat, and downstream about 2.5 inch the substrate was stationed. Additional tungsten seed powders (3N) were distributed evenly onto the substrate. After pumping down, 1 sccm oxygen and 10 sccm Ar were admitted into the chamber, respectively. The heating temperature was ramped up to 1000 \si{\degreeCelsius} in 30 min and lasted for 240 min. Subsequently the apparatus was allowed to naturally cool down to room temperature. 

\ce{WO3}-\ce{WS2} core-shell NWs were synthesized using direct sulfurization of \ce{WO3} NWs. As depicted in Fig.~\ref{fig:ch5grow}b, the as-grown \ce{WO3} NWs were loaded into the center of heating furnace, and $\sim200$ mg sulfur (Alfa Aesar 10785, 99.5\%) was positioned just outside the upstream edge of furnace, where the maximum temperature was about 240 \si{\degreeCelsius}. After pumping down, the reaction chamber was flushed two times to expel residual air. A cold trap filled with liquid nitrogen in downstream was used to collect possible sulfur condensation. Then, the furnace was heated to 750 \si{\degreeCelsius} in 30 min, held for 15 min, and allowed to naturally cool down to room temperature. During entire growth process, 30 sccm Ar was used as carrier gas.

\section{Results and Discussion}
\subsection{Ensemble Measurements on Core-Shell Nanowires Array}

The morphology of \ce{WO3} NWs in micron scale almost stayed the same after sulfurization, as shown in the inset SEM imaging of Fig.~\ref{fig:ch5ws2sem}. The length of individual nanowire were up to 15 $\mu$m, and the diameter varied from about 40 to 200 nm. The surface of the NWs were rather smooth without other attachments. The NWs array grew slightly larger than the W seeds ($\sim$12 $\mu$m), mostly due to the elongation of NW.
\begin{figure}[htb]
\centering
\includegraphics[width=0.7\textwidth]{ws2_sem_edx_labeled}
\caption[Morphology and composition analysis on the core-shell structures]{Morphology and composition analysis on the as-synthesized \ce{WO3}-\ce{WS2} core-shell structures. (a) Low magnification and (b) high magnification SEM images showing the NWs morphologies after sulfurization. (c) EDX spectroscopy revealing the presence of sulfur element after sulfurization.}
\label{fig:ch5ws2sem}
\end{figure}
The EDX spectrum in Fig.~\ref{fig:ch5ws2sem} revealed the presence of sulfur element, indicating the effectiveness of sulfurization. Compared to the XRD pattern on seeded growth \ce{WO3} sample, post-sulfurization pattern (Fig.~\ref{fig:ch5ws2xrd}) show a much reduced \ce{WO3} intensity with W phase becoming relatively prominent. The peaks at 23.44$^{\circ}$ and 24.36$^{\circ}$ were indexed to \ce{WO3} (020) and (200) reflection, respectively. This result as well as the SEM imaging indicated \ce{WO3} NWs preserved some long range order after the sulfurization process of 15 min at 750 \si{\degreeCelsius}. 
%fig xrd ws2 
\begin{figure}[htb]
\centering
\includegraphics[width=0.7\textwidth]{ws2_cs_xrd}
\caption[X-ray diffraction spectra before and after sulfurization]{X-ray diffraction spectra before and after sulfurization. The reference spectra for $c$-W, 2H-\ce{WS2} and $m$-\ce{WO3} are included with major peaks labeled.}
\label{fig:ch5ws2xrd}
\end{figure}
This observation can also be used to account for the extreme small peak at 13.87$^{\circ}$ in the post-sulfurization spectrum. In comparison to the reference 2H-\ce{WS2} (002) peak (14.37$^{\circ}$, ICDD PDF 04-003-4478), the redshift of specimen peak here indicated an increase of lattice spacing along $c$ axis (from 6.16 \si{\angstrom} to 6.38 \si{\angstrom}). The FWHM of this (002) peak was about 1.1$^{\circ}$, giving an estimation on the crystalline size of 7.5 nm according to Scherrer equation. 

Owing to this small spacing, TEM was primarily utilized to observe the detailed morphology of \ce{WO3}-\ce{WS2} core-shell NWs. Fig.~\ref{fig:ch5ws2tem1}c and d show HRTEM imaging on typical tip and body region of the core-shell NW, respectively. A comparison with the pre-sulfurization specimen (Fig.~\ref{fig:woseedtem2} on page~\pageref{fig:woseedtem2}) readily revealed the formation of core-shell morphology after sulfurization. The measured wall spacing of 6.5 \si{\angstrom} appeared slightly larger than bulk 2H-\ce{WS2} value of 6.16 \si{\angstrom}, but matched well with the XRD result in Fig.~\ref{fig:ch5ws2xrd}. The sharp contrast in core region also proved the lattice structure of \ce{WO3} was not seriously distorted upon sulfurization.
% tem core shell layer
\begin{figure}[htb]
\centering
\includegraphics[width=0.8\textwidth]{ws2_tem_labeled}
\caption[HRTEM imaging before and after sulfurization]{HRTEM imaging before and after sulfurization illustrating the presence of core-shell structure. (a) Low magnification and (b) HRTEM images on \ce{WO3} NW specimen from seeded growth;  HRTEM images on the core-shell nanowire of (c) tip region and (d) body region showing wall number variation. The contrast between (b) and (d) readily identify the formation of core-shell structure. }
\label{fig:ch5ws2tem1}
\end{figure}

After examining dozens of core-shell NWs, this study found the \ce{WS2} wall number at NW tip was usually the largest, reaching more than ten, and the wall number decreased to one along the NW body; but there also existed some wall number fluctuation and shape modification. Fig.~\ref{fig:ch5ws2tem3} show a smooth transition of \ce{WS2} wall number on the tip region of one NW. The wall number was about 7 on the tip corner, gradually decreased from 4 to 3 along the NW body, and stopped at 2 walls. Closer observation indicated the wall spacing was not quite uniform in the tip area, probably due to the complex curvature and bending. \ce{WS2} fullerene also show similar structures. Since the closing of the tip region could be of great complexity and the induced strain could modify Raman signal considerably, the TEM-Raman integrated measurement in this study focused on the body region of the core-shell NWs with smooth transition wall number.
% coreshell tem
\begin{figure}[htb]
\centering
\includegraphics[width=0.7\textwidth]{cs_tem_g1}
\caption[HRTEM imaging on the tip area of one core-shell nanowire]{HRTEM imaging on the tip area of one core-shell nanowire.}
\label{fig:ch5ws2tem3}
\end{figure}
As illustrated in Fig.~\ref{fig:ch5ws2tem4}, complicated \ce{WS2} encapsulation was observed on one tapered core-shell NW. This tapering probably arose from the sulfurization process but could be also due to the parent \ce{WO3} NW. The diameter varied from about 200 nm to 15 nm, and the \ce{WS2} wall number also changed in a nonlinear manner. On some turning region, the \ce{WS2} seemed penetrating into the body of primary \ce{WO3} NW. 
% coreshell tem
\begin{figure}[htb]
\centering
\includegraphics[width=0.7\textwidth]{cs_tem_g3_lbd}
\caption[HRTEM imaging on one tapered core-shell nanowire]{HRTEM imaging on one tapered core-shell nanowire showing fluctuation of wall numbers.}
\label{fig:ch5ws2tem4}
\end{figure}

A systematic study on the sulfurization must be performed to understand the growth dynamics. The growth mechanism of \ce{WS2} nanotubes from oxides has been thoroughly studied,\cite{Feldman1998} and it was generally agreed that the sulfurization of \ce{WO3} was a diffusion limited reaction.\cite{Feldman1996} It was also suggested that the core \ce{WO3} NWs became non-stoichiometric phases upon sulfurization.\cite{Feldman1996,ZAK2009} This study recognized this possibility but also considered two factors different from previous studies: one was the usage of sulfur as precursor instead of \ce{H2} and/or \ce{H2S} as in previous reports, which accelerated the reduction of \ce{WO3} significantly due to the formation of \ce{H2O}; the other was the short reaction time and light degree of sulfurization, which probably would not distort the \ce{WO3} lattice significantly, as evidenced by the XRD and HRTEM analysis in this section. The sulfurization dynamic was then proposed as following: The tips of the NWs exposed directly to the sulfur vapors resulting in a high sulfurization rate. On the other hand,  sulfurizaiton along the NWs is limited by the diffusion of the sulfur vapors through the dense NWs array. Therefore, the \ce{WS2} wall number is not uniform along the as-grown core-shell NWs.

% coreshell tem
\begin{figure}[htb]
\centering
\includegraphics[width=0.7\textwidth]{core-shell_void_growth}
\caption[HRTEM imaging on one nanowire showing consumed core]{HRTEM imaging on one core-shell nanowire showing consumed core.}
\label{fig:ch5ws2tem2}
\end{figure}
No obviously consumed core region was observed among most of the core-shell NWs examined under TEM. An exception was shown in Fig.~\ref{fig:ch5ws2tem2}. Note the rectangular area show abrupt termination of one \ce{WS2} wall, indicating a formation of dislocation.


\subsection{An Integrated TEM-Raman Study on Individual Nanowire}\label{tms:raman} 

To perform TEM-Raman integrated study, the core-shell specimen was first dispersed into ethanol solution; and then the solution was dipped onto polydimethylsiloxane (PDMS) support. A home-built micromanipulator with probe tip (tungsten, 100nm point radius, Micromaipulator) was used to pick up the core-shell NWs and transfer them onto TEM grids (Lacey C, 300 mesh, Cu, 01895-F Ted Pella, Inc) under 100X objective. The steps were visualized in Fig.~\ref{fig:ch5ws2trans}. 
% ws2 transfer
\begin{figure}[htb]
\centering
\includegraphics[width=0.5\textwidth]{cs_NW_transfer}
\caption[Core-shell nanowires transfer steps]{Core-shell nanowires transfer steps. (a) tungsten tip under X100 objective, (b) NWs dispersed onto PDMS, (c) one NW picked up using tungsten tip and (d) tranferred NW on TEM grid with two tips indicated by the arrows.}
\label{fig:ch5ws2trans}
\end{figure}
The transferred NWs were first examined using TEM to identify the \ce{WS2} wall configuration and the associated geometrical orientation. Then, the core-shell NWs on TEM grids were probed under Raman in a location-resolved manner. Only those core-shell NWs exhibiting relative smooth \ce{WS2} wall number\footnote{wall number and layer number were used interchangeably in this work, but note that 1 wall is not equal to 1 layer due to the tubular structure} transition were used in subsequent Raman measurement. To perform TEM and Raman analysis on individual NWs, this study placed one NW only in each cell of the TEM grid (65 $\times$ 65 $\mu$m). And the low magnification TEM images and optical images were compared to map the NW location and orientation. The carbon film patterns served as a unique background to resolve possible ambiguity. An example was given in Fig.~\ref{fig:ch5ws2map}. 
% ws2 map
\begin{figure}[htb]
\centering
\includegraphics[width=0.6\textwidth]{cs_nw_map}
\caption[Core-shell nanowires mapping]{Core-shell nanowires mapping. Optical image of one NW on the lacey  carbon TEM grid under (a) x100 magnification and (c) digital zoom-in view with two end indicated by the arrows, TEM images of (b) low magnification and (d) high magnification acquired on the same NW specimen.}
\label{fig:ch5ws2map}
\end{figure}

Fig.~\ref{fig:ch5ws2ram}a-d depicted typical morphologies of the \ce{WO3}-\ce{WS2} NWs in the TEM-Raman integrated study. The diameter of these core-shell NWs was about 100 nm, and the length about 10 $\mu$m, allowing for up to five distinctive Raman scattering sites. Fig.~\ref{fig:ch5ws2ram}e show the associated Raman spectra acquired on the core-shell NWs using 532 nm excitation wavelength. Two strong peaks were observed between 300 and 450 \si{cm^{-1}}. The resonant nature\cite{Stacy1985} of Raman shift was recognized by the broad band at about 350 \si{cm^{-1}}.  Recent theoretical\cite{Molina-Sanchez2011} and experimental\cite{Staiger2012} studies on 2D and tubular \ce{WS2} agreed that this band consisted of first order mode $E_{2g}^1$ and second order mode $2LA(M)$, although there was still discrepancy on the exact symmetry assignment, i.e., the presence of $E_{2g}^1(M)$ mode.\cite{Berkdemir2013,Peimyoo2013}. The other peak at about 420 \si{cm^{-1}} was assigned to $A_{1g}$ mode in both bulk\cite{Sekine1980} and few layer \ce{WS2}. Tubular \ce{WS2} structure can be viewed as rolling up the planar \ce{WS2} sheet, and new Raman features arose subsequently, e.g., a shoulder mode emerged on the low frequency side of $A_{1g}$ mode. Theoretical investigation suggested assigning this mode as $B_{1u}$, the Davydov doublet of $A_{1g}$ mode.\cite{Ataca2012} And experimental study confirmed this $B_{1u}$ mode arose from structural disorder of \ce{WS2} layers, and the intensity ratio between $B_{1u}$ and $A_{1g}$ strongly depended on excitation wavelength.\cite{Staiger2012} This study observed $B_{1u}$ mode under 532 nm was much less prominent than that of 632 nm, in consistent with previous reports.\cite{Krause2009,Krause2009a} 
% fig Raman TEM
\begin{figure}[htb]
\centering
\includegraphics[width=0.8\textwidth]{cs_temram}
\caption[TEM-Raman integrated characterization on core-shell NWs]{TEM-Raman integrated characterization on core-shell NWs. (a)-(d) HRTEM images on four NWs showing the \ce{WS2} wall number variation from one to about ten. (e) Raman spectra acquired from the same core-shell NWs supported on TEM grids, indicating a sharp contrast between the in-plane and out-of-plane vibrations. Note that the Raman spectra primarily arise from the outer \ce{WS2} shell, and only weak Raman shift at about 800 \si{cm^{-1}} were observed at some bare core region (not shown here).}
\label{fig:ch5ws2ram}
\end{figure}

Rigid analysis of the Raman spectra must be performed to reach solid conclusion. Hence, multi-peak Lorentzian fitting was employed on the spectra in Fig.~\ref{fig:ch5ws2ram}e, and the result for one wall \ce{WS2} core-shell NW was displayed in Fig.~\ref{fig:ch5ws2ml}. The de-convolution peaks compared favorably with original Raman spectrum, and similar operation were used for other three spectra. Fig.~\ref{fig:ch5ws2pr} show the wall number dependence on three group of intensity ratios extracted from apparent value, de-convoluted height, and integrated intensity, respectively. Similar trend was found among all three set of data, though the integrated intensity appeared more sensitive towards wall number variation. For a qualitative purpose, the apparent intensity ratio $I_{E/A}$ between in-plane E band ($E_{2g}^1+2LA$) and out-of-plane A band ($A_{1g}+B_{1u}$) was used as a quick criterion to distinguish \ce{WS2} few wall in this study. 
% fig Raman TEM
\begin{figure}[htb]
\centering
\subfloat[]{\label{fig:ch5ws2ml}\includegraphics[width=0.45\textwidth]{ws2_1L_fitting}}\hspace{0.04\textwidth}
\subfloat[]{\label{fig:ch5ws2pr}\includegraphics[width=0.45\textwidth]{ws2_peak_ratio}}
\caption[Multi-peak Lorentzian fitting on core-shell NWs]{(a) Multi-peak Lorentzian fitting on core-shell NWs, and (b) wall number dependence on averaged intensity ratio between the in-plane and out-of-plane Raman bands.}
\label{fig:ch5ws2prl}
\end{figure}
$I_{E/A}$ was then extracted to correlate with wall number variation on tens of core-shell NWs. As shown in Fig.~\ref{fig:ch5ws2ramap}, the mapping of \ce{WS2} wall dependent $I_{E/A}$ almost followed the same trend in Fig.~\ref{fig:ch5ws2pr} with some variation attributed to the core-shell NW diameter difference.
\begin{figure}[htb]
\centering
\includegraphics[width=0.6\textwidth]{ws2_ratio_map}
\caption[Mapping on the \ce{WS2} wall dependent intensity ratio]{Mapping on the \ce{WS2} wall dependent intensity ratio. The Raman spectra were acquired on individual NW suspended on 300 mesh lacey C TEM grid.}
\label{fig:ch5ws2ramap}
\end{figure}
The intensity ratio $I_{E/A}$ decreased monotonically with the increase of \ce{WS2} wall number, a similar trend as observed on two-dimensional (2D) few layer \ce{WS2},\cite{Berkdemir2013} and the exact value of $I_{E/A}$ show modest difference with 2D sheet, which could be attributed to the suspended configuration of current core-shell NWs specimen and the presence of \ce{WO3} core. Furthermore, these results indicated that $I_{E/A} > 2$ for single-walled \ce{WS2} core-shell NW, and when the \ce{WS2} wall number was about 3, $I_{E/A}=1$. These two values served as a quick determination of \ce{WS2} wall number when the NW was suspended. 

Before using this observation in more general scenarios, one should evaluate the influence of \ce{WO3} and other dielectric substrates. It has been demonstrated that the optical interference from dielectric environment significantly modified the intensity of Raman scattering from layered \ce{MoS2}, regardless lattice vibration modes.\cite{Li2012} In this work, the author calculated the reflectivity of three different stacking scenarios to estimate the influence of \ce{WO3} and insulating substrate, as shown in Fig.~\ref{fig:ch5ws2stk}. It was concluded that under 532 nm excitation, the \ce{WO3} core show little influence on the Raman intensity when \ce{WS2} nanotube was suspended in air; whereas the \ce{SiO2}/Si substrates dramatically enhanced the Raman scattering of the core-shell NWs. It was further predicted that current Raman spectra on the core-shell NWs can be used to estimate \ce{WS2} nanotube wall numbers. 

\begin{figure}[htb]
\centering
\subfloat[]{\label{fig:stk1}\includegraphics[width=0.30\textwidth]{stacking1}}\hspace{0.03\textwidth}
\subfloat[]{\label{fig:stk2}\includegraphics[width=0.30\textwidth]{stacking2}}\hspace{0.03\textwidth}
\subfloat[]{\label{fig:stk3}\includegraphics[width=0.30\textwidth]{stacking3}}
\caption[Optical reflectance simulation on three stacking configurations]{Optical reflectance simulation on three  \ce{WS2} wall stacking configurations.}
\label{fig:ch5ws2stk}
\end{figure}

\Gls{pl} was examined with core-shell NWs on insulating \ce{SiO2}-Si substrates. A typical Raman spectrum from 160 to 6000 \si{cm^{-1}} was presented in Fig.~\ref{fig:ws2ramall}. This pattern was acquired on the middle spot of the core-shell NW shown as inset. Using 532 nm excitation, the multi-phonon resonant feature was clearly resolved. A broad \gls{pl} peak was also observed with center at 628 nm and FWHM of 23 nm, corresponding to a photon energy of 1.974 eV with width of 72 meV.  
% ws2/wo3 on sio2,1 
\begin{figure}[htb]
\centering
\includegraphics[width=0.7\textwidth]{ws2_sio2_all}
\caption[Raman spectrum on the middle spot of core-shell NW]{Raman spectrum acquired on the middle spot of core-shell NW (inset) using 532 nm. The $x$ axis is in logarithmic scale.}
\label{fig:ws2ramall}
\end{figure}
Another feature was the presence of three peaks centering at 1860, 3695, and 5500 \si{cm^{-1}}, respectively, where the latter two were probably the secondary and third overtones of the first peak. It was confirmed that this three peaks belonged to Raman shift instead of \gls{pl}, since similar peaks were observed using different excitation wavelength of 441 nm and 532 nm. The origin of this peak family was still under research, and one possible candidate was the \ce{C=O} bonding, which presumably came from the carbon impurity in tungsten powders as seeds. It was also possible that the \ce{C=O} bonding arose from the hydrocarbon contamination attached onto the core-shell NWs. Interestingly, the intensity of this three-peak family attenuated significantly when using 441 nm excitation, as shown in Fig.~\ref{fig:ws2ramblue}. This comparison suggested there might be some enhancement mechanism from \ce{WS2} resonant Raman vibrations. 
\begin{figure}[htb]
\centering
\includegraphics[width=0.5\textwidth]{ws2_sio2_all_blue}
\caption[Raman spectrum acquired on the middle spot]{Raman spectrum acquired on the middle spot of core-shell NW shown in the inset of Fig.~\ref{fig:ws2ramall}.}
\label{fig:ws2ramblue}
\end{figure}

Three Raman sites were chosen to probe possible wall variation using both 532 and 441 nm excitation with the naming displayed in the inset of Fig.~\ref{fig:ws2ramall}. The Raman patterns were shown in Fig.~\ref{fig:ws2ram3site}a-b. The off-resonance profile in Fig.~\ref{fig:ws2ram3site}b show only two modes: $E_{2g}^1$ mode at about 357 \si{cm^{-1}} and $A_{1g}$ mode at $418\sim419$ \si{cm^{-1}}. In contrast, the on-resonance profile in Fig.~\ref{fig:ws2ram3site}a exhibited much richer spectra. The shoulder mode at tip2 spectrum was a manifestation of $B_{1u}$ mode from curved \ce{WS2} wall. According to previous analysis, the apparent ratio between in-plane and out-of-plane vibration indicated site tip2 had multi-wall \ce{WS2}, and the other two possibly had few-wall \ce{WS2}. 
\begin{figure}[htb]
\centering
\subfloat[]{\label{fig:green}\includegraphics[width=0.45\textwidth]{ws2_sio2_raman_green}}\hspace{0.04\textwidth}
\subfloat[]{\label{fig:blue}\includegraphics[width=0.45\textwidth]{ws2_sio2_raman_blue}}
\caption[Raman scattering from two excitation wavelength]{Raman scattering from (a) 532 nm and (b) 441 nm excitation scanned along the core-shell NW.}
\label{fig:ws2ram3site}
\end{figure}
This predication was confirmed by the associated \gls{pl} measurement displayed in Fig.~\ref{fig:ws2plgreen}. Strong luminescent peaks were found at site mid and tip1, and an extremely weak peak were found from site tip2. The \gls{pl} peaks centered at 1.958 and 1.934 eV for mid and tip1, respectively. The second and third order peaks also show up on all three Raman sites. 
\begin{figure}[htb]
\centering
\includegraphics[width=0.6\textwidth]{core-shell-sio2-pl_green}
\caption[PL spectra of \ce{WO3}-\ce{WS2} on \ce{SiO2}-Si]{PL spectrum acquired on the core-shell NW shown in the inset of Fig.~\ref{fig:ws2ramall}.}
\label{fig:ws2plgreen}
\end{figure}

\section{Summary} 

A comprehensive understanding of the structure-property relation, which is essential to successful application of nanomaterials, is often hindered by the ensemble averaged characterizations in conventional methods. The present work in this chapter, to some extent, circumvented this limitation by leveraging a home-built micromanipulator apparatus to transfer the nanostructures between TEM grids and other desired substrates. Considerable insight on the \ce{WO3}-\ce{WS2} core-shell nanostructures was obtained: few-wall, even single-wall tubular \ce{WS2} nanostructure were observed with \ce{WO3} cores; wall number dependent Raman spectra were studied at on-resonance and off-resonance conditions; the fingerprint of few-wall \ce{WS2} was revealed; and photoluminescence from core-shell NWs was identified. These results held great values to the application of \ce{WO3}-\ce{WS2} heterostrucutres in photocatalysis and photoelectrochemical applications. More important, this technique of TEM-Raman integrated study could be readily applied to a variety of other nanomaterials, therefore opening a broad spectrum for the understanding of the structure-property relation, propelling the potential nano-engineering applications. 

\chapter{SUMMARY AND FUTURE WORK}

In summary, synthesis of \ce{WO3} and \ce{MoO3} nanostructures based on CVD method were thoroughly investigated in this study. Sodium impurity effect in tungsten powder source was revealed, and the formation of \ce{Na5W14O44} nanowire was first reported. Partially enlightened by this discovery, a new VSS growth mechanism of \ce{MoO3} was hypothesized, and the growth dynamics was systematically probed. Comprehensive electron microscopy characterizations were performed on both \ce{WO3} and \ce{MoO3}. And the insight of growth kinetics can be used to establish a systematic connection between the interaction of transition metal oxide and alkali metal ions. As an example, preliminary results of \ce{WO3} NWs growth using \ce{NaOH} treated Si substrate and controlled \ce{MoO3} 1D nanostructures growth on glass were presented in Sec.~\ref{sec:nawo} and Sec.~\ref{sec:mogls}, respectively. Seeded growth of \ce{WO3} was demonstrated to be an effective and scalable approach for obtaining single crystalline 1D \ce{WO3} NWs. Based on these \ce{WO3} nanowires, \ce{WO3}-\ce{WS2} heterostructure was synthesized using a simple sulfurization process, and single-walled \ce{WS2} tubular core-shell nanowire was observed. More important, a TEM-Raman integrated approach was tested aiming at providing a location-resolved map between crystal structure and optical properties. As a result, wall number dependence of Raman spectra on \ce{WS2} tubular nanostructure was fingerprinted, and direct band gap \gls{pl} from \ce{WS2} tubular structure was observed. These results indicated a broad space for band structure tuning of \ce{WO3}-\ce{WS2}. Moreover, the integration of TEM-Raman served as an excellent tool to correlate the structure-property relationship for other 1D nanomaterials.

\section{Preliminary Results on Catalytic Growth of Tungsten Oxide}\label{sec:nawo}
The growth conditions almost remained the same as that of the non-seeded \ce{WO3} growth except that the substrate were located at 7.5 inch (refer to page~\pageref{sec:woxnonseed} for details). As shown in Fig.~\ref{fig:naohwsem}, dense NWs were observed on \ce{NaOH} treated substrate by SEM imaging, where the length ranged from 3 to 10 $\mu$m and diameter from 50 to 200 nm. Compared with the \ce{WO3} NWs in seeded growth, the deposition here also exhibited some plausible plate morphology.
\begin{figure}[htb]
\centering
\includegraphics[width=0.7\textwidth]{naohwo3_sem.jpg}
\caption[SEM characterization of \ce{NaOH} catalyzed \ce{WO3} growth]{SEM characterization of \ce{NaOH} catalyzed \ce{WO3} growth. (a) Low-magnification SEM graphs showing dense \ce{WO3} NWs, and (b) a high magnification view showing thin NW growth.}
\label{fig:naohwsem}
\end{figure}

\begin{figure}[htb]
\centering
\includegraphics[width=0.5\textwidth]{naoh-wo3_edx}
\caption[Composition analysis on \ce{NaOH} catalyzed \ce{WO3} growth NWs]{Composition analysis on \ce{NaOH} catalyzed \ce{WO3} growth NWs identifying small fraction Na content.}
\label{fig:naohwedx}
\end{figure}
\gls{edx} analysis on the \ce{NaOH} \ce{WO3} sample was shown in Fig.~\ref{fig:naohwedx}. Tungsten, oxygen, and sodium elements were detected on the plate growth region with Na/W atomic ratio close to 0.05/1. 
%082713 sample
\begin{figure}[htb]
\centering
\subfloat[]{\label{fig:nawxrd}\includegraphics[width=0.45\textwidth]{naoh-wo3_xrd}}\hspace{0.04\textwidth}
\subfloat[]{\label{fig:nawram}\includegraphics[width=0.45\textwidth]{naoh-wo3_raman}}
\caption[Characterization of \ce{NaOH} catalyzed \ce{WO3} NWs: XRD and Raman]{Characterization of \ce{NaOH} catalyzed \ce{WO3} NWs: XRD and Raman. (a) XRD pattern of the as-prepared sample, and (b) Raman spectrum on NWs region.}
\label{fig:naohwxrd}
\end{figure}
Fig.~\ref{fig:nawxrd} displayed the XRD pattern on one representative sample. The phase identification turned out to be ambiguous. One candidate was orthorhombic \ce{WO3} phase (ICDD PDF 00-020-1324, \emph{a}=7.384 \si{\angstrom}, \emph{b}=7.512 \si{\angstrom}, \emph{c}=3.846 \si{\angstrom}); another one was monoclinic \ce{W10O29} phase (ICDD PDF 04-007-0501, \emph{a}=12.1 \si{\angstrom}, \emph{b}=3.78 \si{\angstrom}, \emph{c}=23.4 \si{\angstrom}, $\beta$=90.5$^\circ$). Micro-Raman scattering spectroscopy was performed on the as-synthesized sample as well. As shown in Fig.~\ref{fig:nawram}, five distinct bands were well resolved with peaks located at 230, 380, 428, 650 and 816 \si{cm^{-1}}, respectively. Another small hump was also observed at 946 \si{cm^{-1}}. This Raman spectrum deviated from typical features of \ce{WO3},\cite{Salje1975a,Dixit1986} but matched well with \ce{WO3.H2O} in one previous study;\cite{Daniel1987} (also see Table.~\ref{tab:woram2} on page.~\pageref{tab:woram2}) however, the XRD pattern did not reveal tungsten oxide hydrates feature.\footnote{ICDD PDF 04-011-1708, \emph{a}=7.35 \si{\angstrom}, \emph{b}=12.5 \si{\angstrom}, \emph{c}=7.70 \si{\angstrom}} Further analysis was therefore required to confirm the exact phase of \ce{NaOH} catalyzed \ce{WO3} NWs.\cite{Azimirad2009a} 

\begin{figure}[htb]
\centering
\includegraphics[width=0.6\textwidth]{naoh_wo3_tem.jpg}
\caption[Characterization of \ce{NaOH} catalyzed \ce{WO3}: TEM]{TEM Characterization of \ce{NaOH} catalyzed \ce{WO3}: (a) TEM image of one nanowire with diameter ca. 20 nm, and (b) HRTEM images on the tip region showing absence of catalytic particle.}
\label{fig:naohwtem}
\end{figure}

TEM specimen was prepared by using carbon grid to slightly scratch the as-grown sample. Fig.~\ref{fig:naohwtem} show the feature of majority NWs. The growth direction was difficult to determined due to the presence of streaking in SAED pattern (not shown here), which presumably arose from stacking defaults during the NW growth. HRTEM revealed dominant plane spacings of 3.90 and 3.80 \si{\angstrom}, in favorable comparison to the XRD peak at $22.92^\circ$ and $23.66^\circ$. It was worth noting that catalyst particles were absent on the tip of NWs during the TEM observation. 
\begin{figure}[htb]
\centering
\includegraphics[width=0.5\textwidth]{Na2WO4_phase}
\caption[Phase diagram of \ce{Na2WO4}-\ce{WO3}]{Phase diagram of \ce{Na2WO4}-\ce{WO3} reproduced from Ref.~\cite{Hoermann1929}}
\label{fig:nawopd}
\end{figure}
Based on the above characterizations, this thesis concluded that the majority of NW belonged to tungsten oxide, although Na doping cannot be ruled out completely. According to the \ce{Na2WO4}-\ce{WO3} phase diagram in Fig.~\ref{fig:nawopd}, the eutectic temperature is ca. 720 \si{\degreeCelsius}.\cite{Mann2007} Further investigations are required to better understanding the growth mechanism. The author suggested using other alkali metal compound as catalyst to probe the \ce{WO3} growth. 

\section{Preliminary Results on Molybdenum Oxide Growth Using Glass}\label{sec:mogls}

The various growth morphologies found using alkali metal based materials as catalysts demonstrated the proposed \gls{vss} mechanism and transverse mode was a general phenomenon, and broad engineering space could be explored to control the dimension of \ce{MoO3}. As an example, this study further illustrated the tunable growth of \ce{MoO3} nanoribbons on glass substrate. Achieving morphology controlled growth of \ce{MoO3} on transparent substrates, e.g. glass, is an important progress towards process integration and device fabrication in photon-energy related applications. 

This thesis studied the effect of temperature, oxygen partial pressure, and growth time towards \ce{MoO3} deposition on glass substrates. Each factor was divided into two levels with $T$ at 700 and 800 \si{\degreeCelsius}, \ce{O2} flow at 1 and 10 sccm, and growth time of 15 and 60 min, respectively. And the morphological variations revealed by SEM imaging were summarized in Table~\ref{tab:mo3glass}.
% glass growth matrix
\begin{table}[htb]
\centering
\caption{\ce{MoO3} growth on glass}\label{tab:mo3glass}
\begin{tabular}{lccp{3in}}
\toprule
\multicolumn{3}{c}{Growth Conditions} \\
\cmidrule(l){1-3}
$T$ (\si{\degreeCelsius}) & \ce{O2} (sccm) & Time (min) & Morphology  \\
\midrule
700    &  10   & 15  &   about 50 $\mu$m belts, some towers \\
700    &  1    & 60  &   irregular shape, not too much growth\\
700    &  10   & 60  &   particle almost all on tip of belts, many towers \\
800    &  1    & 15  &   tens $\mu$m belts with sharp edge (100 nm thick and 1 micrometer wide) dominate\\
800    &  1    & 60  &   exhibiting ITO features, particles on tips of belt, many tower like structures\\
800    &  10   & 15  &   rounded edge flakes dominate, modest density tapering belt, and some possible tower\\
800    &  10   & 60  &   ITO/glass features, aggregated belts, some towers\\
\bottomrule
\end{tabular}
\end{table}
The 800-1-15 combination produced dense \ce{MoO3} nanoplate with length about 10 $\mu$m, width $300\sim500$ nm and thickness less than 100 nm, as illustrated in Fig.~\ref{fig:ch4glass}. These \ce{MoO3} nanoplates were of high crystalline quality and could potentially show enhanced photoelectrochemical and electrochromic activities. 
%010714r1 sample
\begin{figure}[htb]
\centering
\includegraphics[width=0.7\textwidth]{moo3_glass_lbd}
\caption[Optimized \ce{MoO3} nanoplates growth on glass]{Optimized \ce{MoO3} nanoplates growth on glass.}
\label{fig:ch4glass}
\end{figure}
%raman data on moo3 glass available

For future work, the author suggested several following perspectives:
\begin{enumerate}
\item further investigation of \ce{WO3} growth using alkali metal ions as catalysts;
\item probing the VSS growth method of \ce{MoO3} using HRTEM and analyzing the catalyst state; and
\item applying TEM-Raman integrated study onto other nanostructures.
\end{enumerate}

    \chapter{paper draft}
\section{wo3}

\subsection{used}

SG and FG experiments remain essentially the same as that of OT growth, except that in SG additional tungsten powders were distributed onto the receiving substrate, and in FG more than one piece of substrate was employed. The modification is schematically illustrated in Fig.~\ref{fig:wogrowsf}. More details will be provided when it comes to the discussion.
% sg fg
\begin{figure}[htb]
\centering
\includegraphics[width=0.5\textwidth]{sg_and_fg.jpg}
\caption[\ce{WO3} NW growth: SG and FG]{\ce{WO3} NW growth: SG and FG. (a) Seeded growth with additional powders on substrate (b) Flow growth with multiple substrates}
\label{fig:wogrowsf}
\end{figure}

With four kinds of W powders and three layouts, we designed the experimental matrix as illustrated in Table.~\ref{tab:matrix}. The symbol $\times$ means this combination is covered in this work, and NA means otherwise.
% Tungsten powders growth design
\begin{table}[htb]
\centering
\caption{Tungsten powders growth design}\label{tab:matrix}
\begin{tabular}{lccr}
\toprule
 & Ordinary Transport & Seeded growth & Flow growth \\
\midrule
3N   &  $\times$ & NA &  NA   \\
3N5  &  $\times$ & NA &  NA   \\
4N5  &  $\times$ & $\times$ & $\times$ \\
5N   &  $\times$ & $\times$ &  $\times$ \\
\bottomrule
\end{tabular}
\end{table}

We will first present the OT growth results in section~\ref{sec:nawox}, and discuss the other two in section~\ref{sec:sgfg}.

We found the growth using 3N source show distinctive features in comparison to the rest. This mainly arise from the higher sodium concentration in 3N source than that in others. Therefore we focus on 3N source first, and move on to other latter.


% Na5 raman fitting
\begin{figure}[htb]
\centering
\includegraphics[width=0.6\textwidth]{naxwo_ramfit}
\caption[\ce{Na5W_{14}O_{44}} Raman fitting]{Multi-peaks Lorentzian fitting on two major peaks region of \ce{Na5W_{14}O_{44}}. The peaks sum height difference is caused by different baseline value adopted in each fitting.}
\label{fig:naworamfit}
\end{figure}

% W-O bond length
\begin{table}[htb]
\centering
\caption{\ce{W-O} bond length predication}\label{tab:nawobond}
\begin{tabular}{lccr}
\toprule
peak center & length (\AA) & peak center & length (\AA) \\
\midrule
694.6 & 1.900 &  808.6 &  1.821 \\
745.4 & 1.863 &  911.5 &  1.758 \\
764.4 & 1.850 &  933.0 &  1.745 \\
778.7 & 1.840 &   943.5 & 1.740 \\
788.4 & 1.834 &   965.4 & 1.728 \\
\bottomrule
\end{tabular}
\end{table}

An empirical formula to relate the Raman peaks and \ce{W-O} bonding lengths \cite{Hardcastle1995} is
\begin{equation}\label{eq:wobond}
\nu = 25823 \exp{-1.902\cdot R}.
\end{equation}
And the standard deviation of estimating \ce{W-O} bond distance from Raman stretching wavenumber is $\pm0.034$\AA.
The observed Raman peaks of \ce{Na5W_{14}O_{44}} phase lies at 965, 943, 913, 808, 786, 778, 765, 695 and 107 cm. Multi-peaks Lorentzian fitting is preformed to precisely determine the central maximum. Good fitting is obtained, as shown in Fig.~\ref{fig:naworamfit}. We then calculated \ce{W-O} bond distance using Eq.~\ref{eq:wobond}, as illustrated in Table~\ref{tab:nawobond}. The predicted \ce{W-O} bond length comply very well with the crystallographic value of \ce{Na5W_{14}O_{44}} phase.\cite{Triantafyllou1999a} The 107 peak probably is caused by \ce{Na-O} bond.


\subsection{not used}
Both tungsten (W) and molybdenum (Mo) belong to Group VIB transition metal, with outer shell electrons configuration as $4d^55s^1$ and $5d^46s^2$, respectively. Therefore we refer their oxides and chalcogenides as \gls{tmo} and \gls{tmdc}.\footnote{Obviously transition metals include many other elements, all of which have partially filled $d$-electron shell. But here we use TM to denote W and Mo exclusively.}

Nucleation is a process of generating a new phase from a metastable old phase, where the Gibbs energy per molecule of the bulk of the emerging new phase is less than that of the old phase.

General CVD knowledge, substrate preparation, and\cite{MichealK.Zuraw2003}


The energetic sources are ion bombardment, electron beam, laser ablation, and combustion flame\cite{Rao2011}.

The sol–gel process is a well-known, intensively studied wetchemical technique that is widely used in materials synthesis. This method generally starts with a precursor solution (the ``sol") to form discrete particles or a networked gel structure. During the course of gelation (aging process), various forms of hydrolysis and polycondensation take place.
In addition, doped \ce{WO3} was also demonstrated
The composition and phase of final product highly depend on the synthesis conditions.

We do not discuss tungsten oxide hydrates (\ce{WO3.nH2O}) in this work since the product of thermal CVD approach is not plagued with this complexity. It's necessary, however, to deal with hydrated \ce{WO3} in the liquid synthesis routes, as indicated in Section.~\ref{sec:woxgrowth}.


Nonstoichiometric tungsten oxides \ce{WO_x} (i.e. \ce{WO_{2.92}}, \ce{WO_{2.87}}) are known as Magn$\acute{e}$li phases.


Theoretical computation of electronic band structures for \ce{WO_x} proves difficult due to the aforementioned phase transition. oxygen deficiency, structure change, electronic properties vary according.

the ubiquity of \ce{WO6} octahedra is essential for not only the optical properties but the ability to insert and extract ions in the EC oxides, due to the tunnels in three dimensions serving as path for transport of small ions. The intercalation of hydrogen or alkali ions into \ce{WO3} created electron donor level. By absorbing the red part of incident spectrum, electrons at donor level make transition to the conduction band, causing the blue coloration in \ce{H_xWO3}.

its one dimensional (1D) nanostructure has attained intensive research efforts in recent years due to the potential applications in advanced nano-electric and nano-optoelectronic devices.

\begin{quote}
a viable electrochromic smart window must exhibit a cycling life time \textgreater $10^5$ cycles, corresponding to an operation life at 10 - 20 years.
\end{quote}

\subsection{to be used}

W plasma oxidation.\cite{Romanyuk2005} 200nm W coating on Si (100) sub, temperature at RT, 390, and 490 C, oxygen pressure 0.5 Pa, oxidation time for 10 to 3600 s. The resultant thickness of \ce{WO3} at RT  and time of 10 s and 3600 s is found to be 0.2 nm and 11 nm respectively.

\[
 d = d_0 exp(kt)
\]

after fitting, $d_0 = 0.19777$ nm, $k = 0.00112 $ s$^{-1}$, so to oxidize 1nm W coating completely, oxygen plasma time is 1450 s at 0.5 Pa partially pressure; 2 nm for 2000 s.



\ce{WOx} and optical electric field enhancement. The enhancement arise from the structure composed of a conductive layer and an insulating layer that are laminated therein.\footnote{US patent 8601610B2} In \ce{WOx} nanorods, the oxygen deficient planes are conductive, each having atomic thickness and separated by several nm \ce{WO3}. Localized surface plasmons could possibly exist on these conductive planes. Therefore SERS applies and single molecule Raman scattering using a tungsten oxide nanorod has been demonstrated. The \ce{W_nO_{3n-1}} ($n \geq 2$) exhibit $\{ 001 \}$ CS structure. Chemical formulae corresponding to n=2, 3, 4, 5 and 6 are \ce{W2O5=WO_{2.5}}, \ce{W3O8=W_{2.67}}, \ce{W4O_{11}=WO_{2.75}}, \ce{W5O_{14}=WO_{2.8}}, and \ce{W6O_{17}=WO_{2.83}}, which indicates the existence of a oxygen deficient plane at every n row. Actually the value x in \ce{WOx} could almost continuously vary within a range of 2.5 to 3. \ce{W_{18}O_{49}=\ce{WO_{2.72}}} is an exception without $\{ 001 \}$ CS structure. Moreover, the oxygen deficient planes could extend along directions other than $\{ 001 \}$. For instance, the $\{ 102 \}$ CS planes appears in \ce{WOx} where x is within 2.93 to 2.98, and  the $\{ 103 \}$ CS planes for x within 2.87 to 2.93.\cite{Sloan1999}  \citeauthor{Shingaya2013} also synthesized \ce{WS2}-\ce{WO_x} structures and found similar Raman scattering enhancement. The x value is estimated by the Raman spectra peaks.\cite{Shingaya2013}(Data not shown in patent)

For photochemical water reduction to occur, the flat-band potential of the semiconductor (for highly doped semiconductors, this equals the bottom of the conductance band) must exceed the proton reduction potential of 0.0 V vs NHE at pH =0. \cite{Osterloh2008} flat-band potentials strongly depend on ion absorption (protonation of surface hydroxyl groups), crystallographic orientation of the exposed surface, surface defects, and surface oxidation processes.


\ce{W_{18}O_{49}} Raman, IR shielding.\cite{Guo2012} \cite{Guo2011}
broad peak between 750-780 cm-1.

\ce{WnO_{3n-1}} NPs. \cite{Frey2001}


WO3-x raman info, the encapsulated WOx core has been investigated in depth. several stable phase could occur, including \{001\} CS phases, \{103\} CS phases. no evidence of $\gamma$-\ce{W_{18}O_{49}} phase is found. The cross-section ($\sigma$) for Raman scattering and the absorption coefficient of the WS2 layers are much larger than those of the suboxide phase encapsulated inside.

WO2:168(w),189(w), 286(vs), 345(w), 423(w), 479(m), 512(m), 599(m),
617(m) cm-1, and a mode at 781(s) cm-1 which tails to higher energies (w-weak; m-medium; s-strong; vs-very strong).

W5O14: 215, 264, 325, 349, 418, 425,707, and 800 cm-1, 900 maybe

WO3: 808, 719, 275;

W3O8: 870;

no 950 peak indicates no hydrated phases.

\ce{WO_{3-x}} Raman peak at 778. \cite{Deb2007}

% wo3-x phases
\begin{table}[htb]
\centering
\caption{List of \ce{WO_{3-x}} phases}\label{tab:wo3xphase}
\begin{tabular}{lccccc}
\toprule
&&&\multicolumn{3}{c}{Lattice constants \AA} \\
\cmidrule(l){4-6}
 Symbol    & PDF  & Phase & a & b & c   \\
\midrule
\ce{W18O49}  & 00-036-0101 & monoclinic & 18.324 & 3.784 & 14.035  \\
$\delta$-\ce{WO3}   & $-50 \sim 17$  & triclinic & 7.309 & 7.522 & 7.686  \\
$\gamma$-\ce{WO3}   & $17 \sim 330$  & monoclinic I & 7.306 & 7.540 & 7.692  \\
$\beta$-\ce{WO3}    & $330 \sim 740$  & orthorhombic & 7.384 & 7.512 & 3.846  \\
$\alpha$-\ce{WO3}   & $> 740$  & tetragonal & 5.25 & NA & 3.91  \\
$h$-\ce{WO3}        &  $<400$  & hexagonal & 7.298 & NA & 3.899  \\
\bottomrule
\end{tabular}
\end{table}



\section{moo3}

\subsection{used}



\ce{MoO3}, an alternative interpretation in terms of tetrahedral coordination of Mo atoms is also proposed. This is caused by the fact that four of the six surrounding O atom are at distances from 1.67\AA to 1.95\AA, while the remaining two are as far as 2.25 and 2.33\AA. This also stress that the \ce{MOO6} octahedra are rather distorted.


\subsection{to be used}


\cite{Matar2011} Using electronegativity $\chi$ and chemical hardness $\eta$ to assess electron affinity $E_a$, work function $W_f$, Fermi energy $E_f$ and band gap $E_g$.
\begin{align}
\chi &= 0.5(W_f + E_a)\\
\eta & = 0.5(W_f - E_a)
\end{align}
where I is ionization potential and $E_a$ is electron affinity.

Correlation between optical band gap and formation enthalpy; reaction occurs in order to form compounds with a larger gap.  $E_g = A \exp(0.34E_{\Delta H^0})$, and A adopts different values depending on the metal elements:
\begin{itemize}
\item A=0.8 for s and f block elements,
\item A = 1 for d block elements,
\item A = 1.35 for p block elements.
\end{itemize}


\citeauthor{Sreedhara2013} studied the kinetics of photodegradation of methylene blue\footnote{\ce{C16H18N3SCl},319.8 g/mol, MP: 100C accompanied with decomposition \url{http://en.wikipedia.org/wiki/Methylene_blue}} dye by few layer \ce{MoO3}.
For the photodegradation method, it was stated that `` the samples were collected after the photoreaction had been centrifuged for 5 min to remove the photocatalyst before UV-Vis measurement.''


% Melting points 
\begin{table}[htb]
\centering
\renewcommand*{\thetable}{S\arabic{table}}
\caption{physical constants of reactants }\label{tb:thermo}
\begin{tabular}{lccr}
\toprule
Material & MP(\si{\degreeCelsius}) & BP(\si{\degreeCelsius}) & reference\\
\midrule
\ce{NaOH}        & 318 & 1388 & handbook  \\
\ce{NaI}        & 651 & 1300 & MSDS    \\
\ce{KI}        & 681 & 1330 & MSDS   \\
\ce{Na2CO3}        & 851 & Not determined & MSDS    \\
\ce{Na2MoO4}        & 687 & Not available & handbook   \\
\ce{MoO3}    & 795 & 1155 & MSDS   \\
\ce{MoO2}    & 1100(decomp) & Not available & MSDS   \\
\bottomrule
\end{tabular}
\end{table}



\section{ws2}


\textbf{\ce{WS2}-\ce{WO3}}: 1 kW light source(Hg, or Xe lamp), photon flux, phenol (\ce{C6H5OH}, 94.1g/mol, MP 40C)concentration is 20 mg/L, hydroxyl group. The quantitative analysis of phenol was performed via a standard colorimetric method.\footnote{\url{http://omlc.ogi.edu/spectra/PhotochemCAD/html/072.html}}
\citeauthor{DiPaola1999} prepared \ce{WS2}-\ce{WO3} mixture in two methods, sulfurization of \ce{WO3} and oxidation of \ce{WS2},with the latter are more active.
\citeauthor{DiPaola1999} also concluded that the actual efficiency of mixed \ce{WS2}-\ce{WO3} catalysts depends on the ratio of each composition present of the surface of the particles, and the maximum of photoactivity is obtained with 40-50\% surface molar ratio of \ce{WS2}.

ref 25, 28 and 41.


\subsection{used}

As the experimental setup for direct tensile tests of nanotubes is state-of-the-art,\cite{Tang2013} the application of tensile stress on 2D TMD systems is rather difficult due to the excellent lubricating properties of these materials.

\citeauthor{Zhang2013e} investigated the shear (C) and layer breathing mode (LBM) in the low frequency region of \ce{MoS2}.\cite{Zhang2013e} Even layer \ce{MS2} belong to point group D$_{6h}$ with inversion symmetry, while odd layer \ce{MS2} correspond to D$_{3h}$ without inversion symmetry. The excitation wavelength is 532nm from a diode-pumped solid-state laser. A power$\sim$0.23mW is used to avoid sample heating.

reaction mechanism of \ce{MoO3} to \ce{Mo2S}.\cite{Weber1996}

\citeauthor{Ling2014} studied the role of seeding promoters in CVD growth of FL \ce{MoS2}.\cite{Ling2014} PTAS treated substrates provided nucleation site and thus enable uniform deposition of \ce{MS2}.  This enhancement perhaps arise from the \ce{K+} ions.

\citeauthor{Splendiani2010} reported the PL in monolayer \ce{MoS2}.  Calculation indicated the indirect gap become larger when thinning, while the previous direct one almost stays as the same, the value is about 1.85eV (direct gap).\cite{Splendiani2010}


thermal decomposition of (NH4)2MoO2S2 and intermediate product MoOS2 was studied. application: hyfrodesulfurization in refinery \cite{Weber1996}

\cee{MoCl5 + 1/4S8 + 5/2H2 \rightarrow MoS2 + 5HCl} \cite{Stoffels1999}


A direct gap of $\sim 2eV$ at the corners of BZ is formed in 1L \ce{WS2}, Growth on bottom piece show the multiple domain flakes occurs at initial stage of the growth, starting from \ce{WO3} particles.
%\cite{Cong2013}
\subsection{to be used}

Exfoliated WS2 few layer PL.\cite{Zhao2012} excitonic absorption peaks A and B arising from direction transition at K point are found around 625nm (1.98eV) and 550nm, respectively, which are in agreement with results from bulk layers. The A, B excitons difference was a result of strong spin-orbital coupling. Relative PL quantum yield of WS2 between 1L and 2L is on the order of 2. The FWHM of WS2 peak is about 75 meV. wider than thermal energy at room temperature,

Electro microscopy on stacking sequences of WS2 NT.\cite{Houben2012} The probability of parallel stacking is about 30\%. a metal-semi superstructures. In NT, the layers are slightly shifted with respect to each other due to the constraints, thus the stacking is not exactly as pure phases of 2H(prismatic antiparallel), 3R(prismatic parallel) or 1T (octahedral parallel) with their perfect translational symmetry.

chevron pattern, contradictory, contradicting, Debye scattering model for XRD.

\begin{quote}
hexagonal polytype 2Hb with two molecular layers (spacegroup P63/mmc) and a rhombohedral polytype 3R with three molecular layers per unit cell (space group R3m), a high pressure polytype that is stable in plane geometry at pressures above 4 GPa. The two prismatic phases are semiconducting, and the octahedral one is metallic-like.

1T phase may be the result of a transformation from the 3R to the 2H phase by an intermediate 1T phase that is trapped by fast quenching

\end{quote}

aberration corrected TEM is used.

HRTEM on WS2 NT.\cite{Sadan2008} negative spherical-aberration imaging (NCSI). NCSI condiction were achieved at a negative spherical aberration of -20um balanced by an overfocus of +17 nm. Focal series reconstruction to retrieve the phase of electron exit plan wavefunction. Zigzag, armchair revealed.


In centrosymmetric crystals, the vibrational modes must either have even (Raman-active) or odd (IR-active) parity under inversion, which is known as rule of mutual exclusion. When this symmetry is broken, some modes may be simultaneously IR and Raman active.

inelastic neutron scattering to study the non-zone center LA mode. Zone-edge scattering can occur due to zone-folding process. The formation of superlattice could activate formerly inactive zone-edge phonons. The folding of BZ along $\Gamma-M$ would cause the M point to coincide with $\Gamma$ point, so LA(M) phonons would become Raman active in a first-order process.


\ce{SiO_x}-Si, \ce{WS2} absorption coefficient $10^{-7}m^{-1}$, mean free path of photo-excited charge carriers 1 $\mu m$. the wave vector of photon is considerably small than size of BZ, therefore The wave vector of phonon in Raman scattering usually close to zero.

Multiple phonon scattering, For two identical phonons, the corresponding Raman peak in the spectrum is called an overtone of the peak from the corresponding one-phonon process. And the wave vector conservation rule is automatically filled, therefore the phonon involved is not limited to BZ center anymore.
\[
I(G) \approx \sum_k \frac{\langle f|H_M|b\rangle \langle b|H_{ep}|a\rangle \langle a|H_M|i\rangle}{(E_p - E_k^{\pi *}- E_k^{\pi}-i\gamma)(E_p - E_k^{\pi *}- E_k^{\pi}-i\gamma- \hbar\Omega_{G})}
\]

the average distance traveled by an excited electron-hole pair before combination $l=\nu_F/\omega_D=4nm$.

Confocal Raman spectrometer:to obtain Raman spectrum in a specific depth of sample. Edge filter to cut off Rayleigh emission.


resolution $d= 1.22 \lambda/NA$,

Light Scattering in Solids II,. Springer, Berlin, 1982

influence of core WOx, Raman scattering by plasma-LO coupling to determine carrier concentration. measure resonant cross sections in absolute units.

disorder-induced light scattering, Van Hove critical points,
In resonant second-order scattering:
overtone: the same phonon,
combination: two different phonons;

\[
\frac{\ud\sigma}{\ud\Omega}= \omega_s^4 cm^6 Sr^{-1}
\]

scattering volume V in number of unit cells can be considered as one big molecule.


a single nanowire tends to minimize its surface. 2D isoperimetric quotient or circularity $C= \frac{4\pi A}{P^2}$, where A is area and P is perimeter of the cross-section.




\section{ECD}


Characterization of ECD (work like a thin-film batteries) includes transmission measurement and associated EC calculation, charge-discharge time, current-time curve and the fitting of obtained data.

The coloration efficiency (CE) represents the change in the optical density (OD) per unit charge density ($Q/A$, in units of \si{\cm^2\per\coulomb}) during switching and can be calculated according to the formula:
\begin{equation}
CE = \frac{\Delta~OD}{(Q/A)} [cm^2/C],
\end{equation}
where OD = $log(T_{bleach}/T_{color})$. The EC and optical density depend on the wavelength and are usually higher in the near IR than in the visible region.
Using Ohm's law($U_s = IR = RQ/t_s$) with switch voltage $U_s$, resistance R and surface area A, switching time $t_s$ could be estimated as
\begin{equation}
t_s = \Delta~OD\cdot A \cdot R /(CE\cdot U_s).
\end{equation}



battery and ECD.\cite{Granqvist2012} electrolyte: PVB (poly vinyl buteral).
alternative materials and design: organic, Prussian Blue as EC materials, metal hydrides, suspended particle device, liquid crystal, electroplating,
challenges: large area nanoporosity, transparent conducting contact, electrolyte with good ionic conductivity and poor electronic conductivity, stable under UV; assembly and large scale manufacturing;
cathodic coloration:
anodic coloration:
The coloration mechanism: \ce{MO6} octahedrons lead to $e_g$ and $t_{2g}$ level and ion channelling.
ref54,60,65,66,200,209,


\ce{WO3} as cathodic and either polyaniline(PANI) or Prussian white (PW) as anodic electrochromic half cells. \cite{Heckner2002}

Characterization of ECD includes transmission measurement and associated EC calculation, charge-discharge time, current-time curve and the fitting of obtained data.

\begin{quote}
a viable electrochromic smart window must exhibit a cycling life time \textgreater $10^5$ cycles, corresponding to an operation life at 10 -- 20 years.
\end{quote}


\citeauthor{Sella1998} studied the optical and structural properties of RF sputtered thin film of \ce{WO3} and \ce{VO2} for electrochromic devices. Ionic conductor was built using transparent polymer electrolyte, which was prepared from a solution of 1M \ce{LiClO4} in propylene carbonate which was mixed with methylmetharcylate (MMA). The main characteristics of polymer electrolyte were: viscosity at 25 \si{\degreeCelsius} $\approx$ 12920 Pa.s, conductivity $\approx 10^{-2}-10^{-4}$ \si{\per\ohm\per cm},non-hygroscopic if PMMA concentration \textgreater 30\%. A specific counter-electrode layer was not used since the encapsulated polymer electrolyte processes a very high ion storage capacity.\cite{Sella1998}

The device proposed was reproduced as shown in Fig.~\ref{fig:Sella1998ECD}
\begin{figure}[htb]
    \centering
    \includegraphics[angle=270,width=0.8\textwidth]{Sella1998ECD}
    \caption{citation, see original captions} \label{fig:Sella1998ECD}
\end{figure}


\appendix
   %\glsaddallunused
%    
\chapter{Source code for contrast}\label{app:matlab}

The following codes calculated monolayer \ce{MoO3} contrast on \ce{SiO2}-Si substrate with respect to varying \ce{SiO2} thickness within visible wavelength region. 
\begin{singlespace}
\begin{lstlisting}

hv = linspace(1.65,3.1,100);% eV
% refractive index of MoO3 along c
epsl_c = 6.83142 - 0.89876*hv + 0.31211*hv.^2;
% along a direction
epsl_a = 5.27613 - 1.42197*hv + 1.62433*hv.^2 - ...
    0.65474*hv.^3 + 0.09216*hv.^4;
n_c = sqrt(epsl_c);
n_a = sqrt(epsl_a);
d_fl = 1.3858; % monolayer MoO3: nm

d2 = linspace(50,300,100); % SiO2/Si thickness
lambda = 1.2398./hv;% in unit of um

% fused silica 0.21~3.71 um
n_SiO2 = sqrt( 1 + 0.6961663*lambda.^2./(lambda.^2-0.0684043^2) + ...
    0.4079426*lambda.^2./(lambda.^2-0.1162414^2) + ...
    0.8974794*lambda.^2./(lambda.^2-9.896161^2) );
n_2 = n_SiO2;
% for Si valid from 400 to 780 nm
P1 = [358.370854235646,-929.730598395864,...
902.283058938096,-390.016790281725,67.5147263898247;];
n_si = polyval(P1,lambda);
P2 = [126.723254169127,-321.724727403874,...
303.571928160301,-126.259081591924,19.5651256513963;];
k_si = polyval(P2, lambda);

n_3 = n_si - 1i*k_si;


p = length(lambda);
q = length(d2);
Con_c = zeros(p,q);

% n_1: refractive index of FL in lambda range
% d1: thickness of FL
% d2: thickness of SiO2 layer on Si;
for m = 1:q
    Con_c(:,m) = Few_layers(lambda,n_c,n_2,n_3,d_fl,d2(m));
end

contourf(d2,lambda,Con_c);
xlabel('SiO_2 thickness:nm');
ylabel('Wavelength:\mu m');
colorbar;
title('1L MoO_3 contrast along c direction');
\end{lstlisting}

% The function ''Few_layers" returns optical contrast at specified wavelength and layer configuration. 
\begin{lstlisting}
function con = Few_layers(lambda,n_1,n_2,n_3,d1,d2)
% FL contrast on SiO2/Si
% lambda:
% n_1: refractive index of FL in lambda range
% d1: thickness of FL
% d2: thickness of SiO2 layer on Si;

n_0 = 1;  % air or vacuum
lambda = 1000*lambda;
phi_1 = 2*pi*d1*n_1./lambda;
phi_2 = 2*pi*d2*n_2./lambda;

r1 = (n_0 - n_1)./(n_0 + n_1);
r2 = (n_1 - n_2)./(n_1 +n_2);
r3 = (n_2 - n_3)./(n_2 +n_3);
r2_p = (n_0 - n_2)./(n_0 + n_2);

up = r1.*exp(1i*(phi_1 + phi_2)) + r2.*exp(1i*(phi_2-phi_1)) + ...
r3.*exp(-1i*(phi_1 + phi_2))+r1.*r2.*r3.*exp(1i*(phi_1 - phi_2));

down = exp(1i*(phi_1 + phi_2)) + r1.*r2.*exp(1i*(phi_2-phi_1)) + ...
    r1.*r3.*exp(-1i*(phi_1 + phi_2))+ r2.*r3.*exp(1i*(phi_1 - phi_2));

R_n1 = abs(up./down).^2;
R_1 = abs((r2_p.*exp(1i*phi_2) + r3.*exp(-1i*phi_2))./(exp(1i*phi_2)+ ...
    r2_p.*r3.*exp(-1i*phi_2))).^2;

con = (R_1 - R_n1)./R_1;

\end{lstlisting}

\end{singlespace} 
%     \chapter{CVD gas dynamics}

Reynolds number is define for laminar flow as
\[
Re = D \cdot \frac{ V_c \rho}{\eta} = D \cdot \frac{ V_c }{\nu}.
\]

D is inner diameter of pipe, $V_c$ is velocity, $\rho$ is mass density and $\eta$ is dynamic viscosity $\nu$ is and kinematic viscosity. 

Reynolds number is the square of the ratio of the system size to the diffusion length for momentum: Re is the Peclet number for momentum. We can now see that small Re corresponds to diffusion lengths comparable to system size: just as before, gradients (in this case in velocity) are small, and the system behaves as a single unit. When Re is very large (flows with Re \textgreater 10,000 or even \textgreater 1,000,000 are common in the real world, though fortunately not in CVD reactors) diffusion of momentum only proceeds locally within very tiny cells, and huge gradients are possible: i.e. recirculating and turbulent flows. Laminar flow is considered to exist if the Re numbers are significantly below 1000.

chamber volume estimation. We use 30 sccm Ar to fill the sealed chamber. At ambient condition, 1 sccm will fill a one liter vessel from zero pressure to 1 atm in a thousand minutes. And in 87 sec, the pressure increase to 80 Torr. So we estimate the chamber volume to be 410 mL.

kinematic viscosity $\nu$ in unit of $m^2/s$. Dynamic viscosity $\eta = \rho*\nu$ in unit of kg/m/s. The relationship between Pascal second and centipoise is 1 Pas = 1 $Ns/m^2$=1 kg/(ms)=103 cP.

Ideal gas equation
\[
PV = n R T;
\]
where R = 8.3 J/mole K.

\begin{itemize}
\item 1 mole = 22.4 liters at "STP" (0 C, 1 atmosphere)
\item 1 liter = 0.045 moles at STP
\item 1 cm3 = 4.5E-5 moles @ STP
\item 1 cm3 = 6.4E-8 moles @ 1 Torr, 23 C
\item 1 atm = 760 Torr = 101,000 Pa
\item 1 Pa = 7.6 mTorr
\item 1 SLPM = 7.4E-4 mole/second
\item 1 sccm = 7.4E-7 mole/second
\end{itemize}

Maxwellian velocity distribution is given by
\[
\frac{\partial N}{\partial \nu} = N 4\pi \nu^2 (\frac{m}{2\pi k T})^{3/2} exp(-\frac{m\nu^2}{2k T})
\]
with mean velocity $c_m = \sqrt{\frac{8 RT}{\pi M}}$.

pumping ability in unit of $cm^3/s$ is estimated to be.

\begin{table}[htb]
\centering
\caption{CVD parameters}\label{tab:cvd2}
    \begin{tabular}{lccr}
    \toprule
    parameter       & value    & Ar     & \ce{O2}  \\
    \midrule
    Inner Dia       & 25.4mm    &       &      \\
    pressure (mTorr)&           & 100   & 10   \\
    pressure (Pa)   &           & 13.33 & 1.333 \\
    Molar weight (g/mol) &      & 40     & 32  \\
    chamber volume  &           & $4.1e^{-4}m^3$  & 410ml \\
    molar con       & $mol/cm^3$ & $5.35e^{-9}$  & $5.35e^{-10}$ \\
    molecular con   & $molecules/cm^3$  & $3.22e^{15}$ & $3.22e^{14}$ \\
    average velocity (m/s) & 300K & 398   & 445  \\
    average velocity (m/s) & 1200K& 796   & 890  \\
    Flux  &$ molecules/cm^2/s$ & 3.21e19 & 3.58e18 \\
    $\eta$  kg/(ms) & 300K      &  $2.272e^{-5}$  & $2.063e^{-5}$   \\
    Reynolds number &          & 95.2      & 9.4   \\
    MFP  ($\mu m$)   &         &           &     \\
    \bottomrule
    \end{tabular}
\end{table}
The molar concentration at 1200 K will be only 25\% of that at RT. 

Knudsen equation is 
$$
J = \alpha3.51\times10^{22}\frac{P_{torr}}{\sqrt{MT}},
$$
Condensation rate $J(\text{atoms}/\text{cm}^2 \text{sec})$, M in grams/mol. 
This set the upper bound on the rate of deposition. Usually an accommodation coefficient $\alpha$ is used to account for the adsorption. 

Mean free path of gas in vacuum:
$$
\lambda_p = \frac{5\times10^{-3}}{P} \approx 0.05,
$$
where $\lambda_p$ is in centimeter, and P is pressure in Torr. Gas impingement flux $\Phi$, is a measure of the frequency with which gas molecules impinge on or collide with a surface.
$$
\Phi = 3.513\times10^{22}\frac{P}{\sqrt{M T}},
$$

Gas flow could be divided into three regimes: molecular flow, intermediate flow and viscous flow.
Knudsen number is
$$
K_n = D/\lambda_p \approx 50,
$$
where D is characteristic dimension of the CVD system. 
 %   \chapter{bulk tungsten disulfide}



Tungsten disulfide (\ce{WS2}) is a group VI dichalcogenide semiconductor compound. The molecular weight is 249.97 \si{g\per \mole}. Almost all natural \ce{WS2} belongs to P63/mmc space group (2H-\ce{WS2}), where $a$ is 3.153 \AA, $c$ is 12.323 \AA (PDF 04-003-4478). Two other crystal structures were found in man-made \ce{WS2}, i.e. 1T and 3R, where 3R can be prepared by bromine chemical vapor transport (CVT) method \cite{Schutte1987} and 1T was generally found in alkaline intercalated \ce{WS2}.\cite{Yang1996a, Enyashin2011}


\section{Crystallography and Thermodynamics}

The refined crystallographic data, including bond length, angle is listed in Fig.~\ref{app:bond}. 

\begin{figure}[htb]
\centering
\includegraphics[width=0.9\textwidth]{tms_crys}
\caption{Adopted from Ref.\cite{Schutte1987}}
\label{app:bond}
\end{figure}

The melt point of \ce{WS2} is 1523 K (decompose). A phase diagram between W and S is shown in Fig.~\ref{app:pd}. 

\begin{figure}[htb]
\centering
\includegraphics[width=0.6\textwidth]{ws_phase}
\caption{Adopted from Ref.\cite{Tenne1995,Tenne1998}}
\label{app:pd}
\end{figure}

\section{Band Structures}

A theoretical calculation is shown in Fig.~\ref{app:band} 
\begin{figure}[htb]
\centering
\includegraphics[width=0.7\textwidth]{ws2_bandcal}
\caption{Adopted from Ref.\cite{Kuc2011}}
\label{app:band}
\end{figure}



\section{Dielectric Function}


$\epsilon(h\nu)$ is shown in Fig.~\ref{app:nk}. 
\begin{figure}[htb]
\centering
\includegraphics[width=0.5\textwidth]{ws2_nk}
\caption{Adopted from Ref.\cite{Hughes1976}}
\label{app:nk}
\end{figure}

The $n$ and $\kappa$ can be retrieved using other software, such as Tracer 2.0.\footnote{\url{https://sites.google.com/site/kalypsosimulation/Home/data-analysis-software-1}}





   \chapter{paper reading}



\section{WO3}

Tungsten bronzes was coined by Wohler in 1837.\cite{Deb2008} \ce{Na_{x}WO3}



\subsection{applications}


\citeauthor{Wang2009a} mentioned that amorphous \ce{WO3} can only be used in lithium-based electrolytes due to its in-compact structure and high dissolution rate in acidic electrolyte solutions. Electrochromic materials that can endure acidic electrolytes without degradation should be developed. Crystalline \ce{WO3} nanostructures with their much denser structures and small particle sizes are promising to be used as suitable electrochromic material in acidic electrolytes.

photocatalytic applications in solar hydrogen generation and organic pollutant degradation.

photocatalyst\cite{Macphee2010},
photoelectrochemical energy application \cite{Su2010}

Raman \cite{Xiao2007}. Silver has the strongest SERS enhancement due to the larger imaginary part of the dielectric constant and higher thermal conductivity. Milli-Q grade water ((Milli-pore)\textgreater 18.2Mohm).

MB Raman peaks: 445, 1618, ref20. some peak splitting and shift observed on SERS, attributed to chemical adsorption. definition of Raman enhancement factor (9,26).

SERR MB on Ag. \cite{Nicolai2003}
MB: the absorption spectrum in VIS is used to infer about different adsorbed forms of MB. the formation of large aggregates. I call attention to the fact that.

WO3: effective mass of bipolaron = 1.9me. for electron, for hole:
unzip nanotube. passivate BN ribbons with O and S; another player terrones psu.

\ce{WO3} catalyst.\cite{Miyauchi2013}  potential of CB e more negative than redox potential of \ce{O2}-\ce{O2^-} (-0.046 V vs NHE at pH 0). Z-scheme two photo absorption. photogenerated ele in CB of WO3 can reduce itself by formation of color centers.

electrochromic films. \cite{Yoshimura1985}

ECD \cite{Jiao2012} recent review \cite{Mortimer2011}

PEC, photoelectrode, WO3 and Si tandem structures.\cite{Coridan2013}

WO3 photoactivity MB. \cite{Watcharenwong2008}
A low recombination rate is preferred for high photocatalytic efficiency. The simultaneous migration of electrons and holes.



\subsection{properties}

\citeauthor{Chatten2005} also studied the oxygen vacancy in different phases of \ce{WO3}.\cite{Chatten2005}

2D wo3.\cite{Kalantar-zadeh2010a} 
WO3 plasmon \cite{Manthiram2012}

DFT doped \ce{WO3} for photocatalytic reaction.\cite{Wang2012} CBM arises from W $5d$ states and splits into $t_{2g}$ and $e_g$ states under crystal field. VBM comes from O $2p$ states, including $2p_\sigma$ (along \ce{W-O} bonds) and $2p_\pi$ (normal to \ce{W-O} bonds).


oxygen vacancies in \ce{WO_{3-x}}.\cite{Wang2011b}  Coloration and electron conductivity changes. \citeauthor{Wang2011b} found strong dependence of WO3-x electronic properties on $V_O$ concentration and the the crystallographic direction on which O is removed. DFT band gap calculation is close to experimental value. Vacancy levels are found at 2.1eV.

The Raman spectra of \ce{WO_x} is rare because of the difficulty of preparing pure suboxides phase and the strong shielding of \ce{WS2}. Yet it does exhibit distinct Raman spectra. \cite{Tenne2005} The 870 line is attributed to \ce{W3O8}.\cite{Hardcastle1995}


\citeauthor{Huang2006} studied the \ce{W3On} cluster with n from 7 to 10.\cite{Huang2006} It was found \ce{W3O9} clusters possess a HOMO-LUMO gap about 3.4eV. This closeness to bulk value suggests \ce{W3O9} could be viewed as the smallest molecular unit for bulk \ce{WO3}.


\ce{WO3} indirect gap 2.6eV, direct gap 3.4eV. \cite{Koffyberg1979}

Raman fingerprint of m-\ce{WO3}, h-\ce{WO3} and \ce{WO3.nH2O} were summarized in ref\cite{Daniel1987}.

\ce{WO3} on FTO by flame synthesis.\cite{Rao2014} \cite{Xu2006}

Seeded \ce{W_{18}O_{49}} NWs growth on W foil.\cite{Hong2006a}

\ce{Na2W4O_{13}} growth and optical properties. \cite{Oishi2001} \cite{Itoh2001}

\ce{Na2W4O_{13}} crystal phase \cite{Viswanathan1974}

\citeauthor{Salje1984} studied the transport in \ce{WO_{3-x}} ($0\leq x \leq 0.28$).\cite{Salje1984} It was found \ce{WO_{3-x}} show metallic conductivity when $x > 0.1$.

\ce{WO_{3-x}} \cite{Migas2010}

\ce{WO3} high temperature phase. \cite{Vogt1999}
tungsten bronzes \cite{Wiseman1976}

Phase transformation of \ce{Na2MoO4} and \ce{Na2WO4} by Raman scattering. \cite{Lima2011}

\ce{WO2} NWs synthesis and raman \cite{Ma2005}.

\ce{WO_{3-x}} CS planes and conductivity.\cite{Sahle1983}

\ce{W-O} equilibrium diagram \cite{Wriedt1989}

\ce{W_{18}O_{49}} electrochromic devices.\cite{Liu2013d} should compare with this one \cite{Wang2008}

nucleation catalysis \cite{Turnbull1952}

\ce{WO3} NWs aggregates. \cite{Kozan2008a}

optical properties of \ce{WO3} gaps\cite{Saygin-Hinczewski2008}

\ce{WO3} atomic layer by exfoliation and annealing \ce{WO3.H2O}. \cite{Kalantar-zadeh2010a}

sodium tungstates raman \cite{Redkin2010}

charge density wave in K-doped \ce{WO3} \cite{Raj2008}

\ce{W_{18}O_{49}} Raman, IR shielding.\cite{Guo2012} \cite{Guo2011}
broad peak between 750-780 cm-1.

\ce{WnO_{3n-1}} NPs. \cite{Frey2001}

\ce{WO3} growth hydrothermal.\cite{Moshofsky2012}

\ce{W_{18}O_{49}} on tungsten foil by thermal growth\cite{VanHieu2012}

Cathodoluminescence \cite{Parish2007}

optical characterization of WOx film.\cite{Valyukh2010a}

E-beam penetration \cite{Kanaya2002}

optics in electron microscopy. \cite{GarciadeAbajo2010a}


\section{MoO3}

VLS:
Ge NW growth using Ga as catalyst. \cite{Chandrasekaran2006}


plasmon dispersion in 2D materials, plasmon resonances in visible regions by doping induced free carrier density. 2D plasmonics, depolarization factors, partial reduction of Mo to a lower valence state. \cite{Alsaif2014a}

\subsection{app}
applications: electrically controlled optical shutters for heat and light modulation, smart windows associated with solar cell to provide dynamical control of incoming illumination.

piranha clean of FTO. 50ms switch.\cite{Scherer2012} 
nanoscale Kirkendall effect: the outward diffusion of metal cations are balanced by an influx of vacancies. For example, diffusion coefficient of Ni in NiO is higher than that of oxygen.

\ce{MoO3} photocatalytic \cite{Chithambararaj2013}
photocatalytic experimental setup.\cite{Hupka2006}
\ce{MoO3} pseudocapacitor  \cite{Brezesinski2010}

\ce{MoOx} few layer as hole selective contact in solar cell.\cite{Battaglia2014}
\ce{MoO_x} on n-type Si acts as a high work function metal (6.6eV), enabling a dopant-free contact and thus junction-less devices.


\subsection{properties}



\begin{table}[htb]
\caption{Combinations of ECD configuration}\label{tb:ecd}
\begin{tabular}{lcccr}
\toprule
TC(both side) & electrochromic & ion conductor & counter electrode  & reference\\
\midrule
ITO &  \ce{WO3} & \ce{H^+\hyphen} polymer & PANI &\citeauthor{Heckner2002}\\
FTO &  \ce{WO3} & \ce{K^+\hyphen} polymer & PW &\cite{Heckner2002}\\
ITO & \ce{WO3} NWs & \ce{LiClO4\hyphen}PC & none & author design \\
\ce{Na_xWO3} NWs &\ce{WO3} NWs & \ce{LiClO4\hyphen}PC & none & author design\\
\bottomrule
\end{tabular}
\end{table}



\begin{table}[htb]
\centering
\caption{Comparison of MoOx ECD}\label{tab:moxecd}
\begin{tabular}{lcccr}
\toprule
$\lambda$ & $\Delta T$ & $t_c$ & $t_b$ & $CE$  \\
         (nm) & (\%)    & (s) & (s) & ($cm^2/C$)  \\
\midrule
Range      & RT-1100    & 10mTorr-1atm & 0 - 100 & 0-30  \\
\bottomrule
\end{tabular}
\end{table}


10nm MoOx as hole extraction layer (HEL). Without HEL, Holes accumulates at QD/anode interface, causing increased recombination rate. With HEL, hole diffuse into this layer, reducing the recombination. 

The molecular unit in crystal exhibits different vibrational frequencies from that in solution or gas phases.

\ce{Na2Mo4O_{13}} phases monoclinic at RT, solid solubility of \ce{Na2MoO4} in solid \ce{MoO3} is high. vapor pressure of \ce{Na2Mo4O_{13}} over \ce{MoO3}.

melting point of \ce{Na2Mo4O_{13}}
Mp: \ce{Na2Mo2O7} 960K

\ce{MoO3} vapor pressure:

The real phase diagram is the one between \ce{Na2Mo4O_{13}} and \ce{MoO3}.
the growth temperature could be much lower than the eutectic point.

KI MP:  681
NaI MP: 661

NaOH Raman peaks lie at 3633 cm. \cite{walrafen2006} Raman scattering of \ce{Na2SiO3} exhibit major peak at 966 and 589 cm.\cite{Richet1996}

hydrogen absorption in \ce{MoO3}.\cite{Sha2009}

\ce{Na6Mo_{11}O_{36}} phase. \cite{Bramnik2004}

\ce{Na6Mo_{10}O_{33}} phase, \cite{Gatehouse1983}

\ce{MoO3} thin film. \cite{Carcia1987}

\ce{H_xMoO3} raman.\cite{Hirata1996}

MoO3 spreading \cite{Leyrer1990}

Na2Mo2O7, Na2Mo4O13 phase transition \cite{SinghMudher2005}\cite{Tangri1992}

visibility of FL \cite{Benameur2011}

exfoliation IPA \cite{Halim2013}  \cite{Zhou2011a}

\ce{MoO3} good style. \cite{Siciliano2009} \cite{Abdellaoui1997}

\ce{MoO3}  DFT study \cite{B511044K} \cite{Cora1997} \cite{Sayede2005}

\ce{MoO3} raman \cite{Lee2002}

visibility of mica thin layer on \ce{SiO2}-Si. \cite{Castellanos-gomez2011} 1.5\% contrast is almost at the threshold of human eye sensitivity.  When the thickness is below 60nm, Raman could not detect mica.

\ce{MoO3} (010) surface defect. \cite{Chen2001}

mass spectrometry data to extract vapor pressure of \ce{NaxMoO3}.

\ce{MoO3} SWNT by hydrothermal method.\cite{Hu2008a} Raman spectra is off compared to single crystal \ce{MoO3}.  Van der Waals interaction and layered structure make NT possible.

TMO review.\cite{Goodenough2013}

h-\ce{MoO3} \cite{Lunk2010} \cite{Zheng2009}



\section{TMS}

petroleum oil catalytic refinement, solid lubricants in aerospace industry.

heterojunction is employed to transferred photo-generated carriers. Schottky barrier conduction band electron trapping and consequent longer electron-hole pair lifetimes. Numerous studies have suggested that fine particles of transition metals or their oxides, when dispersed on the surface of a photocatalyst matrix, can act as electron traps on n-type semiconductors.\cite{Zhou2010} 


\citeauthor{Cao2014} studied the layer-dependence \ce{MoS2} electrocatalysis and propose the vertical hopping efficiency of electrons instead of the edge site numbers is a key factor for catalytic reaction.\cite{Cao2014} ref19,20


\subsection{strain}

$E_{2g}$ mode is strain sensitive. 

\citeauthor{Ghorbani-Asl2013} studied the strain in tubular TMDC and found a linear dependence of Raman scattering on strain (3 \si{cm^{-1}} per percentage for $E_{2g}$mode).\cite{Ghorbani-Asl2013} 

For 2D materials, strain may be induced by elongation of an appropriate substrate, e.g. by uniform mechanical strain, or by using a material with high thermal expansion coefficient and varying the temperature. For TMD MWNT, tensile tests have been reported by various groups. However, to date, it is not perfectly clear whether inner and outer walls are stretched simultaneously, or rather the outer walls slide on the inner ones. The latter hypothesis would result in a broadening of the Raman signals, while the first one would leave the signal widths rather unaffected. In any case, there would be a shift of the Raman signals that can serve as precise scale for determining the strain.\cite{Ghorbani-Asl2013}


\citeauthor{Virsek2007} performed a Raman-TEM integrated study on multiwalled \ce{WS2} NT with diameter \textgreater 200 nm. The tubes were synthesized using chemical transport method. Up-shift of Raman is explained by strain in the walls. This shift is not observed in the specimen by sulfurization process of oxides. Applied hydrostatic pressure is isotropic,\cite{Staiger2012} while the strain is expected to anisotropic. Strain can also be relaxed by chirality.\cite{Virsek2007} 

strain effect by first-principles calculations. direct gap is only maintain in a narrow strain range (-1.3 -- 0.3 \%), \cite{Yun2012}.

Semiconducting to metallic transition in \ce{MoS2} at compressive strain of 15\% or tensile strain of 8\%; direct-to-indirect gap transition for 1L \ce{MoS2} at about 2\%. \cite{Scalise2012}

strain and Raman theoretical analysis.\cite{Chang2013a} 

magnetic properties of ws2.\cite{Zhang2013j} 

growth mechanism of WS2 NT:

It was found that the critical step in this process is the fast conversion of the oxide nanoparticle surface into a closed monolayer of WS2. W18O49 as an intermediate phase is observed. XRD peaks shift to monitor strain.(002) peak of nanotube shifted to lower angles, the interlayer spacing increase by about 2\% as compared to the bulk powder, likely due to the build-in strain.\cite{ZAK2009} 
$\epsilon = (a - a_0)/a_0$ =(6.4-6.16)/6.16 = 3.8\%. tensile strain ($\epsilon > 0$)



`` Nanotubes not fully converted appeared also during short
runs with higher working pressure. HRTEM observations
revealed an amorphous phase inside some of the nanotubes’
hollow cores, generally near the nanotubes tip (Fig. 5a). The
amorphous phase occupies only a small fraction of the nanotube’s
core volume. A meniscus is found to form at the contact
point between the amorphous matter and the nanotube’s walls.
Fig. 5b displays such a meniscus in the nanotube core (marked
by arrows). The presence of this meniscus indicates that this
amorphous material solidified from a molten phase during the
cool-down period of the sample. The meniscus of the amorphous
phase suggest that the amorphous matter wets the
nanotubes’ walls. Since the WS2 nanotubes are hydrophobic,
this observation indicates that a monomolecular layer of oxide
is left on the entire hollow core of the nanotubes. The nanotube
walls near the contact area with the meniscus are quite defective,
probably due to the large differences between the thermal
expansion coefficients of the WS2 and the amorphous matter,
which induces strain during the cool-down period of the
reaction product. These observations are consistent with the
notion that the amorphous material inside the core is an oxide
phase which is hydrophilic and does not wet the hydrophobic
WS2 layers''\cite{Margolin2004}

\subsection{growth and properties}

WS2 photoluminescence spectra of few layer and nanotube:
NT electrical structures depend on chirality, diameter and layer No as well as strain. Theoretical calculation indicates the SWNT with diameter larger than 4nm should approach the single layer limit.\cite{Ghorbani-Asl2013}

Other chalcogenide has also been synthesized using this one-end sealed layout.\cite{Mukherjee2013}

\citeauthor{Zou2007} prepared W/\ce{WS2} core-shell NPs by reaction of tungsten and sulfur under hydrogen atmosphere.\cite{Zou2007}

CVD 1L WS2 PL.\cite{Peimyoo2013} (of NTU Yu group) PL peak at 635nm, width 40 meV, 

CVD 1L WS2.\cite{Cong2013} (of NTU Yu group) 457 nm excitation, PL at 525nm and 630nm, 

\ce{MoS2} sing-walled nanotube.\cite{Xiao2014}

1T MoS2: metallic phase a negative temperature coefficient for conductivity, XRD pattern identified. \cite{Wypych1992}

stable 1T WS2 multiwalled NT by Re doping.\cite{Enyashin2011}. 2H to 1T transition formerly known only for WS2 and MoS2 intercalated by alkali metals. 3R transition to 2H upon heating since 2H is the most stable one.

1T \ce{MoS2} Raman. \cite{Yang1991} strong peaks at 156, 226, and 330 cm-1. M point frequencies measured by neutron scattering. M point is folded into BZ zone center due to the formation of superlattice.

Electrons and Phonons in Layered Crystal Structures, edited by T. J. Wieting (Reidel, Dordrecht, Holland, 1979).

\ce{WS2} p-type or n-type.  Fermi level at the surface of semiconductor is pinned to a fixed position relative to the CBM and VBM by a sufficient density of surface states situated between CBM and VBM. \cite{Baglio1983}

Electronic structure of \ce{MoS2}.\cite{Eknapakul2014} K intercalating into bulk to create quasi-standing 1L. Large effective mass 0.6 $m_e$ found, implying low mobility. Direct gap 1.88eV is measured.

Self-assembled monolayer (SAM) on \ce{SiO2} and its effect on \ce{MoS2} 1L.\cite{Najmaei2014}

\ce{WS2} 1L doping calculation. \cite{Ma2011}




multipeak Lorentzian fitting. 270 to 410 cm


\citeauthor{Shi2013} studied the strained monolayer \ce{MoS2} and WS2. The results show that exciton binding energy is insensitive to the strain, while optical band gap becomes smaller as strain increases. Monolayer WS2 PL maximum located at about 1.95 eV. Calculation shows the electron effective mass of WS2 is the smallest, rendering higher mobility in device.\cite{Shi2013}

\citeauthor{Kosmider2013} studied the heterojunction between two monolayers of \ce{MoS2} and WS2. Top of VB in W layer and bottom of CB in Mo layer, forming type II structure. bilayer gap 1.2 eV.\cite{Kosmider2013}


Band structure  of \ce{MoS2} in bulk form was calculated by \citeauthor{Mattheiss1973}.The calculation result is 1.2eV (indirect gap).\cite{Mattheiss1973}

Alkali metal intercalated \ce{WS2} film was prepared.\cite{Homyonfer1997} Stage 6 superlattice formation was suggested according to X-ray diffraction, and photoresponse spectra and electron tunneling measurement were done.



quantitative Raman of MoS2 on insulating subs. intensity difference between supported and suspended was highlighted, detailed model in support info.Li2013

WO3-x (1nm) on SiO2/Si sulfurization at 750-950 degree,\cite{Elias2013}

decrease in dielectric screening and thereby enhanced excitonic effect.
DFT is not good at describing photoemission, GW approximation overcome this deficiency but still not enough for photoabsorption process in which ehps are created. BSE equation is used to compensate this discrepancy, WX2 exhibits larger spin-orbit splitting as compared to MX2 family.\cite{Ramasubramaniam2012}



arise as a result of, dispersal of Na by electron probe.



\section{CNT}

SOI:

VSS, growth kinetics,
BN nanotube capping, zigzag is more stable than armchair. \cite{Menon1999}


To develop large-size single-crystal graphene on dielectric substrates. small carbon flow near-equilibrium CVD process. Grain size about 10 microns, precursor \ce{CH4} and \ce{H2} (ratio 2.3:50) at 1180 C. \ce{SiO2}-Si surface roughness. Although the growth substrates (quartz,\ce{SiO2}-Si and \ce{Si3N4}-\ce{SiO2}-Si ) have a complicated stereo network similar to diamond, regular hexagonal G growth is obtained, which indicates the deposition is determined by equilibrium kinetics, and this should be applicable to other 2D materials as well. I2D/IG exceeds two on \ce{SiO2}-Si subs (514.5nm), indicating monolayer G. armchair (AC) G edge grows faster than zigzag (ZZ) edge.\cite{Chen2013j}

catalytic graphitization of solid carbon sources. catalytic transformation, the source is in solid state, low temperature (less than 600C), 2nm  \ce{Al2O3} by ALD as carbon diffusion barrier. amorphous silicon (a-Si), Ni lower the activation barrier ,  tetrahedral amorphous carbon (ta-C).\cite{Weatherup2013}

low energy (50eV) ion implantation doping in G. Ions penetrate pristine G at energy larger than 100eV. Individual substitutional incorporation of B into G lattice is demonstrated. 1\% doping level was obtained. \cite{Bangert2013}


CVD G on copper. Size of single crystal domain and nucleation site density.\cite{Wu2013b}

Concentration of charge carrier $n$ is related to gate voltage $V_g$ by:
\[
n = \frac{\epsilon_0 \epsilon V_g}{ed}
\]
where $\epsilon_r = \epsilon_0 \epsilon$ is dielectric constant of gate materials.

massless relativistic chiral particles, Klein paradox, 100\% tunneling and extreme high mobility.

\ce{CaF2} a material suitable for scattering efficiency S comparison measurement due to its large band gap ($S\times \omega_L^4$ is constant below 5eV).

symmetry-breaking mechanism,

low energy ion doping of graphene.\cite{Ahlgren2011}

\section{misc}

E-beam spatial coherence.\cite{Morishita2013} phase contrast transfer function, coherence estimated by the visibility of double slits interference fringes, an effective diameter in specimen plane.  Image is a result of convolution between object and lens, point source on the focal plane, after lens the EM wavefront intersect image plane at different angle $\theta = d/f$, 

nucleation and film growth \cite{Hanbucken1984}

intrinsic silicon equilibrium charge carrier concentration at RT is $n_i = p_i = 1.5 \times 10^{10} cm^{-3}$, much smaller than silicon atoms density as $5\times 10 ^{22} cm^{-3}$.

The average distance between dopant atoms is cubed root of concentration, $d = (10^{18} cm^{-3})^{-1/3} = 10nm$.

The electron mobility $\mu_n = 1500 cm^2/V\cdot sec $ at RT for Si, and hole mobility $\mu_p = 450 cm^2/V\cdot sec$ at RT.

for p-type silicon, when the conductivity $\sigma = 1 (ohm cm )^{-1}$, the doping level is
$N_A = \frac{\sigma}{q \mu_p}= 1 / (1.6E-19 \times 450) = 1.4E16 cm^{-3}$.

Built-in voltage $V_0 = \frac{kT}{q}ln(N_A N_D/n_i^2)$, depletion region width $W = \sqrt{\frac{2 \epsilon_{Si} V_0}{q}(1/N_A + 1/N_D)}$, where $\epsilon_{Si} = 11.7 \epsilon_0$. When applying external field, depletion width $W = \sqrt{\frac{2 \epsilon_{Si} (V_0 - V) }{q}(1/N_A + 1/N_D)}$

The capacitance of p-n junction is $C = A \sqrt{\frac{q \epsilon_{Si}}{2(V_0 -V)}(N_D N_A/(N_A + N_D))}$.


oxygen plasma treatment on HF-etched Si (001). reaction among $e$, \ce{O^+}, \ce{O2^+}, \ce{O^-},\ce{O2}. \ce{OH}-terminated surface obtained.\cite{Habib2010}

\subsection{MB}
MB is a heterocyclic aromatic dye which is blue colored in oxidizing environment. Upon reduction, MB is turned into colorless leuco MB. This can be used as an oxygen indicator in food industry. Photo-bleaching of MB can be also due to its leuco formation rather than total decomposition. Photocatalytic decomposition can be minimized by keeping the solution at acidic condition (PH = 4), which will limit the formation of oxidative hydroxyl radicals (E = 2.8eV vs normal hydrogen electrode). Oxygen dissolved in the solution play a key role in conversion of LMB to MB under visible light. Purging with \ce{N2} for 20mins can remove dissolved oxygen. \cite{Wang2014a}

solar energy harvesting representative study.\cite{Yoneyama1972} MB to LMB (\ce{C16H19N3S}) in aqueous solution upon illumination of \ce{TiO2}. The colorimetric analysis was performed in a glove box under nitrogen atmosphere. The absence of oxygen is important to prevent the oxidation of LMB to blue MB.
\[
\cee{MB^+ H2O + H^+ \rightarrow MBH3^{2+} + 1/2O2}
\]
where MB represents the uncharged center of MB molecule.

common wisdom expect that a dye incapable of injecting an electron at the excited state to CdS. MB, which process N-methyl groups in its molecular structure and does not sensitize CdS is an exemplary candidate. quantum efficiency is defined as probability of MB converted to azure B per incident photon. QE of CdS to MB decomposition is reduced in nitrogen bubbling treated solutions, indicating the necessity of oxygen. Two possible mechanisms: a) adsorbed oxygen acts as a trap for the conduction electron and prevent the accumulation of negative charge within space charge region of CdS, supported by the formation of \ce{O2^-} in excitation of CdS in aqueous suspension.\cite{Takizawa1978}

ref 16, MB aqueous solution stability. Liquid chromatogram, azure B (trimethylthionine), and thionine. Electrochemical measurement,

MB adsorption.  photocatalytic oxidation of MB by \ce{TiO2} film. photo-oxidation reaction occurs at the surface of photocatalyst. Mb molar extinction coefficient was found to be 66700 1/cm 1/M. Langmuir adsorption isotherm.\cite{Matthews1989}

\[
[MB]_{ads} = \frac{k_1 k_2 [MB]}{1 + k_1[MB]}
\]
and integrated form of Langmuir adsorption isotherm
\[
t = \frac{1}{k_1K} In\frac{[S]^0}{[S]} + \frac{1}{K}([S]^0 - [S])
\]
where $K = k_2 \phi N T_r$, with $\phi$ as quantum yield, N as total absorbed photons, and $T_r$ as rate of transport.
\[
\cee{C16H18N3SCl + 25.5O2 \rightarrow 16CO2 + 6H2O + 3HNO3 + H2SO4 +HCl}
\]
which indicates the total oxidation of $10 \mu M$ MB would exhaust the ambient oxygen concentration of initially air-equilibrated solutions (about $250 \mu M$ ). ref 28 Thus the transport of both oxygen and MB to the photocatalyst surface are anticipated to be key factors.

photoelectrochromism at \ce{TiO2}/MB interface and its control. Efficient capture of photogenerated holes by a reducing agent is crucial to the reversibility of bleach-recoloration transition. This transition is kinetically dictated by electron transfer. Holes transfer is not desired.\cite{DeTacconi1997}

256 nm band is associated to the presence of LMB. LMB formation is not favored at alkaline pH values in aqueous solution. The OH radicals are generated either with the surface hydroxyl groups on \ce{TiO2} or with water, and its high oxidizing power cause photocatalytic decomposition of the dye.

An elementary step in decomposition of MB is N-dealkylation, which is preceded by radical cation formation.\cite{Takizawa1978} This radical cation can be spectroscopically monitored by the presence of 520nm band for MB. In MB absorption spectrum, 664 and 614 nm band ratio is related to monomer and dimer relaxation.
\begin{align}
\cee{TiO2 &\rightarrow e_{CB}^- + h_{VB}^+ \\
h_{VB}^+ + red &\rightarrow ox\\
MB^+ + 2e_{CB}^ + H^+ &\rightarrow LMB}
\end{align}

Measure the ratio between 614 and 663 nm before and after adding WS2 can indicate the adsorption of monomer and dimer MB.

MB can act as sacrificial electron acceptor in the reduction to leuco form. The decomposition is favored under oxygen-rich environment. MB feature peaks at 663, 614 and 292 nm, and $\epsilon_{660}=10^5 M^{-1}cm^{-1}$. The doubly reduced form of MB, LMB has feature peak at 256 nm. The singly reduced form of MB, \ce{MB.^-} is pale yellow, with peak at 420nm.\cite{Mills1999}
\[
\cee{MB + e_{CB}^- ->[pH<7] MB.^-}
\cee{2MB.^- \rightarrow MB + LMB}
\cee{O2 + e_{CB}^- \rightarrow O2.^-}
\]

The oxidized form of MB, \ce{MB.^+} has peak at 520nm, which is stable in acidic solution, but decomposes irreversibly in slight alkaline solution(pH = 9).
thionine peaks at 600nm.
MB forms dimers in aqueous solution,
\ce{
2MB <=>[K_D] (MB)_2
}
A typical value of $K_D$ is 3970 1/M. A quadratic equation can be solved to obtain the monomer concentration:
\[
2K_D [MB]^2 + [MB] - [MB]_{total} = 0
\]
MB adsorption on metal oxides. Monomer size is less than 1.5nm.
Logarithmic acid dissociation constant $pK_a= -\log_10 \frac{[A^-][H^+]}{[HA]}$. The oxidation potential for \ce{H2O}-\ce{O2} couple is 1.23V and 0.817V versus NHE at pH 0 and pH 7, respectively.

%\begin{align}
%\cee{MB + SED &->[TiO2][h\nu \leq 3.2eV] LMB + SED^{2+}\\
%2LMB + O2 &\rightarrow 2MB + 2H2O}
%\end{align}


S.L. Murov, I. Carmichael, G.L. Hug, Handbook of Photochemistry, 2nd revised ed. Marcel Dekker, New York, 1993.

aerobic or anaerobic, dimerise, photominerlization, gas to liquid transfer.

Mb to LMB transition as visual time monitor. commercial colorimetric oxygen indicators. radical-bearing carbon with unpaired electrons. MB = \ce{MB^+Cl^-}.\cite{Galagan2008}

monomer MB and dimer MB kinetics.\cite{Spencer1979}



MB. \cite{Lee2003a}
\begin{align}
\cee{ 2LMB &->[\text{UV}] LMB^*\\
2LMB^* + O2 &\rightarrow 2MB^+ + 2OH^-}
\end{align}

\[
\cee{2LMB ->[\alpha] LMB^*}
\cee{2LMB ->[\text{above}] LMB^*}
\]


\section{solar cell}


\ce{TiO2} NPs for high loading of sensitizing dye. Hole conducting electrolyte with \ce{I^-} and \ce{I3^-} concentration close to $10^19 cm^3$. chemical anchoring groups.

electron injection rate. e transfer rate is several order faster than hole. 1 M = $6\times10^{20} cm^{-3}$. \ce{I^-} is known to coordinate with the sulfur atoms on NCS ligand. ref 31.

FRET: dipole-dipole coupling, energy relay dye to sentisizing dye and then to \ce{TiO2}. analogous to photosynthesis bacteria. Time-resolved PL to measure FRET $R_0$.

photocatalyst review.\cite{Mills1997} Definition: catalysis should not be used unless it can be demonstrated that the turnover number\footnote{the number of product molecules per number of active sites.} is greater than unity. Otherwise, semiconductor-assisted photoreaction is more appropriate. aerated, flush with air; nitrogen-purged. Degussa P25 \ce{TiO2} high temperature flame hydrolysis of \ce{TiCl4} in presence of hydrogen and oxygen. Oxidization of organic species is presumably obtained by \ce{Ti^{IV}OH^{.-}}, rather than direct hole transfer. carrier decay pathways. deactivation of catalyst by intermediate product.

MB natural decoloration under sunlight is found to be about 18\%.\cite{Nogueira1993} Latitude: 24 south, $3.4mW/cm^2$. natural evaporation should be prevented or corrected.

\section{dissertation}

WS2 surface hydrophilic or hydrophobic.
\[
\cee{Ti(IV)-OCH3 + h^+ -> Ti(IV)-O^+CH3}
\]

\ce{WO3} photoanode should be a n-type semiconductor, stable in acidic aqueous solution.

Swagelok TM. List of equipments,

Zheng thesis: \ce{TiO2} anodic nanotube by sputtering Ti on FTO and anodizing in F-organic electrolyte.

heat cure gasket ( ionomer surlyn 1702 Dupont), 125 C for 30s.
vacuum back filling.



\section{vocabulary}
$\delta\omega/\epsilon$ 

\subsection{pronunciation}

chamber, vias, valve, figure, energetic, managerial, inert, volatile, chromic,
laminar, palladium, platinum, photovoltaic, acronym, chirality, stoichiometric, cyclic voltammetry, quasi, pseudo, 

\section{job related paper}

OLED with \ce{MoO3} as charge generation layer. \cite{Kanno2006} The PI is stephen forrest, also a cofounder of udc oled.\footnote{http://www.umich.edu/~ocm/research.html}

stacked led, luminescence linearly increase with layer no at fix current density. hole transporting layer by MoO3, and electron transporting layer by Li.

OLED photovoltaic.\cite{Xiao2012a} functionalized squaraines as donor

excitation state management \cite{Zhang2012b}.
triplet-triplet annihilation (TTA) and singlet-triplet annihilation (STA). spin number of exciton is either 0 (singlet) or 1 (triplet).

TTA, bound electron-hole pairs, introducing a heavy metal in organic molecule to enhance the spin-orbital coupling, enabling triplet emitters.\cite{Zhang2013i} yet when operating at high current, the efficiency is decreased, and TTA is considered as a intrinsic limit.


(T. Tsujimura OLED Displays: Fundamentals and Applications John Wiley \& Sons Inc., Hoboken, New Jersey (2012))

why graphene on copper: a review article. \cite{Mattevi2011} research thrust, post CMOS fab tech. Exposure of hydrocarbon or evaporated carbon onto transition metal, the formation of graphite was surmised as a consequence of diffusion and segregation of carbon impurities from bulk to surface. carbon solubility in the metal and. 

The lack of control over layer No on Ni is partially attributed to the face that the segregation of carbon from the metal carbide upon cooling occurs rapidly within the Ni grains and heterogeneously at the grain boundaries. phase diagram of Ni and C reveals that the solubility of carbon in nickel at high T > 800 C form a solid solution. metastable phase \ce{Ni3C} promote the precipitation of C out of Ni. Carbon preferentially precipitates out at the grain boundaries of polycrystalline Ni subs so the thickness at graphite is higher than within grains. On Fe, due to the high affinity between Fe and C (\ce{Fe3C} is a stable carbide), the formation of $sp^2$ crystalline carbon film is suppressed. 

on copper (decomposition of methane gas at 1100 C), independent of heating or cooling rate. For copper, the 3d shell is filled, leading to less reactive configuration and weaker affinity towards Carbon. Cu can only form soft bonds with C via the charge transfer from the $\pi$ electron in  sp2 hybridized C to the empty 4s states of copper, as supported by the fact that copper does not form any carbide phase, and low C solubility. This low affinity and weak bond makes copper a true catalyst fro graphitic carbon formation. pre-treatment of copper foil, \ce{CuO}, \ce{Cu2O} removal by reducing annealing at 1000 C. 

epitaxial and lattice mismatch is present, 

\textbf{four-point probe}

sheet resistivity. a current source in an infinite sheet gives rise to the logarithmic potential
\[
\phi - \phi_0 = - \frac{I\rho}{2\pi}\log r,
\]
the potential for a dipole becomes
\[
\phi - \phi_0 = \frac{I\rho}{2\pi}\log r_1/r_2,
\]

When equal spacing probes are used, then potential difference between two inner points is 
\[
\delta\phi = V = \frac{I \rho}{\pi}\log2,
\]
so sheet resistivity $\rho$ is obtained as
\[
\rho = 4.5324V/I.
\]




\section{The physics of semiconductor}


quantum well devices exploit spatial quantization effects to increase the efficiency as well as alter the lasing threshold, and  novel semiconductor is made to emit light at wavelengths different from those possible when only bulk material is used.


The discontinuity in the CB and CB occurs at the interface. The corresponding potential discontinuity create a potential difference, forming a trap in which electron or hole can only have discrete energies.

a material of smaller bandgap is sandwiched between two layers of material of greater bandgap, or vice versa.

transmissivity coupled in multiple quantum well structure.

band structures=potential energy diagrams.

\textbf{junctions}
p-n homojunctions, p-n or n-n heterojunctions, metal-semi junctions, most important types: Schottky barriers, or ohmic contact.

chemical potential (Fermi level) as a measure of particle concentration.  Fermi level at 0K is equal to Fermi energy, which is defined as the energy of the topmost filled orbital. In equilibrium, the Fermi level $E_f$ is uniform throughout the material. n electron carrier, p holes carrier.

band diagram represents the electron's potential energy.

band bending: the electron energies are greater on the p side than on the n side, or the electrostatic potential is greater on the n side than on the p side, since potential $V = E/-q$.

The built-in potential for a homojunction is equal to the full band bending in equilibrium.

heterojunction: bandgap discontinuity, to solve the Poisson's equation for band bending.

Metal-Semi: metal cannot support any potential difference across it. Fermi level in Semi is pinned at the interface. Electron transfer from n-type semi to metal, leaving ionized donors behind.

ohmic contact forms if the work function of the metal is less than that of the semi. net flow of electron from metal into semi, no depletion layer forms.

metal-oxide-semi as MIS structures is capacitive in nature since no dc current flows under bias.

ch11.5 nonequilibrium conditions
quasi-fermi level $\phi_n$ or $\phi_p$.

\section{Gary semiconductor fabrication} 

photolithography a patter transferring process from mask to photoresist. clean room needed to remove dust particles, which could cause dislocation on an epitaxial film, low breakdown voltage in gate oxide, or short circuit. 

resolution, registration for effectiveness, throughput for efficiency, shadow printing where mask and wafer in direct contact: cons, dust could case permanent damage to mask, a small gap $d$ (10-50 $\mu$m ) used usually, and the minimium linewideth CD is roughly $\sqrt{\lambda d}$; and projection printing , resolution $l_m = \frac{k_1 \lambda}{NA}$, and depth of focus $\frac{k_2 \lambda}{NA^2}$. 193-nm using ArF excimer laser, and 157 nm using \ce{F2} excimer laser. 

365 nm probably used for \ce{LiNbO3} waveguide. 

resolution enhancement using phase shifting mask,  using the electric field of EM wave to chemically activate the photoresist. A $\pi$ phase change is obtained by using a transparent layer of $d = \lambda/2(n - 1)$ thickness. near-field diffraction, 

EUV 10 - 14 nm, C inner shell electron transition, 


\section{non-imaging optics}

black body radiation power density $S$ is about 1 \si{kW\per m^2}, the balanced temperature $T = \sqrt[4]{S/\sigma} = 364$ K, where $\sigma$ is Stefan-Boltzmann constant. 

ray tracing in vector form as $n^{\prime} r^{\prime} \times n = n r \times n$, invariant $na\theta$, 
























\renewcommand{\bibname}{REFERENCES}
 \begin{singlespace}
  \printbibliography
 \end{singlespace}

\end{document}

 
%\section{Crystal Structures and Electronic Properties}

Solid and orderliness.

Two theories arise to describe the outer shell electrons and to correlate the structure and physical properties: \gls{cft} and band theory.\cite{Goodenough1971} \gls{cft} assumes weak interaction between neighboring atoms and localization of electron towards parent atom, whereas band theory assumes that electron is shared equally by all nuclei and therefore a many-electron problems follows. Description of a single electron in periodic potential fail to treat the electron correlations adequately, as the interaction between atoms becomes weaker.For transition metals, $s$ and $p$ electrons are well described by a collective-electron model, while the 4f or 5f electrons are tightly bound to nuclei and screened from the neighboring atoms by 5s, 5p or 6s, 6p core electrons, hence it matches well with a localized-electron model. d electrons show intermediate character.


In vapor synthesis process, two growth mechanism exists: VS and VLS. VS process is widely accepted for the growth of \ce{MoO3}. Yet we caution that synthesis conditions should be scrutinized to determine the exact mechanism. \citeauthor{Li2002c} suggested a VS mechanism at 700 \si{\degreeCelsius} and VLS at 750 \si{\degreeCelsius} and higher.\cite{Li2002c} \citeauthor{Fibers2007} proposed a modified VS mechanism probably because the deposition occurs on \ce{Al2SiO5} with possible \ce{Al_{0.95}SiNa_{0.06}O_x} involved. Therefore temperature and possible impurity could potentially alter the growth mechanism. We divide the growth results into two categories: on Si substrate and on non-Si substrate, and describe them respectively. We also briefly mention using liquid exfoliation to prepare few layer \ce{MoO3}.


This rectangular shape implies the boundary plane along growth direction (long axis) is (001), in consistency with previous experimental reports\cite{Zeng1998,Li2002b} and theoretical studies.\cite{Firment1983,Cora1997} We also observed different shapes, i.e., elongated hexagonal using similar growth conditions. This is not an indication of different growth mode. It is a normal thermodynamic fluctuation. The growth rates along different crystalline direction of \ce{MoO3} are determined by the free surface energy. In fact, (201), (101) and (102) planes have all been observed as terminating planes.\cite{Zeng1998} The stacking rate of \ce{MoO6} octahedra along $a$ and $c$ axis could develop some other ratio. And the coexistence of different planes in one growth suggests the similarity of free surface energies between these surfaces. In other words, the migration barrier of adatoms on (010) plane is presumably much lower than that on other low index planes due to the Van der Waals interaction nature along [010] direction.

\citeauthor{Wang2013c} changed the NW growth direction by changing the shape of droplets; \citeauthor{Biswas2013} reported enhanced aspect ratio of Ge NWs from bimetallic alloy.




We first repeated the previous growth on ITO/glass, and then found similar deposition phenomena on glass and mica substrates. We suspected that the morphology difference on the substrates employed in this study arises from the compositions difference. Both mica and glass contains alkaline elements which are absent in Si and \ce{SiO2}/Si substrates. We primarily verify this hypothesis by utilizing \ce{NaOH} treated Si substrates and found that the overall morphology almost reproduces that on ITO and mica substrates. We then propose a new growth mechanism based on the VLS process to explain the deposition of \ce{MoO3} on alkaline-ions-containing substrates.

ITO versus NaOH images.



According to the phase diagram shown in Fig.~\ref{fig:pd}, \ce{Na2MoO4} - \ce{MoO3} phase above 800 \si{\degreeCelsius} are \ce{Na2MoO4}(l) + \ce{MoO3}(l), More discussion should follow $\ldots$.


%\subsubsection{Phase evolution probing by Raman}

\citeauthor{Hardcastle1990} summarized an empirical formula to relate the Raman peaks and \ce{Mo-O} bonding lengths.\cite{Hardcastle1990} This correlation assumes general form as
\begin{equation}\label{eq:mobond}
\nu = A \exp{B\cdot R},
\end{equation}
where $A=32895$ and $B=-2.073$ are fitting parameters, R is bond distance in unit of \AA. Given a stretching frequency, the resolution for calculated bond distance is $\pm0.016$ \AA. Another empirical expression connect the bond valence $s$ and bond distance R: $s(M-O) \approx (R/X)^{-6} $, where X=1.882 when M is Mo, and 1.904 when M is W. The valence sum rule could be then used to check the state of Mo cation. It should be noticed that not all observed Raman lines could be correlated to a \ce{Mo-O} bond distance by extrapolation of Eq.~\ref{eq:mobond}. It is then regarded as a symmetry related vibrational mode, i.e. 820 \si{cm^{-1} in \ce{MoO3}. From the correlation of various Mo compounds, a general conclusion is the lower the stretching frequency for the shortest metal-oxygen bond, the more regular is the structure.



% droplet position statistics
\begin{figure}[htb]
\centering
\includegraphics[width=0.6\textwidth]{droplets_sta}
\caption[Droplet position statistics]{Droplet position statistics. On-tip and not-on-tip counts variation with respect to growth time. }
\label{fig:mo3dropsta}
\end{figure}


Nanotower is a less common morphology in the structures of nanomaterials.\cite{Kharissova2010} We have not found report on  \ce{MoO3} nanotower structures, although there are several studies on \ce{In2O3},\cite{Jean2010,Yan2007}, GaN\cite{Xiao2012} and ZnO\cite{Zhang2013c}. \citeauthor{Jean2010} investigated the growth mechanism of \ce{In2O3} nanotowers and attributed it to the periodical axial and continuous lateral growth interaction, where the former was first guided by Au-In alloy liquid, and subsequent by self-catalytic In droplets (MP 156C) while the latter due to VS process. In contrast, \citeauthor{Yan2007} observed similar \ce{In2O3} nanotower growth and proposed that the axial growth is controlled by Au-catalytic VLS process. \citeauthor{Xiao2012} reported the GaN nanotower grown on Ni-coated Si(111) substrates and explained the growth as asymmetric self-copy process based on VLS mechanism. \citeauthor{Zhang2013c} studied the growth of ZnO nanotower and provided a competitive model between axial and lateral growth controlled by Zn vapor to explain the as-synthesized structures. (When ZnO and carbon mixture was used as source instead of ZnO alone, nanorods array results, presumably due to the stable supply of Zn vapor owing to the gentle carbothermal reduction.) Therefore the fluctuation of reagents is probably responsible for tapering and periodical modification, which is absent in the coin roll style growth of \ce{MoO3} tower on Si. The occasionally observed tapering or enlarging on \ce{MoO3} tower can be attributed to the cooling down growth.

We divide the growth into two stages according to the emerging time of long belts and towers. In first stage the growth is dominated by lateral growth, as illustrated in . In the second stage, \ce{NaxMoO3} is evaporated and recondensed in the low temperature region, and some droplet will form, and when at proper locations, introduce secondary VLS growth.

In first stage, the Na-Mo-O liquid is formed and have much more higher absorption rate than that of bare substrate. The absorbed species will diffuse along the surface and inside the liquid. It was difficult to estimate which diffusion path one is faster. 


\textbf{open questions}
\begin{itemize}
\item Is the liquid consumed? if so, how? : evaporation
\item How the size of liquid is related to the morphology evolution?
\item Possible doping level of alkaline ions
\item other lateral growth mode such as Ni-W-S system
\item VLS-VS interaction, or possible solution-liquid-solid, VSS,
\item the interpretation of phase diagram
\item the overall molar amount of NaOH and MoO3 deposition
\item Young's parameters, contact angle informations, Gibbs free energy calculation
\item catalyst composition variation during growth
\item \ce{Na2Mo4O_{13}} phases, solid solubility of \ce{Na2MoO4} in solid \ce{MoO3} is high.
\item vapor pressure of \ce{Na2Mo4O_{13}} and \ce{MoO3}.
\end{itemize}


As shown in Fig, we found several solid clusters with diameter about 1 $\mu m$ on the top part of long belts. According to the \ce{Na2MoO4}-\ce{MoO3} binary compounds phase diagram, we expect the terminating solid clusters have a Na:Mo ratio close to 0.6. The experimental results match well with this predictions. EDX analysis reveal the presence of Na on the solid cluster but not at about 1$\mu m$ away from previous location.\footnote{the sensitivity of our instrument is estimated to be 1\% atomic level} Quantitative atomic ratio derived from the solid cluster is not close to the one predicted by phase diagram probably due to the high Mo content in adjacent area. However, XRD examinations on a short time growth sample revealed a possible \ce{Na2Mo4O13} phase(PDF 028-1112), providing an indirection evidence of the droplet composition. We also suspect that the growth will become different below 500 \si{\degreeCelsius} since no liquid would remain. Substrate placed at low temp end show.
We can also deduce that other sodium compounds can be employed as substrate treatment agent. This argument is confirmed by \ce{Na2CO3} and \ce{KI} assisted growth.

